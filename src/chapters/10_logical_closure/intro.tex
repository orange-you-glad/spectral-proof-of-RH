\subsection*{Introduction to the Closure of the Spectral Program}
\label{subsec:closure_of_spectral_program}

This chapter completes the analytic reformulation of the Riemann Hypothesis by recasting it as a spectral rigidity statement for a canonical trace-class operator. We synthesize the operator-theoretic, determinant-theoretic, and trace-theoretic components developed in previous chapters into a logically acyclic, formally closed equivalence.

\paragraph{Canonical Framework Summary.}

\begin{itemize}
  \item \textbf{Canonical Operator Construction:}  
  A compact, self-adjoint, trace-class operator
  \[
  \Lsym \in \TC(\HPsi)
  \]
  is constructed via trace-norm limits of mollified convolution operators with kernels derived from \( \FT^{-1}[\Xi(s)] \). Exponential decay ensures trace-class inclusion.

  \item \textbf{Determinant Identity:}  
  The zeta-regularized Fredholm determinant satisfies:
  \[
  \detz(I - \lambda \Lsym) = \frac{\Xi\left(\tfrac{1}{2} + i\lambda\right)}{\Xi\left(\tfrac{1}{2}\right)},
  \]
  establishing a canonical analytic encoding of \( \Xi(s) \) through \( \Lsym \).

  \item \textbf{Spectral Multiplicity Encoding:}  
  The Hadamard factorization of \( \Xi(s) \) implies:
  \[
  \operatorname{ord}_{\rho}(\zeta) = \operatorname{mult}_{\mu_\rho}(\Lsym), \quad
  \mu_\rho := \tfrac{1}{i}(\rho - \tfrac{1}{2}),
  \]
  showing that the determinant records both location and multiplicity of nontrivial zeros.

  \item \textbf{Spectral Bijection:}  
  The map \( \rho \mapsto \mu_\rho \) defines a bijection from \( \zeta \)-zeros to the nonzero spectrum of \( \Lsym \), with multiplicities preserved (\thmref{thm:spectral_zero_bijection_revised}).

  \item \textbf{Spectral Symmetry:}  
  Functional symmetry \( \Xi(s) = \Xi(1 - s) \) yields:
  \[
  \mu \in \Spec(\Lsym) \Rightarrow -\mu \in \Spec(\Lsym),
  \]
  encoding reflection symmetry in the spectrum.

  \item \textbf{Spectral Rigidity and Logical Equivalence:}  
  The Riemann Hypothesis is shown to be equivalent to spectral reality:
  \[
  \RH \iff \Spec(\Lsym) \subset \R,
  \]
  as proven unconditionally in \thmref{thm:rh_spectrum_equiv}.

  \item \textbf{Spectral Determinacy:}  
  The spectrum of \( \Lsym \) reconstructs \( \Xi(s) \) via its determinant. In particular, all spectral data determines the zero structure of \( \zeta(s) \) (\corref{cor:spectral_determines_zeta}).

  \item \textbf{Positivity and Regularization:}  
  The trace functional
  \[
  \phi \mapsto \Tr(\phi(\Lsym))
  \]
  defines a positive tempered distribution on \( \R \) (\lemref{lem:trace_distribution_positivity}), supporting analytic continuation and regularized zeta determinants.

  \item \textbf{Analytic Closure:}  
  All results rely solely on:
  \begin{itemize}
    \item spectral theory for compact self-adjoint operators;
    \item trace-class convergence and Schatten ideals;
    \item Hadamard factorization of entire functions;
    \item Laplace–Mellin regularization and heat kernel analysis.
  \end{itemize}
  No unproven input from arithmetic geometry, automorphic theory, or modular forms is used.
\end{itemize}

\paragraph{Logical Structure.}
The complete DAG for this construction appears in Appendix~\ref{app:dependency_graph} (Figure~\ref{fig:dag_visual}). The final theorem
\[
\RH \iff \Spec(\Lsym) \subset \R
\]
is established in Chapter~\ref{sec:logical_closure}, closing the analytic–spectral loop with formal rigor.
