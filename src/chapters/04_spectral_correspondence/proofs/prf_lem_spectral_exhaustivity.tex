\begin{proof}[Proof of \lemref{lem:spectral_exhaustivity}]
Let \( \{ \mu_n \} \subset \Spec(\Lsym) \setminus \{0\} \) denote the nonzero eigenvalues of the canonical compact, self-adjoint operator \( \Lsym \in \TC(H_{\Psi_\alpha}) \), counted with algebraic multiplicity.

\paragraph{Step 1: Determinant Zeros Correspond to Zeta Zeros.}
By \thmref{thm:det_identity_revised}, the canonical Fredholm determinant satisfies
\[
\detz(I - \lambda \Lsym) = \frac{\Xi(\tfrac{1}{2} + i\lambda)}{\Xi(\tfrac{1}{2})},
\]
which is an entire function of order one and exponential type \( \pi \). The right-hand side vanishes precisely at \( \lambda_\rho := i(\rho - \tfrac{1}{2}) \in \C \), where \( \rho \in \C \) is a nontrivial zero of \( \zeta(s) \). The order of vanishing equals the multiplicity of the zero in the Hadamard product of \( \Xi(s) \).

\paragraph{Step 2: Spectral Inclusion via Fredholm Theory.}
Since \( \Lsym \in \TC \), analytic Fredholm theory (cf.~\cite[Thm.~3.1]{Simon2005TraceIdeals}) implies:
\[
\lambda^{-1} \in \Spec(\Lsym) \setminus \{0\} \quad \Longleftrightarrow \quad \detz(I - \lambda \Lsym) = 0,
\]
with multiplicities preserved. Thus for each \( \lambda_\rho = i(\rho - \tfrac{1}{2}) \), we obtain
\[
\mu := \lambda_\rho^{-1} = \frac{1}{i}(\rho - \tfrac{1}{2}) \in \Spec(\Lsym).
\]

\paragraph{Step 3: Spectral Exhaustivity.}
The Hadamard factorization of \( \Xi(s) \) guarantees that \( \detz(I - \lambda \Lsym) \) has no zeros other than the \( \lambda_\rho \) above. Hence, all nonzero eigenvalues of \( \Lsym \) arise from the spectral map
\[
\mu_\rho := \tfrac{1}{i}(\rho - \tfrac{1}{2}),
\]
for some zero \( \rho \) of \( \zeta(s) \). The multiplicities match because both determinant and spectrum admit order-one Hadamard structures, and the determinant encodes all of \( \Spec(\Lsym) \setminus \{0\} \).

\paragraph{Conclusion.}
Every nonzero eigenvalue of \( \Lsym \) corresponds to a unique nontrivial zeta zero \( \rho \), and multiplicities match exactly. This establishes surjectivity and completes the spectral bijection.
\end{proof}
