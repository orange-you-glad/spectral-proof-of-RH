\begin{proof}
By assumption, \( L \in \mathcal{S}_1(H_{\Psi_\alpha}) \) is compact and self-adjoint, and satisfies the normalized determinant identity
\[
\det\nolimits_\zeta(I - \lambda L) = \frac{\Xi\left(\tfrac{1}{2} + i\lambda\right)}{\Xi\left(\tfrac{1}{2}\right)}.
\]
This uniquely determines the eigenvalue multiset of \( L \) (with multiplicities) by trace-class determinant theory (see~\cite[Theorem~4.2]{Simon2005TraceIdeals}).

\smallskip

\textit{(1)} Follows from equality of entire determinant functions: the zeros of the determinant match precisely those of the canonical model, hence the spectral data of \( L \) and \( L_{\mathrm{sym}} \) agree.

\smallskip

\textit{(2)} As both operators are compact and self-adjoint on the same Hilbert space with identical spectrum, the spectral theorem (cf.~\cite[Theorem~VI.16]{ReedSimon1980I}) ensures there exists a unitary operator \( U \) with
\[
L = U L_{\mathrm{sym}} U^{-1}.
\]

\smallskip

\textit{(3)} Therefore, \( L_{\mathrm{sym}} \) is unique up to unitary equivalence within the class of self-adjoint, trace-class realizations of the completed zeta function's determinant identity.

\smallskip

\textit{(4)} If a non-self-adjoint \( \tilde{L} \in \mathcal{S}_1 \) also satisfies the determinant identity, then its eigenvalue multiset must still match. However, the lack of normality can lead to inequivalent Jordan forms or non-diagonalizability. Thus, such an operator must be similar (but not unitarily equivalent) to \( L_{\mathrm{sym}} \), and only under additional structural constraints can the determinant equality be preserved.

\smallskip

Hence, the spectral determinant identity canonically determines \( L_{\mathrm{sym}} \) within the space of compact self-adjoint trace-class operators.
\end{proof}
%  