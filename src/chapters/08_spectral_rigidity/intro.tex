\subsection*{Introduction}

This chapter recasts the Riemann Hypothesis as a statement of spectral rigidity: the assertion that the spectrum of the canonical trace-class operator
\[
L_{\mathrm{sym}} \in \mathcal{C}_1(H_{\Psi_\alpha}),
\]
constructed via mollified convolution from the inverse Fourier transform of the completed Riemann zeta function \( \Xi(s) \), lies entirely in \( \mathbb{R} \) if and only if all nontrivial zeros of \( \zeta(s) \) lie on the critical line \( \Re(s) = \tfrac{1}{2} \).

\paragraph{Goals.}
\begin{itemize}
  \item \textit{Spectral Encoding.}  
  The Carleman \(\zeta\)-regularized determinant identity
  \[
  \det\nolimits_{\zeta}(I - \lambda L_{\mathrm{sym}})
  = \frac{\Xi\left(\tfrac{1}{2} + i\lambda\right)}{\Xi\left(\tfrac{1}{2}\right)}
  \]
  induces a multiplicity-preserving map
  \[
  \rho \mapsto \mu_\rho := \frac{1}{i}(\rho - \tfrac{1}{2}),
  \]
  sending nontrivial zeros \( \rho \) of \( \zeta(s) \) to eigenvalues \( \mu_\rho \in \operatorname{Spec}(L_{\mathrm{sym}}) \). This encoding intertwines the Hadamard product expansion of \( \Xi(s) \) with the spectral structure of \( L_{\mathrm{sym}} \). We prove that every zero of the determinant corresponds to a spectral eigenvalue.

  \item \textit{Spectral Rigidity.}  
  While \( L_{\mathrm{sym}} \) is self-adjoint and hence has real spectrum by construction, we prove the converse: if all encoded eigenvalues \( \mu_\rho \in \mathbb{R} \), then the corresponding zeros \( \rho \) lie on the critical line. That is,
  \[
  \operatorname{Spec}(L_{\mathrm{sym}}) \subset \mathbb{R}
  \quad \Longleftrightarrow \quad
  \rho = \tfrac{1}{2} + i\gamma, \quad \gamma \in \mathbb{R}.
  \]
  Determinantal vanishing implies spectral inclusion via analytic continuation and Fredholm theory, without requiring an explicit spectral bijection.

  \item \textit{Spectral Symmetry.}  
  The functional equation \( \Xi(\tfrac{1}{2} + i\lambda) = \Xi(\tfrac{1}{2} - i\lambda) \) implies that the spectrum is symmetric:
  \[
  \mu \in \operatorname{Spec}(L_{\mathrm{sym}}) \quad \Rightarrow \quad -\mu \in \operatorname{Spec}(L_{\mathrm{sym}}),
  \]
  with equal multiplicity. This symmetry follows directly from the evenness of the canonical Fourier profile \( \phi(\lambda) := \Xi(\tfrac{1}{2} + i\lambda) \).

  \item \textit{Trace Positivity.}  
  The spectral trace pairing
  \[
  \phi \mapsto \operatorname{Tr}(\phi(L_{\mathrm{sym}}))
  \]
  defines a positive tempered distribution on \( \mathbb{R} \). In particular,
  \[
  \operatorname{Tr}(e^{-tL_{\mathrm{sym}}^2}) \ge 0 \quad \text{for all } t > 0,
  \]
  reflecting positivity of the spectral measure under heat kernel regularization. This distributional positivity holds for all nonnegative test functions \( \phi \in \mathcal{S}(\mathbb{R}) \), supporting a harmonic-analytic interpretation of the spectral trace.

  \item \textit{Analytic Independence.}  
  All results in this chapter are derived using classical analytic techniques:
  \begin{itemize}
    \item spectral theory for compact self-adjoint operators;
    \item Gaussian decay and trace-class convergence of \( e^{-tL_{\mathrm{sym}}^2} \);
    \item Hadamard factorization and functional identity of \( \Xi(s) \);
    \item Fredholm determinant theory and zero-to-spectrum transfer;
    \item and spectral unitarity from orthonormal diagonalization.
  \end{itemize}
  No appeal is made to modular forms, trace formulas, or Langlands theory.
\end{itemize}
