\begin{theorem}[Spectral Reformulation of the Riemann Hypothesis]
\label{thm:eq_of_rh}
Let \( \Lsym \in \TC(\HPsi) \cap \KC(\HPsi) \) denote the canonical compact, self-adjoint, trace-class operator on the exponentially weighted Hilbert space
\[
\HPsi := L^2(\R, e^{\alpha |x|} dx), \qquad \alpha > \pi,
\]
constructed via trace-norm limits of symmetric mollified convolution operators as described in Chapters~\ref{sec:operator_construction}–\ref{sec:heat_kernel_asymptotics} and Appendix~\ref{app:heat_kernel_construction}.

\medskip

Suppose the associated Carleman \(\zeta\)-regularized Fredholm determinant satisfies the canonical identity:
\[
\detz(I - \lambda \Lsym) = \frac{\Xi\left( \tfrac{1}{2} + i\lambda \right)}{\Xi\left( \tfrac{1}{2} \right)},
\qquad \forall \lambda \in \C,
\]
as proven in \thmref{thm:det_identity_revised}.

\medskip

Then the Riemann Hypothesis is equivalent to the spectral reality of \( \Lsym \):
\[
\RH \quad \Longleftrightarrow \quad \Spec(\Lsym) \subset \R.
\]

\medskip

\noindent
More precisely, for each nontrivial zero \( \rho = \beta + i\gamma \) of \( \zeta(s) \), define the canonical spectral image:
\[
\mu_\rho := \frac{1}{i}(\rho - \tfrac{1}{2}) = \gamma \in \C.
\]
Then:
\begin{itemize}
  \item \( \mu_\rho \in \R \) if and only if \( \operatorname{Re}(\rho) = \tfrac{1}{2} \);
  \item Therefore,
  \[
  \Spec(\Lsym) \subset \R
  \quad \Longleftrightarrow \quad
  \text{All nontrivial zeros } \rho \text{ lie on the critical line}.
  \]
\end{itemize}

\medskip

\noindent
This equivalence follows from:
\begin{enumerate}
  \item The canonical determinant identity for \( \Lsym \);
  \item The bijective, multiplicity-preserving spectral map
  \[
  \rho \longmapsto \mu_\rho := \tfrac{1}{i}(\rho - \tfrac{1}{2}),
  \]
  proven in \thmref{thm:spectral_zero_bijection_revised}, which identifies the nontrivial zeros of \( \zeta(s) \) with the nonzero spectrum of \( \Lsym \);
  \item The algebraic implication \( \mu_\rho \in \R \iff \Re(\rho) = \tfrac{1}{2} \).
\end{enumerate}

\medskip

\noindent
Thus, the Riemann Hypothesis is logically equivalent to the condition that the entire spectrum of a canonical trace-class operator lies on the real line. This establishes a fully operator-theoretic reformulation of RH, grounded in analytic Fredholm theory and zeta-regularized determinant calculus.
\end{theorem}
