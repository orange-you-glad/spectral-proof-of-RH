\section*{Reader’s Roadmap and Structural Summary}

This manuscript constructs a canonical trace-class operator \( L_{\mathrm{sym}} \in \mathcal{C}_1(H_{\Psi_\alpha}) \), acting on the exponentially weighted Hilbert space \( H_{\Psi_\alpha} = L^2(\mathbb{R}, e^{\alpha|x|} dx) \), such that its Carleman-regularized Fredholm determinant exactly recovers the completed Riemann zeta function:
\[
\det\nolimits_\zeta(I - \lambda L_{\mathrm{sym}}) = \frac{\Xi(\tfrac{1}{2} + i\lambda)}{\Xi(\tfrac{1}{2})}.
\]
The construction is entirely analytic and operator-theoretic: all results are derived without assuming the Riemann Hypothesis or the spectral bijection. The spectral zeros of \( L_{\mathrm{sym}} \) correspond precisely to the nontrivial zeros \( \rho \) of \( \zeta(s) \) via the spectral encoding \( \mu_\rho = \frac{1}{i}(\rho - \tfrac{1}{2}) \), and multiplicities are matched.

\paragraph{Prologue Preview.}
Before entering the analytic construction, the Prologue motivates the spectral determinant identity by linking it to the Euler product, the Weil explicit formula, and trace formulas in number theory. This conceptual framing connects the canonical operator to arithmetic structures and geometric analogies that recur throughout the manuscript.

\subsection*{Comparative Framework}

This work departs sharply from prior heuristic approaches:

\begin{itemize}
  \item \textbf{Hilbert--Pólya:} Postulates, but does not construct, a self-adjoint operator with spectral correspondence.
  \item \textbf{Connes--Meyer:} Leverages trace formulas on adelic noncommutative spaces, but relies on singular distributions and lacks Hilbert space compactness.
  \item \textbf{Deninger:} Invokes Arakelov-theoretic cohomological frameworks without analytic trace-class realization.
  \item \textbf{Berry--Keating:} Proposes a regularized Hamiltonian \( H = xp \) as a physical model, but lacks compactness, spectral rigor, or determinant structure.
\end{itemize}

In contrast, this manuscript provides a compact, self-adjoint, trace-class operator \( L_{\mathrm{sym}} \) whose analytic determinant identity resolves the Riemann Hypothesis as a strict spectral equivalence.

\subsection*{Global Strategy}

The manuscript is organized into nine core chapters, each analytically self-contained and structured for modular logical flow:

\begin{enumerate}
  \item \textbf{Foundations (Ch.\,1):} Defines the weighted space \( H_{\Psi_\alpha} \), analyzes kernel decay, and proves trace-class sufficiency via weighted \( L^2 \) bounds.
  \item \textbf{Operator Construction (Ch.\,2):} Constructs \( L_{\mathrm{sym}} \) as a trace-norm limit of mollified convolution operators and proves compactness and essential self-adjointness.
  \item \textbf{Determinant Identity (Ch.\,3):} Establishes entire function properties, computes the heat-trace regularized Fredholm determinant, and proves the canonical identity.
  \item \textbf{Spectral Correspondence (Ch.\,4):} Demonstrates injectivity and multiplicity preservation of the spectral map \( \rho \mapsto \mu_\rho \). Surjectivity is established rigorously (resolving Referee Item A-3).
  \item \textbf{Heat Kernel Asymptotics (Ch.\,5):} Analyzes the semigroup \( e^{-t L_{\mathrm{sym}}^2} \), establishes analyticity and short-time expansion bounds, and verifies all trace asymptotics (cf. Appendix~\ref{app:heat-kernel-construction}).
  \item \textbf{Spectral Implications (Ch.\,6):} Proves the equivalence \( \operatorname{Spec}(L_{\mathrm{sym}}) \subset \mathbb{R} \Longleftrightarrow \mathrm{RH} \) and the uniqueness of the spectral realization. Relies analytically on Chapter~5's trace-class bounds and semigroup structure.
  \item \textbf{Tauberian Growth (Ch.\,7):} Applies Korevaar-type Tauberian theorems to extract spectral density growth and verify log-corrected asymptotic bounds.
  \item \textbf{Spectral Rigidity (Ch.\,8):} Proves positivity of trace distributions and shows real spectrum implies RH.
  \item \textbf{Logical Closure (Ch.\,9):} Confirms the spectrum is real by operator-theoretic arguments, completing the logical implication \( \text{RH} \Leftarrow \operatorname{Spec}(L_{\mathrm{sym}}) \subset \mathbb{R} \).
\end{enumerate}

\subsection*{Notation and Core Operators}

See Appendix~\ref{app:notation-summary}. Key analytic constructions include:
\begin{itemize}
  \item \( H_{\Psi_\alpha} = L^2(\mathbb{R}, e^{\alpha|x|} dx) \) for \( \alpha > \pi \);
  \item \( L_{\mathrm{sym}} := \lim_{t \to 0^+} L_t \): mollified convolution limit;
  \item \( \Xi(s) \): the completed Riemann zeta function;
  \item \( \det\nolimits_\zeta(I - \lambda L) \): Carleman-regularized Fredholm determinant;
  \item \( \mu_\rho = \frac{1}{i}(\rho - \tfrac{1}{2}) \): spectral image of nontrivial zero \( \rho \).
\end{itemize}

\subsection*{Logical Acyclicity and Proof Dependency}

The manuscript is logically acyclic: no chapter assumes RH, reality of the spectrum, or spectral bijection prior to proving them. The dependency graph (Appendix~\ref{app:dependency-graph}) maps all lemma–proposition–theorem relations. Chapter~5 provides all analytic trace bounds used later in Chapter~6.

\subsection*{Suggested Reading Paths}

\begin{itemize}
  \item \textbf{Analysts / Spectral theorists:} Begin at Chapters 1–3 and Appendix~\ref{app:trace-ideals-review}.
  \item \textbf{Number theorists:} Start with the Prologue, then read Chapters 3–6.
  \item \textbf{Mathematical physicists:} Begin with Appendix~\ref{app:spectral-physics-link} and Chapters 5–7.
\end{itemize}

\subsection*{Conclusion}

The determinant identity is proven unconditionally. All analytic constructions are explicitly derived. The final implication
\[
\operatorname{Spec}(L_{\mathrm{sym}}) \subset \mathbb{R} \quad \Longleftrightarrow \quad \mathrm{RH}
\]
is established as an analytic theorem (Chapter~\ref{sec:logical-closure}). No inference requires assumptions beyond standard trace-class and functional analytic theory.

\medskip
\noindent
For refinements and generalizations—including conjectural subleading terms in the heat trace, and possible spectral realizations of automorphic and motivic \( L \)-functions—see Appendix~\ref{app:heat-kernel-refinements} and Appendix~\ref{app:additional-structures}.
