\subsection*{Summary}

This chapter constructed the canonical operator \( L_{\sym} \in \TC(H_{\Psi_\alpha}) \) as the trace-norm limit of mollified convolution operators. The core results are organized as follows:

\paragraph{Function Spaces and Spectral Input}
\begin{itemize}
  \item \defref{def:weighted_fourier_space} — Exponentially weighted Hilbert space \( H_{\Psi_\alpha} := L^2(\R, e^{\alpha |x|} dx) \), where \( \alpha > \pi \) ensures trace-class decay.
  \item \defref{def:canonical_fourier_profile} — Canonical spectral profile \( \phi(\lambda) := \Xi(\tfrac{1}{2} + i\lambda) \), entire of exponential type \( \pi \).
  \item \defref{def:mollified_fourier_profile} — Mollified profiles \( \varphi_t(\lambda) := e^{-t\lambda^2} \phi(\lambda) \) lie in \( \Schwartz \) and regulate high-frequency tails.
  \item \defref{def:convolution_operators_Lt_Lsym} — Definition of convolution operators \( L_t \), with trace-norm limit \( L_{\sym} := \lim_{t \to 0^+} L_t \in \TC(H_{\Psi_\alpha}) \).
\end{itemize}

\paragraph{Kernel Estimates and Convergence}
\begin{itemize}
  \item \lemref{lem:mollified_profile_decay} — Uniform control of inverse Fourier kernels \( k_t \); exponential decay inherited from \( \phi \in \PW{\pi} \).
  \item \lemref{lem:trace_class_Lt} — Trace-class inclusion of \( L_t \in \TC(H_{\Psi_\alpha}) \) via Simon's exponential kernel criterion.
  \item \lemref{lem:trace_norm_convergence_Lt_to_Lsym}, \lemref{lem:trace_norm_rate_convergence} — Convergence \( \|L_t - L_{\sym}\|_{\TC} \to 0 \) with quantitative rate \( \lesssim t^\beta \).
  \item \lemref{lem:trace_norm_limit_unique}, \lemref{lem:construction_canonical_data} — Canonicality: limit operator \( L_{\sym} \) is unique and mollifier-independent.
  \item \lemref{lem:mollifier_independence_kernel_limit} — Convergence of kernel sequences is invariant under mollifier choice.
\end{itemize}

\paragraph{Operator Properties and Domain Closure}
\begin{itemize}
  \item \lemref{lem:boundedness_Lsym} — Boundedness: \( L_{\sym} \in \mathcal{B}(H_{\Psi_\alpha}) \) with operator norm control.
  \item \lemref{lem:core_essential_sa}, \lemref{lem:Lsym_square_selfadjoint} — Schwartz space is a common core; both \( L_{\sym} \) and \( L_{\sym}^2 \) are essentially self-adjoint.
  \item \remref{rem:selfadjoint_analytic_vectors} — Analytic vectors via mollifier convergence justify self-adjointness using Nelson’s theorem.
  \item \lemref{lem:heat_semigroup_existence} — Semigroup generation: \( \{ e^{-t L_{\sym}^2} \}_{t > 0} \subset \TC(H_{\Psi_\alpha}) \) is trace-class and holomorphic in \( t \).
\end{itemize}

\paragraph{Canonical Classification}
\begin{itemize}
  \item \thmref{thm:existence_Lsym} — Existence: \( L_{\sym} \in \TC(H_{\Psi_\alpha}) \) realized as the trace-norm limit of \( L_t \).
  \item \thmref{thm:sa_trace_class_Lsym} — Final classification: \( L_{\sym} \) is compact, self-adjoint, and trace class.
  \item \thmref{thm:trace_zero_Lsym} — Spectral centering: \( \Tr(L_{\sym}) = 0 \) ensures canonical normalization of the determinant.
\end{itemize}

\paragraph{Chapter Closure.}
This chapter delivers the full analytic construction of the canonical operator \( L_{\sym} \), with trace-class regularity, semigroup structure, and self-adjoint convergence established from first principles. Built from mollified inverse Fourier transforms of the completed zeta function \( \Xi \), this operator satisfies all criteria for encoding zeta zero data spectrally.

It underpins the determinant identity
\[
\det\nolimits_{\zeta}(I - \lambda^2 L_{\sym}^2) = \Xi\left(\tfrac{1}{2} + i\lambda\right),
\]
which is proven in Chapter~\ref{sec:determinant_identity}.
