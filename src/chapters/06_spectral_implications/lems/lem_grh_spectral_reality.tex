\begin{lemma}[Spectral Reality Implies Generalized Riemann Hypothesis]
\label{lem:grh_spectral_reality}

Let \( L_{\sym}^{(\pi)} \) be the canonical compact, self-adjoint, trace-class operator associated with the automorphic L-function \( L(s, \pi) \) for \( \pi \in \mathcal{A}_{\text{cusp}}(\mathrm{GL}_n) \). If the spectrum of \( L_{\sym}^{(\pi)} \) is real, i.e., \( \Spec(L_{\sym}^{(\pi)}) \subset \mathbb{R} \), then the nontrivial zeros of the automorphic L-function \( L(s, \pi) \) all lie on the critical line \( \Re(s) = \frac{1}{2} \). That is, the \emph{Generalized Riemann Hypothesis} (GRH) holds for \( L(s, \pi) \).

\end{lemma}
