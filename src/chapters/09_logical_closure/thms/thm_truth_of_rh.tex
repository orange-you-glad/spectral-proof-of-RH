\begin{theorem}[Truth of the Riemann Hypothesis]
\label{thm:truth_of_rh}

Every nontrivial zero \( \rho \in \C \) of the Riemann zeta function \( \zetaR(s) \) satisfies
\[
\Re(\rho) = \tfrac{1}{2}.
\]

\medskip

\noindent
That is, the nontrivial zero set of \( \zetaR(s) \) lies entirely on the critical line:
\[
\Spec(\zetaR) \subset \left\{ s \in \C : \Re(s) = \tfrac{1}{2} \right\}.
\]

\medskip

\noindent
This completes the analytic proof of the Riemann Hypothesis, via the canonical determinant identity and the spectral theory of the self-adjoint, trace-class operator \( L_{\sym} \in \TC(\HPsi) \). Under the spectral map
\[
\mu_\rho := \frac{1}{i(\rho - \tfrac{1}{2})},
\]
the bijective correspondence between zeta zeros and spectrum, together with the real-valuedness of \( \Spec(L_{\sym}) \), implies that all \( \rho \in \Spec(\zetaR) \) satisfy \( \Re(\rho) = \tfrac{1}{2} \).
\end{theorem}
