\begin{proposition}[Spectral Reality Implies RH and Simplicity]
\label{prop:spectrum_reality_implies_rh_multiplicity}

Let \( L_{\sym} \in \TC(\HPsi) \) be the canonical compact, self-adjoint operator associated with the completed Riemann zeta function \( \XiR(s) \), and define its spectral determinant:
\[
f(\lambda) := \det\nolimits_\zeta(I - \lambda L_{\sym}) = \frac{\XiR\left( \tfrac{1}{2} + i\lambda \right)}{\XiR\left( \tfrac{1}{2} \right)}.
\]

Assume:
\begin{enumerate}
  \item[\textup{(i)}] The spectrum of \( L_{\sym} \) lies entirely on the real axis:
  \[
  \Spec(L_{\sym}) \subset \R.
  \]

  \item[\textup{(ii)}] Each nonzero eigenvalue \( \mu \in \Spec(L_{\sym}) \setminus \{0\} \) is simple (algebraic multiplicity one).
\end{enumerate}

Then every nontrivial zero \( \rho \) of the Riemann zeta function satisfies:
\[
\Re(\rho) = \tfrac{1}{2}, \qquad \operatorname{ord}_\rho(\zetaR) = 1,
\]
i.e., the Riemann Hypothesis holds and all nontrivial zeros are simple.
\end{proposition}
