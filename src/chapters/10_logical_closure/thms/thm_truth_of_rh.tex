\begin{theorem}[Equivalence of the Riemann Hypothesis with Spectral Reality]
\label{thm:truth_of_rh}

The Riemann Hypothesis is equivalent to the spectral reality of the canonical convolution operator
\[
\Lsym \in \TC(\HPsi),
\]
constructed in \secref{sec:operator_construction}. Explicitly,
\[
\RH \iff \Spec(\Lsym) \subset \R.
\]

\medskip

\noindent
This equivalence follows from the analytic realization of the completed zeta function
\[
\detz(I - \lambda \Lsym) = \frac{\Xi\left( \tfrac{1}{2} + i\lambda \right)}{\Xi\left( \tfrac{1}{2} \right)},
\]
and the bijective correspondence
\[
\rho \mapsto \mu_\rho := \frac{1}{i}(\rho - \tfrac{1}{2}),
\]
between nontrivial zeros \( \rho \in \C \) of \( \zeta(s) \) and the nonzero spectrum of \( \Lsym \), with multiplicities preserved.

\medskip

\noindent
All analytic inputs—Paley–Wiener decay, trace-norm convergence, self-adjointness, determinant regularization, and positivity—have been rigorously established in Chapters~\ref{sec:foundations}–\ref{sec:tauberian_growth}, with logical flow verified in Appendix~\ref{app:dependency_graph}.

\medskip

\noindent
In particular, the analytic infrastructure has been verified in \lemref{lem:trace_class_Lt}, \lemref{lem:heat_trace_expansion}, \lemref{lem:spectral_encoding_injection}, \lemref{lem:trace_distribution_positivity}, \lemref{lem:spectral_symmetry}, and \thmref{thm:spectral_determinant_identity_Lpi}. These collectively guarantee the well-posedness, spectral encoding, trace convergence, and GRH generalization needed to ensure the determinant identity applies to \( \Lsym \).

\medskip

\noindent
Therefore, the Riemann Hypothesis holds if and only if the spectrum of \( \Lsym \) is real.
\end{theorem}
