\begin{lemma}[Short-Time Lower Bound for the Heat Trace]
\label{lem:hk_lower_bound}
Let \( L_{\mathrm{sym}} \in \mathcal{C}_1(H_{\Psi_\alpha}) \) be the canonical compact, self-adjoint operator, and define its trace norm
\[
C_1 := \| L_{\mathrm{sym}} \|_{\mathcal{C}_1}.
\]
Then for all \( t \in (0,1] \), the spectral heat trace satisfies the lower bound:
\[
\operatorname{Tr}\left( e^{-t L_{\mathrm{sym}}^2} \right) \ge \frac{1}{4C_1} \, t^{-1/2}.
\]

\medskip
\noindent
This estimate reflects the dominant contribution of the low-frequency spectrum in the short-time regime. The constant \( \frac{1}{4C_1} \) is explicit and depends only on the Schatten–1 norm of \( L_{\mathrm{sym}} \). The semigroup regularity and existence of the trace are guaranteed by \lemref{lem:heat_semigroup_wellposed}, and the decay behavior of \( \mu_\rho \to 0 \) from \lemref{lem:spectral_decay_bounds} ensures spectral concentration near the origin. For the corresponding upper bound, see \lemref{lem:hk_upper_bound}.
\end{lemma}
