\begin{proof}[Proof of \lemref{lem:trace_class_conjugated_kernel}]
Let \( T \) be the integral operator defined by
\[
(Tf)(x) := \int_{\R} K(x, y) f(y) \, dy,
\]
initially acting on the dense subspace \( \Schwartz \subset \HPsi := L^2(\R, \Psi_\alpha(x)\, dx) \), where \( \Psi_\alpha(x) := e^{\alpha |x|} \), with \( \alpha > 0 \). Assume that
\[
\iint_{\R^2} |K(x, y)|\, \Psi_\alpha(x) \Psi_\alpha(y)\, dx\, dy < \infty.
\]

\medskip
\noindent\textbf{Step 1: Unitary Conjugation to Flat \( L^2 \).}
Define the unitary map
\[
U \colon \HPsi \to L^2(\R), \qquad (Uf)(x) := \Psi_\alpha(x)^{1/2} f(x).
\]
Then \( U \) is an isometric isomorphism, with inverse \( (U^{-1}g)(x) = \Psi_\alpha(x)^{-1/2} g(x) \).

The conjugated operator \( \widetilde{T} := U T U^{-1} \) acts on \( L^2(\R) \) via the kernel
\[
\widetilde{K}(x, y) := \frac{K(x,y)}{\sqrt{\Psi_\alpha(x) \Psi_\alpha(y)}}.
\]

\medskip
\noindent\textbf{Step 2: Trace-Class Criterion in Flat \( L^2 \).}
Since
\[
\iint_{\R^2} |\widetilde{K}(x, y)|\, dx\, dy = \iint_{\R^2} |K(x, y)|\, \Psi_\alpha(x) \Psi_\alpha(y)\, dx\, dy < \infty,
\]
we apply Simon’s trace-class criterion \cite[Thm.~4.2]{Simon2005TraceIdeals}. Therefore,
\[
\widetilde{T} \in \mathcal{C}_1(L^2(\R)), \quad \text{with } \|\widetilde{T}\|_{\mathcal{C}_1} \le \iint_{\R^2} |\widetilde{K}(x, y)|\, dx\, dy.
\]

\medskip
\noindent\textbf{Step 3: Transfer Back to Weighted Space.}
Since \( T = U^{-1} \widetilde{T} U \) and \( U \) is unitary, we conclude:
\[
T \in \mathcal{C}_1(\HPsi), \quad \|T\|_{\mathcal{C}_1(\HPsi)} = \|\widetilde{T}\|_{\mathcal{C}_1(L^2)} \le \iint_{\R^2} |K(x, y)|\, \Psi_\alpha(x)\Psi_\alpha(y)\, dx\, dy.
\]

\medskip
\noindent\textbf{Conclusion.}
The kernel integrability condition implies that the conjugated kernel \( \widetilde{K} \in L^1(\R^2) \), and hence \( T \in \mathcal{C}_1(\HPsi) \). If \( K \) is symmetric, \( T \) is also symmetric on the Schwartz core, which underpins its spectral analysis.
\end{proof}
