\begin{remark}[Functional Calculus for Spectral Trace Pairings]
\label{rem:functional_calculus_trace}
For any \( \phi \in \mathcal{S}(\mathbb{R}) \), the spectral operator \( \phi(L_{\mathrm{sym}}) \) is well-defined via the spectral theorem. Since \( L_{\mathrm{sym}} \in \mathcal{C}_1(H_{\Psi_\alpha}) \), its eigenvalues \( \{ \mu_n \} \subset \mathbb{R} \) satisfy \( \mu_n \to 0 \) and \( \sum_n |\mu_n| < \infty \).

The rapid decay of \( \phi(\mu_n) \) ensures:
\[
\sum_n |\phi(\mu_n)| < \infty,
\]
so the trace
\[
\operatorname{Tr}(\phi(L_{\mathrm{sym}})) := \sum_n \phi(\mu_n)
\]
is absolutely convergent. Therefore, all trace pairings \( \phi \mapsto \operatorname{Tr}(\phi(L_{\mathrm{sym}})) \) used in this chapter are rigorously defined for test functions \( \phi \in \mathcal{S}(\mathbb{R}) \).

\medskip

\noindent
This justifies interpreting \( \operatorname{Tr}(\phi(L_{\mathrm{sym}})) \) as a tempered distribution without requiring additional regularization.
\thmref{thm:canonical_operator_realization}
\lemref{lem:trace_distribution_positive}
\thmref{thm:det_identity_revised}
\end{remark}
