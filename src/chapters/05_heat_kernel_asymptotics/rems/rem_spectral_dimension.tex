\begin{remark}[Spectral Interpretation of Heat Trace Scaling]
\label{rem:spectral_dimension}
The two-sided short-time asymptotic
\[
\Tr(e^{-t L_{\sym}^2}) \asymp t^{-1/2} \qquad \text{as } t \to 0^+
\]
admits a natural spectral interpretation: it reflects an effective spectral dimension \( d = 1 \), consistent with a log-corrected Weyl law.

\medskip
\noindent
Let \( N(\lambda) := \#\{ n : \mu_n^2 \le \lambda \} \) denote the eigenvalue counting function for the squared spectrum of \( L_{\sym} \). Then the asymptotic growth
\[
N(\lambda) \sim C \lambda^{1/2} \log \lambda, \qquad \lambda \to \infty,
\]
implies, via Tauberian inversion (see Chapter~\ref{sec:tauberian_growth}), that
\[
\Tr(e^{-t L_{\sym}^2}) \sim \frac{1}{\sqrt{4\pi t}} \log\left( \frac{1}{t} \right),
\]
as established in \propref{prop:short_time_heat_expansion}.

\medskip
\noindent
Although \( L_{\sym} \) is not a local differential operator, its smoothing kernel inherits exponential localization from the Paley--Wiener class of the spectral profile \( \phi(\lambda) \in \PW{\pi} \). The resulting heat kernel \( K_t(x,y) \) decays Gaussianly in \( |x - y| \) and is real-analytic for all \( t > 0 \), permitting classical short-time analysis up to a logarithmic correction.

\medskip
\noindent
Thus, \( L_{\sym} \) behaves spectrally like a pseudodifferential operator of dimension one, modulated by logarithmic density. This supports the structure of the spectral determinant:
\[
\det\nolimits_\zeta(I - \lambda L_{\sym}) = \exp\left( - \int_0^\infty \frac{\Tr(e^{-t L_{\sym}^2}) - P(t)}{t} \, e^{-\lambda^2 t} \, dt \right),
\]
where \( P(t) \sim \frac{1}{\sqrt{4\pi t}} \log(1/t) + \cdots \) is the parametrix singular term. The logarithmic divergence near \( t = 0^+ \) dictates the entire order and exponential type of the determinant via Laplace--Mellin regularization.
\thmref{thm:canonical_operator_realization}
\thmref{thm:det_identity_revised}
\lemref{lem:heat_trace_expansion}
\end{remark}
