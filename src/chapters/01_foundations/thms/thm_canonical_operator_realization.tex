\begin{theorem}[Canonical Compact Operator and Spectral Realization]
\label{thm:canonical_operator_realization}
Let \( \phi(\lambda) := \Xi\left( \tfrac{1}{2} + i\lambda \right) \) denote the centered spectral profile of the completed Riemann zeta function, and let \( \alpha > \pi \) be fixed. Define the exponentially weighted Hilbert space
\[
\HPsi := L^2(\R, \Psi_\alpha(x)\, dx), \qquad \text{where } \Psi_\alpha(x) := e^{\alpha |x|}.
\]

Construct the mollified convolution operators
\[
L_t f(x) := \int_{\R} \FT^{-1}\left[ e^{-t\lambda^2} \phi(\lambda) \right](x - y)\, f(y)\, dy,
\]
which are real, symmetric, compact operators in \( \TC(\HPsi) \). Then the following hold:

\begin{itemize}
    \item[\textup{(i)}] The trace-norm limit
    \[
    \Lsym := \lim_{t \to 0^+} L_t \in \TC(\HPsi)
    \]
    exists, is self-adjoint, and compact.

    \item[\textup{(ii)}] The zeta-regularized determinant
    \[
    \detz(I - \lambda \Lsym) := \prod_n (1 - \lambda \lambda_n)\, e^{\lambda \lambda_n}
    = \frac{\Xi\left( \tfrac{1}{2} + i\lambda \right)}{\Xi\left( \tfrac{1}{2} \right)}
    \]
    is entire of order one and encodes the nontrivial zero set of \( \zeta(s) \) via the spectral realization
    \[
    \Spec(\Lsym) = \left\{ \mu_\rho := \tfrac{1}{i}(\rho - \tfrac{1}{2}) \,\middle|\, \zeta(\rho) = 0,\; \Re(\rho) = \tfrac{1}{2} \right\}.
    \]

    \item[\textup{(iii)}] The space \( \Schwartz \subset \HPsi \) is a core for \( \Lsym \), and the kernel of \( \Lsym \) is symmetric, real, and exponentially decaying off the diagonal.
\end{itemize}

\noindent
This operator \( \Lsym \) is the analytic centerpiece of the spectral determinant identity developed in Chapters~\ref{sec:determinant_identity} through~\ref{sec:spectral_implications}, and governs the spectral encoding of the Riemann Hypothesis via its real spectrum.
\end{theorem}
