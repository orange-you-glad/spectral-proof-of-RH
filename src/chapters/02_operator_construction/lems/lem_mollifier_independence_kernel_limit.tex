\begin{lemma}[Mollifier Independence of Canonical Kernel Limit]
\label{lem:mollifier_independence_kernel_limit}
Let \( \alpha > \pi \), and define the exponentially weighted Hilbert space
\[
H_{\Psi_\alpha} := L^2(\mathbb{R}, \Psi_\alpha(x)\, dx), \qquad \Psi_\alpha(x) := e^{\alpha |x|}.
\]

Let \( \widehat{\Xi} \in \mathcal{S}'(\mathbb{R}) \) denote the inverse Fourier transform of the canonical spectral profile
\[
\phi(\lambda) := \Xi\left( \tfrac{1}{2} + i\lambda \right),
\]
where \( \Xi \) is the completed Riemann zeta function.

Let \( \{\varphi_t\}_{t > 0} \subset \mathcal{S}(\mathbb{R}) \) be any mollifier family satisfying:
\begin{itemize}
    \item \textnormal{(Normalization)} \( \int_{\mathbb{R}} \varphi_t(x)\, dx = 1 \);
    \item \textnormal{(Approximate Identity)} \( \varphi_t \to \delta \) in \( \mathcal{S}' \) as \( t \to 0^+ \);
    \item \textnormal{(Symmetry)} \( \varphi_t(x) = \varphi_t(-x) \);
    \item \textnormal{(Decay)} \( \varphi_t \in L^1 \cap L^2 \cap L^1(\Psi_\alpha\, dx) \) for all \( t > 0 \).
\end{itemize}

Define the mollified kernels and corresponding convolution operators:
\[
\widehat{\Xi}_t := \varphi_t * \widehat{\Xi}, \qquad
(L_t^{(\varphi)} f)(x) := \int_{\mathbb{R}} \widehat{\Xi}_t(x - y) f(y)\, dy.
\]

Then:
\begin{enumerate}
    \item[\textnormal{(i)}] For each \( t > 0 \), the mollified kernel \( \widehat{\Xi}_t \in \mathcal{S}(\mathbb{R}) \cap L^1(\mathbb{R}, \Psi_\alpha(x)\, dx) \), and the convolution operator \( L_t^{(\varphi)} \in \mathcal{C}_1(H_{\Psi_\alpha}) \) is trace class. This follows from classical kernel decay and weighted integral criteria for trace-class operators \cite[Ch.~4]{Simon2005TraceIdeals}.

    \item[\textnormal{(ii)}] The limit
    \[
    L_{\mathrm{sym}} := \lim_{t \to 0^+} L_t^{(\varphi)} \quad \text{in } \mathcal{C}_1(H_{\Psi_\alpha})
    \]
    exists and is independent of the mollifier family \( \{\varphi_t\} \). That is, for any two such families \( \varphi_t \), \( \tilde{\varphi}_t \), we have
    \[
    \lim_{t \to 0^+} \| L_t^{(\varphi)} - L_t^{(\tilde{\varphi})} \|_{\mathcal{C}_1} = 0.
    \]
    This follows from convolution continuity in trace-norm and the fact that mollifiers converge to the identity in \( \mathcal{S}' \), preserving symmetry and exponential integrability.
\end{enumerate}

\noindent
Hence, the canonical operator \( L_{\mathrm{sym}} \in \mathcal{C}_1(H_{\Psi_\alpha}) \) is uniquely defined by the analytic profile \( \phi = \Xi(1/2 + i\lambda) \) and the exponential weight \( \Psi_\alpha \), independent of mollification choices. This confirms the analytic rigidity of the construction and provides the foundation for the determinant identity in \defref{def:convolution-operators-Lt-Lsym}.
\end{lemma}
% 