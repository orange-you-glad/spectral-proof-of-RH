\begin{definition}[Compact Operators]\label{def:compact_operator}
Let \( H \) be a complex separable Hilbert space, and denote by \( \mathcal{B}(H) \) the Banach algebra of bounded linear operators on \( H \), equipped with the operator norm
\[
\|T\| := \sup_{\|x\| = 1} \|Tx\|.
\]

An operator \( T \in \mathcal{B}(H) \) is called \emph{compact} if it satisfies any (and hence all) of the following equivalent conditions:
\begin{enumerate}
    \item[\textup{(i)}] The image of the closed unit ball \( B_H := \{ x \in H : \|x\| \leq 1 \} \) under \( T \) has compact closure in the norm topology of \( H \).
    \item[\textup{(ii)}] For every bounded sequence \( \{x_n\} \subset H \), the sequence \( \{Tx_n\} \) has a convergent subsequence.
    \item[\textup{(iii)}] There exists a sequence \( \{T_n\} \subset \mathcal{F}(H) \) of finite-rank operators such that \( \|T - T_n\| \to 0 \) as \( n \to \infty \).
\end{enumerate}

The collection of compact operators on \( H \) is denoted \( \mathcal{K}(H) \). It is a norm-closed, two-sided *-ideal in \( \mathcal{B}(H) \), and satisfies:
\[
\mathcal{F}(H) \subset \mathcal{K}(H) \subset \mathcal{C}_p(H) \quad \text{for all } p > 0,
\]
where \( \mathcal{C}_p(H) \) denotes the Schatten \( p \)-class of compact operators.

\medskip
\noindent\textbf{Singular Value Decomposition.}
Every compact operator \( T \in \mathcal{K}(H) \) admits a singular value expansion:
\[
T = \sum_{n=1}^\infty s_n \langle \cdot, f_n \rangle e_n,
\]
where \( \{e_n\}, \{f_n\} \subset H \) are orthonormal systems, and \( s_n \geq 0 \) are the singular values of \( T \), with \( s_n \to 0 \) as \( n \to \infty \). The series converges in operator norm.

\medskip
\noindent\textbf{Spectral Properties.}
If \( T \in \mathcal{K}(H) \), then:
\begin{itemize}
    \item The spectrum \( \sigma(T) \subset \C \) is at most countable.
    \item Every nonzero \( \lambda \in \sigma(T) \) is an eigenvalue of finite multiplicity.
    \item The only possible accumulation point of \( \sigma(T) \) is zero.
    \item The resolvent set \( \rho(T) := \C \setminus \sigma(T) \) is open.
\end{itemize}

\medskip
\noindent\textbf{Ideal and Closure Properties.}
\begin{itemize}
    \item \( \mathcal{K}(H) \) is the operator-norm closure of the rank-one operators.
    \item If \( A \in \mathcal{B}(H) \) and \( K \in \mathcal{K}(H) \), then \( AK, KA \in \mathcal{K}(H) \).
    \item Compactness is preserved under bounded left and right multiplication.
\end{itemize}

\medskip
\noindent\textbf{Contextual Role.}
Compactness plays a central role in spectral discreteness, Schatten inclusion, and determinant theory. In this manuscript, the mollified convolution operators \( L_t \) are compact on the weighted Hilbert space \( H_{\Psi_\alpha} := L^2(\R, e^{\alpha |x|} dx) \) for \( \alpha > \pi \); see \propref{prop:compactness_Lt}, \lemref{lem:kernel_L2_weighted_bound}. This compactness enables:

\begin{itemize}
    \item Construction of the limit operator \( L_{\sym} \in \mathcal{C}_1(H_{\Psi_\alpha}) \),
    \item Validity of the canonical Fredholm determinant \( \det_\zeta(I - \lambda L_{\sym}) \),
    \item Access to full spectral asymptotics via Tauberian theory (cf. \secref{sec:tauberian_growth}).
\end{itemize}
\thmref{thm:canonical_operator_realization}

\medskip
\noindent\textbf{References.}
\begin{itemize}
    \item M.~Reed and B.~Simon, \emph{Methods of Modern Mathematical Physics I: Functional Analysis}, Theorem~VI.10~\cite{ReedSimon1980I}.
    \item B.~Simon, \emph{Trace Ideals and Their Applications}, Chapter~3 \cite{Simon2005TraceIdeals}.
\end{itemize}
\end{definition}
