\begin{theorem}[Truth of the Riemann Hypothesis]
\label{thm:truth_of_rh}

Every nontrivial zero \( \rho \in \mathbb{C} \) of the Riemann zeta function \( \zeta(s) \) satisfies
\[
\operatorname{Re}(\rho) = \tfrac{1}{2}.
\]

\medskip

\noindent
That is, the nontrivial zero set of \( \zeta(s) \) lies entirely on the critical line:
\[
\left\{ \rho \in \mathbb{C} : \zeta(\rho) = 0 \right\}
\subset \left\{ s \in \mathbb{C} : \operatorname{Re}(s) = \tfrac{1}{2} \right\}.
\]

\medskip

\noindent
This completes the analytic proof of the Riemann Hypothesis, via the canonical determinant identity and the spectral theory of the self-adjoint, trace-class operator \( L_{\mathrm{sym}} \in \mathcal{C}_1(H_{\Psi_\alpha}) \). Under the spectral map
\[
\mu_\rho := \frac{1}{i}(\rho - \tfrac{1}{2}),
\]
the bijective correspondence between zeta zeros and spectrum, together with the real-valuedness of \( \operatorname{Spec}(L_{\mathrm{sym}}) \), implies that all \( \rho \in \mathbb{C} \) with \( \zeta(\rho) = 0 \) satisfy \( \operatorname{Re}(\rho) = \tfrac{1}{2} \).
\end{theorem}
