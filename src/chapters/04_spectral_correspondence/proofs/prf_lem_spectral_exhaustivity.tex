\begin{proof}[Proof of Lemma~\ref{lem:spectral_exhaustivity}]
Let \( \{ \mu_n \} \subset \Spec(L_{\mathrm{sym}}) \setminus \{0\} \) denote the nonzero eigenvalues of the canonical compact, self-adjoint operator \( L_{\mathrm{sym}} \), counted with algebraic multiplicity.

\paragraph{Step 1: Determinant Zeros Correspond to Zeta Zeros.}
By Theorem~\ref{thm:det_identity_revised}, we have the exact identity
\[
\det\nolimits_\zeta(I - \lambda L_{\mathrm{sym}}) = \frac{\Xi(\tfrac{1}{2} + i\lambda)}{\Xi(\tfrac{1}{2})}.
\]
The right-hand side vanishes precisely at those \( \lambda_\rho := i(\rho - \tfrac{1}{2}) \in \mathbb{C} \) for which \( \rho \) is a nontrivial zero of \( \zeta(s) \). The order of vanishing equals the multiplicity of the zero of \( \zeta \) at \( \rho \).

\paragraph{Step 2: Spectral Reciprocity.}
For operators in the trace class \( \mathcal{C}_1 \), the Fredholm determinant \(\det_\zeta(I - \lambda L)\) satisfies the spectral identity:
\[
\log \det\nolimits_\zeta(I - \lambda L) = -\sum_{n=1}^\infty \log(1 - \lambda \mu_n),
\]
analytically continued where necessary. Hence, the determinant vanishes at \( \lambda = \lambda_\rho \) if and only if \( \mu_\rho := \lambda_\rho^{-1} = \frac{1}{i(\rho - \tfrac{1}{2})} \) lies in \( \Spec(L_{\mathrm{sym}}) \setminus \{0\} \), and the order of vanishing matches the algebraic multiplicity of the corresponding eigenvalue.

\paragraph{Step 3: Exhaustivity.}
Since \( \Xi(s) \) has no other zeros than those inherited from the nontrivial zeros of \( \zeta(s) \), and the determinant’s zero set fully encodes the spectrum of \( L_{\mathrm{sym}} \) away from zero, it follows that every nonzero eigenvalue of \( L_{\mathrm{sym}} \) must arise from some \( \rho \). The zero eigenvalue is excluded by normalization: \(\Xi(\tfrac{1}{2}) \neq 0\) ensures \(\det_\zeta(I) = 1 \Rightarrow 0 \notin \Spec(L_{\mathrm{sym}})\).

\medskip
\noindent
Thus, the map
\[
\rho \mapsto \mu_\rho := \frac{1}{i(\rho - \tfrac{1}{2})}
\]
is surjective onto the nonzero spectrum of \( L_{\mathrm{sym}} \), preserving algebraic multiplicities.
\end{proof}
%  