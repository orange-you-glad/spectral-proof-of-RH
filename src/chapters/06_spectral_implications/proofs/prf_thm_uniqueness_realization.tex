\begin{proof}[Proof of Theorem~\ref{thm:uniqueness_realization}]
Let \( L \in \mathcal{S}_1(H_{\Psi_\alpha}) \cap \mathcal{K}(H_{\Psi_\alpha}) \) be a compact, self-adjoint trace-class operator on the weighted Hilbert space \( H_{\Psi_\alpha} := L^2(\mathbb{R}, e^{\alpha|x|}dx) \) for fixed \( \alpha > \pi \). Assume:
\[
\det\nolimits_\zeta(I - \lambda L) = \frac{\Xi\left(\tfrac{1}{2} + i\lambda\right)}{\Xi\left(\tfrac{1}{2}\right)}
= \det\nolimits_\zeta(I - \lambda L_{\mathrm{sym}}),
\quad \text{for all } \lambda \in \mathbb{C}.
\]

\paragraph{Step 1: Spectral Multiplicity from Determinant Identity.}
By analytic determinant theory for trace-class operators (see~\cite[Thm. 4.2]{Simon2005TraceIdeals}), the Fredholm determinant
\[
\det\nolimits_\zeta(I - \lambda L)
= \prod_{n=1}^\infty (1 - \lambda \mu_n),
\]
encodes the nonzero eigenvalues \( \{ \mu_n \} \subset \mathbb{R} \setminus \{0\} \), counted with multiplicity. Since the determinant of \( L \) agrees identically with that of \( L_{\mathrm{sym}} \), and both are entire functions of order one with the same normalization at \( \lambda = 0 \), we conclude:
\[
\Spec(L) = \Spec(L_{\mathrm{sym}}),
\quad \text{as multisets}.
\]

\paragraph{Step 2: Application of Spectral Theorem.}
Both \( L \) and \( L_{\mathrm{sym}} \) are compact and self-adjoint, and they act on the same separable Hilbert space \( H_{\Psi_\alpha} \). By the spectral theorem for compact self-adjoint operators (see~\cite[Thm. VI.16]{ReedSimon1980I}), any two such operators with matching discrete spectra (counted with multiplicities) are unitarily equivalent. Therefore, there exists a unitary operator
\[
U \colon H_{\Psi_\alpha} \to H_{\Psi_\alpha}
\quad \text{such that} \quad
L = U L_{\mathrm{sym}} U^{-1}.
\]

\paragraph{Conclusion.}
The operator \( L_{\mathrm{sym}} \) is thus unique, up to unitary equivalence, among all compact, self-adjoint, trace-class operators whose zeta-regularized Fredholm determinant realizes the normalized spectral identity associated with \( \Xi(s) \). This completes the proof.
\end{proof}
%  