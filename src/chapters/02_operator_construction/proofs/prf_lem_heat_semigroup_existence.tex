\begin{proof}[Proof of \lemref{lem:heat_semigroup_existence}]
Let \( L := L_{\mathrm{sym}} \in \mathcal{C}_1(H_{\Psi_\alpha}) \) be the canonical self-adjoint operator constructed as the trace-norm limit of mollified convolution operators with symmetric kernels.

\paragraph{(i) Positivity and self-adjointness of \( L^2 \).}
Since \( L \) is self-adjoint on \( H_{\Psi_\alpha} \), it follows that \( L^2 \) is also self-adjoint, with domain \( \mathcal{D}(L^2) \subset H_{\Psi_\alpha} \) dense. Moreover, for any \( f \in \mathcal{D}(L) \), we have
\[
\langle L^2 f, f \rangle = \langle Lf, Lf \rangle = \| Lf \|^2 \ge 0,
\]
so \( L^2 \ge 0 \) in the sense of quadratic forms. Therefore, \( L^2 \) is positive and self-adjoint.

\paragraph{(ii) Generation of heat semigroup.}
By the spectral theorem for unbounded self-adjoint operators~\cite[Ch.~VIII]{ReedSimon1975II}, the positive self-adjoint operator \( L^2 \) generates a strongly continuous, holomorphic semigroup
\[
e^{-t L^2} = \int_0^\infty e^{-t\lambda} \, dE_\lambda,
\]
where \( \{E_\lambda\} \) is the spectral measure of \( L^2 \). Since \( L \in \mathcal{C}_1 \), its spectrum is discrete, so the spectrum of \( L^2 \) consists of squares of real eigenvalues of \( L \), accumulating only at 0.

Hence, for all \( t > 0 \), the operator \( e^{-t L^2} \) is trace class:
\[
e^{-t L^2} \in \mathcal{C}_1(H_{\Psi_\alpha}).
\]
Moreover, the semigroup is holomorphic in \( t \), since \( L^2 \ge 0 \) admits an entire spectral resolution.

\paragraph{(iii) Trace bounds and regularity.}
Let \( \{ \mu_n \} \subset \mathbb{R} \setminus \{0\} \) be the nonzero eigenvalues of \( L \), so that \( \{ \mu_n^2 \} \) are the eigenvalues of \( L^2 \). Then:
\[
\operatorname{Tr}(e^{-t L^2}) = \sum_n e^{-t \mu_n^2}.
\]
Asymptotically, the eigenvalues satisfy \( \mu_n^2 \sim c n^2 \log^2 n \), so the heat trace satisfies
\[
\operatorname{Tr}(e^{-t L^2}) \lesssim t^{-1/2} \log(1/t) \quad \text{as } t \to 0^+,
\]
by Laplace–Mellin inversion of the spectral zeta function \( \zeta_{L^2}(s) \). For large \( t \), exponential decay yields
\[
\operatorname{Tr}(e^{-t L^2}) \lesssim e^{-\delta t}, \quad \text{as } t \to \infty,
\]
for some \( \delta > 0 \) depending on the spectral gap.

\paragraph{Conclusion.}
The semigroup \( \{ e^{-t L_{\mathrm{sym}}^2} \} \) is strongly continuous, holomorphic in \( t \), and satisfies trace-class smoothing bounds across all \( t > 0 \). This justifies the analytic use of the Laplace representation for the Fredholm determinant and the spectral zeta function.
\end{proof}
