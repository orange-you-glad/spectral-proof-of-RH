\section{Summary of Notation}
\label{app:notation-summary}

\noindent
Notation is introduced contextually within each chapter. The following analytic objects and conventions are defined globally and recur throughout the manuscript:

\begin{itemize}
  \item \( \HPsi := L^2(\R, e^{\alpha |x|} \, dx) \): exponentially weighted Hilbert space with fixed exponential weight \( \alpha > \pi \). The weight function is denoted \( \Psi_\alpha(x) := e^{\alpha |x|} \).

  \item \( L_{\sym} := \lim_{t \to 0^+} L_t \in \mathcal{C}_1(\HPsi) \): the canonical compact, self-adjoint convolution operator in the trace-class \( \mathcal{C}_1 \), defined as the trace-norm limit of mollified approximants \( L_t \) derived from the inverse Fourier transform of the completed zeta function.

  \item \( \phi(\lambda) := \Xi\left(\tfrac{1}{2} + i\lambda\right) \): the centered Fourier profile of the completed Riemann zeta function, defining the convolution kernel \( k(x) := \FT^{-1}[\phi](x) \) via inverse transform.

  \item \( \mu_\rho := \frac{1}{i}(\rho - \tfrac{1}{2}) \): the canonical spectral image of a nontrivial zero \( \rho \in \C \) of \( \zeta(s) \), used in the bijective encoding of the spectrum of \( L_{\sym} \).

  \item \( \det\nolimits_\zeta(I - \lambda L_{\sym}) := \frac{\Xi\left(\tfrac{1}{2} + i\lambda\right)}{\Xi\left(\tfrac{1}{2}\right)} \): the Carleman \(\zeta\)-regularized Fredholm determinant of \( L_{\sym} \), canonically encoding the spectral zero set of the Riemann zeta function.
\end{itemize}

\noindent
The completed Riemann zeta function \( \Xi(s) \) is an entire function of order one and exponential type \( \pi \), satisfying the functional equation \( \Xi(s) = \Xi(1 - s) \). For functional identities, normalizations, and Fourier conventions, see Appendix~\ref{app:reductions-and-conventions}.

\medskip

\noindent
All remaining notation is introduced locally in the chapters where it is first used. Indexing of key theorems and symbols by chapter is provided in Appendix~\ref{app:dependency-graph}.
