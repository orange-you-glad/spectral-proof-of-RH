\begin{proof}[Proof of Proposition~\ref{prop:spectrum-reality-implies-rh-multiplicity}]
Assume:
\begin{itemize}
  \item \( L_{\mathrm{sym}} \in \mathcal{C}_1(H_{\Psi_\alpha}) \) is compact and self-adjoint;
  \item \( \operatorname{Spec}(L_{\mathrm{sym}}) \subset \mathbb{R} \);
  \item All nonzero eigenvalues \( \mu \in \operatorname{Spec}(L_{\mathrm{sym}}) \setminus \{0\} \) have multiplicity one.
\end{itemize}

Let
\[
f(\lambda) := \det\nolimits_\zeta(I - \lambda L_{\mathrm{sym}}) = \frac{\Xi\left(\tfrac{1}{2} + i\lambda\right)}{\Xi\left(\tfrac{1}{2}\right)},
\]
be the canonical spectral determinant. This function is entire of genus one and exponential type \( \pi \), and admits the Hadamard product:
\[
f(\lambda) = \prod_{\rho} \left(1 - \frac{\lambda}{\mu_\rho}\right) \exp\left( \frac{\lambda}{\mu_\rho} \right),
\]
where each \( \mu_\rho := \frac{1}{i}(\rho - \tfrac{1}{2}) \) is the image of a nontrivial zero \( \rho \in \mathbb{C} \) of \( \zeta(s) \).

\paragraph{Step 1: Reality Implies RH.}
By assumption, all \( \mu_\rho \in \mathbb{R} \). Then:
\[
\mu_\rho = \frac{1}{i}(\rho - \tfrac{1}{2}) \in \mathbb{R} \quad \Longrightarrow \quad \rho - \tfrac{1}{2} \in i\mathbb{R} \quad \Longrightarrow \quad \Re(\rho) = \tfrac{1}{2}.
\]
Hence, all nontrivial zeros \( \rho \) lie on the critical line.

\paragraph{Step 2: Simplicity.}
The multiplicity of each zero \( \mu_\rho \in \operatorname{Spec}(L_{\mathrm{sym}}) \) is one by assumption. Since the determinant product preserves multiplicity under the map \( \rho \mapsto \mu_\rho \), the order of vanishing of \( \Xi(s) \) at \( s = \tfrac{1}{2} + i\lambda \) must also be one. Therefore:
\[
\operatorname{ord}_\rho(\zeta) = \operatorname{ord}_{\mu_\rho}(f) = 1.
\]

\paragraph{Conclusion.}
All nontrivial zeros of \( \zeta(s) \) lie on the critical line and are simple, as claimed.
\end{proof}
