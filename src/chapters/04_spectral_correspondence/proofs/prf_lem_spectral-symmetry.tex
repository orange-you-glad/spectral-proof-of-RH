\begin{proof}[Proof of Lemma~\ref{lem:spectral-symmetry}]
Let \( \phi(\lambda) := \Xi\left(\tfrac{1}{2} + i\lambda\right) \). Since \( \Xi(s) \) is entire and satisfies the functional equation \( \Xi(s) = \Xi(1 - s) \), the function \( \lambda \mapsto \phi(\lambda) \) is real-valued and even on \( \mathbb{R} \).

Define the convolution kernel
\[
K(x - y) := \frac{1}{2\pi} \int_{\mathbb{R}} e^{i\lambda(x - y)} \phi(\lambda)\, d\lambda,
\]
so that the integral kernel \( K(x,y) := K(x - y) \) satisfies \( K(x,y) = K(y,x) \in \mathbb{R} \). The symmetry follows from the evenness of \( \phi(\lambda) \), since
\[
K(x - y) = \frac{1}{2\pi} \int_{\mathbb{R}} e^{i\lambda(x - y)} \phi(\lambda)\, d\lambda = K(y - x).
\]

Let \( \widetilde{L}_{\mathrm{sym}} \colon L^2(\mathbb{R}) \to L^2(\mathbb{R}) \) be the convolution operator with kernel \( K(x,y) \). Then \( \widetilde{L}_{\mathrm{sym}} \) is real, symmetric, compact, and self-adjoint.

\paragraph{Step 1: Spectral Symmetry in \( L^2(\mathbb{R}) \).}
By the spectral theorem for compact real symmetric operators on \( L^2(\mathbb{R}) \), the spectrum is symmetric about zero:
\[
\mu \in \Spec(\widetilde{L}_{\mathrm{sym}}) \quad \Longrightarrow \quad -\mu \in \Spec(\widetilde{L}_{\mathrm{sym}}),
\]
with algebraic multiplicities preserved.

\paragraph{Step 2: Transfer via Unitary Equivalence.}
Let \( U \colon H_{\Psi_\alpha} \to L^2(\mathbb{R}) \) denote the unitary transformation \( (Uf)(x) := \sqrt{\Psi_\alpha(x)} f(x) \). Then
\[
L_{\mathrm{sym}} = U^{-1} \widetilde{L}_{\mathrm{sym}} U,
\]
so \( L_{\mathrm{sym}} \) is unitarily equivalent to \( \widetilde{L}_{\mathrm{sym}} \). Since unitary equivalence preserves compactness, self-adjointness, and spectral multiplicities, it follows that
\[
\mu \in \Spec(L_{\mathrm{sym}}) \quad \Longrightarrow \quad -\mu \in \Spec(L_{\mathrm{sym}}),
\]
with matching algebraic multiplicities.
\end{proof}
%  