\begin{proof}[Proof of \cref{lem:exact_type_pi}]
Let \( f(\lambda) := \Xi\left( \tfrac{1}{2} + i\lambda \right) \). It is classical that \( \Xi(s) \) is an entire function of order one and exponential type \( \pi \). This follows from:

\begin{itemize}
  \item The functional equation and integral representation of \( \Xi(s) \),
  \item Stirling’s expansion for \( \Gamma(s/2) \),
  \item Hadamard factorization for entire functions with real zeros of bounded density.
\end{itemize}

\paragraph{(i) Upper bound on type.}
From Titchmarsh~\cite[§10.5]{Titchmarsh1986Zeta}, for every \( \varepsilon > 0 \), there exists \( C_\varepsilon > 0 \) such that
\[
\left| \Xi\left( \tfrac{1}{2} + i\lambda \right) \right| \le C_\varepsilon e^{(\pi + \varepsilon)|\lambda|},
\]
establishing exponential type \( \le \pi \).

\paragraph{(ii) Lower bound on type.}
The lower bound follows from classical results of de Bruijn and Levin~\cite[Ch.~3]{Levin1996EntireLectures}, which show that the exponential type of an even, real-entire function is determined by the asymptotic density of its zeros.

In particular, the counting function for the imaginary parts of the nontrivial zeros satisfies
\[
N(T) \sim \frac{T}{2\pi} \log\left( \frac{T}{2\pi e} \right) + \mathcal{O}(\log T),
\]
which implies that the exponential type of \( \Xi(s) \) is at least \( \pi \) via Hadamard theory. Thus, the exponential type is exactly \( \pi \).

\paragraph{(iii) Determinant correspondence.}
By \cref{thm:det_identity_revised}, the canonical zeta-determinant satisfies
\[
\det\nolimits_\zeta(I - \lambda L_{\sym}) = \frac{\Xi\left( \tfrac{1}{2} + i\lambda \right)}{\Xi\left( \tfrac{1}{2} \right)},
\]
and therefore inherits the exact exponential type of the numerator.

\paragraph{Conclusion.}
Both \( \Xi\left( \tfrac{1}{2} + i\lambda \right) \) and the canonical determinant \( \lambda \mapsto \det\nolimits_\zeta(I - \lambda L_{\sym}) \) belong to the Hadamard class \( \mathcal{E}_1^\pi \), completing the proof.
\end{proof}
