\begin{lemma}[Uniqueness of Trace-Norm Limit]
\label{lem:trace_norm_limit_unique}
Let \( \{L_t\}_{t > 0} \subset \TC(H_{\Psi_\alpha}) \) be a family of mollified convolution operators converging in trace norm:
\[
\lim_{t \to 0^+} \| L_t - L \|_{\TC} = 0,
\]
for some operator \( L \in \TC(H_{\Psi_\alpha}) \).

Then the limit \( L \) is unique and independent of the mollifier sequence \( \{L_t\} \), provided the convergence occurs in the trace norm. That is, if two mollifier families \( \{L_t^{(1)}\} \), \( \{L_t^{(2)}\} \subset \TC(H_{\Psi_\alpha}) \) satisfy
\[
\|L_t^{(1)} - L_t^{(2)}\|_{\TC} \to 0 \quad \text{as } t \to 0^+,
\]
and each admits a trace-norm limit, then their limits coincide:
\[
\lim_{t \to 0^+} L_t^{(1)} = \lim_{t \to 0^+} L_t^{(2)} = L.
\]

\medskip
\noindent
This follows from the completeness of \( \TC(H_{\Psi_\alpha}) \) as a Banach space. The trace-norm topology ensures uniqueness of the limit and stability under perturbation by mollifiers.
\end{lemma}
