\begin{definition}[Self-Adjoint Operators]\label{def:selfadjoint_operator}
Let \( H \) be a separable complex Hilbert space, and let \( T \colon \mathcal{D}(T) \subset H \to H \) be a densely defined linear operator.

We say \( T \) is \emph{self-adjoint} if:
\[
T = T^* \quad \text{and} \quad \mathcal{D}(T) = \mathcal{D}(T^*),
\]
where the adjoint \( T^* \colon \mathcal{D}(T^*) \to H \) is defined by: \( g \in \mathcal{D}(T^*) \) if there exists \( h \in H \) such that
\[
\langle T f, g \rangle = \langle f, h \rangle \quad \text{for all } f \in \mathcal{D}(T), \quad \text{in which case } T^* g := h.
\]

The adjoint \( T^* \) is always closed. Hence, every self-adjoint operator is closed and densely defined.

\medskip
\noindent\textbf{Bounded Case.}
If \( T \in \mathcal{B}(H) \) is bounded and everywhere defined, then \( T \) is self-adjoint if and only if it is symmetric:
\[
\langle T f, g \rangle = \langle f, T g \rangle \quad \text{for all } f, g \in H.
\]

\medskip
\noindent\textbf{Symmetric Operators.}
A densely defined operator \( T \colon \mathcal{D}(T) \to H \) is \emph{symmetric} if
\[
\langle T f, g \rangle = \langle f, T g \rangle \quad \text{for all } f, g \in \mathcal{D}(T),
\]
i.e., \( T \subseteq T^* \). Such an operator is self-adjoint precisely when equality holds: \( \mathcal{D}(T) = \mathcal{D}(T^*) \) and \( T = T^* \).

\medskip
\noindent\textbf{Graph Characterization.}
Let
\[
\operatorname{graph}(T) := \{ (f, Tf) \in H \times H : f \in \mathcal{D}(T) \}.
\]
Then \( T \) is self-adjoint if and only if \( \operatorname{graph}(T) \) is closed and equals \( \operatorname{graph}(T^*) \). In particular, self-adjoint operators are maximal among symmetric ones.

\medskip
\noindent\textbf{Spectral Theorem.}
Every self-adjoint operator \( T \colon \mathcal{D}(T) \to H \) admits a unique spectral resolution:
\[
T = \int_{\sigma(T)} \lambda \, dE_\lambda,
\]
where \( E_\lambda \) is a projection-valued measure (PVM) on the Borel \( \sigma \)-algebra of \( \mathbb{R} \), supported on the spectrum \( \sigma(T) \subset \mathbb{R} \). In particular:
\begin{itemize}
    \item \( \sigma(T) \subset \mathbb{R} \) is closed and nonempty;
    \item For every bounded Borel function \( f \colon \mathbb{R} \to \mathbb{C} \), the spectral calculus
    \[
    f(T) := \int_{\sigma(T)} f(\lambda)\, dE_\lambda
    \]
    defines a bounded operator \( f(T) \in \mathcal{B}(H) \);
    \item The one-parameter unitary group \( \{ e^{itT} \}_{t \in \mathbb{R}} \subset \mathcal{U}(H) \) is strongly continuous.
\end{itemize}

\medskip
\noindent\textbf{Compact Self-Adjoint Operators.}
If \( T \in \mathcal{C}_1(H) \cap \operatorname{SA}(H) \), then \( \sigma(T) \subset \mathbb{R} \) consists entirely of isolated eigenvalues of finite multiplicity, with \( \lambda_n \to 0 \). The corresponding eigenfunctions form a complete orthonormal basis of \( H \).

\medskip
\noindent\textbf{Remarks.}
\begin{itemize}
    \item Self-adjointness implies a full spectral theory and guarantees the reality of the spectrum.
    \item A symmetric operator \( T_0 \colon \mathcal{D}_0 \to H \) is \emph{essentially self-adjoint} if its closure \( \overline{T_0} \) is self-adjoint. This property ensures unique spectral evolution.
    \item In this manuscript, the convolution operators \( L_t \), and their limit \( L_{\sym} \), are essentially self-adjoint on \( \mathcal{S}(\mathbb{R}) \subset H_{\Psi_\alpha} \); see \secref{sec:operator_construction}.
    \item For integral operators with Hermitian kernels and exponential damping, essential self-adjointness on a core follows from Friedrichs' extension theorem.
\end{itemize}

\medskip
\noindent\textbf{References.}
\begin{itemize}
    \item M.~Reed and B.~Simon, \emph{Methods of Modern Mathematical Physics I: Functional Analysis}, Chapter~VI \cite{ReedSimon1980I}.
    \item B.~Simon, \emph{Trace Ideals and Their Applications}, Chapter~3 \cite{Simon2005TraceIdeals}.
\end{itemize}
\end{definition}
