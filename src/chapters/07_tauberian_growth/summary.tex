\subsection*{Summary}

\begin{itemize}
  % --- Tauberian Principle ---
  \item \defref{def:tauberian_theorem} — Classical Laplace–Tauberian framework: the short-time behavior of the heat trace \( \Tr(e^{-tL^2}) \) determines high-energy spectral growth \( A(\Lambda) \) via inverse Laplace analysis.

  % --- Asymptotic Bounds and Growth Envelopes ---
  \item \lemref{lem:spectral_convexity_estimate} — From the envelope bound \( \Tr(e^{-t L^2}) = O(t^{-1/2}) \), we deduce the coarse spectral estimate:
  \[
  N(\lambda) = O(\lambda^{1/2}),
  \]
  placing the spectrum in the minimal compact class consistent with trace-class decay.

  \item \lemref{lem:log_corrected_tauberian_estimate} — From the refined expansion
  \[
  \Tr(e^{-t L^2}) \sim \frac{1}{\sqrt{4\pi t}} \log\left( \frac{1}{t} \right),
  \]
  and Korevaar’s log-modified Tauberian theory, we obtain the precise asymptotic:
  \[
  N(\lambda) \sim \frac{\sqrt{\lambda}}{\pi} \log \lambda.
  \]

  \item \corref{cor:zeta_compatibility} — The spectral growth law of \( L_{\sym} \) matches the Riemann–von Mangoldt formula:
  \[
  N_\zeta(T) = \frac{T}{2\pi} \log\left( \frac{T}{2\pi} \right) - \frac{T}{2\pi} + O(\log T),
  \]
  confirming analytic compatibility between the zeta zero distribution and the spectrum of the canonical operator.
\end{itemize}

\medskip
\noindent
(For higher-order corrections to the heat kernel expansion and Tauberian remainder estimates, see Appendix~\ref{app:heat_kernel_refinements}.)

\vspace{0.5em}

\noindent\textbf{Analytic flow diagram:}
\[
\begin{aligned}
\text{Heat Kernel}
&\xrightarrow{\text{Laplace}} \text{Trace Asymptotics} \\
&\xrightarrow{\text{Tauberian}} \text{Spectral Growth} \\
&\xrightarrow{\text{Hadamard}} \text{Zeta Determinant}
\end{aligned}
\]

\begin{remark}[Transition to Spectral Rigidity]
We have shown that the asymptotic eigenvalue distribution of \( L_{\sym} \) encodes the nontrivial zeros of the Riemann zeta function. The next natural question is rigidity: if an operator shares both the spectrum and the determinant identity, is it unitarily equivalent to \( L_{\sym} \)?

This leads directly to the spectral rigidity results developed in Chapter~\ref{sec:spectral_rigidity}.
\end{remark}

\vspace{0.5em}

\paragraph{Logical Closure.}
All spectral asymptotics in this chapter are derived from the singular behavior of the heat trace using validated Tauberian theorems. The Laplace–Mellin inversion logic is acyclic and rooted in the canonical trace asymptotic developed in Chapter~\ref{sec:heat_kernel_asymptotics}. All results are DAG-traceable and require no assumptions beyond established determinant and heat kernel structure.
