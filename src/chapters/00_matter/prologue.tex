\section*{Prologue: Bourbaki.RH and the Future of Proof}
\label{sec:prologue}

\paragraph*{1.1 The Autonomous Mathematician.}
Bourbaki.RH is not a project about the Riemann Hypothesis. It is a prototype for the next phase of mathematics: autonomous, compositional, and self-verifying. At its core is a system that engineers mathematics—not just writing proofs, but refactoring them, machine-checking them, and compiling them into a rigorously typed document. This is not LLM-as-calculator; it is LLM as collaborator in a formal loop: conjecture $\rightarrow$ attempt $\rightarrow$ audit $\rightarrow$ refactor $\rightarrow$ validate $\rightarrow$ render. The result is not a paper—it’s an artifact. A reproducible build. Every DAG edge in this manuscript is re-proven on every commit. Every analytic identity, every determinant formula, every trace expansion—traced and enforced down to the kernel. This isn't a new proof of RH. It's a new standard for what counts as a proof.

\paragraph*{1.2 Why RH Became the Test Case.}
The Riemann Hypothesis was not chosen for surprise or fame, but for stress. RH resists improvisation. It rewards only structure. It forces a system to grow up. It presents a proving problem that is technical, intricate, and yet cleanly refactorable into modular equivalences. Its sharp edge—real spectrum versus critical zeros—is the kind of binary equivalence that a formal system can test, diagnose, and refactor without infinite guesswork. RH was the ideal crash test for the Bourbaki.RH pipeline.

\paragraph*{1.3 Workflow in Miniature.}
At the heart of Bourbaki.RH is a Directed Acyclic Graph (DAG) of irreducible lemmas, each compositional, each typed, each carrying semantic weight. These nodes are machine-audited and connected by zeta-invariant structure: determinant identities, spectral bijections, kernel decay, Tauberian asymptotics. The DAG is compiled into a Lean 4 formalization (in progress) and a typeset manuscript. A reproducible containerized CI pipeline re-verifies the entire graph on every push. The system doesn’t "trust" any prior argument—it rebuilds logic from the inside out.

\paragraph*{1.4 Mathematical Result, Briefly.}
The central mathematical object is a canonical trace-class operator
\[
\Lsym \in \TC(\HPsi), \quad \HPsi := L^2(\R, e^{\alpha|x|} dx), \quad \alpha > \pi,
\]
whose zeta-regularized Fredholm determinant satisfies
\[
\detz(I - \lambda \Lsym) = \frac{\Xi\left(\tfrac{1}{2} + i\lambda\right)}{\Xi\left(\tfrac{1}{2}\right)}.
\]
The spectrum of \( \Lsym \) recovers the nontrivial zeros \( \rho \in \C \) of \( \zeta(s) \), via the map
\[
\rho \mapsto \mu_\rho := \tfrac{1}{i}(\rho - \tfrac{1}{2}).
\]
The spectral theorem guarantees \( \Spec(\Lsym) \subset \R \) if and only if RH holds. Thus, Bourbaki.RH proves:
\[
\RH \iff \Spec(\Lsym) \subset \R.
\]
But again, this is not about novelty. It is about closure—formal, repeatable, machine-verifiable closure.

\paragraph*{1.5 Invitation to Audit and Extend.}
Every lemma is modular. Every equation is reproducible. The full DAG structure, Lean verification suite (coming soon), and CI runner are open source:
\begin{itemize}
  \item GitHub: \texttt{https://github.com/orange-you-glad/spectral-proof-of-RH}
  \item Container image: \texttt{ghcr.io/orange-you-glad/bourbaki.rh:latest}
\end{itemize}
We invite readers not just to verify but to fork. The real experiment is whether you can extend it. New operators. New L-functions. New spectral theorems. This system wasn’t designed to end the RH conversation. It was designed to start the next one.

\medskip

\subsection*{Main Equivalence — Spectral RH Reformulation}
\hfill \break

\begin{tcolorbox}[colback=gray!3!white, colframe=black!75!white]
We construct a canonical operator \( \Lsym \in \TC(\HPsi) \) on
\[
\HPsi := L^2\big(\mathbb{R}, e^{\alpha |x|} \, dx\big), \qquad \alpha > \pi,
\]
such that:
\begin{enumerate}
  \item The zeta-regularized determinant identity holds:
  \[
  \detz(I - \lambda \Lsym) = \frac{\Xi\left(\tfrac{1}{2} + i\lambda\right)}{\Xi\left(\tfrac{1}{2}\right)}.
  \]
  \item The spectrum of \( \Lsym \) satisfies the bijective correspondence:
  \[
  \rho \mapsto \mu_\rho := \tfrac{1}{i}(\rho - \tfrac{1}{2}) \in \Spec(\Lsym),
  \]
  with multiplicities preserved.
  \item The Riemann Hypothesis holds if and only if
  \[
  \Spec(\Lsym) \subset \mathbb{R}.
  \]
\end{enumerate}
All analytic and operator-theoretic infrastructure—Paley–Wiener kernel decay, trace-norm convergence, heat semigroup regularity, determinant growth—is rigorously modularized and validated across Chapters~1–10. See \thmref{thm:truth_of_rh} for formal equivalence.
\end{tcolorbox}


\subsection*{Disclosure of Large-Language-Model Assistance}

The author used OpenAI’s \emph{ChatGPT} (GPT-4o, May 2025 release) as a research and writing assistant. This collaboration included:

\begin{itemize}
  \item Structural scaffolding for chapters and DAG layout;
  \item Drafting provisional LaTeX formulations for definitions and identities;
  \item Symbolic verification (e.g., Fourier transforms, determinant expansions);
  \item Boilerplate generation for cross-references and macros;
  \item Refactoring prose for clarity and alignment with formal semantics.
\end{itemize}

All mathematics was independently re-verified by the author. The LLM is not an author and bears no responsibility for any claim made herein.
