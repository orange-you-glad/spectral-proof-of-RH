\section{Classical Properties of the Riemann Zeta Function}
\label{app:zeta_function_background}

This appendix summarizes the classical analytic properties of the Riemann zeta function \( \zeta(s) \) and its completion \( \Xi(s) \), based on foundational results from Titchmarsh~\cite{Titchmarsh1986Zeta} and Edwards~\cite{Edwards1974Zeta}. These properties form the analytic foundation for the determinant identity and spectral constructions developed in Chapters~\ref{sec:determinant_identity}–\ref{sec:logical_closure}.

\subsection*{The Completed Zeta Function \( \Xi(s) \)}

Define:
\[
\Xi(s) := \tfrac{1}{2} s(s - 1) \pi^{-s/2} \Gamma\left( \tfrac{s}{2} \right) \zeta(s),
\qquad \text{with } \zeta(s) := \sum_{n=1}^\infty n^{-s}, \quad \Re(s) > 1.
\]

Then \( \Xi(s) \) satisfies:
\begin{itemize}
  \item \textbf{Entirety:} \( \Xi(s) \) extends to an entire function of order one and genus one;
  \item \textbf{Functional Equation:} \( \Xi(s) = \Xi(1 - s) \);
  \item \textbf{Reality Symmetry:} \( \Xi(s) \in \R \) for real \( s \), and \( \Xi(\bar{s}) = \overline{\Xi(s)} \).
\end{itemize}

These facts follow from the functional equation of \( \zeta(s) \), analytic continuation, and Stirling’s formula applied to \( \Gamma(s/2) \).

\subsection*{Zeros and the Riemann Hypothesis}

All nontrivial zeros of \( \zeta(s) \) lie in the critical strip \( 0 < \Re(s) < 1 \), and coincide with the zeros of \( \Xi(s) \). The Riemann Hypothesis asserts:
\[
\Re(\rho) = \tfrac{1}{2}, \qquad \text{for all nontrivial zeros } \rho.
\]

Centering via \( s = \tfrac{1}{2} + i\lambda \) aligns the critical line with the real axis, facilitating spectral realization via a self-adjoint operator \( L_{\sym} \in \TC(\HPsi) \) (see \cref{sec:operator_construction}).

\subsection*{Hadamard Factorization and Exponential Type}

As an entire function of order one and genus one, \( \Xi(s) \) admits the Hadamard product:
\[
\Xi(s) = e^{C_0 + C_1 s} \prod_{\rho} \left( 1 - \frac{s}{\rho} \right) e^{s/\rho},
\]
with convergence uniform on compact subsets. The product runs over all nontrivial zeros \( \rho \in \C \), counted with multiplicity.

The centered profile satisfies:
\[
\Xi\left( \tfrac{1}{2} + i\lambda \right) = \mathcal{O}\left( e^{\pi |\lambda| / 2} \right), \qquad |\lambda| \to \infty,
\]
due to the decay of the gamma factor and Mellin representation. Hence, the map \( \lambda \mapsto \Xi(\tfrac{1}{2} + i\lambda) \) is entire of exponential type exactly \( \pi \) (see \cref{lem:exact_type_pi}).

\subsection*{Centered Spectral Profile and Fourier Analysis}

Define the centered profile:
\[
\phi(\lambda) := \Xi\left( \tfrac{1}{2} + i\lambda \right), \qquad \lambda \in \R.
\]
Then:
\[
\phi(-\lambda) = \phi(\lambda), \qquad \phi(\lambda) \in \R, \qquad \phi \in \PW{\pi} \subset \Schwartz'(\R).
\]

By the Paley–Wiener theorem, \( \phi \in \PW{\pi} \) implies:
\[
\FT^{-1}[\phi](x) \sim e^{-\pi |x|}, \qquad x \to \infty.
\]
This inverse Fourier transform defines the convolution kernel \( k(x) := \FT^{-1}[\phi](x) \), which yields the compact operator \( L_{\sym} \in \TC(\HPsi) \) by trace-norm convergence. The exponential decay of \( k \) enables heat trace asymptotics and determinant regularization.

\subsection*{Conclusion}

The completed Riemann zeta function \( \Xi(s) \) satisfies:
\begin{itemize}
  \item Entirety and functional symmetry;
  \item Genus-one Hadamard factorization;
  \item Sharp exponential type \( \pi \);
  \item Paley–Wiener decay of the convolution kernel.
\end{itemize}

These properties justify the spectral determinant identity realized via \( L_{\sym} \in \TC(\HPsi) \), and enable the analytic equivalence \( \RH \iff \Spec(L_{\sym}) \subset \R \) proven in \cref{sec:spectral_implications}. The reparametrization \( \rho \mapsto \mu_\rho := \frac{1}{i}(\rho - \tfrac{1}{2}) \) transforms the critical strip into the real spectrum of \( L_{\sym} \).
