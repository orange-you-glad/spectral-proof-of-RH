\begin{proof}[Proof of \propref{prop:selfadjointness-Lt}]
Let \( H_\Psi := L^2(\mathbb{R}, \Psi(x)\, dx) \), where \( \Psi \) is a smooth, strictly positive exponential weight satisfying \( \Psi(x) \sim e^{\alpha|x|} \) as \( |x| \to \infty \), for some \( \alpha > 0 \).

\medskip
\noindent\textbf{Step 1: Boundedness and Symmetry on a Dense Core.}
By Proposition~\ref{prop:boundedness-Lt-weighted}, the operator
\[
(L_t f)(x) := \int_{\mathbb{R}} \phi_t(x - y)\, f(y)\, dy
\]
extends to a bounded linear operator \( L_t \in \mathcal{B}(H_\Psi) \). By Proposition~\ref{prop:symmetry-Lt-Schwartz}, \( L_t \) is symmetric on the dense subspace \( \mathcal{S}(\mathbb{R}) \subset H_\Psi \), meaning:
\[
\langle L_t f, g \rangle_{H_\Psi} = \langle f, L_t g \rangle_{H_\Psi}, \quad \forall f, g \in \mathcal{S}(\mathbb{R}).
\]

\medskip
\noindent\textbf{Step 2: Self-Adjointness of Bounded Symmetric Operator.}
By a standard result in operator theory (see \cite[Theorem~VI.1]{ReedSimon1980I}), any bounded symmetric operator defined on a dense subspace of a Hilbert space extends uniquely to a self-adjoint operator. Thus,
\[
L_t^* = L_t \quad \text{on all of } H_\Psi.
\]

\medskip
\noindent\textbf{Conclusion.}
The operator \( L_t \in \mathcal{B}(H_\Psi) \) is self-adjoint. Hence, it admits a spectral resolution via the spectral theorem, supporting zeta-function regularization, semigroup analysis, and Fredholm determinant identities developed in later chapters.
\end{proof}
