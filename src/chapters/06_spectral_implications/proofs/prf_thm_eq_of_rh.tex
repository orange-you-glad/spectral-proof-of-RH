\begin{proof}[Proof of \thmref{thm:eq_of_rh}]
Let \( \rho \mapsto \mu_\rho := \dfrac{1}{i(\rho - \tfrac{1}{2})} \) denote the canonical spectral reparametrization of the nontrivial zeros \( \rho \) of the Riemann zeta function \( \zetaR(s) \). This spectral mapping is bijective by \thmref{thm:spectral_zero_bijection_revised} and preserves multiplicity by \lemref{lem:multiplicity_preservation}, ensuring trace-class spectral correspondence.

\medskip

Under this map, the normalized Carleman–\(\zeta\)-regularized determinant of the canonical operator \( L_{\sym} \in \TC(\HPsi) \cap \KC(\HPsi) \) satisfies:
\[
\det\nolimits_\zeta(I - \lambda L_{\sym}) = \frac{\XiR\left( \tfrac{1}{2} + i\lambda \right)}{\XiR\left( \tfrac{1}{2} \right)},
\]
as established unconditionally in \thmref{thm:det_identity_revised}. The analytic infrastructure for this identity—including short-time trace bounds, semigroup holomorphy, and Laplace convergence—is developed in Chapter~\ref{sec:heat_kernel_asymptotics} and Appendix~\ref{app:heat_kernel_construction}.

\medskip

Now observe the algebraic identity: for any nontrivial zero \( \rho = \sigma + i\gamma \),
\[
\mu_\rho = \frac{1}{i(\rho - \tfrac{1}{2})} = \frac{1}{i(\sigma - \tfrac{1}{2}) - \gamma}.
\]
Then \( \mu_\rho \in \R \) if and only if \( \sigma = \tfrac{1}{2} \), i.e., if and only if \( \rho \) lies on the critical line. Thus:
\[
\mu_\rho \in \R \iff \Re(\rho) = \tfrac{1}{2}.
\]

\paragraph*{(\( \Rightarrow \)) Spectral Reality Implies \(\RH\).}
Assume \( \Spec(L_{\sym}) \subset \R \). Then all eigenvalues \( \mu_\rho \in \R \), so the identity above implies that every nontrivial zero \( \rho \) satisfies \( \Re(\rho) = \tfrac{1}{2} \). Hence, \RH{} holds.

\paragraph*{(\( \Leftarrow \)) \(\RH\) Implies Spectral Reality.}
Conversely, assume \RH{} holds. Then for each nontrivial zero \( \rho \), we have \( \Re(\rho) = \tfrac{1}{2} \), which implies \( \mu_\rho \in \R \). Therefore, all nonzero eigenvalues of \( L_{\sym} \) are real.

\paragraph*{Conclusion.}
Since \( L_{\sym} \) is compact and self-adjoint (\thmref{thm:sa_trace_class_Lsym}), the spectral theorem ensures its spectrum lies in \( \R \). Hence,
\[
\Spec(L_{\sym}) \subset \R \iff \mu_\rho \in \R \text{ for all nontrivial zeros } \rho,
\]
which yields the analytic–spectral equivalence:
\[
\RH \iff \Spec(L_{\sym}) \subset \R,
\]
as claimed.
\end{proof}
