\subsection*{Summary}

This chapter constructed the canonical convolution operator \( L_{\sym} \in \TC(H_{\Psi_\alpha}) \), beginning with mollified spectral profiles and culminating in a complete operator-theoretic classification. The results are modularly organized below:

\paragraph{Function Spaces and Core Definitions}
\begin{itemize}
  \item \defref{def:weighted_fourier_space} — Exponentially weighted Hilbert space \( H_{\Psi_\alpha} := L^2(\R, e^{\alpha |x|} dx) \), with \( \alpha > \pi \).
  \item \defref{def:mollified_fourier_profile} — Gaussian-mollified profiles \( \varphi_t(\lambda) := e^{-t\lambda^2} \phi(\lambda) \in \Schwartz \).
  \item \defref{def:convolution_operators_Lt_Lsym} — Convolution operators \( L_t \) and their trace-norm limit \( L_{\sym} := \lim_{t \to 0^+} L_t \).
\end{itemize}

\paragraph{Analytic Estimates and Kernel Decay}
\begin{itemize}
  \item \lemref{lem:mollified_profile_decay} — Decay bounds, integrability, and convergence of \( \varphi_t \) and \( k_t := \widehat{\varphi_t} \).
  \item \lemref{lem:trace_class_Lt} — Each mollified operator \( L_t \in \TC(H_{\Psi_\alpha}) \) via Simon's kernel criterion.
  \item \lemref{lem:trace_norm_convergence_Lt_to_Lsym} — Convergence: \( \|L_t - L_{\sym}\|_{\TC} \to 0 \).
  \item \lemref{lem:trace_norm_rate_convergence} — Explicit convergence rate: \( \|L_t - L_{\sym}\| \lesssim t^\beta \).
  \item \lemref{lem:construction_canonical_data} — Uniqueness of \( L_{\sym} \) from analytic data \( (\Xi, \alpha) \), independent of mollifier.
  \item \lemref{lem:mollifier_independence_kernel_limit} — Independence of \( L_{\sym} \) under generalized mollifier families.
\end{itemize}

\paragraph{Operator Properties and Domain Closure}
\begin{itemize}
  \item \lemref{lem:boundedness_Lsym} — Boundedness of \( L_{\sym} \in \mathcal{B}(H_{\Psi_\alpha}) \) from mollified estimates.
  \item \lemref{lem:core_essential_sa} — \( L_{\sym} \) is essentially self-adjoint on \( \Schwartz \subset H_{\Psi_\alpha} \).
  \item \lemref{lem:Lsym_square_selfadjoint} — \( L_{\sym}^2 \) is also essentially self-adjoint on \( \Schwartz \) and admits a discrete spectral resolution.
\end{itemize}

\paragraph{Canonical Operator Theorems}
\begin{itemize}
  \item \thmref{thm:existence_Lsym} — Existence of \( L_{\sym} \) as trace-norm limit of \( L_t \), constructed from mollified profiles.
  \item \thmref{thm:sa_trace_class_Lsym} — Final classification: \( L_{\sym} \in \TC \cap \KC \), self-adjoint with Paley–Wiener kernel.
  \item \thmref{thm:trace_zero_Lsym} — Trace vanishing: \( \Tr(L_{\sym}) = 0 \), ensuring determinant normalization.
\end{itemize}

\medskip
\noindent
These results establish the full analytic and spectral-theoretic foundation for the canonical determinant identity in Chapter~\ref{sec:determinant_identity}, where the spectrum of \( L_{\sym} \) will be shown to encode the nontrivial zeros of the Riemann zeta function \( \zeta(s) \).
