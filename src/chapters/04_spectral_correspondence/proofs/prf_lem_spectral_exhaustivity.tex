\begin{proof}[Proof of \lemref{lem:spectral_exhaustivity}]
Let \( \{ \mu_n \} \subset \operatorname{Spec}(L_{\mathrm{sym}}) \setminus \{0\} \) denote the nonzero eigenvalues of the canonical compact, self-adjoint operator \( L_{\mathrm{sym}} \in \mathcal{C}_1(H_{\Psi_\alpha}) \), counted with algebraic multiplicity.

\paragraph{Step 1: Determinant Zeros Correspond to Zeta Zeros.}
By \thmref{thm:det_identity_revised}, the regularized Fredholm determinant satisfies the identity:
\[
\det\nolimits_\zeta(I - \lambda L_{\mathrm{sym}}) = \frac{\Xi(\tfrac{1}{2} + i\lambda)}{\Xi(\tfrac{1}{2})}.
\]
The right-hand side vanishes precisely at values \( \lambda_\rho := i(\rho - \tfrac{1}{2}) \in \mathbb{C} \), corresponding to nontrivial zeros \( \rho \in \mathbb{C} \) of the Riemann zeta function \( \zeta(s) \), with the order of vanishing equal to the multiplicity of \( \rho \).

\paragraph{Step 2: Spectral Reciprocity.}
For trace-class operators \( L \in \mathcal{C}_1(H_{\Psi_\alpha}) \), the logarithmic Fredholm determinant admits the spectral expansion
\[
\log \det\nolimits_\zeta(I - \lambda L) = -\sum_{\mu \in \operatorname{Spec}(L)} \log(1 - \lambda \mu),
\]
valid for \( \|\lambda L\| < 1 \) and extended by analytic continuation. Therefore, the determinant vanishes at \( \lambda = \lambda_\rho \) if and only if
\[
\mu_\rho := \lambda_\rho^{-1} = \frac{1}{i}(\rho - \tfrac{1}{2})
\]
is a nonzero eigenvalue of \( L_{\mathrm{sym}} \), and the multiplicity of the zero of the determinant equals the algebraic multiplicity of the eigenvalue \( \mu_\rho \).

\paragraph{Step 3: Exhaustivity.}
Since the entire function \( \Xi(s) \) has no zeros other than the nontrivial zeros of \( \zeta(s) \), and the determinant encodes the complete nonzero spectrum of \( L_{\mathrm{sym}} \), it follows that every eigenvalue \( \mu \in \operatorname{Spec}(L_{\mathrm{sym}}) \setminus \{0\} \) arises from some zeta zero \( \rho \), via the relation:
\[
\mu = \mu_\rho = \frac{1}{i}(\rho - \tfrac{1}{2}).
\]
Moreover, the normalization \( \Xi(\tfrac{1}{2}) \ne 0 \) implies \( \det\nolimits_\zeta(I) = 1 \), so \( 0 \notin \operatorname{Spec}(L_{\mathrm{sym}}) \). Thus, the spectral map \( \rho \mapsto \mu_\rho \) is surjective onto \( \operatorname{Spec}(L_{\mathrm{sym}}) \setminus \{0\} \), with multiplicities preserved.
\end{proof}
