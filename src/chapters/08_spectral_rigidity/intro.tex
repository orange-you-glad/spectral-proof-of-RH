\subsection*{Introduction}
\label{sec:intro_spectral_rigidity}

This chapter recasts the Riemann Hypothesis as a statement of spectral rigidity: the spectrum of the canonical trace-class operator
\[
\Lsym \in \TC(\HPsi),
\]
is real if and only if all nontrivial zeros of the Riemann zeta function lie on the critical line \( \Re(s) = \tfrac{1}{2} \). The operator \( \Lsym \) is constructed via mollified convolution from the inverse Fourier transform of the completed zeta function \( \XiR(s) \).

\paragraph{Goals.}
\begin{itemize}
  \item \textit{Spectral Encoding.}  
  The Carleman \(\zeta\)-regularized determinant identity
  \[
  \detz(I - \lambda \Lsym)
  = \frac{\XiR\left(\tfrac{1}{2} + i\lambda\right)}{\XiR\left(\tfrac{1}{2}\right)}
  \]
  defines a multiplicity-preserving encoding
  \[
  \rho \mapsto \mu_\rho := \frac{1}{i}(\rho - \tfrac{1}{2}),
  \]
  sending nontrivial zeros \( \rho \) of \( \zetaR(s) \) to eigenvalues \( \mu_\rho \in \Spec(\Lsym) \). This intertwines the Hadamard factorization of \( \XiR(s) \) with the spectral structure of \( \Lsym \). We show that every determinant zero corresponds to a spectral eigenvalue.

  \item \textit{Spectral Rigidity.}  
  Although \( \Lsym \) is self-adjoint and thus has real spectrum, we prove the converse: if all encoded eigenvalues \( \mu_\rho \in \R \), then each corresponding zero \( \rho \) lies on the critical line. That is,
  \[
  \Spec(\Lsym) \subset \R
  \quad \Longleftrightarrow \quad
  \Re(\rho) = \tfrac{1}{2}, \quad \text{for all } \rho \in \Spec(\zetaR).
  \]
  Determinantal vanishing implies spectral inclusion via analytic continuation and Fredholm theory, without requiring a prior bijection.

  \item \textit{Spectral Symmetry.}  
  The functional identity \( \XiR(\tfrac{1}{2} + i\lambda) = \XiR(\tfrac{1}{2} - i\lambda) \) implies
  \[
  \mu \in \Spec(\Lsym) \quad \Rightarrow \quad -\mu \in \Spec(\Lsym),
  \]
  with multiplicities preserved. This reflects the evenness of the centered spectral profile \( \phi(\lambda) := \XiR(\tfrac{1}{2} + i\lambda) \).

  \item \textit{Trace Positivity.}  
  The trace pairing
  \[
  \phi \mapsto \Tr(\phi(\Lsym))
  \]
  defines a positive tempered distribution on \( \R \). In particular,
  \[
  \Tr(e^{-t\Lsym^2}) \ge 0 \quad \forall\, t > 0,
  \]
  reflecting positivity of the heat kernel regularized spectral measure. This positivity extends to all \( \phi \in \Schwartz(\R) \), reinforcing the harmonic-analytic structure of the trace.

  \item \textit{Analytic Independence.}  
  All results in this chapter follow from classical analytic theory:
  \begin{itemize}
    \item spectral theory of compact, self-adjoint operators;
    \item kernel decay and semigroup regularity;
    \item Hadamard product theory and functional symmetry;
    \item Fredholm theory and determinant–spectrum correspondence;
    \item unitary equivalence via orthonormal diagonalization.
  \end{itemize}
  No input is required from modular forms, trace formulas, or Langlands theory.
\end{itemize}
