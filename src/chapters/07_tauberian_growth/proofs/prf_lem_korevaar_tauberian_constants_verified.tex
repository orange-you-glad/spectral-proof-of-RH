\begin{proof}[Proof of Lemma~\ref{lem:korevaar_tauberian_constants_verified}]
We verify the hypotheses of Korevaar’s Tauberian theorem for log-modulated Laplace transforms~\cite[Ch.~III, Thm.~5.5]{Korevaar2004Tauberian} for the spectral heat trace
\[
\Theta(t) := \operatorname{Tr}(e^{-t L_{\mathrm{sym}}^2}) = \sum_{n=1}^\infty e^{-t \mu_n^2},
\]
associated with the canonical compact, self-adjoint operator \( L_{\mathrm{sym}} \in \mathcal{C}_1(H_{\Psi_\alpha}) \).

\paragraph{(i) Nonnegativity and Local Boundedness.}
Each term in the series satisfies \( e^{-t \mu_n^2} > 0 \), so \( \Theta(t) > 0 \) for all \( t > 0 \), and the function is smooth, nonnegative, and locally bounded.

\paragraph{(ii) Laplace–Stieltjes Representation.}
Define the squared eigenvalue counting function:
\[
A(\Lambda) := \#\left\{ n \in \mathbb{N} : \mu_n^2 \le \Lambda \right\}.
\]
Then \( A(\Lambda) \) is right-continuous, monotone nondecreasing, and diverges as \( \Lambda \to \infty \). The Laplace–Stieltjes representation holds:
\[
\Theta(t) = \int_0^\infty e^{-t \lambda} \, dA(\lambda),
\]
with convergence guaranteed by the sub-Weyl growth \( A(\Lambda) \lesssim \Lambda^{1/2} \log \Lambda \).

\paragraph{(iii) Log-Modulated Regular Variation.}
From Proposition~\ref{prop:short_time_heat_expansion}, the heat trace satisfies the refined singular expansion:
\[
\Theta(t) \sim \frac{1}{\sqrt{4\pi t}} \log\left( \frac{1}{t} \right), \qquad \text{as } t \to 0^+.
\]
This implies \( \Theta \in \mathcal{R}^{\log}_{1/2}(0^+) \), the class of log-modulated regularly varying functions of index \( \alpha = 1/2 \).

\paragraph{(iv) Asymptotic Invertibility.}
By Lemma~\ref{lem:laplace_kernel_growth_class} (and equivalently Lemma~\ref{lem:tauberian_heat_trace_application}), the inverse Laplace–Stieltjes transform satisfies
\[
A(\Lambda) \sim \frac{\sqrt{\Lambda}}{\pi} \log \Lambda \qquad \text{as } \Lambda \to \infty.
\]

\paragraph{Conclusion.}
All conditions of Korevaar’s theorem are verified. The application of the log-modulated Tauberian inversion is fully justified, completing the proof.
\end{proof}
