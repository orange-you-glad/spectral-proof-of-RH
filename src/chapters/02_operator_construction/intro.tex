\subsection*{Introduction}

This chapter constructs the canonical compact, self-adjoint, trace-class operator
\[
\Lsym \in \TC(H_{\Psi_\alpha}),
\]
on the exponentially weighted Hilbert space \( H_{\Psi_\alpha} := L^2(\R, e^{\alpha |x|} dx) \), designed to spectrally encode the nontrivial zeros of the completed Riemann zeta function \( \Xi(s) \) via a zeta-regularized determinant.

The analytic realization unfolds in five rigorously structured stages:

\begin{itemize}
    \item \textbf{Weighted space and spectral profile:} The spectral profile \( \phi(\lambda) := \Xi(\tfrac{1}{2} + i\lambda) \) belongs to the Paley--Wiener class \( \PW{\pi} \), and its inverse Fourier transform \( k(x) \) satisfies exponential decay. For \( \alpha > \pi \), the associated kernel is integrable in \( H_{\Psi_\alpha} \), yielding trace-class convolution operators. The optimal threshold \( \alpha > \pi \) is proved to be both necessary and sufficient; see \propref{prop:trace_class_sharpness}.
    
    \item \textbf{Mollifier family \( L_t \):} Introducing mollified spectral profiles
    \[
    \varphi_t(\lambda) := e^{-t\lambda^2} \phi(\lambda),
    \]
    we define convolution operators \( L_t \in \TC(H_{\Psi_\alpha}) \) with uniformly bounded trace norms. Each \( L_t \) is compact, symmetric, and self-adjoint, forming a regularized approximation scheme for \( \Lsym \).

    \item \textbf{Trace-norm limit and canonicality:} We show that \( L_t \to \Lsym \) in the trace-norm topology as \( t \to 0^+ \), and that this limit is independent of the mollifier. Canonical convergence and uniqueness are established in \lemref{lem:trace_norm_limit_unique} and \thmref{thm:existence_Lsym}.

    \item \textbf{Core domain and essential self-adjointness:} The Schwartz space \( \Schwartz(\R) \) serves as a common core for both \( \Lsym \) and \( \Lsym^2 \). By Nelson’s analytic vector theorem, these operators are essentially self-adjoint on \( \Schwartz \); see \lemref{lem:core_essential_sa} and \lemref{lem:Lsym_square_selfadjoint}.

    \item \textbf{Trace normalization:} The operator \( \Lsym \) is centered, with vanishing trace:
    \[
    \Tr(\Lsym) = 0,
    \]
    ensuring normalization of the zeta-determinant; see \thmref{thm:trace_zero_Lsym}.
\end{itemize}

\medskip

\noindent
The proof of trace-norm convergence deserves special emphasis. Using dominated convergence, uniform trace-norm bounds on the mollified family \( \{L_t\} \), and kernel decay estimates, we show:
\[
  \|L_t - \Lsym\|_{\TC} \le C\,t^{1/2}
\]
for small \( t \), providing quantitative control on the approximation. This, combined with the self-adjoint structure on a common core, yields a well-defined spectral operator.

\medskip

The operator \( \Lsym \) thus constructed is compact, self-adjoint, trace class, and generates a holomorphic heat semigroup \( \{e^{-t \Lsym^2}\}_{t > 0} \) on \( H_{\Psi_\alpha} \); see \lemref{lem:heat_semigroup_existence}.

\medskip

\noindent
This operator forms the analytic base of the determinant identity:
\[
\detz(I - \lambda^2 \Lsym^2) = \Xi\left(\tfrac{1}{2} + i\lambda\right),
\]
which is rigorously established in Chapter~\ref{sec:determinant_identity} using Laplace–Mellin regularization and Paley--Wiener theory.

\paragraph{Comparison to Prior Spectral Models.}
Unlike heuristic frameworks such as Hilbert--Pólya or the spectral traces of Connes~\cite{Connes1999TraceFormula} and Deninger~\cite{Deninger1998Frobenius}, the operator \( \Lsym \) is rigorously constructed within classical Hilbert space analysis. It enjoys explicit control over domain, trace, convergence, and self-adjoint structure, and fulfills all analytic prerequisites for a canonical spectral realization of \( \Xi(s) \) and its zeros.
