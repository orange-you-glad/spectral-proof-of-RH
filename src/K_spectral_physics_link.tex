\section{Spectral Physics Interpretation}
\label{app:spectral-physics-link}

\noindent\textbf{[Noncritical Appendix]}  
This appendix explores speculative physical interpretations of the canonical operator \( L_{\mathrm{sym}} \in \mathcal{C}_1(H_{\Psi_\alpha}) \). While no physical model is constructed here, the operator admits a formal analogy with a quantum Hamiltonian whose spectrum encodes the (rescaled) nontrivial zeros of the Riemann zeta function:
\[
\operatorname{Spec}(L_{\mathrm{sym}}) = \left\{ \mu_n := \frac{1}{\gamma_n} \;\middle|\; \zeta\left(\tfrac{1}{2} + i\gamma_n\right) = 0 \right\}.
\]

\subsection*{Partition Function Analogy}

The heat trace
\[
Z(t) := \operatorname{Tr}(e^{-t L_{\mathrm{sym}}})
\]
formally resembles a partition function for a quantum system with eigenvalues \( \mu_n \). Its singular short-time expansion,
\[
Z(t) \sim \frac{1}{\sqrt{4\pi t}} \log\left( \frac{1}{t} \right) + o(t^{-1/2}), \quad t \to 0^+,
\]
is typical of Laplace-type operators on singular or arithmetic manifolds, and reflects the logarithmic divergence found in trace formulas for noncompact spaces.

The zeta-regularized spectral determinant
\[
\log \det\nolimits_\zeta(I - \lambda L_{\mathrm{sym}})
\]
acts analogously to a free energy term, consistent with the spectral action principle in noncommutative geometry~\cite{Connes1999TraceFormula, Chamseddine2007SpectralAction}.

\subsection*{GUE Statistics and Inverse Spectrum}

Under the spectral map
\[
\gamma_n \longmapsto \mu_n := \frac{1}{\gamma_n},
\]
the conjectural GUE distribution of zeta zeros~\cite{Montgomery1973PairCorrelation, Berry1986RiemannSpectra} is transformed into a nonlinear spacing distribution on the inverse spectrum \( \{ \mu_n \} \). Level repulsion and rigidity are preserved qualitatively, but the inverse mapping compresses large eigenvalues and emphasizes low-frequency arithmetic structure.

This suggests that \( L_{\mathrm{sym}} \) may model a nonstandard Hamiltonian with inverted arithmetic energy scales—compressing high-energy dynamics into a trace-class framework.

\subsection*{Caveats and Interpretation}

These analogies are illustrative only and do not influence any analytic result in the manuscript. No physical Hamiltonian, Lagrangian, or path integral formalism is required to define \( L_{\mathrm{sym}} \).

Nonetheless, this perspective may guide future inquiry into:
\begin{itemize}
  \item Quantum mechanical realizations of spectral zeta functions;
  \item Inverse-spectral ensembles and statistical mechanics of determinant models;
  \item Hamiltonian or path-integral interpretations of arithmetic trace formulas.
\end{itemize}

\medskip
\noindent
The operator \( L_{\mathrm{sym}} \) offers a canonical analytic realization of the nontrivial zeta spectrum. Whether it also admits a meaningful quantization or physical interpretation remains an open question—one that sits at the intersection of spectral theory, arithmetic, and quantum geometry.

\medskip
\noindent
For the analytic realization of the spectral equivalence \( \mathrm{RH} \iff \operatorname{Spec}(L_{\mathrm{sym}}) \subset \mathbb{R} \), see Chapter~\ref{sec:spectral-implications}.
