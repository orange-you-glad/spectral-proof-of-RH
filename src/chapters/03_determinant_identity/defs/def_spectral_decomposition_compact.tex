\begin{definition}[Spectral Decomposition of Compact Self-Adjoint Operators]
\label{def:spectral_decomposition_compact}
Let \( H \) be a separable complex Hilbert space, and let \( T \in \TC(H) \) be a compact, self-adjoint operator.

Then there exists an orthonormal basis \( \{ e_n \}_{n=1}^\infty \subset H \) consisting of eigenvectors of \( T \), with associated eigenvalues \( \{ \lambda_n \}_{n=1}^\infty \subset \R \), counted with algebraic multiplicity and satisfying \( \lambda_n \to 0 \) as \( n \to \infty \), such that for all \( f \in H \),
\[
T f = \sum_{n=1}^\infty \lambda_n \ip{f}{e_n} e_n,
\]
where the series converges in the norm topology of \( H \).

\medskip
\noindent
This diagonalization expresses \( T \) via the spectral theorem as a normal operator with pure point spectrum and no continuous or residual part. In particular:
\begin{itemize}
  \item The trace is given by
  \[
  \Tr(T) = \sum_{n=1}^\infty \lambda_n,
  \]
  which converges absolutely by the trace-class condition;

  \item The trace norm satisfies
  \[
  \|T\|_{\TC} = \sum_{n=1}^\infty |\lambda_n|;
  \]

  \item The spectral functional calculus applies: for any holomorphic function \( \phi \) defined on a neighborhood of the spectrum \( \{ \lambda_n \} \), the operator \( \phi(T) \colon H \to H \) is given by
  \[
  \phi(T) f = \sum_{n=1}^\infty \phi(\lambda_n) \ip{f}{e_n} e_n.
  \]
\end{itemize}

\medskip
\noindent
This Hilbert–Schmidt spectral resolution forms the analytic foundation for trace expansions, heat kernel asymptotics, spectral zeta functions, and Fredholm or Carleman determinant identities associated with \( T \).
\thmref{thm:det_identity_revised}
\end{definition}
