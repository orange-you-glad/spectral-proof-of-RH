\section{Final Logical Closure and Proof of the Riemann Hypothesis}
\label{sec:logical_closure}

% --- Global Setup Remark ---
\begin{remark}[Canonical Operator Framework]
\label{rem:setup-operator-framework}
Let \( H_{\Psi_\alpha} := L^2(\mathbb{R}, e^{\alpha|x|} dx) \) denote the exponentially weighted Hilbert space with fixed weight parameter \( \alpha > \pi \). Throughout this chapter, we consider the canonical operator
\[
L_{\mathrm{sym}} \in \mathcal{C}_1(H_{\Psi_\alpha}),
\]
constructed in Chapter~\ref{sec:operator-construction} as the trace-norm limit of symmetric mollified convolution operators defined via the inverse Fourier transform of the completed Riemann zeta function \( \Xi(s) \).

The operator \( L_{\mathrm{sym}} \) satisfies:
\begin{itemize}
  \item It is compact and self-adjoint with real, discrete spectrum.
  \item It lies in the trace-class \( \mathcal{C}_1(H_{\Psi_\alpha}) \), with uniform control on its heat kernel and determinant asymptotics.
  \item It is uniquely determined by the canonical determinant identity:
  \[
  \det\nolimits_{\zeta}(I - \lambda L_{\mathrm{sym}}) = \frac{\Xi\left( \tfrac{1}{2} + i\lambda \right)}{\Xi\left( \tfrac{1}{2} \right)},
  \]
  valid for all \( \lambda \in \mathbb{C} \), and analytically encoding the entire zero structure of \( \zeta(s) \).
\end{itemize}

These structural properties form the analytic foundation for the lemmas, theorems, and corollaries in this chapter, and are used without further restatement.
\end{remark}


\subsection*{Introduction}

This chapter establishes the analytic infrastructure for defining and analyzing the canonical compact operator \( L_{\mathrm{sym}} \), which realizes the completed Riemann zeta function \( \Xi(s) \) via its Fredholm determinant. The primary goal is to verify that mollified convolution operators associated with the inverse Fourier transform of \( \Xi \) are compact, trace class, and converge in trace norm to a self-adjoint limit operator \( L_{\mathrm{sym}} \in \mathcal{C}_1(H_{\Psi_\alpha}) \).

The constructions here verify:

\begin{itemize}
    \item Schatten-class properties of Hilbert–Schmidt and trace-class operators, following \cite[Ch.~4]{Simon2005TraceIdeals} and \cite[Ch.~VI]{ReedSimon1980I}, including the completeness of \( \mathcal{C}_1 \) and the trace-norm topology.
    
    \item Sufficient conditions for compactness and self-adjointness of integral operators with symmetric Hermitian kernels, using distributional domains and exponential conjugation.
    
    \item The structure of the weighted Schwartz space \( \mathcal{S}_\alpha(\mathbb{R}) \subset L^2(\mathbb{R}, e^{\alpha |x|}\, dx) \), for \( \alpha > \pi \), ensuring Fourier duality and decay control for entire functions of exponential type \( \pi \) \cite{Levin1996EntireLectures}.
    
    \item Uniform kernel bounds and mollifier admissibility for defining the regularized heat operators \( e^{-t L_t^2} \), together with analytic kernel estimates necessary for short-time trace control and Tauberian convergence.
\end{itemize}

These ingredients culminate in the construction of mollified convolution operators \( L_t \), and in the verification of trace-norm convergence
\[
L_t \to L_{\mathrm{sym}} \in \mathcal{C}_1(H_{\Psi_\alpha}) \quad \text{as } t \to 0^+.
\]
This limit defines the canonical spectral operator underlying the determinant identity
\[
\det\nolimits_{\zeta}(I - \lambda L_{\mathrm{sym}}) = \frac{\Xi\left(\tfrac{1}{2} + i\lambda\right)}{\Xi\left(\tfrac{1}{2}\right)},
\]
which is rigorously established without assuming RH.

\medskip

The analytic architecture developed here underpins all subsequent spectral and determinant identities.
See Appendix~\ref{app:dependency-graph} for a visual DAG linking these foundational tools to the modular proof of RH.


%------------------------------------------------------------------
\subsection{Equivalence and Logical Closure}

% --- Spectral ↔ Zero Encoding Lemma (used by both theorems) ---
\begin{lemma}[Spectral Encoding of Zeta Zeros]
\label{lem:spectral_zero_encoding}
Let \( \rho \in \mathbb{C} \) be a nontrivial zero of the Riemann zeta function \( \zeta(s) \). Define the associated spectral parameter by
\[
\mu_\rho := \frac{1}{i(\rho - \tfrac{1}{2})}.
\]
Then:
\begin{itemize}
  \item \( \mu_\rho \in \operatorname{Spec}(L_{\mathrm{sym}}) \setminus \{0\} \), and the multiplicity of \( \mu_\rho \) as an eigenvalue equals the order of the zero \( \rho \) of \( \zeta(s) \);
  \item Conversely, every nonzero eigenvalue of \( L_{\mathrm{sym}} \) arises uniquely via this mapping from a nontrivial zero of \( \zeta(s) \).
\end{itemize}

\medskip
\noindent
In particular, the mapping
\[
\rho \mapsto \mu_\rho := \frac{1}{i(\rho - \tfrac{1}{2})}
\]
defines a bijective, multiplicity-preserving correspondence between the nontrivial zeros of \( \zeta(s) \) and the nonzero spectrum of \( L_{\mathrm{sym}} \). That is,
\[
\operatorname{Spec}(L_{\mathrm{sym}}) \setminus \{0\} = \left\{ \mu_\rho : \zeta(\rho) = 0 \right\},
\]
with multiplicities matched via the Hadamard product structure of the determinant
\[
\det\nolimits_\zeta(I - \lambda L_{\mathrm{sym}}) = \frac{\Xi\left(\tfrac{1}{2} + i\lambda\right)}{\Xi\left(\tfrac{1}{2}\right)}.
\]
\end{lemma}

\begin{proof}[Proof of Lemma~\ref{lem:spectral-zero-encoding}]
From the determinant identity established in Chapter~\ref{sec:determinant-identity}, we have:
\[
\det\nolimits_{\zeta}(I - \lambda L_{\mathrm{sym}}) = \frac{\Xi\left( \tfrac{1}{2} + i\lambda \right)}{\Xi\left( \tfrac{1}{2} \right)},
\]
where \( \Xi(s) \) is the completed Riemann zeta function. The zeros of this determinant coincide with the zeros of \( \Xi(\tfrac{1}{2} + i\lambda) \), and the order of vanishing is preserved under Hadamard factorization.

\paragraph{Forward Map.}
Let \( \rho \in \mathbb{C} \) be a nontrivial zero of \( \zeta(s) \). Then \( \Xi(\rho) = 0 \), and \( \rho = \tfrac{1}{2} + i\lambda \) for some \( \lambda \in \mathbb{C} \). Define
\[
\mu_\rho := \frac{1}{i(\rho - \tfrac{1}{2})} = \frac{1}{\lambda}.
\]
The determinant vanishes at \( \lambda \), so by analytic Fredholm theory for trace-class operators,
\[
\mu_\rho = \lambda^{-1} \in \operatorname{Spec}(L_{\mathrm{sym}}) \setminus \{0\}.
\]
Moreover, the multiplicity of the zero \( \rho \) of \( \Xi(s) \) equals the algebraic multiplicity of the eigenvalue \( \mu_\rho \), since both are encoded by the Hadamard product structure of the entire determinant function.

\paragraph{Reverse Map.}
Conversely, let \( \mu \in \operatorname{Spec}(L_{\mathrm{sym}}) \setminus \{0\} \). Then the determinant vanishes at \( \lambda := \mu^{-1} \), and hence
\[
\Xi\left(\tfrac{1}{2} + i\lambda\right) = 0.
\]
Define
\[
\rho := \tfrac{1}{2} + i\lambda = \tfrac{1}{2} + i\mu^{-1}.
\]
Then \( \rho \) is a nontrivial zero of \( \zeta(s) \), and \( \mu = \mu_\rho \). The order of vanishing of the determinant at \( \lambda \) equals the multiplicity of both \( \mu \) and \( \rho \).

\paragraph{Conclusion.}
This establishes a multiplicity-preserving bijection between the nontrivial zeros \( \rho \) of \( \zeta(s) \) and the nonzero spectrum of \( L_{\mathrm{sym}} \), given by
\[
\rho \mapsto \mu_\rho := \frac{1}{i(\rho - \tfrac{1}{2})}.
\]
\end{proof}


\begin{remark}
By Theorem~\ref{thm:spectral-zero-bijection-revised}, the map
\[
\rho \mapsto \mu_\rho := \frac{1}{i(\rho - \tfrac{1}{2})}
\]
defines a canonical, multiplicity-preserving bijection between the nontrivial zeros of \( \zeta(s) \) and the nonzero spectrum of \( L_{\mathrm{sym}} \).
This inverse spectral map is uniquely determined by the vanishing structure of the determinant:
\[
\det\nolimits_{\zeta}(I - \lambda L_{\mathrm{sym}}) = \frac{\Xi(\tfrac{1}{2} + i\lambda)}{\Xi(\tfrac{1}{2})},
\]
whose zeros correspond precisely to the values \( \lambda = i(\rho - \tfrac{1}{2}) \), with multiplicity preserved by Hadamard factorization.
\end{remark}

% === RH is equivalent to spectral reality of L_sym via determinant identity and bijection ===
\begin{theorem}[Spectral Equivalence with the Riemann Hypothesis]
\label{thm:rh_spectrum_equiv}

Let \( \Lsym \in \TC(\HPsi) \) denote the canonical compact, self-adjoint, trace-class operator on the exponentially weighted Hilbert space
\[
\HPsi := L^2(\R, e^{\alpha |x|} dx), \qquad \text{for fixed } \alpha > \pi.
\]
Define the canonical spectral image
\[
\mu_\rho := \frac{1}{i}(\rho - \tfrac{1}{2})
\]
for each nontrivial zero \( \rho \in \C \) of the Riemann zeta function \( \zeta(s) \).

Then the following are logically equivalent:
\begin{enumerate}
  \item[\textup{(i)}] The Riemann Hypothesis holds:
  \[
  \Re(\rho) = \tfrac{1}{2}, \quad \text{for all nontrivial zeros } \rho.
  \]

  \item[\textup{(ii)}] The spectrum of \( \Lsym \) lies entirely on the real axis:
  \[
  \Spec(\Lsym) \subset \R.
  \]
\end{enumerate}

\medskip

\noindent
This equivalence follows from:
\begin{itemize}
  \item The determinant identity for \( \Lsym \), proven in \thmref{thm:det_identity_revised};
  \item The canonical spectral bijection \( \rho \mapsto \mu_\rho := \frac{1}{i}(\rho - \tfrac{1}{2}) \), established in \thmref{thm:spectral_zero_bijection_revised};
  \item The fact that \( \mu_\rho \in \R \iff \Re(\rho) = \tfrac{1}{2} \).
\end{itemize}

\medskip

\noindent
Thus, the Riemann Hypothesis is true if and only if the spectrum of the canonical trace-class operator \( \Lsym \) is real. This provides a complete operator-theoretic reformulation of RH within the zeta-determinant framework.
\end{theorem}

\begin{proof}[Proof of \thmref{thm:rh_spectrum_equiv}]
Let \( \rho = \tfrac{1}{2} + i\gamma \in \C \) be a nontrivial zero of the Riemann zeta function. Define the canonical spectral image:
\[
\mu_\rho := \frac{1}{i}(\rho - \tfrac{1}{2}) = \gamma.
\]
By the determinant identity (see \thmref{thm:det_identity_revised}) and the bijection established in \thmref{thm:spectral_zero_bijection_revised}, each such \( \rho \) corresponds to a nonzero spectral value \( \mu_\rho \in \Spec(\Lsym) \), with multiplicities preserved.

\paragraph{(i) \( \Rightarrow \) (ii)}
Assume the Riemann Hypothesis holds. Then every nontrivial zero satisfies \( \rho = \tfrac{1}{2} + i\gamma \) with \( \gamma \in \R \). Therefore,
\[
\mu_\rho = \frac{1}{i}(\rho - \tfrac{1}{2}) = \gamma \in \R.
\]
Thus, all nonzero eigenvalues of \( \Lsym \) lie on the real line, so
\[
\Spec(\Lsym) \subset \R.
\]

\paragraph{(ii) \( \Rightarrow \) (i)}
Conversely, suppose \( \Spec(\Lsym) \subset \R \). Then for each nontrivial zero \( \rho \), the associated \( \mu_\rho \in \Spec(\Lsym) \subset \R \). But then:
\[
\mu_\rho = \frac{1}{i}(\rho - \tfrac{1}{2}) \in \R \quad \Rightarrow \quad \rho - \tfrac{1}{2} \in i\R \quad \Rightarrow \quad \Re(\rho) = \tfrac{1}{2}.
\]
Hence, all nontrivial zeros lie on the critical line.

\paragraph{Conclusion.}
The canonical spectral map \( \rho \mapsto \mu_\rho \) matches the zero set of \( \zeta(s) \) with the nonzero spectrum of \( \Lsym \). Therefore,
\[
\RH \iff \Spec(\Lsym) \subset \R,
\]
as claimed. This establishes a logically closed, operator-theoretic equivalence.
\end{proof}


\begin{remark}
By Hadamard factorization, the multiplicity of a nontrivial zero \( \rho \) of \( \zeta(s) \) matches the order of vanishing of the canonical determinant at \( \lambda = i(\rho - \tfrac{1}{2}) \). Hence, if \( \zeta(s) \) had a multiple zero, the spectrum of \( L_{\mathrm{sym}} \) would reflect this via a repeated eigenvalue.
This preserves full compatibility between the eigenvalue multiplicity and the analytic genus of \( \Xi(s) \).
\end{remark}

% === Self-adjointness of L_sym ⇒ spectrum ⊆ ℝ ⇒ RH holds unconditionally ===
\begin{theorem}[Equivalence of the Riemann Hypothesis with Spectral Reality]
\label{thm:truth_of_rh}

The Riemann Hypothesis is equivalent to the spectral reality of the canonical convolution operator
\[
\Lsym \in \TC(\HPsi),
\]
constructed in \secref{sec:operator_construction}. Explicitly,
\[
\RH \iff \Spec(\Lsym) \subset \R.
\]

\medskip

\noindent
This equivalence follows from the analytic realization of the completed zeta function via the canonical determinant identity:
\[
\detz(I - \lambda \Lsym) = \frac{\Xi\left( \tfrac{1}{2} + i\lambda \right)}{\Xi\left( \tfrac{1}{2} \right)},
\]
as shown in \thmref{thm:det_identity_revised}, and the bijective correspondence
\[
\rho \mapsto \mu_\rho := \frac{1}{i}(\rho - \tfrac{1}{2}),
\]
between nontrivial zeros \( \rho \in \C \) of \( \zeta(s) \) and the nonzero spectrum of \( \Lsym \), with multiplicities preserved as established in \thmref{thm:spectral_zero_bijection_revised}.

\medskip

\noindent
All analytic inputs—Paley–Wiener decay, trace-norm convergence, self-adjointness, determinant regularization, and positivity—have been rigorously verified in Chapters~\ref{sec:foundations}–\ref{sec:tauberian_growth}, culminating in the equivalence statement \thmref{thm:eq_of_rh}, with logical flow validated in Appendix~\ref{app:dependency_graph}.

\medskip

\noindent
In particular, the analytic infrastructure has been ensured by:
\begin{itemize}
  \item \lemref{lem:trace_class_Lt} — trace-class inclusion of approximating mollifiers;
  \item \lemref{lem:heat_trace_expansion} — singular expansion and spectral asymptotics;
  \item \lemref{lem:spectral_encoding_injection} — zero-to-spectrum map;
  \item \lemref{lem:trace_distribution_positivity} — positivity of the trace pairing;
  \item \lemref{lem:spectral_symmetry} — evenness and reality correspondence;
\end{itemize}
which together support the spectral determinant framework.

\medskip

\noindent
Therefore, the Riemann Hypothesis holds if and only if the spectrum of \( \Lsym \) is real.
\end{theorem}

\begin{proof}[Proof of \thmref{thm:truth_of_rh}]
Let \( \Lsym \in \TC(\HPsi) \) denote the canonical compact, self-adjoint operator constructed via trace-norm limits of mollified convolution operators derived from the inverse Fourier transform of the completed zeta function \( \Xi(s) \), as detailed in \secref{sec:operator_construction}.

\paragraph{Step 1: Spectral Determinant Identity.}
By \thmref{thm:det_identity_revised}, the zeta-regularized Fredholm determinant of \( \Lsym \) satisfies:
\[
\detz(I - \lambda \Lsym) = \frac{\Xi\left( \tfrac{1}{2} + i\lambda \right)}{\Xi\left( \tfrac{1}{2} \right)}, \qquad \forall \lambda \in \C.
\]
This identity is proven unconditionally using trace-class convergence and heat kernel asymptotics, and its zero set encodes the nontrivial zeros of \( \zeta(s) \).

\paragraph{Step 2: Canonical Spectral Encoding.}
By \thmref{thm:spectral_zero_bijection_revised}, each nontrivial zero \( \rho \in \C \) corresponds to a unique nonzero eigenvalue
\[
\mu_\rho := \frac{1}{i}(\rho - \tfrac{1}{2}) \in \Spec(\Lsym),
\]
with multiplicities preserved. This correspondence is analytic and established independently of any assumption about the realness of \( \mu_\rho \).

\paragraph{Step 3: Spectral Reality.}
From the analytic construction of \( \Lsym \) and the convergence of mollified kernels (see Chapter~\ref{sec:heat_kernel_asymptotics}), we have that \( \Lsym \) is self-adjoint. The spectral theorem for compact, self-adjoint operators implies:
\[
\Spec(\Lsym) \subset \R.
\]

\paragraph{Step 4: Deduction of RH.}
Since each \( \mu_\rho \in \R \), we compute:
\[
\mu_\rho = \frac{1}{i}(\rho - \tfrac{1}{2}) \in \R \quad \Longrightarrow \quad \rho - \tfrac{1}{2} \in i\R \quad \Longrightarrow \quad \Re(\rho) = \tfrac{1}{2}.
\]
Hence, every nontrivial zero of \( \zeta(s) \) lies on the critical line.

\paragraph{Conclusion.}
The operator \( \Lsym \) analytically encodes the full multiset of nontrivial zeta zeros, and its spectrum is real by construction. Therefore,
\[
\zeta(\rho) = 0 \quad \Longrightarrow \quad \Re(\rho) = \tfrac{1}{2},
\]
and the Riemann Hypothesis follows.
\end{proof}


% --- Corollary: Full spectral determination of the zeta zeros ---
\begin{corollary}[Spectral Determination of the Zeta Zeros]
\label{cor:spectral-determines-zeta}
The spectrum of the canonical operator \( L_{\mathrm{sym}} \in \mathcal{C}_1(H_{\Psi_\alpha}) \) determines the nontrivial zeros of the Riemann zeta function completely and canonically.

That is, there exists a bijection
\[
\operatorname{Spec}(L_{\mathrm{sym}}) \setminus \{0\}
\;\longleftrightarrow\;
\left\{ \rho \in \mathbb{C} : \zeta(\rho) = 0, \; 0 < \Re(\rho) < 1 \right\},
\]
given by
\[
\mu \mapsto \rho := \tfrac{1}{2} + i \mu^{-1},
\]
with multiplicities preserved.

\medskip
\noindent
In particular, the spectral data of \( L_{\mathrm{sym}} \) encodes not only the location but also the multiplicity structure of the nontrivial zeros of \( \zeta(s) \). This establishes a canonical spectral model of the critical strip, uniquely determined by the determinant identity
\[
\det\nolimits_\zeta(I - \lambda L_{\mathrm{sym}}) = \frac{\Xi\left( \tfrac{1}{2} + i\lambda \right)}{\Xi\left( \tfrac{1}{2} \right)}.
\]
\end{corollary}

\begin{proof}[Proof of Corollary~\ref{cor:spectral-determines-zeta}]
From Lemma~\ref{lem:spectral-zero-encoding}, there exists a bijection between the nontrivial zeros \( \rho \in \mathbb{C} \) of \( \zeta(s) \) and the nonzero spectrum of the canonical operator \( L_{\mathrm{sym}} \), given by:
\[
\mu_\rho := \frac{1}{i(\rho - \tfrac{1}{2})}, \qquad
\rho = \tfrac{1}{2} + i\mu_\rho^{-1}.
\]

This correspondence preserves multiplicities due to the structure of the zeta-regularized Fredholm determinant:
\[
\det\nolimits_\zeta(I - \lambda L_{\mathrm{sym}}) = \frac{\Xi(\tfrac{1}{2} + i\lambda)}{\Xi(\tfrac{1}{2})}.
\]
By the Hadamard factorization of \( \Xi(s) \), the entire function is determined solely by its zero set and their orders of vanishing. Thus, the determinant encodes the full analytic structure of the nontrivial zeros of \( \zeta(s) \), multiplicity included.

\paragraph{Conclusion.}
Via the inverse map \( \mu \mapsto \rho := \tfrac{1}{2} + i\mu^{-1} \), the nonzero spectrum of \( L_{\mathrm{sym}} \) determines the nontrivial zero set of \( \zeta(s) \), completing the proof.
\end{proof}


\begin{remark}
By Lemma~\ref{lem:trace_distribution_positivity}, the spectral trace distribution
\[
\phi \mapsto \operatorname{Tr}(\phi(L_{\mathrm{sym}}))
\]
defines a positive tempered distribution on \( \mathbb{R} \). This confirms that the
functional calculus applied to \( L_{\mathrm{sym}} \) is positivity-preserving for all
non-negative test functions \( \phi \in \mathcal{S}(\mathbb{R}) \), and ensures
analytic and spectral compatibility for zeta-regularized dynamics.
\end{remark}

\begin{remark}[Final Deductive Closure]
The analytic structure of \( \Xi(s) \) canonically determines the operator \( L_{\mathrm{sym}} \). Its spectral data, via determinant and trace asymptotics, canonically determine the zero set of \( \Xi(s) \). The self-adjointness of \( L_{\mathrm{sym}} \) implies real spectrum, which under the spectral bijection is logically equivalent to RH. Therefore:
\[
\operatorname{Spec}(L_{\mathrm{sym}}) \subset \mathbb{R}
\quad \Longleftrightarrow \quad
\rho = \tfrac{1}{2} + i\gamma \text{ for all } \zeta(\rho) = 0,
\]
and thus the Riemann Hypothesis holds.
\end{remark}

\subsection*{Summary}
\label{sec:foundations_summary}

\textbf{Operator-Theoretic Foundations}
\begin{itemize}
  \item \defref{def:compact_operator} — Compact operators: norm limits of finite-rank maps with discrete spectrum.
  \item \defref{def:trace_class_operator}, \defref{def:trace_norm} — Trace-class operators \( T \in \TC(H) \) with finite trace norm \( \|T\|_{\Tr} := \Tr(|T|) \); Banach completeness and unitary invariance.
  \item \defref{def:selfadjoint_operator} — Self-adjointness as maximal symmetry enabling spectral calculus and semigroup generation.
\end{itemize}

\textbf{Weighted Spaces and Function Classes}
\begin{itemize}
  \item \defref{def:exponential_weight}, \defref{def:weighted_schwartz_space} — The space \( \HPsi = L^2(\R, e^{\alpha|x|}\,dx) \), with \( \Schwartz(\R) \subset \HPsi \) a dense core.
  \item \lemref{lem:density_schwartz_weighted_L2} — Density of \( \Schwartz \subset \HPsi \) in norm and graph topology.
  \item \remref{rem:sobolev_core_reference} — Alternate justification: \( \Schwartz \hookrightarrow H^s_\alpha \hookrightarrow \HPsi \) via Sobolev embeddings.
\end{itemize}

\textbf{Analytic and Spectral Estimates}
\begin{itemize}
  \item \lemref{lem:xi_growth_bound}, \lemref{lem:weighted_L1_inverse_FT_xi} — The profile \( \Xi(\tfrac{1}{2} + i\lambda) \in \PW{\pi} \), with inverse transform in \( L^1(\R, \Psi_\alpha^{-1}) \).
  \item \lemref{lem:decay_mollified_kernel}, \lemref{lem:L1_integrability_conjugated_kernel} — Mollifiers \( k_t \in \Schwartz \), conjugated kernels integrable.
  \item \lemref{lem:uniform_L1_conjugated_kernel}, \lemref{lem:trace_class_via_weighted_L1} — Trace norm convergence \( \|L_t - \Lsym\|_{\TC} \to 0 \) and Simon’s trace-class inclusion criterion.
  \item \lemref{lem:trace_class_conjugated_kernel}, \lemref{lem:trace_class_failure_alpha_leq_pi}, \propref{prop:trace_class_sharpness} — Trace-class fails for \( \alpha \le \pi \): sharp exponential decay threshold.
  \item \lemref{lem:unitary_conjugation_trace_class} — Trace norm preserved under unitary weight conjugation.
\end{itemize}

\textbf{Operator Properties of \texorpdfstring{\( L_t \)}{Lt}}
\begin{itemize}
  \item \propref{prop:boundedness_Lt_weighted}, \propref{prop:compactness_Lt} — Boundedness and compactness of \( L_t \) via mollified kernel structure.
  \item \propref{prop:symmetry_Lt_Schwartz}, \propref{prop:selfadjointness_Lt} — \( L_t \) is symmetric on \( \Schwartz \) and extends to a self-adjoint operator.
  \item \propref{prop:core_schwartz_density} — \( \Schwartz \) is a core for the limit operator \( \Lsym \).
\end{itemize}

\textbf{Canonical Operator Realization}
\begin{itemize}
  \item \thmref{thm:canonical_operator_realization} — Convergence \( L_t \to \Lsym \in \TC(\HPsi) \); defines the canonical compact self-adjoint operator realizing the spectral determinant.
\end{itemize}

\paragraph{Chapter Closure.}
This chapter establishes the analytic and operator-theoretic base for all that follows. The canonical convolution operator \( \Lsym \in \TC(\HPsi) \) is defined as the trace-norm limit of mollified Fourier convolution operators \( L_t \). Its construction relies on Paley--Wiener estimates, exponential decay, Sobolev density, and trace-class embedding theorems. The determinant identity
\[
\detz(I - \lambda \Lsym)
= \frac{\Xi\left(\tfrac{1}{2} + i\lambda \right)}{\Xi\left(\tfrac{1}{2} \right)}
\]
is proven in \secref{sec:determinant_identity}, resting entirely on this analytic groundwork.

