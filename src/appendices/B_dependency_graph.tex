\section{Logical Dependency Graph (Modular Proof Architecture)}
\label{app:dependency_graph}

This appendix presents the formal structure of the manuscript as a directed acyclic graph (DAG), in which each chapter builds only on previously established analytic foundations. No theorem appeals to any result logically equivalent to the Riemann Hypothesis prior to its proof, ensuring strict acyclicity and audit transparency.

For symbol definitions, see Appendix~\ref{app:notation_summary}.

\vspace{1ex}
\hrule
\vspace{1ex}

\begin{quote}
\textit{This proof flow diagram captures the modular structure of the analytic–spectral program. Each node reflects an acyclic dependency, ensuring strict logical sequencing from foundational definitions to the final equivalence with RH.}
\end{quote}

\vspace{1ex}
\hrule
\vspace{2ex}

\subsection*{Analytic Preconditions for the Determinant Identity}
\label{dag:determinant_preconditions}

The analytic identity
\[
\detz(I - \lambda \Lsym) = \frac{\XiR(\tfrac{1}{2} + i\lambda)}{\XiR(\tfrac{1}{2})}
\]
relies on the following validated properties, derived across Chapters~1–2 and Appendix~\ref{app:zeta_trace_background}:

\begin{center}
\renewcommand{\arraystretch}{1.4}
\begin{tabularx}{\textwidth}{|l|X|}
\hline
\textbf{Property} & \textbf{Summary and Source} \\
\hline
\textbf{Exponential Weight Threshold} & \( \alpha > \pi \) is necessary and sufficient for kernel inclusion in \( \TC(\HPsi) \); sharp trace-class cutoff established in \cref{lem:trace_class_failure_alpha_leq_pi}. \\
\textbf{Spectral Profile Class} & \( \phi(\lambda) := \XiR(\tfrac{1}{2} + i\lambda) \in \PW{\pi} \) ensures exponential control (\cref{lem:xi_growth_bound}). \\
\textbf{Kernel Localization} & \( \widehat{\Xi}(x) \in L^1(\R, e^{-\alpha |x|} dx) \) for all \( \alpha > \pi \) (\cref{lem:weighted_L1_inverse_FT_xi}). \\
\textbf{Operator Regularity} & \( \Lsym \in \TC(\HPsi) \) is compact, self-adjoint, and uniquely defined (\cref{lem:trace_norm_convergence_Lt_to_Lsym}, \cref{lem:trace_norm_limit_unique}). \\
\textbf{Heat Semigroup Well-Posedness} & \( e^{-t\Lsym^2} \in \TC(\HPsi) \), analytic with decay bounds (\cref{lem:heat_semigroup_existence}, \cref{lem:heat_semigroup_wellposed}). \\
\textbf{Determinant Growth and Entirety} & \( \detz \) is entire of exponential type \( \pi \), matching trace data (\cref{lem:det_identity_entire_order_one}). \\
\hline
\end{tabularx}
\end{center}

\subsection*{Directed Acyclic Graph (Visual Overview)}
\begin{figure}[ht]
\centering
\scalebox{0.87}{
\begin{tikzpicture}[
  node distance=1.4cm and 2.6cm,
  every node/.style={draw, align=center, rounded corners=3pt, font=\small, fill=gray!5, text width=4.8cm},
  arrow/.style={-{Stealth}, thick}
]

\node (ch1) {Ch.~1\\Foundational Structures};
\node (ch2) [below=of ch1] {Ch.~2\\Operator Construction};
\node (ch3) [below=of ch2] {Ch.~3\\Determinant Identity};
\node (ch4) [right=of ch3] {Ch.~4\\Spectral Bijection};
\node (ch5) [below=of ch3] {Ch.~5\\Heat Kernel Asymptotics};
\node (ch6) [right=of ch5] {Ch.~6\\RH Equivalence};
\node (ch7) [below=of ch5] {Ch.~7\\Tauberian Growth};
\node (ch8) [right=of ch7] {Ch.~8\\Spectral Rigidity};
\node (ch9) [below=of ch7] {Ch.~9\\Logical Closure};

\draw[arrow] (ch1) -- (ch2);
\draw[arrow] (ch2) -- (ch3);
\draw[arrow] (ch3) -- (ch5);
\draw[arrow] (ch5) -- (ch7);
\draw[arrow] (ch7) -- (ch9);

\draw[arrow] (ch3) -- (ch4);
\draw[arrow] (ch4) -- (ch6);
\draw[arrow] (ch5) -- (ch6);
\draw[arrow] (ch6) -- (ch8);
\draw[arrow] (ch8) -- (ch9);

\end{tikzpicture}
}
\caption{Directed acyclic graph of analytic dependencies. Arrows represent logical flow between chapters.}
\label{fig:dag_visual}
\end{figure}


Each lemma, proposition, and theorem in this manuscript depends only on prior analytic infrastructure or trace-class spectral calculus. No assumption of the Riemann Hypothesis, spectral bijection, or spectral reality is made until explicitly proven.

The analytic–spectral chain from the determinant identity to the RH equivalence
\[
\RH \iff \Spec(\Lsym) \subset \R
\]
is modular, acyclic, and fully closed in \cref{sec:logical_closure}, relying only on results from Chapters~1–8 and the analytic infrastructure in Appendix~\ref{app:zeta_trace_background}.

A structurally critical constraint in this chain is the exponential weight threshold \( \alpha > \pi \), which governs the trace-class viability of all kernel realizations and appears as the base node of this DAG (\cref{lem:trace_class_failure_alpha_leq_pi}).

\begin{remark}[Closure of Forward Dependencies]
The determinant identity in Chapter~\ref{sec:determinant_identity} invokes heat trace asymptotics and Laplace integrability properties that are rigorously derived in Chapter~\ref{sec:heat_kernel_asymptotics} and Appendix~\ref{app:heat_kernel_construction}. These analytic inputs validate the determinant’s growth class, log-derivative identity, and short-time singular structure. Their forward use is modular and does not induce any logical cycles.
\end{remark}

Forward dependency disclosure is given in \remref{rem:forward_spectral_closure}, and the full analytic equivalence is formally stated in \thmref{thm:rh_spectrum_equiv} and concluded in \thmref{thm:truth_of_rh}.

The same DAG structure extends conjecturally to the automorphic setting discussed in Chapter~\ref{sec:spectral_generalization}.  There one postulates analogous operators \(L_\pi\) whose spectra encode the zeros of \(\Lambda_\pi(s)\), giving a pathway toward proving \(\GRH(\pi) \iff \Spec(L_\pi) \subset \R\).

For symbol reference, see Appendix~\ref{app:notation_summary}.
