\begin{proof}[Proof of Lemma~\ref{lem:spectral-symmetry-lsym}]
The canonical determinant identity is
\[
\det\nolimits_{\zeta}(I - \lambda L_{\mathrm{sym}}) = \frac{\Xi(\tfrac{1}{2} + i\lambda)}{\Xi(\tfrac{1}{2})},
\]
where \( \Xi(s) \) is entire of order one and satisfies the functional symmetry
\[
\Xi\left(\tfrac{1}{2} + i\lambda\right) = \Xi\left(\tfrac{1}{2} - i\lambda\right).
\]
It follows that
\[
\det\nolimits_{\zeta}(I - \lambda L_{\mathrm{sym}}) = \det\nolimits_{\zeta}(I + \lambda L_{\mathrm{sym}}).
\]

\paragraph{Spectral Interpretation.}
For trace-class self-adjoint operators, the zeros of \( \lambda \mapsto \det\nolimits_{\zeta}(I - \lambda L) \) coincide with the reciprocals of the nonzero eigenvalues of \( L \), counted with multiplicity:
\[
\lambda = \frac{1}{\mu} \quad \Longleftrightarrow \quad \mu \in \operatorname{Spec}(L) \setminus \{0\}.
\]
Thus, if \( \mu \in \operatorname{Spec}(L_{\mathrm{sym}}) \), then so is \( -\mu \), since the set of determinant zeros is invariant under \( \lambda \mapsto -\lambda \).

\paragraph{Conclusion.}
Therefore, the spectrum of \( L_{\mathrm{sym}} \) is symmetric under \( \mu \mapsto -\mu \), with multiplicities preserved.
\end{proof}
