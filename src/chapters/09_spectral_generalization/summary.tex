\subsection*{Summary}

In this chapter we extended the canonical spectral framework from the Riemann zeta function to the class of completed automorphic \( L \)-functions \( \Xi(s, \pi) \), where \( \pi \in \mathcal{A}_{\mathrm{cusp}}(\mathrm{GL}_n) \). Assuming standard analytic continuation and boundedness properties for \( \Xi(s, \pi) \), along with exponential decay of its inverse Fourier kernel, we constructed:

\begin{itemize}
  \item A family of mollified convolution operators \( \{ \Ltpi \}_{t > 0} \in \TC(\Hpsipi) \), shown to converge in trace norm;
  \item A canonical limiting operator \( \Lsympi \in \TC(\Hpsipi) \) satisfying a determinant identity:
  \[
  \detz(I - \lambda \Lsympi) = \frac{\Xi\left(\tfrac{1}{2} + i\lambda, \pi\right)}{\Xi\left(\tfrac{1}{2}, \pi\right)};
  \]
  \item A spectral correspondence identifying the eigenvalues of \( \Lsympi \) with scaled nontrivial zeros of \( \Xi(s, \pi) \);
  \item An equivalence: \( \GRH(\pi) \Longleftrightarrow \Spec(\Lsympi) \subset \R \).
\end{itemize}

These results fully validate the generalization of the spectral determinant framework to automorphic representations of \( \mathrm{GL}_n \), under analytic hypotheses compatible with the Langlands program. The structure remains modular and provably uniform in its operator-theoretic formulation, providing a robust foundation for future extensions to motivic and arithmetic \( L \)-functions beyond the automorphic spectrum.

The next chapter returns to the Riemann case and completes the logical closure, establishing the full equivalence between spectrum reality and the Riemann Hypothesis within the validated zeta context.
