\begin{proof}[Proof of \lemref{lem:det_identity_entire_order_one}]
Let \( L := L_{\mathrm{sym}} \in \mathcal{C}_1(H_{\Psi_\alpha}) \) be compact and self-adjoint. Then the zeta-regularized determinant admits the trace-logarithmic expansion:
\[
\det\nolimits_\zeta(I - \lambda L)
= \exp\left( - \sum_{n=1}^\infty \frac{\lambda^n}{n} \operatorname{Tr}(L^n) \right),
\]
which converges absolutely for all \( \lambda \in \mathbb{C} \), since
\[
|\operatorname{Tr}(L^n)| \le \|L^n\|_{\mathcal{C}_1} \le \|L\|^n.
\]
Hence, \( \det\nolimits_\zeta(I - \lambda L) \) defines an entire function on \( \mathbb{C} \).

\paragraph{Growth bound.}
By \lemref{lem:det_growth_bound}, the determinant satisfies the exponential growth estimate:
\[
\log \left| \det\nolimits_\zeta(I - \lambda L) \right|
\le C |\lambda| \log(1 + |\lambda|),
\]
for some constant \( C > 0 \), implying the function is entire of order one.

\paragraph{Exponential type via Fourier decay.}
The operator \( L \) is a convolution operator with kernel \( K(x - y) \), whose Fourier transform is the centered profile
\[
\widehat{K}(\lambda) = \Xi\left( \tfrac{1}{2} + i\lambda \right).
\]
Since \( \Xi(s) \) is entire of exponential type \( \pi \), the Paley--Wiener theorem~\cite[Ch.~9]{Levin1996EntireLectures} implies that \( K \in L^2(\mathbb{R}) \) has frequency support in \( [-\pi, \pi] \). Consequently, \( K(x) \) decays exponentially as \( |x| \to \infty \) with rate at least \( \pi - \varepsilon \). This decay transfers to the entire function structure of the determinant via trace-kernel correspondence~\cite[Ch.~3--4]{Simon2005TraceIdeals}.

\paragraph{Conclusion.}
The determinant \( \det\nolimits_\zeta(I - \lambda L) \) is entire of order one and exponential type \( \pi \), that is,
\[
\det\nolimits_\zeta(I - \lambda L) \in \mathcal{E}_1^\pi,
\]
as claimed.
\end{proof}
