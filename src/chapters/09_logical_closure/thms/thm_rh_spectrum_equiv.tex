\begin{theorem}[Spectral Equivalence with the Riemann Hypothesis]
\label{thm:rh-spectrum-equiv}
Let \( L_{\mathrm{sym}} \in \mathcal{C}_1(H_{\Psi_\alpha}) \) denote the canonical compact, self-adjoint, trace-class operator on the exponentially weighted Hilbert space
\[
H_{\Psi_\alpha} := L^2(\mathbb{R}, e^{\alpha |x|} dx),
\]
for fixed weight parameter \( \alpha > \pi \). Let
\[
\mu_\rho := \frac{1}{i(\rho - \tfrac{1}{2})}
\]
denote the spectral image of a nontrivial zero \( \rho \in \mathbb{C} \) of the Riemann zeta function.

Then the following are equivalent:
\begin{enumerate}
  \item[\textup{(i)}] The Riemann Hypothesis holds: every nontrivial zero \( \rho \in \mathbb{C} \) of \( \zeta(s) \) satisfies
  \[
  \Re(\rho) = \tfrac{1}{2}.
  \]

  \item[\textup{(ii)}] The spectrum of the canonical operator \( L_{\mathrm{sym}} \) lies entirely on the real line:
  \[
  \operatorname{Spec}(L_{\mathrm{sym}}) \subset \mathbb{R}.
  \]
\end{enumerate}

\medskip
\noindent
This equivalence follows from the canonical spectral mapping
\[
\rho \mapsto \mu_\rho := \frac{1}{i(\rho - \tfrac{1}{2})},
\]
which defines a multiplicity-preserving bijection between the nontrivial zeros of \( \zeta(s) \) and the nonzero spectrum of \( L_{\mathrm{sym}} \), as established in Theorem~\ref{thm:spectral-zero-bijection-revised}.

The determinant identity
\[
\det\nolimits_\zeta(I - \lambda L_{\mathrm{sym}}) = \frac{\Xi\left( \tfrac{1}{2} + i\lambda \right)}{\Xi\left( \tfrac{1}{2} \right)}
\]
encodes the zero set of \( \zeta(s) \) in the spectral data of \( L_{\mathrm{sym}} \). Thus, reality of the spectrum is logically equivalent to the critical-line condition \( \Re(\rho) = \tfrac{1}{2} \) for all nontrivial zeros \( \rho \).
\end{theorem}
