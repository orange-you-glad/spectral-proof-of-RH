\begin{lemma}[Trace-Class Criterion for Conjugated Kernels]
\label{lem:trace_class_conjugated_kernel}
Let \( K(x, y) \in C^\infty(\R^2) \) be a real-valued, symmetric kernel: \( K(x,y) = K(y,x) \) for all \( x,y \in \R \).

Fix \( \alpha > 0 \), and define the weighted Hilbert space
\[
\HPsi := L^2(\R, \Psi_\alpha(x)\, dx), \qquad \text{where } \Psi_\alpha(x) := e^{\alpha |x|}.
\]

Suppose the exponentially conjugated kernel satisfies the integrability condition:
\[
\iint_{\R^2} |K(x, y)| \, \Psi_\alpha(x) \Psi_\alpha(y) \, dx\, dy < \infty.
\]

Then the integral operator
\[
(T f)(x) := \int_{\R} K(x, y)\, f(y)\, dy
\]
extends to a bounded trace-class operator on \( \HPsi \):
\[
T \in \mathcal{C}_1(\HPsi),
\]
with trace norm estimate:
\[
\| T \|_{\mathcal{C}_1(\HPsi)}
\le \iint_{\R^2} |K(x, y)|\, \Psi_\alpha(x)\Psi_\alpha(y)\, dx\, dy.
\]

\paragraph{Remarks.}
\begin{itemize}
    \item The symmetry and real-valuedness of \( K \) imply that \( T \) is formally self-adjoint on the Schwartz core \( \Schwartz \subset \HPsi \). If \( K \) is sufficiently regular, this lifts to self-adjointness on the closure.

    \item This result follows from \lemref{lem:trace_class_via_weighted_L1} via unitary conjugation to flat \( L^2(\R) \), where the kernel
    \[
    \widetilde{K}(x,y) := \frac{K(x,y)}{\sqrt{\Psi_\alpha(x)\Psi_\alpha(y)}}
    \]
    belongs to \( L^1(\R^2) \), activating Simon's trace-class criterion \cite[Thm.~4.2]{Simon2005TraceIdeals}.
\end{itemize}
\end{lemma}
