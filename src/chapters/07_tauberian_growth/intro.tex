\subsection*{Introduction}
\label{sec:intro_tauberian_growth}

This chapter establishes the spectral asymptotics of the canonical compact, self-adjoint, trace-class operator
\[
\Lsym \in \TC(\HPsi),
\]
via Laplace–Tauberian inversion of its squared heat semigroup. Using refined short-time asymptotics of the spectral trace
\[
\Theta(t) := \Tr(e^{-t \Lsym^2}),
\]
we derive sharp growth estimates for the eigenvalue counting function
\[
N(\Lambda) := \#\left\{ n \in \N : \mu_n^2 \le \Lambda \right\},
\]
and confirm analytic compatibility with the Riemann–von Mangoldt zero-counting law.

\paragraph{Objectives.}
\begin{itemize}
  \item \textit{Spectral Growth Bounds:}  
  From the asymptotic envelope
  \[
  \Theta(t) \asymp t^{-1/2} \quad \text{as } t \to 0^+,
  \]
  established in Chapter~\ref{sec:heat_kernel_asymptotics}, we deduce the subconvex bound
  \[
  N(\Lambda) = O(\Lambda^{1/2+\varepsilon}) \quad \text{for all } \varepsilon > 0,
  \]
  via \lemref{lem:spectral_convexity_estimate}. This confirms effective spectral dimension \( d = 1 \), with asymptotic shape \( N(\Lambda) \sim \Lambda^{1/2} \log \Lambda \).

  \item \textit{Tauberian Inversion:}  
  Applying Korevaar’s log-refined version of Karamata theory (\lemref{lem:log_corrected_tauberian_estimate}), we invert the Laplace relation
  \[
  \Theta(t) = \int_0^\infty e^{-t\lambda} \, dN(\lambda),
  \]
  and recover the sharp asymptotic:
  \[
  N(\Lambda) = \frac{\sqrt{\Lambda}}{\pi} \log \Lambda + O(\sqrt{\Lambda}),
  \quad \text{as } \Lambda \to \infty.
  \]
  This follows from classifying \( \Theta \in \mathcal{R}_{1/2}^{\log}(0^+) \): the class of log-modulated regularly varying functions.

  \item \textit{Zeta-Theoretic Compatibility:}  
  The derived asymptotic agrees with the classical Riemann–von Mangoldt formula for the zero-counting function. This match, proven in \corref{cor:zeta_compatibility}, confirms that the spectrum of \( \Lsym \) spectrally encodes the distribution of \( \zeta(s) \) zeros.
\end{itemize}

\paragraph{Analytic Inputs from Chapter~\ref{sec:heat_kernel_asymptotics}.}

\begin{center}
\renewcommand{\arraystretch}{1.3}
\begin{tabularx}{\textwidth}{|c|X|X|}
\hline
\textbf{Source} & \textbf{Analytic Quantity} & \textbf{Role in This Chapter} \\
\hline
\propref{prop:two_sided_heat_trace_bounds} &
\( \Tr(e^{-t \Lsym^2}) \asymp t^{-1/2} \) &
Ensures admissibility for Tauberian envelope bounds \\

\propref{prop:short_time_heat_expansion} &
\( \Tr(e^{-t \Lsym^2}) = \frac{1}{\sqrt{4\pi t}} \log(1/t) + \cdots \) &
Triggers Korevaar inversion with logarithmic precision \\

\lemref{lem:log_derivative_determinant} &
Spectral zeta representation via heat trace &
Links eigenvalue asymptotics to determinant growth \\

\hline
\end{tabularx}
\end{center}

\paragraph{Link to Spectral Equivalence.}
The Tauberian analysis developed here provides the final analytic link needed to validate the spectral criterion for the Riemann Hypothesis. By matching the counting function \( N(\Lambda) \) to the Riemann–von Mangoldt law, we confirm that:
\[
  \RH \iff \Spec(\Lsym) \subset \R,
\]
as formalized in \thmref{thm:rh_spectral_closure}. This closes the analytic loop from determinant identity to spectral encoding to RH equivalence.
