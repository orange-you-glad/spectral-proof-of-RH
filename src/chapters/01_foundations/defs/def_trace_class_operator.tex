\begin{definition}[Trace-Class Operators]\label{def:trace_class_operator}
Let \( H \) be a separable complex Hilbert space, and let \( \mathcal{K}(H) \subset \mathcal{B}(H) \) denote the ideal of compact operators.

A compact operator \( T \in \mathcal{K}(H) \) is said to be of \emph{trace class} if its trace norm
\[
\| T \|_{\mathcal{C}_1} := \sum_{n=1}^\infty \sigma_n(T)
\]
is finite, where \( \{ \sigma_n(T) \} \) are the singular values of \( T \), i.e., the eigenvalues of the positive operator \( |T| := \sqrt{T^* T} \), arranged in non-increasing order:
\[
\sigma_1(T) \ge \sigma_2(T) \ge \cdots \ge 0, \qquad \lim_{n \to \infty} \sigma_n(T) = 0.
\]

The space \( \mathcal{C}_1(H) \) of trace-class operators satisfies the following properties:
\begin{enumerate}
    \item[\textup{(i)}] \( \mathcal{C}_1(H) \) is a Banach space under the norm \( \| \cdot \|_{\mathcal{C}_1} \), and a norm-closed, two-sided *-ideal in \( \mathcal{B}(H) \), obeying the inclusions
    \[
    \mathcal{F}(H) \subset \mathcal{C}_1(H) \subsetneq \mathcal{K}(H),
    \]
    where \( \mathcal{F}(H) \) denotes the space of finite-rank operators.

    \item[\textup{(ii)}] \( \mathcal{C}_1(H) \) is stable under bounded multiplication: for all \( A \in \mathcal{B}(H) \) and \( T \in \mathcal{C}_1(H) \),
    \[
    \| A T \|_{\mathcal{C}_1} \le \|A\| \cdot \|T\|_{\mathcal{C}_1}, \qquad
    \| T A \|_{\mathcal{C}_1} \le \|A\| \cdot \|T\|_{\mathcal{C}_1}.
    \]

    \item[\textup{(iii)}] The trace map
    \[
    \operatorname{Tr}(T) := \sum_{n=1}^\infty \langle T e_n, e_n \rangle
    \]
    is absolutely convergent, independent of the choice of orthonormal basis \( \{ e_n \} \subset H \), and satisfies the cyclicity identity:
    \[
    \operatorname{Tr}(AB) = \operatorname{Tr}(BA), \qquad \forall A \in \mathcal{B}(H),\; B \in \mathcal{C}_1(H).
    \]
\end{enumerate}

\medskip
\noindent\textbf{Remarks.}
\begin{itemize}
    \item \( \mathcal{C}_1(H) = \mathcal{S}_1(H) \) is the first Schatten ideal: the set of compact operators whose singular values lie in \( \ell^1 \). It generalizes the class of nuclear operators in Hilbert space theory.

    \item For integral operators \( T \) with kernel \( K(x, y) \in L^1(\R^2) \), one has \( T \in \mathcal{C}_1(L^2) \), with
    \[
    \| T \|_{\mathcal{C}_1} \le \| K \|_{L^1(\R^2)} \quad \text{\cite[Thm.~4.2]{Simon2005TraceIdeals}}.
    \]

    \item In this manuscript, the mollified convolution operators \( L_t \), and their trace-norm limit \( L_{\sym} \), are shown to lie in \( \mathcal{C}_1(\HPsi) \) for all \( \alpha > \pi \), where \( \HPsi := L^2(\R, e^{\alpha |x|} dx) \). This guarantees that the Fredholm determinant
    \[
    \det\nolimits_{\zeta}(I - \lambda L_{\sym})
    \]
    is well-defined, entire of order one, and admits a spectral representation compatible with the Hadamard factorization of \( \Xi(s) \) (see \secref{sec:determinant_identity}).

    \item The classical Fredholm determinant
    \[
    \det(I + zT) := \prod_{n=1}^\infty (1 + z \lambda_n),
    \]
    where \( \{ \lambda_n \} \) are the eigenvalues of \( T \in \mathcal{C}_1(H) \), converges absolutely and defines an entire function of exponential type (see \appref{app:trace_ideals_review}).
\end{itemize}

\medskip
\noindent\textbf{References.}
\begin{itemize}
    \item B.~Simon, \emph{Trace Ideals and Their Applications}, Chapter~3 \cite{Simon2005TraceIdeals}.
    \item M.~Reed and B.~Simon, \emph{Methods of Modern Mathematical Physics I: Functional Analysis}, Chapters~VI--VII \cite{ReedSimon1980I}.
\end{itemize}
\end{definition}
