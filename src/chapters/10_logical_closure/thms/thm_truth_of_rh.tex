\begin{theorem}[Equivalence of the Riemann Hypothesis with Spectral Reality]
\label{thm:truth_of_rh}

The Riemann Hypothesis is equivalent to the spectral reality of the canonical convolution operator
\[
\Lsym \in \TC(\HPsi),
\]
constructed in \secref{sec:operator_construction}. Explicitly,
\[
\RH \iff \Spec(\Lsym) \subset \R.
\]

\medskip

\noindent
This equivalence follows from the analytic realization of the completed zeta function via the canonical determinant identity:
\[
\detz(I - \lambda \Lsym) = \frac{\Xi\left( \tfrac{1}{2} + i\lambda \right)}{\Xi\left( \tfrac{1}{2} \right)},
\]
as shown in \thmref{thm:det_identity_revised}, and the bijective correspondence
\[
\rho \mapsto \mu_\rho := \frac{1}{i}(\rho - \tfrac{1}{2}),
\]
between nontrivial zeros \( \rho \in \C \) of \( \zeta(s) \) and the nonzero spectrum of \( \Lsym \), with multiplicities preserved as established in \thmref{thm:spectral_zero_bijection_revised}.

\medskip

\noindent
All analytic inputs—Paley–Wiener decay, trace-norm convergence, self-adjointness, determinant regularization, and positivity—have been rigorously verified in Chapters~\ref{sec:foundations}–\ref{sec:tauberian_growth}, culminating in the equivalence statement \thmref{thm:eq_of_rh}, with logical flow validated in Appendix~\ref{app:dependency_graph}.

\medskip

\noindent
In particular, the analytic infrastructure has been ensured by:
\begin{itemize}
  \item \lemref{lem:trace_class_Lt} — trace-class inclusion of approximating mollifiers;
  \item \lemref{lem:heat_trace_expansion} — singular expansion and spectral asymptotics;
  \item \lemref{lem:spectral_encoding_injection} — zero-to-spectrum map;
  \item \lemref{lem:trace_distribution_positivity} — positivity of the trace pairing;
  \item \lemref{lem:spectral_symmetry} — evenness and reality correspondence;
\end{itemize}
which together support the spectral determinant framework.

\medskip

\noindent
Therefore, the Riemann Hypothesis holds if and only if the spectrum of \( \Lsym \) is real.
\end{theorem}
