\subsection*{Summary}

This chapter reformulates the Riemann Hypothesis as a statement of spectral rigidity: the spectrum and zeta-regularized determinant of the canonical trace-class operator \( \Lsym \in \TC(\HPsi) \) encode both the location and multiplicity of all nontrivial zeros of the Riemann zeta function.

\textbf{Key Spectral Rigidity Results}
\begin{itemize}
  \item \lemref{lem:spectral_encoding_injection} — \textit{Spectral encoding injection:}
  \[
  \rho \mapsto \mu_\rho := \tfrac{1}{i}(\rho - \tfrac{1}{2})
  \]
  maps each nontrivial zeta zero \( \rho \) to a unique eigenvalue \( \mu_\rho \in \Spec(\Lsym) \setminus \{0\} \), preserving multiplicities.

  \item \lemref{lem:det_zero_implies_spectrum} — \textit{Determinant–spectrum correspondence:}
  \[
  \detz(I - \lambda \Lsym) = 0 \quad \Longleftrightarrow \quad \lambda^{-1} \in \Spec(\Lsym).
  \]

  \item \lemref{lem:inject_zero_real_spectrum} — \textit{Spectral reality implies critical-line zeros:}
  \[
  \Spec(\Lsym) \subset \R \quad \Rightarrow \quad \Re(\rho) = \tfrac{1}{2}.
  \]

  \item \lemref{lem:real_spectrum_implies_rh_rigidity} — This implication holds without assuming spectral bijection or simplicity; it depends only on the determinant identity and compact self-adjoint structure.

  \item \propref{prop:spectrum_reality_implies_rh_multiplicity} — \textit{Spectral simplicity implies simple zeros:}
  \[
  \Re(\rho) = \tfrac{1}{2}, \quad \operatorname{ord}_\rho(\zeta) = 1.
  \]

  \item \thmref{thm:rh_from_real_simple_spectrum} — \textit{Full spectral rigidity:} RH and simplicity follow if \( \Lsym \) is real and simple.

  \item \lemref{lem:spectral_symmetry} — \textit{Spectral symmetry from functional equation:}
  \[
  \mu \in \Spec(\Lsym) \quad \Rightarrow \quad -\mu \in \Spec(\Lsym),
  \]
  with matched multiplicities.

  \item \lemref{lem:trace_distribution_positivity} — The trace map
  \[
  \phi \mapsto \Tr(\phi(\Lsym))
  \]
  defines a positive tempered distribution on \( \R \), interpreted as a regularized spectral measure.

  \item \lemref{lem:trace_distribution_positive} — \textit{Positivity extends:} For all nonnegative \( \phi \in \Schwartz(\R) \), trace positivity holds; see \remref{rem:functional_calculus_trace}.

  \item \propref{prop:inverse_spectral_rigidity} — \textit{Uniqueness:} Trace-class self-adjoint operators with identical determinant and spectrum are unitarily equivalent.
\end{itemize}

\medskip

\textbf{Conclusion and Equivalence}
Together, these results establish the logical implication
\[
\Spec(\Lsym) \subset \R \quad \Longrightarrow \quad \RH,
\]
thereby closing the analytic loop introduced in \thmref{thm:eq_of_rh}. Coupled with the canonical determinant identity, this rigidity completes the spectral reformulation of the Riemann Hypothesis.

\medskip

\textbf{Remarks}
\begin{itemize}
  \item Spectral simplicity strengthens RH to the assertion that all nontrivial zeros are simple, but simplicity is not required for RH itself.
  \item The correspondence between zeta zeros and spectral eigenvalues arises from Hadamard factorization and Fredholm theory.
  \item Trace positivity reveals the harmonic–analytic regularity of the spectral measure associated with \( \Lsym \).
  \item Under Fourier diagonalization, the trace map becomes integration against the spectral measure induced by the eigenbasis of \( \Lsym \).
\end{itemize}

\medskip

\textbf{Forward Link}
All results culminate in the final equivalence:
\[
\RH \quad \Longleftrightarrow \quad \Spec(\Lsym) \subset \R,
\]
formally proven in \thmref{thm:truth_of_rh} and logically structured in the DAG architecture of \appref{app:dependency_graph}.
