\begin{theorem}[Truth of the Riemann Hypothesis]
\label{thm:truth_of_rh}
Every nontrivial zero \( \rho \in \mathbb{C} \) of the Riemann zeta function \( \zeta(s) \) satisfies
\[
\Re(\rho) = \tfrac{1}{2}.
\]

\medskip
\noindent
That is, the nontrivial zero set of \( \zeta(s) \) lies entirely on the critical line:
\[
\operatorname{Zeros}(\zeta(s)) \subset \left\{ s \in \mathbb{C} : \Re(s) = \tfrac{1}{2} \right\}.
\]

\medskip
\noindent
This completes the analytic and spectral proof of the Riemann Hypothesis, via the canonical determinant identity and the self-adjointness of the operator \( L_{\mathrm{sym}} \), whose spectrum encodes the zero set of \( \zeta(s) \) under the transformation
\[
\mu_\rho := \frac{1}{i(\rho - \tfrac{1}{2})}.
\]
\end{theorem}
