\section{Refinements of Heat Kernel Asymptotics}
\label{app:heat_kernel_refinements}

\noindent\textbf{[LOGICALLY INERT Appendix — Noncritical]}  
This appendix is analytically and logically independent from Chapters~1–9. It explores conjectural refinements of the short-time asymptotic expansion of the heat trace
\[
\Theta(t) := \Tr(e^{-t \Lsym^2}),
\]
under assumptions stronger than those required for the determinant identity or the RH equivalence.

\subsection*{Refined Expansion Structure}

If the eigenvalues \( \{\mu_n\} \subset \Spec(\Lsym) \) exhibit additional arithmetic structure—such as regular spacing, symmetry under a motivic duality, or controlled multiplicity decay—then \( \Theta(t) \) may admit a sharper expansion of the form:
\[
\Theta(t) = \frac{1}{\sqrt{4\pi t}} \log\left( \frac{1}{t} \right)
+ \frac{c_0}{\sqrt{t}} + o\left( \frac{1}{\sqrt{t}} \right),
\quad t \to 0^+,
\]
for some constant \( c_0 \in \R \) encoding subleading spectral contributions.

Such refinements mirror expansions for Laplace-type operators on singular or stratified spaces and are structurally comparable to known asymptotics in~\cite{Seeley1967ComplexPowers, Gilkey1995Invariance, Vaillant2001HeatKernel}.

\subsection*{Implications for Tauberian Asymptotics}

If such refined trace expansions hold, then the eigenvalue counting function
\[
A(\Lambda) := \#\left\{ \mu_n^2 \le \Lambda \right\}
\]
admits a subleading Weyl-like asymptotic:
\[
A(\Lambda) = \frac{1}{2\pi} \Lambda^{1/2} \log \Lambda
+ C_1 \Lambda^{1/2} + o(\Lambda^{1/2}),
\]
where \( C_1 \in \R \) reflects spectral torsion, degeneracy, or motivic correction terms.

These refinements are consistent with extended Tauberian theorems and appear in higher-rank trace formulas and heat asymptotics~\cite{Korevaar2004Tauberian}.

\subsection*{Analytic Outlook}

Identification of \( c_0 \) and \( C_1 \) via:
\begin{itemize}
  \item Mellin transforms of \( \Theta(t) \),
  \item Residues and continuation of the spectral zeta function \( \zeta_{\Lsym^2}(s) = \Tr(\Lsym^{-2s}) \),
  \item Or analytic structure of auxiliary Dirichlet series,
\end{itemize}
could refine the analytic profile of the determinant and inform higher-order spectral statistics for \( \XiR(s) \).

\medskip

These terms play no role in the core determinant identity or RH equivalence proven in \cref{sec:spectral_implications}, but they may prove useful for:

\begin{itemize}
  \item Computing analytic torsion and spectral entropy;
  \item Extending trace identities to modular and functorial contexts;
  \item Testing compatibility with automorphic lifts (\appref{app:functorial_extensions});
  \item Quantifying subleading corrections in zeta-regularized determinants~\cite{Elizalde1994ZetaRegularization}.
\end{itemize}

\subsection*{Conclusion}

These conjectural refinements are analytically consistent with the canonical spectral model but remain speculative. They invite further investigation into the finer spectral geometry of \( L_{\mathrm{sym}} \), beyond what is required for the determinant identity or the equivalence with RH.
