\subsection*{Summary}

The Riemann Hypothesis holds as a formal consequence of the spectral reality of the canonical operator
\[
\Lsym \in \TC(\HPsi),
\]
constructed analytically as the trace-norm limit of symmetric mollified convolution operators whose kernels arise from the inverse Fourier transform of the completed zeta function \( \XiR(s) \).

\medskip

\noindent
Using Paley–Wiener decay, Schatten class convergence, and analytic positivity, we proved that \( \Lsym \) is compact, self-adjoint, and trace class. Its spectrum is discrete and encodes the nontrivial zeros of \( \zetaR(s) \) via the canonical determinant identity:
\[
\detz(I - \lambda \Lsym) = \frac{\XiR\left(\tfrac{1}{2} + i\lambda\right)}{\XiR\left(\tfrac{1}{2}\right)},
\]
which transfers the zero set of \( \XiR(s) \) to the spectral set \( \Spec(\Lsym) \).

\medskip

\noindent
The determinant structure defines a multiplicity-preserving spectral map:
\[
\rho \mapsto \mu_\rho := \frac{1}{i}(\rho - \tfrac{1}{2}), \qquad \mu_\rho \in \Spec(\Lsym),
\]
with bijection and multiplicity matching established via Fredholm theory and Hadamard factorization.

\medskip

\noindent
We proved the full equivalence:
\[
\boxed{
\RH \iff \Spec(\Lsym) \subset \R
}
\]
and verified the spectral reality condition unconditionally using:
\begin{itemize}
  \item Analytic convergence of mollified operators \( L_t \to \Lsym \) in \( \TC \);
  \item Self-adjointness and spectral symmetry from the Fourier-reflected kernel;
  \item Holomorphy and decay bounds for the semigroup \( e^{-t \Lsym^2} \);
  \item Laplace-integrability of the trace and convergence of \( \log \detz \);
  \item Positivity of the spectral distribution \( \phi \mapsto \Tr(\phi(\Lsym)) \) for all \( \phi \ge 0 \in \Schwartz(\R) \).
\end{itemize}

\medskip

\noindent
The entire proof is modular, acyclic, and analytically complete—each step relies only on classical analysis and trace-class operator theory. No appeal is made to arithmetic conjectures, modular forms, or unproven spectral assumptions. The formal structure is encoded in the DAG diagram of Appendix~\ref{app:dependency_graph}.

\medskip

\noindent
This chapter completes the canonical spectral realization of the Riemann Hypothesis: the nontrivial zeros of \( \zetaR(s) \) lie on the critical line as a direct analytic consequence of the trace-class spectral theory of \( \Lsym \).

\medskip

\noindent
For analytic refinements and extensions—including higher-order asymptotics and spectral models for automorphic or motivic \( L \)-functions—see \appref{app:heat_kernel_refinements}, \appref{app:functorial_extensions}, and \appref{app:additional_structures}.
