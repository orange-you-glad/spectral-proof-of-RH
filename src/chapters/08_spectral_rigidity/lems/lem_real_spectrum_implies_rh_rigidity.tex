\begin{lemma}[Spectral Reality Implies the Riemann Hypothesis]
\label{lem:real_spectrum_implies_rh_rigidity}

Let \( L_{\mathrm{sym}} \in \mathcal{C}_1(H_{\Psi_\alpha}) \) be the canonical compact, self-adjoint operator constructed via mollified convolution from the inverse Fourier transform of the completed Riemann zeta function \( \Xi(s) \), acting on the weighted Hilbert space:
\[
H_{\Psi_\alpha} := L^2(\mathbb{R}, e^{\alpha |x|} dx), \qquad \alpha > \pi.
\]

Assume:
\begin{enumerate}
  \item[\textup{(i)}] The Carleman \(\zeta\)-regularized Fredholm determinant satisfies:
  \[
  \det\nolimits_\zeta(I - \lambda L_{\mathrm{sym}}) = \frac{\Xi\left(\tfrac{1}{2} + i\lambda\right)}{\Xi\left(\tfrac{1}{2}\right)}, \qquad \forall \lambda \in \mathbb{C}.
  \]

  \item[\textup{(ii)}] The spectrum of \( L_{\mathrm{sym}} \) lies entirely on the real axis:
  \[
  \operatorname{Spec}(L_{\mathrm{sym}}) \subset \mathbb{R}.
  \]
\end{enumerate}

Then every nontrivial zero \( \rho \in \mathbb{C} \) of the Riemann zeta function satisfies:
\[
\operatorname{Re}(\rho) = \tfrac{1}{2}.
\]
That is, the Riemann Hypothesis holds.
\end{lemma}
