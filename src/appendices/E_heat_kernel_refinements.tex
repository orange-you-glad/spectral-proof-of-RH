\section{Refinements of Heat Kernel Asymptotics}
\label{app:heat-kernel-refinements}

\noindent\textbf{[LOGICALLY INERT Appendix — Noncritical]}  
This appendix is not referenced by any result in the main manuscript. It explores potential refinements to the short-time heat trace asymptotics of the canonical operator
\[
\Theta(t) := \operatorname{Tr}(e^{-t L_{\mathrm{sym}}^2}),
\]
under stronger spectral regularity assumptions beyond those required to prove the Riemann Hypothesis.

\subsection*{Refined Expansion Structure}

If the spectrum \( \{\mu_n\} \) of \( L_{\mathrm{sym}} \) exhibits additional arithmetic or analytic structure—such as uniform gaps, arithmetic progression, or controlled multiplicity decay—then the heat trace may admit a sharper asymptotic expansion:
\[
\Theta(t) = \frac{1}{\sqrt{4\pi t}} \log\left( \frac{1}{t} \right)
+ \frac{c_0}{\sqrt{t}} + o\left( \frac{1}{\sqrt{t}} \right)
\quad \text{as } t \to 0^+,
\]
for some constant \( c_0 \in \mathbb{R} \) encoding subleading spectral structure.

Such expansions appear in Laplace-type operators on singular spaces, logarithmic spectral densities, and scattering manifolds. See~\cite{Seeley1967ComplexPowers, Gilkey1995Invariance, Vaillant2001HeatKernel} for foundational cases.

\subsection*{Implications for Tauberian Asymptotics}

If this refined expansion holds, then the eigenvalue counting function
\[
A(\Lambda) := \#\left\{ \mu_n^2 \le \Lambda \right\}
\]
would satisfy a subleading Weyl-type asymptotic:
\[
A(\Lambda) = \frac{1}{2\pi} \Lambda^{1/2} \log \Lambda + C_1 \Lambda^{1/2} + o(\Lambda^{1/2}),
\]
where \( C_1 \in \mathbb{R} \) may encode additional structure such as motivic degeneracies or Hecke symmetry.

These terms arise in refined Selberg trace formulas and logarithmic Tauberian expansions~\cite{Korevaar2004Tauberian}.

\subsection*{Analytic Outlook}

Determining \( c_0 \) or \( C_1 \) explicitly—e.g., via Mellin transforms of \( \Theta(t) \), residues of the spectral zeta function \( \zeta_{L^2}(s) := \operatorname{Tr}(L_{\mathrm{sym}}^{-2s}) \), or Dirichlet series interpolants—could refine the analytic structure of \( \Xi(s) \) and inform deeper arithmetic extensions.

While not needed for the determinant identity or the proof of RH, such refinements may guide future developments involving:
\begin{itemize}
  \item Spectral torsion and analytic torsion invariants;
  \item Modular trace formula lifts and automorphic extensions;
  \item Functorial compatibility with motivic and Langlands \( L \)-functions (cf.~Appendix~\ref{app:functorial-extensions});
  \item Subleading logarithmic correction theory in determinant frameworks~\cite{Elizalde1994ZetaRegularization}.
\end{itemize}

\medskip
\noindent
These refinements remain conjectural but are analytically compatible with the canonical spectral realization of \( \Xi(s) \) developed in the main manuscript.
