\begin{definition}[Spectral Zeta Function]
\label{def:spectral_zeta_function}

Let \( H \) be a separable Hilbert space, and let
\[
T \in \TC(H)
\]
be a compact, self-adjoint, positive semi-definite operator.

Let \( \{ \lambda_n \}_{n=1}^\infty \subset (0,\infty) \) denote the nonzero eigenvalues of \( T \), listed with algebraic multiplicity and ordered so that \( \lambda_n \to 0 \) as \( n \to \infty \).

\medskip

The \emph{spectral zeta function} associated to \( T \) is defined by the Dirichlet series:
\[
\zeta_T(s) := \sum_{n=1}^\infty \lambda_n^{-s},
\]
which converges absolutely for \( \Re(s) > s_0 \), for some \( s_0 > 0 \) depending on the eigenvalue decay.

\medskip

Under suitable spectral asymptotics—e.g., Weyl-type or logarithmic decay—\( \zeta_T(s) \) admits meromorphic continuation to a larger domain, often to all of \( \C \). In particular, for operators such as \( T = L_{\sym}^2 \), the small-time asymptotics of the heat trace,
\[
\Tr(e^{-tT}) \sim \frac{1}{\sqrt{4\pi t}} \log\left( \frac{1}{t} \right)
\quad \text{as } t \to 0^+,
\]
imply meromorphic continuation of \( \zeta_T(s) \) via the Mellin transform of the heat kernel.

\medskip

Spectral zeta functions play a central role in analytic spectral theory, especially in the definition of zeta-regularized determinants. The \emph{shifted spectral zeta function}, defined by
\[
\zeta_T(s, \lambda) := \sum_n (\lambda_n - \lambda)^{-s},
\]
admits analytic continuation in \( s \) for fixed \( \lambda \notin \{ \lambda_n \} \), and gives rise to the determinant via
\[
\log \det\nolimits_\zeta(I - \lambda T^{1/2}) := -\left.\frac{d}{ds} \zeta_T(s, \lambda)\right|_{s = 0}.
\]
\end{definition}
