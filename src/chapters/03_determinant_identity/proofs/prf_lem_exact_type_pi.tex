\begin{proof}[Proof of Lemma~\ref{lem:exact_type_pi}]
Let \( f(\lambda) := \Xi\left( \tfrac{1}{2} + i\lambda \right) \). It is classical that \( \Xi(s) \) is an entire function of order one and exponential type \( \pi \). This follows from:

\begin{itemize}
  \item The functional equation and integral representation of \( \Xi(s) \),
  \item The Stirling expansion of \( \Gamma(s/2) \), and
  \item The Hadamard factorization for entire functions with real zeros of bounded density.
\end{itemize}

\paragraph{(i) Upper bound on type.}
From Titchmarsh~\cite[§10.5]{Titchmarsh1986Zeta}, one has for all \( \varepsilon > 0 \),
\[
|\Xi(\tfrac{1}{2} + i\lambda)| \le C e^{(\pi + \varepsilon)|\lambda|},
\]
establishing exponential type \( \le \pi \).

\paragraph{(ii) Lower bound on type.}
The lower bound follows from classical results of de Bruijn and Levin~\cite[Chap.~3]{Levin1996EntireLectures}, which state that the exponential type is equal to \( \pi \) if the distribution of the zeros has limiting density \( \frac{\pi}{\log T} \) and the function is real-entire and even.

Furthermore, the exact type is encoded in the asymptotic location of the zeros of \( \Xi(s) \), whose imaginary parts have counting function
\[
N(T) \sim \frac{T}{2\pi} \log\left(\frac{T}{2\pi e}\right) + O(\log T),
\]
implying exponential type at least \( \pi \) by Hadamard theory.

\paragraph{(iii) Determinant match.}
By Theorem~\ref{thm:det_identity_revised}, we have
\[
\det\nolimits_\zeta(I - \lambda L_{\mathrm{sym}}) = \frac{\Xi(\tfrac{1}{2} + i\lambda)}{\Xi(\tfrac{1}{2})},
\]
and therefore the determinant inherits the exact same exponential type.

\paragraph{Conclusion.}
Both \( \Xi(\tfrac{1}{2} + i\lambda) \) and the canonical determinant belong to the class \( \mathcal{E}_1^\pi \), completing the proof.
\end{proof}
%  