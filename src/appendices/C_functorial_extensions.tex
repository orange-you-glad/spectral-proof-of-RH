\section{Functorial Extensions Beyond \texorpdfstring{$\mathrm{GL}_n$}{GLn}}
\label{app:functorial_extensions}

\noindent\textbf{[Noncritical Appendix.]}  
This appendix is analytically and logically independent from Chapters~1–10. It no longer motivates constructions used in the main body, as the spectral realization of automorphic \( L \)-functions for \( \mathrm{GL}_n \) has been validated in \secref{sec:spectral_generalization}. Instead, it outlines speculative directions for generalizing the canonical determinant framework to broader classes of global \( L \)-functions arising in arithmetic geometry.

%--------------------------------------------------------------
\subsection*{Beyond \texorpdfstring{$\mathrm{GL}_n$}{GLn}: Artin and Motivic \texorpdfstring{$L$}{L}-Functions}

Let \( \Lambda_\pi(s) \) denote the completed \( L \)-function associated to a motive over \( \Q \), a Galois representation, or a representation of a reductive group \( G \) beyond \( \mathrm{GL}_n \). Assume:
\begin{itemize}
  \item[(i)] \( \Lambda_\pi(s) \) is entire of order one;
  \item[(ii)] \( \Lambda_\pi(s) = \varepsilon_\pi \Lambda_\pi(1 - s) \), with \( |\varepsilon_\pi| = 1 \);
  \item[(iii)] \( \Lambda_\pi(s) \) admits a genus-one Hadamard product.
\end{itemize}

These conditions hold in many expected cases: Artin \( L \)-functions, symmetric powers of modular forms, Hasse–Weil \( L \)-functions of curves and higher-dimensional varieties, and Langlands \( L \)-functions attached to nonstandard representations~\cite{Langlands1970EulerProducts, Deligne1971WeilI}.

\medskip
\noindent\textbf{Examples.}
\begin{center}
\renewcommand{\arraystretch}{1.3}
\begin{tabular}{|c|c|c|}
\hline
\textbf{Object \( \pi \)} & \textbf{\( \Lambda_\pi(s) \)} & \textbf{Source} \\
\hline
Modular form on \( \Gamma_0(N) \) & Hecke \( L \)-function & \cite{Cogdell2007Lectures} \\
Elliptic curve over \( \Q \) & Hasse–Weil \( L \)-function & \cite{Deligne1971WeilI} \\
Artin representation & Artin \( L \)-function & \cite{Langlands1970EulerProducts} \\
\( \mathrm{GSp}_{2n} \) Siegel cusp form & Standard \( L \)-function & speculative \\
\hline
\end{tabular}
\end{center}

To extend the spectral framework, one would define a Hilbert space \( H_{\Psi_\pi} := L^2(\R, w(x) dx) \) for a weight \( w(x) \gtrsim |\Lambda_\pi(1/2 + i x)|^2 \), and construct operators
\[
\varphi_{t,\pi}(\lambda) := e^{-t\lambda^2} \Lambda_\pi\left( \tfrac{1}{2} + i\lambda \right), \qquad K_t^{(\pi)} := \FT^{-1}[\varphi_{t,\pi}],
\]
as in \secref{sec:spectral_generalization}.

%--------------------------------------------------------------
\subsection*{Extension Hypothesis}

\textit{Hypothesis.} Suppose \( \Lambda_\pi(s) \) satisfies (i)--(iii), and that \( K_t^{(\pi)}(x) \in L^1(e^{\alpha |x|} dx) \) for some \( \alpha > 0 \). Then the operator
\[
L_{\sym,\pi} := \lim_{t \to 0^+} \int_{\R} K_t^{(\pi)}(x - y) f(y) dy
\]
defines a compact, self-adjoint trace-class operator on \( H_{\Psi_\pi} \), with determinant
\[
\detz(I - \lambda L_{\sym,\pi}) = \frac{\Lambda_\pi(1/2 + i\lambda)}{\Lambda_\pi(1/2)}.
\]

This statement is structurally parallel to the validated theorem of \secref{sec:spectral_generalization} and may be subjected to the same analytic audit in future work.

%--------------------------------------------------------------
\subsection*{Research Directions}

\begin{itemize}
  \item[(1)] \textbf{Artin \( L \)-functions}: These lack a known modular origin, so Fourier kernel decay must be studied directly. 
  \item[(2)] \textbf{Non-GL(n) groups}: For example, spectral realization of \( L \)-functions for classical groups like \( \mathrm{GSp}_{2n} \), \( \mathrm{SO}_{n} \), \( \mathrm{U}_n \), requires new trace-norm embeddings.
  \item[(3)] \textbf{Motivic weight normalization}: For cohomological \( L \)-functions, shift \( s \mapsto s + w/2 \) must be incorporated into the spectral scaling.
  \item[(4)] \textbf{Symmetric power functoriality}: Does the spectral operator for \( \mathrm{Sym}^k(\pi) \) relate algebraically to \( L_{\sym,\pi} \)?
\end{itemize}

%--------------------------------------------------------------
\subsection*{Conclusion}

With \secref{sec:spectral_generalization} establishing the spectral determinant identity for all automorphic \( L \)-functions on \( \mathrm{GL}_n \), this appendix shifts focus to speculative generalizations. These include Artin and motivic \( L \)-functions, and potential extensions of the spectral framework to groups beyond \( \mathrm{GL}_n \). Each of these paths invites a new set of analytic verifications and kernel constructions, building on the blueprint established in this manuscript.

\medskip
\noindent
These questions do not affect the closed proof of the Riemann Hypothesis presented in \secref{sec:logical_closure}, but provide a frontier for extending the spectral paradigm across arithmetic geometry.
