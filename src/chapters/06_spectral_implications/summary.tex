\subsection*{Summary}

\begin{itemize}
  \item \corref{cor:spectrum-real-equiv-rh} — \textbf{Equivalence of RH with Spectral Reality:}
  \[
  \mathrm{RH} \quad \Longleftrightarrow \quad \operatorname{Spec}(L_{\mathrm{sym}}) \subset \mathbb{R}.
  \]
  This equivalence is established via Lemma~\ref{lem:reality-of-spectrum-and-rh}, which analyzes the canonical spectral map \( \rho \mapsto \mu_\rho = \tfrac{1}{i}(\rho - \tfrac{1}{2}) \). The eigenvalue \( \mu_\rho \in \operatorname{Spec}(L_{\mathrm{sym}}) \) is real if and only if \( \Re(\rho) = \tfrac{1}{2} \). Hence, the Riemann Hypothesis is equivalent to spectral reality of \( L_{\mathrm{sym}} \).

  \item \thmref{thm:uniqueness_realization} — \textbf{Spectral Uniqueness of \( L_{\mathrm{sym}} \):}
  Any compact, self-adjoint operator \( L \in \mathcal{C}_1(H_{\Psi_\alpha}) \) satisfying
  \[
  \det\nolimits_\zeta(I - \lambda L)
  = \frac{\Xi\left(\tfrac{1}{2} + i\lambda\right)}{\Xi\left(\tfrac{1}{2}\right)} \quad \text{for all } \lambda \in \mathbb{C}
  \]
  is unitarily equivalent to \( L_{\mathrm{sym}} \). Thus, the spectral realization of \( \Xi(s) \) is unique—up to unitary equivalence—within the trace-class self-adjoint category.

  \item \lemref{lem:canonical-closure} — \textbf{Canonical Closure of the Spectral Program:}
  The completed zeta function \( \Xi(s) \), via its Hadamard structure and functional symmetry, canonically determines a unique operator \( L_{\mathrm{sym}} \in \mathcal{C}_1(H_{\Psi_\alpha}) \). No non-self-adjoint realization can satisfy the determinant identity without violating trace-class closure or spectral multiplicity.
\end{itemize}

\medskip

Together, these results complete the analytic–spectral realization phase of the modular proof program. The Riemann Hypothesis is thereby reinterpreted as a real-spectrum criterion for a uniquely defined self-adjoint operator. This sets the stage for final logical closure in Chapter~\ref{sec:logical-closure}.
