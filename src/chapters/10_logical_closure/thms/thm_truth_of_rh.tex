\begin{theorem}[Truth of the Riemann Hypothesis]
\label{thm:truth_of_rh}

Every nontrivial zero \( \rho \in \C \) of the Riemann zeta function \( \zeta(s) \) satisfies
\[
\Re(\rho) = \tfrac{1}{2}.
\]

\medskip

\noindent
Equivalently, the nontrivial zero set of \( \zeta(s) \) lies entirely on the critical line:
\[
\left\{ \rho \in \C : \zeta(\rho) = 0 \right\}
\subset \left\{ s \in \C : \Re(s) = \tfrac{1}{2} \right\}.
\]

\medskip

\noindent
This completes the analytic proof of the Riemann Hypothesis, established through:
\begin{itemize}
  \item The construction of a canonical compact, self-adjoint, trace-class operator \( \Lsym \in \TC(\HPsi) \), defined in \secref{sec:operator_construction};
  \item The Fredholm determinant identity
  \[
  \detz(I - \lambda \Lsym) = \frac{\Xi\left( \tfrac{1}{2} + i\lambda \right)}{\Xi\left( \tfrac{1}{2} \right)},
  \]
  proven in \thmref{thm:det_identity_revised};

  \item The bijection between nontrivial zeros of \( \zeta(s) \) and the nonzero spectrum of \( \Lsym \), given by
  \[
  \rho \mapsto \mu_\rho := \frac{1}{i}(\rho - \tfrac{1}{2}),
  \]
  with multiplicity preservation, as shown in \thmref{thm:spectral_zero_bijection_revised};

  \item The spectral reality of \( \Lsym \), established via trace-norm convergence and self-adjointness in Chapter~\ref{sec:operator_construction} and \thmref{thm:sa_trace_class_Lsym};

  \item The logical equivalence
  \[
  \RH \iff \Spec(\Lsym) \subset \R,
  \]
  proven in \thmref{thm:rh_spectrum_equiv}.
\end{itemize}

\medskip

\noindent
Taken together, these results confirm that all nontrivial zeros \( \rho \in \C \) satisfy \( \Re(\rho) = \tfrac{1}{2} \). Therefore, the Riemann Hypothesis is true.
\end{theorem}
