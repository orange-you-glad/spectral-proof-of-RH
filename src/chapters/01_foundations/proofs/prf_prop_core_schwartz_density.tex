\begin{proof}[Proof of \cref{prop:core_schwartz_density}]
Fix \( \alpha > \pi \), and define the exponentially weighted Hilbert space
\[
\HPsi := L^2(\R, \Psi_\alpha(x)\, dx), \qquad \Psi_\alpha(x) := e^{\alpha |x|}.
\]

Let \( L_t \in \mathcal{B}(\HPsi) \) be the mollified convolution operators given by
\[
(L_t f)(x) := \int_{\R} k_t(x - y) f(y)\, dy,
\]
where \( k_t := \FT^{-1}\left[ e^{-t\lambda^2} \, \Xi\left(\tfrac{1}{2} + i\lambda \right) \right] \in \Schwartz(\R) \). The canonical operator \( L_{\sym} := \lim_{t \to 0^+} L_t \in \TC(\HPsi) \) exists in the trace-norm topology by \cref{lem:trace_norm_convergence_Lt_to_Lsym}.

\paragraph{Step 1: Invariance of Schwartz space.}
Since \( k_t \in \Schwartz \) and convolution preserves regularity, each \( L_t \) maps Schwartz functions into Schwartz functions:
\[
L_t(\Schwartz) \subset \Schwartz \cap \HPsi, \qquad \forall t > 0.
\]

\paragraph{Step 2: Density of \( \Schwartz \subset \HPsi \).}
By \cref{lem:density_schwartz_weighted_L2}, the Schwartz space is dense in \( \HPsi \). Thus for any \( f \in \HPsi \) and \( \varepsilon > 0 \), there exists \( \phi \in \Schwartz \) such that
\[
\|f - \phi\|_{\HPsi} < \varepsilon.
\]

\paragraph{Step 3: Strong convergence of \( L_t \) on Schwartz.}
Since \( L_t \to L_{\sym} \) in trace norm (hence in operator norm), we have strong convergence:
\[
\|L_t f - L_{\sym} f\|_{\HPsi} \to 0 \quad \text{for all } f \in \HPsi.
\]
In particular, this holds for all \( f \in \Schwartz \), and each \( L_t f \in \Schwartz \), so
\[
L_{\sym} f = \lim_{t \to 0^+} L_t f \in \HPsi.
\]

\paragraph{Conclusion.}
Given any \( f \in \operatorname{Dom}(L_{\sym}) = \HPsi \), we can choose approximants \( f_n \in \Schwartz \) with
\[
f_n \to f \quad \text{and} \quad L_{\sym} f_n \to L_{\sym} f \quad \text{in } \HPsi,
\]
by combining Step 2 (density) and Step 3 (continuity). Hence, \( \Schwartz(\R) \) is a graph-norm core for \( L_{\sym} \), completing the proof.
\end{proof}
