\begin{remark}[Functional Calculus for Spectral Trace Pairings]
\label{rem:functional_calculus_trace}
For any \( \phi \in \Schwartz(\R) \), the spectral operator \( \phi(L_{\sym}) \) is well-defined via the spectral theorem. Since \( L_{\sym} \in \TC(\HPsi) \), its eigenvalues \( \{ \mu_n \} \subset \R \) satisfy \( \mu_n \to 0 \) and \( \sum_n |\mu_n| < \infty \).

The rapid decay of \( \phi(\mu_n) \) ensures:
\[
\sum_n |\phi(\mu_n)| < \infty,
\]
so the trace
\[
\Tr(\phi(L_{\sym})) := \sum_n \phi(\mu_n)
\]
is absolutely convergent. Therefore, all trace pairings \( \phi \mapsto \Tr(\phi(L_{\sym})) \) used in this chapter are rigorously defined for test functions \( \phi \in \Schwartz(\R) \).

This justifies interpreting \( \Tr(\phi(L_{\sym})) \) as a tempered distribution without requiring additional regularization.
\end{remark}
