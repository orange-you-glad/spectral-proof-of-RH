\begin{proof}[Proof of \lemref{lem:heat_trace_expansion}]
\textbf{(i)} Since \( L_{\mathrm{sym}} \in \mathcal{C}_1(H_{\Psi_\alpha}) \subset \mathcal{C}_2 \), we may factor \( L_{\mathrm{sym}} = AB \) with \( A, B \in \mathcal{C}_2 \). Then
\[
L_{\mathrm{sym}}^2 = A(BA)B \in \mathcal{C}_1,
\]
by Schatten ideal multiplication \( \mathcal{C}_2 \cdot \mathcal{C}_2 \subset \mathcal{C}_1 \). Hence, the semigroup \( e^{-tL_{\mathrm{sym}}^2} \in \mathcal{C}_1 \) for all \( t > 0 \), and the heat trace
\[
\Theta(t) := \operatorname{Tr}(e^{-tL_{\mathrm{sym}}^2})
\]
is smooth and finite for all \( t > 0 \) by \thmref{thm:canonical_operator_realization} and \lemref{lem:trace_norm_convergence_Lt_to_Lsym}.

\textbf{(ii)} Let \( \{ \mu_n \}_{n \in \mathbb{N}} \subset \mathbb{R} \) denote the eigenvalues of \( L_{\mathrm{sym}} \), counted with multiplicity. Since \( L_{\mathrm{sym}}^2 \in \mathcal{C}_1 \), the spectral trace formula gives
\[
\operatorname{Tr}(e^{-tL_{\mathrm{sym}}^2}) = \sum_{n=1}^{\infty} e^{-t \mu_n^2} < \infty.
\]
By the spectral theorem, \( e^{-tL_{\mathrm{sym}}^2} \) is a strongly continuous, trace-class semigroup.

From \lemref{lem:heat_kernel_diagonal_positivity} and \lemref{lem:hk_expansion_uniform}, the kernel \( K_t(x,y) \) of \( e^{-tL_{\mathrm{sym}}^2} \) is smooth, symmetric, and exponentially decaying. Using Paley–Wiener theory and parametrix expansions (cf.~\cite[Ch.~III]{Korevaar2004Tauberian}), we obtain the pointwise asymptotic:
\[
K_t(x,x) \sim \sum_{n=0}^{\infty} a_n(x)\, t^{n - 1/2}, \qquad t \to 0^+.
\]
Because \( K_t \in \mathcal{C}_1 \), term-by-term integration yields:
\[
\Theta(t) = \operatorname{Tr}(e^{-tL_{\mathrm{sym}}^2}) = \int_{\mathbb{R}} K_t(x,x) \, dx \sim \sum_{n=0}^{\infty} A_n t^{n - 1/2}, \qquad A_n := \int_{\mathbb{R}} a_n(x) \, dx,
\]
where convergence follows from the decay bounds in \lemref{lem:kernel_L2_weighted_bound}, \lemref{lem:mollified_profile_decay}, and \lemref{lem:xi_growth_bound}.

\textbf{(iii)} The leading singularity is governed by the spectral profile of \( \Xi(s) \), which via Hadamard factorization and Gaussian damping implies:
\[
\Theta(t) = \frac{1}{\sqrt{4\pi t}} \log\left( \frac{1}{t} \right) + \frac{c_0}{\sqrt{t}} + o(t^{-1/2}), \qquad \text{as } t \to 0^+,
\]
with
\[
c_0 := \int_{\mathbb{R}} a_1(x)\, dx.
\]

This logarithmic divergence is not Lebesgue integrable and necessitates analytic continuation for zeta-regularized determinant construction, as developed in Chapter~\ref{sec:determinant_identity}. While \( c_0 \) is formally defined, it plays no role in spectral bijection or determinant normalization.
\end{proof}
