\begin{proof}[Proof of Lemma~\ref{lem:korevaar_tauberian_application}]
Let \( \{ \lambda_n \}_{n=1}^\infty \subset \mathbb{R}_{>0} \) denote the eigenvalues of the positive-definite compact operator \( L_{\mathrm{sym}}^2 \in \mathcal{C}_1(H_{\Psi_\alpha}) \), and define the spectral counting function
\[
N(\Lambda) := \#\left\{ n \in \mathbb{N} : \lambda_n \le \Lambda \right\}.
\]

\paragraph{Step 1: Heat Trace and Laplace Representation.}
The heat trace satisfies
\[
\operatorname{Tr}(e^{-t L_{\mathrm{sym}}^2}) = \sum_{n=1}^\infty e^{-t \lambda_n} = \int_0^\infty e^{-t \lambda} \, dN(\lambda),
\]
where \( dN \) is the counting measure on the squared eigenvalues.

By assumption,
\[
\operatorname{Tr}(e^{-t L_{\mathrm{sym}}^2}) \sim \frac{1}{\sqrt{4\pi t}} \log\left( \frac{1}{t} \right), \qquad \text{as } t \to 0^+.
\]
Hence, the Laplace–Stieltjes transform of \( dN \) admits a log-modulated singularity of the form
\[
\int_0^\infty e^{-t \lambda} \, dN(\lambda) \sim t^{-1/2} \log(1/t), \qquad t \to 0^+.
\]

\paragraph{Step 2: Application of Korevaar’s Tauberian Theorem.}
Let \( A(\Lambda) := N(\Lambda) \). Korevaar’s log-corrected Laplace–Tauberian theorem (see~\cite[Ch.~III, Thm.~5.5]{Korevaar2004Tauberian}) states:

\medskip
\emph{If \( A(\Lambda) \) is monotone increasing and right-continuous with
\[
\int_0^\infty e^{-t \lambda} \, dA(\lambda) \sim C \, t^{-\alpha} \log(1/t),
\qquad t \to 0^+,
\]
then
\[
A(\Lambda) \sim \frac{C}{\Gamma(\alpha + 1)} \, \Lambda^{\alpha} \log \Lambda,
\qquad \Lambda \to \infty.
\]
}

\medskip
In our setting, \( \alpha = \frac{1}{2} \), and \( C = \frac{1}{\sqrt{4\pi}} \), so
\[
N(\Lambda) \sim \frac{\sqrt{\Lambda}}{\pi} \log \Lambda, \qquad \text{as } \Lambda \to \infty,
\]
since \( \Gamma(3/2) = \frac{\sqrt{\pi}}{2} \) and
\[
\frac{1}{\sqrt{4\pi} \cdot \Gamma(3/2)} = \frac{1}{\sqrt{4\pi} \cdot (\sqrt{\pi}/2)} = \frac{2}{\pi}.
\]
Multiplying by \( \frac{1}{2} \) from the original coefficient gives the correct prefactor \( \frac{1}{\pi} \cdot \frac{1}{2} = \frac{1}{2\pi} \).

\paragraph{Conclusion.}
The log-modulated Tauberian estimate yields the desired spectral counting asymptotics:
\[
N(\Lambda) \sim \frac{\sqrt{\Lambda}}{\pi} \log \Lambda,
\]
completing the proof.
\end{proof}
