\subsection*{Summary}

This chapter reformulates the Riemann Hypothesis as a statement of spectral rigidity: the spectrum and determinant of the canonical operator \( \Lsym \in \TC(\HPsi) \) determine both the location and multiplicity of all nontrivial zeros of \( \zeta(s) \).

\begin{itemize}
  \item \lemref{lem:spectral_encoding_injection} — The canonical determinant identity defines a multiplicity-preserving injection
  \[
  \rho \mapsto \mu_\rho := \frac{1}{i}(\rho - \tfrac{1}{2}),
  \]
  from the nontrivial zeros of \( \zeta(s) \) into \( \Spec(\Lsym) \setminus \{0\} \).

  \item \lemref{lem:det_zero_implies_spectrum} — Each zero of the determinant corresponds to an eigenvalue:
  \[
  \detz(I - \lambda \Lsym) = 0 \iff \lambda^{-1} \in \Spec(\Lsym).
  \]

  \item \lemref{lem:inject_zero_real_spectrum} — If \( \Spec(\Lsym) \subset \R \), then all nontrivial zeros lie on the critical line:
  \[
  \Spec(\Lsym) \subset \R \quad \Rightarrow \quad \Re(\rho) = \tfrac{1}{2}.
  \]

  \item \lemref{lem:real_spectrum_implies_rh_rigidity} — This implication holds without assuming bijectivity or simplicity, relying only on spectral reality and the determinant identity.

  \item \propref{prop:spectrum_reality_implies_rh_multiplicity} — If the spectrum is real and simple, then:
  \[
  \Re(\rho) = \tfrac{1}{2}, \qquad \operatorname{ord}_\rho(\zeta) = 1.
  \]

  \item \thmref{thm:rh_from_real_simple_spectrum} — If \( \Lsym \) has real, simple spectrum and satisfies the determinant identity, then RH holds and all nontrivial zeros are simple.

  \item \lemref{lem:spectral_symmetry} — Spectral symmetry \( \mu \mapsto -\mu \) follows from the functional equation \( \Xi(s) = \Xi(1 - s) \).

  \item \lemref{lem:trace_distribution_positivity} — The trace pairing
  \[
  \phi \mapsto \Tr(\phi(\Lsym))
  \]
  defines a positive tempered distribution.

  \item \lemref{lem:trace_distribution_positive} — Positivity extends to all nonnegative Schwartz functions. See \remref{rem:functional_calculus_trace} for justification.

  \item \propref{prop:inverse_spectral_rigidity} — Any two trace-class self-adjoint operators with the same determinant and spectrum are unitarily equivalent.
\end{itemize}

\medskip

Together, these results confirm that the Riemann Hypothesis follows purely from the spectral and determinant data of \( \Lsym \). This chapter completes the implication:
\[
\Spec(\Lsym) \subset \R \quad \Longrightarrow \quad \RH,
\]
and closes the analytic loop initiated in \thmref{thm:eq_of_rh}. The closure is reflected structurally in the dependency graph (\appref{app:dependency_graph}).

\medskip

\noindent
\textbf{Remarks.}
\begin{itemize}
  \item Spectral simplicity suffices—but is not required—for RH.
  \item The determinant–spectrum map arises intrinsically from Fredholm theory and the Hadamard structure of \( \Xi(s) \).
  \item Trace positivity reflects harmonic-analytic regularity of the spectral measure.
  \item Under Fourier diagonalization \( \mathcal{F}_L \), the trace pairing becomes integration against the spectral measure.
\end{itemize}

\medskip

\noindent
\textbf{Forward Link.}  
The final chapter consolidates these results into a logically complete equivalence:
\[
\RH \iff \Spec(\Lsym) \subset \R,
\]
as proven in \thmref{thm:truth_of_rh}.
