\begin{lemma}[Analyticity of the Automorphic Kernel]
\label{lem:automorphic_kernel_analytic}
Let \( \pi \in \mathcal{A}_{\mathrm{cusp}}(\GL_n) \) and suppose
\[
\phi_\pi(\lambda) := \Xi\left( \tfrac{1}{2} + i\lambda, \pi \right) \in \PW{{\pi_\pi}},
\]
where \( \PW{{\pi_\pi}} \) denotes the Paley–Wiener class of exponential type \( \pi_\pi \), and \( \phi_\pi \) is entire.

Then the inverse Fourier transform
\[
k^{(\pi)}(x) := \mathcal{F}^{-1}[\phi_\pi](x)
\]
defines a real-analytic, even, rapidly decaying function \( k^{(\pi)} \in \mathcal{S}(\R) \). Consequently, the kernel \( K^{(\pi)}(x, y) := k^{(\pi)}(x - y) \) inherits analytic dependence on the spectral parameter \( \lambda \in \mathbb{C} \), via the Paley–Wiener transform.
\end{lemma}
