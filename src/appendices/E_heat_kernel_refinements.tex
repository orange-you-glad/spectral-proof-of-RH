\section{Refinements of Heat Kernel Asymptotics}
\label{app:heat_kernel_refinements}

\noindent\textbf{[LOGICALLY INERT Appendix — Noncritical]}  
This appendix is not used in any analytic argument within Chapters~1–9. It explores potential refinements of the short-time heat trace asymptotics for the canonical operator
\[
\Theta(t) := \Tr(e^{-t L_{\sym}^2}),
\]
under assumptions stronger than those required to prove the Riemann Hypothesis.

\subsection*{Refined Expansion Structure}

If the spectrum \( \{ \mu_n \} \subset \Spec(L_{\sym}) \) exhibits additional arithmetic structure—such as regular gaps, symmetry constraints, or controlled multiplicity decay—then the heat trace may admit a sharper expansion:
\[
\Theta(t) = \frac{1}{\sqrt{4\pi t}} \log\left( \frac{1}{t} \right)
+ \frac{c_0}{\sqrt{t}} + o\left( \frac{1}{\sqrt{t}} \right),
\qquad t \to 0^+,
\]
for some constant \( c_0 \in \R \) reflecting subleading spectral contributions.

Analogous expansions appear for Laplace-type operators on singular manifolds, stratified spaces, or logarithmic densities. See~\cite{Seeley1967ComplexPowers, Gilkey1995Invariance, Vaillant2001HeatKernel} for classical treatments.

\subsection*{Implications for Tauberian Asymptotics}

If such refined expansions hold, then the eigenvalue counting function
\[
A(\Lambda) := \#\left\{ \mu_n^2 \le \Lambda \right\}
\]
admits a subleading Weyl-type asymptotic:
\[
A(\Lambda) = \frac{1}{2\pi} \Lambda^{1/2} \log \Lambda + C_1 \Lambda^{1/2} + o(\Lambda^{1/2}),
\]
where \( C_1 \in \R \) reflects spectral torsion or motivic degeneracies.

These refinements echo corrections found in higher-rank Selberg trace formulas and extended Tauberian theories~\cite{Korevaar2004Tauberian}.

\subsection*{Analytic Outlook}

Explicit identification of \( c_0 \) and \( C_1 \)—via Mellin transforms of \( \Theta(t) \), residues of the spectral zeta function \( \zeta_{L^2}(s) := \Tr(L_{\sym}^{-2s}) \), or via analytic continuation of associated Dirichlet series—could refine the analytic profile of \( \XiR(s) \).

While these terms play no role in the determinant identity or the RH equivalence proven in \cref{sec:spectral_implications}, they may be useful for:

\begin{itemize}
  \item Computing analytic torsion or secondary spectral invariants;
  \item Extending trace identities to modular or functorial lifts;
  \item Verifying compatibility with automorphic \( L \)-functions (\appref{app:functorial_extensions});
  \item Studying subleading corrections to zeta-regularized determinants~\cite{Elizalde1994ZetaRegularization}.
\end{itemize}

\medskip
\noindent
These conjectural refinements are analytically consistent with the canonical spectral model developed here but are not required for any core result.
