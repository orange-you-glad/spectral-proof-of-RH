\begin{proof}[Proof of Lemma~\ref{lem:real_spectrum_implies_rh}]
Let \( \rho \) be an arbitrary nontrivial zero of the Riemann zeta function \( \zeta(s) \), and define the associated spectral image under the canonical encoding:
\[
\mu_\rho := \frac{1}{i}(\rho - \tfrac{1}{2}).
\]

\paragraph{Step 1: Determinant Vanishing Implies Spectral Point.}
From the canonical determinant identity (Theorem~\ref{thm:det_identity_revised}), we have
\[
\det\nolimits_{\zeta}(I - \lambda L_{\mathrm{sym}}) = \frac{\Xi\left( \tfrac{1}{2} + i\lambda \right)}{\Xi\left( \tfrac{1}{2} \right)},
\]
which vanishes if and only if \( \lambda = \gamma \), where \( \rho = \tfrac{1}{2} + i\gamma \) is a nontrivial zero of \( \zeta(s) \). In particular,
\[
\mu_\rho = \frac{1}{i(\rho - \tfrac{1}{2})} = \frac{1}{\gamma} = \lambda^{-1}.
\]

By analytic Fredholm theory for trace-class operators, the zeros of the determinant correspond to nonzero reciprocal eigenvalues of \( L_{\mathrm{sym}} \). Thus,
\[
\mu_\rho \in \operatorname{Spec}(L_{\mathrm{sym}}).
\]

\paragraph{Step 2: Spectral Reality Implies Critical-Line Symmetry.}
Assumption (ii) states \( \operatorname{Spec}(L_{\mathrm{sym}}) \subset \mathbb{R} \), so \( \mu_\rho \in \mathbb{R} \). Therefore,
\[
\frac{1}{i}(\rho - \tfrac{1}{2}) \in \mathbb{R}
\quad \Rightarrow \quad \rho - \tfrac{1}{2} \in i\mathbb{R}
\quad \Rightarrow \quad \Re(\rho) = \tfrac{1}{2}.
\]

\paragraph{Conclusion.}
Since \( \rho \) was arbitrary among the nontrivial zeros of \( \zeta(s) \), it follows that all such zeros lie on the critical line. Hence, the Riemann Hypothesis holds.
\end{proof}
