\section{Logical Dependency Graph (Modular Proof Architecture)}
\label{app:dependency-graph}

This appendix presents the formal structure of the manuscript as a directed acyclic graph (DAG), in which each chapter builds only on previously established analytic foundations. No theorem appeals to any result logically equivalent to the Riemann Hypothesis itself prior to its proof, ensuring strict acyclicity and audit transparency.

For symbol definitions, see Appendix~\ref{app:notation-summary}.

\vspace{1ex}
\hrule
\vspace{1ex}

\begin{quote}
\textit{This proof flow diagram captures the modular structure of the analytic–spectral program. Each node reflects an acyclic dependency, ensuring strict logical sequencing from foundational definitions to the final equivalence with RH.}
\end{quote}

\vspace{1ex}
\hrule
\vspace{2ex}

\subsection*{Modular Proof Hierarchy}

\begin{itemize}
  \item \textbf{Chapter 1 — Foundational Structures:}  
  Introduces weighted Hilbert spaces \( H_{\Psi_\alpha} \) with \( \alpha > \pi \), establishes trace-class operator theory, and sets Paley–Wiener–type Fourier bounds. See Chapter~\ref{sec:foundations} for summary.

  \item \textbf{Chapter 2 — Canonical Operator Construction:}  
  Constructs mollified convolution operators \( L_t \) from the inverse Fourier transform of \( \Xi(s) \), and proves trace-norm convergence \( L_t \to L_{\mathrm{sym}} \) and self-adjointness of the limit.

  \item \textbf{Chapter 3 — Determinant Identity:}  
  Proves the canonical identity
  \[
  \det\nolimits_\zeta(I - \lambda L_{\mathrm{sym}}) = \frac{\Xi(\tfrac{1}{2} + i\lambda)}{\Xi(\tfrac{1}{2})},
  \]
  as an entire function of exponential type \( \pi \), using heat kernel expansions, logarithmic derivative matching, and Hadamard uniqueness.

  \item \textbf{Chapter 4 — Spectral Bijection:}  
  Establishes a multiplicity-preserving bijection between nontrivial zeros \( \rho \) of \( \zeta(s) \) and eigenvalues \( \mu_\rho = \tfrac{1}{i}(\rho - \tfrac{1}{2}) \in \operatorname{Spec}(L_{\mathrm{sym}}) \).

  \item \textbf{Chapter 5 — Heat Kernel and Trace Asymptotics:}  
  Derives upper and lower bounds on the heat trace, proves full short-time expansion, and applies Laplace integrability to recover spectral growth. Includes the key logarithmic derivative identity (Lemma~\ref{lem:log-derivative-determinant}) linking trace expansions to determinant regularization.

  \item \textbf{Chapter 6 — Spectral Rigidity and Equivalence:}  
  Proves spectral symmetry \( \mu \mapsto -\mu \), trace distribution positivity, and logical equivalence
  \[
  \mathrm{RH} \iff \operatorname{Spec}(L_{\mathrm{sym}}) \subset \mathbb{R},
  \]
  via determinant identity and bijective encoding (Theorem~\ref{thm:eq_of_rh}, Lemma~\ref{lem:spectral_rigidity_determinant}).

  \item \textbf{Chapter 7 — Determinantal Uniqueness:}  
  Shows that any compact, self-adjoint trace-class operator satisfying the same zeta-determinant identity is unitarily equivalent to \( L_{\mathrm{sym}} \).

  \item \textbf{Chapter 8 — Logical Closure:}  
  Assembles the analytic and spectral components to conclude that the Riemann Hypothesis holds.
\end{itemize}

\subsection*{Directed Proof Flow Diagram}
\phantomsection
\label{fig:dag_appendix_b}

\[
\begin{array}{c}
\textbf{Foundations (Ch.~1, $\alpha > \pi$)} \\[1ex]
\downarrow \\[1ex]
\textbf{Operator Construction (Ch.~2)} \\[1ex]
\downarrow \\[1ex]
\textbf{Determinant Identity (Ch.~3)} \Leftarrow \text{App.~\ref{app:zeta-function-background}, \ref{app:trace-ideals-review}} \\
\text{(Exponential type $\pi$, Hadamard zero structure)} \\[1ex]
\downarrow \\[1ex]
\textbf{Spectral Bijection (Ch.~4)} \\
\text{(Thm~\ref{thm:spectral-zero-bijection-revised})} \\[1ex]
\downarrow \\[1ex]
\textbf{Heat Trace Asymptotics (Ch.~5)} \\
\text{(Laplace integrability, log-derivative identity)} \\[1ex]
\downarrow \\[1ex]
\textbf{Spectral Rigidity \& RH Equivalence (Ch.~6)} \\
\text{(Thm~\ref{thm:eq_of_rh}, Lem~\ref{lem:spectral_rigidity_determinant})} \\[1ex]
\downarrow \\[1ex]
\textbf{Determinantal Uniqueness (Ch.~7)} \\[1ex]
\downarrow \\[1ex]
\textbf{Final Closure: RH Holds (Ch.~8)}
\end{array}
\]

\subsection*{Conclusion}

Every definition, lemma, and theorem is built upon previously validated constructions. The logical structure is strictly acyclic, with no forward references or circular dependencies. This modular proof design ensures full transparency under analytic audit and supports future formalization. The final theorem is derived from a tightly layered spectral hierarchy rooted in functional analysis, trace-class operator theory, and entire function factorization.

For notation, see Appendix~\ref{app:notation-summary}.
