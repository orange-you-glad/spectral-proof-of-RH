\section{Trace Ideals and Operator Norms}
\label{app:trace_ideals_review}

This appendix reviews the classical theory of compact operators, Schatten–von Neumann ideals, trace norms, and determinant expansions. These tools underpin the spectral realization of the completed zeta function \( \Xi(s) \) via the canonical trace-class convolution operator \( L_{\sym} \), and support the analytic infrastructure of Chapters~\ref{sec:operator_construction}–\ref{sec:logical_closure}.

The exposition follows Simon~\cite{Simon2005TraceIdeals}, Reed–Simon~\cite{ReedSimon1980I, ReedSimon1975II}, and classical Fredholm theory.

\subsection*{Schatten–von Neumann Ideals}

Let \( H \) be a separable complex Hilbert space. For \( p \in [1, \infty) \), define:
\[
\mathcal{C}_p(H) := \left\{ T \in \mathcal{K}(H) \;\middle|\; \|T\|_{\mathcal{C}_p} := \left( \sum_{n=1}^\infty \sigma_n(T)^p \right)^{1/p} < \infty \right\},
\]
where \( \{ \sigma_n(T) \} \) are the singular values of \( T \). Each \( \mathcal{C}_p \) is a Banach *-ideal, closed under adjoints and invariant under unitary conjugation.

\medskip
\noindent
Important special cases:
\begin{itemize}
  \item \( \mathcal{C}_1(H) \): trace-class operators;
  \item \( \mathcal{C}_2(H) \): Hilbert–Schmidt operators.
\end{itemize}

\subsection*{Trace-Class Operators}

If \( T \in \mathcal{C}_1(H) \), the trace is defined by
\[
\Tr(T) := \sum_n \langle T e_n, e_n \rangle,
\]
for any orthonormal basis \( \{e_n\} \subset H \); this definition is independent of the basis.

One has:
\[
|\Tr(T)| \le \|T\|_{\TC}, \qquad \|T\| \le \|T\|_{\TC}.
\]

\subsection*{Structural Properties}

\begin{itemize}
  \item Ideal nesting:
  \[
  \mathcal{C}_1 \subset \mathcal{C}_2 \subset \KC \subset \mathcal{B},
  \]
  where \( \KC \) is the ideal of compact operators and \( \mathcal{B} := \mathcal{B}(H) \) is the bounded operator algebra.

  \item All \( \mathcal{C}_p \) operators with \( p < \infty \) are compact.

  \item Trace cyclicity:
  \[
  \Tr(AB) = \Tr(BA) \quad \text{whenever } A \in \mathcal{C}_1(H),\; B \in \mathcal{B}(H).
  \]
\end{itemize}

\subsection*{Fredholm and Carleman Determinants}

Let \( T \in \mathcal{C}_1(H) \) be compact and trace-class.

\paragraph{Classical Fredholm determinant:}
\[
\det(I + T) := \prod_n (1 + \lambda_n), \qquad
\log \det(I + T) = \sum_{k=1}^\infty \frac{(-1)^{k+1}}{k} \Tr(T^k),
\]
where \( \lambda_n \) are the eigenvalues of \( T \), counted with multiplicity.

\paragraph{Carleman \(\zeta\)-regularized determinant:}
\[
\det\nolimits_\zeta(I - \lambda T) := \exp\left( -\sum_{k=1}^\infty \frac{\lambda^k}{k} \Tr(T^k) \right), \quad |\lambda| < \|T\|_{\TC}^{-1}.
\]

For \( T \in \mathcal{C}_1 \) self-adjoint, this series converges absolutely and defines an entire function of order one and exponential type determined by the trace norm. See \cite[Ch.~4]{Simon2005TraceIdeals}.

\subsection*{Spectral Mapping Diagram}

\begin{figure}[ht]
\centering
% Insert validated TikZ diagram from Chapter 3
\caption{Zeta zeros \( \rho \) map via \( \rho \mapsto \mu_\rho = \tfrac{1}{i}(\rho - \tfrac{1}{2}) \) to eigenvalues of the canonical operator \( L_{\sym} \).}
\label{fig:spectral-mapping}
\end{figure}

\subsection*{Conclusion}

The analytic theory of compact and trace-class operators, together with zeta-regularized determinants, justifies the canonical identity
\[
\det\nolimits_\zeta(I - \lambda L_{\sym}) = \frac{\Xi\left(\tfrac{1}{2} + i\lambda\right)}{\Xi\left(\tfrac{1}{2}\right)},
\]
for a unique compact, self-adjoint operator \( L_{\sym} \in \TC(H_{\Psi_\alpha}) \). This framework underpins the determinant expansion, heat kernel convergence, and spectral encoding mechanisms culminating in the proof of the Riemann Hypothesis in Chapter~\ref{sec:logical_closure}.
