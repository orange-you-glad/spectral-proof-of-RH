\subsection*{Summary}

\textbf{Determinant Foundations and Spectral Setup}
\begin{itemize}
  \item \defref{def:fredholm_determinant} — Fredholm determinant: eigenvalue product representation for \( T \in \TC \); analytic in \( \lambda \in \C \).
  \item \defref{def:carleman_zeta_determinant} — Carleman \(\zeta\)-regularized determinant: defined via Laplace transform of the heat trace; extended by analytic continuation.
  \item \defref{def:spectral_decomposition_compact} — Spectral decomposition for compact self-adjoint operators: orthonormal eigenbasis with discrete spectrum.
  \item \defref{def:spectral_zeta_function} — Spectral zeta function \( \zeta_{\Lsym^2}(s) \): defined via Mellin transform of \( \Tr(e^{-t \Lsym^2}) \), analytically continued via subtraction of singularities.
\end{itemize}

\textbf{Kernel Convergence and Determinant Construction}
\begin{itemize}
  \item \lemref{lem:kernel_trace_norm_convergence} — Kernel convergence: \( L_t \to \Lsym \) in trace norm.
  \item \lemref{lem:heat_semigroup_wellposed} — Semigroup \( \{e^{-t\Lsym^2}\} \subset \TC(\HPsi) \): holomorphic in \( t \), with exponential bounds.
  \item \lemref{lem:det_via_heat_trace} — Determinant via Laplace transform:
  \[
  \log \detz(I - \lambda \Lsym) = - \int_0^\infty \frac{e^{-\lambda t}}{t} \Tr(e^{-t \Lsym}) \, dt.
  \]
  \item \lemref{lem:laplace_heat_trace_convergence} — Absolute convergence of the Laplace integral ensures entire extension.
  \item \lemref{lem:spectral_zeta_from_heat} — Mellin continuation of the spectral zeta function \( \zeta_{\Lsym^2}(s) \) via heat trace expansion.
\end{itemize}

\textbf{Growth Control and Hadamard Classification}
\begin{itemize}
  \item \lemref{lem:det_growth_bound} — Growth bound:
  \[
  \log |\detz(I - \lambda \Lsym)| \le C |\lambda| \log(1 + |\lambda|),
  \]
  showing membership in \( \mathcal{E}_1^{\leq \pi} \).
  \item \lemref{lem:det_identity_entire_order_one}, \lemref{lem:exact_type_pi} — Entire function has exact exponential type \( \pi \), matching the Paley--Wiener class of \( \Xi(\tfrac{1}{2} + i\lambda) \).
  \item \lemref{lem:gamma_embedding_kernel_determinant} — Verifies proper embedding of Gamma factors in the spectral kernel and determinant growth.
\end{itemize}

\textbf{Hadamard Structure and Spectral Matching}
\begin{itemize}
  \item \lemref{lem:hadamard_linear_form} — Hadamard product form:
  \[
  \Xi\left( \tfrac{1}{2} + i\lambda \right) = \Xi\left( \tfrac{1}{2} \right) \prod_{\rho} \left(1 - \frac{\lambda}{\lambda_\rho}\right) e^{\lambda/\lambda_\rho}.
  \]
  \item \lemref{lem:trace_zero} — Trace centering:
  \[
  \Tr(\Lsym) = 0 \quad \Rightarrow \quad \detz(I) = 1.
  \]
  \item \lemref{lem:A_log_derivative} — Logarithmic derivative of the determinant:
  \[
  \frac{d}{d\lambda} \log \detz(I - \lambda \Lsym) = \Tr\left( (I - \lambda \Lsym)^{-1} \Lsym \right).
  \]
  \item \lemref{lem:spectrum_zero_bijection} — Spectral encoding:
  \[
  \rho \mapsto \mu_\rho := \tfrac{1}{i}(\rho - \tfrac{1}{2}) \in \Spec(\Lsym),
  \]
  establishing a bijection between \( \zeta \)-zeros and \( \Lsym \)-spectrum with multiplicities matched.
  \item \lemref{lem:hadamard_uniqueness_E1pi} — Uniqueness in \( \mathcal{E}_1^\pi \): entire function determined by zeros and normalization.
\end{itemize}

\textbf{Canonical Identity and Spectral Implications}
\begin{itemize}
  \item \thmref{thm:det_identity_revised} — Canonical determinant identity:
  \[
  \detz(I - \lambda \Lsym) = \frac{\Xi(\tfrac{1}{2} + i\lambda)}{\Xi(\tfrac{1}{2})}.
  \]
  \item \propref{prop:spectral_counting_weyl} — Spectral asymptotics:
  \[
  N(\Lambda) := \#\{ \mu_n^2 \le \Lambda \} \sim \frac{\sqrt{\Lambda}}{\pi} \log \Lambda,
  \]
  matching the classical zero-counting law.
\end{itemize}

\paragraph{Chapter Closure.}
This chapter completes the analytic core of the spectral program. The canonical trace-class operator \( \Lsym \in \TC(\HPsi) \) realizes the completed zeta function \( \Xi(s) \) as a Carleman \(\zeta\)-regularized Fredholm determinant. All analytic invariants—growth rate, exponential type, zero structure, multiplicities, and normalization—are uniquely captured.

The determinant identity forms the cornerstone for the spectral equivalence with RH, while the zero-to-spectrum bijection is established independently in Chapter~\ref{sec:spectral_correspondence}. The spectral implications of this encoding begin in \thmref{thm:eq_of_rh} and are logically closed in \thmref{thm:truth_of_rh}.
