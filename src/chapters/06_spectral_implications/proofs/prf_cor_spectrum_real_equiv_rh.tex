\begin{proof}[Proof of \corref{cor:spectrum_real_equiv_rh}]
This is an immediate consequence of \lemref{lem:reality_of_spectrum_and_rh}. For each nontrivial zero \( \rho = \beta + i\gamma \) of the Riemann zeta function \( \zetaR(s) \), we define the associated spectral parameter:
\[
\mu_\rho := \frac{1}{i(\rho - \tfrac{1}{2})}.
\]

By \lemref{lem:reality_of_spectrum_and_rh}, this quantity is real if and only if \( \beta = \tfrac{1}{2} \), i.e., \( \rho \) lies on the critical line. Hence:
\[
\mu_\rho \in \R \quad \Longleftrightarrow \quad \Re(\rho) = \tfrac{1}{2}.
\]

Therefore, all spectral values \( \mu_\rho \in \Spec(L_{\sym}) \) are real if and only if all nontrivial zeros \( \rho \) satisfy \( \Re(\rho) = \tfrac{1}{2} \), which is precisely the Riemann Hypothesis.

\medskip

\noindent
Thus, we obtain the analytic–spectral equivalence:
\[
\Spec(L_{\sym}) \subset \R \quad \Longleftrightarrow \quad \RH.
\]
This closes the spectral chain of implications initiated by the determinant identity and completes the operator-theoretic reformulation of RH.
\end{proof}
