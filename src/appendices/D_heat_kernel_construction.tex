\section{Heat Kernel Construction and Spectral Asymptotics}
\label{app:heat_kernel_construction}

This appendix provides technical derivations and analytic justifications for the heat kernel estimates used in \cref{sec:heat_kernel_asymptotics}. All main statements are stated as lemmas or propositions there; here we give detailed proofs relying on classical spectral theory, trace-class semigroup convergence, and parametrix expansions.

\subsection*{Notation}

Let \( L_{\sym} \in \TC(\HPsi) \) be the canonical compact, self-adjoint operator with discrete real spectrum \( \{ \mu_n \} \subset \R \), listed with multiplicities. The heat semigroup is defined via spectral calculus:
\[
e^{-t L_{\sym}^2} := \sum_{n=1}^\infty e^{-t \mu_n^2} P_n,
\]
where \( P_n \) is the projection onto the eigenspace for \( \mu_n \). Since \( L_{\sym}^2 \ge 0 \) and compact, \( e^{-tL_{\sym}^2} \in \TC \) for all \( t > 0 \), and the trace is:
\[
\Tr(e^{-t L_{\sym}^2}) = \sum_{n=1}^\infty e^{-t \mu_n^2} \cdot \operatorname{mult}(\mu_n).
\]

The kernel \( K_t(x,y) \) of this operator admits off-diagonal exponential decay and a singular short-time expansion. These structures form the analytic base for the spectral trace identities of \secref{sec:heat_kernel_asymptotics} and \secref{sec:determinant_identity}.

\subsection*{Scope of Results}

The lemmas proven here establish:

\begin{itemize}
  \item Exponential decay of the heat kernel \( K_t(x,y) \in \Schwartz(\R^2) \);
  \item Positivity and trace-class convergence of \( e^{-t L_{\sym}^2} \);
  \item Laplace–Mellin representation of the spectral zeta function and zeta determinant;
  \item Parametrix expansions yielding singular trace asymptotics:
  \[
  \Tr(e^{-t L_{\sym}^2}) \sim \frac{1}{\sqrt{4\pi t}} \log\left( \frac{1}{t} \right) + \cdots,
  \]
  as in \propref{prop:short_time_heat_expansion};

  \item Eigenvalue counting law:
  \[
  N(\lambda) := \#\{ \mu_n^2 \le \lambda \}
  \sim C \sqrt{\lambda} \log \lambda, \quad \lambda \to \infty,
  \]
  as proven in \propref{prop:spectral_counting_weyl}.
\end{itemize}

These results justify the well-posedness of the zeta-regularized determinant and validate the Tauberian analysis in \cref{sec:tauberian_growth}. They also support the rigidity implications derived in \cref{sec:spectral_rigidity}.

\subsection*{Conclusion}

This appendix secures the analytic infrastructure underlying the spectral model for the completed zeta function \( \XiR(s) \). It confirms that the heat semigroup \( e^{-t L_{\sym}^2} \) governs both the trace expansion and the determinant growth required to link \( \Spec(L_{\sym}) \) to the zero distribution of \( \zetaR(s) \). The logarithmic singularity in the heat trace—see \remref{rem:log_singularity_necessary}—determines the growth law and necessitates zeta-regularization of the determinant.
