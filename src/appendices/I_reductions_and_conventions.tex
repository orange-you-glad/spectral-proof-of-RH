\section{Reductions and Conventions}
\label{app:reductions_and_conventions}

\noindent
This appendix records global analytic conventions and structural assumptions used throughout the manuscript. These ensure convergence, compactness, and spectral closure across all operator-theoretic constructions. No result in the core chapters relies on unstated analytic input.

\subsection*{Hilbert Space Framework}

All Hilbert spaces \( H \) are complex, separable, and equipped with the standard Hermitian inner product. Unless otherwise stated, all operators \( T \colon H \to H \) are linear, bounded, and densely defined.

Operator convergence statements (e.g., \( L_t \to \Lsym \)) are understood in the trace norm \( \|\cdot\|_{\TC} \), unless specified otherwise. Self-adjointness and Schatten class inclusions are verified via kernel decay and Paley–Wiener theory~\cite{Simon2005TraceIdeals, ReedSimon1980I}.

\subsection*{Fourier Transform Convention}

We use the unitary Fourier transform on \( L^2(\R) \), defined by:
\[
\FT f(\lambda) := \frac{1}{\sqrt{2\pi}} \int_{\R} f(x)\, e^{-i\lambda x}\, dx,
\qquad \FT^{-1} = \FT, \qquad \FT^2 f(x) = f(-x).
\]

Convolution operators diagonalize under \( \FT \), and even real-valued convolution profiles yield self-adjoint operators.

\subsection*{Spectral Domain and Normalization}

Spectral variables \( \lambda \in \R \) are centered via \( s = \tfrac{1}{2} + i\lambda \), so that the critical line \( \Re(s) = \tfrac{1}{2} \) maps to \( \lambda \in \R \). This convention aligns spectral reality with the Riemann Hypothesis and is used throughout for determinant normalization.

No use of global fields, adèles, or automorphic representations appears in Chapters~\ref{sec:foundations}–\ref{sec:logical_closure}. All core results derive from classical real and complex analysis. Extensions to automorphic \( L \)-functions are proposed separately in \appref{app:functorial_extensions}.

\subsection*{Weight Function Normalization}

The canonical exponential weight is:
\[
\Psi_\alpha(x) := e^{\alpha |x|}, \qquad \alpha > \pi.
\]
This ensures:
\begin{itemize}
  \item Integrability of the kernel \( \FT^{-1}[\Xi(\tfrac{1}{2} + i\lambda)] \in L^1(\R, \Psi_\alpha) \);
  \item Trace-class convergence \( L_t \to \Lsym \in \TC(\HPsi) \);
  \item Validity of heat trace asymptotics and Laplace regularization for the zeta determinant.
\end{itemize}

Alternate weights such as \( |\Xi(\tfrac{1}{2} + ix)|^2 \) are not used due to insufficient decay for compactness in the trace-norm topology.

\subsection*{Spectral Parameterization}

The centered spectral profile is defined by:
\[
\phi(\lambda) := \Xi\left( \tfrac{1}{2} + i\lambda \right), \qquad \lambda \in \R.
\]
This parameterization ensures:
\begin{itemize}
  \item Symmetry: \( \phi(-\lambda) = \phi(\lambda) \), hence \( \Lsym \) is self-adjoint;
  \item Spectral kernel structure \( k(x) := \FT^{-1}[\phi](x) \), with real, even convolution profile;
  \item Spectral encoding via the canonical map:
  \[
  \rho = \tfrac{1}{2} + i\gamma \quad \longmapsto \quad \mu_\rho := \tfrac{1}{i}(\rho - \tfrac{1}{2}) = \gamma;
  \]
  \item Canonical realization of \( \Xi(s) \) as a zeta-regularized Fredholm determinant.
\end{itemize}
