\begin{theorem}[Analytic Identity for the Canonical Determinant]
\label{thm:det_identity_revised}
\leavevmode
\begin{tcolorbox}[colback=gray!3!white,colframe=black!75!white,title={\textbf{Canonical Determinant Identity}}]
Let \( \Lsym \in \TC(\HPsi) \) be the compact, self-adjoint operator constructed in \cref{sec:operator_construction} via convolution with the inverse Fourier transform of the completed Riemann zeta profile:
\[
\lambda \mapsto \Xi\left( \tfrac{1}{2} + i\lambda \right).
\]

\medskip

Then the Carleman \(\zeta\)-regularized Fredholm determinant
\[
f(\lambda) := \detz(I - \lambda \Lsym)
\]
is an entire function of order one and exact exponential type \( \pi \), satisfying the canonical analytic identity:
\begin{equation}
\detz(I - \lambda \Lsym)
= \frac{\Xi\left( \tfrac{1}{2} + i\lambda \right)}{\Xi\left( \tfrac{1}{2} \right)},
\qquad \forall \lambda \in \C.
\label{eq:determinant_identity}
\end{equation}

\medskip

This identity holds canonically with the following features:
\begin{itemize}
  \item \textbf{Spectral Encoding.} The zeros of the determinant coincide with the nontrivial zeros \( \rho \in \C \) of \( \zeta(s) \), via the spectral map
  \[
  \mu_\rho := \frac{1}{i}(\rho - \tfrac{1}{2}) \in \Spec(\Lsym),
  \]
  preserving multiplicities.

  \item \textbf{Normalization.} The value \( \Xi(\tfrac{1}{2}) \ne 0 \) is known (see, e.g.,~\cite[Thm.~2.3]{Titchmarsh1986Zeta}), so the identity is canonically normalized at \( \lambda = 0 \), and
  \[
  f(0) = \detz(I) = 1.
  \]

  \item \textbf{Hadamard Classification.} The function \( f(\lambda) \in \mathcal{E}_1^\pi \), the Hadamard class of entire functions of order one and exponential type \( \pi \), uniquely determined by their zero set and normalization.

  \item \textbf{Uniqueness.} By Hadamard’s factorization theorem~\cite[Ch.~3]{Levin1996EntireLectures}, the identity \eqref{eq:determinant_identity} is the unique such function in \( \mathcal{E}_1^\pi \) whose zero set matches the spectrum \( \{ \lambda_\rho = i(\rho - \tfrac{1}{2}) \} \) and whose normalization satisfies \( f(0) = 1 \).
\end{itemize}
\thmref{thm:canonical_operator_realization}
\lemref{lem:trace_class_Lt}
\lemref{lem:trace_norm_convergence_Lt_to_Lsym}
\lemref{lem:trace_norm_rate_convergence}
\lemref{lem:kernel_L2_weighted_bound}
\lemref{lem:kernel_symmetry}
\propref{prop:selfadjointness_Lt}
\end{tcolorbox}
\end{theorem}
