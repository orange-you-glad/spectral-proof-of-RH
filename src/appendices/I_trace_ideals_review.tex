\section{Trace Ideals and Operator Norms}
\label{app:trace_ideals_review}

This appendix reviews the classical theory of compact operators, Schatten–von Neumann ideals, trace norms, and determinant expansions. These analytic tools underlie the spectral realization of the completed zeta function \( \XiR(s) \) via the canonical trace-class convolution operator \( L_{\sym} \), and support the infrastructure developed across Chapters~\ref{sec:operator_construction}–\ref{sec:logical_closure}.

\subsection*{Schatten–von Neumann Ideals}

Let \( H \) be a separable Hilbert space. For \( p \in [1, \infty) \), define:
\[
\mathcal{C}_p(H) := \left\{ T \in \KC(H) \;\middle|\; \|T\|_{\mathcal{C}_p} := \left( \sum_{n=1}^\infty \sigma_n(T)^p \right)^{1/p} < \infty \right\},
\]
where \( \{\sigma_n(T)\} \) are the singular values of \( T \). Each \( \mathcal{C}_p \) is a Banach *-ideal, stable under adjoint and unitary conjugation.

\smallskip
Important special cases:
\begin{itemize}
  \item \( \mathcal{C}_1(H) \): trace-class operators;
  \item \( \mathcal{C}_2(H) \): Hilbert–Schmidt operators.
\end{itemize}

\subsection*{Trace-Class Operators}

If \( T \in \TC(H) \), then the trace is defined as:
\[
\Tr(T) := \sum_n \langle T e_n, e_n \rangle,
\]
for any orthonormal basis \( \{e_n\} \subset H \); the definition is basis-independent.

One has the estimates:
\[
|\Tr(T)| \le \|T\|_{\TC}, \qquad \|T\| \le \|T\|_{\TC}.
\]

\subsection*{Structural Properties}

\begin{itemize}
  \item Ideal inclusions:
  \[
  \TC \subset \mathcal{C}_2 \subset \KC \subset \mathcal{B},
  \]
  where \( \KC \) denotes compact operators and \( \mathcal{B} := \mathcal{B}(H) \) the full bounded algebra.

  \item All \( \mathcal{C}_p \) ideals with \( p < \infty \) are contained in \( \KC \).

  \item Trace cyclicity:
  \[
  \Tr(AB) = \Tr(BA), \quad \text{for } A \in \TC, \; B \in \mathcal{B}.
  \]
\end{itemize}

\subsection*{Fredholm and Zeta-Regularized Determinants}

Let \( T \in \TC(H) \) be compact and self-adjoint.

\paragraph{Classical Fredholm Determinant:}
\[
\det(I + T) := \prod_n (1 + \lambda_n), \quad
\log \det(I + T) = \sum_{k=1}^\infty \frac{(-1)^{k+1}}{k} \Tr(T^k),
\]
where \( \{\lambda_n\} \) are the eigenvalues of \( T \), counted with multiplicity.

\paragraph{Carleman \(\zeta\)-Regularized Determinant:}
\[
\det\nolimits_\zeta(I - \lambda T) := \exp\left( -\sum_{k=1}^\infty \frac{\lambda^k}{k} \Tr(T^k) \right), \quad |\lambda| < \|T\|_{\TC}^{-1}.
\]
For self-adjoint \( T \in \TC \), this converges absolutely and defines an entire function of order one and exponential type governed by the trace norm. See~\cite[Ch.~4]{Simon2005TraceIdeals}.

\subsection*{Spectral Mapping Diagram}

\begin{figure}[ht]
\centering
% Insert validated TikZ or external image file
\caption{Zeta zeros \( \rho \mapsto \mu_\rho = \tfrac{1}{i}(\rho - \tfrac{1}{2}) \) correspond to the spectrum of \( L_{\sym} \).}
\label{fig:spectral-mapping}
\end{figure}

\subsection*{Conclusion}

The analytic theory of trace-class operators and \(\zeta\)-regularized determinants justifies the canonical identity:
\[
\det\nolimits_\zeta(I - \lambda L_{\sym}) = \frac{\XiR(\tfrac{1}{2} + i\lambda)}{\XiR(\tfrac{1}{2})},
\]
for the unique compact, self-adjoint operator \( L_{\sym} \in \TC(\HPsi) \). This determinant framework supports the heat kernel expansion, the spectral map \( \rho \mapsto \mu_\rho \), and the analytic equivalence \( \Spec(L_{\sym}) \subset \R \iff \RH \) proven in \cref{sec:logical_closure}.
