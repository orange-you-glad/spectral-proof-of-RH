\begin{proof}[Proof of \lemref{lem:spectral_convexity_estimate}]
Let \( \{ \lambda_n \} \subset \R_{>0} \) denote the nonzero eigenvalues of \( L_{\sym}^2 \in \TC(\HPsi) \), ordered non-decreasingly and counted with multiplicity. Define the spectral counting function
\[
N(\lambda) := \#\left\{ n \in \N : \lambda_n \le \lambda \right\}.
\]

\paragraph{Step 1: Heat Trace as Laplace–Stieltjes Transform.}
Since \( L_{\sym}^2 \) is positive and trace class, its spectral representation yields:
\[
\Tr(e^{-t L_{\sym}^2}) = \sum_{n=1}^\infty e^{-t \lambda_n}
= \int_0^\infty e^{-t\lambda} \, dN(\lambda),
\]
where \( dN(\lambda) = \sum_n \delta_{\lambda_n} \) is a locally finite measure.

\paragraph{Step 2: Short-Time Envelope from \cref{sec:heat_kernel_asymptotics}.}
By Proposition~\ref{prop:two_sided_heat_trace_bounds}, the heat trace satisfies:
\[
\Tr(e^{-t L_{\sym}^2}) \leq c_2 \, t^{-1/2}, \qquad \text{as } t \to 0^+.
\]
Hence, the Laplace integral satisfies:
\[
\int_0^\infty e^{-t\lambda} \, dN(\lambda) \lesssim t^{-1/2}.
\]

\paragraph{Step 3: Tauberian Inversion.}
Applying the classical Tauberian theorem for Laplace–Stieltjes transforms (see \defref{def:tauberian_theorem}), we obtain:
\[
\int_0^\infty e^{-t\lambda} \, dN(\lambda) \lesssim t^{-\alpha}
\quad \Longrightarrow \quad
N(\lambda) \lesssim \lambda^{\alpha} \quad \text{as } \lambda \to \infty.
\]
Setting \( \alpha = 1/2 \), we conclude:
\[
N(\lambda) \leq C\, \lambda^{1/2}, \qquad \text{for all } \lambda \ge \lambda_0,
\]
for some constants \( C > 0 \), \( \lambda_0 > 0 \).

\paragraph{Conclusion.}
This establishes the convex growth envelope
\[
N(\lambda) = O(\lambda^{1/2}),
\]
and confirms that \( N \in \mathcal{R}_{1/2} \), justifying Tauberian admissibility for the refined log-modulated asymptotics proven in \cref{sec:tauberian_growth}.
\end{proof}
