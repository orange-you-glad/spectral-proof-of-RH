\subsection*{Summary}

\begin{itemize}
  % --- Tauberian Principle ---
  \item \defref{def:tauberian-theorem} — Classical Laplace–Tauberian framework: the short-time behavior of the heat trace \( \operatorname{Tr}(e^{-tL^2}) \) determines high-energy spectral growth \( A(\Lambda) \) via inverse Laplace analysis.

  % --- Asymptotic Bounds and Growth Envelopes ---
  \item \lemref{lem:spectral_convexity_estimate} — From the envelope bound \( \operatorname{Tr}(e^{-t L^2}) = O(t^{-1/2}) \), we deduce the coarse spectral estimate:
  \[
  N(\lambda) = O(\lambda^{1/2}),
  \]
  situating the spectrum within the minimal compact class consistent with trace-class control.

  \item \lemref{lem:korevaar_tauberian_application} — Using the refined expansion
  \[
  \operatorname{Tr}(e^{-t L^2}) \sim \frac{1}{\sqrt{4\pi t}} \log\left( \frac{1}{t} \right),
  \]
  and Korevaar’s Tauberian inversion, we derive:
  \[
  N(\lambda) \sim \frac{\sqrt{\lambda}}{\pi} \log \lambda.
  \]

  \item \lemref{lem:tauberian-heat-trace-application} — Two-sided heat trace bounds confirm:
  \[
  A(\Lambda) \asymp \Lambda^{1/2},
  \]
  identifying the regular variation envelope implied by kernel compactness and sparse spectrum.

  \item \lemref{lem:log-corrected-tauberian-application} — Under logarithmic refinement, we obtain:
  \[
  A(\Lambda) \sim \frac{1}{2\pi} \Lambda^{1/2} \log \Lambda
  \quad \text{as } \Lambda \to \infty,
  \]
  placing the spectrum in the log-modulated Tauberian class \( \mathcal{R}_{1/2}^{\log} \).

  \item \corref{cor:zeta-compatibility} — The spectral growth law of \( L_{\mathrm{sym}} \) matches the Riemann–von Mangoldt formula for the nontrivial zeros of \( \zeta(s) \):
  \[
  N_\zeta(T) = \frac{T}{2\pi} \log\left( \frac{T}{2\pi} \right) - \frac{T}{2\pi} + O(\log T),
  \]
  confirming analytic equivalence between the zeta zero distribution and the spectrum of the canonical operator.
\end{itemize}

\medskip
\noindent
(For higher-order corrections to the heat kernel expansion and Tauberian remainder estimates, see Appendix~\ref{app:heat-kernel-refinements}.)

\vspace{0.5em}

\noindent\textbf{Analytic flow diagram:}
\[
\begin{aligned}
\text{Heat Kernel}
&\xrightarrow{\text{Laplace}} \text{Trace Asymptotics} \\
&\xrightarrow{\text{Tauberian}} \text{Spectral Growth} \\
&\xrightarrow{\text{Hadamard}} \text{Zeta Determinant}
\end{aligned}
\]

\begin{remark}[Transition to Spectral Rigidity]
We have shown that the asymptotic eigenvalue distribution of \( L_{\mathrm{sym}} \) encodes the nontrivial zeros of the Riemann zeta function. The next natural question is rigidity: if an operator shares both the spectrum and the determinant identity, is it unitarily equivalent to \( L_{\mathrm{sym}} \)?

This question leads directly to the rigidity results developed in Chapter~\ref{sec:spectral-rigidity}.
\end{remark}

\vspace{0.5em}

\noindent
While this chapter establishes the global spectral envelope of \( L_{\mathrm{sym}} \), the local statistics—e.g., pair correlation or GUE-type universality—remain open. Speculative connections to random matrix theory are discussed in Appendix~\ref{app:additional-structures}.
