\begin{definition}[Canonical Spectral Map]
\label{def:spectral_zero_map}
Let \( \rho \in \C \) be a nontrivial zero of the Riemann zeta function \( \zetaR(s) \), so that \( \zetaR(\rho) = 0 \) and \( \rho \ne \tfrac{1}{2} \). Define the canonical spectral map
\[
\mu_\rho := \frac{1}{i(\rho - \tfrac{1}{2})},
\]
which sends a zero \( \rho = \tfrac{1}{2} + i\gamma \) to the real number \( \mu_\rho = \frac{1}{\gamma} \in \R \setminus \{0\} \).

\medskip
\noindent
This map identifies nontrivial zeta zeros with the nonzero spectrum of the canonical trace-class operator \( L_{\sym} \in \TC(\HPsi) \), via the determinant identity
\[
\det\nolimits_\zeta(I - \lambda L_{\sym}) = \frac{\XiR\left(\tfrac{1}{2} + i\lambda\right)}{\XiR\left(\tfrac{1}{2}\right)},
\]
in which the poles of the logarithmic derivative correspond precisely to the points \( \lambda = \mu_\rho \in \Spec(L_{\sym}) \).

\medskip
\noindent
The inverse map assigns to each nonzero spectral value \( \mu \in \Spec(L_{\sym}) \setminus \{0\} \) the corresponding nontrivial zeta zero:
\[
\rho_\mu := \tfrac{1}{2} - \frac{i}{\mu}.
\]

\medskip
\noindent
This bijection preserves multiplicities and encodes the nontrivial zeros of \( \zetaR(s) \) in the discrete spectrum of \( L_{\sym} \), providing the analytic substrate for the spectral realization of the Riemann Hypothesis.
\end{definition}
