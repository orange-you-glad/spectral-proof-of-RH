\begin{proposition}[Two-Sided Heat Trace Bounds]
\label{prop:two_sided_heat_trace_bounds}
Let \( L_{\sym} \in \TC(\HPsi) \) be the canonical compact, self-adjoint operator on the exponentially weighted Hilbert space \( \HPsi := L^2(\R, e^{\alpha|x|} dx) \) for \( \alpha > \pi \), with discrete real spectrum \( \{\mu_n\} \subset \R \).

Then there exist constants \( c_1, c_2 > 0 \) and \( t_0 > 0 \) such that for all \( t \in (0, t_0] \),
\[
c_1\, t^{-1/2}
\;\le\;
\Tr\left( e^{-t L_{\sym}^2} \right)
\;\le\;
c_2\, t^{-1/2}.
\]
Explicitly, one may take
\[
c_1 := \frac{1}{4 \, \| L_{\sym} \|_{\TC}}, \qquad
c_2 := (1 + e^{-1}) \cdot \| L_{\sym} \|_{\TC},
\]
as established in \lemref{lem:hk_lower_bound} and \lemref{lem:hk_upper_bound}, respectively.

\medskip

\noindent
This two-sided estimate confirms that the heat trace asymptotics obey the scaling law \( \Theta(t) \sim t^{-1/2} \) as \( t \to 0^+ \), consistent with spectral dimension one. The bound holds uniformly for small time \( t \in (0, t_0] \), independently of the spectral multiplicity structure, and confirms that \( \Theta(t) \in R_{1/2} \) in the Tauberian class of regularly varying functions.
\end{proposition}
