\begin{proof}[Proof of \lemref{lem:spectral_bijection_consistency}]
Let \( \rho = \tfrac{1}{2} + i\gamma \) be a nontrivial zero of the Riemann zeta function \( \zetaR(s) \), and define
\[
\mu_\rho := \frac{1}{i(\rho - \tfrac{1}{2})}.
\]

\paragraph{Bijection Properties.}
The bijection follows by combining the results of the preceding lemmas:

\begin{itemize}
  \item By \lemref{lem:zero_to_eigenvalue_injection}, each nontrivial zero \( \rho \) yields a nonzero eigenvalue \( \mu_\rho \in \Spec(L_{\sym}) \), and the map \( \rho \mapsto \mu_\rho \) is injective with multiplicities preserved.

  \item \lemref{lem:spectral_exhaustivity} establishes that every nonzero eigenvalue \( \mu \in \Spec(L_{\sym}) \setminus \{0\} \) arises from such a \( \rho \), thereby proving surjectivity of the map.

  \item \lemref{lem:spectral_multiplicity_matching} confirms that the algebraic multiplicity of each eigenvalue \( \mu_\rho \) equals the order of vanishing of \( \zetaR(s) \) at \( \rho \).
\end{itemize}

\paragraph{Conclusion.}
Therefore, the map
\[
\rho \longmapsto \mu_\rho := \frac{1}{i(\rho - \tfrac{1}{2})}
\]
defines a bijection of multisets between the nontrivial zeros of \( \zetaR(s) \) and the nonzero spectrum of the canonical operator \( L_{\sym} \in \TC(\HPsi) \), with multiplicities preserved. This completes the spectral correspondence implied analytically by the determinant identity.
\end{proof}
