\begin{remark}[No RH Assumption Used in Spectral Encoding]
\label{rem:no_rh_assumption}
Throughout Chapters~\ref{sec:determinant_identity}–\ref{sec:spectral_correspondence}, all constructions, estimates, and identities are derived unconditionally, without assuming the Riemann Hypothesis.

In particular:
\begin{itemize}
  \item The construction of the canonical operator \( L_{\sym} \in \TC(\HPsi) \) via mollified convolution is entirely analytic, based on Fourier-analytic kernel estimates and trace-norm convergence, with no spectral assumptions on the zeros of \( \zetaR \).
  
  \item The determinant identity
  \[
  \det\nolimits_\zeta(I - \lambda L_{\sym}) = \frac{\XiR\left( \tfrac{1}{2} + i\lambda \right)}{\XiR\left( \tfrac{1}{2} \right)}
  \]
  is derived explicitly from semigroup estimates, Paley–Wiener bounds, and analytic regularization of the Fredholm determinant, independently of RH.

  \item The Hadamard product and spectral correspondence \( \rho \mapsto \mu_\rho := \tfrac{1}{i(\rho - \tfrac{1}{2})} \) are constructed from the zero set of \( \XiR(s) \), and the bijective structure—including multiplicity preservation—relies solely on analytic properties of entire functions of order one.

  \item All trace-class, Hilbert–Schmidt, and Schatten norm estimates used to control semigroup convergence, determinant growth, and spectral discreteness are proven via analytic kernel bounds and do not require any a priori assumptions about the distribution of zeta zeros.
\end{itemize}

\noindent
The Riemann Hypothesis first enters as a derived theorem in Chapter~\ref{sec:spectral_implications}, where it is shown to be logically equivalent to the spectral reality condition \( \Spec(L_{\sym}) \subset \R \). No steps in the determinant identity or in the spectral bijection rely on this spectral condition being assumed in advance.
\end{remark}
