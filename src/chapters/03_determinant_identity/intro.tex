\subsection*{Introduction}
\label{sec:det_intro}

This chapter establishes the canonical analytic identity
\begin{equation}
\label{eq:zeta_det_identity}
\det\nolimits_\zeta(I - \lambda L_{\sym}) 
= \frac{\Xi\left( \tfrac{1}{2} + i\lambda \right)}{\Xi\left( \tfrac{1}{2} \right)},
\end{equation}
which realizes the completed Riemann zeta function \( \Xi(s) \) as the Carleman \(\zeta\)-regularized Fredholm determinant of the canonical compact, self-adjoint convolution operator
\[
L_{\sym} \in \TC(\HPsi),
\]
constructed in \cref{sec:operator_construction} as the trace-norm limit of mollified operators \( L_t \), each defined by convolution against a smoothed inverse Fourier transform of \( \Xi \).

\begin{previewbox}
\textbf{Preconditions Summary.}
All analytic claims in this chapter are based on the following rigorously validated results:

\begin{itemize}
  \item \( \phi(\lambda) := \Xi(\tfrac{1}{2} + i\lambda) \) is real, even, entire of exponential type \( \pi \) (see \cref{lem:exact_type_pi}), hence \( \phi \in \PW{\pi} \) and satisfies Paley–Wiener decay.

  \item Its inverse Fourier transform \( k(x) := \FT^{-1}[\phi](x) \in L^1(\R, \Psi_\alpha^{-1}(x)\, dx) \) for any \( \alpha > \pi \), so convolution by \( k \) defines a trace-class operator on \( \HPsi \).

  \item The operator \( L_{\sym} \in \TC(\HPsi) \) is compact and self-adjoint, defined via trace-norm convergence \( L_t \to L_{\sym} \) (see \cref{lem:trace_norm_convergence_Lt_to_Lsym}, \cref{lem:trace_norm_limit_unique}).

  \item The heat semigroup \( \{ e^{-t L_{\sym}^2} \}_{t > 0} \subset \TC(\HPsi) \) exists, is holomorphic in \( t \), and satisfies asymptotic bounds (see \cref{lem:heat_semigroup_wellposed}).

  \item The determinant is defined locally by the power series (\cref{lem:det_via_heat_trace}) and globally via Laplace transform, ensuring entire continuation and exponential type control.
\end{itemize}
\end{previewbox}

\medskip

\noindent
The analytic proof proceeds through the following modular components, grounded in the theory of trace-class determinants~\cite[Ch.~4]{Simon2005TraceIdeals} and Paley–Wiener entire functions~\cite[Ch.~9]{Levin1996EntireLectures}:

\begin{itemize}
  \item \textbf{Heat trace construction and analytic continuation:} We define the determinant via Carleman regularization using the Laplace–Mellin transform of the semigroup \( e^{-t L_{\sym}^2} \); see \cref{lem:heat_semigroup_wellposed}, \cref{lem:det_via_heat_trace}.

  \item \textbf{Short-time singularity and growth control:} The heat trace expansion
  \begin{equation}
  \label{eq:heat_trace_asymptotics}
  \Tr(e^{-t L_{\sym}^2}) = \frac{1}{\sqrt{4\pi t}} \log\left(\frac{1}{t}\right) + \mathcal{O}(t^{-1/2}), \quad t \to 0^+,
  \end{equation}
  determines the genus-one Hadamard structure and confirms that the determinant has exponential type exactly \( \pi \) (see \cref{lem:det_identity_entire_order_one}).

  \item \textbf{Canonical identification via Hadamard theory:} The function \( \lambda \mapsto \det_\zeta(I - \lambda L_{\sym}) \) lies in \( \mathcal{E}_1^\pi \). By Hadamard’s factorization theorem and uniqueness (\cref{lem:hadamard_uniqueness_E1pi}), it must match the normalized spectral profile
  \[
  \frac{\Xi(\tfrac{1}{2} + i\lambda)}{\Xi(\tfrac{1}{2})},
  \]
  via logarithmic derivative matching (\cref{lem:A_log_derivative}) and normalization (\cref{lem:trace_zero}).

  \item \textbf{Canonical normalization:} The trace vanishing
  \[
  \Tr(L_{\sym}) = 0
  \]
  ensures
  \[
  \det\nolimits_\zeta(I) = 1,
  \]
  eliminating ambiguity in the exponential prefactor of the Hadamard product and pinning the determinant canonically.
\end{itemize}

\noindent
This analytic phase is self-contained, strictly acyclic, and free of assumptions about the location of the nontrivial zeros \( \rho \in \C \) of \( \zeta(s) \). The spectral encoding map
\[
\rho \mapsto \mu_\rho := \tfrac{1}{i}(\rho - \tfrac{1}{2}) \in \Spec(L_{\sym})
\]
is constructed separately in \cref{sec:spectral_correspondence}, after the determinant structure is fully classified.
