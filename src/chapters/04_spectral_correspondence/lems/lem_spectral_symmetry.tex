\begin{lemma}[Spectral Symmetry of \( L_{\sym} \)]
\label{lem:spectral_symmetry}

Let \( L_{\sym} \in \TC(\HPsi) \) be the canonical compact, self-adjoint operator on the exponentially weighted Hilbert space
\[
\HPsi := L^2(\R, e^{\alpha |x|} dx), \qquad \alpha > \pi,
\]
constructed via trace-norm convergence from real, even mollified kernels as in \thmref{thm:canonical_operator_realization}. Suppose the determinant identity holds:
\[
\det\nolimits_\zeta(I - \lambda L_{\sym}) = \frac{\XiR\left( \tfrac{1}{2} + i\lambda \right)}{\XiR\left( \tfrac{1}{2} \right)},
\]
as established in \thmref{thm:det_identity_revised}.

Then the spectrum of \( L_{\sym} \) is symmetric under reflection:
\[
\mu \in \Spec(L_{\sym}) \quad \Longrightarrow \quad -\mu \in \Spec(L_{\sym}),
\]
with multiplicities preserved.

\medskip

\noindent
This spectral symmetry follows from the functional identity \( \XiR(\tfrac{1}{2} + i\lambda) = \XiR(\tfrac{1}{2} - i\lambda) \), which implies that the spectral determinant is even in \( \lambda \). The Hadamard factorization structure described in \lemref{lem:hadamard_linear_form} encodes this evenness as spectral root symmetry. Moreover, the convolution kernel of \( L_{\sym} \) is real and even, so the operator commutes with parity, which reinforces the symmetry of its spectrum.
\end{lemma}
