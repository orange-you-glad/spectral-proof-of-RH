\begin{lemma}[Spectral Exhaustivity of \( L_{\mathrm{sym}} \)]
\label{lem:spectral_exhaustivity}
Let \( L_{\mathrm{sym}} \in \mathcal{C}_1(H_{\Psi_\alpha}) \) denote the canonical compact, self-adjoint operator, and suppose the regularized determinant identity holds:
\[
\det\nolimits_\zeta(I - \lambda L_{\mathrm{sym}})
= \frac{\Xi\left(\tfrac{1}{2} + i\lambda\right)}{\Xi\left(\tfrac{1}{2}\right)}
\qquad \forall \lambda \in \mathbb{C}.
\]

Then for every nonzero eigenvalue \( \mu \in \operatorname{Spec}(L_{\mathrm{sym}}) \setminus \{0\} \), there exists a unique nontrivial zero \( \rho = \tfrac{1}{2} + i\gamma \) of \( \zeta(s) \) such that
\[
\mu = \mu_\rho := \frac{1}{i(\rho - \tfrac{1}{2})} = \frac{1}{\gamma}.
\]

Moreover, the algebraic multiplicity of the eigenvalue \( \mu \) coincides with the order of vanishing of \( \zeta(s) \) at \( \rho \).

\medskip
\noindent
Hence, the canonical spectral map \( \rho \mapsto \mu_\rho \) from nontrivial zeta zeros to \( \operatorname{Spec}(L_{\mathrm{sym}}) \setminus \{0\} \) is surjective and multiplicity-preserving.
\end{lemma}
% 