\begin{proof}[Proof of \lemref{lem:real_spectrum_implies_rh_rigidity}]
Let \( \rho \) be a nontrivial zero of \( \zeta(s) \), and define the canonical spectral image:
\[
\mu_\rho := \frac{1}{i}(\rho - \tfrac{1}{2}).
\]

\paragraph{Step 1: Determinant Zero Implies Spectral Inclusion.}
From the determinant identity
\[
\det\nolimits_\zeta(I - \lambda L_{\mathrm{sym}}) = \frac{\Xi\left( \tfrac{1}{2} + i\lambda \right)}{\Xi\left( \tfrac{1}{2} \right)},
\]
we have \( \det\nolimits_\zeta(I - \lambda L_{\mathrm{sym}}) = 0 \) precisely when \( \lambda = \gamma \), where \( \rho = \tfrac{1}{2} + i\gamma \).

Then
\[
\mu_\rho = \frac{1}{i(\rho - \tfrac{1}{2})} = \frac{1}{\gamma} = \lambda^{-1}.
\]
By analytic Fredholm theory, this implies \( \mu_\rho \in \operatorname{Spec}(L_{\mathrm{sym}}) \setminus \{0\} \).

\paragraph{Step 2: Real Spectrum Implies RH.}
By assumption, \( \operatorname{Spec}(L_{\mathrm{sym}}) \subset \mathbb{R} \), so \( \mu_\rho \in \mathbb{R} \). Therefore,
\[
\frac{1}{i}(\rho - \tfrac{1}{2}) \in \mathbb{R} \quad \Rightarrow \quad \rho - \tfrac{1}{2} \in i\mathbb{R} \quad \Rightarrow \quad \operatorname{Re}(\rho) = \tfrac{1}{2}.
\]

\paragraph{Conclusion.}
Since \( \rho \) was arbitrary, all nontrivial zeros of \( \zeta(s) \) lie on the critical line. Hence, the Riemann Hypothesis holds.
\end{proof}
