\begin{proof}[Proof of Lemma~\ref{lem:determinant_fixes_spectrum}]
Let \( f(\lambda) := \det\nolimits_\zeta(I - \lambda L) \) and suppose
\[
f(\lambda) = \frac{\Xi(\tfrac{1}{2} + i\lambda)}{\Xi(\tfrac{1}{2})}
= \det\nolimits_\zeta(I - \lambda L_{\mathrm{sym}}),
\quad \forall \lambda \in \mathbb{C},
\]
with \( L \in \mathcal{S}_1(H_{\Psi_\alpha}) \), compact and self-adjoint, and \( \operatorname{Tr}(L) = 0 \).

\paragraph{Step 1: Entire Function Identity.}
Both \( f(\lambda) \) and \( \det\nolimits_\zeta(I - \lambda L_{\mathrm{sym}}) \) are entire functions of order one and exponential type \( \pi \), normalized so that \( f(0) = 1 \), and with logarithmic derivatives corresponding to trace expansions:
\[
\frac{d}{d\lambda} \log f(\lambda)
= \operatorname{Tr}[(I - \lambda L)^{-1} L].
\]

\paragraph{Step 2: Canonical Zero Sets.}
Since the two determinant functions coincide, their zero sets (with multiplicities) agree. That is, the multiset of poles of their logarithmic derivatives is identical:
\[
\Spec(L) \setminus \{0\} = \Spec(L_{\mathrm{sym}}) \setminus \{0\}.
\]

\paragraph{Step 3: Trace Normalization.}
The trace-zero condition ensures that any exponential ambiguity in the Hadamard factorization of the determinant is eliminated, i.e., there is no residual \( e^{a\lambda} \) term. This ensures complete spectral rigidity of \( L \) via its determinant.

\paragraph{Conclusion.}
The nonzero spectrum of \( L \) is uniquely determined by its determinant, and hence matches that of \( L_{\mathrm{sym}} \) with multiplicities. This completes the proof.
\end{proof}
% 