\section{Heat Kernel Construction and Spectral Asymptotics}
\label{app:heat_kernel_construction}

\noindent\textbf{[Analytic Infrastructure Appendix]}  
This appendix provides detailed analytic derivations of the heat kernel bounds and trace expansions used in Chapters~\ref{sec:heat_kernel_asymptotics} and~\ref{sec:determinant_identity}. While many results are used as inputs for determinant regularization in Chapter~3, they are rigorously proved here. Their forward use is modular and acyclic, as verified in Appendix~\ref{app:dependency_graph}.

\subsection*{Notation}

Let \( \Lsym \in \TC(\HPsi) \) denote the canonical compact, self-adjoint operator with discrete spectrum \( \{ \mu_n \} \subset \R \), listed with multiplicities. The heat semigroup is defined via spectral calculus:
\[
e^{-t \Lsym^2} := \sum_{n=1}^\infty e^{-t \mu_n^2} P_n,
\]
where \( P_n \) is the orthogonal projection onto the eigenspace of \( \mu_n \). Since \( \Lsym^2 \ge 0 \) is compact, we have \( e^{-t \Lsym^2} \in \TC(\HPsi) \) for all \( t > 0 \), and its trace is given by:
\[
\Tr(e^{-t \Lsym^2}) = \sum_{n=1}^\infty e^{-t \mu_n^2} \cdot \operatorname{mult}(\mu_n).
\]

The associated kernel \( K_t(x, y) \) decays rapidly off the diagonal and admits a singular short-time expansion. These analytic properties underlie the trace identities of Chapter~\ref{sec:heat_kernel_asymptotics}, and justify the determinant construction of Chapter~\ref{sec:determinant_identity}.

\subsection*{Scope of Results}

This appendix establishes the following:

\begin{itemize}
  \item Exponential kernel decay: \( K_t(x, y) \in \Schwartz(\R^2) \) for all \( t > 0 \);
  \item Positivity and trace-class continuity of the semigroup \( e^{-t \Lsym^2} \);
  \item Laplace–Mellin representation for the spectral zeta function and its regularization;
  \item Parametrix-based short-time expansion:
  \[
  \Tr(e^{-t \Lsym^2}) \sim \frac{1}{\sqrt{4\pi t}} \log\left( \frac{1}{t} \right) + \cdots,
  \]
  as proved in \propref{prop:short_time_heat_expansion};

  \item Spectral counting law:
  \[
  N(\lambda) := \#\{ \mu_n^2 \le \lambda \} \sim C \sqrt{\lambda} \log \lambda,
  \]
  for large \( \lambda \), as shown in \propref{prop:spectral_counting_weyl}.
\end{itemize}

These analytic statements validate the zeta-determinant construction and underpin the spectral density results derived via Tauberian inversion in Chapter~\ref{sec:tauberian_growth}. They also support the spectral rigidity analysis of Chapter~\ref{sec:spectral_rigidity}.

\begin{remark}[Closure of Forward Dependencies from Chapter~3]
The trace asymptotics and integrability results proved here are invoked in Chapter~\ref{sec:determinant_identity} to define and validate the Carleman \(\zeta\)-regularized determinant \( \detz(I - \lambda \Lsym) \). In particular, the logarithmic singularity and trace continuity retroactively justify the Laplace representation and log-derivative identity (\cref{lem:det_via_heat_trace}, \cref{lem:A_log_derivative}). This dependency is modular and tracked in Appendix~\ref{app:dependency_graph}.
\end{remark}

\subsection*{Conclusion}

This appendix secures the analytic infrastructure underlying the canonical spectral model for \( \XiR(s) \). It shows that the heat semigroup \( e^{-t \Lsym^2} \) governs both determinant growth and spectral asymptotics. The logarithmic singularity in the trace expansion (see \remref{rem:log_singularity_necessary}) confirms the need for zeta regularization and dictates the growth class and entire type of the determinant.

\medskip
\noindent
Although proved downstream, all uses of this material in Chapter~\ref{sec:determinant_identity} are logically sound and noncircular, and tracked explicitly in the DAG of Appendix~\ref{app:dependency_graph}.
