\section*{Summary}
\addcontentsline{toc}{section}{Summary}

In this chapter we constructed a canonical compact, self-adjoint trace-class operator \( \Lsympi \in \TC(\Hpsipi) \) associated to each cuspidal automorphic representation \( \pi \in \mathcal{A}_{\mathrm{cusp}}(\mathrm{GL}_n) \). Under two analytic assumptions — the entire, order-one continuation of \( \Xi(s, \pi) \) and the exponential decay of its inverse Fourier kernel — we proved:

\begin{itemize}
  \item The mollified convolution operators \( \Ltpi \) converge in trace norm to \( \Lsympi \).
  \item The Fredholm determinant of \( \Lsympi \) satisfies the identity
  \[
  \detz(I - \lambda \Lsympi) = \frac{\Xi\left(\tfrac{1}{2} + i\lambda, \pi\right)}{\Xi\left(\tfrac{1}{2}, \pi\right)}.
  \]
  \item The eigenvalues of \( \Lsympi \) encode the scaled zeros of \( \Xi(s, \pi) \), with multiplicities.
  \item The Generalized Riemann Hypothesis for \( \pi \) is equivalent to \( \Spec(\Lsympi) \subset \R \).
\end{itemize}

This generalization functorially extends the zeta spectral framework to a class of automorphic \( L \)-functions. It shows that, conditional on standard analytic bounds, the spectral approach applies uniformly across the \( \GL_n \)-automorphic setting. 

In the next chapter, we close the logical structure by applying this analytic theory to the Riemann case and verifying the equivalence of real spectrum with the Riemann Hypothesis in the specific context of \( \zeta(s) \), thereby finalizing the spectral proof.
