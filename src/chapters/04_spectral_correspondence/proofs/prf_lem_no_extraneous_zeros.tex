\begin{proof}[Proof of \lemref{lem:no_extraneous_zeros}]
The canonical determinant is given by
\[
f(\lambda) := \det\nolimits_\zeta(I - \lambda L_{\mathrm{sym}}) = \frac{\Xi\left(\tfrac{1}{2} + i\lambda\right)}{\Xi\left(\tfrac{1}{2}\right)}.
\]
Since \( \Xi(s) \) is an entire function of order 1 and genus 1, its Hadamard factorization admits the form
\[
\Xi\left(\tfrac{1}{2} + i\lambda\right)
= \Xi\left(\tfrac{1}{2}\right) \prod_{\rho} \left(1 - \frac{\lambda}{\mu_\rho} \right)
\exp\left( \frac{\lambda}{\mu_\rho} \right),
\]
where \( \mu_\rho := \frac{1}{i}(\rho - \tfrac{1}{2}) \), and the product runs over all nontrivial zeros \( \rho \in \mathbb{C} \) of the Riemann zeta function \( \zeta(s) \).

\medskip
\noindent
The logarithmic derivative of \( f \) is
\[
\frac{f'(\lambda)}{f(\lambda)} = \sum_\rho \left( \frac{1}{\lambda - \mu_\rho} + \frac{1}{\mu_\rho} \right),
\]
from which it follows that all poles of \( f'/f \) lie precisely at \( \lambda = \mu_\rho \), with multiplicity equal to the order of vanishing of \( \zeta(s) \) at \( \rho \).

\medskip
\noindent
The exponential factor
\[
\exp\left( \sum_\rho \frac{\lambda}{\mu_\rho} \right)
\]
is entire and nonvanishing, and introduces no additional zeros. If it did, then \( f(\lambda) \) would vanish outside the set \( \{ \mu_\rho \} \), contradicting the spectral realization from \secref{sec:spectral_correspondence} and violating the normalization \( f(0) = 1 \).

\medskip
\noindent
Indeed, since the canonical operator \( L_{\mathrm{sym}} \in \mathcal{C}_1(H_{\Psi_\alpha}) \) is trace class with
\[
\operatorname{Tr}(L_{\mathrm{sym}}) = \sum_{\mu \in \operatorname{Spec}(L_{\mathrm{sym}})} \mu = 0,
\]
the genus-one exponential term introduces no singularities and preserves the entire character of the determinant. Thus, all zeros of \( f(\lambda) \) arise solely from the Hadamard product over spectral values \( \mu_\rho \), and the nonzero spectrum satisfies:
\[
\operatorname{Spec}(L_{\mathrm{sym}}) \setminus \{0\} = \{ \mu_\rho \}.
\]
\end{proof}
