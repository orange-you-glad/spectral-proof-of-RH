\begin{proposition}[Refined Short-Time Heat Trace Expansion]
\label{prop:short_time_heat_expansion}
Let \( L_{\mathrm{sym}} \in \mathcal{C}_1(H_{\Psi_\alpha}) \) be the canonical compact, self-adjoint operator on the exponentially weighted Hilbert space \( H_{\Psi_\alpha} := L^2(\mathbb{R}, e^{\alpha |x|} \, dx) \) for some \( \alpha > \pi \). Then, as \( t \to 0^+ \), the spectral heat trace satisfies the refined singular expansion:
\[
\operatorname{Tr}\left(e^{-t L_{\mathrm{sym}}^2}\right)
= \frac{1}{\sqrt{4\pi t}} \log\left( \frac{1}{t} \right)
+ c_0 \sqrt{t} + o\left( \sqrt{t} \right),
\]
for some constant \( c_0 \in \mathbb{R} \), where the remainder \( o(\sqrt{t}) \) vanishes uniformly as \( t \to 0^+ \) over compact subintervals of \( (0, t_0] \).

\medskip
\noindent
The leading-order singularity \( \frac{1}{\sqrt{4\pi t}} \log(1/t) \) reflects the logarithmic divergence induced by the spectral structure of the mollified convolution kernel defining \( L_{\mathrm{sym}} \). This divergence originates from the genus-one Hadamard structure of the completed zeta function and implies non-integrability of the trace near \( t = 0 \).

\medskip
\noindent
The correction coefficient \( c_0 \) arises from the first regular term in the local heat kernel expansion and satisfies
\[
c_0 := \int_{\mathbb{R}} a_1(x) \, dx,
\]
where the diagonal expansion takes the form
\[
K_t(x,x) \sim \sum_{n=0}^{\infty} a_n(x)\, t^{n - \frac{1}{2}}.
\]

\medskip
\noindent
This refined asymptotic plays a foundational role in the determinant expansion and Tauberian theory developed in Chapters~\ref{sec:determinant-identity} and~\ref{sec:tauberian-growth}. In particular, the logarithmic divergence necessitates analytic continuation in the Laplace transform and underpins the regularized determinant identity:
\[
\log \det\nolimits_{\zeta}(I - \lambda L_{\mathrm{sym}}) = - \int_0^\infty \frac{e^{-\lambda^2 t}}{t} \operatorname{Tr}\left(e^{-t L_{\mathrm{sym}}^2} \right) \, dt,
\]
where the integral must be interpreted in the zeta-regularized sense.
\end{proposition}
