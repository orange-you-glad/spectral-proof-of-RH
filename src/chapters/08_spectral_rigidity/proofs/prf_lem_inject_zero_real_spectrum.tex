\begin{proof}[Proof of Lemma~\ref{lem:inject-zero-real-spectrum}]
Let \( \rho = \tfrac{1}{2} + i\gamma \in \mathbb{C} \) be an arbitrary nontrivial zero of the Riemann zeta function \( \zeta(s) \). Under assumption (i), this zero is mapped to a nonzero eigenvalue of \( L_{\mathrm{sym}} \) via the spectral encoding:
\[
\mu_\rho := \frac{1}{i}(\rho - \tfrac{1}{2}) = \frac{1}{i}(i\gamma) = \gamma.
\]

\paragraph{Step 1: Reality of Spectrum.}
Assumption (ii) states that \( \mu_\rho \in \operatorname{Spec}(L_{\mathrm{sym}}) \subset \mathbb{R} \), and that all eigenvalues are real and simple. Hence,
\[
\mu_\rho = \gamma \in \mathbb{R}.
\]
This immediately implies that \( \rho = \tfrac{1}{2} + i\gamma \) has \( \gamma \in \mathbb{R} \), so
\[
\Re(\rho) = \tfrac{1}{2}.
\]

\paragraph{Step 2: Exhaustion.}
Since the mapping \( \rho \mapsto \mu_\rho \) is bijective and multiplicity-preserving by hypothesis, every nontrivial zero of \( \zeta(s) \) corresponds to a unique real eigenvalue of \( L_{\mathrm{sym}} \). Therefore, all nontrivial zeros lie on the critical line.

\paragraph{Conclusion.}
We conclude that
\[
\rho \in \operatorname{Zeros}(\zeta(s)) \quad \Longrightarrow \quad \Re(\rho) = \tfrac{1}{2},
\]
establishing the Riemann Hypothesis under the stated spectral assumptions.
\end{proof}
