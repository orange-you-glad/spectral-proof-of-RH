\subsection*{Introduction}

This chapter constructs the analytic infrastructure necessary to define and analyze the canonical trace-class operator \( \Lsym \) that underlies the spectral realization of the Riemann Hypothesis (\( \RH \)). The operator is obtained as the trace-norm limit of a family of mollified convolution operators \( \{L_t\}_{t > 0} \), defined via inverse Fourier transforms of mollified zeta spectral profiles. These operators act on the exponentially weighted Hilbert space
\[
\HPsi := L^2(\R, e^{\alpha |x|} dx), \quad \text{with } \alpha > \pi.
\]

\vspace{0.5em}

\begin{center}
\begin{tikzpicture}[node distance=20pt, every node/.style={align=center},>=Stealth]
  \node (phi) [draw, rectangle] {\( \phi_t(\lambda) := e^{-t\lambda^2} \Xi(\tfrac{1}{2} + i\lambda) \)};
  \node (kt) [draw, rectangle, below=of phi] {\( k_t(x) := \FT^{-1}[\phi_t](x) \)};
  \node (kernel) [draw, rectangle, below=of kt] {\( K_t(x,y) := k_t(x - y) \)};
  \node (op) [draw, rectangle, below=of kernel] {\( (L_t f)(x) := \int_{\R} K_t(x,y) f(y)\,dy \)};
  \node (limit) [draw, rectangle, below=of op, yshift=-0cm] {\( \Lsym := \lim_{t \to 0^+} L_t \in \TC(\HPsi) \)};
  \node (det) [draw, rectangle, below=of limit, yshift=-0cm] {\( \detz(I - \lambda \Lsym) = \frac{\Xi(\tfrac{1}{2} + i\lambda)}{\Xi(\tfrac{1}{2})} \)};

  \draw[->] (phi) -- (kt);
  \draw[->] (kt) -- (kernel);
  \draw[->] (kernel) -- (op);
  \draw[->] (op) -- (limit);
  \draw[->] (limit) -- (det);
\end{tikzpicture}
\end{center}

\vspace{0.5em}

The following analytic properties are verified:

\begin{itemize}
    \item \textbf{Trace-class inclusion:} The inverse Fourier transform of \( \Xi(\tfrac{1}{2} + i\lambda) \) decays as \( e^{-\pi|x|} \). Operators \( L_t \in \TC(\HPsi) \) are trace class if and only if \( \alpha > \pi \), with sharpness shown in \propref{prop:trace_class_sharpness}.
    
    \item \textbf{Schatten control and convergence:} Uniform trace-norm bounds hold for \( L_t \), and trace-norm convergence \( L_t \to \Lsym \) is rigorously established.
    
    \item \textbf{Essential self-adjointness:} The limit operator \( \Lsym \) is essentially self-adjoint on \( \Schwartz \), via Nelson's theorem; see \remref{rem:sobolev_core_reference}.
    
    \item \textbf{Heat semigroup structure:} The semigroup \( \{e^{-t \Lsym^2}\}_{t > 0} \) is holomorphic and trace class, with exponential norm bounds; see \lemref{lem:heat_semigroup_existence}.
    
    \item \textbf{Paley--Wiener embedding:} The spectral profile \( \phi(\lambda) := \Xi(\tfrac{1}{2} + i\lambda) \) belongs to the Paley--Wiener class \( \PW{\pi} \), enabling localization and determinant regularization.
\end{itemize}

These properties culminate in a trace-norm convergent operator \( \Lsym \in \TC(\HPsi) \), whose zeta-regularized Fredholm determinant satisfies
\[
\detz(I - \lambda \Lsym) = \frac{\Xi\left(\tfrac{1}{2} + i\lambda\right)}{\Xi\left(\tfrac{1}{2}\right)}.
\]

This identity is rigorously proved in \secref{sec:determinant_identity}, without assuming \( \RH \) or spectral surjectivity. The analytic tools developed here serve as the base layer for all spectral and determinant arguments to follow.

For a modular dependency diagram, see \appref{app:dependency_graph}.
