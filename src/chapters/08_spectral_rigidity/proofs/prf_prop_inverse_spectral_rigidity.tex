\begin{proof}[Proof of \propref{prop:inverse_spectral_rigidity}]
Let \( L_1, L_2 \in \mathcal{C}_1(H_{\Psi_\alpha}) \) be compact, self-adjoint operators satisfying:
\[
\operatorname{Spec}(L_1) = \operatorname{Spec}(L_2), \qquad
\det\nolimits_\zeta(I - \lambda L_1) = \det\nolimits_\zeta(I - \lambda L_2), \quad \forall \lambda \in \mathbb{C}.
\]

\paragraph{Step 1: Spectral Theorem and Orthonormal Bases.}
By the spectral theorem, each \( L_j \) admits an orthonormal basis \( \{e_n^{(j)}\} \subset H_{\Psi_\alpha} \) with corresponding eigenvalues \( \{\lambda_n\} \subset \mathbb{R} \), repeated with multiplicities, satisfying:
\[
L_j f = \sum_{n=1}^\infty \lambda_n \langle f, e_n^{(j)} \rangle e_n^{(j)}, \quad j = 1,2.
\]

\paragraph{Step 2: Determinant Equivalence.}
Since each operator is trace class, its Carleman determinant is expressed as:
\[
\det\nolimits_\zeta(I - \lambda L_j) = \prod_{n=1}^\infty (1 - \lambda \lambda_n),
\]
which converges absolutely on compact subsets of \( \mathbb{C} \). Equality of determinants for all \( \lambda \in \mathbb{C} \) thus implies equality of spectral data, including multiplicities—this follows from \lemref{lem:uniqueness_from_determinant}.

\paragraph{Step 3: Construction of Intertwiner.}
Define a unitary operator \( U \colon H_{\Psi_\alpha} \to H_{\Psi_\alpha} \) by setting \( U e_n^{(1)} := e_n^{(2)} \). This defines a bijective isometry and yields:
\[
U L_1 U^{-1} f = \sum_{n=1}^\infty \lambda_n \langle f, e_n^{(2)} \rangle e_n^{(2)} = L_2 f,
\]
so we conclude:
\[
L_2 = U L_1 U^{-1}.
\]

\paragraph{Conclusion.}
The operators \( L_1 \) and \( L_2 \) are unitarily equivalent. If both arise from the canonical convolution construction realizing the completed zeta function \( \Xi(s) \), then they must share the same eigenbasis. Hence \( U = \operatorname{Id} \), and it follows that \( L_1 = L_2 \) by spectral and convolutional uniqueness.
\end{proof}
