\subsection*{Introduction to the Closure of the Spectral Program}
\label{subsec:closure_of_spectral_program}

This chapter concludes the analytic reformulation of the Riemann Hypothesis as a statement of spectral rigidity. We synthesize the operator-theoretic, determinant-theoretic, and trace-theoretic constructions developed in prior chapters into a logically acyclic, formally closed equivalence.

\medskip

The modular structure proceeds through the following canonical components:

\begin{itemize}
  \item \textbf{Canonical Operator Construction.}  
  A compact, self-adjoint, trace-class operator
  \[
  \Lsym \in \TC(\HPsi)
  \]
  is constructed as the trace-norm limit of mollified convolution operators, with kernels derived from the inverse Fourier transform of the completed zeta function \( \Xi(s) \). Paley–Wiener decay ensures trace-class inclusion and Schatten convergence.

  \item \textbf{Determinant Identity.}  
  The Carleman zeta-regularized Fredholm determinant satisfies
  \[
  \detz(I - \lambda \Lsym) = \frac{\Xi\left(\tfrac{1}{2} + i\lambda\right)}{\Xi\left(\tfrac{1}{2}\right)},
  \qquad \forall \lambda \in \C,
  \]
  transferring the zero set of \( \Xi(s) \) to the spectral data of \( \Lsym \) and establishing its canonical analytic encoding.

  \item \textbf{Spectral Multiplicity Encoding.}  
  The Hadamard factorization of \( \Xi(s) \) ensures:
  \[
  \operatorname{ord}_{\rho} \zeta(s) = \operatorname{mult}_{\mu_\rho}(\Lsym), \quad \text{with } \mu_\rho := \tfrac{1}{i}(\rho - \tfrac{1}{2}),
  \]
  so the determinant records both location and multiplicity of the nontrivial zeros.

  \item \textbf{Spectral Bijection.}  
  The map \( \rho \mapsto \mu_\rho \) defines a bijection between the nontrivial zeros of \( \zeta(s) \) and the nonzero spectrum of \( \Lsym \), with multiplicities preserved (see \thmref{thm:spectral_zero_bijection_revised}).

  \item \textbf{Spectral Symmetry.}  
  The functional identity \( \Xi(s) = \Xi(1 - s) \) implies:
  \[
  \mu \in \Spec(\Lsym) \Rightarrow -\mu \in \Spec(\Lsym),
  \]
  and ensures that the spectrum is symmetric about the origin.

  \item \textbf{Spectral Rigidity and Logical Equivalence.}  
  The spectral condition
  \[
  \Spec(\Lsym) \subset \R
  \]
  is logically equivalent to the Riemann Hypothesis:
  \[
  \RH \iff \Re(\rho) = \tfrac{1}{2} \text{ for all nontrivial zeros } \rho \iff \mu_\rho \in \R.
  \]
  This is established in \thmref{thm:rh_spectrum_equiv} without assuming RH.

  \item \textbf{Spectral Determinacy.}  
  The spectrum of \( \Lsym \) fully determines the zero set of \( \zeta(s) \) and reconstructs \( \Xi(s) \) via its zeta determinant (see \corref{cor:spectral_determines_zeta}).

  \item \textbf{Positivity and Regularization.}  
  The functional trace pairing
  \[
  \phi \mapsto \Tr(\phi(\Lsym))
  \]
  defines a positive tempered distribution on \( \R \) (see \lemref{lem:trace_distribution_positivity}), compatible with analytic continuation and determinant regularization.

  \item \textbf{Analytic Closure.}  
  The RH equivalence follows strictly from:
  \begin{itemize}
    \item spectral theory for compact self-adjoint operators;
    \item trace-class convergence and Schatten analysis;
    \item Hadamard theory of entire functions of order one;
    \item Laplace–Mellin regularization and heat trace asymptotics.
  \end{itemize}
  No use is made of modular forms, L-functions, or unproven arithmetic input.
\end{itemize}

\medskip

The logical flow and dependency graph for this construction is diagrammed in Appendix~\ref{app:dependency_graph} (see Figure~\ref{fig:dag_visual}). The full equivalence
\[
\RH \iff \Spec(\Lsym) \subset \R
\]
is proven formally and acyclically in Chapter~\ref{sec:logical_closure}.
