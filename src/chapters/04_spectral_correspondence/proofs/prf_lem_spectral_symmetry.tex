\begin{proof}[Proof of \lemref{lem:spectral_symmetry}]
Let \( \phi(\lambda) := \XiR\left(\tfrac{1}{2} + i\lambda\right) \). Since \( \XiR(s) \) is entire and satisfies the functional equation \( \XiR(s) = \XiR(1 - s) \), it follows that \( \phi(\lambda) \) is real-valued and even on \( \R \); that is,
\[
\phi(-\lambda) = \phi(\lambda), \qquad \overline{\phi(\lambda)} = \phi(\lambda).
\]

Define the convolution kernel
\[
k(x - y) := \frac{1}{2\pi} \int_{\R} e^{i\lambda(x - y)} \phi(\lambda)\, d\lambda,
\]
and set \( K(x, y) := k(x - y) \). Then \( K(x, y) = K(y, x) \in \R \), as
\[
k(x - y) = \frac{1}{2\pi} \int_{\R} e^{i\lambda(x - y)} \phi(\lambda)\, d\lambda = k(y - x),
\]
using the evenness of \( \phi \).

Let \( \widetilde{L}_{\sym} \colon L^2(\R) \to L^2(\R) \) be the convolution operator defined by
\[
(\widetilde{L}_{\sym} f)(x) := \int_{\R} K(x, y) f(y)\, dy.
\]
Then \( \widetilde{L}_{\sym} \) is a compact, real, symmetric, and self-adjoint operator on \( L^2(\R) \).

\paragraph{Step 1: Spectral Symmetry in \( L^2(\R) \).}
By the spectral theorem for compact self-adjoint operators on real Hilbert spaces, the spectrum of \( \widetilde{L}_{\sym} \) is symmetric about the origin:
\[
\mu \in \Spec(\widetilde{L}_{\sym}) \quad \Longrightarrow \quad -\mu \in \Spec(\widetilde{L}_{\sym}),
\]
with equal algebraic multiplicities.

\paragraph{Step 2: Transfer via Unitary Equivalence.}
Let \( U \colon \HPsi \to L^2(\R) \) be the unitary operator defined by \( (Uf)(x) := \sqrt{\PsiW(x)} f(x) \), where \( \PsiW(x) := e^{\alpha |x|} \). Then
\[
L_{\sym} = U^{-1} \widetilde{L}_{\sym} U.
\]
Since unitary equivalence preserves spectral data, it follows that
\[
\mu \in \Spec(L_{\sym}) \quad \Longrightarrow \quad -\mu \in \Spec(L_{\sym}),
\]
with identical multiplicities.
\end{proof}
