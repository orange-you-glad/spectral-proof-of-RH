\subsection*{Summary}

This chapter reformulates the Riemann Hypothesis as a statement of spectral rigidity: the spectrum and determinant of the canonical operator \( L_{\sym} \in \TC(\HPsi) \) determine both the location and multiplicity of all nontrivial zeros of \( \zetaR(s) \).

\begin{itemize}
  \item \lemref{lem:spectral_encoding_injection} — The canonical determinant identity induces a multiplicity-preserving injection
  \[
  \rho \mapsto \mu_\rho := \frac{1}{i}(\rho - \tfrac{1}{2}),
  \]
  from nontrivial zeros of \( \zetaR(s) \) into the nonzero spectrum of \( L_{\sym} \).

  \item \lemref{lem:det_zero_implies_spectrum} — Each zero of the determinant \( \det\nolimits_\zeta(I - \lambda L_{\sym}) \) corresponds to a reciprocal eigenvalue \( \mu = \lambda^{-1} \in \Spec(L_{\sym}) \).

  \item \lemref{lem:inject_zero_real_spectrum} — If the spectrum is real and simple, then all nontrivial zeta zeros lie on the critical line:
  \[
  \Spec(L_{\sym}) \subset \R \quad \Longrightarrow \quad \Re(\rho) = \tfrac{1}{2}.
  \]

  \item \lemref{lem:real_spectrum_implies_rh_rigidity} — The same implication holds without requiring bijectivity or simplicity, using only determinant vanishing and spectrum reality.

  \item \propref{prop:spectrum_reality_implies_rh_multiplicity} — Real, simple spectrum implies both RH and simplicity of all nontrivial zeros.

  \item \thmref{thm:rh_from_real_simple_spectrum} — The determinant identity, real spectrum, and eigenvalue simplicity together imply:
  \[
  \text{RH holds, and all nontrivial zeros of } \zetaR(s) \text{ are simple.}
  \]

  \item \lemref{lem:spectral_symmetry} — Spectral symmetry under \( \mu \mapsto -\mu \) follows from the functional identity \( \XiR(\tfrac{1}{2} + i\lambda) = \XiR(\tfrac{1}{2} - i\lambda) \).

  \item \lemref{lem:trace_distribution_positivity} — The trace pairing
  \[
  \phi \mapsto \Tr(\phi(L_{\sym}))
  \]
  defines a positive tempered distribution on \( \R \), including:
  \[
  \Tr(e^{-t L_{\sym}^2}) \ge 0.
  \]

  \item \lemref{lem:trace_distribution_positive} — Positivity extends to all nonnegative Schwartz test functions. See also \remref{rem:functional_calculus_trace} for justification of the trace pairing domain.

  \item \propref{prop:inverse_spectral_rigidity} — If two trace-class operators share the same spectrum and determinant identity, they are unitarily equivalent.
\end{itemize}

These results confirm that the Riemann Hypothesis can be derived purely from the spectral and determinant data of \( L_{\sym} \). This chapter completes the analytic implication
\[
\Spec(L_{\sym}) \subset \R \quad \Longrightarrow \quad \RH,
\]
closing the logical loop initiated in Chapter~\ref{sec:spectral_implications}. This closure is reflected structurally in the DAG (\appref{app:dependency_graph}).

\medskip

\noindent
\textbf{Remarks.}
\begin{itemize}
  \item Simplicity of spectrum is sufficient (but not necessary) for RH.
  \item The determinant–spectrum map arises intrinsically from Fredholm theory and the Hadamard product structure of \( \XiR(s) \).
  \item Positivity of the trace distribution reflects spectral measure regularity and harmonic-analytic structure.
  \item The Fourier transform \( \FT_L \) diagonalizing \( L_{\sym} \) interprets the trace pairing as integration against the spectral measure.
\end{itemize}

\medskip

\noindent
\textbf{Forward Link.}  
The final chapter (\cref{sec:logical_closure}) consolidates all analytic implications into a logically closed equivalence:
\[
\RH \iff \Spec(L_{\sym}) \subset \R.
\]
