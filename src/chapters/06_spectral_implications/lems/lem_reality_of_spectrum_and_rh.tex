\begin{lemma}[Reality of Spectrum Equivalent to RH]
\label{lem:reality-of-spectrum-and-rh}
Let \( L_{\mathrm{sym}} \in \mathcal{C}_1(H_{\Psi_\alpha}) \) be the canonical self-adjoint trace-class operator constructed from the completed Riemann zeta function \( \Xi(s) \). Let \( \rho = \tfrac{1}{2} + i\gamma \) be a nontrivial zero of \( \zeta(s) \), and define the associated eigenvalue:
\[
\mu_\rho := \frac{1}{i(\rho - \tfrac{1}{2})} = \frac{1}{\gamma}.
\]

Then:
\[
\mu_\rho \in \mathbb{R} \quad \Longleftrightarrow \quad \Re(\rho) = \tfrac{1}{2}.
\]

\medskip
\noindent
Consequently, the Riemann Hypothesis holds if and only if the entire spectrum of \( L_{\mathrm{sym}} \) lies on the real line:
\[
\operatorname{Spec}(L_{\mathrm{sym}}) \subset \mathbb{R} \quad \Longleftrightarrow \quad \mathrm{RH}.
\]
\end{lemma}
% 