\begin{proof}[Proof of \lemref{lem:hk_upper_bound}]
Let \( L_{\mathrm{sym}} \in \mathcal{C}_1(H_{\Psi_\alpha}) \), with spectral decomposition \( L_{\mathrm{sym}} e_n = \mu_n e_n \) for an orthonormal basis \( \{ e_n \} \subset H_{\Psi_\alpha} \). The heat trace is given by
\[
\operatorname{Tr}(e^{-t L_{\mathrm{sym}}^2}) = \sum_{n=1}^\infty e^{-t \mu_n^2},
\]
which converges absolutely since \( L_{\mathrm{sym}}^2 \in \mathcal{C}_1 \).

\paragraph{Step 1: Partitioning the Spectrum.}
Fix \( 0 < t \le 1 \), and define the index sets
\[
A_1(t) := \left\{ n : |\mu_n| \le t^{-1/2} \right\}, \quad
A_2(t) := \left\{ n : |\mu_n| > t^{-1/2} \right\}.
\]
Then
\[
\operatorname{Tr}(e^{-t L_{\mathrm{sym}}^2})
= \sum_{n \in A_1(t)} e^{-t \mu_n^2} + \sum_{n \in A_2(t)} e^{-t \mu_n^2}.
\]

\paragraph{Step 2: Estimating Each Sum.}
For \( n \in A_1(t) \), we have \( e^{-t \mu_n^2} \le 1 \), so
\[
\sum_{n \in A_1(t)} e^{-t \mu_n^2} \le |A_1(t)|.
\]
For \( n \in A_2(t) \), \( |\mu_n| > t^{-1/2} \Rightarrow t \mu_n^2 > 1 \Rightarrow e^{-t \mu_n^2} < e^{-1} \), so
\[
\sum_{n \in A_2(t)} e^{-t \mu_n^2} \le e^{-1} \cdot |A_2(t)|.
\]

\paragraph{Step 3: Bounding the Cardinalities.}
Using the trace norm:
\[
\sum_n |\mu_n| = \|L_{\mathrm{sym}}\|_{\mathcal{C}_1},
\]
and \( |\mu_n| \ge t^{-1/2} \) for \( n \in A_2(t) \), so
\[
t^{-1/2} \cdot |A_2(t)| \le \sum_{n \in A_2(t)} |\mu_n| \le \|L_{\mathrm{sym}}\|_{\mathcal{C}_1}.
\]
Hence
\[
|A_2(t)| \le t^{-1/2} \cdot \|L_{\mathrm{sym}}\|_{\mathcal{C}_1}, \qquad
|A_1(t)| \le \|L_{\mathrm{sym}}\|_{\mathcal{C}_1} \cdot t^{-1/2}.
\]

\paragraph{Step 4: Final Estimate.}
Combining gives
\[
\operatorname{Tr}(e^{-t L_{\mathrm{sym}}^2})
\le |A_1(t)| + e^{-1} |A_2(t)|
\le (1 + e^{-1}) \cdot \|L_{\mathrm{sym}}\|_{\mathcal{C}_1} \cdot t^{-1/2}.
\]
Setting \( c_2 := (1 + e^{-1}) \cdot \|L_{\mathrm{sym}}\|_{\mathcal{C}_1} \) completes the proof.
\end{proof}
