\begin{proof}[Proof of \thmref{thm:existence_Lsym}]
Fix \( \alpha > \pi \), and define the exponential weight
\[
\Psi_\alpha(x) := e^{\alpha |x|}, \qquad H_{\Psi_\alpha} := L^2(\R, \Psi_\alpha(x)\, dx).
\]

\paragraph{(i) Trace-Class Structure of \( L_t \).}
By \lemref{lem:trace_class_Lt}, for each \( t > 0 \), the convolution operator
\[
L_t f(x) := \int_{\R} k_t(x - y) f(y)\, dy
\]
lies in \( \TC(H_{\Psi_\alpha}) \), and is compact and self-adjoint. Here \( k_t := \widehat{\varphi_t} \in \Schwartz \), with
\[
\varphi_t(\lambda) := e^{-t\lambda^2} \, \Xi\left( \tfrac{1}{2} + i\lambda \right).
\]

\paragraph{(ii) Trace-Norm Convergence.}
By \lemref{lem:trace_norm_convergence_Lt_to_Lsym}, the family \( \{L_t\}_{t > 0} \subset \TC(H_{\Psi_\alpha}) \) converges in trace norm:
\[
\| L_t - \Lsym \|_{\TC} \to 0 \quad \text{as } t \to 0^+,
\]
for a unique limit \( \Lsym \in \TC(H_{\Psi_\alpha}) \), since \( \TC \) is a Banach ideal.

\paragraph{(iii) Compactness of the Limit.}
Trace-norm convergence implies convergence in operator norm. Since \( \KC(H_{\Psi_\alpha}) \) is norm closed, we obtain:
\[
\Lsym \in \KC(H_{\Psi_\alpha}).
\]

\paragraph{(iv) Self-Adjointness of the Limit.}
By \lemref{lem:core_essential_sa}, the restriction of \( \Lsym \) to \( \Schwartz \subset H_{\Psi_\alpha} \) is essentially self-adjoint, and its closure defines a unique self-adjoint operator:
\[
\Lsym = \Lsym^*.
\]

\paragraph{(v) Trace Normalization.}
By \thmref{thm:trace_zero_Lsym}, the trace vanishes:
\[
\Tr(\Lsym) = 0.
\]
This enforces canonical normalization in the zeta-regularized determinant:
\[
\detz(I - \lambda \Lsym) = \frac{\Xi\left( \tfrac{1}{2} + i\lambda \right)}{\Xi\left( \tfrac{1}{2} \right)},
\]
ensuring that \( \detz(I) = 1 \).

\paragraph{Conclusion.}
The operator \( \Lsym \in \TC(H_{\Psi_\alpha}) \cap \KC(H_{\Psi_\alpha}) \) is self-adjoint with zero trace and arises canonically as the analytic limit of mollified spectral convolution operators. This completes the construction.
\end{proof}
