\section*{Prologue: Structural Roadmap and Arithmetic Spectral Context}
\label{sec:prologue}

\medskip

\noindent
This manuscript constructs a canonical trace-class operator whose \(\zeta\)-regularized Fredholm determinant recovers the completed Riemann zeta function. Through this construction, we establish an analytic equivalence:
\[
\RH \iff \Spec(\Lsym) \subset \R,
\]
reformulating the Riemann Hypothesis in operator-theoretic terms. The argument synthesizes classical heuristics—Hilbert–Pólya, Weil's explicit formula, Selberg’s trace method—into a rigorously defined spectral framework. The structure is modular, acyclic, and analytically complete.

\paragraph*{Scope.}
We construct a compact, self-adjoint, trace-class operator
\[
\Lsym \in \TC(H_{\Psi_\alpha}),
\]
on the exponentially weighted Hilbert space \( H_{\Psi_\alpha} = \PsiAlphaSpace{x} \), with \( \alpha > \pi \). Its zeta-regularized Fredholm determinant satisfies:
\[
\detz(I - \lambda \Lsym) = \frac{\Xi(\tfrac{1}{2} + i\lambda)}{\Xi(\tfrac{1}{2})},
\]
where \( \Xi(s) = \tfrac{1}{2}s(s-1)\pi^{-s/2} \Gamma(s/2) \zeta(s) \) is the completed Riemann zeta function. This identity is established unconditionally, without assuming RH.

The decay threshold \( \alpha > \pi \) is sharp and governs the trace-class inclusion of the inverse Fourier kernel \( \FT^{-1}[\Xi(\tfrac{1}{2} + i\lambda)] \). This constraint is foundational and rigorously proved in \lemref{lem:trace_class_failure_alpha_leq_pi}.

\paragraph*{Arithmetic Context.}
The operator \( \Lsym \) satisfies the Hilbert–Pólya criterion: it is self-adjoint and compact, and its spectrum (under RH) corresponds bijectively to the imaginary parts of the nontrivial zeta zeros. The determinant identity analytically lifts the Euler product to Hilbert space. This construction mirrors the trace formula perspectives of Weil, Deninger, and Connes, but achieves a concrete analytic realization of the arithmetic spectrum.

\paragraph*{Spectral Synthesis.}
The logarithmic derivative of \( \detz(I - \lambda \Lsym) \) recovers a spectral version of Riemann–Weil's explicit formula. Heat trace asymptotics,
\[
\Tr(e^{-t \Lsym^2}) \sim \tfrac{1}{\sqrt{t}} \log(1/t),
\]
lead to the classical counting function \( N(T) \sim \tfrac{T}{2\pi} \log \tfrac{T}{2\pi} \), aligning trace-class spectral theory with the zero distribution of \( \zeta(s) \).

\paragraph*{Historical Context.}
Spectral approaches to RH trace back to Hilbert–Pólya. Weil reframed zeta zeros as spectral eigenvalues through the explicit formula. Selberg established trace identities for Laplacians on modular surfaces. Connes, Deninger, and Berry–Keating introduced connections to noncommutative geometry, arithmetic flows, and quantum chaos. None, however, constructed an explicit trace-class operator whose determinant exactly reproduces \( \Xi(s) \). This manuscript provides such a construction.

\paragraph*{Narrative Architecture.}
The manuscript is organized into ten analytic chapters:

\begin{itemize}
  \item[\textbf{1}] \textbf{Foundations:} Kernel decay, Schatten embeddings, analytic thresholds.
  \item[\textbf{2}] \textbf{Operator Construction:} Mollified convolution and trace-norm limits.
  \item[\textbf{3}] \textbf{Determinant Identity:} Entire structure and canonical normalization.
  \item[\textbf{4}] \textbf{Spectral Correspondence:} Bijection between zeros and spectrum.
  \item[\textbf{5}] \textbf{Heat Trace:} Singular expansion and Laplace analysis.
  \item[\textbf{6}] \textbf{Spectral Equivalence:} \( \RH \iff \Spec(\Lsym) \subset \R \).
  \item[\textbf{7}] \textbf{Tauberian Growth:} Asymptotic density via trace inversion.
  \item[\textbf{8}] \textbf{Spectral Rigidity:} Positivity, uniqueness, and inverse map.
  \item[\textbf{9}] \textbf{Spectral Generalization:} Postulated automorphic extensions.
  \item[\textbf{10}] \textbf{Logical Closure:} Final equivalence and DAG validation.
\end{itemize}

\paragraph*{Appendix Guide.}
The appendices supplement the main chapters:

\begin{itemize}
  \item[\textbf{[A]}] Notation and spectral conventions;
  \item[\textbf{[B]}] DAG and logical infrastructure;
  \item[\textbf{[C–D]}] Kernel construction and trace regularization;
  \item[\textbf{[E–F]}] Numerical simulations and refined asymptotics;
  \item[\textbf{[G–J]}] Speculative extensions, functorial lifts, and physics analogies.
\end{itemize}

\begin{tcolorbox}[colback=gray!3!white, colframe=black!75!white, title={\textbf{Main Equivalence — Spectral RH Reformulation}}]

We construct a canonical trace-class operator \( \Lsym \in \TC(\HPsi) \) on the exponentially weighted Hilbert space
\[
\HPsi := L^2(\mathbb{R}, e^{\alpha|x|} dx), \qquad \alpha > \pi,
\]
whose spectrum encodes the nontrivial zeros of the completed Riemann zeta function \( \Xi(s) \) via the spectral map
\[
\rho \mapsto \mu_\rho := \tfrac{1}{i}(\rho - \tfrac{1}{2}) \in \Spec(\Lsym).
\]
This operator satisfies the canonical determinant identity:
\[
\detz(I - \lambda \Lsym) = \frac{\Xi(\tfrac{1}{2} + i\lambda)}{\Xi(\tfrac{1}{2})},
\]
and defines a bijection between spectral roots and zeta zeros, with multiplicities preserved.

\medskip

\noindent The Riemann Hypothesis is then equivalent to the spectral condition:
\[
\RH \iff \Spec(\Lsym) \subset \R.
\]
All analytic infrastructure—Paley–Wiener kernel decay, trace-norm convergence, semigroup regularity, and determinant growth—are rigorously established across Chapters~1–10. See \thmref{thm:truth_of_rh} for formal closure.
\end{tcolorbox}

\subsection*{Disclosure of Large-Language-Model Assistance}

The author used OpenAI's \emph{ChatGPT} (GPT-4o, May 2025 release) as a \textit{research and writing assistant}. Specifically, the model supported:

\begin{itemize}
  \item Brainstorming outlines, subsection titles, and structural flow;
  \item Drafting initial phrasings for formal statements and transitions, subsequently edited line-by-line by the author;
  \item Generating BibTeX entries, \LaTeX{} macros, and formatting code;
  \item Verifying symbolic calculations (e.g., Fourier identities), independently re-derived for final inclusion;
  \item Composing preliminary versions of technical responses, later rewritten in the author’s voice.
\end{itemize}

This assistance was essential given the author’s severe motor impairments, which would have otherwise prevented the production of a manuscript of this scale and density. All mathematical content was independently verified. The LLM is not an author and bears no responsibility for any claim made herein.
