\begin{lemma}[Hadamard Factorization of \( \Xi\left( \tfrac{1}{2} + i\lambda \right) \)]
\label{lem:hadamard_linear_form}
Let \( \Xi(s) \) denote the completed Riemann zeta function. Then the shifted entire function
\[
\lambda \mapsto \Xi\left( \tfrac{1}{2} + i\lambda \right)
\]
is of order one and genus one, and admits the canonical Hadamard factorization:
\[
\Xi\left( \tfrac{1}{2} + i\lambda \right)
= \Xi\left( \tfrac{1}{2} \right)
\prod_{\rho \ne \frac{1}{2}} \left( 1 - \frac{\lambda}{i(\rho - \frac{1}{2})} \right)
\exp\left( \frac{\lambda}{i(\rho - \frac{1}{2})} \right),
\]
where the product is taken over all nontrivial zeros \( \rho \in \C \) of \( \zeta(s) \), counted with multiplicity (see \lemref{lem:spectrum_zero_bijection} for the spectral correspondence).

\medskip
\noindent
This factorization is canonical in the Hadamard sense: since \( \Xi(s) \) is entire of order one and exponential type \( \pi \) (see \lemref{lem:det_identity_entire_order_one}, \lemref{lem:exact_type_pi}), the genus is one, and the minimal Weierstrass primary factor is
\[
E_1(z) = (1 - z)\exp(z).
\]
The exponential term arises from the genus-one constraint in Hadamard’s theorem~\cite[Ch.~9]{Levin1996EntireLectures}.

\medskip
\noindent
The infinite product converges absolutely and uniformly on compact subsets of \( \C \). Moreover, the functional symmetry \( \Xi(s) = \Xi(1 - s) \) implies
\[
\Xi\left( \tfrac{1}{2} + i\lambda \right) = \Xi\left( \tfrac{1}{2} - i\lambda \right),
\]
so the factorization is even in \( \lambda \), and all spectral roots \( \mu_\rho := \tfrac{1}{i}(\rho - \tfrac{1}{2}) \) appear in symmetric pairs about the origin. This factorization underpins the determinant identity of \thmref{thm:det_identity_revised}.
\end{lemma}
