\begin{proof}[Proof of \lemref{lem:trace_class_Lt}]
Let \( H_{\Psi_\alpha} := L^2(\R, \Psi_\alpha(x)\, dx) \), where \( \Psi_\alpha(x) := e^{\alpha |x|} \), and fix \( \alpha > \pi \). Let \( \phi_t \in \Schwartz \) be the mollified profile, and define the convolution kernel
\[
K_t(x, y) := \frac{1}{2\pi} \int_{\R} e^{i(x - y)\lambda} \phi_t(\lambda)\, d\lambda = k_t(x - y),
\]
with \( k_t \in \Schwartz \) by \lemref{lem:mollified_profile_decay}.

\smallskip
\noindent\textbf{Step 1: Unitary Conjugation.}  
Define the unitary map
\[
U : H_{\Psi_\alpha} \to L^2(\R), \qquad (Uf)(x) := \Psi_\alpha(x)^{1/2} f(x),
\]
with inverse \( U^{-1}(g)(x) := \Psi_\alpha(x)^{-1/2} g(x) \). Let \( \widetilde{L}_t := U L_t U^{-1} \) be the conjugated operator on \( L^2(\R) \), with integral kernel
\[
\widetilde{K}_t(x, y) := \frac{K_t(x, y)}{\sqrt{\Psi_\alpha(x) \Psi_\alpha(y)}}.
\]

\smallskip
\noindent\textbf{Step 2: Trace-Class Criterion.}  
By assumption,
\[
\sup_{0 < t \le 1} \| \widetilde{K}_t \|_{L^1(\R^2)} < \infty.
\]
Simon’s trace-class kernel criterion~\cite[Thm.~4.2]{Simon2005TraceIdeals} implies \( \widetilde{L}_t \in \TC(L^2(\R)) \), with norm bound:
\[
\| \widetilde{L}_t \|_{\TC} \le \| \widetilde{K}_t \|_{L^1(\R^2)}.
\]

\smallskip
\noindent\textbf{Step 3: Transfer to Weighted Space.}  
Since \( L_t = U^{-1} \widetilde{L}_t U \) and \( U \) is unitary, we conclude:
\[
L_t \in \TC(H_{\Psi_\alpha}), \qquad \| L_t \|_{\TC(H_{\Psi_\alpha})} = \| \widetilde{L}_t \|_{\TC(L^2)}.
\]

\smallskip
\noindent\textbf{Step 4: Hilbert–Schmidt Inclusion.}  
Moreover, \lemref{lem:kernel_L2_weighted_bound} ensures:
\[
K_t \in L^2(\R^2, \Psi_\alpha(x) \Psi_\alpha(y)\, dx\, dy),
\]
so \( L_t \in \SC(H_{\Psi_\alpha}) \subset \KC(H_{\Psi_\alpha}) \). This independently confirms compactness and gives a secondary check on trace-class regularity.

\smallskip
\noindent\textbf{Conclusion.}  
The mollified convolution operator \( L_t \in \TC(H_{\Psi_\alpha}) \) for all \( t > 0 \), with uniform trace-norm control on \( (0,1] \). This establishes convergence in \( \TC \) and provides the analytic backbone for the construction of the canonical operator \( L_{\sym} \).
\end{proof}
