\begin{lemma}[Explicit Korevaar Tauberian Bound]
\label{lem:explicit-korevaar-tauberian-bound}
Let \( A(\Lambda) := \#\left\{ n \in \mathbb{N} : \mu_n^2 \le \Lambda \right\} \) be the squared spectral counting function associated with the canonical compact, self-adjoint, trace-class operator \( L_{\mathrm{sym}} \in \mathcal{C}_1(H_{\Psi_\alpha}) \), with nonzero eigenvalues \( \mu_n \in \mathbb{R} \setminus \{0\} \).

Assume the spectral heat trace satisfies the refined singular expansion
\[
\operatorname{Tr}(e^{-t L_{\mathrm{sym}}^2}) = \frac{1}{\sqrt{4\pi t}} \log\left( \frac{1}{t} \right) + o(t^{-1/2}) \qquad \text{as } t \to 0^+.
\]

Then for all sufficiently large \( \Lambda > 0 \), the spectral counting function satisfies the explicit two-sided bound:
\[
\left| A(\Lambda) - \frac{\sqrt{\Lambda}}{\pi} \log \Lambda \right| \le C \sqrt{\Lambda},
\]
for some constant \( C > 0 \) depending only on the trace-norm asymptotics of \( L_{\mathrm{sym}} \) and the uniform bounds on the heat trace expansion.

\medskip
\noindent
This result follows from Korevaar’s remainder-refined Tauberian theorem for Laplace transforms of log-modulated regularly varying functions~\cite[Ch.~III, §5]{Korevaar2004Tauberian}, and gives effective control over the remainder in the spectral counting function relative to the log-corrected Weyl leading term.
\end{lemma}
