\begin{proof}[Proof of \lemref{lem:tauberian_heat_trace_application}]
Assume the spectral heat trace satisfies
\[
\Theta(t) := \Tr(e^{-t L_{\sym}^2}) \sim \frac{1}{\sqrt{4\pi t}} \log\left( \frac{1}{t} \right), \quad \text{as } t \to 0^+.
\]
Let \( N(\lambda) := \#\{ n : \mu_n^2 \le \lambda \} \) be the eigenvalue counting function for \( L_{\sym}^2 \), and define its Laplace transform:
\[
\int_0^\infty e^{-t \lambda} \, dN(\lambda) = \Theta(t).
\]

This Laplace relation links the distribution function \( N(\lambda) \) to the behavior of \( \Theta(t) \) as \( t \to 0^+ \). Since the singular term
\[
\Theta(t) \sim \frac{1}{\sqrt{4\pi t}} \log(1/t)
\]
is slowly varying relative to any \( t^{-\alpha} \) with \( \alpha > 1/2 \), it corresponds under Tauberian inversion to the growth law
\[
N(\lambda) \sim C \, \lambda^{1/2} \log \lambda,
\]
as \( \lambda \to \infty \), for some constant \( C > 0 \). This follows from the extended Karamata–de Haan Tauberian theorem for Laplace transforms (see Korevaar~\cite[Thm.~4.12.9]{Korevaar2004Tauberian}, or BGT~\cite[Thm.~1.7.1]{BGT1989RegularVariation}).

\paragraph{Normalization.}
In the special case where
\[
\Theta(t) \sim \frac{1}{\sqrt{4\pi t}} \log(1/t),
\]
the corresponding inverse asymptotic is
\[
N(\lambda) \sim \sqrt{\lambda} \log \lambda \quad \text{as } \lambda \to \infty,
\]
with multiplicative constant inherited from the Mellin kernel \( t^{s - 1} \). This confirms the log-enhanced Weyl scaling for the squared spectrum of \( L_{\sym} \).

\end{proof}
