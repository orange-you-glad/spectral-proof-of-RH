\section{Summary of Notation}
\label{app:notation-summary}

\noindent
Notation is introduced contextually within each chapter. The following analytic objects and conventions are defined globally and recur throughout the manuscript:

\begin{itemize}
  \item \( H_{\Psi_\alpha} := L^2(\mathbb{R}, e^{\alpha |x|} \, dx) \): exponentially weighted Hilbert space with fixed exponential weight \( \alpha > \pi \).

  \item \( L_{\mathrm{sym}} := \lim_{t \to 0^+} L_t \in \mathcal{C}_1(H_{\Psi_\alpha}) \): the canonical compact, self-adjoint convolution operator in the trace-class \( \mathcal{C}_1 \), defined as the trace-norm limit of mollified approximants \( L_t \) derived from the inverse Fourier transform of the completed zeta function.

  \item \( \phi(\lambda) := \Xi(\tfrac{1}{2} + i\lambda) \): the centered Fourier profile of the completed Riemann zeta function, defining the convolution kernel \( K(x) \) via inverse transform.

  \item \( \mu_\rho := \frac{1}{i}(\rho - \tfrac{1}{2}) \): the canonical spectral image of a nontrivial zero \( \rho \in \mathbb{C} \) of \( \zeta(s) \), used in the bijective encoding of the spectrum of \( L_{\mathrm{sym}} \).

  \item \( \det\nolimits_\zeta(I - \lambda L_{\mathrm{sym}}) := \frac{\Xi(\tfrac{1}{2} + i\lambda)}{\Xi(\tfrac{1}{2})} \): the Carleman \(\zeta\)-regularized Fredholm determinant of \( L_{\mathrm{sym}} \), canonically encoding the spectral zero set.
\end{itemize}

\noindent
The completed Riemann zeta function \( \Xi(s) \) is an entire function of order one and exponential type \( \pi \), satisfying the functional equation \( \Xi(s) = \Xi(1 - s) \). For functional identities, normalizations, and Fourier conventions, see Appendix~\ref{app:reductions-and-conventions}.

\noindent
All remaining notation is introduced locally in the chapters where it is first used. Indexing of key theorems and symbols by chapter is provided in Appendix~\ref{app:dependency-graph}.
