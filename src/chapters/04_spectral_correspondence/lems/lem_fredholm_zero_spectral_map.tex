\begin{lemma}[Fredholm Zeros Correspond to Canonical Spectrum]
\label{lem:fredholm_zero_spectral_map}
Let \( L_{\mathrm{sym}} \in \mathcal{C}_1(H_{\Psi_\alpha}) \) be the canonical compact, self-adjoint operator, and let
\[
f(\lambda) := \det\nolimits_\zeta(I - \lambda L_{\mathrm{sym}})
= \frac{\Xi\left( \tfrac{1}{2} + i\lambda \right)}{\Xi\left( \tfrac{1}{2} \right)}.
\]

Then:
\begin{enumerate}
  \item[\textnormal{(i)}] For every zero \( \rho \ne \tfrac{1}{2} \) of the Riemann zeta function, the value \( \mu_\rho := \frac{1}{i(\rho - \tfrac{1}{2})} \) is an eigenvalue of \( L_{\mathrm{sym}} \), with multiplicity equal to the order of vanishing at \( \rho \).

  \item[\textnormal{(ii)}] Conversely, for every nonzero eigenvalue \( \mu \in \Spec(L_{\mathrm{sym}}) \setminus \{0\} \), there exists a unique \( \rho = \tfrac{1}{2} + \frac{1}{i\mu} \) such that \( \zeta(\rho) = 0 \) and \( \mu = \mu_\rho \).

  \item[\textnormal{(iii)}] The zero set of \( f(\lambda) \) coincides (as a multiset) with \( \{ \lambda_\rho := i(\rho - \tfrac{1}{2}) : \zeta(\rho) = 0 \} \), and the map \( \rho \mapsto \mu_\rho \) defines a bijection between the nontrivial zeros of \( \zeta(s) \) and the nonzero spectrum of \( L_{\mathrm{sym}} \).
\end{enumerate}
\end{lemma}
% 