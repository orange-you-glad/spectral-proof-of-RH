\begin{proof}[Proof of \lemref{lem:grh_spectral_reality}]
Let \( L_{\sym}^{(\pi)} \) be the canonical compact, self-adjoint, trace-class operator associated with the automorphic L-function \( L(s, \pi) \) for \( \pi \in \mathcal{A}_{\text{cusp}}(\mathrm{GL}_n) \). We are given that the spectrum of \( L_{\sym}^{(\pi)} \) is real, i.e., \( \Spec(L_{\sym}^{(\pi)}) \subset \mathbb{R} \).

\paragraph{Step 1: Spectral Reality Implies Eigenvalues on the Critical Line.}
By the spectral theory of self-adjoint operators, the eigenvalues of \( L_{\sym}^{(\pi)} \) are real. The Fredholm determinant for \( L_{\sym}^{(\pi)} \) is related to the zeros of the automorphic L-function \( L(s, \pi) \) through the identity
\[
\det_\zeta(I - \lambda L_{\sym}^{(\pi)}) = \frac{\Xi(\frac{1}{2} + i\lambda)}{\Xi(\frac{1}{2})},
\]
where the nontrivial zeros of \( \Xi(s) \), which correspond to the nontrivial zeros of \( L(s, \pi) \), are the points where the determinant vanishes.

If the spectrum of \( L_{\sym}^{(\pi)} \) is real, the corresponding eigenvalues \( \mu_n \in \mathbb{R} \) must satisfy \( \mu_n = \frac{1}{i}(\rho_n - \frac{1}{2}) \), where \( \rho_n \) are the nontrivial zeros of \( L(s, \pi) \).

\paragraph{Step 2: Critical Line Condition.}
Since the eigenvalues \( \mu_n \) are real, we must have \( \Im(\rho_n) = \pm \frac{1}{2} \). This is precisely the condition that the nontrivial zeros \( \rho_n \) of the automorphic L-function lie on the critical line \( \Re(s) = \frac{1}{2} \).

Thus, the \emph{Generalized Riemann Hypothesis (GRH)} holds for \( L(s, \pi) \).

\paragraph{Conclusion.}
Since the spectral reality of \( L_{\sym}^{(\pi)} \) forces the nontrivial zeros of \( L(s, \pi) \) to lie on the critical line, we conclude that the \emph{Generalized Riemann Hypothesis (GRH)} holds for \( L(s, \pi) \). This completes the proof.
\end{proof}
