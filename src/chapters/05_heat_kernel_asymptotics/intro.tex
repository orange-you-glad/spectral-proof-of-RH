\subsection*{Introduction}

This chapter establishes sharp two-sided bounds and a detailed asymptotic structure for the short-time behavior of the spectral heat trace
\[
\Tr(e^{-t L_{\sym}^2}),
\]
where \( L_{\sym} \in \TC(\HPsi) \) denotes the canonical compact, self-adjoint convolution operator constructed in \secref{sec:operator_construction}. The operator \( L_{\sym} \) generates a strongly continuous, trace-class contraction semigroup \( e^{-tL_{\sym}^2} \in \TC(\HPsi) \cap \mathcal{B}(\HPsi) \) for all \( t > 0 \), defined via the spectral theorem and classical heat kernel calculus (cf.~\cite[Ch.~X]{ReedSimon1978IV}, \cite[Ch.~3]{Simon2005TraceIdeals}).

\paragraph{Two-Sided Heat Trace Bounds.}
We prove the existence of constants \( c_1, c_2 > 0 \) and \( t_0 > 0 \) such that
\[
c_1\, t^{-1/2} \le \Tr(e^{-t L_{\sym}^2}) \le c_2\, t^{-1/2}
\quad \text{for all } 0 < t \le t_0,
\]
uniformly on compact subintervals of \( (0, t_0] \). This scaling reflects the spectral dimension \( d = 1 \) and the inverse-square growth of the generator spectrum, modulated by a logarithmic correction from the kernel's analytic structure.

\paragraph{Analytic Framework.}
The asymptotics are derived from the analytic and operator-theoretic structure of the semigroup and its integral kernel. Core components of the analysis include:
\begin{itemize}
  \item Gaussian mollification and Paley--Wiener decay of the Fourier profile defining \( L_{\sym} \), ensuring Schwartz-class behavior for approximants (see~\cite[Ch.~IX]{ReedSimon1975II}, \cite[Ch.~9]{Levin1996EntireLectures});
  \item Positivity and exponential off-diagonal decay of the heat kernel \( K_t(x, y) \), with diagonal positivity ensured by \lemref{lem:heat_kernel_diagonal_positivity};
  \item Trace-norm convergence of mollified operators \( L_t \to L_{\sym} \) as \( t \to 0^+ \), and heat kernel trace convergence by dominated convergence and compactness;
  \item Pointwise asymptotic expansion of the diagonal kernel:
  \[
  K_t(x,x) \sim \sum_{n=0}^\infty a_n(x)\, t^{n - 1/2},
  \]
  yielding the global trace expansion
  \[
  \Tr(e^{-t L_{\sym}^2}) \sim \sum_{n=0}^\infty A_n t^{n - 1/2},
  \quad A_n := \int_{\R} a_n(x)\,dx.
  \]
  This includes an explicit logarithmic singularity and its refined correction (cf. \lemref{lem:heat_trace_expansion}, \propref{prop:short_time_heat_expansion}).
\end{itemize}

\paragraph{Spectral Class.}
The singular structure of \( \Theta(t) := \Tr(e^{-t L_{\sym}^2}) \) places it in the log-modified regular variation class \( \mathcal{R}_{1/2}^{\log}(0^+) \). This classification is crucial for Tauberian inversion and asymptotic reconstruction of spectral density.

\paragraph{Determinant and Spectral Implications.}
The exact asymptotic behavior
\[
\Tr(e^{-t L_{\sym}^2}) = \frac{\log(1/t)}{\sqrt{4\pi t}} + c_0 \sqrt{t} + o(\sqrt{t})
\]
determines the order-one growth and normalization of the Carleman zeta-regularized Fredholm determinant \( \det\nolimits_{\zeta}(I - \lambda L_{\sym}) \). This connection is made rigorous via the Laplace integral representation and logarithmic derivative identity (see \lemref{lem:log_derivative_determinant}, \propref{prop:short_time_heat_expansion}). The necessity of the leading singular term is grounded in Hadamard product structure and spectral Mellin theory (cf.~\cite[Ch.~III]{Korevaar2004Tauberian}).

\paragraph{Outlook.}
These analytic results prepare the foundation for \secref{sec:tauberian_growth}, where Korevaar’s log-corrected Tauberian theorem (\lemref{lem:log_corrected_tauberian_estimate}) is applied to invert the trace asymptotic and derive the spectral growth law
\[
N(\Lambda) = \frac{\sqrt{\Lambda}}{\pi} \log \Lambda + O(\sqrt{\Lambda}).
\]
This confirms that the spectrum of \( L_{\sym} \) encodes the nontrivial zero distribution of \( \zeta(s) \), completing the analytic–spectral connection initiated in \secref{sec:determinant_identity}.
