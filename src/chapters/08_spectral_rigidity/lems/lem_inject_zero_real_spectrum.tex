\begin{lemma}[Spectral Realization and Rigidity Imply the Riemann Hypothesis]
\label{lem:inject_zero_real_spectrum}

Let \( L_{\sym} \in \TC(\HPsi) \) be the canonical compact, self-adjoint operator on the exponentially weighted Hilbert space
\[
\HPsi := L^2(\R, e^{\alpha |x|} dx), \qquad \alpha > \pi,
\]
with dense domain and discrete spectrum.

Assume:
\begin{enumerate}
  \item[\textup{(i)}] There exists a bijective, multiplicity-preserving correspondence between the nontrivial zeros \( \rho \in \C \) of the Riemann zeta function \( \zetaR(s) \) and the nonzero spectrum of \( L_{\sym} \), via:
  \[
  \rho \mapsto \mu_\rho := \frac{1}{i}(\rho - \tfrac{1}{2}) \in \Spec(L_{\sym}) \setminus \{0\}.
  \]

  \item[\textup{(ii)}] The spectrum \( \Spec(L_{\sym}) \subset \R \) is real and all nonzero eigenvalues are simple.
\end{enumerate}

Then every nontrivial zero \( \rho \) of \( \zetaR(s) \) satisfies the Riemann Hypothesis:
\[
\Re(\rho) = \tfrac{1}{2}.
\]
\end{lemma}
