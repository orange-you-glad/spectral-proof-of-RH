\begin{remark}[No RH Assumption Used in Spectral Encoding]
\label{rem:no_rh_assumption}
Throughout Chapters~\ref{sec:determinant_identity}–\ref{sec:spectral_correspondence}, all constructions, estimates, and determinant identities are established unconditionally—without assuming the Riemann Hypothesis, GUE statistics, or any symmetry of the zeta zero distribution.

\medskip

In particular:
\begin{itemize}
  \item The canonical operator \( \Lsym \in \TC(\HPsi) \) is constructed via mollified convolution and inverse Fourier transform of \( \Xi(s) \), with trace-norm convergence and self-adjointness proven analytically. No assumption on the spectral location of zeta zeros is required.

  \item The determinant identity
  \[
  \det\nolimits_\zeta(I - \lambda \Lsym) = \frac{\XiR\left( \tfrac{1}{2} + i\lambda \right)}{\XiR\left( \tfrac{1}{2} \right)}
  \]
  is derived via Laplace regularization of the heat trace and analytic control of its singularity structure, as in Chapter~\ref{sec:determinant_identity}. No implicit assumption of real spectrum is used in this derivation.

  \item The spectral map \( \rho \mapsto \mu_\rho := \tfrac{1}{i}(\rho - \tfrac{1}{2}) \) and its inverse are defined algebraically and proven to be a bijection using Fredholm theory and Hadamard factorization of entire functions of order one. See \thmref{thm:spectral_zero_bijection_revised}. This spectral encoding is verified independently of RH.

  \item All Schatten norm, kernel decay, and operator compactness results in Chapters~\ref{sec:operator_construction}–\ref{sec:heat_kernel_asymptotics} follow from classical operator theory and Fourier analysis. No probabilistic, arithmetic, or spectral assumptions are made about the zeta zeros.
\end{itemize}

\medskip

\noindent
The Riemann Hypothesis enters for the first time in Chapter~\ref{sec:spectral_implications}, where it is shown to be equivalent to the condition \( \Spec(\Lsym) \subset \R \) using the previously established spectral bijection and determinant identity. The proof is strictly modular and acyclic.
\end{remark}
