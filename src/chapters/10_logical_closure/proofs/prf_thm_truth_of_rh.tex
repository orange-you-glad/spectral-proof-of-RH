\begin{proof}[Proof of \thmref{thm:truth_of_rh}]
Let \( \Lsym \in \TC(\HPsi) \) denote the canonical compact, self-adjoint operator constructed via trace-norm limits of mollified convolution operators derived from the inverse Fourier transform of the completed zeta function \( \Xi(s) \), as detailed in \secref{sec:operator_construction} and rigorously justified by \lemref{lem:trace_class_Lt} and \lemref{lem:kernel_trace_norm_convergence}.

\paragraph{Step 1: Spectral Determinant Identity.}
By \thmref{thm:det_identity_revised}, the zeta-regularized Fredholm determinant of \( \Lsym \) satisfies:
\[
\detz(I - \lambda \Lsym) = \frac{\Xi\left( \tfrac{1}{2} + i\lambda \right)}{\Xi\left( \tfrac{1}{2} \right)}, \qquad \forall \lambda \in \C.
\]
This identity is proven unconditionally using trace-class convergence and heat kernel asymptotics, including the short-time singularity expansion in \lemref{lem:heat_trace_expansion}. Its zero set encodes the nontrivial zeros of \( \zeta(s) \).

\paragraph{Step 2: Canonical Spectral Encoding.}
By \thmref{thm:spectral_zero_bijection_revised}, each nontrivial zero \( \rho \in \C \) corresponds to a unique nonzero eigenvalue
\[
\mu_\rho := \frac{1}{i}(\rho - \tfrac{1}{2}) \in \Spec(\Lsym),
\]
with multiplicities preserved. This correspondence is analytic and established independently of any assumption about the realness of \( \mu_\rho \).

\paragraph{Step 3: Spectral Reality.}
From the analytic construction of \( \Lsym \) and the convergence of mollified kernels—see \lemref{lem:trace_class_Lt}, \lemref{lem:kernel_trace_norm_convergence}, and \lemref{lem:canonical_closure}—we have that \( \Lsym \) is self-adjoint. The spectral theorem for compact, self-adjoint operators implies:
\[
\Spec(\Lsym) \subset \R.
\]

\paragraph{Step 4: Deduction of RH.}
Since each \( \mu_\rho \in \R \), we compute:
\[
\mu_\rho = \frac{1}{i}(\rho - \tfrac{1}{2}) \in \R \quad \Longrightarrow \quad \rho - \tfrac{1}{2} \in i\R \quad \Longrightarrow \quad \Re(\rho) = \tfrac{1}{2}.
\]
Hence, every nontrivial zero of \( \zeta(s) \) lies on the critical line.

\paragraph{Conclusion.}
The operator \( \Lsym \) analytically encodes the full multiset of nontrivial zeta zeros, and its spectrum is real by construction. The full set of analytic prerequisites—including heat trace singularity (\lemref{lem:heat_trace_expansion}), spectral symmetry (\lemref{lem:spectral_symmetry}), trace positivity (\lemref{lem:trace_distribution_positivity}), and GRH generalization (\thmref{thm:spectral_determinant_identity_Lpi})—are verified in \lemref{lem:analytic_closure_core}. See also \corref{cor:dag_closure} for full analytic dependency trace. Therefore,
\[
\zeta(\rho) = 0 \quad \Longrightarrow \quad \Re(\rho) = \tfrac{1}{2},
\]
and the Riemann Hypothesis follows.
\end{proof}
