\begin{proof}[Proof of \lemref{lem:weighted_trace_norm_duality}]
Fix \( \alpha > \pi \), and define \( \Psi_\alpha(x) := e^{\alpha |x|} \). Let \( k \in L^1(\R, \Psi_\alpha(x)\, dx) \) be real-valued and even, and define the convolution kernel
\[
K(x,y) := k(x - y),
\]
with associated convolution operator
\[
(L f)(x) := \int_{\R} k(x - y)\, f(y)\, dy.
\]

\medskip
\noindent\textbf{(i) Weighted Kernel Norm Factorization.}
We compute:
\begin{align*}
\iint_{\R^2} |K(x,y)|\, \Psi_\alpha(x)\Psi_\alpha(y)\, dx\, dy
&= \iint_{\R^2} |k(x - y)|\, \Psi_\alpha(x)\Psi_\alpha(y)\, dx\, dy.
\end{align*}
Make the change of variables \( u := x - y \), \( v := y \), so that \( x = u + v \) and \( dx\,dy = du\,dv \). Then:
\[
= \int_{\R} |k(u)| \left( \int_{\R} \Psi_\alpha(u + v)\Psi_\alpha(v)\, dv \right) du.
\]
Using symmetry and convexity of \( \Psi_\alpha \), we observe:
\[
\Psi_\alpha(u + v)\Psi_\alpha(v) = e^{\alpha(|u + v| + |v|)} = e^{\alpha|u|} \cdot e^{2\alpha|v|}.
\]
Hence,
\[
\int_{\R} \Psi_\alpha(u + v)\Psi_\alpha(v)\, dv = \Psi_\alpha(u) \cdot \int_{\R} e^{2\alpha |v|}\, dv = \Psi_\alpha(u) \cdot \|\Psi_\alpha\|_{L^1(\R)}.
\]
Therefore,
\[
\iint_{\R^2} |K(x,y)|\, \Psi_\alpha(x)\Psi_\alpha(y)\, dx\, dy
= \|k\|_{L^1(\R, \Psi_\alpha)} \cdot \|\Psi_\alpha\|_{L^1(\R)}.
\]

\medskip
\noindent\textbf{(ii) Trace-Class Bound.}
Since \( K \in \mathcal{C}_1(\Psi_\alpha) \), the associated integral operator \( L \) lies in \( \mathcal{C}_1(\HPsi) \) by Simon’s kernel criterion \cite[Thm.~4.2]{Simon2005TraceIdeals}. Moreover,
\[
\|L\|_{\mathcal{C}_1(\HPsi)} \le \|K\|_{\mathcal{C}_1(\Psi_\alpha)}
= \|k\|_{L^1(\R, \Psi_\alpha)} \cdot \|\Psi_\alpha\|_{L^1(\R)}.
\]

\medskip
\noindent\textbf{Conclusion.}
This completes the proof of both claims, establishing an explicit factorized relationship between 1D weighted kernel integrability and 2D trace-norm control in \( \HPsi \).
\end{proof}
