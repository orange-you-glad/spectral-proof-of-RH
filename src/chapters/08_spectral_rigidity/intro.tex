\subsection*{Introduction}
\label{sec:intro_spectral_rigidity}

This chapter recasts the Riemann Hypothesis as a statement of spectral rigidity for the canonical trace-class operator
\[
\Lsym \in \TC(\HPsi),
\]
constructed via mollified convolution from the inverse Fourier transform of the completed zeta function \( \XiR(s) \). The central claim is that the spectrum of \( \Lsym \) is real if and only if all nontrivial zeros of \( \zetaR(s) \) lie on the critical line \( \Re(s) = \tfrac{1}{2} \).

\paragraph{Objectives.}
\begin{itemize}
  \item \textit{Spectral Encoding:}  
  The canonical determinant identity
  \[
  \detz(I - \lambda \Lsym) = \frac{\XiR\left(\tfrac{1}{2} + i\lambda\right)}{\XiR\left(\tfrac{1}{2}\right)}
  \]
  defines a multiplicity-preserving map
  \[
  \rho \mapsto \mu_\rho := \tfrac{1}{i}(\rho - \tfrac{1}{2}),
  \]
  sending each nontrivial zeta zero \( \rho \) to an eigenvalue \( \mu_\rho \in \Spec(\Lsym) \). This intertwines the Hadamard factorization of \( \XiR(s) \) with the spectral structure of \( \Lsym \), confirming that each determinant zero yields a spectral eigenvalue.

  \item \textit{Spectral Rigidity:}  
  Although \( \Lsym \) is self-adjoint and thus has real spectrum, we prove the converse: if all eigenvalues \( \mu_\rho \in \R \), then all corresponding zeros \( \rho \) must lie on the critical line. That is,
  \[
  \Spec(\Lsym) \subset \R
  \quad \Longleftrightarrow \quad
  \Re(\rho) = \tfrac{1}{2} \quad \forall \rho \in \Spec(\zetaR).
  \]
  The implication is established via analytic continuation, determinant structure, and Fredholm theory—without presuming spectral bijection.

  \item \textit{Spectral Symmetry:}  
  The functional equation \( \XiR(\tfrac{1}{2} + i\lambda) = \XiR(\tfrac{1}{2} - i\lambda) \) implies spectral symmetry:
  \[
  \mu \in \Spec(\Lsym) \quad \Rightarrow \quad -\mu \in \Spec(\Lsym),
  \]
  with matched multiplicities. This reflects the evenness of the centered spectral profile \( \phi(\lambda) := \XiR(\tfrac{1}{2} + i\lambda) \).

  \item \textit{Trace Positivity:}  
  The spectral trace pairing
  \[
  \phi \mapsto \Tr(\phi(\Lsym))
  \]
  defines a positive tempered distribution on \( \R \). In particular,
  \[
  \Tr(e^{-t \Lsym^2}) \ge 0 \quad \forall\, t > 0,
  \]
  expressing the positivity of the heat trace and its interpretation as a regularized spectral measure. This positivity holds for all \( \phi \in \Schwartz(\R) \), reinforcing the harmonic-analytic structure of the trace.

  \item \textit{Analytic Independence:}  
  All results in this chapter are derived using classical analytic and operator-theoretic tools:
  \begin{itemize}
    \item spectral theory of compact self-adjoint operators;
    \item exponential kernel decay and heat semigroup regularity;
    \item Hadamard factorization and functional symmetry of \( \XiR(s) \);
    \item Fredholm determinant theory and spectrum–zero correspondence;
    \item uniqueness via unitary equivalence and diagonalization.
  \end{itemize}
  No use is made of automorphic forms, trace formulas, or Langlands theory.
\end{itemize}
