\begin{proof}[Proof of \lemref{lem:xi_growth_bound}]
Let \( s := \tfrac{1}{2} + i\lambda \), and recall the representation
\[
\Xi(s) := \tfrac{1}{2}s(s-1)\pi^{-s/2} \Gamma\left( \tfrac{s}{2} \right) \zeta(s),
\]
which defines an entire function of order one and exponential type \( \pi \), satisfying the functional equation \( \Xi(s) = \Xi(1 - s) \).

\medskip
\noindent\textbf{Step 1: Gamma Term Estimate.}
Set \( z := \tfrac{s}{2} = \tfrac{1}{4} + \tfrac{i\lambda}{2} \). By Stirling’s bound for \( \Gamma(z) \) in vertical strips (see \cite[Eq.~(1.5.3)]{Titchmarsh1986Zeta}), there exists \( C_1 > 0 \) such that
\[
|\Gamma(z)| \le C_1 (1 + |\lambda|)^{-1/2} e^{\pi |\lambda| / 4}, \qquad \forall \lambda \in \R.
\]

\medskip
\noindent\textbf{Step 2: Remaining Factors.}
We estimate:
\begin{align*}
|s(s - 1)| &= \left| \left( \tfrac{1}{2} + i\lambda \right)\left( -\tfrac{1}{2} + i\lambda \right) \right| = \tfrac{1}{4} + \lambda^2, \\
|\pi^{-s/2}| &= \pi^{-\Re(s)/2} = \pi^{-1/4}, \\
|\zeta(s)| &\le C_2 \log(2 + |\lambda|), \qquad \text{for } \Re(s) = \tfrac{1}{2},
\end{align*}
for some constant \( C_2 > 0 \), using convexity bounds for \( \zeta(s) \) on the critical line.

\medskip
\noindent\textbf{Step 3: Real Axis Growth.}
Combining the above, we obtain
\[
|\Xi(s)| \le C_3\, (1 + \lambda^2) \cdot (1 + |\lambda|)^{-1/2} \cdot \log(2 + |\lambda|) \cdot e^{\pi |\lambda| / 4},
\]
for some \( C_3 > 0 \). All algebraic and logarithmic terms are subexponential, so we absorb them into a constant \( A_1 > 0 \) and write:
\[
|\phi(\lambda)| = \left| \Xi\left( \tfrac{1}{2} + i\lambda \right) \right| \le A_1\, e^{\tfrac{\pi}{2} |\lambda|},
\]
establishing part (ii) of the lemma.

\medskip
\noindent\textbf{Step 4: Global Complex Growth.}
Since \( \Xi(s) \) is entire of order one and exponential type \( \pi \), Hadamard factorization and Phragmén--Lindelöf bounds imply (see \cite[Ch.~3]{Levin1996EntireLectures}):
\[
|\Xi(s)| \le A\, e^{\pi |s|}, \qquad \forall\, s \in \C,
\]
for some constant \( A > 0 \). Setting \( s = \tfrac{1}{2} + i\lambda \), we obtain
\[
|\phi(\lambda)| = \left| \Xi\left( \tfrac{1}{2} + i\lambda \right) \right| \le A\, e^{\pi |\lambda|},
\]
completing part (i) of the lemma.

\medskip
\noindent\textbf{Conclusion.}
The centered profile \( \phi(\lambda) := \Xi\left(\tfrac{1}{2} + i\lambda\right) \) satisfies:
\[
|\phi(\lambda)| \le A_1\, e^{\tfrac{\pi}{2}|\lambda|} \quad \text{on } \R, \qquad |\phi(\lambda)| \le A\, e^{\pi|\lambda|} \quad \text{on } \C.
\]
Thus \( \phi \in PW_\pi(\R) \), and the mollified profiles \( \phi_t(\lambda) := e^{-t\lambda^2} \phi(\lambda) \) lie in \( \Schwartz \), with exponential spatial decay of their Fourier transforms \( k_t := \FT^{-1}[\phi_t] \), as needed in \lemref{lem:decay_mollified_kernel}.
\end{proof}
