\begin{lemma}[Hadamard Factorization of \( \Xi\left(\tfrac{1}{2} + i\lambda\right) \)]
\label{lem:hadamard_linear_form}
Let \( \Xi(s) \) denote the completed Riemann zeta function. Then the shifted entire function
\[
\lambda \mapsto \Xi\left( \tfrac{1}{2} + i\lambda \right)
\]
is of order one and genus one, and admits the canonical Hadamard factorization:
\[
\Xi\left( \tfrac{1}{2} + i\lambda \right)
= \Xi\left( \tfrac{1}{2} \right)
\prod_{\rho \ne \tfrac{1}{2}} \left( 1 - \frac{\lambda}{i(\rho - \tfrac{1}{2})} \right)
\exp\left( \frac{\lambda}{i(\rho - \tfrac{1}{2})} \right),
\]
where the product is taken over all nontrivial zeros \( \rho \in \mathbb{C} \) of \( \zeta(s) \), counted with multiplicity.

\medskip
\noindent
This factorization is canonical in the Hadamard sense: since \( \Xi(s) \) is entire of order one, the associated genus is also one, and the minimal Weierstrass primary factor is of the form
\[
E_1(z) = (1 - z) \exp(z).
\]
The exponential terms arise from this genus-one constraint.

\medskip
\noindent
The product converges absolutely and uniformly on compact subsets of \( \mathbb{C} \), and the symmetry \( \Xi(s) = \Xi(1 - s) \) implies
\[
\Xi\left( \tfrac{1}{2} + i\lambda \right) = \Xi\left( \tfrac{1}{2} - i\lambda \right),
\]
so the factorization is invariant under \( \lambda \mapsto -\lambda \), and all zeros appear in symmetric pairs about the origin.
\end{lemma}
%  