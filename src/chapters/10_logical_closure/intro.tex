\subsection*{Closure of the Spectral Program}
\label{subsec:closure_of_spectral_program}

This chapter concludes the analytic reformulation of the Riemann Hypothesis as a statement of spectral rigidity. We synthesize the spectral, determinant, and trace-theoretic constructions from preceding chapters into a logically acyclic, formally closed equivalence.

\medskip
\noindent
The framework proceeds through the following modular components:

\begin{itemize}
  \item \textbf{Canonical Operator Construction.}  
  A compact, self-adjoint, trace-class operator
  \[
  L_{\sym} \in \TC(\HPsi)
  \]
  is constructed as the trace-norm limit of mollified symmetric convolution operators with kernels derived from the inverse Fourier transform of \( \XiR(s) \). Gaussian decay ensures convergence in Schatten norm.

  \item \textbf{Determinant Identity.}  
  The Carleman \(\zeta\)-regularized Fredholm determinant satisfies
  \[
  \det\nolimits_\zeta(I - \lambda L_{\sym}) = \frac{\XiR\left(\tfrac{1}{2} + i\lambda\right)}{\XiR\left(\tfrac{1}{2}\right)},
  \]
  for all \( \lambda \in \C \). This identity transfers the zero structure of \( \XiR(s) \) to the spectral data of \( L_{\sym} \).

  \item \textbf{Spectral Multiplicity Matching.}  
  The Hadamard factorization of \( \XiR(s) \) ensures that
  \[
  \operatorname{ord}_{\rho} \zetaR = \operatorname{mult}_{\mu_\rho}(L_{\sym}),
  \quad \text{with} \quad \mu_\rho := \frac{1}{i}(\rho - \tfrac{1}{2}).
  \]
  Thus, the determinant encodes both the location and multiplicity of the nontrivial zeros.

  \item \textbf{Spectral Realization.}  
  Every nontrivial zero \( \rho \) of \( \zetaR(s) \) corresponds to a nonzero eigenvalue
  \[
  \mu_\rho := \frac{1}{i}(\rho - \tfrac{1}{2}) \in \Spec(L_{\sym}),
  \]
  as shown in \lemref{lem:spectral_zero_encoding}.

  \item \textbf{Spectral Symmetry.}  
  The identity \( \XiR(\tfrac{1}{2} + i\lambda) = \XiR(\tfrac{1}{2} - i\lambda) \) implies:
  \[
  \mu \in \Spec(L_{\sym}) \quad \Rightarrow \quad -\mu \in \Spec(L_{\sym}),
  \]
  with equal multiplicities (see \lemref{lem:spectral_symmetry}).

  \item \textbf{Spectral Rigidity and Logical Equivalence.}  
  The condition
  \[
  \Spec(L_{\sym}) \subset \R
  \]
  is logically equivalent to the Riemann Hypothesis:
  \[
  \RH \iff \Re(\rho) = \tfrac{1}{2} \;\text{for all } \rho \in \Spec(\zetaR)
       \iff \mu_\rho \in \R \;\text{for all } \mu_\rho \in \Spec(L_{\sym}).
  \]
  This equivalence is proven in \thmref{thm:rh_spectrum_equiv}, without assuming spectral simplicity.

  \item \textbf{Spectral Completeness.}  
  The spectrum of \( L_{\sym} \) fully determines the nontrivial zero set of \( \zetaR(s) \), as shown in \corref{cor:spectral_determines_zeta}. This yields a canonical trace-class realization of the critical line.

  \item \textbf{Trace Positivity and Functional Calculus.}  
  The trace pairing
  \[
  \phi \mapsto \Tr(\phi(L_{\sym}))
  \]
  defines a positive tempered distribution on \( \R \) for all nonnegative \( \phi \in \Schwartz(\R) \) (see \lemref{lem:trace_distribution_positivity}, \remref{rem:functional_calculus_trace}). This confirms compatibility of the determinant identity with positivity and trace regularization.

  \item \textbf{Analytic Closure.}  
  The RH equivalence is derived entirely from:
  \begin{itemize}
    \item spectral theory for compact self-adjoint operators,
    \item short-time trace asymptotics and Tauberian inversion,
    \item Hadamard factorization and exponential type of \( \XiR(s) \),
    \item determinant theory and distributional positivity.
  \end{itemize}
  No input is required from modular forms, automorphic representations, or arithmetic conjectures.
\end{itemize}

\medskip
\noindent
The complete analytic dependency graph is diagrammed in Appendix~\ref{app:dependency_graph} (see Figure~\ref{fig:dag_appendix_b}) and confirms full acyclic derivation of the RH equivalence. The main theorem that follows formalizes this as a canonical spectral proof.
