\section{Functorial Extensions Beyond \texorpdfstring{$\mathrm{GL}_n$}{GLn}}
\label{app:functorial_extensions}

\noindent\textbf{[Noncritical Appendix.]}  
This appendix is analytically and logically independent of the proof architecture presented in Chapters~1–10. It no longer motivates constructions used in the main body. Instead, it outlines prospective generalizations of the canonical spectral determinant framework to a broader class of global \( L \)-functions arising in arithmetic geometry.

%--------------------------------------------------------------
\subsection*{Beyond \texorpdfstring{$\mathrm{GL}_n$}{GLn}: Artin and Motivic \( L \)-Functions}

Let \( \Lambda_\pi(s) \) denote the completed \( L \)-function associated to a motive over \( \Q \), a Galois representation, or a nonstandard automorphic representation of a reductive group \( G \). Assume:

\begin{itemize}
  \item[(i)] \( \Lambda_\pi(s) \) is entire of order one;
  \item[(ii)] \( \Lambda_\pi(s) = \varepsilon_\pi \Lambda_\pi(1 - s) \), with \( |\varepsilon_\pi| = 1 \);
  \item[(iii)] \( \Lambda_\pi(s) \) admits a genus-one Hadamard factorization.
\end{itemize}

These conditions are expected to hold for Artin \( L \)-functions, symmetric powers of modular forms, Hasse–Weil \( L \)-functions of curves and varieties, and Langlands–Shahidi \( L \)-functions for classical groups~\cite{Langlands1970EulerProducts, Deligne1971WeilI}.

\medskip
\noindent\textbf{Examples.}
\begin{center}
\renewcommand{\arraystretch}{1.3}
\begin{tabular}{|c|c|c|}
\hline
\textbf{Object \( \pi \)} & \textbf{\( \Lambda_\pi(s) \)} & \textbf{Source} \\
\hline
Modular form on \( \Gamma_0(N) \) & Hecke \( L \)-function & \cite{Cogdell2007Lectures} \\
Elliptic curve over \( \Q \) & Hasse–Weil \( L \)-function & \cite{Deligne1971WeilI} \\
Artin representation & Artin \( L \)-function & \cite{Langlands1970EulerProducts} \\
\( \mathrm{GSp}_{2n} \) cusp form & Standard \( L \)-function & conjectural \\
\hline
\end{tabular}
\end{center}

\paragraph{Proposed Generalization.}
To extend the spectral construction, define a Hilbert space
\[
H_{\Psi_\pi} := L^2(\R, w_\pi(x) dx)
\]
for a weight function \( w_\pi(x) \gtrsim |\Lambda_\pi(\tfrac{1}{2} + ix)|^2 \), and construct mollified Fourier profiles
\[
\varphi_{t,\pi}(\lambda) := e^{-t\lambda^2} \Lambda_\pi\left( \tfrac{1}{2} + i\lambda \right),
\quad
K_t^{(\pi)} := \FT^{-1}[\varphi_{t,\pi}],
\]
analogously to the construction in \secref{sec:spectral_generalization}.

%--------------------------------------------------------------
\subsection*{Extension Hypothesis}

\textit{Hypothesis.} Suppose \( \Lambda_\pi(s) \) satisfies (i)--(iii), and that
\[
K_t^{(\pi)}(x) \in L^1(\R, e^{\alpha |x|} dx) \quad \text{for some } \alpha > 0.
\]
Then the operator
\[
L_{\sym,\pi} := \lim_{t \to 0^+} \int_{\R} K_t^{(\pi)}(x - y) f(y) dy
\]
defines a compact, self-adjoint, trace-class operator on \( H_{\Psi_\pi} \), with determinant identity:
\[
\detz(I - \lambda L_{\sym,\pi}) = \frac{\Lambda_\pi(\tfrac{1}{2} + i\lambda)}{\Lambda_\pi(\tfrac{1}{2})}.
\]

This generalizes the validated construction from Chapter~\ref{sec:spectral_generalization}, and can be subjected to the same analytic audit framework.

%--------------------------------------------------------------
\subsection*{Research Directions}

\begin{itemize}
  \item[(1)] \textbf{Artin \( L \)-functions:} As non-automorphic objects, these require direct control over Fourier decay from Galois-theoretic data.
  \item[(2)] \textbf{Beyond \( \GL_n \):} Extending the spectral determinant framework to \( \mathrm{GSp}_{2n} \), \( \mathrm{SO}_n \), or \( \mathrm{U}_n \) requires new kernel bounds and weight embeddings.
  \item[(3)] \textbf{Motivic Scaling:} Shift \( s \mapsto s + \tfrac{w}{2} \) in cohomological \( L \)-functions must be reflected in the spectral normalization.
  \item[(4)] \textbf{Functoriality of Operators:} Is the spectral operator for \( \mathrm{Sym}^k(\pi) \) algebraically related to that for \( \pi \)? If so, what is the induced map on determinants?
\end{itemize}

%--------------------------------------------------------------
\subsection*{Conclusion}

Having validated the spectral determinant identity for \( \GL_n \)-automorphic representations in Chapter~\ref{sec:spectral_generalization}, this appendix outlines next-stage extensions. These include Artin and motivic \( L \)-functions, and generalization to groups beyond \( \GL_n \). Each case invites its own kernel construction, weight analysis, and trace-class audit.

\medskip

\noindent
These extensions do not affect the completed proof of the Riemann Hypothesis (\cref{sec:logical_closure}), but delineate a broader landscape in which the spectral paradigm may continue to propagate across arithmetic geometry.
