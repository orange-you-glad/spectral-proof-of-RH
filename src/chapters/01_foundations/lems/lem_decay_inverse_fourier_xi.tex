\begin{lemma}[Exponential Decay of the Inverse Fourier Transform of \( \Xi \)]
\label{lem:decay_inverse_fourier_xi}
Let \( \phi(\lambda) := \Xi\left( \tfrac{1}{2} + i\lambda \right) \), where \( \Xi(s) \) is the completed Riemann zeta function. Define the inverse Fourier transform
\[
k(x) := \widehat{\phi}(x) := \frac{1}{2\pi} \int_{\R} e^{i\lambda x} \phi(\lambda)\, d\lambda.
\]

Then \( k \in C^\infty(\R) \) is real-valued, even, and satisfies the exponential decay estimate:
\[
|k(x)| \le C_\alpha\, e^{-\alpha |x|}, \qquad \forall\, x \in \R, \quad \text{for any } \alpha > \pi,
\]
where \( C_\alpha > 0 \) depends only on \( \alpha \).

\medskip
\noindent In particular:
\begin{itemize}
  \item \( k \in L^1(\R, \Psi_\alpha(x)\, dx) \), where \( \Psi_\alpha(x) := e^{\alpha |x|} \);
  \item The kernel \( K(x,y) := k(x - y) \) lies in \( L^1(\R^2, \Psi_\alpha(x)\Psi_\alpha(y)\, dx\, dy) \);
  \item The convolution operator
  \[
  (Lf)(x) := \int_{\R} k(x - y) f(y)\, dy
  \]
  belongs to the trace class \( \TC(\HPsi) \), by Simon’s kernel criterion \cite[Thm.~4.2]{Simon2005TraceIdeals}.
\end{itemize}

\medskip
\noindent
This decay follows from the Paley--Wiener theorem: since \( \phi \in \PW{\pi} \) (see \cref{def:paley_wiener_class}), its inverse Fourier transform \( k(x) \) decays faster than \( e^{-\alpha |x|} \) for every \( \alpha > \pi \). The result quantifies the optimal spatial localization of Paley--Wiener kernels in \( \HPsi \).
\end{lemma}
