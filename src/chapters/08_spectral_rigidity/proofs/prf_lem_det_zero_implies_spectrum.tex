\begin{proof}[Proof of \lemref{lem:det_zero_implies_spectrum}]
Let \( T \in \TC(H) \) be compact and self-adjoint. Then its spectrum consists of a discrete set of real eigenvalues \( \{ \mu_n \} \subset \R \setminus \{0\} \), with \( \mu_n \to 0 \), counted with multiplicity.

The Carleman \(\zeta\)-regularized determinant admits the canonical Hadamard product:
\[
\det\nolimits_\zeta(I - \lambda T) = \prod_{n=1}^\infty (1 - \lambda \mu_n),
\]
which converges absolutely on compact subsets of \( \C \), since \( T \in \TC(H) \Rightarrow \sum |\mu_n| < \infty \).

\paragraph{Step 1: Determinant Zero Implies Reciprocal Eigenvalue.}
Suppose
\[
\det\nolimits_\zeta(I - \lambda_0 T) = 0, \qquad \lambda_0 \in \C \setminus \{0\}.
\]
Then for some index \( n \), we have:
\[
1 - \lambda_0 \mu_n = 0 \quad \Longrightarrow \quad \lambda_0 = \mu_n^{-1}.
\]

\paragraph{Step 2: Spectrum Inclusion.}
Thus,
\[
\lambda_0^{-1} = \mu_n \in \Spec(T),
\quad \text{and} \quad \lambda_0 \in \Spec(T)^{-1}.
\]

\paragraph{Conclusion.}
Every nonzero zero of the determinant corresponds to a nonzero eigenvalue of \( T \), and:
\[
\lambda \in \C, \quad \det\nolimits_\zeta(I - \lambda T) = 0
\quad \Longrightarrow \quad
\lambda^{-1} \in \Spec(T).
\]
This completes the proof.
\end{proof}
