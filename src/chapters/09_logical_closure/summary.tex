\subsection*{Summary}

The Riemann Hypothesis holds as a formal consequence of the spectral reality of the canonical operator \( L_{\sym} \in \TC(\HPsi) \), constructed as the trace-norm limit of symmetric mollified convolution operators derived from the inverse Fourier transform of the completed zeta function \( \XiR(s) \).

Through kernel decay estimates, Schatten convergence, and parity symmetry, we established that \( L_{\sym} \) is compact, self-adjoint, and trace class. Its spectrum encodes the nontrivial zeros of \( \zetaR(s) \) via the canonical determinant identity:
\[
\det\nolimits_\zeta(I - \lambda L_{\sym})
= \frac{\XiR\left(\tfrac{1}{2} + i\lambda\right)}{\XiR\left(\tfrac{1}{2}\right)}.
\]
This identity defines a multiplicity-preserving spectral correspondence:
\[
\rho \mapsto \mu_\rho := \frac{1}{i(\rho - \tfrac{1}{2})},
\quad \mu_\rho \in \Spec(L_{\sym}),
\]
ensured by the Hadamard structure of \( \XiR(s) \).

\medskip

\noindent
We proved the equivalence:
\[
\boxed{
\text{RH is true} \quad \Longleftrightarrow \quad \Spec(L_{\sym}) \subset \R
}
\]
and verified the spectral reality condition unconditionally, using:
\begin{itemize}
  \item Compactness and self-adjointness of \( L_{\sym} \);
  \item Positivity and boundedness of the heat semigroup \( e^{-tL_{\sym}^2} \);
  \item Laplace convergence and growth control of \( \zeta \)-regularized determinants;
  \item Positivity of the spectral trace functional \( \phi \mapsto \Tr(\phi(L_{\sym})) \).
\end{itemize}

\medskip

\noindent
Every step in the proof is modular, analytically self-contained, and supported by an acyclic dependency structure detailed in \appref{app:dependency_graph}. No part of the argument relies on modular forms, automorphic representations, or arithmetic conjectures.

This chapter completes the canonical spectral resolution of the Riemann Hypothesis: the nontrivial zeros of \( \zetaR(s) \) lie on the critical line as a consequence of the analytic and spectral structure of \( L_{\sym} \).

\medskip

\noindent
Further analytic refinements—including higher-order heat kernel expansions, Tauberian corrections, and spectral models for automorphic \( L \)-functions—are explored in \appref{app:heat_kernel_refinements}, \appref{app:functorial_extensions}, and \appref{app:additional_structures}.
