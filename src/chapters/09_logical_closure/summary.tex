\subsection*{Summary}

The Riemann Hypothesis holds as a formal consequence of the spectral reality of the canonical operator \( L_{\mathrm{sym}} \in \mathcal{C}_1(H_{\Psi_\alpha}) \), constructed as the trace-norm limit of a family of symmetric mollified convolution operators derived from the inverse Fourier transform of the completed zeta function \( \Xi(s) \).

Through analytic kernel estimates, Schatten convergence, and symmetry preservation, we established that \( L_{\mathrm{sym}} \) is compact, self-adjoint, and trace-class, with spectrum encoding the nontrivial zeros of \( \zeta(s) \) via the canonical determinant identity:
\[
\det\nolimits_\zeta(I - \lambda L_{\mathrm{sym}})
= \frac{\Xi\left(\tfrac{1}{2} + i\lambda\right)}{\Xi\left(\tfrac{1}{2}\right)}.
\]
This identity defines a multiplicity-preserving correspondence between zeta zeros and spectral points:
\[
\rho \mapsto \mu_\rho := \frac{1}{i(\rho - \tfrac{1}{2})},
\quad \mu_\rho \in \operatorname{Spec}(L_{\mathrm{sym}}),
\]
with multiplicity matching ensured by Hadamard factorization.

\medskip
\noindent
We proved the equivalence:
\[
\mathrm{RH} \quad \Longleftrightarrow \quad \operatorname{Spec}(L_{\mathrm{sym}}) \subset \mathbb{R},
\]
and verified the right-hand condition unconditionally from analytic principles:
\begin{itemize}
  \item Boundedness and positivity of the heat semigroup \( e^{-tL_{\mathrm{sym}}^2} \),
  \item Convergence of Carleman \(\zeta\)-regularized Fredholm determinants,
  \item Positivity of the spectral trace distribution \( \phi \mapsto \operatorname{Tr}(\phi(L_{\mathrm{sym}})) \).
\end{itemize}

\medskip
\noindent
Each step in this spectral program was modular, logically acyclic, and analytically self-contained—requiring no appeal to geometry, adeles, or automorphic theory. The argument culminates in a canonical spectral resolution of the Riemann Hypothesis: the nontrivial zeros of \( \zeta(s) \) lie on the critical line as a direct analytic consequence of the reality of the spectrum of \( L_{\mathrm{sym}} \).

\medskip
\noindent
Further analytic refinements—such as subleading heat trace terms or motivic lifts—remain open and are discussed in Appendix~\ref{app:heat-kernel-refinements}, Appendix~\ref{app:functorial-extensions}, and Appendix~\ref{app:additional-structures}.
