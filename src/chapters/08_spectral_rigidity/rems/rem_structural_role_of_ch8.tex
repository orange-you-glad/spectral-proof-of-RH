\begin{remark}[Structural Role of Chapter~\ref{sec:spectral_rigidity}]
\label{rem:structural_role_of_ch8}

This chapter establishes the converse direction of the analytic–spectral equivalence:
\[
\operatorname{Spec}(L_{\mathrm{sym}}) \subset \mathbb{R} \quad \Longrightarrow \quad \RH,
\]
thereby closing the logical loop initiated in Chapter~\ref{sec:spectral_implications}. All analytic prerequisites—trace-class convergence, determinant identity, and spectral encoding—are proven in prior chapters. No appeal is made to \(\RH\) itself.

\medskip

\noindent
Thus, the equivalence
\[
\RH \iff \operatorname{Spec}(L_{\mathrm{sym}}) \subset \mathbb{R}
\]
is derived entirely from the canonical operator's spectrum and its zeta-regularized Fredholm determinant, without invoking modular, motivic, or trace formula machinery.
\thmref{thm:canonical_operator_realization}
\thmref{thm:rh_spectral_closure}
\end{remark}
