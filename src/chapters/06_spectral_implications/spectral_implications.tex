\section{Spectral Implications: Logical Equivalence and Rigidity}
\label{sec:spectral-implications}

\subsection*{Introduction}

This chapter establishes the analytic infrastructure for defining and analyzing the canonical compact operator \( L_{\mathrm{sym}} \), which realizes the completed Riemann zeta function \( \Xi(s) \) via its Fredholm determinant. The primary goal is to verify that mollified convolution operators associated with the inverse Fourier transform of \( \Xi \) are compact, trace class, and converge in trace norm to a self-adjoint limit operator \( L_{\mathrm{sym}} \in \mathcal{C}_1(H_{\Psi_\alpha}) \).

The constructions here verify:

\begin{itemize}
    \item Schatten-class properties of Hilbert–Schmidt and trace-class operators, following \cite[Ch.~4]{Simon2005TraceIdeals} and \cite[Ch.~VI]{ReedSimon1980I}, including the completeness of \( \mathcal{C}_1 \) and the trace-norm topology.
    
    \item Sufficient conditions for compactness and self-adjointness of integral operators with symmetric Hermitian kernels, using distributional domains and exponential conjugation.
    
    \item The structure of the weighted Schwartz space \( \mathcal{S}_\alpha(\mathbb{R}) \subset L^2(\mathbb{R}, e^{\alpha |x|}\, dx) \), for \( \alpha > \pi \), ensuring Fourier duality and decay control for entire functions of exponential type \( \pi \) \cite{Levin1996EntireLectures}.
    
    \item Uniform kernel bounds and mollifier admissibility for defining the regularized heat operators \( e^{-t L_t^2} \), together with analytic kernel estimates necessary for short-time trace control and Tauberian convergence.
\end{itemize}

These ingredients culminate in the construction of mollified convolution operators \( L_t \), and in the verification of trace-norm convergence
\[
L_t \to L_{\mathrm{sym}} \in \mathcal{C}_1(H_{\Psi_\alpha}) \quad \text{as } t \to 0^+.
\]
This limit defines the canonical spectral operator underlying the determinant identity
\[
\det\nolimits_{\zeta}(I - \lambda L_{\mathrm{sym}}) = \frac{\Xi\left(\tfrac{1}{2} + i\lambda\right)}{\Xi\left(\tfrac{1}{2}\right)},
\]
which is rigorously established without assuming RH.

\medskip

The analytic architecture developed here underpins all subsequent spectral and determinant identities.
See Appendix~\ref{app:dependency-graph} for a visual DAG linking these foundational tools to the modular proof of RH.


%------------------------------------------------------------------
\subsection{Equivalence with the Riemann Hypothesis}

\begin{theorem}[Spectral Reformulation of the Riemann Hypothesis]
\label{thm:eq_of_rh}

Let \( L_{\mathrm{sym}} \in \mathcal{C}_1(H_{\Psi_\alpha}) \cap \mathcal{K}(H_{\Psi_\alpha}) \) denote the canonical compact, self-adjoint, trace-class operator on the exponentially weighted Hilbert space
\[
H_{\Psi_\alpha} := L^2(\mathbb{R}, e^{\alpha |x|} dx), \qquad \alpha > \pi,
\]
constructed via mollified convolution with the inverse Fourier transform of the completed zeta profile
\[
\phi(\lambda) := \Xi\left( \tfrac{1}{2} + i\lambda \right),
\]
as rigorously developed in Chapters~\ref{sec:operator_construction}–\ref{sec:heat_kernel_asymptotics} and Appendix~\ref{app:heat_kernel_construction}.

\medskip

Suppose its Carleman–\(\zeta\)-regularized Fredholm determinant satisfies the canonical identity:
\[
\det\nolimits_\zeta(I - \lambda L_{\mathrm{sym}}) = \frac{\Xi\left( \tfrac{1}{2} + i\lambda \right)}{\Xi\left( \tfrac{1}{2} \right)},
\qquad \forall \lambda \in \mathbb{C},
\]
as proven unconditionally in \thmref{thm:det_identity_revised}.

\medskip

Then the Riemann Hypothesis is equivalent to the spectral reality of \( L_{\mathrm{sym}} \):
\[
\RH \quad \Longleftrightarrow \quad \operatorname{Spec}(L_{\mathrm{sym}}) \subset \mathbb{R}.
\]

\medskip

\noindent
Explicitly, for each nontrivial zero \( \rho = \beta + i\gamma \) of \( \zeta(s) \), define the canonical spectral image:
\[
\mu_\rho := \frac{1}{i}(\rho - \tfrac{1}{2}) = \gamma \in \mathbb{C}.
\]
Then:
\begin{itemize}
  \item \( \mu_\rho \in \mathbb{R} \iff \operatorname{Re}(\rho) = \tfrac{1}{2} \);
  \item Hence,
  \[
  \operatorname{Spec}(L_{\mathrm{sym}}) \subset \mathbb{R}
  \quad \Longleftrightarrow \quad
  \text{All nontrivial zeros } \rho \text{ lie on the critical line.}
  \]
\end{itemize}

\medskip

\noindent
This equivalence follows from:
\begin{enumerate}
  \item The analytic–spectral identity for the determinant of \( L_{\mathrm{sym}} \);
  \item The bijective spectral map \( \rho \mapsto \mu_\rho \) between nontrivial zeros and the nonzero spectrum of \( L_{\mathrm{sym}} \) (see \thmref{thm:spectral_zero_bijection_revised});
  \item The algebraic inversion identity:
  \[
  \mu_\rho \in \mathbb{R} \iff \operatorname{Re}(\rho) = \tfrac{1}{2}.
  \]
\end{enumerate}

\medskip

\noindent
Thus, the Riemann Hypothesis is equivalent to the condition that the entire spectrum of a canonical trace-class operator lies on the real line. This establishes a logically acyclic, operator-theoretic reformulation of RH within the framework of Fredholm theory and spectral determinant calculus.
\end{theorem}

\begin{proof}[Proof of \thmref{thm:eq_of_rh}]
Let \( \rho \mapsto \mu_\rho := \dfrac{1}{i}(\rho - \tfrac{1}{2}) \) denote the canonical spectral reparametrization of the nontrivial zeros \( \rho \) of the Riemann zeta function \( \zeta(s) \). This map is bijective and multiplicity-preserving by \thmref{thm:spectral_zero_bijection_revised} and \lemref{lem:multiplicity_preservation}, ensuring complete spectral correspondence with the nonzero eigenvalues of \( \Lsym \in \TC(\HPsi) \).

\medskip

By \thmref{thm:det_identity_revised}, the Carleman \(\zeta\)-regularized Fredholm determinant of \( \Lsym \) satisfies:
\[
\detz(I - \lambda \Lsym) = \frac{\Xi\left( \tfrac{1}{2} + i\lambda \right)}{\Xi\left( \tfrac{1}{2} \right)},
\qquad \forall \lambda \in \C.
\]
This identity is constructed without assuming RH and rests on analytic inputs from Chapter~\ref{sec:heat_kernel_asymptotics} and Appendix~\ref{app:heat_kernel_construction}, particularly the convergence and singularity structure derived in \lemref{lem:heat_trace_expansion}, \lemref{lem:trace_class_Lt}, and \lemref{lem:kernel_trace_norm_convergence}.

\medskip

Now observe the algebraic implication: for any nontrivial zero \( \rho = \sigma + i\gamma \),
\[
\mu_\rho := \frac{1}{i}(\rho - \tfrac{1}{2}) = \gamma + i(\sigma - \tfrac{1}{2}).
\]
Thus,
\[
\mu_\rho \in \R \iff \Re(\rho) = \tfrac{1}{2}.
\]

\paragraph*{\( \Rightarrow \): Spectral Reality Implies \(\RH\).}
Assume \( \Spec(\Lsym) \subset \R \). Then each \( \mu_\rho \in \R \), so the identity above implies that \( \Re(\rho) = \tfrac{1}{2} \). Hence, all nontrivial zeros lie on the critical line and RH holds.

\paragraph*{\( \Leftarrow \): \(\RH\) Implies Spectral Reality.}
Conversely, assume RH holds. Then for each \( \rho \), \( \Re(\rho) = \tfrac{1}{2} \), which implies \( \mu_\rho \in \R \). Therefore, all nonzero eigenvalues of \( \Lsym \) are real. By self-adjointness and the spectral theorem, \( \Spec(\Lsym) \subset \R \).

\paragraph*{Conclusion.}
The spectrum of \( \Lsym \) is real if and only if all nontrivial zeros of \( \zeta(s) \) lie on the critical line. This establishes the analytic–spectral equivalence
\[
\RH \iff \Spec(\Lsym) \subset \R,
\]
completing the modular operator-theoretic reformulation of the Riemann Hypothesis.
\end{proof}


% Additional lemma: preservation of spectral multiplicity via determinant
\begin{lemma}[Spectral Multiplicity Preservation]
\label{lem:multiplicity_preservation}

Let \( \rho \in \C \) be a nontrivial zero of the Riemann zeta function \( \zetaR(s) \), and define its canonical spectral image
\[
\mu_\rho := \frac{1}{i(\rho - \tfrac{1}{2})}.
\]
Then \( \mu_\rho \in \Spec(L_{\sym}) \) appears with algebraic multiplicity equal to the order of vanishing of \( \zetaR(s) \) at \( \rho \).

\medskip

\noindent
This multiplicity correspondence follows from the Hadamard factorization of the completed zeta function \( \XiR(s) \), which governs the zero structure of the normalized Carleman–\(\zeta\)-regularized Fredholm determinant of \( L_{\sym} \in \TC(\HPsi) \cap \KC(\HPsi) \):
\[
\det\nolimits_\zeta(I - \lambda L_{\sym}) = \frac{\XiR(\tfrac{1}{2} + i\lambda)}{\XiR(\tfrac{1}{2})}.
\]

Taking the logarithmic derivative, we obtain a meromorphic function whose poles correspond to the spectral values:
\[
\frac{d}{d\lambda} \log \det\nolimits_{\zeta}(I - \lambda L_{\sym}) = \sum_{\rho} \frac{m_\rho}{\lambda - \mu_\rho},
\]
where \( m_\rho \) is the multiplicity of the zero \( \rho \) of \( \zetaR \), and \( \mu_\rho \) is its spectral image. This expansion reflects the classical Hadamard product representation of \( \XiR(s) \), and matches the spectral resolvent trace identity for trace-class self-adjoint operators.

\medskip

\noindent
Since \( L_{\sym} \) is compact and self-adjoint, its spectrum consists of isolated real eigenvalues with finite algebraic multiplicity. The residues of the logarithmic derivative coincide with these multiplicities. Therefore, the multiplicity of each spectral point \( \mu_\rho \) matches exactly the order of vanishing of \( \zetaR(s) \) at \( \rho \), as claimed.
\end{lemma}

\begin{proof}
The determinant identity (Theorem~\ref{thm:det-identity-revised}) states:
\[
\det\nolimits_{\zeta}(I - \lambda L_{\mathrm{sym}})
= \frac{\Xi\left(\tfrac{1}{2} + i\lambda\right)}{\Xi\left(\tfrac{1}{2}\right)},
\]
where \( \Xi(s) \) admits a Hadamard factorization over the nontrivial zeros \( \rho \) of \( \zeta(s) \):
\[
\Xi(s) = \Xi\left(\tfrac{1}{2}\right)
\prod_\rho \left(1 - \frac{s - \tfrac{1}{2}}{\rho - \tfrac{1}{2}}\right)
\exp\left( \frac{s - \tfrac{1}{2}}{\rho - \tfrac{1}{2}} \right).
\]
Letting \( \mu_\rho := \frac{1}{i(\rho - \tfrac{1}{2})} \), the determinant vanishes at \( \lambda = 1/\mu_\rho \), with multiplicity equal to the order of vanishing of \( \Xi(s) \) at \( \rho \).

By Lemma~\ref{lem:A_log_derivative}, the logarithmic derivative satisfies:
\[
\frac{d}{d\lambda} \log \det\nolimits_\zeta(I - \lambda L_{\mathrm{sym}})
= \operatorname{Tr}\left( (I - \lambda L_{\mathrm{sym}})^{-1} L_{\mathrm{sym}} \right),
\]
whose pole structure at \( \lambda = 1/\mu_\rho \) reflects the multiplicity of the eigenvalue \( \mu_\rho \). Thus, the eigenvalue multiplicity of \( \mu_\rho \) equals the order of vanishing of \( \zeta(s) \) at \( \rho \), as claimed.
\end{proof}
% 

% Spectral reality lemma: RH ⇔ spectrum real
\begin{lemma}[Spectrum Reality and RH Equivalence]
\label{lem:spectrum-reality-implies-rh}
If all zeros of \( \Xi(\tfrac{1}{2} + i\lambda) \) lie on the real axis, then the canonical convolution operator \( L_{\mathrm{sym}} \) has real spectrum.

Conversely, if \( L_{\mathrm{sym}} \in \mathcal{C}_1(H_{\Psi_\alpha}) \) is self-adjoint, then all eigenvalues \( \mu_\rho \in \mathbb{R} \) imply that the associated nontrivial zeta zeros \( \rho \in \mathbb{C} \) satisfy \( \operatorname{Re}(\rho) = \tfrac{1}{2} \), i.e., the Riemann Hypothesis holds.
\end{lemma}
%  
\begin{proof}[Proof of \lemref{lem:reality_of_spectrum_and_rh}]
Let \( \rho = \beta + i\gamma \) be a nontrivial zero of the Riemann zeta function \( \zetaR(s) \). Define its canonical spectral image:
\[
\mu_\rho = \frac{1}{i(\rho - \tfrac{1}{2})} = \frac{1}{i((\beta - \tfrac{1}{2}) + i\gamma)}.
\]

Set \( z := \beta - \tfrac{1}{2} + i\gamma \in \C \). Then
\[
\mu_\rho = \frac{-i}{z} = \frac{-i((\beta - \tfrac{1}{2}) - i\gamma)}{(\beta - \tfrac{1}{2})^2 + \gamma^2}
= \frac{\gamma}{(\beta - \tfrac{1}{2})^2 + \gamma^2}
- i \cdot \frac{\beta - \tfrac{1}{2}}{(\beta - \tfrac{1}{2})^2 + \gamma^2}.
\]

Hence, \( \mu_\rho \in \R \) if and only if the imaginary part vanishes:
\[
\Im(\mu_\rho) = 0 \quad \Longleftrightarrow \quad \beta = \tfrac{1}{2}.
\]

That is,
\[
\mu_\rho \in \R \iff \rho \in \tfrac{1}{2} + i\R.
\]

Since the canonical spectral map \( \rho \mapsto \mu_\rho \) is injective and covers all nontrivial zeros of \( \zetaR \), this correspondence implies:
\[
\Spec(L_{\sym}) \subset \R \quad \Longleftrightarrow \quad \text{all } \rho \in \mathcal{Z}_\zeta \text{ satisfy } \Re(\rho) = \tfrac{1}{2}.
\]

\medskip

\noindent
Equivalently,
\[
\boxed{
\Spec(L_{\sym}) \subset \R \quad \Longleftrightarrow \quad \RH
}
\]
as claimed.
\end{proof}


% Boxed corollary: RH ⇔ spectrum real
\begin{corollary}[Equivalence of RH with Spectrum Reality]
\label{cor:spectrum-real-equiv-rh}
Let \( L_{\mathrm{sym}} \in \mathcal{C}_1(H_{\Psi_\alpha}) \) be the canonical self-adjoint trace-class operator whose Fredholm determinant satisfies
\[
\det\nolimits_\zeta(I - \lambda L_{\mathrm{sym}})
= \frac{\Xi\left(\tfrac{1}{2} + i\lambda\right)}{\Xi\left(\tfrac{1}{2}\right)}.
\]

Then the Riemann Hypothesis holds if and only if the spectrum of \( L_{\mathrm{sym}} \) is real:
\[
\boxed{
\operatorname{Spec}(L_{\mathrm{sym}}) \subset \mathbb{R}
\quad \Longleftrightarrow \quad \mathrm{RH}
}
\]

\noindent
That is, the nontrivial zeros of \( \zeta(s) \) lie on the critical line \( \Re(s) = \tfrac{1}{2} \) if and only if all eigenvalues of \( L_{\mathrm{sym}} \) are real.

\end{corollary}
% 
\begin{proof}[Proof of Corollary~\ref{cor:spectrum-real-equiv-rh}]
This is an immediate consequence of Lemma~\ref{lem:reality-of-spectrum-and-rh}. For each nontrivial zero \( \rho = \beta + i\gamma \) of \( \zeta(s) \), we define the associated eigenvalue
\[
\mu_\rho := \frac{1}{i(\rho - \tfrac{1}{2})}.
\]

By Lemma~\ref{lem:reality-of-spectrum-and-rh}, we have:
\[
\mu_\rho \in \mathbb{R} \quad \Longleftrightarrow \quad \beta = \tfrac{1}{2}.
\]
Therefore, all \( \mu_\rho \in \operatorname{Spec}(L_{\mathrm{sym}}) \) are real if and only if all nontrivial zeros \( \rho \) lie on the critical line \( \Re(\rho) = \tfrac{1}{2} \). This is precisely the Riemann Hypothesis.

Hence,
\[
\operatorname{Spec}(L_{\mathrm{sym}}) \subset \mathbb{R} \quad \iff \quad \mathrm{RH}.
\]
\end{proof}
% 

% Physics analogy remark
\begin{remark}[Spectral Physics Perspective]
The equivalence \( \mathrm{RH} \iff \operatorname{Spec}(L_{\mathrm{sym}}) \subset \mathbb{R} \) admits a speculative interpretation in the context of quantum mechanics. Under the spectral map \( \rho = \tfrac{1}{2} + i\gamma \mapsto \mu_\rho := \tfrac{1}{\gamma} \), the canonical operator \( L_{\mathrm{sym}} \) can be formally viewed as a Hamiltonian with inverse arithmetic energy levels. Its heat trace resembles a quantum partition function with singular short-time behavior, and its determinant has parallels with spectral free energy. See Appendix~\ref{app:spectral-physics-link} for further discussion of this physical analogy.
\end{remark}

% Determinant trace-log identity is used but not proved here
\medskip
\noindent
For the analytic justification of the trace–logarithmic derivative identity, see Lemma~\ref{lem:log-derivative-determinant} in Chapter~\ref{sec:heat-kernel-asymptotics} and the supporting analysis in Appendix~\ref{app:heat-kernel-construction}.

%------------------------------------------------------------------
\subsection{Uniqueness of Spectral Realization}

% This result builds directly on the spectral-zero bijection and the determinant identity,
% enforcing uniqueness within the trace-class, self-adjoint realization space.
\begin{theorem}[Uniqueness of Spectral Realization]
\label{thm:uniqueness_realization}

Let \( L \in \mathcal{C}_1(H_{\Psi_\alpha}) \cap \mathcal{K}(H_{\Psi_\alpha}) \) be a compact, self-adjoint, trace-class operator on the exponentially weighted Hilbert space
\[
H_{\Psi_\alpha} := L^2(\mathbb{R}, e^{\alpha |x|} \, dx), \qquad \alpha > \pi.
\]

Suppose \( L \) satisfies the canonical zeta-regularized determinant identity:
\[
\det\nolimits_\zeta(I - \lambda L) = \frac{\Xi\left( \tfrac{1}{2} + i\lambda \right)}{\Xi\left( \tfrac{1}{2} \right)},
\quad \forall \lambda \in \mathbb{C},
\]
where \( \Xi(s) \) is the completed Riemann zeta function, entire of order one and exact exponential type \( \pi \). Assume the normalization:
\[
\det\nolimits_\zeta(I) = 1.
\]

Then \( L \) is unitarily equivalent to the canonical operator \( L_{\mathrm{sym}} \in \mathcal{C}_1(H_{\Psi_\alpha}) \). That is, there exists a unitary operator
\[
U \colon H_{\Psi_\alpha} \to H_{\Psi_\alpha} \quad \text{such that} \quad L = U L_{\mathrm{sym}} U^{-1}.
\]

\medskip
\noindent
In particular:
\begin{itemize}
  \item The spectrum of \( L \), including all algebraic multiplicities, coincides with that of \( L_{\mathrm{sym}} \);
  \item \( L_{\mathrm{sym}} \) is the unique (up to unitary equivalence) compact, self-adjoint, trace-class realization of the completed zeta function’s canonical spectral determinant;
  \item The analytic data encoded in \( \Xi(s) \)—via its Hadamard factorization and spectral trace regularization—rigidly determines the operator-theoretic structure of \( L_{\mathrm{sym}} \).
\end{itemize}
\end{theorem}

\begin{proof}[Proof of \thmref{thm:uniqueness_realization}]
Let \( L \in \mathcal{C}_1(H_{\Psi_\alpha}) \cap \mathcal{K}(H_{\Psi_\alpha}) \) be a compact, self-adjoint, trace-class operator on the weighted Hilbert space \( H_{\Psi_\alpha} := L^2(\mathbb{R}, e^{\alpha|x|}dx) \), with fixed \( \alpha > \pi \). Suppose:
\[
\det\nolimits_\zeta(I - \lambda L) = \frac{\Xi\left(\tfrac{1}{2} + i\lambda\right)}{\Xi\left(\tfrac{1}{2}\right)}
= \det\nolimits_\zeta(I - \lambda L_{\mathrm{sym}}),
\quad \forall \lambda \in \mathbb{C}.
\]

\paragraph{Step 1: Spectral Data from Determinant Identity.}
By classical trace-class determinant theory (see~\cite[Thm. 4.2]{Simon2005TraceIdeals}), the normalized Carleman–\(\zeta\)-regularized determinant admits the product representation:
\[
\det\nolimits_\zeta(I - \lambda L)
= \prod_{n=1}^\infty (1 - \lambda \mu_n),
\]
where \( \{ \mu_n \} \subset \mathbb{R} \setminus \{0\} \) are the nonzero eigenvalues of \( L \), counted with algebraic multiplicity. Since this determinant agrees identically with that of \( L_{\mathrm{sym}} \), and both are entire functions of order one normalized by \( \det\nolimits_\zeta(I) = 1 \), we conclude:
\[
\operatorname{Spec}(L) = \operatorname{Spec}(L_{\mathrm{sym}}),
\quad \text{as multisets}.
\]

\paragraph{Step 2: Spectral Equivalence Implies Unitary Equivalence.}
Since \( L \) and \( L_{\mathrm{sym}} \) are both compact, self-adjoint operators on the same separable Hilbert space \( H_{\Psi_\alpha} \), and since their spectra (with multiplicities) coincide, the spectral theorem implies that \( L \) is unitarily equivalent to \( L_{\mathrm{sym}} \). That is, there exists a unitary operator
\[
U \colon H_{\Psi_\alpha} \to H_{\Psi_\alpha}
\quad \text{such that} \quad
L = U L_{\mathrm{sym}} U^{-1}.
\]

\paragraph{Conclusion.}
The canonical operator \( L_{\mathrm{sym}} \) is thus uniquely determined (up to unitary equivalence) among all compact, self-adjoint, trace-class operators realizing the normalized spectral determinant identity for \( \Xi(s) \). The analytic fingerprint of \( \Xi \)—its order-one entire structure, exponential type, and Hadamard factorization—rigidly determines the operator-theoretic data of \( L_{\mathrm{sym}} \), completing the proof.
\end{proof}


% Spectral rigidity lemma: determinant identity forces spectral agreement
\begin{lemma}[Spectral Rigidity from Determinant Identity]
\label{lem:spectral_rigidity_determinant}

Let \( L_1, L_2 \in \TC(\HPsi) \cap \KC(\HPsi) \) be compact, self-adjoint, trace-class operators on the exponentially weighted Hilbert space \( \HPsi := L^2(\R, e^{\alpha|x|}dx) \) with \( \alpha > \pi \).

Suppose their Carleman–\(\zeta\)-regularized Fredholm determinants coincide:
\[
\det\nolimits_\zeta(I - \lambda L_1) = \det\nolimits_\zeta(I - \lambda L_2),
\quad \forall \lambda \in \C,
\]
with both normalized at the origin:
\[
\det\nolimits_\zeta(I) = 1.
\]

Then \( L_1 \) and \( L_2 \) have identical nonzero spectra, including algebraic multiplicities:
\[
\Spec(L_1) \setminus \{0\} = \Spec(L_2) \setminus \{0\}
\quad \text{as multisets}.
\]

If both operators act on the same Hilbert space, then the spectral theorem implies they are unitarily equivalent.
\end{lemma}

\begin{proof}[Proof of \lemref{lem:spectral_rigidity_determinant}]
Let \( L_1, L_2 \in \TC(\HPsi) \cap \KC(\HPsi) \) be compact, self-adjoint, trace-class operators satisfying:
\[
\det\nolimits_\zeta(I - \lambda L_1) = \det\nolimits_\zeta(I - \lambda L_2), \quad \forall \lambda \in \C,
\]
with both determinants normalized at the origin: \( \det\nolimits_\zeta(I) = 1 \), as ensured by \lemref{lem:trace_zero}.

\paragraph{Step 1: Spectral Encoding via Determinant Structure.}
For compact, self-adjoint operators in \( \TC \), the zeta-regularized Fredholm determinant admits the canonical Hadamard product expansion:
\[
\det\nolimits_\zeta(I - \lambda L_j) = \prod_{\mu \in \Spec(L_j) \setminus \{0\}} (1 - \lambda \mu)^{\operatorname{mult}_{L_j}(\mu)},
\quad j = 1,2,
\]
by \lemref{lem:hadamard_linear_form}. Since the two determinants coincide as entire functions of order one and exponential type \( \pi \), and share the normalization \( \det\nolimits_\zeta(I) = 1 \), the identity theorem for entire functions implies that their zero sets (counted with multiplicity) must coincide. Hence:
\[
\Spec(L_1) \setminus \{0\} = \Spec(L_2) \setminus \{0\}
\quad \text{as multisets}.
\]

\paragraph{Step 2: Completion via Spectral Theorem.}
If \( L_1 \) and \( L_2 \) act on the same Hilbert space \( \HPsi \), then the spectral theorem for compact self-adjoint operators ensures the existence of a unitary operator
\[
U \colon \HPsi \to \HPsi
\quad \text{such that} \quad
L_2 = U L_1 U^{-1}.
\]

\paragraph{Conclusion.}
Thus, the Carleman–\(\zeta\)-regularized Fredholm determinant serves as a complete spectral fingerprint for compact, self-adjoint trace-class operators: the analytic data of the determinant determines the operator spectrum uniquely, and—on a fixed Hilbert space—determines the operator itself up to unitary equivalence. This rigidity underlies the uniqueness result in \thmref{thm:uniqueness_realization}.
\end{proof}


% Determinant uniquely fixes spectral multiset
\begin{lemma}[Determinant Identity Fixes the Spectrum]
\label{lem:determinant_fixes_spectrum}

Let \( L \in \TC(\HPsi) \cap \KC(\HPsi) \) be a compact, self-adjoint, trace-class operator satisfying the normalized spectral determinant identity:
\[
\det\nolimits_\zeta(I - \lambda L) = \frac{\XiR\left(\tfrac{1}{2} + i\lambda\right)}{\XiR\left(\tfrac{1}{2}\right)},
\quad \forall \lambda \in \C,
\]
and assume its trace vanishes:
\[
\Tr(L) = 0.
\]

Then the nonzero spectrum of \( L \), counted with algebraic multiplicity, coincides with that of the canonical operator \( L_{\sym} \). That is,
\[
\Spec(L) \setminus \{0\} = \Spec(L_{\sym}) \setminus \{0\},
\quad \text{as multisets}.
\]
\thmref{thm:canonical_operator_realization}
\thmref{thm:det_identity_revised}
\lemref{lem:trace_zero}
\end{lemma}

\begin{proof}[Proof of \lemref{lem:determinant_fixes_spectrum}]
Let \( f(\lambda) := \det\nolimits_\zeta(I - \lambda L) \), and suppose
\[
f(\lambda) = \frac{\XiR(\tfrac{1}{2} + i\lambda)}{\XiR(\tfrac{1}{2})}
= \det\nolimits_\zeta(I - \lambda L_{\sym}),
\quad \forall \lambda \in \C,
\]
where \( L \in \TC(\HPsi) \cap \KC(\HPsi) \) is compact, self-adjoint, trace-class, and satisfies \( \Tr(L) = 0 \).

\paragraph{Step 1: Entire Function Identity and Trace Normalization.}
Both determinant functions are entire of order one and exponential type \( \pi \), and normalized so that \( f(0) = 1 \). The trace-zero condition removes any exponential prefactor ambiguity in their Hadamard factorization—i.e., no term of the form \( e^{a\lambda} \) appears.

\paragraph{Step 2: Logarithmic Derivative and Spectral Poles.}
The logarithmic derivative of the determinant is governed by the resolvent trace formula:
\[
\frac{d}{d\lambda} \log f(\lambda)
= \Tr\left[(I - \lambda L)^{-1} L\right],
\]
which is meromorphic with simple poles at \( \lambda = 1/\mu \) for each nonzero eigenvalue \( \mu \in \Spec(L) \), with residue equal to the algebraic multiplicity of \( \mu \).

Since the determinant agrees with that of \( L_{\sym} \), these poles match those of the canonical model, and thus:
\[
\Spec(L) \setminus \{0\} = \Spec(L_{\sym}) \setminus \{0\}
\quad \text{as multisets}.
\]

\paragraph{Conclusion.}
The spectral data of \( L \), away from zero, is completely encoded by the determinant under the trace normalization condition. Therefore, \( L \) and \( L_{\sym} \) have identical nonzero spectra, completing the proof.
\end{proof}


%------------------------------------------------------------------
\subsection{Canonical Closure of the Spectral Program}

\begin{lemma}[Canonical Closure of the Spectral Model]
\label{lem:canonical_closure}

Let \( L \in \TC(\HPsi) \cap \KC(\HPsi) \) be a compact, self-adjoint, trace-class operator on the exponentially weighted Hilbert space
\[
\HPsi := L^2(\R, e^{\alpha |x|} \, dx), \qquad \alpha > \pi,
\]
and suppose \( L \) satisfies the normalized spectral determinant identity:
\[
\det\nolimits_\zeta(I - \lambda L) = \frac{\XiR\left(\tfrac{1}{2} + i\lambda\right)}{\XiR\left(\tfrac{1}{2}\right)}, \qquad \forall \lambda \in \C,
\]
with normalization \( \det\nolimits_\zeta(I) = 1 \), and where \( \XiR(s) \) is the completed Riemann zeta function.

\medskip
\noindent
Then:
\begin{enumerate}
  \item The nonzero spectrum of \( L \) coincides with that of the canonical operator \( L_{\sym} \), as multisets with algebraic multiplicities:
  \[
  \Spec(L) \setminus \{0\} = \Spec(L_{\sym}) \setminus \{0\};
  \]
  
  \item \( L \) is unitarily equivalent to \( L_{\sym} \): there exists a unitary operator \( U \colon \HPsi \to \HPsi \) such that
  \[
  L = U L_{\sym} U^{-1};
  \]
  
  \item \( L_{\sym} \) is the unique (up to unitary equivalence) compact, self-adjoint trace-class operator whose zeta-regularized determinant realizes the spectral identity associated with \( \XiR(s) \);
  
  \item If \( \widetilde{L} \in \TC \) satisfies the same determinant identity but is not self-adjoint, then \( \widetilde{L} \) is similar to \( L_{\sym} \) in the algebraic sense: there exists an invertible operator \( S \in \mathcal{B}(\HPsi) \) such that
  \[
  \widetilde{L} = S L_{\sym} S^{-1},
  \]
  preserving the nonzero spectrum and multiplicities, though not necessarily realized via a unitary conjugation.
\end{enumerate}

\medskip
\noindent
Hence, the canonical spectral determinant associated with \( \XiR(s) \), under trace-class and self-adjointness, uniquely determines the operator \( L_{\sym} \) up to unitary equivalence, and rigidly constrains all other determinant-realizing models to algebraic similarity. This completes the canonical closure of the spectral model.
\end{lemma}

\begin{proof}[Proof of \lemref{lem:canonical_closure}]
By assumption, \( L \in \TC(\HPsi) \cap \KC(\HPsi) \) is compact, self-adjoint, and satisfies the normalized spectral determinant identity:
\[
\det\nolimits_\zeta(I - \lambda L) = \frac{\XiR\left(\tfrac{1}{2} + i\lambda\right)}{\XiR\left(\tfrac{1}{2}\right)}, \quad \forall \lambda \in \C.
\]

\paragraph{(1) Spectral Equality from Determinant Identity.}
By trace-class determinant theory (see~\cite[Theorem~4.2]{Simon2005TraceIdeals}), the zeta-regularized determinant encodes the nonzero spectrum of \( L \) as a multiset (including algebraic multiplicities). Since the determinant of \( L \) matches that of \( L_{\sym} \), we conclude:
\[
\Spec(L) \setminus \{0\} = \Spec(L_{\sym}) \setminus \{0\}.
\]

\paragraph{(2) Unitary Equivalence for Self-Adjoint Case.}
Both \( L \) and \( L_{\sym} \) are compact, self-adjoint operators on the same separable Hilbert space \( \HPsi \), with matching spectra and multiplicities. By the spectral theorem for compact self-adjoint operators (see~\cite[Theorem~VI.16]{ReedSimon1980I}), there exists a unitary operator \( U \colon \HPsi \to \HPsi \) such that
\[
L = U L_{\sym} U^{-1}.
\]

\paragraph{(3) Uniqueness of the Canonical Realization.}
The above shows that \( L_{\sym} \) is unique up to unitary equivalence within the class of compact, self-adjoint, trace-class operators realizing the spectral determinant identity for \( \XiR(s) \).

\paragraph{(4) Similarity Class for Non-Self-Adjoint Realizations.}
Suppose \( \widetilde{L} \in \TC(\HPsi) \) is not self-adjoint but still satisfies the same determinant identity. Then it must have the same nonzero spectral multiset as \( L_{\sym} \), including multiplicities. While lack of normality may prevent diagonalizability or self-adjointness, spectral similarity implies the existence of an invertible operator \( S \in \mathcal{B}(\HPsi) \) such that
\[
\widetilde{L} = S L_{\sym} S^{-1}.
\]
This shows that \( \widetilde{L} \) lies in the similarity class of \( L_{\sym} \), even if not in its unitary equivalence class.

\paragraph{Conclusion.}
The spectral determinant identity associated with \( \XiR(s) \), together with trace-class compactness and self-adjointness, canonically determines the operator \( L_{\sym} \) up to unitary equivalence. Any non-self-adjoint realization is algebraically similar to this canonical model, thus completing the closure of the spectral program.
\end{proof}


%------------------------------------------------------------------

% Final logical closure of the analytic-spectral architecture.
\subsection*{Summary}
\label{sec:foundations_summary}

\textbf{Operator-Theoretic Foundations}
\begin{itemize}
  \item \defref{def:compact_operator} — Compact operators: norm limits of finite-rank maps with discrete spectrum.
  \item \defref{def:trace_class_operator}, \defref{def:trace_norm} — Trace-class operators \( T \in \TC(H) \) with finite trace norm \( \|T\|_{\Tr} := \Tr(|T|) \); Banach completeness and unitary invariance.
  \item \defref{def:selfadjoint_operator} — Self-adjointness as maximal symmetry enabling spectral calculus and semigroup generation.
\end{itemize}

\textbf{Weighted Spaces and Function Classes}
\begin{itemize}
  \item \defref{def:exponential_weight}, \defref{def:weighted_schwartz_space} — The space \( \HPsi = L^2(\R, e^{\alpha|x|}\,dx) \), with \( \Schwartz(\R) \subset \HPsi \) a dense core.
  \item \lemref{lem:density_schwartz_weighted_L2} — Density of \( \Schwartz \subset \HPsi \) in norm and graph topology.
  \item \remref{rem:sobolev_core_reference} — Alternate justification: \( \Schwartz \hookrightarrow H^s_\alpha \hookrightarrow \HPsi \) via Sobolev embeddings.
\end{itemize}

\textbf{Analytic and Spectral Estimates}
\begin{itemize}
  \item \lemref{lem:xi_growth_bound}, \lemref{lem:weighted_L1_inverse_FT_xi} — The profile \( \Xi(\tfrac{1}{2} + i\lambda) \in \PW{\pi} \), with inverse transform in \( L^1(\R, \Psi_\alpha^{-1}) \).
  \item \lemref{lem:decay_mollified_kernel}, \lemref{lem:L1_integrability_conjugated_kernel} — Mollifiers \( k_t \in \Schwartz \), conjugated kernels integrable.
  \item \lemref{lem:uniform_L1_conjugated_kernel}, \lemref{lem:trace_class_via_weighted_L1} — Trace norm convergence \( \|L_t - \Lsym\|_{\TC} \to 0 \) and Simon’s trace-class inclusion criterion.
  \item \lemref{lem:trace_class_conjugated_kernel}, \lemref{lem:trace_class_failure_alpha_leq_pi}, \propref{prop:trace_class_sharpness} — Trace-class fails for \( \alpha \le \pi \): sharp exponential decay threshold.
  \item \lemref{lem:unitary_conjugation_trace_class} — Trace norm preserved under unitary weight conjugation.
\end{itemize}

\textbf{Operator Properties of \texorpdfstring{\( L_t \)}{Lt}}
\begin{itemize}
  \item \propref{prop:boundedness_Lt_weighted}, \propref{prop:compactness_Lt} — Boundedness and compactness of \( L_t \) via mollified kernel structure.
  \item \propref{prop:symmetry_Lt_Schwartz}, \propref{prop:selfadjointness_Lt} — \( L_t \) is symmetric on \( \Schwartz \) and extends to a self-adjoint operator.
  \item \propref{prop:core_schwartz_density} — \( \Schwartz \) is a core for the limit operator \( \Lsym \).
\end{itemize}

\textbf{Canonical Operator Realization}
\begin{itemize}
  \item \thmref{thm:canonical_operator_realization} — Convergence \( L_t \to \Lsym \in \TC(\HPsi) \); defines the canonical compact self-adjoint operator realizing the spectral determinant.
\end{itemize}

\paragraph{Chapter Closure.}
This chapter establishes the analytic and operator-theoretic base for all that follows. The canonical convolution operator \( \Lsym \in \TC(\HPsi) \) is defined as the trace-norm limit of mollified Fourier convolution operators \( L_t \). Its construction relies on Paley--Wiener estimates, exponential decay, Sobolev density, and trace-class embedding theorems. The determinant identity
\[
\detz(I - \lambda \Lsym)
= \frac{\Xi\left(\tfrac{1}{2} + i\lambda \right)}{\Xi\left(\tfrac{1}{2} \right)}
\]
is proven in \secref{sec:determinant_identity}, resting entirely on this analytic groundwork.

