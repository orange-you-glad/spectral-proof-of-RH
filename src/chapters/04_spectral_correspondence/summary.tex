\subsection*{Chapter Summary}

This chapter establishes a canonical spectral encoding of the nontrivial zeros of the Riemann zeta function via the compact, self-adjoint operator \( \Lsym \in \TC(\HPsi) \). The key results are:

\begin{itemize}
  \item \lemref{lem:zero_to_eigenvalue_injection} — Each nontrivial zero \( \rho = \tfrac{1}{2} + i\gamma \) of \( \zeta(s) \) defines a nonzero eigenvalue via
  \[
  \mu_\rho := \frac{1}{i}(\rho - \tfrac{1}{2}) = \gamma \in \Spec(\Lsym).
  \]

  \item \lemref{lem:spectral_exhaustivity} — Surjectivity: every \( \mu \in \Spec(\Lsym) \setminus \{0\} \) arises from some \( \rho \), i.e., from a nontrivial zeta zero.

  \item \lemref{lem:spectral_multiplicity_matching} — Multiplicity preservation:
  \[
  \operatorname{mult}(\mu_\rho) = \operatorname{ord}_\rho(\zeta).
  \]

  \item \lemref{lem:spectral_symmetry} — Spectral symmetry:
  \[
  \mu \in \Spec(\Lsym) \quad \Longrightarrow \quad -\mu \in \Spec(\Lsym),
  \]
  reflecting the functional symmetry \( \zeta(s) = \zeta(1 - s) \).

  \item \lemref{lem:spectral_bijection_consistency} — The map \( \rho \mapsto \mu_\rho \) is a multiplicity-preserving bijection from nontrivial zeros of \( \zeta(s) \) to the nonzero spectrum of \( \Lsym \).

  \item \thmref{thm:spectral_zero_bijection_revised} — Consolidated result:
  \[
  \zeta(\rho) = 0 \quad \Longleftrightarrow \quad \mu_\rho := \frac{1}{i}(\rho - \tfrac{1}{2}) \in \Spec(\Lsym),
  \]
  with multiplicities exactly matching.
\end{itemize}

\begin{quote}
  \textbf{Remark (Spectral Encoding as Analytic Dual).}~
  This bijection is canonical: the full spectrum of \( \Lsym \) is uniquely determined by the analytic structure of \( \Xi(s) \) via the determinant identity. Conversely, as shown in \corref{cor:spectrum_determines_zeta}, the spectrum of \( \Lsym \), including multiplicities, fully reconstructs \( \Xi(s) \) and thereby recovers the nontrivial zeros of \( \zeta(s) \).
\end{quote}

See Table~\ref{fig:schematic_bijection_table} for a visual overview of the map \( \rho \mapsto \mu_\rho \), its symmetry \( \pm\mu \), and the bijection structure.

\medskip

This canonical spectral encoding underpins the equivalence
\[
\mathrm{RH} \quad \Longleftrightarrow \quad \Spec(\Lsym) \subset \R,
\]
which we prove in Chapter~\ref{sec:spectral_implications}.
