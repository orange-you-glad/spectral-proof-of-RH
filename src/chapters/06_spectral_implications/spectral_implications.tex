\section{Spectral Implications: Logical Equivalence and Rigidity}
\label{sec:spectral_implications}

\subsection*{Introduction}
\begin{remark}[Structural Role of Chapter~\ref{sec:spectral_rigidity}]
\label{rem:structural_role_of_ch8}

This chapter establishes the converse direction of the analytic–spectral equivalence:
\[
\operatorname{Spec}(L_{\mathrm{sym}}) \subset \mathbb{R} \quad \Longrightarrow \quad \RH,
\]
thereby closing the logical loop initiated in Chapter~\ref{sec:spectral_implications}. All analytic prerequisites—trace-class convergence, determinant identity, and spectral encoding—are proven in prior chapters. No appeal is made to \(\RH\) itself.

\medskip

\noindent
Thus, the equivalence
\[
\RH \iff \operatorname{Spec}(L_{\mathrm{sym}}) \subset \mathbb{R}
\]
is derived entirely from the canonical operator's spectrum and its zeta-regularized Fredholm determinant, without invoking modular, motivic, or trace formula machinery.
\end{remark}


This chapter recasts the Riemann Hypothesis as a statement of spectral rigidity: the spectrum of the canonical trace-class operator
\[
L_{\sym} \in \TC(\HPsi),
\]
is real if and only if all nontrivial zeros of the Riemann zeta function lie on the critical line \( \Re(s) = \tfrac{1}{2} \). The operator \( L_{\sym} \) is constructed via mollified convolution from the inverse Fourier transform of the completed zeta function \( \XiR(s) \).

\paragraph{Goals.}
\begin{itemize}
  \item \textit{Spectral Encoding.}  
  The Carleman \(\zeta\)-regularized determinant identity
  \[
  \det\nolimits_{\zeta}(I - \lambda L_{\sym})
  = \frac{\XiR\left(\tfrac{1}{2} + i\lambda\right)}{\XiR\left(\tfrac{1}{2}\right)}
  \]
  defines a multiplicity-preserving encoding
  \[
  \rho \mapsto \mu_\rho := \frac{1}{i}(\rho - \tfrac{1}{2}),
  \]
  sending nontrivial zeros \( \rho \) of \( \zetaR(s) \) to eigenvalues \( \mu_\rho \in \Spec(L_{\sym}) \). This intertwines the Hadamard factorization of \( \XiR(s) \) with the spectral structure of \( L_{\sym} \). We show that every determinant zero corresponds to a spectral eigenvalue.

  \item \textit{Spectral Rigidity.}  
  Although \( L_{\sym} \) is self-adjoint and thus has real spectrum, we prove the converse: if all encoded eigenvalues \( \mu_\rho \in \R \), then each corresponding zero \( \rho \) lies on the critical line. That is,
  \[
  \Spec(L_{\sym}) \subset \R
  \quad \Longleftrightarrow \quad
  \Re(\rho) = \tfrac{1}{2}, \quad \text{for all } \rho \in \Spec(\zetaR).
  \]
  Determinantal vanishing implies spectral inclusion via analytic continuation and Fredholm theory, without requiring a prior bijection.

  \item \textit{Spectral Symmetry.}  
  The functional identity \( \XiR(\tfrac{1}{2} + i\lambda) = \XiR(\tfrac{1}{2} - i\lambda) \) implies
  \[
  \mu \in \Spec(L_{\sym}) \quad \Rightarrow \quad -\mu \in \Spec(L_{\sym}),
  \]
  with multiplicities preserved. This reflects the evenness of the centered spectral profile \( \phi(\lambda) := \XiR(\tfrac{1}{2} + i\lambda) \).

  \item \textit{Trace Positivity.}  
  The trace pairing
  \[
  \phi \mapsto \Tr(\phi(L_{\sym}))
  \]
  defines a positive tempered distribution on \( \R \). In particular,
  \[
  \Tr(e^{-tL_{\sym}^2}) \ge 0 \quad \forall\, t > 0,
  \]
  reflecting positivity of the heat kernel regularized spectral measure. This positivity extends to all \( \phi \in \Schwartz(\R) \), reinforcing the harmonic-analytic structure of the trace.

  \item \textit{Analytic Independence.}  
  All results in this chapter follow from classical analytic theory:
  \begin{itemize}
    \item spectral theory of compact, self-adjoint operators;
    \item kernel decay and semigroup regularity;
    \item Hadamard product theory and functional symmetry;
    \item Fredholm theory and determinant–spectrum correspondence;
    \item unitary equivalence via orthonormal diagonalization.
  \end{itemize}
  No input is required from modular forms, trace formulas, or Langlands theory.
\end{itemize}


%------------------------------------------------------------------
\subsection{Equivalence with the Riemann Hypothesis}

\begin{theorem}[Spectral Reformulation of the Riemann Hypothesis]
\label{thm:eq_of_rh}

Let \( L_{\sym} \in \TC(\HPsi) \cap \KC(\HPsi) \) denote the canonical compact, self-adjoint, trace-class operator on the exponentially weighted Hilbert space
\[
\HPsi := \PsiAlphaSpace{x}, \qquad \alpha > \pi,
\]
constructed via mollified convolution with the inverse Fourier transform of the completed zeta profile
\[
\phi(\lambda) := \XiR\left( \tfrac{1}{2} + i\lambda \right),
\]
as rigorously developed in Chapters~\ref{sec:operator_construction}–\ref{sec:heat_kernel_asymptotics} and Appendix~\ref{app:heat_kernel_construction}.

\medskip

Suppose its Carleman–\(\zeta\)-regularized Fredholm determinant satisfies the canonical identity:
\[
\det\nolimits_\zeta(I - \lambda L_{\sym}) = \frac{\XiR\left( \tfrac{1}{2} + i\lambda \right)}{\XiR\left( \tfrac{1}{2} \right)},
\qquad \forall \lambda \in \C,
\]
as proven unconditionally in \thmref{thm:det_identity_revised}.

\medskip

Then the Riemann Hypothesis is equivalent to the spectral reality of \( L_{\sym} \):
\[
\RH \iff \Spec(L_{\sym}) \subset \R.
\]

\medskip

\noindent
Explicitly, for each nontrivial zero \( \rho = \beta + i\gamma \) of \( \zetaR(s) \), define the canonical spectral image:
\[
\mu_\rho := \frac{1}{i(\rho - \tfrac{1}{2})} = \frac{1}{\gamma} \in \C.
\]
Then:
\begin{itemize}
  \item \( \mu_\rho \in \R \iff \Re(\rho) = \tfrac{1}{2} \);
  \item Hence,
  \[
  \Spec(L_{\sym}) \subset \R \iff \text{ all nontrivial zeros } \rho \text{ lie on the critical line.}
  \]
\end{itemize}

\medskip

\noindent
This equivalence follows from:
\begin{enumerate}
  \item The analytic–spectral identity for the determinant of \( L_{\sym} \);
  \item The bijective spectral map \( \rho \mapsto \mu_\rho \) between nontrivial zeros and the nonzero spectrum of \( L_{\sym} \) (see \thmref{thm:spectral_zero_bijection_revised});
  \item The algebraic inversion identity:
  \[
  \mu_\rho \in \R \iff \Re(\rho) = \tfrac{1}{2}.
  \]
\end{enumerate}

\medskip

\noindent
Thus, the Riemann Hypothesis is equivalent to the condition that the entire spectrum of a canonical trace-class operator lies on the real line. This establishes a logically acyclic, operator-theoretic reformulation of RH within the framework of Fredholm theory and spectral determinant calculus.
\end{theorem}

\begin{proof}[Proof of \thmref{thm:eq_of_rh}]
Let \( \rho \mapsto \mu_\rho := \dfrac{1}{i(\rho - \tfrac{1}{2})} \) denote the canonical spectral reparametrization of the nontrivial zeros \( \rho \) of the Riemann zeta function \( \zetaR(s) \). This spectral mapping is bijective by \thmref{thm:spectral_zero_bijection_revised} and preserves multiplicity by \lemref{lem:multiplicity_preservation}, ensuring trace-class spectral correspondence.

\medskip

Under this map, the normalized Carleman–\(\zeta\)-regularized determinant of the canonical operator \( L_{\sym} \in \TC(\HPsi) \cap \KC(\HPsi) \) satisfies:
\[
\det\nolimits_\zeta(I - \lambda L_{\sym}) = \frac{\XiR\left( \tfrac{1}{2} + i\lambda \right)}{\XiR\left( \tfrac{1}{2} \right)},
\]
as established unconditionally in \thmref{thm:det_identity_revised}. The analytic infrastructure for this identity—including short-time trace bounds, semigroup holomorphy, and Laplace convergence—is developed in Chapter~\ref{sec:heat_kernel_asymptotics} and Appendix~\ref{app:heat_kernel_construction}.

\medskip

Now observe the algebraic identity: for any nontrivial zero \( \rho = \sigma + i\gamma \),
\[
\mu_\rho = \frac{1}{i(\rho - \tfrac{1}{2})} = \frac{1}{i(\sigma - \tfrac{1}{2}) - \gamma}.
\]
Then \( \mu_\rho \in \R \) if and only if \( \sigma = \tfrac{1}{2} \), i.e., if and only if \( \rho \) lies on the critical line. Thus:
\[
\mu_\rho \in \R \iff \Re(\rho) = \tfrac{1}{2}.
\]

\paragraph*{(\( \Rightarrow \)) Spectral Reality Implies \(\RH\).}
Assume \( \Spec(L_{\sym}) \subset \R \). Then all eigenvalues \( \mu_\rho \in \R \), so the identity above implies that every nontrivial zero \( \rho \) satisfies \( \Re(\rho) = \tfrac{1}{2} \). Hence, \RH{} holds.

\paragraph*{(\( \Leftarrow \)) \(\RH\) Implies Spectral Reality.}
Conversely, assume \RH{} holds. Then for each nontrivial zero \( \rho \), we have \( \Re(\rho) = \tfrac{1}{2} \), which implies \( \mu_\rho \in \R \). Therefore, all nonzero eigenvalues of \( L_{\sym} \) are real.

\paragraph*{Conclusion.}
Since \( L_{\sym} \) is compact and self-adjoint (\thmref{thm:sa_trace_class_Lsym}), the spectral theorem ensures its spectrum lies in \( \R \). Hence,
\[
\Spec(L_{\sym}) \subset \R \iff \mu_\rho \in \R \text{ for all nontrivial zeros } \rho,
\]
which yields the analytic–spectral equivalence:
\[
\RH \iff \Spec(L_{\sym}) \subset \R,
\]
as claimed.
\end{proof}


% Additional lemma: preservation of spectral multiplicity via determinant
\begin{lemma}[Spectral Multiplicity Preservation]
\label{lem:multiplicity_preservation}

Let \( \rho \in \C \) be a nontrivial zero of the Riemann zeta function \( \zetaR(s) \), and define its canonical spectral image
\[
\mu_\rho := \frac{1}{i(\rho - \tfrac{1}{2})}.
\]
Then \( \mu_\rho \in \Spec(L_{\sym}) \) appears with algebraic multiplicity equal to the order of vanishing of \( \zetaR(s) \) at \( \rho \).

\medskip

\noindent
This multiplicity correspondence follows from the Hadamard factorization of the completed zeta function \( \XiR(s) \), which governs the zero structure of the normalized Carleman–\(\zeta\)-regularized Fredholm determinant of \( L_{\sym} \in \TC(\HPsi) \cap \KC(\HPsi) \):
\[
\det\nolimits_\zeta(I - \lambda L_{\sym}) = \frac{\XiR(\tfrac{1}{2} + i\lambda)}{\XiR(\tfrac{1}{2})}.
\]

Taking the logarithmic derivative, we obtain a meromorphic function whose poles correspond to the spectral values:
\[
\frac{d}{d\lambda} \log \det\nolimits_{\zeta}(I - \lambda L_{\sym}) = \sum_{\rho} \frac{m_\rho}{\lambda - \mu_\rho},
\]
where \( m_\rho \) is the multiplicity of the zero \( \rho \) of \( \zetaR \), and \( \mu_\rho \) is its spectral image. This expansion reflects the classical Hadamard product representation of \( \XiR(s) \), and matches the spectral resolvent trace identity for trace-class self-adjoint operators.

\medskip

\noindent
Since \( L_{\sym} \) is compact and self-adjoint, its spectrum consists of isolated real eigenvalues with finite algebraic multiplicity. The residues of the logarithmic derivative coincide with these multiplicities. Therefore, the multiplicity of each spectral point \( \mu_\rho \) matches exactly the order of vanishing of \( \zetaR(s) \) at \( \rho \), as claimed.
\end{lemma}

\begin{proof}
The canonical determinant identity (\thmref{thm:det_identity_revised}) asserts:
\[
\det\nolimits_{\zeta}(I - \lambda L_{\sym})
= \frac{\XiR\left(\tfrac{1}{2} + i\lambda\right)}{\XiR\left(\tfrac{1}{2}\right)},
\]
where \( \XiR(s) \) admits the classical Hadamard factorization over the nontrivial zeros \( \rho \) of \( \zetaR(s) \):
\[
\XiR(s) = \XiR\left(\tfrac{1}{2}\right)
\prod_\rho \left(1 - \frac{s - \tfrac{1}{2}}{\rho - \tfrac{1}{2}}\right)
\exp\left( \frac{s - \tfrac{1}{2}}{\rho - \tfrac{1}{2}} \right).
\]

\medskip

Define the canonical spectral parameter \( \mu_\rho := \dfrac{1}{i(\rho - \tfrac{1}{2})} \). Then:
\[
\det\nolimits_\zeta(I - \lambda L_{\sym}) = 0 \quad \Longleftrightarrow \quad \lambda = \frac{1}{\mu_\rho},
\]
and the order of vanishing of the determinant at \( \lambda = 1/\mu_\rho \) matches the order of vanishing of \( \zetaR(s) \) at \( \rho \).

\medskip

By \lemref{lem:log_derivative_determinant}, the logarithmic derivative of the determinant satisfies the trace identity:
\[
\frac{d}{d\lambda} \log \det\nolimits_\zeta(I - \lambda L_{\sym})
= \Tr\left( (I - \lambda L_{\sym})^{-1} L_{\sym} \right),
\]
which is meromorphic with poles precisely at the reciprocal spectral values \( \lambda = 1/\mu_\rho \). The residue at each such pole equals the algebraic multiplicity of the corresponding eigenvalue \( \mu_\rho \in \Spec(L_{\sym}) \).

\medskip

Since \( L_{\sym} \in \TC(\HPsi) \cap \KC(\HPsi) \) is compact and self-adjoint, its spectrum is real and discrete, and the eigenvalues have finite algebraic multiplicities. The spectral trace calculus thus confirms that the poles of the logarithmic derivative coincide (in both location and multiplicity) with those arising from the Hadamard factorization of \( \XiR(s) \).

\medskip

Therefore, the order of vanishing of \( \zetaR(s) \) at each nontrivial zero \( \rho \) equals the algebraic multiplicity of the spectral point \( \mu_\rho \in \Spec(L_{\sym}) \), completing the analytic–spectral correspondence.
\end{proof}


% Spectral reality lemma: RH ⇔ spectrum real
\begin{lemma}[Reality of Spectrum Equivalent to \(\RH\)]
\label{lem_reality_of_spectrum_and_rh}

Let \( L_{\sym} \in \TC(\HPsi) \) be the canonical self-adjoint trace-class operator constructed from the completed Riemann zeta function \( \XiR(s) \). Let \( \rho = \tfrac{1}{2} + i\gamma \) be a nontrivial zero of \( \zetaR(s) \), and define its associated spectral image:
\[
\mu_\rho := \frac{1}{i(\rho - \tfrac{1}{2})} = \frac{1}{\gamma}.
\]

Then:
\[
\mu_\rho \in \R \iff \Re(\rho) = \tfrac{1}{2}.
\]

\medskip

\noindent
Consequently, the Riemann Hypothesis is equivalent to the spectral reality of the canonical operator:
\[
\Spec(L_{\sym}) \subset \R \iff \RH.
\]
\end{lemma}

\begin{proof}[Proof of \lemref{lem:reality_of_spectrum_and_rh}]
Assume first that all zeros of the function \( \lambda \mapsto \Xi(\tfrac{1}{2} + i\lambda) \) lie on the real axis. Then the canonical determinant identity
\[
\det\nolimits_\zeta(I - \lambda L_{\mathrm{sym}}) = \frac{\Xi(\tfrac{1}{2} + i\lambda)}{\Xi(\tfrac{1}{2})}
\]
has zeros only for real \( \lambda \). Since this determinant arises from the spectrum of a compact, self-adjoint operator \( L_{\mathrm{sym}} \), the location of its zeros implies that all eigenvalues \( \mu_\rho \in \mathbb{R} \), by the spectral theorem and the identity \( \mu_\rho = \lambda_\rho^{-1} \).

\medskip
\noindent
Conversely, suppose \( L_{\mathrm{sym}} \in \mathcal{C}_1(H_{\Psi_\alpha}) \) is self-adjoint and all eigenvalues \( \mu_\rho \in \mathbb{R} \). Then by the canonical spectral encoding
\[
\mu_\rho = \frac{1}{i}(\rho - \tfrac{1}{2}),
\]
it follows that
\[
\rho = \tfrac{1}{2} + \frac{1}{i\mu_\rho}.
\]
Since \( \mu_\rho \in \mathbb{R} \), this implies \( \operatorname{Re}(\rho) = \tfrac{1}{2} \). Therefore, all nontrivial zeros of \( \zeta(s) \) lie on the critical line, and the Riemann Hypothesis holds.
\end{proof}


% Boxed corollary: RH ⇔ spectrum real
\begin{corollary}[Equivalence of RH with Spectrum Reality]
\label{cor:spectrum_real_equiv_rh}

Let \( L_{\sym} \in \TC(\HPsi) \cap \KC(\HPsi) \) be the canonical compact, self-adjoint, trace-class operator whose Carleman–\(\zeta\)-regularized Fredholm determinant satisfies:
\[
\det\nolimits_\zeta(I - \lambda L_{\sym})
= \frac{\XiR\left(\tfrac{1}{2} + i\lambda\right)}{\XiR\left(\tfrac{1}{2}\right)},
\qquad \forall \lambda \in \C.
\]

Then the Riemann Hypothesis is equivalent to the spectral reality of \( L_{\sym} \):
\[
\boxed{
\Spec(L_{\sym}) \subset \R
\quad \Longleftrightarrow \quad \RH
}
\]

\noindent
That is, all nontrivial zeros of the Riemann zeta function \( \zetaR(s) \) lie on the critical line \( \Re(s) = \tfrac{1}{2} \) if and only if every eigenvalue of \( L_{\sym} \) is real.

\medskip

\noindent
This equivalence follows directly from \thmref{thm:eq_of_rh}, and rests analytically on the determinant identity in \thmref{thm:det_identity_revised}, the spectral bijection in \thmref{thm:spectral_zero_bijection_revised}, and the multiplicity preservation proven in \lemref{lem:multiplicity_preservation}.

\medskip

It constitutes the operator-theoretic core of the analytic–spectral reformulation of the Riemann Hypothesis.
\end{corollary}

\begin{proof}[Proof of Corollary~\ref{cor:spectrum-real-equiv-rh}]
This is an immediate consequence of Lemma~\ref{lem:reality-of-spectrum-and-rh}. For each nontrivial zero \( \rho = \beta + i\gamma \) of \( \zeta(s) \), we define the associated eigenvalue
\[
\mu_\rho := \frac{1}{i(\rho - \tfrac{1}{2})}.
\]

By Lemma~\ref{lem:reality-of-spectrum-and-rh}, we have:
\[
\mu_\rho \in \mathbb{R} \quad \Longleftrightarrow \quad \beta = \tfrac{1}{2}.
\]
Therefore, all \( \mu_\rho \in \operatorname{Spec}(L_{\mathrm{sym}}) \) are real if and only if all nontrivial zeros \( \rho \) lie on the critical line \( \Re(\rho) = \tfrac{1}{2} \). This is precisely the Riemann Hypothesis.

Hence,
\[
\operatorname{Spec}(L_{\mathrm{sym}}) \subset \mathbb{R} \quad \iff \quad \mathrm{RH}.
\]
\end{proof}
% 

% Physics analogy remark
\begin{remark}[Spectral Physics Perspective]
The equivalence \( \mathrm{RH} \iff \operatorname{Spec}(L_{\mathrm{sym}}) \subset \mathbb{R} \) admits a speculative interpretation in the context of quantum mechanics. Under the spectral map \( \rho = \tfrac{1}{2} + i\gamma \mapsto \mu_\rho := \tfrac{1}{\gamma} \), the canonical operator \( L_{\mathrm{sym}} \) can be formally viewed as a Hamiltonian with inverse arithmetic energy levels. Its heat trace resembles a quantum partition function with singular short-time behavior, and its determinant has parallels with spectral free energy. See Appendix~\ref{app:spectral-physics-link} for further discussion of this physical analogy.
\end{remark}

% Determinant trace-log identity is used but not proved here
\medskip
\noindent
For the analytic justification of the trace–logarithmic derivative identity, see Lemma~\ref{lem:log-derivative-determinant} in Chapter~\ref{sec:heat-kernel-asymptotics} and the supporting analysis in Appendix~\ref{app:heat-kernel-construction}.

%------------------------------------------------------------------
\subsection{Uniqueness of Spectral Realization}

% This result builds directly on the spectral-zero bijection and the determinant identity,
% enforcing uniqueness within the trace-class, self-adjoint realization space.
\begin{theorem}[Uniqueness of Spectral Realization]
\label{thm:uniqueness_realization}

Let \( L \in \TC(\HPsi) \cap \KC(\HPsi) \) be a compact, self-adjoint, trace-class operator on the exponentially weighted Hilbert space
\[
\HPsi := L^2(\R, e^{\alpha |x|} \, dx), \qquad \alpha > \pi.
\]

Suppose \( L \) satisfies the canonical zeta-regularized determinant identity:
\[
\det\nolimits_\zeta(I - \lambda L) = \frac{\XiR\left( \tfrac{1}{2} + i\lambda \right)}{\XiR\left( \tfrac{1}{2} \right)},
\quad \forall \lambda \in \C,
\]
where \( \XiR(s) \) is the completed Riemann zeta function, entire of order one and exact exponential type \( \pi \). Assume the normalization:
\[
\det\nolimits_\zeta(I) = 1.
\]

Then \( L \) is unitarily equivalent to the canonical operator \( L_{\sym} \in \TC(\HPsi) \). That is, there exists a unitary operator
\[
U \colon \HPsi \to \HPsi \quad \text{such that} \quad L = U L_{\sym} U^{-1}.
\]

\medskip
\noindent
In particular:
\begin{itemize}
  \item The spectrum of \( L \), including all algebraic multiplicities, coincides with that of \( L_{\sym} \);
  \item \( L_{\sym} \) is the unique (up to unitary equivalence) compact, self-adjoint, trace-class realization of the completed zeta function’s canonical spectral determinant;
  \item The analytic data encoded in \( \XiR(s) \)—via its Hadamard factorization and spectral trace regularization—rigidly determines the operator-theoretic structure of \( L_{\sym} \).
\end{itemize}
\end{theorem}

\begin{proof}[Proof of \thmref{thm:uniqueness_realization}]
Let \( L \in \TC(\HPsi) \cap \KC(\HPsi) \) be a compact, self-adjoint, trace-class operator on the weighted Hilbert space \( \HPsi := L^2(\R, e^{\alpha|x|}dx) \), with fixed \( \alpha > \pi \). Suppose:
\[
\det\nolimits_\zeta(I - \lambda L) = \frac{\XiR\left(\tfrac{1}{2} + i\lambda\right)}{\XiR\left(\tfrac{1}{2}\right)}
= \det\nolimits_\zeta(I - \lambda L_{\sym}),
\quad \forall \lambda \in \C.
\]

\paragraph{Step 1: Spectral Data from Determinant Identity.}
By classical trace-class determinant theory (see~\cite[Thm. 4.2]{Simon2005TraceIdeals}), the normalized Carleman–\(\zeta\)-regularized determinant admits the product representation:
\[
\det\nolimits_\zeta(I - \lambda L)
= \prod_{n=1}^\infty (1 - \lambda \mu_n),
\]
where \( \{ \mu_n \} \subset \R \setminus \{0\} \) are the nonzero eigenvalues of \( L \), counted with algebraic multiplicity. Since this determinant agrees identically with that of \( L_{\sym} \), and both are entire functions of order one normalized by \(\det\nolimits_\zeta(I) = 1\), we conclude:
\[
\Spec(L) = \Spec(L_{\sym}),
\quad \text{as multisets}.
\]

\paragraph{Step 2: Spectral Equivalence Implies Unitary Equivalence.}
Since \( L \) and \( L_{\sym} \) are both compact, self-adjoint operators on the same separable Hilbert space \( \HPsi \), and since their spectra (with multiplicities) coincide, the spectral theorem implies that \( L \) is unitarily equivalent to \( L_{\sym} \). That is, there exists a unitary operator
\[
U \colon \HPsi \to \HPsi
\quad \text{such that} \quad
L = U L_{\sym} U^{-1}.
\]

\paragraph{Conclusion.}
The canonical operator \( L_{\sym} \) is thus uniquely determined (up to unitary equivalence) among all compact, self-adjoint, trace-class operators realizing the normalized spectral determinant identity for \( \XiR(s) \). The analytic fingerprint of \( \XiR \)—its order-one entire structure, exponential type, and Hadamard factorization—rigidly determines the operator-theoretic data of \( L_{\sym} \), completing the proof.
\end{proof}


% Spectral rigidity lemma: determinant identity forces spectral agreement
\begin{lemma}[Spectral Rigidity from Determinant Identity]
\label{lem:spectral_rigidity_determinant}

Let \( L_1, L_2 \in \TC(\HPsi) \cap \KC(\HPsi) \) be compact, self-adjoint, trace-class operators on the exponentially weighted Hilbert space \( \HPsi := L^2(\R, e^{\alpha|x|}dx) \) with \( \alpha > \pi \).

Suppose their Carleman–\(\zeta\)-regularized Fredholm determinants coincide:
\[
\det\nolimits_\zeta(I - \lambda L_1) = \det\nolimits_\zeta(I - \lambda L_2),
\quad \forall \lambda \in \C,
\]
with both normalized at the origin:
\[
\det\nolimits_\zeta(I) = 1.
\]

Then \( L_1 \) and \( L_2 \) have identical nonzero spectra, including algebraic multiplicities:
\[
\Spec(L_1) \setminus \{0\} = \Spec(L_2) \setminus \{0\}
\quad \text{as multisets}.
\]

If both operators act on the same Hilbert space, then the spectral theorem implies they are unitarily equivalent.
\end{lemma}

\begin{proof}[Proof of Lemma~\ref{lem:spectral_rigidity_determinant}]
Let \( L_1, L_2 \in \mathcal{S}_1(H_{\Psi_\alpha}) \) be compact, self-adjoint trace-class operators such that
\[
\det\nolimits_\zeta(I - \lambda L_1) = \det\nolimits_\zeta(I - \lambda L_2), \quad \forall \lambda \in \mathbb{C},
\]
and assume \( \det\nolimits_\zeta(I) = 1 \) for both.

\paragraph{Step 1: Entire Determinants Encode Spectra.}
For compact, self-adjoint operators in \( \mathcal{S}_1 \), the zeta-regularized Fredholm determinant has the canonical Hadamard product:
\[
\det\nolimits_\zeta(I - \lambda L_j) = \prod_{\mu \in \Spec(L_j) \setminus \{0\}} (1 - \lambda \mu)^{\operatorname{mult}_{L_j}(\mu)},
\quad j = 1,2.
\]
Since the two determinant functions coincide as entire functions of order one and exponential type \( \pi \), with identical normalization at \( \lambda = 0 \), it follows that their sets of zeros (with multiplicity) coincide. That is,
\[
\Spec(L_1) \setminus \{0\} = \Spec(L_2) \setminus \{0\}
\quad \text{as multisets}.
\]

\paragraph{Step 2: Spectral Theorem Completion.}
If both \( L_1 \) and \( L_2 \) act on the same Hilbert space and are self-adjoint with matching spectra and multiplicities, then by the spectral theorem for compact self-adjoint operators, there exists a unitary operator \( U \colon H_{\Psi_\alpha} \to H_{\Psi_\alpha} \) such that
\[
L_2 = U L_1 U^{-1}.
\]

\paragraph{Conclusion.}
The determinant identity, combined with trace-class structure and normalization, rigidly determines the operator spectrum up to unitary equivalence. This completes the proof.
\end{proof}
% 

% Determinant uniquely fixes spectral multiset
\begin{lemma}[Determinant Identity Fixes the Spectrum]
\label{lem:determinant_fixes_spectrum}
Let \( L \in \mathcal{S}_1(H_{\Psi_\alpha}) \) be a compact, self-adjoint trace-class operator with
\[
\det\nolimits_\zeta(I - \lambda L) = \frac{\Xi(\tfrac{1}{2} + i\lambda)}{\Xi(\tfrac{1}{2})},
\]
for all \( \lambda \in \mathbb{C} \), and assume \( \operatorname{Tr}(L) = 0 \).

Then the nonzero spectrum of \( L \), counted with algebraic multiplicity, coincides with that of the canonical operator \( L_{\mathrm{sym}} \). That is,
\[
\Spec(L) \setminus \{0\} = \Spec(L_{\mathrm{sym}}) \setminus \{0\},
\quad \text{as multisets}.
\]
\end{lemma}
% 
\begin{proof}[Proof of \lemref{lem:determinant_fixes_spectrum}]
Let \( f(\lambda) := \det\nolimits_\zeta(I - \lambda L) \), and suppose
\[
f(\lambda) = \frac{\XiR(\tfrac{1}{2} + i\lambda)}{\XiR(\tfrac{1}{2})}
= \det\nolimits_\zeta(I - \lambda L_{\sym}),
\quad \forall \lambda \in \C,
\]
where \( L \in \TC(\HPsi) \cap \KC(\HPsi) \) is compact, self-adjoint, trace-class, and satisfies \( \Tr(L) = 0 \).

\paragraph{Step 1: Entire Function Identity and Trace Normalization.}
Both determinant functions are entire of order one and exponential type \( \pi \), and normalized so that \( f(0) = 1 \). The trace-zero condition removes any exponential prefactor ambiguity in their Hadamard factorization—i.e., no term of the form \( e^{a\lambda} \) appears.

\paragraph{Step 2: Logarithmic Derivative and Spectral Poles.}
The logarithmic derivative of the determinant is governed by the resolvent trace formula:
\[
\frac{d}{d\lambda} \log f(\lambda)
= \Tr\left[(I - \lambda L)^{-1} L\right],
\]
which is meromorphic with simple poles at \( \lambda = 1/\mu \) for each nonzero eigenvalue \( \mu \in \Spec(L) \), with residue equal to the algebraic multiplicity of \( \mu \).

Since the determinant agrees with that of \( L_{\sym} \), these poles match those of the canonical model, and thus:
\[
\Spec(L) \setminus \{0\} = \Spec(L_{\sym}) \setminus \{0\}
\quad \text{as multisets}.
\]

\paragraph{Conclusion.}
The spectral data of \( L \), away from zero, is completely encoded by the determinant under the trace normalization condition. Therefore, \( L \) and \( L_{\sym} \) have identical nonzero spectra, completing the proof.
\end{proof}


%------------------------------------------------------------------
\subsection{Canonical Closure of the Spectral Program}

\begin{lemma}[Canonical Closure of the Spectral Model]
\label{lem:canonical_closure}

Let \( L \in \TC(\HPsi) \cap \KC(\HPsi) \) be a compact, self-adjoint, trace-class operator on the exponentially weighted Hilbert space
\[
\HPsi := L^2(\R, e^{\alpha |x|} \, dx), \qquad \alpha > \pi,
\]
and suppose \( L \) satisfies the normalized spectral determinant identity:
\[
\det\nolimits_\zeta(I - \lambda L) = \frac{\XiR\left(\tfrac{1}{2} + i\lambda\right)}{\XiR\left(\tfrac{1}{2}\right)}, \qquad \forall \lambda \in \C,
\]
with normalization \( \det\nolimits_\zeta(I) = 1 \), and where \( \XiR(s) \) is the completed Riemann zeta function.

\medskip
\noindent
Then:
\begin{enumerate}
  \item The nonzero spectrum of \( L \) coincides with that of the canonical operator \( L_{\sym} \), as multisets with algebraic multiplicities:
  \[
  \Spec(L) \setminus \{0\} = \Spec(L_{\sym}) \setminus \{0\};
  \]
  
  \item \( L \) is unitarily equivalent to \( L_{\sym} \): there exists a unitary operator \( U \colon \HPsi \to \HPsi \) such that
  \[
  L = U L_{\sym} U^{-1};
  \]
  
  \item \( L_{\sym} \) is the unique (up to unitary equivalence) compact, self-adjoint trace-class operator whose zeta-regularized determinant realizes the spectral identity associated with \( \XiR(s) \);
  
  \item If \( \widetilde{L} \in \TC \) satisfies the same determinant identity but is not self-adjoint, then \( \widetilde{L} \) is similar to \( L_{\sym} \) in the algebraic sense: there exists an invertible operator \( S \in \mathcal{B}(\HPsi) \) such that
  \[
  \widetilde{L} = S L_{\sym} S^{-1},
  \]
  preserving the nonzero spectrum and multiplicities, though not necessarily realized via a unitary conjugation.
\end{enumerate}

\medskip
\noindent
Hence, the canonical spectral determinant associated with \( \XiR(s) \), under trace-class and self-adjointness, uniquely determines the operator \( L_{\sym} \) up to unitary equivalence, and rigidly constrains all other determinant-realizing models to algebraic similarity. This completes the canonical closure of the spectral model.
\thmref{thm:canonical_operator_realization}
\thmref{thm:det_identity_revised}
\thmref{thm:eq_of_rh}
\end{lemma}

\begin{proof}[Proof of \lemref{lem:canonical_closure}]
By assumption, \( L \in \TC(\HPsi) \cap \KC(\HPsi) \) is compact, self-adjoint, and satisfies the normalized spectral determinant identity:
\[
\det\nolimits_\zeta(I - \lambda L) = \frac{\XiR\left(\tfrac{1}{2} + i\lambda\right)}{\XiR\left(\tfrac{1}{2}\right)}, \quad \forall \lambda \in \C.
\]
This identity is canonically associated with the operator \( L_{\sym} \) via \thmref{thm:det_identity_revised}, and encodes the full spectral data of the Riemann zeta function.

\paragraph{(1) Spectral Equality from Determinant Identity.}
By trace-class determinant theory (see~\cite[Theorem~4.2]{Simon2005TraceIdeals}), the zeta-regularized determinant encodes the nonzero spectrum of \( L \) as a multiset (including algebraic multiplicities). Since the determinant of \( L \) matches that of \( L_{\sym} \), we conclude:
\[
\Spec(L) \setminus \{0\} = \Spec(L_{\sym}) \setminus \{0\}.
\]
This spectral identity follows from the Hadamard factorization argument in \lemref{lem:spectral_rigidity_determinant}.

\paragraph{(2) Unitary Equivalence for Self-Adjoint Case.}
Both \( L \) and \( L_{\sym} \) are compact, self-adjoint operators on the same separable Hilbert space \( \HPsi \), with matching spectra and multiplicities. By the spectral theorem for compact self-adjoint operators (see~\cite[Theorem~VI.16]{ReedSimon1980I}), there exists a unitary operator \( U \colon \HPsi \to \HPsi \) such that
\[
L = U L_{\sym} U^{-1}.
\]

\paragraph{(3) Uniqueness of the Canonical Realization.}
The above shows that \( L_{\sym} \) is unique up to unitary equivalence within the class of compact, self-adjoint, trace-class operators realizing the spectral determinant identity for \( \XiR(s) \). This confirms the uniqueness result of \thmref{thm:uniqueness_realization}.

\paragraph{(4) Similarity Class for Non-Self-Adjoint Realizations.}
Suppose \( \widetilde{L} \in \TC(\HPsi) \) is not self-adjoint but still satisfies the same determinant identity. Then it must have the same nonzero spectral multiset as \( L_{\sym} \), including multiplicities. While lack of normality may prevent diagonalizability or self-adjointness, spectral similarity implies the existence of an invertible operator \( S \in \mathcal{B}(\HPsi) \) such that
\[
\widetilde{L} = S L_{\sym} S^{-1}.
\]
This shows that \( \widetilde{L} \) lies in the similarity class of \( L_{\sym} \), even if not in its unitary equivalence class.

\paragraph{Conclusion.}
The spectral determinant identity associated with \( \XiR(s) \), together with trace-class compactness and self-adjointness, canonically determines the operator \( L_{\sym} \) up to unitary equivalence. Any non-self-adjoint realization is algebraically similar to this canonical model, thus completing the closure of the spectral program.
\end{proof}


%------------------------------------------------------------------

% Final logical closure of the analytic-spectral architecture.
\subsection*{Chapter Summary}

This chapter establishes the canonical spectral encoding of the nontrivial zeros of the Riemann zeta function via the compact, self-adjoint operator \( L_{\sym} \in \TC(\HPsi) \). The key results are as follows:

\begin{itemize}
  \item \lemref{lem:zero_to_eigenvalue_injection} — Every nontrivial zero \( \rho = \tfrac{1}{2} + i\gamma \) of \( \zeta(s) \) defines a nonzero eigenvalue of \( L_{\sym} \) via the spectral map
  \[
  \mu_\rho := \frac{1}{i(\rho - \tfrac{1}{2})} = \frac{1}{\gamma} \in \Spec(L_{\sym}).
  \]

  \item \lemref{lem:spectral_exhaustivity} — Every nonzero eigenvalue \( \mu \in \Spec(L_{\sym}) \setminus \{0\} \) arises from such a zero \( \rho \): surjectivity of the map.

  \item \lemref{lem:spectral_multiplicity_matching} — The algebraic multiplicity of \( \mu_\rho \) matches the order of vanishing of \( \zeta(s) \) at \( \rho \):
  \[
  \operatorname{mult}(\mu_\rho) = \operatorname{ord}_\rho(\zeta).
  \]

  \item \lemref{lem:spectral_symmetry} — The spectrum of \( L_{\sym} \) is symmetric under reflection:
  \[
  \mu \in \Spec(L_{\sym}) \quad \Longrightarrow \quad -\mu \in \Spec(L_{\sym}),
  \]
  reflecting the functional symmetry \( \zeta(s) = \zeta(1 - s) \).

  \item \lemref{lem:spectral_bijection_consistency} — The spectral map \( \rho \mapsto \mu_\rho \) is a multiplicity-preserving bijection between nontrivial zeros of \( \zeta(s) \) and the nonzero spectrum of \( L_{\sym} \).

  \item \thmref{thm:spectral_zero_bijection_revised} — Consolidated bijection:
  \[
  \zeta(\rho) = 0 \quad \Longleftrightarrow \quad \mu_\rho := \frac{1}{i(\rho - \tfrac{1}{2})} \in \Spec(L_{\sym}),
  \]
  with multiplicities preserved on both sides.
\end{itemize}

\begin{quote}
  \textbf{Remark (Spectral Encoding as Analytic Dual).}~
  This bijection is canonical: the entire spectrum of \( L_{\sym} \) is uniquely determined by the analytic structure of \( \Xi(s) \) via the Fredholm determinant identity. Conversely, as shown in \corref{cor:spectrum_determines_zeta}, the spectrum of \( L_{\sym} \), including multiplicities, fully determines \( \Xi(s) \), and thus recovers the nontrivial zeros of \( \zeta(s) \).
\end{quote}

For a visual overview, see Table~\ref{fig:schematic_bijection_table}, which illustrates the spectral encoding map \( \rho \mapsto \mu_\rho \), the symmetry \( \pm \mu \), and the bijection structure between zeros and eigenvalues.

\medskip
\noindent
This spectral encoding underpins the reformulation of the Riemann Hypothesis in Chapter~\ref{sec:spectral_implications}, where we prove:
\[
\mathrm{RH} \quad \Longleftrightarrow \quad \Spec(L_{\sym}) \subset \R.
\]

