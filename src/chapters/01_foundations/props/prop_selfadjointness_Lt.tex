\begin{proposition}[Self-Adjointness of \( L_t \)]
\label{prop:selfadjointness_Lt}
Let \( H_\Psi := L^2(\R, \Psi(x)\, dx) \) be a weighted Hilbert space, where \( \Psi \colon \R \to (0,\infty) \) is smooth and satisfies
\[
\Psi(x) \sim e^{\alpha|x|} \quad \text{as } |x| \to \infty,
\]
for some \( \alpha > 0 \); that is, there exist constants \( c_1, c_2 > 0 \) such that
\[
c_1 e^{\alpha |x|} \le \Psi(x) \le c_2 e^{\alpha |x|}, \qquad \forall x \in \R.
\]

Let \( \phi_t \in \Schwartz \) be a real-valued, even mollifier, and define the convolution operator
\[
(L_t f)(x) := \int_{\R} \phi_t(x - y)\, f(y)\, dy, \qquad f \in \Schwartz,
\]
with dense domain \( \Dom(L_t) := \Schwartz \subset H_\Psi \).

Assume:
\begin{itemize}
    \item[\textup{(i)}] \( L_t \) extends to a bounded linear operator on \( H_\Psi \) (see \propref{prop:boundedness_Lt_weighted});
    \item[\textup{(ii)}] \( L_t \) is symmetric on the dense core \( \Schwartz \subset H_\Psi \); that is,
    \[
    \langle L_t f, g \rangle_{H_\Psi} = \langle f, L_t g \rangle_{H_\Psi}, \qquad \forall f, g \in \Schwartz,
    \]
    as established in \propref{prop:symmetry_Lt_Schwartz}.
\end{itemize}

Then the bounded operator \( L_t \in \mathcal{B}(H_\Psi) \) is self-adjoint:
\[
L_t^* = L_t.
\]

\paragraph{Consequences.}
As a bounded self-adjoint operator, \( L_t \) admits a spectral resolution via the spectral theorem:
\[
L_t = \int_{\sigma(L_t)} \lambda\, dE_\lambda,
\]
where \( E_\lambda \) is a projection-valued measure. This enables the analytic definition of:
\[
\det\nolimits_\zeta(I - \lambda L_t), \qquad e^{-t L_t^2}, \qquad \text{and} \qquad \zeta_{L_t}(s),
\]
as functions of \( \lambda \) and \( s \), respectively. These constructions underpin the canonical determinant identity and heat kernel asymptotics in subsequent chapters.
\end{proposition}
