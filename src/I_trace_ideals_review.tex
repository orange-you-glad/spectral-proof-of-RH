\section{Trace Ideals and Operator Norms}
\label{app:trace-ideals-review}

This appendix reviews key analytic properties of compact operators, Schatten–von Neumann ideals, trace norms, and determinant expansions.

These constructions underpin the spectral realization of the completed zeta function \( \Xi(s) \) via the canonical trace-class operator \( L_{\mathrm{sym}} \), and rely on foundational results in functional analysis and operator theory, particularly~\cite{Simon2005TraceIdeals, ReedSimon1980I, ReedSimon1975II}.

For the determinant identity linking the spectrum of \( L_{\mathrm{sym}} \) to the zeros of \( \Xi(s) \), see Theorem~\ref{thm:det-identity-revised} in Chapter~\ref{sec:determinant-identity}.

\subsection*{Schatten–von Neumann Ideals}

Let \( H \) be a separable complex Hilbert space. For \( p \in [1, \infty) \), define:
\[
\mathcal{C}_p(H) := \left\{ T \in \mathcal{K}(H) \;\middle|\; \|T\|_{\mathcal{C}_p} := \left( \sum_{n=1}^\infty \sigma_n(T)^p \right)^{1/p} < \infty \right\},
\]
where \( \sigma_n(T) \) are the singular values of \( T \). Each \( \mathcal{C}_p \) is a Banach *-ideal, closed under adjoints and invariant under unitary conjugation.

\medskip
Important cases:
\begin{itemize}
  \item \( \mathcal{C}_1(H) \): trace-class operators;
  \item \( \mathcal{C}_2(H) \): Hilbert–Schmidt operators.
\end{itemize}

\subsection*{Trace-Class Operators}

If \( T \in \mathcal{C}_1(H) \), then the trace
\[
\operatorname{Tr}(T) := \sum_n \langle T e_n, e_n \rangle
\]
is independent of the choice of orthonormal basis \( \{ e_n \} \). One has:
\[
|\operatorname{Tr}(T)| \le \|T\|_{\mathcal{C}_1}, \qquad \|T\| \le \|T\|_{\mathcal{C}_1}.
\]

\subsection*{Structural Properties}

\begin{itemize}
  \item Ideal hierarchy: \( \mathcal{C}_1 \subset \mathcal{C}_2 \subset \mathcal{K} \subset \mathcal{B} \), where \( \mathcal{K} \) is the ideal of compact operators and \( \mathcal{B} \) the bounded operators.
  \item All \( \mathcal{C}_p \) operators (for \( p < \infty \)) are compact.
  \item Trace cyclicity: \( \operatorname{Tr}(AB) = \operatorname{Tr}(BA) \) when \( A \in \mathcal{C}_1 \), \( B \in \mathcal{B} \).
\end{itemize}

\subsection*{Fredholm and Carleman Determinants}

Let \( T \in \mathcal{C}_1(H) \). The classical Fredholm determinant is defined by
\[
\det(I + T) := \prod_n (1 + \lambda_n),
\]
with logarithmic expansion
\[
\log \det(I + T) = \sum_{k=1}^\infty \frac{(-1)^{k+1}}{k} \operatorname{Tr}(T^k),
\]
where \( \lambda_n \) are the eigenvalues of \( T \), counted with multiplicity.

The Carleman zeta-regularized determinant is defined (formally) for \( |\lambda| < \|T\|_{\mathcal{C}_1}^{-1} \) by
\[
\det\nolimits_\zeta(I - \lambda T) := \exp\left( -\sum_{k=1}^\infty \frac{\lambda^k}{k} \operatorname{Tr}(T^k) \right).
\]
For compact, self-adjoint trace-class operators—such as \( L_{\mathrm{sym}} \)—this determinant extends analytically and defines an entire function of order one~\cite{Simon2005TraceIdeals}.

\subsection*{Spectral Mapping Diagram}

\begin{figure}[ht]
\centering
% See above: TikZ diagram code already given
\caption{Zeta zeros \( \rho \) map via \( \rho \mapsto \mu_\rho = \frac{1}{i(\rho - \tfrac{1}{2})} \) to eigenvalues of \( L_{\mathrm{sym}} \).}
\label{fig:spectral-mapping}
\end{figure}

\subsection*{Conclusion}

The theory of trace-class operators, Schatten ideals, and zeta-regularized determinants underpins the canonical identity
\[
\det\nolimits_\zeta(I - \lambda L_{\mathrm{sym}}) = \frac{\Xi(\tfrac{1}{2} + i\lambda)}{\Xi(\tfrac{1}{2})},
\]
for the self-adjoint operator \( L_{\mathrm{sym}} \in \mathcal{C}_1(H_{\Psi_\alpha}) \). This framework justifies the analytic trace expansions, heat kernel asymptotics, and spectral rigidity principles that culminate in the proof of the Riemann Hypothesis in this manuscript.
