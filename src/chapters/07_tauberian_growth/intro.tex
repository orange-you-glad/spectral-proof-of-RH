\subsection*{Introduction}
\label{sec:intro_tauberian_growth}

This chapter establishes the spectral asymptotics of the canonical compact, self-adjoint, trace-class operator
\[
\Lsym \in \TC(\HPsi),
\]
via Laplace–Tauberian inversion of its squared heat semigroup. Using refined short-time asymptotics of the spectral trace
\[
\Theta(t) := \Tr(e^{-t \Lsym^2}),
\]
we derive precise growth estimates for the eigenvalue counting function
\[
N(\Lambda) := \#\left\{ n \in \N : \mu_n^2 \le \Lambda \right\},
\]
and confirm analytic compatibility with the Riemann–von Mangoldt zero-counting law.

\paragraph{Objectives.}
\begin{itemize}
  \item \textit{Spectral Growth Bounds.}  
  From the envelope
  \[
  \Theta(t) \asymp t^{-1/2} \quad \text{as } t \to 0^+,
  \]
  established in Chapter~\ref{sec:heat_kernel_asymptotics}, we deduce via \lemref{lem:spectral_convexity_estimate} the subconvex bound
  \[
  N(\Lambda) = O(\Lambda^{1/2+\varepsilon}) \quad \text{for all } \varepsilon > 0.
  \]
  The effective spectral dimension is \( d = 1 \), consistent with leading growth \( N(\Lambda) \sim \Lambda^{1/2} \log \Lambda \).

  \item \textit{Tauberian Inversion.}  
  Applying Korevaar’s refinement of Karamata theory (\lemref{lem:log_corrected_tauberian_estimate}), we invert the Laplace relation
  \[
  \Theta(t) = \int_0^\infty e^{-t\lambda} \, dN(\lambda),
  \]
  and obtain the sharp asymptotic
  \[
  N(\Lambda) = \frac{\sqrt{\Lambda}}{\pi} \log \Lambda + O(\sqrt{\Lambda}) \quad \text{as } \Lambda \to \infty.
  \]
  This follows from classifying \( \Theta \in \mathcal{R}_{1/2}^{\log} \), i.e., regularly varying with index \( 1/2 \) and logarithmic modulation.

  \item \textit{Zeta-Theoretic Compatibility.}  
  The derived asymptotic matches the classical Riemann–von Mangoldt formula for zeta zeros. This match, proven in \corref{cor:zeta_compatibility}, confirms that the spectrum of \( \Lsym \) encodes the zero distribution of \( \zeta(s) \).
\end{itemize}

\paragraph{Analytic Inputs from Chapter~\ref{sec:heat_kernel_asymptotics}.}

\begin{center}
\renewcommand{\arraystretch}{1.3}
\begin{tabularx}{\textwidth}{|c|X|X|}
\hline
\textbf{Source} & \textbf{Analytic Quantity} & \textbf{Role in This Chapter} \\
\hline
\propref{prop:two_sided_heat_trace_bounds} &
\( \Tr(e^{-t \Lsym^2}) \asymp t^{-1/2} \) &
Ensures admissibility for Tauberian envelope bounds \\

\propref{prop:short_time_heat_expansion} &
\( \Tr(e^{-t \Lsym^2}) = \frac{1}{\sqrt{4\pi t}} \log(1/t) + \cdots \) &
Triggers Korevaar inversion with logarithmic precision \\

\lemref{lem:log_derivative_determinant} &
Spectral zeta representation via heat trace &
Links eigenvalue asymptotics to determinant structure \\
\hline
\end{tabularx}
\end{center}

\paragraph{Link to Spectral Equivalence.}
By matching the eigenvalue counting function with the Riemann--von Mangoldt formula, \lemref{lem:log_corrected_tauberian_estimate} supplies the final analytic ingredient used in Chapter~\ref{sec:spectral_implications}.  In particular, it shows that
\[
  \RH \iff \Spec(\Lsym) \subset \R,
\]
as formalized in \thmref{thm:rh_spectral_closure}.  The Tauberian analysis here therefore closes the loop from heat kernel bounds to operator-theoretic reformulation of RH.
