\section*{Reader’s Roadmap and Structural Summary}

This manuscript constructs a canonical trace-class operator \( L_{\sym} \in \TC(\HPsi) \), acting on the exponentially weighted Hilbert space \( \HPsi = L^2(\R, e^{\alpha|x|} dx) \), such that its Carleman-regularized Fredholm determinant recovers the completed Riemann zeta function:
\[
\det\nolimits_\zeta(I - \lambda L_{\sym}) = \frac{\XiR(\tfrac{1}{2} + i\lambda)}{\XiR(\tfrac{1}{2})}.
\]
The construction is entirely analytic and operator-theoretic: all results are proven without assuming the Riemann Hypothesis or the spectral bijection. The spectrum of \( L_{\sym} \) corresponds bijectively to the nontrivial zeros \( \rho \in \Spec(\zetaR) \) via the canonical map \( \mu_\rho = \frac{1}{i}(\rho - \tfrac{1}{2}) \), and multiplicities are matched.

\paragraph{Prologue Preview.}
The Prologue motivates the spectral determinant identity by linking it to the Euler product, the Weil explicit formula, and trace formulas in number theory. This framing situates the canonical operator within known arithmetic structures and geometric dualities.

\subsection*{Comparative Framework}

This work diverges sharply from earlier approaches:

\begin{itemize}
  \item \textbf{Hilbert--Pólya:} Postulates—but does not construct—a self-adjoint operator encoding RH.
  \item \textbf{Connes--Meyer:} Works with trace formulas on adelic spaces using singular distributions, lacking Hilbert space compactness.
  \item \textbf{Deninger:} Proposes Arakelov-style cohomological frameworks but without analytic trace-class realization.
  \item \textbf{Berry--Keating:} Models a formal Hamiltonian \( H = xp \), with no spectral compactness or determinant formulation.
\end{itemize}

In contrast, this manuscript gives an explicit compact, self-adjoint, trace-class operator \( L_{\sym} \) whose determinant identity establishes RH as a spectral equivalence.

\subsection*{Global Strategy}

The manuscript is structured into nine analytic chapters, each logically self-contained and modular:

\begin{enumerate}
  \item \textbf{Foundations (Ch.~\ref{sec:foundations}):} Defines \( \HPsi \), proves kernel decay, and establishes trace-class embedding.
  \item \textbf{Operator Construction (Ch.~\ref{sec:operator_construction}):} Constructs \( L_{\sym} \) via mollified convolution, proves compactness and essential self-adjointness.
  \item \textbf{Determinant Identity (Ch.~\ref{sec:determinant_identity}):} Proves entire determinant identity via Laplace-regularized heat traces.
  \item \textbf{Spectral Correspondence (Ch.~\ref{sec:spectral_correspondence}):} Proves multiplicity-preserving spectral bijection; resolves surjectivity rigorously.
  \item \textbf{Heat Kernel Asymptotics (Ch.~\ref{sec:heat_kernel_asymptotics}):} Analyzes \( e^{-t L_{\sym}^2} \), establishes analytic bounds and trace expansions; supported by \appref{app:heat_kernel_construction}.
  \item \textbf{Spectral Implications (Ch.~\ref{sec:spectral_implications}):} Proves \( \Spec(L_{\sym}) \subset \R \iff \RH \); confirms spectral uniqueness from determinant identity.
  \item \textbf{Tauberian Growth (Ch.~\ref{sec:tauberian_growth}):} Applies Korevaar’s Tauberian theory to extract asymptotic spectral density.
  \item \textbf{Spectral Rigidity (Ch.~8):} Proves positivity of trace densities and the necessity of real spectrum for RH.
  \item \textbf{Logical Closure (Ch.~\ref{sec:logical_closure}):} Concludes \( \RH \Leftarrow \Spec(L_{\sym}) \subset \R \) by inverse determinant uniqueness.
\end{enumerate}

\subsection*{Notation and Core Operators}

See \appref{app:notation_summary}. Central constructions include:
\begin{itemize}
  \item \( \HPsi = L^2(\R, e^{\alpha|x|} dx) \), \( \alpha > \pi \);
  \item \( L_{\sym} := \lim_{t \to 0^+} L_t \): mollified convolution limit;
  \item \( \XiR(s) \): completed Riemann zeta function;
  \item \( \mu_\rho = \frac{1}{i}(\rho - \tfrac{1}{2}) \): spectral encoding of \( \zeta \)-zeros;
  \item \( \det\nolimits_\zeta \): Carleman-regularized Fredholm determinant.
\end{itemize}

\subsection*{Logical Acyclicity and Proof Dependency}

The manuscript is formally acyclic: no result assumes RH, real spectrum, or spectral surjectivity prior to proof. All analytic inputs are closed in previous chapters. See \appref{app:dependency_graph} for the DAG of logical dependencies. Chapter~\ref{sec:heat_kernel_asymptotics} supplies all trace bounds used in Chapter~\ref{sec:spectral_implications}.

\subsection*{Suggested Reading Paths}

\begin{itemize}
  \item \textbf{Analysts / Spectral Theorists:} Start with Chapters 1–3 and \appref{app:trace_ideals_review}.
  \item \textbf{Number Theorists:} Begin with the Prologue, then Chapters 3–6.
  \item \textbf{Mathematical Physicists:} Begin with \appref{app:spectral_physics_link}, then Chapters 5–7.
\end{itemize}

\subsection*{Conclusion}

The determinant identity is proved unconditionally. All analytic estimates are explicitly derived. The spectral equivalence
\[
\Spec(L_{\sym}) \subset \R \quad \Longleftrightarrow \quad \RH
\]
is established as an analytic theorem in Chapter~\ref{sec:logical_closure}. No step requires unproven assumptions beyond trace-class and spectral calculus.

\medskip
\noindent
For refinements and extensions—including subleading heat trace terms and conjectural realizations of automorphic and motivic \( L \)-functions—see \appref{app:heat_kernel_refinements} and \appref{app:additional_structures}.
