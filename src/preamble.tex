%==============================
% Preamble: Core Packages and Custom Definitions
%==============================

%--- Essential Math Packages ---
\usepackage{amsmath, amssymb, amsfonts, mathrsfs}
\usepackage{graphicx}
\usepackage{lmodern}
\usepackage{microtype}
\usepackage{tikz}
\usetikzlibrary{arrows.meta, positioning, cd}

%--- Table and Caption Formatting ---
\usepackage{booktabs}
\usepackage{tabularx}
\usepackage{caption}

%--- Hyperlinking and Cross-Referencing ---
\usepackage{hyperref}
\hypersetup{
    bookmarksdepth=section,
    breaklinks=true,
    colorlinks=true,
    linkcolor=black,
    citecolor=black,
    urlcolor=black,
    pdfauthor={R.A. Jacob Martone},
    pdftitle={A Canonical Spectral Determinant Realizing the Riemann Hypothesis},
    pdfsubject={Mathematics -- Zeta Functions, Spectral Theory, Operator Algebras},
    pdfkeywords={Riemann Hypothesis, Spectral Determinant, Trace-Class Operator, Fredholm Theory, Zeta Regularization, Heat Kernel, Hilbert Space, Operator Theory, Entire Functions}
}

\usepackage[capitalize,nameinlink]{cleveref}

%==============================
% Custom Macro Definitions
%==============================

% Number Systems
\newcommand{\R}{\mathbb{R}}
\newcommand{\C}{\mathbb{C}}
\newcommand{\Z}{\mathbb{Z}}
\newcommand{\N}{\mathbb{N}}
\newcommand{\Q}{\mathbb{Q}}

% Function Spaces and Weighted Spaces
\newcommand{\Lp}[1]{L^{#1}(\R)}
\newcommand{\Schwartz}{\mathcal{S}(\R)}
\newcommand{\HPsi}{H_{\Psi}}
\newcommand{\Hilb}{\mathcal{H}}
\newcommand{\PsiAlpha}[1]{\exp\!\bigl(\alpha |#1|\bigr)}
\newcommand{\PsiAlphaSpace}[1]{L^2(\R, \PsiAlpha{#1}\,dx)}

% Domain
\DeclareMathOperator{\Dom}{Dom}

% Operator-Theoretic Objects
\newcommand{\TC}{\mathcal{B}_1}
\newcommand{\KC}{\mathcal{K}}
\newcommand{\SC}{\mathcal{S}_2}
\newcommand{\Tr}{\operatorname{Tr}}
\newcommand{\Spec}{\operatorname{Spec}}
\newcommand{\Ran}{\operatorname{Ran}}
\newcommand{\Ker}{\operatorname{Ker}}
\newcommand{\HS}{\mathrm{HS}}
\newcommand{\Lop}{L_t}
\newcommand{\Lsym}{L_{\mathrm{sym}}}

% Special Functions and Function Classes
\newcommand{\XiR}{\Xi}
\newcommand{\zetaR}{\zeta}
\newcommand{\PsiW}{\Psi}
\newcommand{\PW}[1]{\mathcal{PW}_{#1}(\R)}

% Fourier Transforms
\newcommand{\FT}{\mathcal{F}}
\newcommand{\ft}[1]{\widehat{#1}}

% Math Utilities
\newcommand{\abs}[1]{\left|#1\right|}
\newcommand{\norm}[1]{\left\|#1\right\|}
\newcommand{\ip}[2]{\left\langle #1, #2 \right\rangle}

% Structural Labels
\newcommand{\sym}{\mathrm{sym}}
\newcommand{\op}{\mathrm{op}}

% Named Global Hypotheses (Robust Inline Constants)
\newcommand{\RH}{\mathsf{RH}}     % Riemann Hypothesis
\newcommand{\GRH}{\mathsf{GRH}}   % Generalized RH
\newcommand{\ABC}{\mathsf{ABC}}   % ABC Conjecture

% Cross-Reference Shortcuts
\newcommand{\thmref}[1]{Theorem~\ref{#1}}
\newcommand{\lemref}[1]{Lemma~\ref{#1}}
\newcommand{\corref}[1]{Corollary~\ref{#1}}
\newcommand{\propref}[1]{Proposition~\ref{#1}}
\newcommand{\defref}[1]{Definition~\ref{#1}}
\newcommand{\secref}[1]{Section~\ref{#1}}
\newcommand{\appref}[1]{Appendix~\ref{#1}}
\newcommand{\eqnref}[1]{\eqref{#1}}
\newcommand{\remref}[1]{Remark~\ref{#1}}

%==============================
% Spectral Generalization (Chapter 9)
%==============================

% Automorphic Spectral Notation
\newcommand{\alphapi}{\alpha_{\pi}}
\newcommand{\Hpsipi}{H_{\Psi_{\pi}}}
\newcommand{\Ltpi}{L_t^{(\pi)}}
\newcommand{\Lsympi}{L_{\mathrm{sym}}^{(\pi)}}
\newcommand{\ktpi}{k_t^{(\pi)}}
\newcommand{\KpiT}{K_t^{(\pi)}}
\newcommand{\muRho}{\mu_\rho}
\newcommand{\SpecPi}{\Spec(\Lsympi)}

% Spectral Determinants for L-functions
\newcommand{\detz}{\det\nolimits_{\zeta}}
\newcommand{\XiPi}{\Xi(s, \pi)}
\newcommand{\XiPiHalf}{\Xi\left(\tfrac{1}{2}, \pi\right)}
\newcommand{\XiPiShifted}{\Xi\left(\tfrac{1}{2} + i\lambda, \pi\right)}
\newcommand{\ZetaDetPi}{\detz(I - \lambda \Lsympi)}
\newcommand{\SpecEncodingPi}{\left\{ \mu_\rho := \frac{1}{i}(\rho - \tfrac{1}{2}) : L(\rho, \pi) = 0 \right\}}
\newcommand{\GL}{\mathrm{GL}}

% Validation and DAG Labels
\newcommand{\postref}[1]{Postulate~\ref{#1}}
\newcommand{\thmpostref}[1]{Postulated Theorem~\ref{#1}}

%==============================
% Theorem Environments
%==============================

\theoremstyle{plain}
\newtheorem{theorem}{Theorem}[section]
\newtheorem{lemma}[theorem]{Lemma}
\newtheorem{proposition}[theorem]{Proposition}
\newtheorem{corollary}[theorem]{Corollary}
\newtheorem{conjecture}[theorem]{Conjecture}
\newtheorem{postulate}[theorem]{Postulate}

\theoremstyle{definition}
\newtheorem{definition}[theorem]{Definition}
\newtheorem{example}[theorem]{Example}

\theoremstyle{remark}
\newtheorem{remark}[theorem]{Remark}
\newtheorem*{auditnote}{Audit Note}

%==============================
% Preview Box for Spectral Highlights
%==============================
\usepackage[most]{tcolorbox}
\newtcolorbox{previewbox}{
  colback=gray!5,
  colframe=black,
  fonttitle=\bfseries,
  coltitle=black,
  sharp corners,
  boxrule=0.4pt,
  title={Spectral Preview},
  before skip=10pt,
  after skip=10pt,
  breakable
}

%==============================
% References Environment
%==============================
\newenvironment{references}{
  \medskip
  \noindent\textbf{References.}
  \begin{list}{}{\leftmargin=1.5em \itemindent=-1.5em \itemsep=0pt}
  \item[]
}{
  \end{list}
  \medskip
}

%==============================
% TikZ and PGFPlots
%==============================
\usepackage{pgfplots}
\pgfplotsset{compat=1.17}

%==============================
% Wide Check Symbol for Inverse Transforms
%==============================
\makeatletter
\DeclareRobustCommand{\widecheck}[1]{
  \mathpalette\@widecheck{#1}
}
\def\@widecheck#1#2{%
  \setbox\z@\hbox{\m@th#1#2}%
  \setbox\tw@\hbox{\m@th#1\widehat{\box\z@}}%
  \setbox\tw@\hbox to \wd\z@{\hss\box\tw@\hss}%
  {\vbox{\offinterlineskip
      \box\tw@
      \vskip -\ht\z@
      \box\z@}}}
\makeatother

% Suppress TOC entries for \subsection*, \subsubsection*, etc.
\makeatletter
\renewcommand{\@seccntformat}[1]{%
  \ifcsname if@#1star\endcsname
    \expandafter\@gobble
  \else
    \csname the#1\endcsname\quad
  \fi
}
\makeatother
