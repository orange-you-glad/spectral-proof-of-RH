\begin{proof}[Proof of \lemref{lem:Lsym_square_selfadjoint}]
We establish essential self-adjointness of \( L_{\sym}^2 \) on the Schwartz core \( \Schwartz \subset H_{\Psi_\alpha} \).

\paragraph{Step 1: Invariance of the Core.}
Since \( L_{\sym} \) is defined via convolution with kernel \( k \in \Schwartz \), and convolution preserves \( \Schwartz \), we have
\[
L_{\sym} : \Schwartz \to \Schwartz, \quad \Rightarrow \quad L_{\sym}^2 : \Schwartz \to \Schwartz.
\]
Thus, the domain \( \Schwartz \) is invariant under \( L_{\sym} \) and its square.

\paragraph{Step 2: Symmetry.}
Because \( L_{\sym} \) is self-adjoint, it follows that
\[
\langle L_{\sym}^2 f, g \rangle_{H_{\Psi_\alpha}} = \langle f, L_{\sym}^2 g \rangle_{H_{\Psi_\alpha}}, \qquad \forall f, g \in \Schwartz,
\]
so \( L_{\sym}^2 \) is symmetric on \( \Schwartz \).

\paragraph{Step 3: Nelson's Analytic Vector Criterion.}
Since \( L_{\sym}^2 \) preserves \( \Schwartz \), and each \( f \in \Schwartz \) satisfies
\[
\| (L_{\sym}^2)^n f \| < C_n,
\]
the elements of \( \Schwartz \) are analytic vectors for \( L_{\sym}^2 \). Thus, by Nelson’s analytic vector theorem~\cite[Thm.~X.36]{ReedSimon1975II}, we conclude:
\[
L_{\sym}^2 \text{ is essentially self-adjoint on } \Schwartz.
\]

\paragraph{Step 4: Spectral Discreteness.}
Since \( L_{\sym} \in \TC(H_{\Psi_\alpha}) \), we have \( L_{\sym}^2 \in \TC(H_{\Psi_\alpha}) \) as well. Then, by spectral theory of compact self-adjoint operators, the spectrum of \( L_{\sym}^2 \) consists of a discrete sequence of real eigenvalues:
\[
\Spec(L_{\sym}^2) \subset [0, \infty), \quad \text{accumulating only at zero},
\]
each with finite multiplicity.

\paragraph{Conclusion.}
The operator \( L_{\sym}^2 \) is essentially self-adjoint on \( \Schwartz \), with closure admitting a discrete spectral resolution. This confirms the functional calculus infrastructure for defining zeta regularizations and heat kernels.
\end{proof}
