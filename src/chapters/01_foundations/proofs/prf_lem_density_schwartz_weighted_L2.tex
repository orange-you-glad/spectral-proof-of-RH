\begin{proof}[Proof of \lemref{lem:density_schwartz_weighted_L2}]
Fix \( \alpha > \pi \), and define the exponential weight
\[
\Psi_\alpha(x) := e^{\alpha |x|}.
\]
Then the associated weighted Hilbert space
\[
H_{\Psi_\alpha} := L^2(\mathbb{R}, \Psi_\alpha(x)\,dx)
\]
is equipped with inner product
\[
\langle f, g \rangle_{H_{\Psi_\alpha}} := \int_{\mathbb{R}} f(x)\, \overline{g(x)}\, \Psi_\alpha(x)\, dx.
\]

\medskip
\noindent\textbf{Step 1: Unitary Reduction to Flat \( L^2 \).}
Define the unitary transformation
\[
U \colon H_{\Psi_\alpha} \to L^2(\mathbb{R}), \quad (Uf)(x) := \Psi_\alpha(x)^{1/2} f(x),
\]
with inverse
\[
U^{-1}(h)(x) := \Psi_\alpha(x)^{-1/2} h(x).
\]
Then for all \( f, g \in H_{\Psi_\alpha} \), one has
\[
\langle Uf, Ug \rangle_{L^2} = \langle f, g \rangle_{H_{\Psi_\alpha}}, \quad \text{and} \quad \|Uf\|_{L^2} = \|f\|_{H_{\Psi_\alpha}}.
\]

\medskip
\noindent\textbf{Step 2: Density in Standard \( L^2 \).}
Let \( f \in H_{\Psi_\alpha} \), and define \( g := Uf \in L^2(\mathbb{R}) \). Since \( \mathcal{S}(\mathbb{R}) \) is dense in \( L^2(\mathbb{R}) \), for any \( \varepsilon > 0 \) there exists \( \varphi \in \mathcal{S}(\mathbb{R}) \) such that
\[
\|g - \varphi\|_{L^2} < \varepsilon.
\]
Define \( f_\varepsilon := U^{-1}(\varphi) = \Psi_\alpha^{-1/2}(x) \varphi(x) \in H_{\Psi_\alpha} \). Then
\[
\|f - f_\varepsilon\|_{H_{\Psi_\alpha}} = \|Uf - \varphi\|_{L^2} < \varepsilon.
\]

\medskip
\noindent\textbf{Step 3: Image of Schwartz is Schwartz.}
Since \( \varphi \in \mathcal{S}(\mathbb{R}) \) and \( \Psi_\alpha^{-1/2} \in C^\infty(\mathbb{R}) \) with exponential decay, the product
\[
f_\varepsilon(x) = \varphi(x) \cdot \Psi_\alpha(x)^{-1/2}
\]
belongs to \( \mathcal{S}(\mathbb{R}) \), by standard closure properties of Schwartz space under multiplication by smooth functions of sub-exponential growth. Therefore,
\[
f_\varepsilon \in \mathcal{S}(\mathbb{R}) \cap H_{\Psi_\alpha}, \quad \text{and} \quad \|f - f_\varepsilon\|_{H_{\Psi_\alpha}} < \varepsilon.
\]

\medskip
\noindent\textbf{Conclusion.}
As \( f \in H_{\Psi_\alpha} \) and \( \varepsilon > 0 \) were arbitrary, we conclude that \( \mathcal{S}(\mathbb{R}) \) is dense in \( H_{\Psi_\alpha} \). This holds for all \( \alpha > 0 \), and in particular for \( \alpha > \pi \), which ensures compatibility with the exponential type of Paley–Wiener kernels derived from \( \Xi \).

\medskip
\noindent\textbf{Functional Consequence.}
The density \( \mathcal{S}(\mathbb{R}) \subset H_{\Psi_\alpha} \) provides a common core domain for the convolution operators \( L_t \) and their limit \( L_{\mathrm{sym}} \). In particular, it ensures:
\begin{itemize}
    \item Symmetry of \( L_t \) under reflection: \( L_t^* = L_t \) on \( \mathcal{S}(\mathbb{R}) \);
    \item Essential self-adjointness of \( L_{\mathrm{sym}} \) on \( \mathcal{S}(\mathbb{R}) \);
    \item Justification of heat trace and determinant convergence via mollifier approximation.
\end{itemize}
\end{proof}
