\begin{lemma}[Spectrum Reality and RH Equivalence]
\label{lem:reality_of_spectrum_and_rh}
If all zeros of the function \( \lambda \mapsto \Xi(\tfrac{1}{2} + i\lambda) \) lie on the real axis, then the spectrum of the canonical operator \( L_{\mathrm{sym}} \in \mathcal{C}_1(H_{\Psi_\alpha}) \) is real.

Conversely, if \( L_{\mathrm{sym}} \in \mathcal{C}_1(H_{\Psi_\alpha}) \) is self-adjoint and all eigenvalues \( \mu_\rho \in \mathbb{R} \), then the corresponding nontrivial zeros \( \rho \in \mathbb{C} \) of the Riemann zeta function satisfy \( \operatorname{Re}(\rho) = \tfrac{1}{2} \). That is, the Riemann Hypothesis holds.

\medskip

\noindent
In other words, the spectral condition
\[
\operatorname{Spec}(L_{\mathrm{sym}}) \subset \mathbb{R}
\quad \Longleftrightarrow \quad
\rho \in \mathbb{C},\ \zeta(\rho) = 0 \Rightarrow \operatorname{Re}(\rho) = \tfrac{1}{2}.
\]
\end{lemma}
