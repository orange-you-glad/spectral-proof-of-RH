\begin{proof}[Proof of \thmref{thm:spectral_zero_bijection_revised}]
Let \( \rho = \tfrac{1}{2} + i\gamma \) be a nontrivial zero of the Riemann zeta function \( \zetaR(s) \), and define the corresponding spectral value
\[
\mu_\rho := \frac{1}{i(\rho - \tfrac{1}{2})}.
\]

\paragraph{Injectivity.}
By \lemref{lem:zero_to_eigenvalue_injection}, each nontrivial zero \( \rho \) maps to a unique nonzero eigenvalue \( \mu_\rho \in \Spec(L_{\sym}) \), and distinct zeros yield distinct eigenvalues. This confirms the injectivity of the spectral map \( \rho \mapsto \mu_\rho \).

\paragraph{Surjectivity.}
By \lemref{lem:spectral_exhaustivity}, every nonzero eigenvalue \( \mu \in \Spec(L_{\sym}) \setminus \{0\} \) arises from some nontrivial zero \( \rho \in \C \), satisfying \( \mu = \mu_\rho \). This proves surjectivity.

\paragraph{Multiplicity Preservation.}
By \lemref{lem:spectral_multiplicity_matching}, the algebraic multiplicity of each eigenvalue \( \mu_\rho \) equals the order of vanishing of \( \zetaR(s) \) at the corresponding zero \( \rho \). Hence, the spectral map preserves multiplicities.

\paragraph{Conclusion.}
The map
\[
\rho \longmapsto \mu_\rho := \frac{1}{i(\rho - \tfrac{1}{2})}
\]
defines a canonical multiplicity-preserving bijection between the multiset of nontrivial zeros of \( \zetaR(s) \) and the multiset of nonzero eigenvalues of \( L_{\sym} \in \TC(\HPsi) \). This correspondence is uniquely determined by the regularized Fredholm determinant identity and fully realizes the spectral encoding of the Riemann zeta function's zeros.
\end{proof}
