\begin{proof}[Proof of \lemref{lem:weighted_L1_inverse_FT_xi}]
Let \( \alpha > \pi \), and define
\[
\widehat{\Xi}(x) := \frac{1}{2\pi} \int_{\mathbb{R}} e^{i\lambda x} \, \Xi\left( \tfrac{1}{2} + i\lambda \right)\, d\lambda.
\]

\medskip
\noindent\textbf{Step 1: Spectral Profile and Exponential Type.}
Define
\[
\phi(\lambda) := \Xi\left( \tfrac{1}{2} + i\lambda \right).
\]
As shown in Lemma~\ref{lem:xi_growth_bound}, this function is entire of order one and exponential type \( \pi \), i.e., \( \phi \in PW_\pi(\mathbb{R}) \), with
\[
|\phi(\lambda)| \le A_1\, e^{\pi |\lambda|}, \quad \forall \lambda \in \mathbb{R},
\]
due to Hadamard factorization and asymptotics for \( \Gamma(s/2)\zeta(s) \) on vertical lines \cite[Ch.~3]{Levin1996EntireLectures}, \cite[Ch.~2]{Titchmarsh1986Zeta}.

\medskip
\noindent\textbf{Step 2: Paley–Wiener Decay.}
By the Paley–Wiener theorem for exponential type \( \pi \) \cite[Thm.~3.2.4]{Levin1996EntireLectures}, the inverse Fourier transform
\[
\widehat{\phi}(x) = \frac{1}{2\pi} \int_{\mathbb{R}} e^{i\lambda x} \phi(\lambda)\, d\lambda
\]
satisfies
\[
\widehat{\phi} \in L^1(\mathbb{R}, e^{-\beta |x|} dx), \quad \forall \beta > \pi.
\]
Hence for any fixed \( \alpha > \pi \),
\[
\widehat{\Xi}(x) = \widehat{\phi}(x) \in L^1(\mathbb{R}, e^{-\alpha |x|} dx).
\]

\medskip
\noindent\textbf{Step 3: Quantitative Bound.}
For any \( \varepsilon > 0 \), there exists \( C_\alpha > 0 \) such that
\[
|\widehat{\Xi}(x)| \le C_\alpha\, e^{-(\alpha - \varepsilon)|x|}, \quad \forall x \in \mathbb{R}.
\]
Therefore,
\[
\int_{\mathbb{R}} |\widehat{\Xi}(x)|\, e^{-\alpha |x|} dx
\le C_\alpha \int_{\mathbb{R}} e^{-(\alpha + \varepsilon)|x|} dx
= \frac{2 C_\alpha}{\alpha + \varepsilon} < \infty.
\]

\medskip
\noindent\textbf{Conclusion.}
Thus \( \widehat{\Xi} \in L^1(\mathbb{R}, \Psi_\alpha^{-1}(x) dx) \), where \( \Psi_\alpha(x) := e^{\alpha |x|} \). Define the convolution kernel
\[
K(x,y) := \widehat{\Xi}(x - y).
\]
Then \( K \in L^1(\mathbb{R}^2, \Psi_\alpha(x)\Psi_\alpha(y)\, dx\,dy) \), and the associated operator
\[
L f(x) := \int_{\mathbb{R}} \widehat{\Xi}(x - y)\, f(y)\, dy
\]
belongs to \( \mathcal{C}_1(H_{\Psi_\alpha}) \) by Simon’s trace-class kernel criterion \cite[Thm.~4.2]{Simon2005TraceIdeals}.
\end{proof}
