\begin{proof}[Proof of \thmref{thm:truth_of_rh}]
Let \( L_{\mathrm{sym}} \in \mathcal{C}_1(H_{\Psi_\alpha}) \) be the canonical compact, self-adjoint operator constructed via mollified convolution from the inverse Fourier transform of the completed zeta function \( \Xi(s) \).

By the determinant identity (see \thmref{thm:det_identity_revised}),
\[
\det\nolimits_\zeta(I - \lambda L_{\mathrm{sym}}) = \frac{\Xi\left( \tfrac{1}{2} + i\lambda \right)}{\Xi\left( \tfrac{1}{2} \right)},
\]
and the spectral bijection (see \thmref{thm:spectral_zero_bijection_revised}, \lemref{lem:spectral_encoding_injection}) ensures that each nontrivial zero \( \rho \in \mathbb{C} \) corresponds to a spectral point
\[
\mu_\rho := \frac{1}{i}(\rho - \tfrac{1}{2}) \in \operatorname{Spec}(L_{\mathrm{sym}}) \setminus \{0\}.
\]

\paragraph{Spectral Reality.}
From the analytic construction of \( L_{\mathrm{sym}} \) and its mollifier convergence in \cref{sec:operator_construction}, self-adjointness and real spectrum are established. Hence:
\[
\operatorname{Spec}(L_{\mathrm{sym}}) \subset \mathbb{R}.
\]

\paragraph{Implication for Zeros.}
For each \( \rho \), we have:
\[
\mu_\rho \in \mathbb{R} \quad \Rightarrow \quad \frac{1}{i}(\rho - \tfrac{1}{2}) \in \mathbb{R}
\quad \Rightarrow \quad \rho - \tfrac{1}{2} \in i\mathbb{R}
\quad \Rightarrow \quad \operatorname{Re}(\rho) = \tfrac{1}{2}.
\]

\paragraph{Conclusion.}
The spectral determinant fully encodes the nontrivial zeros of \( \zeta(s) \), and \( L_{\mathrm{sym}} \) has real spectrum. Hence:
\[
\zeta(\rho) = 0 \quad \Longrightarrow \quad \operatorname{Re}(\rho) = \tfrac{1}{2},
\]
and the Riemann Hypothesis is proven.
\end{proof}
