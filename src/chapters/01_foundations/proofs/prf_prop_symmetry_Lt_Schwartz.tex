\begin{proof}[Proof of \propref{prop:symmetry_Lt_Schwartz}]
Let \( f, g \in \mathcal{S}(\mathbb{R}) \subset H_\Psi \), and define the convolution operator
\[
(L_t f)(x) := \int_{\mathbb{R}} \phi_t(x - y)\, f(y)\, dy,
\]
where \( \phi_t \in \mathcal{S}(\mathbb{R}) \) is real-valued and even.

\medskip
\noindent\textbf{Step 1: Compute the Weighted Inner Product.}
We compute:
\[
\begin{aligned}
\langle L_t f, g \rangle_{H_\Psi}
&= \int_{\mathbb{R}} (L_t f)(x)\, \overline{g(x)}\, \Psi(x)\, dx \\
&= \iint_{\mathbb{R}^2} \phi_t(x - y)\, f(y)\, \overline{g(x)}\, \Psi(x)\, dy\, dx.
\end{aligned}
\]

\medskip
\noindent\textbf{Step 2: Fubini and Symmetry of \( \phi_t \).}
Since \( \phi_t \in \mathcal{S}(\mathbb{R}) \) and \( \Psi(x) \sim e^{\alpha |x|} \), the integrand is absolutely integrable. By Fubini’s theorem:
\[
\langle L_t f, g \rangle_{H_\Psi}
= \int_{\mathbb{R}} f(y) \left( \int_{\mathbb{R}} \phi_t(x - y)\, \overline{g(x)}\, \Psi(x)\, dx \right) dy.
\]

Since \( \phi_t \) is even, \( \phi_t(x - y) = \phi_t(y - x) \), and
\[
(L_t g)(y) = \int_{\mathbb{R}} \phi_t(y - x)\, g(x)\, dx = \int_{\mathbb{R}} \phi_t(x - y)\, g(x)\, dx.
\]
Therefore,
\[
\overline{(L_t g)(y)} = \int_{\mathbb{R}} \phi_t(x - y)\, \overline{g(x)}\, dx.
\]

\medskip
\noindent\textbf{Step 3: Complete the Symmetry Argument.}
Substituting into the outer integral:
\[
\langle L_t f, g \rangle_{H_\Psi}
= \int_{\mathbb{R}} f(y)\, \overline{(L_t g)(y)}\, \Psi(y)\, dy = \langle f, L_t g \rangle_{H_\Psi}.
\]

\medskip
\noindent\textbf{Conclusion.}
This verifies that \( L_t \) is symmetric on the Schwartz core \( \mathcal{S}(\mathbb{R}) \subset H_\Psi \). The symmetry follows directly from the real-valuedness and evenness of \( \phi_t \) and ensures that \( L_t \subset L_t^* \). This property is a foundational step in establishing essential self-adjointness of the strong limit \( L_{\mathrm{sym}} \).
\end{proof}
