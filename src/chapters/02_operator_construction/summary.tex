\subsection*{Summary}
\label{sec:operator_construction_summary}

\textbf{Function Spaces and Spectral Input}
\begin{itemize}
  \item \defref{def:weighted_fourier_space} — The exponentially weighted Hilbert space \( H_{\Psi_\alpha} := L^2(\R, e^{\alpha |x|} dx) \), where \( \alpha > \pi \) ensures trace-class regularity.
  \item \defref{def:canonical_fourier_profile} — Spectral profile \( \phi(\lambda) := \Xi(\tfrac{1}{2} + i\lambda) \), entire and of exponential type \( \pi \).
  \item \defref{def:mollified_fourier_profile} — Mollified profiles \( \varphi_t(\lambda) := e^{-t\lambda^2} \phi(\lambda) \) yield kernel smoothing for trace approximation.
  \item \defref{def:convolution_operators_Lt_Lsym} — Defines convolution operators \( L_t \), with trace-norm limit \( \Lsym := \lim_{t \to 0^+} L_t \in \TC(H_{\Psi_\alpha}) \).
\end{itemize}

\textbf{Kernel Estimates and Trace Convergence}
\begin{itemize}
  \item \lemref{lem:mollified_profile_decay} — Uniform decay of mollified kernels \( k_t \) follows from Paley–Wiener bounds on \( \phi \).
  \item \lemref{lem:trace_class_Lt} — Trace-class inclusion: \( L_t \in \TC(H_{\Psi_\alpha}) \) via Simon’s kernel criterion.
  \item \lemref{lem:trace_norm_convergence_Lt_to_Lsym}, \lemref{lem:trace_norm_rate_convergence} — Quantitative trace-norm convergence \( \|L_t - \Lsym\|_{\TC} \lesssim t^{1/2} \).
  \item \lemref{lem:trace_norm_limit_unique}, \lemref{lem:construction_canonical_data} — Canonicality and mollifier independence of the operator \( \Lsym \).
  \item \lemref{lem:mollifier_independence_kernel_limit} — Kernel convergence is independent of mollifier choice under exponential control.
\end{itemize}

\textbf{Operator Properties and Domain Closure}
\begin{itemize}
  \item \lemref{lem:boundedness_Lsym} — \( \Lsym \in \mathcal{B}(H_{\Psi_\alpha}) \) with controlled norm.
  \item \lemref{lem:core_essential_sa}, \lemref{lem:Lsym_square_selfadjoint} — The Schwartz space \( \Schwartz(\R) \) is a common core for \( \Lsym \) and \( \Lsym^2 \); both are essentially self-adjoint.
  \item \remref{rem:selfadjoint_analytic_vectors} — Analytic vectors derived from mollified evolution validate Nelson’s theorem.
  \item \lemref{lem:heat_semigroup_existence} — The heat semigroup \( \{ e^{-t \Lsym^2} \}_{t > 0} \) is trace class and holomorphic in \( t \).
\end{itemize}

\textbf{Canonical Classification}
\begin{itemize}
  \item \thmref{thm:existence_Lsym} — \( \Lsym \in \TC(H_{\Psi_\alpha}) \) exists as the trace-norm limit of the mollified family.
  \item \thmref{thm:sa_trace_class_Lsym} — Final classification: \( \Lsym \) is compact, self-adjoint, and trace class.
  \item \thmref{thm:trace_zero_Lsym} — Spectral centering: \( \Tr(\Lsym) = 0 \) ensures determinant normalization.
\end{itemize}

\paragraph{Chapter Closure.}
This chapter completes the analytic construction of the canonical operator \( \Lsym \), which is compact, self-adjoint, and trace class. Constructed from mollified inverse Fourier transforms of the completed zeta function \( \Xi(s) \), this operator satisfies all spectral, functional, and convergence conditions required for determinant regularization.

The resulting operator forms the backbone of the spectral determinant identity
\[
\detz(I - \lambda^2 \Lsym^2) = \Xi\left(\tfrac{1}{2} + i\lambda\right),
\]
which is formally established in Chapter~\ref{sec:determinant_identity}. The full spectral realization of RH rests on this canonical trace-class model.
