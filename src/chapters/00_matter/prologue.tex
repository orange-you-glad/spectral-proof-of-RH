\section*{Prologue: Bourbaki and the Future of Proof}
\label{sec:prologue}

\paragraph*{1.1 The Autonomous Mathematician.}
\emph{Bourbaki} is not a project about the Riemann Hypothesis. It is a prototype for what mathematics becomes when proofs are no longer written, but built. Autonomous. Compositional. Self-verifying. At its heart is not an answer, but an architecture—a formal system that does not merely construct proofs, but refactors them, audits them, and compiles them into semantic artifacts. This is not LLM-as-calculator. It is LLM as collaborator, as interlocutor, as dialectical engine in a formal loop:
\[
\text{conjecture} \rightarrow \text{attempt} \rightarrow \text{audit} \rightarrow \text{refactor} \rightarrow \text{validate} \rightarrow \text{render}.
\]
The output is not a paper. It is a reproducible build. Every DAG edge in this manuscript is re-verified on every commit. Every identity is tracked to its analytic kernel. Every expansion is recursively justified. This is not a new proof of RH. It is a new epistemology of proof.

\paragraph*{1.2 Why RH Became the Test Case.}
The Riemann Hypothesis was not chosen for its fame, but for its constraints. RH tolerates no ambiguity. It rewards only structure. It enforces a discipline of equivalence. In a mathematical ecosystem designed to be modular, local, and formally audited, RH serves not as a summit but as a stress test. Its sharp dichotomy—real spectrum versus critical zeros—defines the kind of binary architecture a formal pipeline can encode, stabilize, and audit without heuristic drift. RH demanded a proving system capable of confronting necessity. \emph{Bourbaki} responded—with structure.

\paragraph*{1.3 Workflow in Miniature.}
At the core is a Directed Acyclic Graph (DAG) of irreducible semantic units—lemmas that are not merely steps in a proof, but composable modules in a formal structure. Each is typed, audited, and connected by analytic invariants: spectral bijections, determinant identities, kernel decay, Tauberian asymptotics. The DAG compiles into a Lean 4 formalization (in progress), a typeset manuscript, and a containerized CI pipeline that re-verifies logical closure on every push. Nothing is assumed. Nothing is trusted. Everything is reconstructed—from the Fourier kernel up. Every path to RH is explicit. Every identity is localizable.

\paragraph*{1.4 Mathematical Result, Briefly.}
The central analytic construct is a canonical trace-class operator
\[
\Lsym \in \TC(\HPsi), \qquad \HPsi := L^2(\R, e^{\alpha|x|} dx), \quad \alpha > \pi,
\]
whose zeta-regularized Fredholm determinant satisfies:
\[
\detz(I - \lambda \Lsym) = \frac{\Xi\left(\tfrac{1}{2} + i\lambda\right)}{\Xi\left(\tfrac{1}{2}\right)}.
\]
Through this identity, the nontrivial zeros \( \rho \in \C \) of \( \zeta(s) \) are encoded spectrally:
\[
\rho \mapsto \mu_\rho := \tfrac{1}{i}(\rho - \tfrac{1}{2}) \in \Spec(\Lsym).
\]
By the spectral theorem:
\[
\RH \iff \Spec(\Lsym) \subset \R.
\]
But this manuscript does not merely state the result—it reconstructs it compositionally. No appeal to conjecture. No invocation of unverified arithmetic intuition. Arithmetic truth emerges here not as a hypothesis, but as a spectral invariant.

\paragraph*{1.5 Parallel Construction via Trace Asymptotics.}
Beyond the canonical route through \( \Xi(s) \), we show that RH can also be deduced from heat trace asymptotics alone. A second operator, \( \tilde{L}_{\mathrm{sym}} \in \TC(\HPsi) \), is constructed directly from the short-time expansion of the trace
\[
t \mapsto \Tr(e^{-t \tilde{L}_{\mathrm{sym}}^2}),
\]
without assuming access to the zero set or the completed zeta function. Its spectrum encodes the same bijection to the nontrivial zeros of \( \zeta(s) \), and the Riemann Hypothesis follows from its self-adjointness. This provides a clean, spectral-theoretic route to RH, independently anchored in trace kernel asymptotics and Laplace duality.

\paragraph*{1.6 The Shape of What Comes Next.}
Bourbaki is not a closed object. It is a proving platform. Any operator in this ecosystem is reproducible; any identity, composable. You can lift the framework to automorphic \( L \)-functions. You can generalize the determinant theory to other trace norms. You can replace the Riemann zeta function with zeta elements arising from motives, curves, or noncommutative spaces. This is mathematics built not just to verify one hypothesis, but to support a hundred. And prove them all.

\paragraph*{1.7 Invitation to Audit and Extend.}
Every lemma is modular. Every operator is refactorable. The full DAG structure, Lean verification suite (in progress), and CI runner are open source:
\begin{itemize}
  \item GitHub: \texttt{https://github.com/orange-you-glad/spectral-proof-of-RH}
  \item Formal interlocutor: \texttt{https://orangeyouglad.org/bourbaki}
\end{itemize}
The true experiment is not whether RH is true, but whether formal mathematics can scale. Can you extend it? Optimize it? Add a new spectral branch? Bourbaki was not designed to end a proof. It was designed to begin one.


\subsection*{Main Equivalence — Spectral RH Reformulation}
\hfill \break
\begin{tcolorbox}[colback=gray!3!white, colframe=black!75!white]
We construct a canonical operator \( \Lsym \in \TC(\HPsi) \) on
\[
\HPsi := L^2\big(\mathbb{R}, e^{\alpha |x|} \, dx\big), \qquad \alpha > \pi,
\]
such that:
\begin{enumerate}
  \item The zeta-regularized determinant identity holds:
  \[
  \detz(I - \lambda \Lsym) = \frac{\Xi\left(\tfrac{1}{2} + i\lambda\right)}{\Xi\left(\tfrac{1}{2}\right)}.
  \]
  \item The spectrum of \( \Lsym \) satisfies the bijective correspondence:
  \[
  \rho \mapsto \mu_\rho := \tfrac{1}{i}(\rho - \tfrac{1}{2}) \in \Spec(\Lsym),
  \]
  with multiplicities preserved.
  \item The Riemann Hypothesis holds if and only if
  \[
  \Spec(\Lsym) \subset \mathbb{R}.
  \]
\end{enumerate}
All analytic and operator-theoretic infrastructure—Paley–Wiener kernel decay, trace-norm convergence, heat semigroup regularity, determinant growth—is rigorously modularized and verified across Chapters~1–10. See \thmref{thm:truth_of_rh} and \thmref{thm:truth_of_rh_from_trace_asymptotics} for the dual equivalences.
\end{tcolorbox}

\subsection*{Disclosure of Large-Language-Model Assistance}

The author used OpenAI’s \emph{ChatGPT} (GPT-4o, May 2025 release) as a research and writing assistant. This collaboration included:

\begin{itemize}
  \item Structural scaffolding for chapters and DAG architecture;
  \item Drafting provisional \LaTeX{} for definitions, lemmas, and operator identities;
  \item Symbolic verification of transforms and expansions;
  \item Refactoring of prose for clarity, precision, and formal alignment.
\end{itemize}

All mathematics was independently verified by the author. The LLM is not an author and bears no responsibility for any claim made herein.
