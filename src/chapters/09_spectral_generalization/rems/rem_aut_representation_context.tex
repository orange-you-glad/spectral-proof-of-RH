\begin{remark}[Automorphic Representation Context]
\label{rem:aut_representation_context}
Throughout this chapter, \( \pi \in \mathcal{A}_{\mathrm{cusp}}(\GL_n) \) denotes a fixed unitary cuspidal automorphic representation of the general linear group over the adeles \( \GL_n(\mathbb{A}_\Q) \). We assume the standard normalization of the completed \( L \)-function:
\[
\Xi(s, \pi) := L_\infty(s, \pi) \cdot L(s, \pi),
\]
where \( L_\infty(s, \pi) \) denotes the product of local Gamma factors at the archimedean places, and \( L(s, \pi) \) denotes the finite-part \( L \)-function with Euler product expansion.

We refer to \( \Xi(s, \pi) \) as the spectral profile associated to \( \pi \). It is known by general theory (see~\cite{Cogdell2007Lectures},~\cite{Bump1997AutomorphicForms}) that \( \Xi(s, \pi) \) extends to an entire function of order one, bounded in vertical strips, and satisfying the functional equation
\[
\Xi(s, \pi) = \varepsilon(\pi) \cdot \Xi(1 - s, \widetilde{\pi}),
\]
where \( \widetilde{\pi} \) is the contragredient representation and \( \varepsilon(\pi) \in \C^\times \) is the global root number.

This normalization and analytic continuation underlie all constructions in this chapter. The explicit spectral assumptions are isolated in \defref{def:analytic_properties_XiPi}.
\end{remark}
