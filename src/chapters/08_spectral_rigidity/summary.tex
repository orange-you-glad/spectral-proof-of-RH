\subsection*{Summary}

This chapter reformulates the Riemann Hypothesis as a statement of spectral rigidity: the spectrum and determinant of the canonical operator \( \Lsym \in \TC(\HPsi) \) determine both the location and multiplicity of all nontrivial zeros of the Riemann zeta function.

\textbf{Key Spectral Rigidity Results}
\begin{itemize}
  \item \lemref{lem:spectral_encoding_injection} — Spectral encoding injection:
  \[
  \rho \mapsto \mu_\rho := \tfrac{1}{i}(\rho - \tfrac{1}{2}),
  \]
  sends zeta zeros \( \rho \) into \( \Spec(\Lsym) \setminus \{0\} \), preserving multiplicity.

  \item \lemref{lem:det_zero_implies_spectrum} — Every zero of the determinant corresponds to a spectral eigenvalue:
  \[
  \detz(I - \lambda \Lsym) = 0 \iff \lambda^{-1} \in \Spec(\Lsym).
  \]

  \item \lemref{lem:inject_zero_real_spectrum} — Spectral reality implies critical-line alignment:
  \[
  \Spec(\Lsym) \subset \R \quad \Rightarrow \quad \Re(\rho) = \tfrac{1}{2}.
  \]

  \item \lemref{lem:real_spectrum_implies_rh_rigidity} — This implication holds without assuming spectral simplicity or bijection; it relies only on the determinant identity.

  \item \propref{prop:spectrum_reality_implies_rh_multiplicity} — If the spectrum is real and simple, then all zeta zeros are simple:
  \[
  \Re(\rho) = \tfrac{1}{2}, \quad \operatorname{ord}_\rho(\zeta) = 1.
  \]

  \item \thmref{thm:rh_from_real_simple_spectrum} — Under spectral simplicity and reality, RH and simplicity of all nontrivial zeros follow.

  \item \lemref{lem:spectral_symmetry} — Spectral symmetry:
  \[
  \mu \in \Spec(\Lsym) \Rightarrow -\mu \in \Spec(\Lsym),
  \]
  from functional symmetry of \( \Xi(s) \).

  \item \lemref{lem:trace_distribution_positivity} — The map \( \phi \mapsto \Tr(\phi(\Lsym)) \) defines a positive tempered distribution.

  \item \lemref{lem:trace_distribution_positive} — Positivity extends to all nonnegative \( \phi \in \Schwartz(\R) \); see \remref{rem:functional_calculus_trace}.

  \item \propref{prop:inverse_spectral_rigidity} — Any two trace-class self-adjoint operators with identical determinant and spectrum are unitarily equivalent.
\end{itemize}

\medskip

\textbf{Conclusion and Equivalence}
These results prove the logical implication:
\[
\Spec(\Lsym) \subset \R \quad \Rightarrow \quad \RH,
\]
closing the analytic loop initiated in \thmref{thm:eq_of_rh}. Together with the determinant identity, this rigidity completes the spectral reformulation of RH.

\medskip

\textbf{Remarks}
\begin{itemize}
  \item Spectral simplicity strengthens the RH claim but is not required.
  \item The spectrum–zero correspondence arises from Hadamard structure and Fredholm theory.
  \item Trace positivity reflects harmonic-analytic regularity of the spectral measure.
  \item Under Fourier diagonalization, the trace pairing becomes integration against the spectral spectral measure.
\end{itemize}

\medskip

\textbf{Forward Link}
All results culminate in the final logical equivalence:
\[
\RH \iff \Spec(\Lsym) \subset \R,
\]
proven in \thmref{thm:truth_of_rh}, and logically grounded in the DAG structure of \appref{app:dependency_graph}.
