\begin{remark}[Canonical Operator Framework]
\label{rem:setup_operator_framework}
Let \( H_{\Psi_\alpha} := L^2(\mathbb{R}, e^{\alpha|x|} dx) \) denote the exponentially weighted Hilbert space with fixed weight parameter \( \alpha > \pi \). Throughout this chapter, we consider the canonical operator
\[
L_{\mathrm{sym}} \in \mathcal{C}_1(H_{\Psi_\alpha}),
\]
constructed in Chapter~\ref{sec:operator-construction} as the trace-norm limit of symmetric mollified convolution operators defined via the inverse Fourier transform of the completed Riemann zeta function \( \Xi(s) \).

The operator \( L_{\mathrm{sym}} \) satisfies:
\begin{itemize}
  \item It is compact and self-adjoint with real, discrete spectrum.
  \item It lies in the trace-class \( \mathcal{C}_1(H_{\Psi_\alpha}) \), with uniform control on its heat kernel and determinant asymptotics.
  \item It is uniquely determined by the canonical determinant identity:
  \[
  \det\nolimits_{\zeta}(I - \lambda L_{\mathrm{sym}}) = \frac{\Xi\left( \tfrac{1}{2} + i\lambda \right)}{\Xi\left( \tfrac{1}{2} \right)},
  \]
  valid for all \( \lambda \in \mathbb{C} \), and analytically encoding the entire zero structure of \( \zeta(s) \).
\end{itemize}

These structural properties form the analytic foundation for the lemmas, theorems, and corollaries in this chapter, and are used without further restatement.
\end{remark}
