\begin{lemma}[Weighted \( L^1 \)-Integrability of the Inverse Fourier Transform of \( \Xi \)]
\label{lem:weighted_L1_inverse_FT_xi}
Let \( \alpha > \pi \), and define the centered spectral profile
\[
\phi(\lambda) := \Xi\left( \tfrac{1}{2} + i\lambda \right),
\]
where \( \Xi(s) \) is the completed Riemann zeta function—entire of exponential type \( \pi \) and order one.

Define its inverse Fourier transform:
\[
\widehat{\Xi}(x) := \frac{1}{2\pi} \int_{\mathbb{R}} e^{i\lambda x} \, \phi(\lambda)\, d\lambda,
\]
interpreted in the distributional sense.

Then:
\[
\widehat{\Xi} \in L^1(\mathbb{R}, e^{-\alpha |x|}\, dx),
\]
i.e., there exists \( A_\alpha > 0 \) such that
\[
\int_{\mathbb{R}} |\widehat{\Xi}(x)| \, e^{-\alpha |x|}\, dx \le A_\alpha.
\]

\medskip
\noindent
In particular, defining the exponential weight \( \Psi_\alpha(x) := e^{\alpha |x|} \), we have \( \widehat{\Xi} \in L^1(\mathbb{R}, \Psi_\alpha^{-1}(x)\, dx) \). Therefore, the convolution kernel
\[
K(x,y) := \widehat{\Xi}(x - y)
\]
belongs to \( L^1(\mathbb{R}^2, \Psi_\alpha(x)\Psi_\alpha(y)\, dx\,dy) \), and the associated convolution operator
\[
(L f)(x) := \int_{\mathbb{R}} \widehat{\Xi}(x - y) f(y)\, dy
\]
lies in the trace class \( \operatorname{TC}(H_{\Psi_\alpha}) \) by Simon’s kernel criterion \cite[Thm.~4.2]{Simon2005TraceIdeals}.

\medskip
\noindent
This decay follows from the Paley--Wiener theorem: since \( \phi \in \operatorname{PW}_\pi(\mathbb{R}) \) (see \defref{def:paley_wiener_class}), its inverse Fourier transform satisfies
\[
|\widehat{\Xi}(x)| = \mathcal{O}(e^{-\pi |x|}),
\]
and thus lies in \( L^1(\mathbb{R}, e^{-\alpha |x|} dx) \) for all \( \alpha > \pi \).

\medskip
\noindent
\textbf{Optional.} For explicit pointwise exponential decay and differentiability of \( \widehat{\Xi}(x) \), see \lemref{lem:decay_inverse_fourier_xi}.
\end{lemma}
