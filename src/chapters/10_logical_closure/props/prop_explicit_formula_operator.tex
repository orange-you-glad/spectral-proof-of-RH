\begin{proposition}[Operator-Theoretic Form of Weil’s Explicit Formula]
\label{prop:explicit_formula_operator}

Let \( L_{\sym} \in \mathcal{C}_1(H_{\Psi_\alpha}) \) be the canonical self-adjoint trace-class operator defined by convolution with the inverse Fourier transform of \( \Xi\left(\tfrac{1}{2} + i\lambda\right) \). Let \( \{ \mu_\rho \} \subset \mathbb{R} \) denote its spectrum, where \( \mu_\rho = \tfrac{1}{i}(\rho - \tfrac{1}{2}) \) and \( \zeta(\rho) = 0 \).

Then for any Schwartz-class test function \( h \in \mathcal{S}(\mathbb{R}) \) with inverse Fourier transform \( g := \mathcal{F}^{-1}[h] \in \mathcal{S}(\mathbb{R}) \), the spectral trace pairing satisfies:
\[
\sum_{\rho} h(\mu_\rho) = \sum_{n=1}^{\infty} \frac{\Lambda(n)}{\sqrt{n}} \left( g(\log n) + g(-\log n) \right)
+ \textnormal{(Archimedean contribution)}.
\]

Here:
- \( \Lambda(n) \) is the von Mangoldt function,
- The left-hand side arises as the spectral trace \( \sum h(\mu_\rho) \),
- The right-hand side recovers Weil’s explicit formula in distributional form.

This equality realizes the Riemann explicit formula as a spectral identity for \( L_{\sym} \).
\end{proposition}
