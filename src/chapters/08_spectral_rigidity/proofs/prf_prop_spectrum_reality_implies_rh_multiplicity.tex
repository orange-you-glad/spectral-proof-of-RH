\begin{proof}[Proof of \propref{prop:spectrum_reality_implies_rh_multiplicity}]
Assume:
\begin{itemize}
  \item \( L_{\mathrm{sym}} \in \mathcal{C}_1(H_{\Psi_\alpha}) \) is compact and self-adjoint;
  \item \( \operatorname{Spec}(L_{\mathrm{sym}}) \subset \mathbb{R} \);
  \item All nonzero eigenvalues \( \mu \in \operatorname{Spec}(L_{\mathrm{sym}}) \setminus \{0\} \) are simple.
\end{itemize}

Let
\[
f(\lambda) := \det\nolimits_\zeta(I - \lambda L_{\mathrm{sym}}) = \frac{\Xi\left(\tfrac{1}{2} + i\lambda\right)}{\Xi\left(\tfrac{1}{2}\right)}
\]
be the canonical spectral determinant. As an entire function of exponential type \( \pi \) and genus one, it admits the Hadamard product:
\[
f(\lambda) = \prod_\rho \left(1 - \frac{\lambda}{\mu_\rho} \right) \exp\left( \frac{\lambda}{\mu_\rho} \right),
\]
where \( \mu_\rho := \frac{1}{i}(\rho - \tfrac{1}{2}) \), with \( \rho \in \operatorname{Spec}(\zeta) \), counted with multiplicity.

\paragraph{Step 1: Real Spectrum Implies RH.}
If \( \mu_\rho \in \mathbb{R} \), then
\[
\mu_\rho = \frac{1}{i}(\rho - \tfrac{1}{2}) \in \mathbb{R} \quad \Rightarrow \quad \rho - \tfrac{1}{2} \in i\mathbb{R} \quad \Rightarrow \quad \operatorname{Re}(\rho) = \tfrac{1}{2}.
\]
Hence, all nontrivial zeros lie on the critical line.

\paragraph{Step 2: Simplicity.}
Each eigenvalue \( \mu_\rho \in \operatorname{Spec}(L_{\mathrm{sym}}) \) has multiplicity one. Since the determinant product preserves multiplicity under the map \( \rho \mapsto \mu_\rho \), the order of vanishing of \( \Xi(s) \) at \( s = \tfrac{1}{2} + i\lambda \) is also one:
\[
\operatorname{ord}_\rho(\zeta) = \operatorname{ord}_{\mu_\rho}(f) = 1.
\]

\paragraph{Conclusion.}
Under the assumptions of real and simple spectrum, all nontrivial zeros of \( \zeta(s) \) lie on the critical line and are simple, as claimed.
\end{proof}
