\begin{proof}[Proof of \lemref{lem:inject_zero_real_spectrum}]
Let \( \rho = \tfrac{1}{2} + i\gamma \in \C \) be a nontrivial zero of \( \zetaR(s) \). By assumption (i), the spectral encoding
\[
\mu_\rho := \frac{1}{i}(\rho - \tfrac{1}{2}) = \gamma
\]
maps \( \rho \) to a nonzero eigenvalue of \( L_{\sym} \in \TC(\HPsi) \).

\paragraph{Step 1: Real Spectrum.}
Assumption (ii) states that \( \Spec(L_{\sym}) \subset \R \) and all nonzero eigenvalues are simple. Therefore,
\[
\mu_\rho = \gamma \in \R.
\]
It follows that \( \rho = \tfrac{1}{2} + i\gamma \) has \( \Im(\rho) = \gamma \in \R \), so
\[
\Re(\rho) = \tfrac{1}{2}.
\]

\paragraph{Step 2: Exhaustion.}
Since the map \( \rho \mapsto \mu_\rho \) is bijective and multiplicity-preserving, every nontrivial zero corresponds to a unique real eigenvalue. Hence, all nontrivial zeros lie on the critical line.

\paragraph{Conclusion.}
The assumption of real, simple spectrum implies
\[
\rho \in \Spec(\zetaR) \quad \Longrightarrow \quad \Re(\rho) = \tfrac{1}{2},
\]
thereby proving the Riemann Hypothesis under the stated spectral conditions.
\end{proof}
