\begin{definition}[Canonical Spectral Map]
\label{def:spectral_zero_map}
Let \( \rho \in \mathbb{C} \) be a nontrivial zero of the Riemann zeta function \( \zeta(s) \), so that \( \zeta(\rho) = 0 \) and \( \rho \ne \tfrac{1}{2} \). Define the canonical spectral map
\[
\mu_\rho := \frac{1}{i(\rho - \tfrac{1}{2})},
\]
which sends a zero \( \rho = \tfrac{1}{2} + i\gamma \) to the real number \( \mu_\rho = 1/\gamma \in \mathbb{R} \setminus \{0\} \).

\medskip
\noindent
This map identifies nontrivial zeta zeros with the nonzero spectrum of the canonical trace-class operator \( L_{\mathrm{sym}} \in \mathcal{C}_1(H_{\Psi_\alpha}) \), via the identity
\[
\det\nolimits_\zeta(I - \lambda L_{\mathrm{sym}}) = \frac{\Xi\left(\tfrac{1}{2} + i\lambda\right)}{\Xi\left(\tfrac{1}{2}\right)},
\]
in which the poles of the logarithmic derivative correspond precisely to \( \lambda = \mu_\rho \in \operatorname{Spec}(L_{\mathrm{sym}}) \).

\medskip
\noindent
The inverse map assigns to each nonzero spectral value \( \mu \in \operatorname{Spec}(L_{\mathrm{sym}}) \setminus \{0\} \) the corresponding zeta zero:
\[
\rho_\mu := \tfrac{1}{2} - \frac{i}{\mu}.
\]

\medskip
\noindent
This bijection preserves multiplicities and encodes the nontrivial zeta zeros in the discrete spectrum of \( L_{\mathrm{sym}} \), forming the foundation for the spectral realization of the Riemann Hypothesis.
\end{definition}
% 