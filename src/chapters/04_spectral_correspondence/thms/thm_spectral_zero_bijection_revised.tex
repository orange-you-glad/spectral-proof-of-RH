\begin{theorem}[Spectral Bijection with Nontrivial Zeta Zeros]
\label{thm:spectral_zero_bijection_revised}
Let \( L_{\mathrm{sym}} \in \mathcal{C}_1(H_{\Psi_\alpha}) \) be the canonical compact, self-adjoint operator whose Carleman \(\zeta\)-regularized Fredholm determinant satisfies:
\[
\det\nolimits_\zeta(I - \lambda L_{\mathrm{sym}}) = \frac{\Xi\left(\tfrac{1}{2} + i\lambda\right)}{\Xi\left(\tfrac{1}{2}\right)} \qquad \forall \lambda \in \mathbb{C}.
\]

Then there exists a canonical multiplicity-preserving bijection between:
\begin{itemize}
  \item the multiset of nontrivial zeros \( \rho = \tfrac{1}{2} + i\gamma \) of the Riemann zeta function \( \zeta(s) \), and
  \item the multiset of nonzero eigenvalues \( \mu \in \operatorname{Spec}(L_{\mathrm{sym}}) \setminus \{0\} \),
\end{itemize}
given by the spectral correspondence:
\[
\rho \longmapsto \mu_\rho := \frac{1}{i}(\rho - \tfrac{1}{2}).
\]

\medskip

\noindent
This bijection satisfies:
\begin{itemize}
  \item \( \mu_\rho \in \mathbb{R} \setminus \{0\} \) for each \( \rho \ne \tfrac{1}{2} \);
  \item \( \operatorname{Spec}(L_{\mathrm{sym}}) \setminus \{0\} = \left\{ \mu_\rho : \zeta(\rho) = 0 \right\} \), as multisets;
  \item \( \operatorname{ord}_\rho(\zeta) = \operatorname{mult}_{\mathrm{spec}}(\mu_\rho) \), i.e., analytic multiplicity equals spectral multiplicity.
\end{itemize}

\medskip

\noindent
This result follows from:
\begin{enumerate}
  \item The determinant identity, which ensures that the zero set of the entire function \( \lambda \mapsto \det\nolimits_\zeta(I - \lambda L_{\mathrm{sym}}) \) coincides with the zero set of \( \Xi(\tfrac{1}{2} + i\lambda) \);
  \item The trace-class spectral theorem, which guarantees that the zeros of the determinant correspond precisely to the nonzero spectrum of \( L_{\mathrm{sym}} \), with multiplicities preserved;
  \item The normalization \( \Xi(\tfrac{1}{2}) \ne 0 \), which implies \( \lambda = 0 \) is not a zero of the determinant, and thus \( 0 \notin \operatorname{Spec}(L_{\mathrm{sym}}) \).
\end{enumerate}
\end{theorem}
