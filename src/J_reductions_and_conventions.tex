\section{Reductions and Conventions}
\label{app:reductions-and-conventions}

This appendix records global analytic conventions and structural reductions used throughout the manuscript. These ensure convergence, compactness, and spectral closure across all operator-theoretic constructions. No result relies on unstated assumptions.

\subsection*{Hilbert Space Framework}

All Hilbert spaces \( H \) are complex, separable, and equipped with the standard inner product. Operators \( T : H \to H \) are assumed linear and bounded unless otherwise stated.

All trace and determinant constructions assume compactness. Convergence of operator approximants (e.g., \( L_t \to L_{\mathrm{sym}} \)) is taken in the trace norm \( \|\cdot\|_{\mathcal{C}_1} \) unless otherwise specified. Self-adjointness and Schatten class inclusions are justified via Fourier-analytic kernel decay, as developed in~\cite{Simon2005TraceIdeals, ReedSimon1980I}.

\subsection*{Fourier Transform Convention}

We use the unitary Fourier transform on \( L^2(\mathbb{R}) \):
\[
\mathcal{F}f(\lambda) := \frac{1}{\sqrt{2\pi}} \int_{\mathbb{R}} f(x) e^{-i\lambda x}\, dx,
\]
with the standard involutive properties:
\[
\mathcal{F}^{-1} = \mathcal{F}, \qquad \mathcal{F}^2 f(x) = f(-x).
\]

Convolution operators become multiplication operators under \( \mathcal{F} \), and even, real-valued convolution profiles yield self-adjoint operators.

\subsection*{Spectral Domain}

All spectral variables lie on the real axis \( \lambda \in \mathbb{R} \), with the change of variable \( s = \tfrac{1}{2} + i\lambda \) centering the critical line. This normalization enables the formulation of the determinant identity in terms of real spectral data.

No adelic, automorphic, or global field structures are assumed in the main chapters. All results are derived from classical real and complex analysis. Extensions to automorphic \( L \)-functions appear in Appendix~\ref{app:functorial-extensions}.

\subsection*{Weight Function Normalization}

We fix the exponential weight:
\[
\Psi_\alpha(x) := e^{\alpha |x|}, \qquad \alpha > \pi,
\]
chosen to ensure:
\begin{itemize}
  \item Integrability of the inverse Fourier transform \( \widehat{\Xi} \);
  \item Trace-class convergence of mollified operators \( L_t \to L_{\mathrm{sym}} \);
  \item Validity of short-time heat kernel expansions and Laplace integrability.
\end{itemize}

Alternative weights, such as \( |\Xi(\tfrac{1}{2} + ix)|^2 \), are not used here, as they lack sufficient exponential decay to guarantee compactness in trace-class operator theory.

\subsection*{Spectral Parameterization}

All spectral data is encoded through the centered profile:
\[
\phi(\lambda) := \Xi\left( \tfrac{1}{2} + i\lambda \right), \qquad \lambda \in \mathbb{R}.
\]
This choice ensures:
\begin{itemize}
  \item Even symmetry: \( \phi(-\lambda) = \phi(\lambda) \);
  \item Self-adjointness of convolution operators with kernel \( \widehat{\phi} \);
  \item Canonical spectral encoding: \( \rho = \tfrac{1}{2} + i\gamma \mapsto \mu_\rho := \frac{1}{\gamma} \);
  \item Realization of \( \Xi(s) \) as a zeta-regularized Fredholm determinant.
\end{itemize}
