\begin{proof}[Proof of Lemma~\ref{lem:trace-zero}]
Let \( L_{\mathrm{sym}} \in \mathcal{C}_1(H_{\Psi_\alpha}) \) be the canonical self-adjoint convolution operator with kernel
\[
K(x - y) := \widehat{\Xi}(x - y),
\]
where
\[
\widehat{\Xi}(x) := \frac{1}{2\pi} \int_{\mathbb{R}} e^{i\lambda x} \Xi\left( \tfrac{1}{2} + i\lambda \right) \, d\lambda
\]
is the inverse Fourier transform of the completed zeta profile.

\paragraph{(i) Symmetry of the Kernel.}
The function
\[
\phi(\lambda) := \Xi\left( \tfrac{1}{2} + i\lambda \right)
\]
is real-valued and even, due to the functional equation \( \Xi(s) = \Xi(1 - s) \) and the fact that \( \Xi \) is real on the critical line. Therefore, \( \widehat{\Xi}(x) \in \mathcal{S}(\mathbb{R}) \) is real-valued and even as well. In particular, \( K(x, x) = \widehat{\Xi}(0) \) is constant.

\paragraph{(ii) Kernel Trace Heuristic.}
Formally, the kernel trace of \( L_{\mathrm{sym}} \) would be
\[
\operatorname{Tr}(L_{\mathrm{sym}}) = \int_{\mathbb{R}} K(x, x)\, dx = \widehat{\Xi}(0) \cdot \int_{\mathbb{R}} dx.
\]
This diverges unless \( \widehat{\Xi}(0) = 0 \), but this argument is heuristic and not valid in the trace-class context on weighted spaces.

\paragraph{(iii) Spectral Interpretation via Hadamard Structure.}
The determinant
\[
\det\nolimits_\zeta(I - \lambda L_{\mathrm{sym}}) = \frac{\Xi\left( \tfrac{1}{2} + i\lambda \right)}{\Xi\left( \tfrac{1}{2} \right)}
\]
is an entire function of exponential type \( \pi \) and genus one, with Hadamard factorization
\[
\Xi\left( \tfrac{1}{2} + i\lambda \right)
= \Xi\left( \tfrac{1}{2} \right)
\prod_\rho \left(1 - \frac{\lambda}{i(\rho - \tfrac{1}{2})} \right)
\exp\left( \frac{\lambda}{i(\rho - \tfrac{1}{2})} \right).
\]
The logarithmic derivative satisfies
\[
\frac{d}{d\lambda} \log \Xi\left( \tfrac{1}{2} + i\lambda \right)
= \sum_\rho \frac{1}{\lambda - i(\rho - \tfrac{1}{2})},
\]
with no term of the form \( \lambda^{-1} \) appearing. Since such a term would contribute a nonzero trace in the spectral expansion of the logarithmic derivative, its absence implies:
\[
\operatorname{Tr}(L_{\mathrm{sym}}) = \sum_n \lambda_n = 0.
\]

\paragraph{Conclusion.}
Thus, the vanishing trace condition follows from the Hadamard factorization and canonical normalization of the determinant. It ensures
\[
\det\nolimits_\zeta(I - \lambda L_{\mathrm{sym}}) = \frac{\Xi\left( \tfrac{1}{2} + i\lambda \right)}{\Xi\left( \tfrac{1}{2} \right)}
\quad \text{satisfies} \quad \det\nolimits_\zeta(I) = 1,
\]
as claimed.
\end{proof}
% 