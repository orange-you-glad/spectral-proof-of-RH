\begin{corollary}[Spectral Determination of the Zeta Zeros]
\label{cor:spectral-determines-zeta}
The spectrum of the canonical operator \( L_{\mathrm{sym}} \in \mathcal{C}_1(H_{\Psi_\alpha}) \) determines the nontrivial zeros of the Riemann zeta function completely and canonically.

That is, there exists a bijection
\[
\operatorname{Spec}(L_{\mathrm{sym}}) \setminus \{0\}
\;\longleftrightarrow\;
\left\{ \rho \in \mathbb{C} : \zeta(\rho) = 0, \; 0 < \Re(\rho) < 1 \right\},
\]
given by
\[
\mu \mapsto \rho := \tfrac{1}{2} + i \mu^{-1},
\]
with multiplicities preserved.

\medskip
\noindent
In particular, the spectral data of \( L_{\mathrm{sym}} \) encodes not only the location but also the multiplicity structure of the nontrivial zeros of \( \zeta(s) \). This establishes a canonical spectral model of the critical strip, uniquely determined by the determinant identity
\[
\det\nolimits_\zeta(I - \lambda L_{\mathrm{sym}}) = \frac{\Xi\left( \tfrac{1}{2} + i\lambda \right)}{\Xi\left( \tfrac{1}{2} \right)}.
\]
\end{corollary}
