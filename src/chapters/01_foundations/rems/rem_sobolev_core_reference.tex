\begin{remark}[Core Density and Sobolev Completion]
\label{rem:sobolev_core_reference}
The density of \( \Schwartz(\R) \subset H_{\Psi_\alpha} \) can also be justified through Sobolev-scale completions adapted to exponential weights.

\medskip

Let \( H^s_\alpha(\R) \) denote the weighted Sobolev space
\[
H^s_\alpha(\R) := \left\{ f \in L^2_{\mathrm{loc}}(\R) \;\middle|\; \langle D \rangle^s f \in L^2(\R, \Psi_\alpha(x)\, dx) \right\}, \quad \Psi_\alpha(x) := e^{\alpha |x|}, \quad \alpha > 0.
\]
Then the continuous embeddings
\[
\Schwartz(\R) \hookrightarrow H^s_\alpha(\R) \hookrightarrow H_{\Psi_\alpha}
\]
are dense for all \( s \ge 0 \), and provide a natural topology for defining graph cores of unbounded convolution operators with exponential kernel decay.

\medskip

This perspective complements the analytic vector argument used to establish essential self-adjointness of \( L_{\sym} \) on \( \Schwartz(\R) \), and justifies the stability of core domains under mollification and unitary conjugation. Under \( U_\alpha f(x) := \Psi_\alpha(x)^{1/2} f(x) \), all Sobolev and Schwartz core properties transfer to the flat setting \( L^2(\R) \), reinforcing the spectral invariance of the domain structure.
\end{remark}
