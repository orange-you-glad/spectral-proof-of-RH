\subsection*{Introduction}
\label{sec:det_intro}

This chapter establishes the canonical identity
\begin{equation}
\label{eq:zeta_det_identity}
\detz(I - \lambda \Lsym) 
= \frac{\Xi\left( \tfrac{1}{2} + i\lambda \right)}{\Xi\left( \tfrac{1}{2} \right)},
\end{equation}
realizing the completed Riemann zeta function \( \Xi(s) \) as the Carleman \(\zeta\)-regularized Fredholm determinant of the canonical convolution operator
\[
\Lsym \in \TC(\HPsi),
\]
constructed in \cref{sec:operator_construction} as the trace-norm limit of mollified convolution operators \( L_t \), each defined by smoothed inverse Fourier transforms of \( \Xi \).

\begin{previewbox}
\textbf{Analytic Preconditions.} The results in this chapter rest on the following rigorously established inputs from Chapters~1–2 and analytic expansions derived in Chapter~\ref{sec:heat_kernel_asymptotics} and Appendix~\ref{app:heat_kernel_construction}:

\begin{itemize}
  \item \( \phi(\lambda) := \Xi(\tfrac{1}{2} + i\lambda) \) is real, even, entire, and lies in the Paley--Wiener class \( \PW{\pi} \) (\cref{lem:exact_type_pi}).
  \item The inverse Fourier transform \( k = \FT^{-1}[\phi] \) lies in \( L^1(\R, \Psi_\alpha^{-1}) \), enabling trace-class convolution on \( \HPsi \) for \( \alpha > \pi \).
  \item The canonical operator \( \Lsym \in \TC(\HPsi) \) is compact, self-adjoint, and defined as the trace-norm limit \( L_t \to \Lsym \) (\cref{lem:trace_norm_convergence_Lt_to_Lsym}, \cref{lem:trace_norm_limit_unique}).
  \item The heat semigroup \( \{ e^{-t \Lsym^2} \}_{t > 0} \) is trace class and holomorphic, with Laplace-integrable decay (\cref{lem:heat_semigroup_wellposed}).
  \item Short-time trace asymptotics:
  \[
  \Tr(e^{-t \Lsym^2}) \sim \tfrac{1}{\sqrt{4\pi t}} \log\left(\tfrac{1}{t}\right) + \cdots
  \]
  determine the genus-one structure and entire growth class of the determinant (\cref{lem:det_identity_entire_order_one}).
  \item The determinant is defined via local power series expansion (\cref{lem:det_via_heat_trace}) and globally extended by Laplace--Mellin continuation.
\end{itemize}

All dependencies are strictly modular and acyclic, tracked in \appref{app:dependency_graph}. No assumption of RH or spectral bijection is used in this chapter.
\end{previewbox}

\begin{remark}[Forward Dependence on Heat Trace Asymptotics]
Although this chapter does not assume RH or spectral realness, it does rely on heat trace asymptotics and Laplace integrability results developed later in Chapter~\ref{sec:heat_kernel_asymptotics} and Appendix~\ref{app:heat_kernel_construction}. These forward links are tracked explicitly in the DAG, and no logical circularity arises.
\end{remark}

\paragraph{Structure of the Proof.}
The argument proceeds through four modular analytic phases, grounded in trace-class determinant theory~\cite[Ch.~4]{Simon2005TraceIdeals} and the Paley--Wiener theory of entire functions~\cite[Ch.~9]{Levin1996EntireLectures}:

\begin{itemize}
  \item \textbf{Heat trace and Laplace continuation:} The Carleman determinant is defined via the Laplace--Mellin transform of the trace \( \Tr(e^{-t \Lsym^2}) \), analytically extended using asymptotic Tauberian analysis (\cref{lem:det_via_heat_trace}).

  \item \textbf{Short-time singularity and growth bounds:} As \( t \to 0^+ \),
  \[
  \Tr(e^{-t \Lsym^2}) = \frac{1}{\sqrt{4\pi t}} \log\left(\tfrac{1}{t}\right) + \mathcal{O}(t^{-1/2}),
  \]
  confirming logarithmic divergence and bounding the growth of \( \detz(I - \lambda \Lsym) \) within \( \mathcal{E}_1^\pi \) (\cref{lem:det_identity_entire_order_one}).

  \item \textbf{Hadamard uniqueness and spectral identity:} The entire function \( \lambda \mapsto \detz(I - \lambda \Lsym) \) is uniquely determined by its zero set and normalization. The spectral trace and log-derivative (\cref{lem:A_log_derivative}) confirm that it matches the normalized zeta profile.

  \item \textbf{Canonical normalization:} The trace identity
  \[
  \Tr(\Lsym) = 0 \quad \Rightarrow \quad \detz(I) = 1
  \]
  ensures uniqueness of the Hadamard representation and eliminates ambiguities in the exponential prefactor.
\end{itemize}

This analytic phase is formally self-contained, free of RH assumptions, and independent of spectral encoding. The bijective spectral map
\[
\rho \mapsto \mu_\rho := \tfrac{1}{i}(\rho - \tfrac{1}{2}) \in \Spec(\Lsym)
\]
is developed separately in Chapter~\ref{sec:spectral_correspondence}, building on this determinant identity.
