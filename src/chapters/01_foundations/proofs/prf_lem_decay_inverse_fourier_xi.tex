\begin{proof}[Proof of \lemref{lem:decay_inverse_fourier_xi}]
Let \( \phi(\lambda) := \Xi\left( \tfrac{1}{2} + i\lambda \right) \), where \( \Xi(s) \) is the completed Riemann zeta function. As established in \lemref{lem:xi_growth_bound}, \( \phi \in \PW{\pi} \) is entire of exponential type \( \pi \), real-valued, and even, with
\[
|\phi(\lambda)| \le C\, e^{\pi |\lambda|}, \qquad \forall \lambda \in \R.
\]

\medskip
\noindent\textbf{Step 1: Exponential Decay via Paley--Wiener.}
By the Paley--Wiener theorem for functions in \( \PW{\pi} \) (see~\cite[Thm.~3.2.4]{Levin1996EntireLectures}, \cite[Ch.~IX.4]{ReedSimon1975II}), the inverse Fourier transform
\[
k(x) := \FT^{-1}[\phi](x) = \frac{1}{2\pi} \int_{\R} e^{i\lambda x} \phi(\lambda)\, d\lambda
\]
lies in \( C^\infty(\R) \cap L^1(\R, e^{-\alpha |x|} dx) \) for all \( \alpha > \pi \), with
\[
|k(x)| \le C_\alpha\, e^{-\alpha |x|}, \qquad \forall x \in \R.
\]

\medskip
\noindent\textbf{Step 2: Symmetry and Regularity.}
Since \( \phi \) is real-valued and even, Fourier inversion implies \( k(x) \in \R \) and \( k(x) = k(-x) \). Moreover, \( k \in \Schwartz \subset C^\infty(\R) \), and all derivatives decay faster than any exponential \( e^{-\beta |x|} \) for \( \beta < \alpha \).

\medskip
\noindent\textbf{Step 3: Weighted Integrability.}
Fix \( \alpha > \pi \), and define \( \Psi_\alpha(x) := e^{\alpha |x|} \). Then
\[
\int_{\R} |k(x)|\, \Psi_\alpha(x)\, dx = \int_{\R} |k(x)|\, e^{\alpha |x|} dx < \infty,
\]
so \( k \in L^1(\R, \Psi_\alpha(x)\, dx) \).

\medskip
\noindent\textbf{Step 4: Trace-Class Kernel Inclusion.}
Define the translation-invariant kernel \( K(x,y) := k(x - y) \). Then
\[
\iint_{\R^2} |K(x,y)|\, \Psi_\alpha(x)\Psi_\alpha(y)\, dx\, dy
= \int_{\R} |k(z)| \left( \int_{\R} \Psi_\alpha(z + y)\Psi_\alpha(y)\, dy \right) dz.
\]
Using the exponential decay of \( k \) and convexity of \( \Psi_\alpha \), the inner integral is uniformly bounded in \( z \) by \( C \Psi_\alpha(z) \). Thus, the full integral is bounded by
\[
C \int_{\R} |k(z)|\, \Psi_\alpha(z)\, dz < \infty.
\]
Hence \( K \in L^1(\R^2, \Psi_\alpha(x)\Psi_\alpha(y)\, dx\, dy) \), and the associated convolution operator
\[
(L f)(x) := \int_{\R} k(x - y) f(y)\, dy
\]
belongs to \( \TC(\HPsi) \) by Simon’s trace-class kernel criterion~\cite[Thm.~4.2]{Simon2005TraceIdeals}.
\end{proof}
