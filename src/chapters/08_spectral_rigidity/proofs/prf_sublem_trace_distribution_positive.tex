\begin{proof}[Proof of Lemma~\ref{sublem:trace_distribution_positive}]
Since \( L_{\mathrm{sym}} \in \mathcal{C}_1(H_{\Psi_\alpha}) \) is compact and self-adjoint, it admits a spectral decomposition of the form
\[
L_{\mathrm{sym}} = \sum_n \mu_n \langle \cdot, \psi_n \rangle \psi_n,
\]
where \( \{ \psi_n \} \subset H_{\Psi_\alpha} \) is an orthonormal basis, and \( \{ \mu_n \} \subset \mathbb{R} \) are the corresponding eigenvalues, counted with multiplicity.

Let \( \varphi \in \mathcal{S}(\mathbb{R}) \) be a real-valued Schwartz function. Then, by the spectral theorem and functional calculus:
\[
\varphi(L_{\mathrm{sym}}) = \sum_n \varphi(\mu_n) \langle \cdot, \psi_n \rangle \psi_n,
\]
and hence
\[
\operatorname{Tr}(\varphi(L_{\mathrm{sym}})) = \sum_n \varphi(\mu_n).
\]

If \( \varphi(\lambda) \ge 0 \) for all \( \lambda \in \mathbb{R} \), then each term in the series is nonnegative. Therefore:
\[
\operatorname{Tr}(\varphi(L_{\mathrm{sym}})) \ge 0.
\]

The map \( \varphi \mapsto \operatorname{Tr}(\varphi(L_{\mathrm{sym}})) \) is thus a positive linear functional on \( \mathcal{S}(\mathbb{R}) \), with at most polynomial growth controlled by the decay of \( \varphi \). Hence, it defines a positive tempered distribution.
\end{proof}
