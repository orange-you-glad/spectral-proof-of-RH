\begin{remark}[Spectral Interpretation of Heat Trace Scaling]
\label{rem:spectral-dimension}
The two-sided asymptotic estimate
\[
\operatorname{Tr}(e^{-t L_{\mathrm{sym}}^2}) \asymp t^{-1/2} \qquad \text{as } t \to 0^+
\]
admits a natural spectral interpretation: it reflects an effective spectral dimension \( d = 1 \) in the sense of a log-enhanced Weyl law.

\medskip
\noindent
Specifically, if the eigenvalue counting function for the squared spectrum of \( L_{\mathrm{sym}} \),
\[
N(\lambda) := \#\{ n : \mu_n^2 \le \lambda \},
\]
satisfies the asymptotic growth law
\[
N(\lambda) \sim C \lambda^{1/2} \log \lambda, \qquad \text{as } \lambda \to \infty,
\]
then a Tauberian inversion (see Chapter~\ref{sec:tauberian-growth}) implies that
\[
\operatorname{Tr}(e^{-t L_{\mathrm{sym}}^2}) \sim \frac{1}{\sqrt{4\pi t}} \log\left( \frac{1}{t} \right),
\]
as confirmed in Proposition~\ref{prop:short_time_heat_expansion}.

\medskip
\noindent
This aligns \( L_{\mathrm{sym}} \) with pseudodifferential-type operators exhibiting one-dimensional spectral behavior, modulated by logarithmic corrections. It also supports the analytic structure of the zeta-regularized spectral determinant:
\[
\det\nolimits_\zeta(I - \lambda L_{\mathrm{sym}}) = \exp\left( - \int_0^\infty \frac{\operatorname{Tr}(e^{-t L_{\mathrm{sym}}^2}) - P(t)}{t} \, e^{-\lambda^2 t} \, dt \right),
\]
where \( P(t) \sim \frac{1}{\sqrt{4\pi t}} \log(1/t) + \cdots \) is the parametrix subtraction term. The singularity of the trace near \( t = 0 \) determines the growth and holomorphic domain of the spectral determinant via Laplace–Mellin regularization.
\end{remark}
