\begin{proof}[Proof of Lemma~\ref{lem:heat-trace-expansion}]
\textbf{(i)} Since \( L_{\mathrm{sym}} \in \mathcal{C}_1(H_{\Psi_\alpha}) \subset \mathcal{C}_2 \), we may factor \( L_{\mathrm{sym}} = AB \) with \( A, B \in \mathcal{C}_2 \), the Hilbert–Schmidt class. Then the square satisfies
\[
L_{\mathrm{sym}}^2 = A (BA) B \in \mathcal{C}_1,
\]
because \( \mathcal{C}_2 \cdot \mathcal{C}_2 \subset \mathcal{C}_1 \), by standard Schatten ideal multiplication. Hence, the heat semigroup \( e^{-tL_{\mathrm{sym}}^2} \in \mathcal{C}_1 \) for all \( t > 0 \), and the heat trace
\[
\Theta(t) := \operatorname{Tr}(e^{-tL_{\mathrm{sym}}^2})
\]
is finite and smooth for all \( t > 0 \).

\textbf{(ii)} Let \( \{ \mu_n \}_{n \in \mathbb{N}} \subset \mathbb{R} \) denote the eigenvalues of \( L_{\mathrm{sym}} \), indexed with multiplicity. Since \( L_{\mathrm{sym}}^2 \in \mathcal{C}_1 \), the spectral trace formula yields
\[
\operatorname{Tr}(e^{-tL_{\mathrm{sym}}^2}) = \sum_{n=1}^{\infty} e^{-t \mu_n^2} < \infty.
\]
The spectral theorem ensures that \( e^{-tL_{\mathrm{sym}}^2} \) is a strongly continuous trace-class semigroup.

From Lemma~1.13 and Lemma~2.10, the integral kernel \( K_t(x,y) \) of \( e^{-tL_{\mathrm{sym}}^2} \) is symmetric, smooth, and exponentially decaying. Paley–Wiener theory and the standard parametrix expansion (cf.~\cite[Ch.~III]{Korevaar2004Tauberian}) yield the pointwise diagonal expansion:
\[
K_t(x,x) \sim \sum_{n=0}^{\infty} a_n(x) \, t^{n - 1/2}, \qquad t \to 0^+.
\]
Since the kernel is trace-class, integrating term-by-term gives the global expansion:
\[
\Theta(t) = \operatorname{Tr}(e^{-tL_{\mathrm{sym}}^2}) = \int_{\mathbb{R}} K_t(x,x) \, dx \sim \sum_{n=0}^{\infty} A_n \, t^{n - 1/2}, \qquad A_n := \int_{\mathbb{R}} a_n(x) \, dx.
\]

\textbf{(iii)} The leading-order singularity is dictated by the spectral profile of \( \Xi(s) \), whose Hadamard product involves genus-one exponential type \( \pi \) and a Gaussian damping prefactor. Via Laplace–Mellin Tauberian asymptotics (see Lemma~7.2), this yields the dominant log-corrected term:
\[
\Theta(t) = \frac{1}{\sqrt{4\pi t}} \log\left( \frac{1}{t} \right) + \frac{c_0}{\sqrt{t}} + o(t^{-1/2}), \qquad \text{as } t \to 0^+,
\]
with
\[
c_0 := \int_{\mathbb{R}} a_1(x) \, dx.
\]
This logarithmic divergence near \( t = 0 \) is not Lebesgue integrable, and necessitates analytic regularization of the associated Fredholm determinant via the zeta prescription detailed in Chapter~3. The coefficient \( c_0 \in \mathbb{R} \), though formally computable, is analytically neutral: it plays no role in determinant normalization or spectral bijection.
\end{proof}
