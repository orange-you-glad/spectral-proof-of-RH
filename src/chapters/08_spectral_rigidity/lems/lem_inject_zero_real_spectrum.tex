\begin{lemma}[Spectral Realization and Rigidity Imply the Riemann Hypothesis]
\label{lem:inject_zero_real_spectrum}
Let \( L_{\mathrm{sym}} \in \mathcal{C}_1(H_{\Psi_\alpha}) \) be the canonical compact, self-adjoint operator on the exponentially weighted Hilbert space \( H_{\Psi_\alpha} := L^2(\mathbb{R}, e^{\alpha |x|} dx) \), for some fixed \( \alpha > \pi \), with dense domain.

Assume the following conditions:
\begin{enumerate}
  \item[\textnormal{(i)}] There exists a bijective, multiplicity-preserving correspondence between the multiset of nontrivial zeros \( \rho \in \mathbb{C} \) of the Riemann zeta function \( \zeta(s) \) and the nonzero spectrum of \( L_{\mathrm{sym}} \), given by
  \[
  \rho \mapsto \mu_\rho := \frac{1}{i}(\rho - \tfrac{1}{2}) \in \operatorname{Spec}(L_{\mathrm{sym}}) \setminus \{0\}.
  \]
  
  \item[\textnormal{(ii)}] The spectrum \( \operatorname{Spec}(L_{\mathrm{sym}}) \) is real and simple.
\end{enumerate}

Then every nontrivial zero \( \rho \) of \( \zeta(s) \) satisfies the Riemann Hypothesis:
\[
\Re(\rho) = \tfrac{1}{2}.
\]
\end{lemma}
