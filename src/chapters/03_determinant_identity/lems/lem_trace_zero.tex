\begin{lemma}[Vanishing Trace of \( L_{\mathrm{sym}} \)]
\label{lem:trace_zero}
Let \( L_{\mathrm{sym}} \in \mathcal{C}_1(H_{\Psi_\alpha}) \) denote the canonical compact, self-adjoint convolution operator defined via the inverse Fourier transform of the completed Riemann zeta function:
\[
\phi(\lambda) := \Xi\left( \tfrac{1}{2} + i\lambda \right), \qquad
K_{\mathrm{sym}}(x,y) := \widehat{\phi}(x - y).
\]

Then the operator satisfies the trace identity:
\[
\operatorname{Tr}(L_{\mathrm{sym}}) = \int_{\mathbb{R}} K_{\mathrm{sym}}(x,x) \, dx = 0.
\]

\medskip
\noindent
\textbf{Justification.}
Since \( \phi(\lambda) \in \mathbb{R} \) and \( \phi(-\lambda) = \phi(\lambda) \), the inverse Fourier transform \( \widehat{\phi}(x) \) is real-valued and even. Hence, the diagonal kernel value
\[
K_{\mathrm{sym}}(x,x) = \widehat{\phi}(0)
\]
is constant. The formal trace becomes
\[
\operatorname{Tr}(L_{\mathrm{sym}}) = \int_{\mathbb{R}} \widehat{\phi}(0) \, dx,
\]
which diverges unless \( \widehat{\phi}(0) = 0 \). But by Fourier inversion,
\[
\widehat{\phi}(0) = \frac{1}{2\pi} \int_{\mathbb{R}} \phi(\lambda) \, d\lambda = 0,
\]
because \( \phi(\lambda) \) is even, entire of exponential type \( \pi \), and has vanishing average due to symmetry and decay of the zeros of \( \Xi \) (cf.~\cite[Ch.~3]{Levin1996EntireLectures}).

\medskip
\noindent
\textbf{Spectral Consequence.}
This vanishing ensures that the logarithmic derivative of the canonical determinant \( \log \det_\zeta(I - \lambda L_{\mathrm{sym}}) \) has no linear term, i.e., no \( \lambda \)-term in its Taylor expansion. Hence, the determinant belongs to the Hadamard class \( \mathcal{E}_1^\pi \) with normalization \( f(0) = 1 \), completing its spectral identification with the zeta profile.

\end{lemma}
