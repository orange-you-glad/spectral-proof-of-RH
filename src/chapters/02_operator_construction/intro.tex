\subsection*{Introduction}

This chapter constructs the canonical compact, self-adjoint, trace-class operator
\[
\Lsym \in \TC(H_{\Psi_\alpha}),
\]
on the exponentially weighted Hilbert space \( H_{\Psi_\alpha} := L^2(\R, e^{\alpha |x|} dx) \), designed to spectrally encode the nontrivial zeros of the completed Riemann zeta function \( \Xi(s) \).

The construction unfolds in five analytically rigorous stages:

\begin{itemize}
    \item \textbf{Weighted space and spectral profile:} The decay of \( \phi(\lambda) := \Xi(\tfrac{1}{2} + i\lambda) \) implies that its inverse Fourier transform lies in \( L^1(\R, \Psi_\alpha^{-1}) \) for \( \alpha > \pi \), ensuring kernel integrability. This yields convolution operators that are trace class on \( H_{\Psi_\alpha} \). The sharpness of this exponential decay threshold is established in \propref{prop:trace_class_sharpness}.

    \item \textbf{Mollifier family \( L_t \):} Mollified profiles
    \[
    \varphi_t(\lambda) := e^{-t\lambda^2} \phi(\lambda)
    \]
    define convolution operators \( L_t \in \TC(H_{\Psi_\alpha}) \) with uniformly bounded trace norms and symmetric kernels. Each \( L_t \) is compact, self-adjoint, and well-posed for regularization.

    \item \textbf{Trace-norm limit and canonicality:} We prove that \( L_t \to \Lsym \) in the trace-norm topology as \( t \to 0^+ \), and that this limit is independent of mollifier choice. Uniqueness is verified in \lemref{lem:trace_norm_limit_unique}, and canonical convergence is formalized in \thmref{thm:existence_Lsym}.

    \item \textbf{Core domain and essential self-adjointness:} The Schwartz space \( \Schwartz(\R) \) is a common core for \( \Lsym \) and \( \Lsym^2 \). By Nelson’s theorem, both are essentially self-adjoint on \( \Schwartz \) (\lemref{lem:core_essential_sa}, \lemref{lem:Lsym_square_selfadjoint}).

    \item \textbf{Trace normalization:} We prove the centering identity
    \[
    \Tr(\Lsym) = 0,
    \]
    ensuring uniqueness of the zeta-determinant normalization (\thmref{thm:trace_zero_Lsym}).
\end{itemize}

These results culminate in the spectral operator \( \Lsym \), rigorously defined in \( \TC(H_{\Psi_\alpha}) \), and endowed with trace-class regularity, self-adjointness, and analytic semigroup generation (\lemref{lem:heat_semigroup_existence}).

\medskip

This operator forms the analytic base of the determinant identity
\[
\detz(I - \lambda^2 \Lsym^2) = \Xi\left(\tfrac{1}{2} + i\lambda\right),
\]
established in Chapter~\ref{sec:determinant_identity} using heat kernel methods and Paley--Wiener theory.

\paragraph{Comparison to Prior Spectral Models.}
Unlike heuristic proposals such as Hilbert--Pólya, or frameworks by Connes~\cite{Connes1999TraceFormula} and Deninger~\cite{Deninger1998Frobenius}, the operator \( \Lsym \) is rigorously constructed within classical Hilbert space theory, with explicit control over domain, norm, trace, and convergence. It satisfies all analytic prerequisites for a canonical determinant realization of the Riemann zeta function.
