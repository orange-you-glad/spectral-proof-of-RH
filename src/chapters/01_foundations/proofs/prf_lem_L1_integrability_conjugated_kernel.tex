\begin{proof}[Proof of \lemref{lem:L1_integrability_conjugated_kernel}]
Assume the kernel \( K(x,y) \) satisfies the decay estimate
\[
|K(x,y)| \le C (1 + |x| + |y|)^{-N},
\]
for some constant \( C > 0 \). Let \( \Psi(x) \) satisfy the two-sided exponential bounds
\[
c_1 e^{a|x|} \le \Psi(x) \le c_2 e^{a|x|}, \qquad \forall x \in \R,
\]
with constants \( a > 0 \), \( c_1, c_2 > 0 \). Define the conjugated kernel
\[
\widetilde{K}(x,y) := \frac{K(x,y)}{\sqrt{\Psi(x)\Psi(y)}}.
\]

\medskip
\noindent\textbf{Step 1: Pointwise Estimate.}
Using the lower bound on \( \Psi \), we have
\[
\sqrt{\Psi(x)\Psi(y)} \ge c_1\, e^{a(|x| + |y|)/2},
\]
so
\[
|\widetilde{K}(x,y)| \le \frac{C}{c_1} (1 + |x| + |y|)^{-N} e^{-a(|x| + |y|)/2}.
\]

\medskip
\noindent\textbf{Step 2: Factorization via Submultiplicativity.}
Using the inequality
\[
1 + |x| + |y| \ge \tfrac{1}{2}(1 + |x|)(1 + |y|),
\]
we obtain
\[
(1 + |x| + |y|)^{-N} \le 2^N (1 + |x|)^{-N/2} (1 + |y|)^{-N/2}.
\]
Thus, for some constant \( C' > 0 \),
\[
|\widetilde{K}(x,y)| \le C' \cdot (1 + |x|)^{-N/2} e^{-a|x|/2} \cdot (1 + |y|)^{-N/2} e^{-a|y|/2}.
\]
Each factor lies in \( L^1(\R) \) provided \( N > 2a \). Therefore, Fubini’s theorem yields
\[
\iint_{\R^2} |\widetilde{K}(x,y)|\, dx\,dy < \infty.
\]

\medskip
\noindent\textbf{Step 3: Operator-Theoretic Interpretation.}
Let \( T \) be the integral operator on \( L^2(\R, \Psi(x)\,dx) \) with kernel \( K(x,y) \). Define the unitary map
\[
U \colon L^2(\R, \Psi(x)\,dx) \to L^2(\R), \qquad (Uf)(x) := \Psi(x)^{1/2} f(x),
\]
and let \( \widetilde{T} := U T U^{-1} \) act on \( L^2(\R) \) with kernel \( \widetilde{K}(x,y) \in L^1(\R^2) \).

By Simon’s trace-class criterion for integral operators \cite[Thm.~4.2]{Simon2005TraceIdeals}, we conclude:
\[
\widetilde{T} \in \mathcal{C}_1(L^2(\R)) \quad \Rightarrow \quad T \in \mathcal{C}_1\big(L^2(\R, \Psi(x)\,dx)\big).
\]
\end{proof}
