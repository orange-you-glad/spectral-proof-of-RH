\subsection*{Introduction}

The goal of this chapter is to construct the canonical compact, self-adjoint, trace-class operator
\[
L_{\mathrm{sym}} \in \mathcal{C}_1(H_{\Psi_\alpha}),
\]
on the exponentially weighted Hilbert space \( H_{\Psi_\alpha} \), tailored to match the spectral architecture of the completed Riemann zeta function \( \Xi(s) \).

We proceed through five analytically rigorous stages:

\begin{itemize}
    \item We define the exponentially weighted Hilbert space
    \[
    H_{\Psi_\alpha} := L^2(\mathbb{R}, \Psi_\alpha(x)\, dx), \quad \text{with } \Psi_\alpha(x) := e^{\alpha |x|}, \quad \alpha > \pi,
    \]
    selected such that the inverse Fourier transform of the canonical profile
    \[
    \phi(\lambda) := \Xi\left(\tfrac{1}{2} + i\lambda\right)
    \]
    lies in \( L^1(\mathbb{R}, \Psi_\alpha^{-1} dx) \). This ensures that convolution against \( \phi^\vee \) yields trace-class operators on \( H_{\Psi_\alpha} \).

    \item We define mollified spectral profiles
    \[
    \varphi_t(\lambda) := e^{-t\lambda^2} \phi(\lambda),
    \]
    which belong to the Schwartz space \( \mathcal{S}(\mathbb{R}) \), and induce a family of convolution operators \( L_t \in \mathcal{C}_1(H_{\Psi_\alpha}) \). Each \( L_t \) is bounded, self-adjoint, and satisfies uniform trace-norm bounds:
    \[
    \| L_t \|_{\mathcal{C}_1} \le C(\alpha), \quad \forall t \in (0,1],
    \]
    where \( C(\alpha) \) is explicitly derived from mollifier decay and exponential localization.

    \item We prove that the family \( \{L_t\}_{t > 0} \) converges in Schatten-1 norm:
    \[
    \| L_t - L_{\mathrm{sym}} \|_{\mathcal{C}_1} \to 0 \quad \text{as } t \to 0^+,
    \]
    and define the canonical operator by trace-norm limit:
    \[
    L_{\mathrm{sym}} := \lim_{t \to 0^+} L_t,
    \]
    where the limit exists uniquely in \( \mathcal{C}_1(H_{\Psi_\alpha}) \) and is independent of mollifier.

    \item We prove that \( L_{\mathrm{sym}} \) is compact, self-adjoint, and densely defined on \( H_{\Psi_\alpha} \), with \( \mathcal{S}(\mathbb{R}) \subset H_{\Psi_\alpha} \) serving as a symmetric core. The essential self-adjointness of \( L_{\mathrm{sym}} \) and its square \( L_{\mathrm{sym}}^2 \) are established via kernel decay and Nelson’s analytic vector theorem.

    \item We prove that the trace of \( L_{\mathrm{sym}} \) vanishes:
    \[
    \operatorname{Tr}(L_{\mathrm{sym}}) = 0,
    \]
    ensuring canonical normalization in the determinant identity. This condition removes the exponential ambiguity in Hadamard-class entire functions.
\end{itemize}

These results culminate in \thmref{thm:existence-Lsym}, \thmref{thm:sa-trace-class-Lsym}, and \thmref{thm:trace-zero-Lsym}, which collectively establish the existence, uniqueness, and analytic classification of \( L_{\mathrm{sym}} \) as the canonical trace-class convolution operator associated with the completed zeta function \( \Xi(s) \).

\medskip

This operator furnishes the analytic foundation for the Carleman–zeta-regularized determinant identity
\[
\det\nolimits_{\zeta}(I - \lambda^2 L_{\mathrm{sym}}^2) = \Xi\left(\tfrac{1}{2} + i\lambda\right),
\]
to be rigorously derived in Chapter~\ref{sec:determinant-identity}.

\medskip

All convergence results and trace-norm bounds are derived from explicit analytic kernel estimates using mollifier-controlled Fourier decay and exponential weights. Each operator-theoretic property of \( L_t \) and \( L_{\mathrm{sym}} \)—including trace-class membership, compactness, symmetry, essential self-adjointness, and normalization—is formally proved. This construction sets the analytic and spectral framework for encoding the nontrivial zeros of \( \zeta \) through canonical Fredholm theory.

\paragraph{Contrast with Prior Spectral Proposals.}
Unlike speculative frameworks such as the Hilbert--Pólya conjecture, Connes's noncommutative trace formulas \cite{Connes1999TraceFormula}, or Deninger's arithmetic cohomology program \cite{Deninger1998Frobenius}, the operator \( L_{\mathrm{sym}} \) is constructed explicitly within classical Hilbert space analysis. It satisfies concrete analytic properties—trace-class compactness, self-adjointness, and kernel decay—and supports a determinantal identity rigorously matching the completed zeta function. This resolves long-standing gaps in prior frameworks, which lacked either a spectral determinant, a Hilbert space realization, or verifiable compactness and trace control.
% 