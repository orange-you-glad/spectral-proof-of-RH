\begin{theorem}[Spectral Equivalence with the Riemann Hypothesis]
\label{thm:rh_spectrum_equiv}

Let \( \Lsym \in \TC(\HPsi) \) denote the canonical compact, self-adjoint, trace-class operator on the exponentially weighted Hilbert space
\[
\HPsi := L^2(\R, e^{\alpha |x|} dx), \qquad \text{for fixed } \alpha > \pi.
\]
Define the canonical spectral image
\[
\mu_\rho := \frac{1}{i}(\rho - \tfrac{1}{2})
\]
for each nontrivial zero \( \rho \in \C \) of the Riemann zeta function \( \zeta(s) \).

Then the following are logically equivalent:
\begin{enumerate}
  \item[\textup{(i)}] The Riemann Hypothesis holds:
  \[
  \Re(\rho) = \tfrac{1}{2}, \quad \text{for all nontrivial zeros } \rho.
  \]

  \item[\textup{(ii)}] The spectrum of \( \Lsym \) lies entirely on the real axis:
  \[
  \Spec(\Lsym) \subset \R.
  \]
\end{enumerate}

\medskip

\noindent
This equivalence follows from:
\begin{itemize}
  \item The determinant identity for \( \Lsym \), proven in \thmref{thm:det_identity_revised};
  \item The canonical spectral bijection \( \rho \mapsto \mu_\rho := \frac{1}{i}(\rho - \tfrac{1}{2}) \), established in \thmref{thm:spectral_zero_bijection_revised};
  \item The fact that \( \mu_\rho \in \R \iff \Re(\rho) = \tfrac{1}{2} \).
\end{itemize}

\medskip

\noindent
Thus, the Riemann Hypothesis is true if and only if the spectrum of the canonical trace-class operator \( \Lsym \) is real. This provides a complete operator-theoretic reformulation of RH within the zeta-determinant framework.
\end{theorem}
