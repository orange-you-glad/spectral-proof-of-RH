\begin{proof}[Proof of Theorem~\ref{thm:rh_spectrum_equiv}]
Let \( \rho = \tfrac{1}{2} + i\gamma \in \mathbb{C} \) be a nontrivial zero of the Riemann zeta function. From the canonical determinant identity,
\[
\det\nolimits_\zeta(I - \lambda L_{\mathrm{sym}}) = \frac{\Xi\left( \tfrac{1}{2} + i\lambda \right)}{\Xi\left( \tfrac{1}{2} \right)},
\]
and the bijective spectral correspondence established in Lemma~\ref{lem:spectral_zero_encoding}, each such zero corresponds to a nonzero spectral value
\[
\mu_\rho := \frac{1}{i(\rho - \tfrac{1}{2})} \in \operatorname{Spec}(L_{\mathrm{sym}}),
\]
with multiplicity preserved via the Hadamard factorization of the determinant.

\paragraph{(i) \( \Rightarrow \) (ii)}
Assume the Riemann Hypothesis holds, i.e., every nontrivial zero satisfies \( \rho = \tfrac{1}{2} + i\gamma \) with \( \gamma \in \mathbb{R} \). Then
\[
\mu_\rho = \frac{1}{i(i\gamma)} = -\frac{1}{\gamma} \in \mathbb{R}.
\]
Since the spectrum of \( L_{\mathrm{sym}} \) consists precisely of the \( \mu_\rho \)'s (along with 0 if needed), it follows that
\[
\operatorname{Spec}(L_{\mathrm{sym}}) \subset \mathbb{R}.
\]

\paragraph{(ii) \( \Rightarrow \) (i)}
Assume \( \operatorname{Spec}(L_{\mathrm{sym}}) \subset \mathbb{R} \). For any nontrivial zero \( \rho \in \mathbb{C} \), we again have
\[
\mu_\rho := \frac{1}{i(\rho - \tfrac{1}{2})} \in \operatorname{Spec}(L_{\mathrm{sym}}) \subset \mathbb{R}.
\]
This implies
\[
\rho - \tfrac{1}{2} \in i\mathbb{R} \quad \Longrightarrow \quad \Re(\rho) = \tfrac{1}{2}.
\]
Hence, the Riemann Hypothesis holds.

\paragraph{Conclusion.}
The spectrum of the canonical operator \( L_{\mathrm{sym}} \) lies entirely in \( \mathbb{R} \) if and only if all nontrivial zeros of \( \zeta(s) \) lie on the critical line. That is,
\[
\mathrm{RH} \iff \operatorname{Spec}(L_{\mathrm{sym}}) \subset \mathbb{R}.
\]
\end{proof}
