\subsection*{Summary}

This chapter derives the spectral growth of the canonical operator \( \Lsym \in \TC(\HPsi) \) from short-time asymptotics of the heat trace \( \Tr(e^{-t \Lsym^2}) \), using Laplace–Tauberian inversion. The key analytic results are:

\textbf{Tauberian Structure and Spectral Envelopes}
\begin{itemize}
  \item \defref{def:tauberian_theorem} — Classical Laplace–Tauberian framework: short-time behavior of the heat trace governs high-energy spectral growth via Laplace inversion.

  \item \lemref{lem:spectral_convexity_estimate} — Spectral envelope estimate:
  \[
  \Tr(e^{-t \Lsym^2}) = O(t^{-1/2}) \quad \Longrightarrow \quad N(\lambda) = O(\lambda^{1/2+\varepsilon}),
  \]
  confirming trace-class compactness and effective spectral dimension \( d = 1 \).
\end{itemize}

\textbf{Refined Asymptotics via Korevaar Theory}
\begin{itemize}
  \item \lemref{lem:log_corrected_tauberian_estimate} — Refined asymptotic:
  \[
  \Tr(e^{-t \Lsym^2}) \sim \frac{1}{\sqrt{4\pi t}} \log\left( \frac{1}{t} \right)
  \quad \Longrightarrow \quad
  N(\lambda) \sim \frac{\sqrt{\lambda}}{\pi} \log \lambda.
  \]
  This places \( \Theta(t) \in \mathcal{R}_{1/2}^{\log}(0^+) \) and enables sharp inversion.

  \item \corref{cor:zeta_compatibility} — Spectral growth matches the classical Riemann–von Mangoldt formula:
  \[
  N_\zeta(T) = \frac{T}{2\pi} \log\left( \frac{T}{2\pi} \right) - \frac{T}{2\pi} + O(\log T),
  \]
  confirming analytic agreement between the spectrum of \( \Lsym \) and the zeta zero distribution.
\end{itemize}

\medskip
\noindent
(For higher-order trace corrections and refined remainders, see \appref{app:heat_kernel_refinements}.)

\vspace{0.5em}

\textbf{Analytic Flow Diagram}
\[
\begin{aligned}
\text{Heat Kernel}
&\xrightarrow{\text{Laplace}} \text{Trace Asymptotics} \\
&\xrightarrow{\text{Tauberian}} \text{Spectral Growth} \\
&\xrightarrow{\text{Hadamard}} \text{Zeta Determinant}
\end{aligned}
\]

\begin{remark}[Transition to Spectral Rigidity]
We have shown that the eigenvalue distribution of \( \Lsym \) precisely encodes the nontrivial zeros of \( \zeta(s) \). The next natural question is rigidity: if another operator shares both the spectral data and the determinant identity, must it be unitarily equivalent to \( \Lsym \)? This is the content of the spectral uniqueness result in \thmref{thm:uniqueness_realization}.
\end{remark}

\paragraph{Logical Closure.}
All asymptotics in this chapter are derived from the heat trace singularity using rigorously validated Tauberian theorems. The analytic flow from kernel to spectral counting is fully DAG-traceable and logically independent of RH or spectral assumptions.
