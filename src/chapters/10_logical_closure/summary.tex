\subsection*{Summary}

We have constructed a canonical compact, self-adjoint, trace-class operator
\[
\Lsym \in \TC(\HPsi),
\]
on the exponentially weighted Hilbert space \( \HPsi := L^2(\mathbb{R}, e^{\alpha|x|} dx) \), with \( \alpha > \pi \), whose discrete spectrum encodes the nontrivial zeros of the completed Riemann zeta function \( \XiR(s) \).

\medskip

\noindent
Via trace-norm convergence of mollified convolution operators and exponential kernel decay from Paley–Wiener theory, we established the canonical determinant identity \thmref{thm:det_identity_revised}:
\[
\detz(I - \lambda \Lsym) = \frac{\XiR\left(\tfrac{1}{2} + i\lambda\right)}{\XiR\left(\tfrac{1}{2}\right)},
\]
which transfers the zero set of \( \XiR(s) \) to the spectral data of \( \Lsym \).

\medskip

\noindent
This determinant identity yields the bijective spectral encoding \thmref{thm:spectral_zero_bijection_revised}:
\[
\rho \mapsto \mu_\rho := \tfrac{1}{i}(\rho - \tfrac{1}{2}) \in \Spec(\Lsym),
\]
with multiplicities preserved through Hadamard factorization and Fredholm theory.

\medskip

\noindent
We proved the formal analytic equivalence \thmref{thm:eq_of_rh}, finalized in \thmref{thm:truth_of_rh}:
\[
\boxed{
\RH \iff \Spec(\Lsym) \subset \R
}
\]
and verified every analytic prerequisite necessary to establish the spectral reality of \( \Lsym \). In particular:
\begin{itemize}
  \item The convergence \( L_t \to \Lsym \) in \( \TC(\HPsi) \) follows from kernel decay and Schatten estimates;
  \item Self-adjointness and spectral symmetry follow from kernel reflection and Paley–Wiener analyticity;
  \item The semigroup \( e^{-t \Lsym^2} \) is holomorphic and trace class;
  \item The heat trace \( \Tr(e^{-t \Lsym^2}) \) admits Laplace–Mellin continuation and defines the determinant;
  \item The trace pairing \( \phi \mapsto \Tr(\phi(\Lsym)) \) is a positive tempered distribution.
\end{itemize}

\medskip

\noindent
These results are modular, acyclic, and grounded entirely in classical analytic techniques. The RH equivalence is derived unconditionally from spectral and determinant theory, without invoking arithmetic conjectures or automorphic frameworks.

\medskip

\noindent
The full logical dependency structure is encoded in the DAG of Appendix~\ref{app:dependency_graph}. The analytic foundations—kernel decay, determinant growth, spectral convergence—are developed in Chapters~1–5 and Appendices~\ref{app:zeta_trace_background} through \ref{app:heat_kernel_construction}.

\medskip

\noindent
For future refinements—including higher-order trace asymptotics, conjectural extensions to automorphic, Artin, or motivic \( L \)-functions, and structural comparisons with noncommutative geometry frameworks—see Appendices~\ref{app:heat_kernel_refinements}, \ref{app:functorial_extensions}, and \ref{app:additional_structures}. In particular, \appref{app:additional_structures} compares the spectral trace formalism developed here with Connes’ noncommutative trace formula and situates \( \Lsym \) within a broader analytic–cohomological context.

\begin{tcolorbox}[colback=gray!2!white, colframe=black!50, title={\textbf{Canonical Equivalence — RH via Spectral Reality}}]
The canonical operator \( \Lsym \in \TC(\HPsi) \), constructed in Chapter~\ref{sec:operator_construction} and analytically normalized in Chapter~\ref{sec:determinant_identity}, satisfies the determinant identity
\[
\detz(I - \lambda \Lsym) = \frac{\Xi(\tfrac{1}{2} + i\lambda)}{\Xi(\tfrac{1}{2})},
\]
whose spectral zeros match the nontrivial zeros of \( \zeta(s) \). This determinant structure, combined with the bijection \( \rho \mapsto \mu_\rho := \tfrac{1}{i}(\rho - \tfrac{1}{2}) \), canonically encodes zeta zeros into the spectrum of \( \Lsym \). The Riemann Hypothesis is then equivalent to spectral reality:
\[
\RH \iff \Spec(\Lsym) \subset \R,
\]
as formally proven in \thmref{thm:truth_of_rh}. This equivalence is derived solely from analytic spectral theory and zeta determinant regularization.
\end{tcolorbox}
