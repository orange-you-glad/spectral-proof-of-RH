\begin{remark}[No RH Assumption Used in Spectral Encoding]
\label{rem:no-rh-assumption}
Throughout Chapters~\ref{sec:determinant-identity}–\ref{sec:spectral-correspondence}, all constructions, estimates, and identities are derived without assuming the Riemann Hypothesis.

In particular:
\begin{itemize}
  \item The construction of the canonical operator \( L_{\mathrm{sym}} \in \mathcal{C}_1(H_{\Psi_\alpha}) \) via mollified convolution is entirely analytic and makes no spectral assumptions.
  \item The determinant identity
  \[
  \det\nolimits_{\zeta}(I - \lambda L_{\mathrm{sym}})
  = \frac{\Xi\left( \tfrac{1}{2} + i\lambda \right)}{\Xi\left( \tfrac{1}{2} \right)}
  \]
  is derived from explicit kernel estimates and heat-trace asymptotics, independently of RH.
  \item The Hadamard product and spectral mapping \( \rho \mapsto \mu_\rho \) are constructed from zero loci of \( \Xi \), and their convergence and multiplicity-preserving structure do not rely on RH.
  \item Trace-class and Schatten norm estimates used to establish convergence and positivity of the heat semigroup are independent of any spectral input from \( \zeta \).
\end{itemize}

The Riemann Hypothesis first appears as a theorem in Chapter~\ref{sec:spectral-implications}, where it is shown to be equivalent to the spectral reality of \( L_{\mathrm{sym}} \). No steps in the determinant identity or bijection arguments depend on assuming this spectral condition in advance.

\end{remark}
% 