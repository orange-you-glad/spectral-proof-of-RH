\section{Summary of Notation}
\label{app:notation_summary}

\noindent
This appendix collects global analytic symbols and conventions. All other notation is introduced locally at first use. For semantic dependencies and usage by chapter, see the DAG in \appref{app:dependency_graph}.

\begin{itemize}
  \item \textbf{Weighted Hilbert Space:}
  \[
  \HPsi := L^2(\R, e^{\alpha |x|} dx), \qquad \Psi_\alpha(x) := e^{\alpha |x|}, \quad \alpha > \pi.
  \]
  This exponential weight ensures trace-class inclusion of convolution kernels with Fourier decay \( \widehat{\phi}(x) \sim e^{-\pi |x|} \). The threshold \( \alpha > \pi \) is sharp: for \( \alpha \le \pi \), the canonical kernel fails to lie in \( L^1(\R^2, \Psi_\alpha \otimes \Psi_\alpha) \), and no trace-class realization exists. All operators \( L_t \), \( \Lsym \), and semigroups \( e^{-tL^2} \) act on \( \HPsi \).

  \item \textbf{Paley–Wiener Class:}
  \[
  \PW{a} := \mathcal{PW}_a(\R)
  \]
  denotes the Paley–Wiener space of entire functions of exponential type \( a \), i.e., those with Fourier transforms supported in \( [-a, a] \). The centered spectral profile satisfies:
  \[
  \phi(\lambda) := \Xi\left( \tfrac{1}{2} + i\lambda \right) \in \PW{\pi}, \quad \Rightarrow \quad \phi^\vee(x) \sim e^{-\pi |x|}.
  \]

  \item \textbf{Canonical Operator:}
  \[
  \Lsym := \lim_{t \to 0^+} L_t \in \TC(\HPsi),
  \]
  defined as the trace-norm limit of mollified convolution operators with kernels from
  \[
  k_t(x) := \FT^{-1}[\phi_t](x), \quad \phi_t(\lambda) := e^{-t\lambda^2} \phi(\lambda).
  \]
  The operator is compact, self-adjoint, and satisfies the canonical determinant identity.

  \item \textbf{Spectral Profile and Kernel:}
  \[
  \phi(\lambda) := \Xi\left( \tfrac{1}{2} + i\lambda \right), \quad
  k(x) := \FT^{-1}[\phi](x), \quad
  K(x, y) := k(x - y).
  \]
  These define the convolution kernel for \( \Lsym \). The decay of \( \phi \in \PW{\pi} \) governs trace-class regularity and determinant growth.

  \item \textbf{Canonical Spectral Map:}
  \[
  \mu_\rho := \tfrac{1}{i}(\rho - \tfrac{1}{2}), \qquad
  \rho \in \mathcal{Z}(\zeta).
  \]
  This bijective reparametrization maps the critical line to the real axis and encodes the zeta zeros into the spectrum of \( \Lsym \).

  \item \textbf{Spectral Determinant:}
  \[
  \detz(I - \lambda \Lsym) := \frac{\Xi\left( \tfrac{1}{2} + i\lambda \right)}{\Xi\left( \tfrac{1}{2} \right)},
  \]
  the Carleman \(\zeta\)-regularized Fredholm determinant of \( \Lsym \), encoding all nontrivial zeros and normalized via trace centering.
\end{itemize}

\medskip

\noindent
The completed zeta function \( \Xi(s) \) is entire of order one and exponential type \( \pi \), satisfying:
\[
\Xi(s) = \Xi(1 - s), \qquad
\Xi\left( \tfrac{1}{2} + i\lambda \right) \in \R \quad \forall\, \lambda \in \R.
\]

\medskip

\noindent
For analytic derivations of these constructions, see \appref{app:zeta_trace_background}. The full dependency DAG appears in \appref{app:dependency_graph}, summarizing which chapters rely on each core analytic object.
