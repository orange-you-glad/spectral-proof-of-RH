\begin{remark}[Spectral Discreteness of Mollified Operators]
\label{rmk:spectrum_Lt_real_discrete}
For each \( t > 0 \), the mollified convolution operator
\[
L_t f(x) := \int_{\R} k_t(x - y)\, f(y)\, dy
\]
acts on the weighted Hilbert space \( H_{\Psi_\alpha} := L^2(\R, \Psi_\alpha(x)\, dx) \) and satisfies:
\begin{itemize}
  \item \( L_t \in \TC(H_{\Psi_\alpha}) \cap \KC(H_{\Psi_\alpha}) \), i.e., it is trace-class and compact;
  \item \( L_t \) is self-adjoint with domain containing \( \Schwartz \subset H_{\Psi_\alpha} \);
  \item By the spectral theorem, \( L_t \) admits a discrete spectrum of real eigenvalues \( \{ \lambda_n \} \subset \R \), with \( \lambda_n \to 0 \), and an orthonormal eigenbasis \( \{ \psi_n \} \subset H_{\Psi_\alpha} \).
\end{itemize}

These properties follow from standard operator theory for compact self-adjoint convolution operators and underpin the analytic convergence \( L_t \to L_{\sym} \) in trace norm.
\end{remark}
