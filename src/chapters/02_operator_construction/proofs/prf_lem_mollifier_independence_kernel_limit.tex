\begin{proof}[Proof of \lemref{lem:mollifier_independence_kernel_limit}]
Let \( \{\varphi_t\}_{t > 0} \subset \Schwartz \) be a mollifier family satisfying:
\begin{itemize}
  \item Normalization: \( \int_{\R} \varphi_t(x)\, dx = 1 \);
  \item Approximate identity: \( \varphi_t \to \delta \) in \( \Schwartz' \) as \( t \to 0^+ \);
  \item Symmetry: \( \varphi_t(x) = \varphi_t(-x) \);
  \item Decay: \( \varphi_t \in L^1 \cap L^2 \cap L^1(\Psi_\alpha\, dx) \) for all \( t > 0 \).
\end{itemize}

Let \( \phi(\lambda) := \Xi\left( \tfrac{1}{2} + i\lambda \right) \), and define its inverse Fourier transform \( \widehat{\Xi}(x) := \phi^\vee(x) \in L^1(\R, \Psi_\alpha^{-1}(x)\, dx) \). The decay of \( \widehat{\Xi} \) is guaranteed by \lemref{lem:xi_growth_bound}. Define mollified spatial kernels:
\[
\widehat{\Xi}_t := \varphi_t * \widehat{\Xi} \in \Schwartz \cap L^1(\R, \Psi_\alpha\, dx),
\]
and the associated convolution operators:
\[
(L_t^{(\varphi)} f)(x) := \int_{\R} \widehat{\Xi}_t(x - y)\, f(y)\, dy.
\]

\paragraph{(i) Trace-Class Structure.}  
By Simon’s trace-class kernel criterion~\cite[Thm.~4.2]{Simon2005TraceIdeals}, the kernel
\[
K_t^{(\varphi)}(x, y) := \widehat{\Xi}_t(x - y)
\]
satisfies
\[
K_t^{(\varphi)} \in L^1(\R^2, \Psi_\alpha(x) \Psi_\alpha(y)\, dx\, dy),
\]
and hence \( L_t^{(\varphi)} \in \TC(H_{\Psi_\alpha}) \). Symmetry of both \( \varphi_t \) and \( \widehat{\Xi} \) ensures that \( L_t^{(\varphi)} \) is self-adjoint. The convergence structure is inherited from \lemref{lem:trace_class_Lt}.

\paragraph{(ii) Independence and Trace-Norm Convergence.}  
Let \( \varphi_t, \tilde{\varphi}_t \) be two mollifier families satisfying the above properties. Then:
\[
\widehat{\Xi}_t := \varphi_t * \widehat{\Xi}, \qquad \widehat{\widetilde{\Xi}}_t := \tilde{\varphi}_t * \widehat{\Xi},
\]
and define the associated convolution operators:
\[
L_t := L_t^{(\varphi)}, \qquad \widetilde{L}_t := L_t^{(\tilde{\varphi})}.
\]
Then,
\[
\| L_t - \widetilde{L}_t \|_{\TC(H_{\Psi_\alpha})}
\le \| \widehat{\Xi}_t - \widehat{\widetilde{\Xi}}_t \|_{L^1(\R, \Psi_\alpha)} \cdot \| \Psi_\alpha \|_{L^1(\R)}.
\]
Since
\[
\widehat{\Xi}_t - \widehat{\widetilde{\Xi}}_t = (\varphi_t - \tilde{\varphi}_t) * \widehat{\Xi},
\]
and \( \varphi_t - \tilde{\varphi}_t \to 0 \) in \( \Schwartz' \), we obtain
\[
\| \widehat{\Xi}_t - \widehat{\widetilde{\Xi}}_t \|_{L^1(\R, \Psi_\alpha)} \to 0
\]
by dominated convergence, using uniform exponential decay of the mollified kernels as established in \lemref{lem:trace_norm_convergence_Lt_to_Lsym}.

\paragraph{Conclusion.}  
The limiting operator
\[
L_{\sym} := \lim_{t \to 0^+} L_t^{(\varphi)} \in \TC(H_{\Psi_\alpha})
\]
is independent of the mollifier. Hence, \( L_{\sym} \) is canonically determined by the analytic profile \( \phi \) and the exponential weight \( \Psi_\alpha \), confirming the intrinsic, mollifier-independent construction.
\end{proof}
