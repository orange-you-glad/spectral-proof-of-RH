\begin{remark}[Scope of Spectral Generalization]
\label{rem:spectral_generalization_scope}
The results in Chapter~\ref{sec:spectral_generalization} extend the canonical operator framework to automorphic \( L \)-functions \( L(s, \pi) \) for \( \pi \in \mathcal{A}_{\mathrm{cusp}}(\mathrm{GL}_n) \), assuming only:
\begin{itemize}
  \item Analytic continuation and functional equation of the completed function \( \Xi(s, \pi) \);
  \item Exponential decay of the inverse Fourier kernel;
  \item Validity of the determinant identity (postulated and justified under trace-class convergence).
\end{itemize}

No additional hypotheses are required from the Langlands program, trace formula, or global arithmetic geometry. The modular architecture of the zeta case lifts to \( L(s, \pi) \) in full, with:
\begin{itemize}
  \item \textbf{Spectral Encoding}: Nontrivial zeros \( \rho \) map to eigenvalues \( \mu_\rho = \tfrac{1}{i}(\rho - \tfrac{1}{2}) \in \Spec(\Lsympi) \);
  \item \textbf{Determinant Identity}: \( \detz(I - \lambda \Lsympi) = \Xi(\tfrac{1}{2} + i\lambda, \pi)/\Xi(\tfrac{1}{2}, \pi) \);
  \item \textbf{GRH Equivalence}: \( \GRH(\pi) \iff \Spec(\Lsympi) \subset \R \).
\end{itemize}

Thus, the spectral model is general in scope and uniform in structure across all automorphic \( \pi \in \mathcal{A}_{\mathrm{cusp}}(\mathrm{GL}_n) \), conditioned only on analytic bounds that are expected from Langlands functoriality. Future extensions to motivic or Artin \( L \)-functions are conjectural and explored in Appendix~\ref{app:functorial_extensions}.
\end{remark}
