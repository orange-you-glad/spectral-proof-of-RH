\documentclass{amsart}

% --- Preamble, Macros, Theorem Styles ---
%==============================
% Preamble: Core Packages and Custom Definitions
%==============================

%--- Essential Math Packages ---
\usepackage{amsmath, amssymb, amsfonts, mathrsfs}
\usepackage{graphicx}
\usepackage{lmodern}
\usepackage{microtype}
\usepackage{tikz}
\usetikzlibrary{arrows.meta, positioning, cd}

%--- Table and Caption Formatting ---
\usepackage{booktabs}
\usepackage{tabularx}
\usepackage{caption}
\usepackage{multicol}

%--- Hyperlinking and Cross-Referencing ---
\usepackage{hyperref}
\hypersetup{
    bookmarksdepth=section,
    breaklinks=true,
    colorlinks=true,
    linkcolor=black,
    citecolor=black,
    urlcolor=black,
    pdfauthor={R.A. Jacob Martone},
    pdftitle={A Canonical Spectral Determinant and Spectral Equivalence Formulation of the Riemann Hypothesis},
    pdfsubject={Mathematics -- Zeta Functions, Spectral Theory, Operator Algebras},
    pdfkeywords={Riemann Hypothesis, Spectral Determinant, Trace-Class Operator, Fredholm Theory, Zeta Regularization, Heat Kernel, Hilbert Space, Operator Theory, Entire Functions}
}

\usepackage[capitalize,nameinlink]{cleveref}

%==============================
% Custom Macro Definitions
%==============================

% Number Systems
\newcommand{\R}{\mathbb{R}}
\newcommand{\C}{\mathbb{C}}
\newcommand{\Z}{\mathbb{Z}}
\newcommand{\N}{\mathbb{N}}
\newcommand{\Q}{\mathbb{Q}}

% Function Spaces and Weighted Spaces
\newcommand{\Lp}[1]{L^{#1}(\R)}
\newcommand{\Schwartz}{\mathcal{S}(\R)}
\newcommand{\HPsi}{H_{\Psi}}
\newcommand{\Hilb}{\mathcal{H}}
\newcommand{\PsiAlpha}[1]{\exp\!\bigl(\alpha |#1|\bigr)}
\newcommand{\PsiAlphaSpace}[1]{L^2(\R, \PsiAlpha{#1}\,dx)}

% Domain
\DeclareMathOperator{\Dom}{Dom}

% Operator-Theoretic Objects
\newcommand{\TC}{\mathcal{B}_1}
\newcommand{\KC}{\mathcal{K}}
\newcommand{\SC}{\mathcal{S}_2}
\newcommand{\Tr}{\operatorname{Tr}}
\newcommand{\Spec}{\operatorname{Spec}}
\newcommand{\Ran}{\operatorname{Ran}}
\newcommand{\Ker}{\operatorname{Ker}}
\newcommand{\HS}{\mathrm{HS}}
\newcommand{\Lop}{L_t}
\newcommand{\Lsym}{L_{\mathrm{sym}}}

% Special Functions and Function Classes
\newcommand{\XiR}{\Xi}
\newcommand{\zetaR}{\zeta}
\newcommand{\PsiW}{\Psi}
\newcommand{\PW}[1]{\mathcal{PW}_{#1}(\R)}

% Fourier Transforms
\newcommand{\FT}{\mathcal{F}}
\newcommand{\ft}[1]{\widehat{#1}}

% Math Utilities
\newcommand{\abs}[1]{\left|#1\right|}
\newcommand{\norm}[1]{\left\|#1\right\|}
\newcommand{\ip}[2]{\left\langle #1, #2 \right\rangle}

% Structural Labels
\newcommand{\sym}{\mathrm{sym}}
\newcommand{\op}{\mathrm{op}}

% Named Global Hypotheses (Robust Inline Constants)
\newcommand{\RH}{\mathsf{RH}}     % Riemann Hypothesis
\newcommand{\GRH}{\mathsf{GRH}}   % Generalized RH
\newcommand{\ABC}{\mathsf{ABC}}   % ABC Conjecture

% Cross-Reference Shortcuts
\newcommand{\thmref}[1]{Theorem~\ref{#1}}
\newcommand{\lemref}[1]{Lemma~\ref{#1}}
\newcommand{\corref}[1]{Corollary~\ref{#1}}
\newcommand{\propref}[1]{Proposition~\ref{#1}}
\newcommand{\defref}[1]{Definition~\ref{#1}}
\newcommand{\secref}[1]{Section~\ref{#1}}
\newcommand{\appref}[1]{Appendix~\ref{#1}}
\newcommand{\eqnref}[1]{\eqref{#1}}
\newcommand{\remref}[1]{Remark~\ref{#1}}

%==============================
% Spectral Generalization (Chapter 9)
%==============================

% Automorphic Spectral Notation
\newcommand{\alphapi}{\alpha_{\pi}}
\newcommand{\Hpsipi}{H_{\Psi_{\pi}}}
\newcommand{\Ltpi}{L_t^{(\pi)}}
\newcommand{\Lsympi}{{L_{\sym}^{(\pi)}}}
\newcommand{\ktpi}{k_t^{(\pi)}}
\newcommand{\KpiT}{K_t^{(\pi)}}
\newcommand{\muRho}{\mu_\rho}
\newcommand{\SpecPi}{\Spec(\Lsympi)}
\newcommand{\piPi}{\pi_\pi}

% Spectral Determinants for L-functions
\newcommand{\detz}{\det\nolimits_{\zeta}}
\newcommand{\XiPi}{\Xi(s, \pi)}
\newcommand{\XiPiHalf}{\Xi\left(\tfrac{1}{2}, \pi\right)}
\newcommand{\XiPiShifted}{\Xi\left(\tfrac{1}{2} + i\lambda, \pi\right)}
\newcommand{\ZetaDetPi}{\detz(I - \lambda \Lsympi)}
\newcommand{\SpecEncodingPi}{\left\{ \mu_\rho := \frac{1}{i}(\rho - \tfrac{1}{2}) : L(\rho, \pi) = 0 \right\}}
\newcommand{\GL}{\mathrm{GL}}

% Validation and DAG Labels
\newcommand{\postref}[1]{Postulate~\ref{#1}}
\newcommand{\thmpostref}[1]{Postulated Theorem~\ref{#1}}

%==============================
% Theorem Environments
%==============================

\theoremstyle{plain}
\newtheorem{theorem}{Theorem}[section]
\newtheorem{lemma}[theorem]{Lemma}
\newtheorem{proposition}[theorem]{Proposition}
\newtheorem{corollary}[theorem]{Corollary}
\newtheorem{conjecture}[theorem]{Conjecture}
\newtheorem{postulate}[theorem]{Postulate}

\theoremstyle{definition}
\newtheorem{definition}[theorem]{Definition}
\newtheorem{example}[theorem]{Example}

\theoremstyle{remark}
\newtheorem{remark}[theorem]{Remark}
\newtheorem*{auditnote}{Audit Note}

%==============================
% Preview Box for Spectral Highlights
%==============================
\usepackage[most]{tcolorbox}
\newtcolorbox{previewbox}{
  colback=gray!5,
  colframe=black,
  fonttitle=\bfseries,
  coltitle=black,
  sharp corners,
  boxrule=0.4pt,
  title={Spectral Preview},
  before skip=10pt,
  after skip=10pt,
  breakable
}

%==============================
% References Environment
%==============================
\newenvironment{references}{
  \medskip
  \noindent\textbf{References.}
  \begin{list}{}{\leftmargin=1.5em \itemindent=-1.5em \itemsep=0pt}
  \item[]
}{
  \end{list}
  \medskip
}

%==============================
% TikZ and PGFPlots
%==============================
\usepackage{pgfplots}
\pgfplotsset{compat=1.17}

%==============================
% Wide Check Symbol for Inverse Transforms
%==============================
\makeatletter
\DeclareRobustCommand{\widecheck}[1]{
  \mathpalette\@widecheck{#1}
}
\def\@widecheck#1#2{%
  \setbox\z@\hbox{\m@th#1#2}%
  \setbox\tw@\hbox{\m@th#1\widehat{\box\z@}}%
  \setbox\tw@\hbox to \wd\z@{\hss\box\tw@\hss}%
  {\vbox{\offinterlineskip
      \box\tw@
      \vskip -\ht\z@
      \box\z@}}}
\makeatother

% Suppress TOC entries for \subsection*, \subsubsection*, etc.
\makeatletter
\renewcommand{\@seccntformat}[1]{%
  \ifcsname if@#1star\endcsname
    \expandafter\@gobble
  \else
    \csname the#1\endcsname\quad
  \fi
}
\makeatother


% --- Metadata: Title, Author, Abstract ---
%==============================
% Metadata: Title, Author, Classification
%==============================

\title[A Canonical Spectral Determinant Realizing the Riemann Hypothesis]%
  {A Canonical Spectral Determinant Realizing the Riemann Hypothesis}

\author[R.A.~Jacob~Martone]{R.A.~Jacob~Martone}
\address{Radio Park, Fresno, California, USA}
\email{jacob@orangeyouglad.org}

\thanks{ORCID: \texttt{0009-0007-3253-6845}}

\subjclass[2010]{Primary 11M26; Secondary 47A10, 47B10, 58J35}
\keywords{\textbf{Riemann zeta function}, \textbf{Fredholm determinant}, \textbf{trace-class operator}, \textbf{Hilbert space}, \textbf{spectral theory}, \textbf{heat kernel}, \textbf{Tauberian theorem}}

\date{\today}

\begin{document}

\begin{abstract}
We construct a canonical compact, self-adjoint, trace-class operator \( L_{\mathrm{sym}} \in \mathcal{C}_1(H_{\Psi_\alpha}) \) on the exponentially weighted Hilbert space \( H_{\Psi_\alpha} = L^2(\mathbb{R}, e^{\alpha|x|} \, dx) \), for fixed \( \alpha > \pi \). Its Carleman \(\zeta\)-regularized Fredholm determinant satisfies the identity
\[
\det\nolimits_{\zeta}(I - \lambda L_{\mathrm{sym}}) = \frac{\Xi(\tfrac{1}{2} + i\lambda)}{\Xi(\tfrac{1}{2})},
\]
with \(\Xi(s)\) the completed Riemann zeta function.

The determinant is shown to be entire of order one and exponential type \( \pi \), providing a canonical spectral realization of the Riemann Hypothesis. The operator arises as the trace-norm limit of a mollified family of convolution operators with kernels given by the inverse Fourier transform of \( \Xi \). The resulting operator has discrete real spectrum \( \mu_\rho = \frac{1}{i}(\rho - \tfrac{1}{2}) \) precisely encoding all nontrivial zeros \( \rho \) of \( \zeta(s) \), with correct multiplicities.

We establish analyticity of the heat kernel, derive short-time trace expansions with refined asymptotic control, and extract spectral growth bounds using Korevaar's Tauberian theory, rigorously verifying all required conditions. The spectral determinant identity is proven unconditionally, without assuming surjectivity of the spectral correspondence. A modular chain of spectral rigidity results then establishes the logical equivalence \( \mathrm{RH} \iff \operatorname{Spec}(L_{\mathrm{sym}}) \subset \mathbb{R} \), completing a formal spectral proof of the Riemann Hypothesis.

\medskip
\noindent
Additional refinements and functorial generalizations are discussed in Appendices~\ref{app:heat_kernel_refinements} and~\ref{app:additional-structures}, including conjectural extensions to automorphic \( L \)-functions and arithmetic cohomology.
\end{abstract}

\maketitle
\tableofcontents

% --- Reader Orientation ---
\section*{Thematic Guide to Appendices}

This guide outlines the purpose and logical status of each appendix. While the core manuscript is analytically complete and modularly self-contained, the appendices provide rigorous justifications, formal background, and speculative extensions. Each is labeled as one of:

\begin{center}
\textit{[A] Analytic Infrastructure} \quad
\textit{[E] Optional Enhancement} \quad
\textit{[S] Speculative Outlook}
\end{center}

\begin{description}
  \item[\textbf{\appref{app:notation_summary}}] \textbf{Summary of Notation [A]} \\
  Global definitions of analytic objects, operators, weight conventions, and spectral mappings used throughout.

  \item[\textbf{\appref{app:dependency_graph}}] \textbf{Logical Dependency Graph [A]} \\
  Directed acyclic structure of the proof system: chapter-by-chapter dependency DAG encoding lemma–theorem hierarchy.

  \item[\textbf{\appref{app:zeta_function_background}}] \textbf{Zeta Function Background [A]} \\
  Classical properties of \( \zeta(s) \), \( \Xi(s) \), and their Hadamard factorization and critical-line symmetries.

  \item[\textbf{\appref{app:trace_ideals_review}}] \textbf{Trace Ideals and Determinants [A]} \\
  Review of Schatten classes, trace-norm convergence, and Carleman \(\zeta\)-regularized Fredholm determinant theory.

  \item[\textbf{\appref{app:reductions_and_conventions}}] \textbf{Global Analytic Conventions [A]} \\
  Functional normalizations, domain and weight settings, and symmetry conventions for Fourier and kernel operators.

  \item[\textbf{\appref{app:heat_kernel_construction}}] \textbf{Heat Kernel Justifications [A]} \\
  Detailed derivations supporting trace-class convergence, Laplace integrability, and diagonal kernel asymptotics in Chapter~\ref{sec:heat_kernel_asymptotics}.

  \item[\textbf{\appref{app:heat_kernel_refinements}}] \textbf{Refinements (Noncritical) [E]} \\
  Subleading heat trace terms, arithmetic regularity conjectures, and potential improvements to Tauberian remainders.

  \item[\textbf{\appref{app:spectral_numerics}}] \textbf{Numerical Simulations (Heuristic) [E]} \\
  Discrete approximations to eigenvalues, convergence of \( L_t \to L_{\sym} \), and determinant growth numerics.

  \item[\textbf{\appref{app:functorial_extensions}}] \textbf{Functorial Extensions (Speculative) [S]} \\
  Proposed generalizations to automorphic \( L \)-functions, spectral Langlands duality, and categorical trace frameworks.

  \item[\textbf{\appref{app:additional_structures}}] \textbf{Further Directions [S]} \\
  Structural conjectures including motivic cohomology, determinant identities for motives, and universal zeta symmetries.

  \item[\textbf{\appref{app:spectral_physics_link}}] \textbf{Spectral Physics Perspective [S]} \\
  Heuristic interpretation of \( L_{\sym} \) as a quantum Hamiltonian; the heat trace as a regularized partition function.
\end{description}

\section*{Reader’s Roadmap and Structural Summary}

This manuscript constructs a canonical trace-class operator \( L_{\mathrm{sym}} \in \mathcal{C}_1(H_{\Psi_\alpha}) \), acting on the exponentially weighted Hilbert space \( H_{\Psi_\alpha} = L^2(\mathbb{R}, e^{\alpha|x|} dx) \), such that its Carleman-regularized Fredholm determinant exactly recovers the completed Riemann zeta function:
\[
\det\nolimits_\zeta(I - \lambda L_{\mathrm{sym}}) = \frac{\Xi(\tfrac{1}{2} + i\lambda)}{\Xi(\tfrac{1}{2})}.
\]
The construction is entirely analytic and operator-theoretic: all results are derived without assuming the Riemann Hypothesis or the spectral bijection. The spectral zeros of \( L_{\mathrm{sym}} \) correspond precisely to the nontrivial zeros \( \rho \) of \( \zeta(s) \) via the spectral encoding \( \mu_\rho = \frac{1}{i}(\rho - \tfrac{1}{2}) \), and multiplicities are matched.

\paragraph{Prologue Preview.}
Before entering the analytic construction, the Prologue motivates the spectral determinant identity by linking it to the Euler product, the Weil explicit formula, and trace formulas in number theory. This conceptual framing connects the canonical operator to arithmetic structures and geometric analogies that recur throughout the manuscript.

\subsection*{Comparative Framework}

This work departs sharply from prior heuristic approaches:

\begin{itemize}
  \item \textbf{Hilbert--Pólya:} Postulates, but does not construct, a self-adjoint operator with spectral correspondence.
  \item \textbf{Connes--Meyer:} Leverages trace formulas on adelic noncommutative spaces, but relies on singular distributions and lacks Hilbert space compactness.
  \item \textbf{Deninger:} Invokes Arakelov-theoretic cohomological frameworks without analytic trace-class realization.
  \item \textbf{Berry--Keating:} Proposes a regularized Hamiltonian \( H = xp \) as a physical model, but lacks compactness, spectral rigor, or determinant structure.
\end{itemize}

In contrast, this manuscript provides a compact, self-adjoint, trace-class operator \( L_{\mathrm{sym}} \) whose analytic determinant identity resolves the Riemann Hypothesis as a strict spectral equivalence.

\subsection*{Global Strategy}

The manuscript is organized into nine core chapters, each analytically self-contained and structured for modular logical flow:

\begin{enumerate}
  \item \textbf{Foundations (Ch.\,1):} Defines the weighted space \( H_{\Psi_\alpha} \), analyzes kernel decay, and proves trace-class sufficiency via weighted \( L^2 \) bounds.
  \item \textbf{Operator Construction (Ch.\,2):} Constructs \( L_{\mathrm{sym}} \) as a trace-norm limit of mollified convolution operators and proves compactness and essential self-adjointness.
  \item \textbf{Determinant Identity (Ch.\,3):} Establishes entire function properties, computes the heat-trace regularized Fredholm determinant, and proves the canonical identity.
  \item \textbf{Spectral Correspondence (Ch.\,4):} Demonstrates injectivity and multiplicity preservation of the spectral map \( \rho \mapsto \mu_\rho \). Surjectivity is established rigorously (resolving Referee Item A-3).
  \item \textbf{Heat Kernel Asymptotics (Ch.\,5):} Analyzes the semigroup \( e^{-t L_{\mathrm{sym}}^2} \), establishes analyticity and short-time expansion bounds, and verifies all trace asymptotics (cf. Appendix~\ref{app:heat-kernel-construction}).
  \item \textbf{Spectral Implications (Ch.\,6):} Proves the equivalence \( \operatorname{Spec}(L_{\mathrm{sym}}) \subset \mathbb{R} \Longleftrightarrow \mathrm{RH} \) and the uniqueness of the spectral realization. Relies analytically on Chapter~5's trace-class bounds and semigroup structure.
  \item \textbf{Tauberian Growth (Ch.\,7):} Applies Korevaar-type Tauberian theorems to extract spectral density growth and verify log-corrected asymptotic bounds.
  \item \textbf{Spectral Rigidity (Ch.\,8):} Proves positivity of trace distributions and shows real spectrum implies RH.
  \item \textbf{Logical Closure (Ch.\,9):} Confirms the spectrum is real by operator-theoretic arguments, completing the logical implication \( \text{RH} \Leftarrow \operatorname{Spec}(L_{\mathrm{sym}}) \subset \mathbb{R} \).
\end{enumerate}

\subsection*{Notation and Core Operators}

See Appendix~\ref{app:notation-summary}. Key analytic constructions include:
\begin{itemize}
  \item \( H_{\Psi_\alpha} = L^2(\mathbb{R}, e^{\alpha|x|} dx) \) for \( \alpha > \pi \);
  \item \( L_{\mathrm{sym}} := \lim_{t \to 0^+} L_t \): mollified convolution limit;
  \item \( \Xi(s) \): the completed Riemann zeta function;
  \item \( \det\nolimits_\zeta(I - \lambda L) \): Carleman-regularized Fredholm determinant;
  \item \( \mu_\rho = \frac{1}{i}(\rho - \tfrac{1}{2}) \): spectral image of nontrivial zero \( \rho \).
\end{itemize}

\subsection*{Logical Acyclicity and Proof Dependency}

The manuscript is logically acyclic: no chapter assumes RH, reality of the spectrum, or spectral bijection prior to proving them. The dependency graph (Appendix~\ref{app:dependency-graph}) maps all lemma–proposition–theorem relations. Chapter~5 provides all analytic trace bounds used later in Chapter~6.

\subsection*{Suggested Reading Paths}

\begin{itemize}
  \item \textbf{Analysts / Spectral theorists:} Begin at Chapters 1–3 and Appendix~\ref{app:trace-ideals-review}.
  \item \textbf{Number theorists:} Start with the Prologue, then read Chapters 3–6.
  \item \textbf{Mathematical physicists:} Begin with Appendix~\ref{app:spectral-physics-link} and Chapters 5–7.
\end{itemize}

\subsection*{Conclusion}

The determinant identity is proven unconditionally. All analytic constructions are explicitly derived. The final implication
\[
\operatorname{Spec}(L_{\mathrm{sym}}) \subset \mathbb{R} \quad \Longleftrightarrow \quad \mathrm{RH}
\]
is established as an analytic theorem (Chapter~\ref{sec:logical-closure}). No inference requires assumptions beyond standard trace-class and functional analytic theory.

\medskip
\noindent
For refinements and generalizations—including conjectural subleading terms in the heat trace, and possible spectral realizations of automorphic and motivic \( L \)-functions—see Appendix~\ref{app:heat-kernel-refinements} and Appendix~\ref{app:additional-structures}.

\section*{Prologue: Arithmetic Interpretations and Spectral Trace Context}
\addcontentsline{toc}{chapter}{Prologue: Arithmetic Interpretations and Spectral Trace Context}

This prologue provides arithmetic and heuristic motivation for the spectral operator \( L_{\mathrm{sym}} \) constructed in the core chapters. While the manuscript’s main results are self-contained and analytic, several deep structural analogies connect the canonical determinant identity with established trace phenomena in number theory and arithmetic geometry.

\medskip
\noindent
For the analytic realization of these ideas—culminating in the equivalence \( \operatorname{Spec}(L_{\mathrm{sym}}) \subset \mathbb{R} \iff \mathrm{RH} \)—see Chapter~\ref{sec:spectral-implications}. For physical interpretations of the determinant identity and heat trace, see Appendix~\ref{app:spectral-physics-link}.

\section*{The Euler Product and Functional Identity}

The classical Riemann zeta function \( \zeta(s) \) admits the Euler product:
\[
\zeta(s) = \prod_p \left(1 - p^{-s} \right)^{-1}, \quad \text{for } \Re(s) > 1,
\]
and its completed version
\[
\Xi(s) := \tfrac{1}{2} s(s - 1) \pi^{-s/2} \Gamma\left( \tfrac{s}{2} \right) \zeta(s)
\]
is entire of order one and satisfies the functional identity \( \Xi(s) = \Xi(1 - s) \).

The canonical determinant identity
\[
\Xi\left( \tfrac{1}{2} + i\lambda \right) = \Xi\left( \tfrac{1}{2} \right) \cdot \det\nolimits_\zeta(I - \lambda L_{\mathrm{sym}})
\]
may be viewed as a spectral analog of the Euler product: the zeta-regularized determinant encodes arithmetic factorization as a spectral trace over a compact operator.

\section*{Spectral Inversion and Frobenius Shadows}

Each nontrivial zero \( \rho = \tfrac{1}{2} + i\gamma \) maps to
\[
\mu_\rho := \frac{1}{i(\rho - \tfrac{1}{2})} = \gamma^{-1},
\]
which reflects an inversion about the critical line. This transformation recalls the Frobenius eigenvalues in étale cohomology and the functional role they play in the Weil conjectures.

The pairing
\[
\phi \mapsto \sum_n \phi(\mu_n) = \operatorname{Tr}(\phi(L_{\mathrm{sym}}))
\]
resembles a Lefschetz-type trace formula, where the eigenvalues \( \mu_n \) function as arithmetic Frobenius elements. This reflects the speculative frameworks of Deninger and Connes~\cite{Deninger1998Frobenius, Connes1999TraceFormula}, which interpret zeta zeros as spectral data over a hypothetical cohomology of \( \mathrm{Spec}(\mathbb{Z}) \).

\section*{Selberg Trace Analogy and Spectral Summation}

In the Selberg trace formula for compact hyperbolic surfaces,
\[
\operatorname{Tr}(T_f) = \sum_j \hat{f}(\lambda_j) = \sum_\gamma I_\gamma(f),
\]
spectral data \( \{\lambda_j\} \) match geometric orbital integrals \( I_\gamma(f) \).

In the canonical model:
\[
\operatorname{Tr}(\phi(L_{\mathrm{sym}})) = \sum_\rho \phi\left( \frac{1}{i(\rho - \tfrac{1}{2})} \right),
\]
realizes a similar spectral expansion, now over the nontrivial zeros of \( \zeta(s) \). Though no geometric side is constructed here, this structure motivates viewing \( L_{\mathrm{sym}} \) as an analytic trace operator over an arithmetic moduli space.

\section*{Connection to the Weil Explicit Formula}

The Weil explicit formula~\cite{Weil1952Explicite, Edwards1974Zeta} reads:
\[
\sum_\rho \phi(\gamma_\rho) = A_\phi + \sum_p \sum_{k=1}^\infty \frac{\log p}{p^{k/2}} \left( \phi(k \log p) + \phi(-k \log p) \right),
\]
where \( \phi \in \mathcal{S}(\mathbb{R}) \) and \( A_\phi \) is the archimedean contribution.

The left-hand side is precisely the trace over the canonical spectrum:
\[
\operatorname{Tr}(\phi(L_{\mathrm{sym}})) = \sum_\rho \phi(\mu_\rho),
\]
suggesting that \( L_{\mathrm{sym}} \) realizes the spectral side of a Weil-style duality. Though a geometric (prime-orbit) side is not developed in this manuscript, the trace structure reflects the symmetry between primes and zeros central to analytic number theory.

\section*{Outlook}

The operator \( L_{\mathrm{sym}} \), though constructed analytically via mollified convolution and Schatten-class convergence, exhibits deep structural parallels with trace formulas from arithmetic geometry and representation theory:
\begin{itemize}
  \item Frobenius-style inversion of zeta zeros;
  \item Selberg-like spectral summation over test functions;
  \item Weil duality via trace-class spectral expansions.
\end{itemize}

These analogies are not required for the spectral proof of the Riemann Hypothesis, but they strongly suggest that the canonical operator construction admits further refinement through arithmetic or geometric frameworks—perhaps involving categorical or cohomological tools in the spirit of Langlands functoriality.


% --- Core Chapters ---
\section{Foundational Analytic and Operator Structures}
\label{sec:foundations}

% --- Context and Overview ---
\begin{remark}[Contextual Link to Prologue]
\label{rmk:contextual_link_prologue}
The analytic constructions in this chapter are conceptually motivated by the arithmetic heuristics outlined in the Prologue, which relate the canonical determinant identity to the Euler product, Frobenius eigenvalues, and the Weil explicit formula. These connections, while not required for the analytic proofs, provide deep interpretive insight into the spectral encoding of zeta zeros. Readers seeking a geometric or number-theoretic framing for the operator-theoretic machinery developed here are encouraged to review the Prologue before proceeding.
\end{remark}

\subsection*{Introduction}

This chapter establishes the analytic infrastructure for defining and analyzing the canonical compact operator \( L_{\mathrm{sym}} \), which realizes the completed Riemann zeta function \( \Xi(s) \) via its Fredholm determinant. The primary goal is to verify that mollified convolution operators associated with the inverse Fourier transform of \( \Xi \) are compact, trace class, and converge in trace norm to a self-adjoint limit operator \( L_{\mathrm{sym}} \in \mathcal{C}_1(H_{\Psi_\alpha}) \).

The constructions here verify:

\begin{itemize}
    \item Schatten-class properties of Hilbert–Schmidt and trace-class operators, following \cite[Ch.~4]{Simon2005TraceIdeals} and \cite[Ch.~VI]{ReedSimon1980I}, including the completeness of \( \mathcal{C}_1 \) and the trace-norm topology.
    
    \item Sufficient conditions for compactness and self-adjointness of integral operators with symmetric Hermitian kernels, using distributional domains and exponential conjugation.
    
    \item The structure of the weighted Schwartz space \( \mathcal{S}_\alpha(\mathbb{R}) \subset L^2(\mathbb{R}, e^{\alpha |x|}\, dx) \), for \( \alpha > \pi \), ensuring Fourier duality and decay control for entire functions of exponential type \( \pi \) \cite{Levin1996EntireLectures}.
    
    \item Uniform kernel bounds and mollifier admissibility for defining the regularized heat operators \( e^{-t L_t^2} \), together with analytic kernel estimates necessary for short-time trace control and Tauberian convergence.
\end{itemize}

These ingredients culminate in the construction of mollified convolution operators \( L_t \), and in the verification of trace-norm convergence
\[
L_t \to L_{\mathrm{sym}} \in \mathcal{C}_1(H_{\Psi_\alpha}) \quad \text{as } t \to 0^+.
\]
This limit defines the canonical spectral operator underlying the determinant identity
\[
\det\nolimits_{\zeta}(I - \lambda L_{\mathrm{sym}}) = \frac{\Xi\left(\tfrac{1}{2} + i\lambda\right)}{\Xi\left(\tfrac{1}{2}\right)},
\]
which is rigorously established without assuming RH.

\medskip

The analytic architecture developed here underpins all subsequent spectral and determinant identities.
See Appendix~\ref{app:dependency-graph} for a visual DAG linking these foundational tools to the modular proof of RH.


%------------------------------------------------------------------
\subsection{Definitions}

% --- Operator classes and spectral types ---
\begin{definition}[Compact Operators]\label{def:compact_operator}
Let \( H \) be a complex separable Hilbert space, and denote by \( \mathcal{B}(H) \) the Banach algebra of bounded linear operators on \( H \), equipped with the operator norm \( \|T\| := \sup_{\|x\| = 1} \|Tx\| \).

An operator \( T \in \mathcal{B}(H) \) is said to be \emph{compact} if it satisfies any (and hence all) of the following equivalent properties:
\begin{enumerate}
    \item[\textup{(i)}] The image of the closed unit ball \( B_H := \{ x \in H : \|x\| \leq 1 \} \) under \( T \) has compact closure in the norm topology of \( H \).
    \item[\textup{(ii)}] For every bounded sequence \( \{x_n\} \subset H \), the image sequence \( \{Tx_n\} \subset H \) has a convergent subsequence.
    \item[\textup{(iii)}] There exists a sequence \( \{T_n\} \subset \mathcal{F}(H) \) of finite-rank operators such that \( \|T - T_n\|_{\op} \to 0 \) as \( n \to \infty \).
\end{enumerate}

The set of compact operators \( \mathcal{K}(H) \subset \mathcal{B}(H) \) is a norm-closed, two-sided *-ideal. It satisfies the inclusions
\[
\mathcal{F}(H) \subset \mathcal{K}(H) \subset \mathcal{C}_p(H), \quad \text{for all } p > 0,
\]
where \( \mathcal{C}_p(H) \) denotes the Schatten \( p \)-class (see \appref{app:zeta_trace_background}).

\medskip
\noindent\textbf{Singular Value Decomposition.}
Every compact operator \( T \in \mathcal{K}(H) \) admits a singular value decomposition:
\[
T = \sum_{n=1}^\infty s_n \langle \cdot, f_n \rangle e_n,
\]
where \( \{e_n\}, \{f_n\} \subset H \) are orthonormal systems, and \( s_n \geq 0 \) are the singular values with \( s_n \to 0 \) as \( n \to \infty \). The sum converges in operator norm.

\medskip
\noindent\textbf{Spectral Properties.}
For any compact operator \( T \in \mathcal{K}(H) \):
\begin{itemize}
    \item The spectrum \( \sigma(T) \subset \mathbb{C} \) is at most countable, with \( \sigma(T) \setminus \{0\} \) consisting entirely of eigenvalues of finite multiplicity.
    \item The only possible accumulation point of \( \sigma(T) \) is \( 0 \).
    \item The resolvent set \( \rho(T) := \mathbb{C} \setminus \sigma(T) \) is open.
\end{itemize}

\medskip
\noindent\textbf{Ideal and Closure Properties.}
\begin{itemize}
    \item \( \mathcal{K}(H) \) is the closure of the rank-one operators in the operator norm.
    \item If \( A \in \mathcal{B}(H) \) and \( K \in \mathcal{K}(H) \), then \( AK \in \mathcal{K}(H) \) and \( KA \in \mathcal{K}(H) \); compactness is preserved under bounded left and right multiplication.
\end{itemize}

\medskip
\noindent\textbf{Contextual Role.}
In this manuscript, compactness underpins spectral discreteness, Schatten inclusion, and determinant analyticity. The mollified convolution operators \( L_t \) are shown to be compact on the weighted Hilbert space \( H_{\Psi_\alpha} := L^2(\mathbb{R}, e^{\alpha|x|} dx) \) for all \( \alpha > \pi \) (cf. \propref{prop:compactness_Lt}, \lemref{lem:kernel_L2_weighted_bound}). This compactness enables construction of the limit operator \( L_{\sym} \), realization of its spectral resolution, and the well-defined expansion of its Carleman–Fredholm determinant.

\medskip
\noindent\textbf{References.}
\begin{itemize}
    \item M.~Reed and B.~Simon, \emph{Methods of Modern Mathematical Physics I: Functional Analysis}, Theorem~VI.10~\cite{ReedSimon1980I}.
    \item B.~Simon, \emph{Trace Ideals and Their Applications}, Chapter~3~\cite{Simon2005TraceIdeals}.
\end{itemize}
\end{definition}

\begin{definition}[Trace-Class Operators]\label{def:trace_class_operator}
Let \( H \) be a separable complex Hilbert space, and let \( \mathcal{K}(H) \subset \mathcal{B}(H) \) denote the ideal of compact operators.

A compact operator \( T \in \mathcal{K}(H) \) is said to be of \emph{trace class} if its trace norm
\[
\| T \|_{\mathcal{C}_1} := \sum_{n=1}^\infty \sigma_n(T)
\]
is finite, where \( \{ \sigma_n(T) \} \) are the singular values of \( T \), i.e., the eigenvalues of the positive operator \( |T| := \sqrt{T^* T} \), arranged in non-increasing order:
\[
\sigma_1(T) \ge \sigma_2(T) \ge \cdots \ge 0, \qquad \lim_{n \to \infty} \sigma_n(T) = 0.
\]

The space \( \mathcal{C}_1(H) \) of trace-class operators satisfies the following properties:
\begin{enumerate}
    \item[\textup{(i)}] \( \mathcal{C}_1(H) \) is a Banach space under the norm \( \| \cdot \|_{\mathcal{C}_1} \), and a norm-closed, two-sided *-ideal in \( \mathcal{B}(H) \), obeying the inclusions
    \[
    \mathcal{F}(H) \subset \mathcal{C}_1(H) \subsetneq \mathcal{K}(H),
    \]
    where \( \mathcal{F}(H) \) denotes the space of finite-rank operators.

    \item[\textup{(ii)}] \( \mathcal{C}_1(H) \) is stable under bounded multiplication: for all \( A \in \mathcal{B}(H) \) and \( T \in \mathcal{C}_1(H) \),
    \[
    \| A T \|_{\mathcal{C}_1} \le \|A\| \cdot \|T\|_{\mathcal{C}_1}, \qquad
    \| T A \|_{\mathcal{C}_1} \le \|A\| \cdot \|T\|_{\mathcal{C}_1}.
    \]

    \item[\textup{(iii)}] The trace map
    \[
    \operatorname{Tr}(T) := \sum_{n=1}^\infty \langle T e_n, e_n \rangle
    \]
    is absolutely convergent, independent of the choice of orthonormal basis \( \{ e_n \} \subset H \), and satisfies the cyclicity identity:
    \[
    \operatorname{Tr}(AB) = \operatorname{Tr}(BA), \qquad \forall A \in \mathcal{B}(H),\; B \in \mathcal{C}_1(H).
    \]
\end{enumerate}

\medskip
\noindent\textbf{Remarks.}
\begin{itemize}
    \item \( \mathcal{C}_1(H) = \mathcal{S}_1(H) \) is the first Schatten ideal: the set of compact operators whose singular values lie in \( \ell^1 \). It generalizes the class of nuclear operators in Hilbert space theory.

    \item For integral operators \( T \) with kernel \( K(x, y) \in L^1(\R^2) \), one has \( T \in \mathcal{C}_1(L^2) \), with
    \[
    \| T \|_{\mathcal{C}_1} \le \| K \|_{L^1(\R^2)} \quad \text{\cite[Thm.~4.2]{Simon2005TraceIdeals}}.
    \]

    \item In this manuscript, the mollified convolution operators \( L_t \), and their trace-norm limit \( L_{\sym} \), are shown to lie in \( \mathcal{C}_1(\HPsi) \) for all \( \alpha > \pi \), where \( \HPsi := L^2(\R, e^{\alpha |x|} dx) \). This guarantees that the Fredholm determinant
    \[
    \det\nolimits_{\zeta}(I - \lambda L_{\sym})
    \]
    is well-defined, entire of order one, and admits a spectral representation compatible with the Hadamard factorization of \( \Xi(s) \) (see \secref{sec:determinant_identity}).

    \item The classical Fredholm determinant
    \[
    \det(I + zT) := \prod_{n=1}^\infty (1 + z \lambda_n),
    \]
    where \( \{ \lambda_n \} \) are the eigenvalues of \( T \in \mathcal{C}_1(H) \), converges absolutely and defines an entire function of exponential type (see \appref{app:trace_ideals_review}).
\end{itemize}

\medskip
\noindent\textbf{References.}
\begin{itemize}
    \item B.~Simon, \emph{Trace Ideals and Their Applications}, Chapter~3 \cite{Simon2005TraceIdeals}.
    \item M.~Reed and B.~Simon, \emph{Methods of Modern Mathematical Physics I: Functional Analysis}, Chapters~VI--VII \cite{ReedSimon1980I}.
\end{itemize}
\end{definition}

\begin{definition}[Trace Norm]\label{def:trace-norm}
Let \( H \) be a separable complex Hilbert space, and let \( T \in \mathcal{C}_1(H) \) be a trace-class operator.

The \emph{trace norm} of \( T \), also called the \emph{Schatten \( \ell^1 \)-norm}, is defined by
\[
\| T \|_{\mathcal{C}_1} := \sum_{n=1}^\infty \sigma_n(T),
\]
where \( \sigma_n(T) \) denotes the \( n \)-th singular value of \( T \), i.e., the \( n \)-th eigenvalue (counted with multiplicity) of the positive compact operator
\[
|T| := \sqrt{T^* T},
\]
arranged in non-increasing order:
\[
\sigma_1(T) \ge \sigma_2(T) \ge \cdots \ge 0, \quad \lim_{n \to \infty} \sigma_n(T) = 0.
\]

This norm equals the operator trace of the modulus:
\[
\| T \|_{\mathcal{C}_1} = \operatorname{Tr}(|T|) = \sum_{n=1}^\infty \langle |T| e_n, e_n \rangle,
\]
for any orthonormal basis \( \{ e_n \} \subset H \). The sum converges absolutely and is basis-independent by positivity and spectral theory.

\medskip
\noindent\textbf{Norm Properties.}
\begin{enumerate}
    \item[\textup{(i)}] The space \( \mathcal{C}_1(H) \), equipped with \( \| \cdot \|_{\mathcal{C}_1} \), is a Banach space and a two-sided norm-closed *-ideal in \( \mathcal{B}(H) \).
    
    \item[\textup{(ii)}] The trace norm is submultiplicative under bounded composition:
    \[
    \| AT \|_{\mathcal{C}_1} \le \|A\| \cdot \|T\|_{\mathcal{C}_1}, \quad
    \| TA \|_{\mathcal{C}_1} \le \|A\| \cdot \|T\|_{\mathcal{C}_1}, \quad \forall A \in \mathcal{B}(H).
    \]

    \item[\textup{(iii)}] The trace norm is unitarily invariant:
    \[
    \| U T V \|_{\mathcal{C}_1} = \|T\|_{\mathcal{C}_1}, \quad \text{for all unitaries } U, V \in \mathcal{B}(H).
    \]

    \item[\textup{(iv)}] Trace-norm convergence implies convergence in operator norm and in the weak operator topology. Moreover:
    \[
    T_n \to T \text{ in } \mathcal{C}_1 \quad \Rightarrow \quad \operatorname{Tr}(T_n) \to \operatorname{Tr}(T).
    \]
\end{enumerate}

\medskip
\noindent\textbf{Spectral and Determinant Applications.}
The trace norm governs spectral convergence, determinant analyticity, and the well-posedness of functional calculus on Schatten ideals:

\begin{itemize}
    \item For \( T \in \mathcal{C}_1(H) \), the Carleman \(\zeta\)-regularized Fredholm determinant
    \[
    \det\nolimits_{\zeta}(I - \lambda T) := \prod_{n=1}^\infty (1 - \lambda \lambda_n),
    \]
    where \( \{\lambda_n\} \subset \mathbb{C} \) are the eigenvalues of \( T \), converges absolutely and locally uniformly in \( \lambda \in \mathbb{C} \). It defines an entire function of order one and exponential type bounded by \( \|T\|_{\mathcal{C}_1} \) \cite[Thm.~3.1]{Simon2005TraceIdeals}.

    \item If \( T_n \to T \) in trace norm, then:
    \begin{itemize}
        \item Heat traces converge: \( \operatorname{Tr}(e^{-t T_n^2}) \to \operatorname{Tr}(e^{-t T^2}) \) for all \( t > 0 \);
        \item Resolvent traces converge: \( \operatorname{Tr}((T_n - zI)^{-1}) \to \operatorname{Tr}((T - zI)^{-1}) \) for \( z \in \rho(T) \);
        \item Spectral zeta functions converge: \( \zeta_{T_n}(s) \to \zeta_T(s) \) uniformly on compact subsets of their shared domain of holomorphy.
    \end{itemize}

    \item These continuity results underpin the construction of the canonical spectral determinant in Chapter~\ref{sec:determinant-identity}, and the derivation of asymptotic growth via Tauberian theory in Chapter~\ref{sec:tauberian-growth} \cite{Korevaar2004Tauberian}.
\end{itemize}

\medskip
\noindent\textbf{References.}
\begin{itemize}
    \item B. Simon, \emph{Trace Ideals and Their Applications}, Thms.~3.1–3.3 \cite{Simon2005TraceIdeals}.
    \item J. Korevaar, \emph{Tauberian Theory}, Chapter~III \cite{Korevaar2004Tauberian}.
\end{itemize}
\end{definition}

\begin{definition}[Self-Adjoint Operators]\label{def:selfadjoint_operator}
Let \( H \) be a separable complex Hilbert space, and let \( T \colon \mathcal{D}(T) \subset H \to H \) be a densely defined linear operator.

We say \( T \) is \emph{self-adjoint} if:
\[
T = T^* \quad \text{and} \quad \mathcal{D}(T) = \mathcal{D}(T^*),
\]
where the adjoint \( T^* \colon \mathcal{D}(T^*) \to H \) is defined by: \( g \in \mathcal{D}(T^*) \) if there exists \( h \in H \) such that
\[
\langle T f, g \rangle = \langle f, h \rangle \quad \text{for all } f \in \mathcal{D}(T), \quad \text{in which case } T^* g := h.
\]

The adjoint \( T^* \) is always closed. Hence, every self-adjoint operator is closed and densely defined.

\medskip
\noindent\textbf{Bounded Case.}
If \( T \in \mathcal{B}(H) \) is bounded and everywhere defined, then \( T \) is self-adjoint if and only if it is symmetric:
\[
\langle T f, g \rangle = \langle f, T g \rangle \quad \text{for all } f, g \in H.
\]

\medskip
\noindent\textbf{Symmetric Operators.}
A densely defined operator \( T \colon \mathcal{D}(T) \to H \) is \emph{symmetric} if
\[
\langle T f, g \rangle = \langle f, T g \rangle \quad \text{for all } f, g \in \mathcal{D}(T),
\]
i.e., \( T \subseteq T^* \). Such an operator is self-adjoint precisely when equality holds: \( \mathcal{D}(T) = \mathcal{D}(T^*) \) and \( T = T^* \).

\medskip
\noindent\textbf{Graph Characterization.}
Let
\[
\operatorname{graph}(T) := \{ (f, Tf) \in H \times H : f \in \mathcal{D}(T) \}.
\]
Then \( T \) is self-adjoint if and only if \( \operatorname{graph}(T) \) is closed and equals \( \operatorname{graph}(T^*) \). In particular, self-adjoint operators are maximal among symmetric ones.

\medskip
\noindent\textbf{Spectral Theorem.}
Every self-adjoint operator \( T \colon \mathcal{D}(T) \to H \) admits a unique spectral resolution:
\[
T = \int_{\sigma(T)} \lambda \, dE_\lambda,
\]
where \( E_\lambda \) is a projection-valued measure (PVM) on the Borel \( \sigma \)-algebra of \( \mathbb{R} \), supported on the spectrum \( \sigma(T) \subset \mathbb{R} \). In particular:
\begin{itemize}
    \item \( \sigma(T) \subset \mathbb{R} \) is closed and nonempty;
    \item For every bounded Borel function \( f \colon \mathbb{R} \to \mathbb{C} \), the spectral calculus
    \[
    f(T) := \int_{\sigma(T)} f(\lambda)\, dE_\lambda
    \]
    defines a bounded operator \( f(T) \in \mathcal{B}(H) \);
    \item The one-parameter unitary group \( \{ e^{itT} \}_{t \in \mathbb{R}} \subset \mathcal{U}(H) \) is strongly continuous.
\end{itemize}

\medskip
\noindent\textbf{Compact Self-Adjoint Operators.}
If \( T \in \mathcal{C}_1(H) \cap \operatorname{SA}(H) \), then \( \sigma(T) \subset \mathbb{R} \) consists entirely of isolated eigenvalues of finite multiplicity, with \( \lambda_n \to 0 \). The corresponding eigenfunctions form a complete orthonormal basis of \( H \).

\medskip
\noindent\textbf{Remarks.}
\begin{itemize}
    \item Self-adjointness implies a full spectral theory and guarantees the reality of the spectrum.
    \item A symmetric operator \( T_0 \colon \mathcal{D}_0 \to H \) is \emph{essentially self-adjoint} if its closure \( \overline{T_0} \) is self-adjoint. This property ensures unique spectral evolution.
    \item In this manuscript, the convolution operators \( L_t \), and their limit \( L_{\sym} \), are essentially self-adjoint on \( \mathcal{S}(\mathbb{R}) \subset H_{\Psi_\alpha} \); see \secref{sec:operator_construction}.
    \item For integral operators with Hermitian kernels and exponential damping, essential self-adjointness on a core follows from Friedrichs' extension theorem.
\end{itemize}

\medskip
\noindent\textbf{References.}
\begin{itemize}
    \item M.~Reed and B.~Simon, \emph{Methods of Modern Mathematical Physics I: Functional Analysis}, Chapter~VI \cite{ReedSimon1980I}.
    \item B.~Simon, \emph{Trace Ideals and Their Applications}, Chapter~3 \cite{Simon2005TraceIdeals}.
\end{itemize}
\end{definition}


% --- Weighted spaces and decay structure ---
\begin{definition}[Weighted Schwartz Space]\label{def:weighted_schwartz_space}
Let \( w \colon \R \to (0, \infty) \) be a smooth, strictly positive weight function satisfying:
\begin{enumerate}
    \item[\textup{(i)}] \( w(x) \ge 1 \) for all \( x \in \R \);
    \item[\textup{(ii)}] \( w(x) \to \infty \) as \( |x| \to \infty \);
    \item[\textup{(iii)}] \( w(x) \ge e^{\alpha |x|} \) for some \( \alpha > 0 \).
\end{enumerate}

The \emph{weighted Schwartz space} \( \mathcal{S}_w(\R) \) is the Fréchet space of all functions \( f \in C^\infty(\R) \) such that
\[
\| f \|^{(w)}_{k,\ell} := \sup_{x \in \R} \left| x^k f^{(\ell)}(x) \right| w(x)^{-1} < \infty \quad \text{for all } k,\ell \in \N_0.
\]
Each seminorm controls weighted decay of derivatives, so functions in \( \mathcal{S}_w(\R) \) decay faster than any polynomial, modulated by exponential weight.

\medskip
\noindent\textbf{Topological Structure.}
\begin{itemize}
    \item One has continuous dense inclusions: \( \Schwartz \subset \mathcal{S}_w(\R) \subset L^2(\R, w(x)^{-2} dx) \).

    \item For exponential weights \( w(x) = e^{\alpha |x|} \), define the corresponding weight potential \( \Psi_\alpha(x) := e^{2\alpha |x|} \), and the associated Hilbert space
    \[
    \HPsi := L^2(\R, \Psi_\alpha(x)\, dx).
    \]
    Then \( \mathcal{S}_w(\R) \subset \HPsi \), with dense embedding for all \( \alpha > 0 \).
\end{itemize}

\medskip
\noindent\textbf{Paley--Wiener Profile and Kernel Decay.}
Let \( \phi_t(\lambda) := e^{-t\lambda^2} \, \Xi\left( \tfrac{1}{2} + i\lambda \right) \) be the mollified spectral profile. Since \( \Xi(s) \) is entire of exponential type \( \pi \), the Paley--Wiener theorem yields
\[
|k_t(x)| := \left| \FT^{-1}[\phi_t](x) \right| \le C_\epsilon \, e^{-(\pi - \epsilon)|x|}, \quad \text{for all } \epsilon > 0.
\]
Thus, for convolution kernels \( K_t(x,y) := k_t(x - y) \) and any \( \alpha > \pi \), we have
\[
K_t \in L^1(\R^2, e^{\alpha(|x| + |y|)} dx\,dy),
\]
and the associated operator
\[
L_t f(x) := \int_{\R} k_t(x - y)\, f(y)\, dy
\]
acts continuously on \( \mathcal{S}_w(\R) \) and preserves the space.

\medskip
\noindent\textbf{Spectral and Functional Role.}
The space \( \mathcal{S}_w(\R) \) serves as a common dense core for mollified convolution operators \( L_t \) and their trace-norm limit \( L_{\sym} \). Its properties include:
\begin{itemize}
    \item Stability under convolution: \( f \in \mathcal{S}_w \Rightarrow k_t * f \in \mathcal{S}_w \), for all \( t > 0 \).
    \item Density: \( \mathcal{S}_w(\R) \) is dense in \( \HPsi \) for all \( \alpha > 0 \), and serves as a domain core in the essential self-adjointness proofs (cf.~\secref{sec:operator_construction}).
    \item Simon’s criterion for trace class applies: if \( K_t \in L^1(\R^2, \Psi_\alpha(x)\Psi_\alpha(y) dx\,dy) \), then
    \[
    L_t \in \mathcal{C}_1(\HPsi) \quad \text{\cite[Thm.~4.2]{Simon2005TraceIdeals}}.
    \]
    \item Moreover, the trace-norm convergence
    \[
    \det\nolimits_\zeta(I - \lambda L_t) \to \det\nolimits_\zeta(I - \lambda L_{\sym})
    \]
    holds uniformly on compact subsets \( \lambda \in \C \) as \( t \to 0^+ \), by norm continuity of the Fredholm determinant.
\end{itemize}

\medskip
\noindent\textbf{References.}
\begin{itemize}
    \item B.~Simon, \emph{Trace Ideals and Their Applications}, Theorem~4.2 \cite{Simon2005TraceIdeals}.
    \item B.~Ya.~Levin, \emph{Lectures on Entire Functions}, Chapter~9 \cite{Levin1996EntireLectures}.
\end{itemize}
\end{definition}

\begin{definition}[Exponential Weight and Weighted Hilbert Space]\label{def:exponential_weight}
Fix \( \alpha > \pi \), and define the exponential weight
\[
\Psi_\alpha(x) := e^{\alpha |x|}, \qquad x \in \mathbb{R}.
\]
Then \( \Psi_\alpha \in C^\infty(\mathbb{R}) \) is strictly positive, even, convex, and satisfies:
\begin{itemize}
    \item Super-exponential growth: \( \Psi_\alpha(x) \to \infty \) as \( |x| \to \infty \);
    \item Rapid decay: \( \Psi_\alpha^{-1}(x) = e^{-\alpha |x|} \in L^1(\mathbb{R}) \) for all \( \alpha > 0 \).
\end{itemize}

Define the weighted Hilbert space
\[
H_{\Psi_\alpha} := L^2(\mathbb{R}, \Psi_\alpha(x)\, dx) = \left\{ f \in L^2_{\mathrm{loc}}(\mathbb{R}) \;\middle|\; \int_{\mathbb{R}} |f(x)|^2 e^{\alpha |x|}\, dx < \infty \right\}.
\]

\medskip
\noindent\textbf{Paley--Wiener Control.}
Let \( F \colon \mathbb{C} \to \mathbb{C} \) be entire of exponential type \( \tau < \alpha \). Then by the Paley--Wiener theorem \cite[Thm.~3.2.4]{Levin1996EntireLectures}, the inverse Fourier transform satisfies
\[
\mathcal{F}^{-1}[F](x) \in L^1(\mathbb{R}, \Psi_\alpha^{-1}(x)\, dx),
\]
i.e., it decays faster than \( e^{-\alpha |x|} \). This provides precise decay estimates for convolution kernels constructed from entire spectral data.

\medskip
\noindent\textbf{Application to \(\Xi\) and Canonical Kernels.}
Let
\[
\phi(\lambda) := \Xi\left( \tfrac{1}{2} + i\lambda \right), \quad
k := \mathcal{F}^{-1}[\phi], \quad
K(x,y) := k(x - y).
\]
Since \( \Xi(s) \) is entire of exponential type \( \pi \), it follows that for all \( \alpha > \pi \),
\[
K \in L^1(\mathbb{R}^2, \Psi_\alpha^{-1}(x)\Psi_\alpha^{-1}(y)\, dx\,dy),
\]
and the convolution operator
\[
(Lf)(x) := \int_{\mathbb{R}} k(x - y) f(y)\, dy
\]
lies in \( \mathcal{C}_1(H_{\Psi_\alpha}) \) by Simon's trace-class criterion \cite[Thm.~4.2]{Simon2005TraceIdeals}.

\medskip
\noindent\textbf{Mollified Heat Kernels and Trace-Norm Limit.}
Define the mollified spectral profile:
\[
\phi_t(\lambda) := e^{-t\lambda^2} \phi(\lambda), \quad
k_t := \mathcal{F}^{-1}[\phi_t], \quad
K_t(x,y) := k_t(x - y).
\]
Then for all \( t > 0 \), we have \( k_t \in \mathcal{S}(\mathbb{R}) \subset L^1(\mathbb{R}, \Psi_\alpha^{-1}(x)\, dx) \), and the associated convolution operator
\[
L_t f(x) := \int_{\mathbb{R}} k_t(x - y) f(y)\, dy
\]
lies in \( \mathcal{C}_1(H_{\Psi_\alpha}) \). Moreover, there exists a canonical trace-norm limit:
\[
L_{\mathrm{sym}} := \lim_{t \to 0^+} L_t \in \mathcal{C}_1(H_{\Psi_\alpha}).
\]

\medskip
\noindent\textbf{Sharpness of \( \alpha > \pi \).}
The condition \( \alpha > \pi \) is sharp: the Paley--Wiener bound for \( k \) yields \( |k(x)| \approx e^{-\pi|x|} \) as \( |x| \to \infty \), so for \( \alpha \le \pi \), the weighted norm
\[
\iint_{\mathbb{R}^2} |K(x,y)| \, \Psi_\alpha(x)\Psi_\alpha(y)\, dx\,dy = \infty.
\]
Thus, \( L \notin \mathcal{C}_1(H_{\Psi_\alpha}) \) unless \( \alpha > \pi \).

\medskip
\noindent\textbf{Spectral and Analytic Consequences.}
The Hilbert space \( H_{\Psi_\alpha} \), with \( \alpha > \pi \), provides the analytic framework for the determinant and trace theory:
\begin{itemize}
    \item Heat trace finiteness: \( \operatorname{Tr}(e^{-tL^2}) < \infty \) for all \( t > 0 \), enabling short-time expansion;
    \item Spectral zeta function: \( \zeta_L(s) = \sum \lambda_n^{-s} \) admits analytic continuation via Tauberian theory \cite{Korevaar2004Tauberian};
    \item Determinant identity: \( \det\nolimits_\zeta(I - \lambda L) \) is entire of order one and recovers \( \Xi(\tfrac{1}{2} + i\lambda) \) up to normalization.
\end{itemize}

\medskip
\noindent\textbf{References.}
\begin{itemize}
    \item B.~Ya.~Levin, \emph{Lectures on Entire Functions}, Theorem~3.2.4 \cite{Levin1996EntireLectures}.
    \item B.~Simon, \emph{Trace Ideals and Their Applications}, Theorem~4.2 \cite{Simon2005TraceIdeals}.
    \item J.~Korevaar, \emph{Tauberian Theory}, Chapter~III \cite{Korevaar2004Tauberian}.
\end{itemize}
\end{definition}

\begin{definition}[Weighted Trace-Norm Space]\label{def:weighted-trace-norm-space}
Fix any \( \alpha > \pi \), and define the exponential weight
\[
\Psi_\alpha(x) := e^{\alpha |x|}, \qquad x \in \mathbb{R}.
\]
Let \( K \colon \mathbb{R}^2 \to \mathbb{C} \) be a measurable kernel. The \emph{weighted trace norm} is defined by
\[
\| K \|_{\mathcal{C}_1(\Psi_\alpha)} := \iint_{\mathbb{R}^2} |K(x,y)| \, \Psi_\alpha(x) \Psi_\alpha(y)\, dx\,dy.
\]
This defines the weighted trace-class kernel space
\[
\mathcal{C}_1(\Psi_\alpha) := \left\{ K \in L^1_{\mathrm{loc}}(\mathbb{R}^2) \ \middle|\ \| K \|_{\mathcal{C}_1(\Psi_\alpha)} < \infty \right\}.
\]

\medskip
\noindent\textbf{Trace-Class Realization.}
If \( K \in \mathcal{C}_1(\Psi_\alpha) \), then the integral operator
\[
(T_K f)(x) := \int_{\mathbb{R}} K(x,y)\, f(y)\, dy
\]
defines a bounded operator on the weighted Hilbert space \( H_{\Psi_\alpha} := L^2(\mathbb{R}, \Psi_\alpha(x)\, dx) \), and satisfies
\[
T_K \in \mathcal{C}_1(H_{\Psi_\alpha}).
\]

\medskip
\noindent\textbf{Trace-Norm Estimate.}
By Simon’s trace-class kernel criterion \cite[Thm.~4.2]{Simon2005TraceIdeals}, one has
\[
\|T_K\|_{\mathcal{C}_1(H_{\Psi_\alpha})} \le \|K\|_{\mathcal{C}_1(\Psi_\alpha)},
\]
so weighted kernel integrability controls trace-class membership in the Schatten \(\mathcal{C}_1\) ideal.

\medskip
\noindent\textbf{Convolution Kernel Case.}
Suppose \( K_t(x,y) = k_t(x - y) \) is a translation-invariant kernel with \( k_t \in L^1(\mathbb{R}, \Psi_\alpha^{-1}(x)\, dx) \). Then
\[
\| K_t \|_{\mathcal{C}_1(\Psi_\alpha)} = \left( \int_{\mathbb{R}} |k_t(z)| \Psi_\alpha(z)\, dz \right) \cdot \left( \int_{\mathbb{R}} \Psi_\alpha(x)\, dx \right) < \infty,
\]
so \( K_t \in \mathcal{C}_1(\Psi_\alpha) \), and the associated convolution operator
\[
L_t f(x) := \int_{\mathbb{R}} k_t(x - y)\, f(y)\, dy
\]
satisfies \( L_t \in \mathcal{C}_1(H_{\Psi_\alpha}) \) for all \( t > 0 \).

\medskip
\noindent\textbf{Spectral Role in Canonical Construction.}
The weighted kernel space \( \mathcal{C}_1(\Psi_\alpha) \) enables explicit and uniform trace-norm control of the mollified operator family \( \{L_t\}_{t > 0} \), with:
\[
L_t \xrightarrow{\ \mathcal{C}_1(H_{\Psi_\alpha})\ } L_{\mathrm{sym}} \in \mathcal{C}_1(H_{\Psi_\alpha}) \quad \text{as } t \to 0^+.
\]
This Schatten-class convergence ensures determinant convergence:
\[
\det\nolimits_\zeta(I - \lambda L_t) \to \det\nolimits_\zeta(I - \lambda L_{\mathrm{sym}})
\]
uniformly on compact subsets \( \lambda \in \mathbb{C} \). The weighted norm \( \|K\|_{\mathcal{C}_1(\Psi_\alpha)} \) thus offers a concrete test for trace-class inclusion, bypassing the need for diagonalization or kernel decomposition.

\medskip
\noindent\textbf{References.}
\begin{itemize}
    \item B. Simon, \emph{Trace Ideals and Their Applications}, Theorem~4.2 \cite{Simon2005TraceIdeals}.
\end{itemize}
\end{definition}

\begin{definition}[Paley--Wiener Class \(\operatorname{PW}_a(\mathbb{R})\)]
\label{def:paley_wiener_class}
Let \( a > 0 \). The Paley--Wiener class \( \operatorname{PW}_a(\mathbb{R}) \) consists of all entire functions \( f \colon \mathbb{C} \to \mathbb{C} \) such that:
\begin{itemize}
  \item \( f \) is of exponential type \( \le a \), i.e., there exists \( C > 0 \) such that
  \[
  |f(\lambda)| \le C e^{a |\lambda|}, \quad \forall \lambda \in \mathbb{C};
  \]
  \item The restriction \( f|_{\mathbb{R}} \in L^2(\mathbb{R}) \), and its Fourier transform \( \widehat{f} \) is supported in the interval \( [-a, a] \).
\end{itemize}

\medskip
\noindent
Equivalently, \( \operatorname{PW}_a(\mathbb{R}) \) is the inverse Fourier image of the compactly supported square-integrable functions:
\[
\operatorname{PW}_a(\mathbb{R}) = \mathcal{F}^{-1}(L^2([-a, a])).
\]

\medskip
\noindent\textbf{Remarks.}
\begin{itemize}
  \item Functions in \( \operatorname{PW}_a(\mathbb{R}) \) extend analytically to entire functions on \( \mathbb{C} \), with exponential type bounded by \( a \).
  \item The space \( \operatorname{PW}_a(\mathbb{R}) \) is a closed subspace of \( L^2(\mathbb{R}) \), and plays a central role in the theory of Fourier-analytic bandlimiting.
\end{itemize}

\begin{references}
  \item R.~E.~A.~C.~Paley and N.~Wiener,\ \emph{Fourier Transforms in the Complex Domain}\cite{PaleyWiener1934Fourier}.
  \item B.~Ya.~Levin,\ \emph{Lectures on Entire Functions}, Chapter~3\cite{Levin1996EntireLectures}.
\end{references}
\end{definition}


%------------------------------------------------------------------
\subsection{Analytic Lemmas}

% --- Duality estimates and kernel integrability ---
\begin{lemma}[Weighted Trace-Norm Duality for Convolution Kernels]
\label{lem:weighted_trace_norm_duality}
Let \( \alpha > \pi \), and define the exponential weight \( \Psi_\alpha(x) := e^{\alpha |x|} \). Let \( k \in L^1(\mathbb{R}, \Psi_\alpha(x)\, dx) \) be a real-valued, even function, and define the translation-invariant kernel
\[
K(x,y) := k(x - y).
\]

Then the following hold:
\begin{enumerate}
    \item[\textup{(i)}] The kernel \( K \in L^1(\mathbb{R}^2, \Psi_\alpha(x)\Psi_\alpha(y)\, dx\,dy) \), with
    \[
    \iint_{\mathbb{R}^2} |K(x,y)|\, \Psi_\alpha(x)\Psi_\alpha(y)\, dx\,dy
    = \|k\|_{L^1(\mathbb{R}, \Psi_\alpha)} \cdot \|\Psi_\alpha\|_{L^1(\mathbb{R})}.
    \]
    In particular, the weighted kernel norm factorizes as a product of one-dimensional integrals.

    \item[\textup{(ii)}] The associated convolution operator
    \[
    (L f)(x) := \int_{\mathbb{R}} k(x - y) f(y)\, dy
    \]
    defines a bounded trace-class operator \( L \in \mathcal{C}_1(H_{\Psi_\alpha}) \), with
    \[
    \| L \|_{\mathcal{C}_1(H_{\Psi_\alpha})}
    \le \|k\|_{L^1(\mathbb{R}, \Psi_\alpha)} \cdot \|\Psi_\alpha\|_{L^1(\mathbb{R})}.
    \]
    The inequality becomes an equality if \( k \ge 0 \).
\end{enumerate}

\medskip
\noindent
This duality underpins explicit trace-norm bounds for mollified convolution operators \( L_t \), and confirms membership in \( \mathcal{C}_1(H_{\Psi_\alpha}) \) whenever \( k_t \in L^1(\mathbb{R}, \Psi_\alpha) \).
\end{lemma}

\begin{proof}[Proof of \lemref{lem:weighted_trace_norm_duality}]
Fix \( \alpha > \pi \), and define \( \Psi_\alpha(x) := e^{\alpha |x|} \). Let \( k \in L^1(\mathbb{R}, \Psi_\alpha(x)\, dx) \) be real-valued and even, and define the convolution kernel
\[
K(x,y) := k(x - y),
\]
with associated operator
\[
(L f)(x) := \int_{\mathbb{R}} k(x - y)\, f(y)\, dy.
\]

\medskip
\noindent\textbf{(i) Weighted Kernel Norm Factorization.}
We compute:
\begin{align*}
\iint_{\mathbb{R}^2} |K(x,y)|\, \Psi_\alpha(x)\Psi_\alpha(y)\, dx\, dy
&= \iint_{\mathbb{R}^2} |k(x - y)|\, \Psi_\alpha(x)\Psi_\alpha(y)\, dx\, dy.
\end{align*}
Make the change of variables \( u := x - y \), \( v := y \), so that \( x = u + v \) and \( dx\,dy = du\,dv \). Then:
\[
= \int_{\mathbb{R}} |k(u)| \left( \int_{\mathbb{R}} \Psi_\alpha(u + v)\Psi_\alpha(v)\, dv \right) du.
\]
Using symmetry and convexity of \( \Psi_\alpha \), we obtain:
\[
\Psi_\alpha(u + v)\Psi_\alpha(v) = e^{\alpha(|u + v| + |v|)} = e^{\alpha|u|} \cdot e^{2\alpha|v|}.
\]
Hence,
\[
\int_{\mathbb{R}} \Psi_\alpha(u + v)\Psi_\alpha(v)\, dv = \Psi_\alpha(u) \cdot \int_{\mathbb{R}} e^{2\alpha |v|}\, dv = \Psi_\alpha(u) \cdot \|\Psi_\alpha\|_{L^1(\mathbb{R})}.
\]
Therefore,
\[
\iint_{\mathbb{R}^2} |K(x,y)|\, \Psi_\alpha(x)\Psi_\alpha(y)\, dx\, dy
= \|k\|_{L^1(\mathbb{R}, \Psi_\alpha)} \cdot \|\Psi_\alpha\|_{L^1(\mathbb{R})}.
\]

\medskip
\noindent\textbf{(ii) Trace-Class Bound.}
Since \( K \in \mathcal{C}_1(\Psi_\alpha) \), the associated integral operator \( L \) lies in \( \mathcal{C}_1(H_{\Psi_\alpha}) \) by Simon’s kernel criterion \cite[Thm.~4.2]{Simon2005TraceIdeals}. Moreover,
\[
\|L\|_{\mathcal{C}_1(H_{\Psi_\alpha})} \le \|K\|_{\mathcal{C}_1(\Psi_\alpha)}
= \|k\|_{L^1(\mathbb{R}, \Psi_\alpha)} \cdot \|\Psi_\alpha\|_{L^1(\mathbb{R})}.
\]

\medskip
\noindent\textbf{Conclusion.}
This completes the proof of both statements, establishing an explicit factorized relationship between 1D weighted kernel integrability and 2D trace-norm control in \( H_{\Psi_\alpha} \).
\end{proof}


\begin{lemma}[\( L^1 \)-Integrability of Conjugated Kernels under Exponential Weights]
\label{lem:L1_integrability_conjugated_kernel}
Let \( K \colon \mathbb{R}^2 \to \mathbb{C} \) be a measurable kernel satisfying the decay estimate
\[
|K(x,y)| \le C (1 + |x| + |y|)^{-N},
\]
for some constants \( C > 0 \), \( N > 0 \). Let \( \Psi_\alpha(x) := e^{\alpha |x|} \) be the exponential weight with fixed \( \alpha > 0 \). Define the conjugated kernel
\[
\widetilde{K}(x,y) := \frac{K(x,y)}{\sqrt{\Psi_\alpha(x)\Psi_\alpha(y)}}.
\]

Then \( \widetilde{K} \in L^1(\mathbb{R}^2) \) provided \( N > 2\alpha \); that is,
\[
\iint_{\mathbb{R}^2} |\widetilde{K}(x,y)|\, dx\,dy < \infty.
\]

\medskip
\noindent
In particular, if \( K(x,y) = k(x - y) \) is a translation-invariant kernel with \( k \in \mathcal{S}(\mathbb{R}) \), then \( K \) satisfies the above estimate for all \( N > 0 \), and hence \( \widetilde{K} \in L^1(\mathbb{R}^2) \) for any \( \alpha > 0 \).

\medskip
\noindent
This lemma applies to mollified canonical kernels \( K_t(x,y) := k_t(x - y) \) as in \lemref{lem:decay_mollified_kernel}, where exponential decay of \( k_t \) follows from the Paley–Wiener growth of \( \phi(\lambda) = \Xi(\tfrac{1}{2} + i\lambda) \) (see \lemref{lem:xi_growth_bound}). Consequently, the conjugated kernel
\[
\widetilde{K}_t(x,y) := \frac{K_t(x,y)}{\sqrt{\Psi_\alpha(x)\Psi_\alpha(y)}}
\]
lies in \( L^1(\mathbb{R}^2) \), and the operator \( L_t \in \mathcal{C}_1(H_{\Psi_\alpha}) \), as confirmed in \lemref{lem:trace_class_Lt}.
\end{lemma}

\begin{proof}[Proof of \lemref{lem:L1_integrability_conjugated_kernel}]
Assume the kernel \( K(x,y) \) satisfies the decay estimate
\[
|K(x,y)| \le C (1 + |x| + |y|)^{-N},
\]
for some constant \( C > 0 \). Let \( \Psi(x) \) satisfy the two-sided exponential bounds
\[
c_1 e^{a|x|} \le \Psi(x) \le c_2 e^{a|x|}, \quad \forall x \in \mathbb{R},
\]
with constants \( a > 0 \), \( c_1, c_2 > 0 \). Define the conjugated kernel
\[
\widetilde{K}(x,y) := \frac{K(x,y)}{\sqrt{\Psi(x)\Psi(y)}}.
\]

\medskip
\noindent\textbf{Step 1: Pointwise Estimate.}
Using the lower bound on \( \Psi \), we have
\[
\sqrt{\Psi(x)\Psi(y)} \ge c_1\, e^{a(|x| + |y|)/2},
\]
so
\[
|\widetilde{K}(x,y)| \le \frac{C}{c_1} (1 + |x| + |y|)^{-N} e^{-a(|x| + |y|)/2}.
\]

\medskip
\noindent\textbf{Step 2: Factorization via Submultiplicativity.}
Using the inequality
\[
1 + |x| + |y| \ge \frac{1}{2}(1 + |x|)(1 + |y|),
\]
we obtain
\[
(1 + |x| + |y|)^{-N} \le 2^N (1 + |x|)^{-N/2} (1 + |y|)^{-N/2}.
\]
Thus, for some constant \( C' > 0 \),
\[
|\widetilde{K}(x,y)| \le C' \cdot (1 + |x|)^{-N/2} e^{-a|x|/2} \cdot (1 + |y|)^{-N/2} e^{-a|y|/2}.
\]
Each factor lies in \( L^1(\mathbb{R}) \) provided \( N > 2a \). Therefore, Fubini’s theorem yields
\[
\iint_{\mathbb{R}^2} |\widetilde{K}(x,y)|\, dx\,dy < \infty.
\]

\medskip
\noindent\textbf{Step 3: Operator-Theoretic Interpretation.}
Let \( T \) be the integral operator on \( L^2(\mathbb{R}, \Psi(x)\,dx) \) with kernel \( K(x,y) \). Define the unitary map
\[
U \colon L^2(\mathbb{R}, \Psi(x)\,dx) \to L^2(\mathbb{R}), \quad (Uf)(x) := \Psi(x)^{1/2} f(x),
\]
and let \( \widetilde{T} := U T U^{-1} \) act on \( L^2(\mathbb{R}) \) with kernel \( \widetilde{K}(x,y) \in L^1(\mathbb{R}^2) \).

By Simon’s trace-class criterion for integral operators \cite[Thm.~4.2]{Simon2005TraceIdeals}, we conclude:
\[
\widetilde{T} \in \mathcal{C}_1(L^2(\mathbb{R})) \quad \Rightarrow \quad T \in \mathcal{C}_1(L^2(\mathbb{R}, \Psi(x)\,dx)).
\]
\end{proof}


% --- Mollifier decay and profile estimates ---
\begin{lemma}[Exponential Decay Estimates for Mollified Kernels]
\label{lem:decay_mollified_kernel}
Let \( t > 0 \), and define the mollified spectral profile
\[
\phi_t(\lambda) := e^{-t\lambda^2} \, \Xi\left(\tfrac{1}{2} + i\lambda\right),
\]
where \( \Xi(s) \) is the completed Riemann zeta function—an entire function of order one and exponential type \( \pi \).

Then the following exponential decay estimates hold:

\begin{enumerate}
    \item[\textup{(i)}] \textbf{Exponential Fourier Envelope:} There exists a constant \( C > 0 \), independent of \( t \), such that
    \[
    |\phi_t(\lambda)| \le C \, e^{\frac{\pi}{2}|\lambda| - t\lambda^2}, \qquad \forall\, \lambda \in \mathbb{R}.
    \]

    \item[\textup{(ii)}] \textbf{Exponential Spatial Kernel Decay:} Define the inverse Fourier kernel
    \[
    K_t(x,y) := \frac{1}{2\pi} \int_{\mathbb{R}} e^{i\lambda(x - y)} \phi_t(\lambda)\,d\lambda.
    \]
    Then for each \( t > 0 \), there exist constants \( C_t > 0 \), \( b_t > 0 \) such that
    \[
    |K_t(x,y)| \le C_t\, e^{-b_t |x - y|}, \qquad \forall\, x, y \in \mathbb{R}.
    \]

    \item[\textup{(iii)}] \textbf{Uniform Bounds for Small \( t \):} There exist constants \( C_0 > 0 \), \( b_0 > 0 \), and \( t_0 > 0 \), such that for all \( t \in (0, t_0] \),
    \[
    |K_t(x,y)| \le C_0\, e^{-b_0 |x - y|}, \qquad \forall\, x, y \in \mathbb{R}.
    \]
\end{enumerate}

\noindent
As a consequence, \( K_t \in L^1(\mathbb{R}^2, \Psi_\alpha(x)\Psi_\alpha(y)\, dx\,dy) \) for any \( \alpha > \pi \), and the associated convolution operator
\[
L_t f(x) := \int_{\mathbb{R}} K_t(x,y) f(y)\, dy
\]
belongs to \( \mathcal{C}_1(H_{\Psi_\alpha}) \) by Simon's exponential trace-class criterion \cite[Thm.~4.2]{Simon2005TraceIdeals}. The uniform bounds in \textup{(iii)} ensure trace-norm convergence
\[
L_t \xrightarrow{\ \mathcal{C}_1(H_{\Psi_\alpha})\ } L_{\mathrm{sym}},
\]
and determinant convergence
\[
\det\nolimits_\zeta(I - \lambda L_t) \to \det\nolimits_\zeta(I - \lambda L_{\mathrm{sym}})
\]
uniformly on compact subsets of \( \lambda \in \mathbb{C} \).
\end{lemma}

\begin{proof}[Proof of \lemref{lem:decay_mollified_kernel}]
Define the mollified spectral profile
\[
\phi_t(\lambda) := e^{-t\lambda^2} \, \Xi\left(\tfrac{1}{2} + i\lambda\right),
\]
and let \( k_t(x) := \FT^{-1}[\phi_t](x) \), so that the convolution kernel is
\[
K_t(x,y) := k_t(x - y) = \frac{1}{2\pi} \int_{\R} e^{i\lambda(x - y)} \phi_t(\lambda)\, d\lambda.
\]

\paragraph{(i) Fourier Envelope Decay.}
The function \( \Xi(s) \) is entire of exponential type \( \pi \), and satisfies (see~\cite[Thm.~3.7.1]{Levin1996EntireLectures}, \cite[§4.12]{Titchmarsh1986Zeta}):
\[
\left| \Xi\left( \tfrac{1}{2} + i\lambda \right) \right| \le C_0\, e^{\frac{\pi}{2} |\lambda|}, \quad \forall \lambda \in \R.
\]
Therefore,
\[
|\phi_t(\lambda)| \le C_0\, e^{-t\lambda^2 + \frac{\pi}{2}|\lambda|}.
\]
Completing the square gives:
\[
-t\lambda^2 + \frac{\pi}{2}|\lambda| \le -\tfrac{t}{2} \lambda^2 + \frac{\pi^2}{8t},
\]
so
\[
|\phi_t(\lambda)| \le C_t\, e^{-a_t \lambda^2}, \quad \text{with } a_t := \tfrac{t}{2}, \quad C_t := C_0 e^{\pi^2/8t}.
\]

\paragraph{(ii) Spatial Kernel Decay via Paley--Wiener.}
Since \( \phi_t \in \Schwartz \) and has exponential type \( \pi \), the Paley--Wiener theorem (see~\cite[Ch.~IX.4]{ReedSimon1975II}) implies:
\[
|k_t(x)| \le C_t' e^{-(\pi - \epsilon)|x|}, \quad \forall \epsilon > 0,
\]
for some constant \( C_t' > 0 \). Thus,
\[
|K_t(x,y)| = |k_t(x - y)| \le C_t'\, e^{-b_t |x - y|}, \quad \text{with } b_t := \pi - \epsilon.
\]

\paragraph{(iii) Uniformity for Small \( t \).}
Since the exponential type of \( \phi_t \) is independent of \( t \), and the Gaussian factor improves decay, the family \( \{k_t\}_{t \in (0, t_0]} \) admits uniform exponential envelope bounds. Therefore, there exist constants \( C_0, b_0 > 0 \) and \( t_0 > 0 \) such that
\[
|K_t(x,y)| \le C_0\, e^{-b_0 |x - y|}, \quad \forall x, y \in \R, \quad \forall t \in (0, t_0].
\]

\paragraph{Conclusion.}
From the above:
\begin{itemize}
    \item[\textup{(i)}] \( \phi_t(\lambda) \in \Schwartz \), with decay controlled by both Gaussian and exponential envelope;
    \item[\textup{(ii)}] \( K_t(x,y) = k_t(x - y) \in L^1(\R^2, \Psi_\alpha(x)\Psi_\alpha(y)\, dx\,dy) \) for all \( \alpha > \pi \);
    \item[\textup{(iii)}] The trace-class bound follows from Simon's kernel\\ criterion~\cite[Thm.~4.2]{Simon2005TraceIdeals}:
    \[
    L_t f(x) := \int_{\R} K_t(x,y)\, f(y)\, dy \in \TC(\HPsi),
    \]
    with trace-norm uniformly bounded for \( t \in (0, t_0] \), ensuring convergence \( L_t \to \Lsym \) in \( \TC(\HPsi) \) and determinant convergence as \( t \to 0^+ \).
\end{itemize}
\end{proof}


\begin{lemma}[Exact Growth Bound for \( \Xi \)]
\label{lem:xi_growth_bound}
Let \( \Xi(s) \) denote the completed Riemann zeta function,
\[
\Xi(s) := \tfrac{1}{2} s(s-1) \pi^{-s/2} \Gamma\left(\tfrac{s}{2}\right) \zeta(s),
\]
which extends to an entire function of order one and exponential type \( \pi \), with Hadamard product over its nontrivial zeros.

Define the centered spectral profile
\[
\phi(\lambda) := \Xi\left( \tfrac{1}{2} + i\lambda \right).
\]

Then the following exponential growth bounds hold:
\begin{itemize}
    \item[\textup{(i)}] \textbf{Global Complex Growth:} There exists a constant \( A > 0 \) such that
    \[
    |\phi(\lambda)| \le A\, e^{\pi |\lambda|}, \qquad \forall\, \lambda \in \C.
    \]
    This reflects the exact exponential type \( \pi \) of \( \Xi(s) \), sharp in the Paley--Wiener sense \cite[Thm.~3.7.1]{Levin1996EntireLectures}.

    \item[\textup{(ii)}] \textbf{Real Axis Growth:} There exists a constant \( A_1 > 0 \) such that
    \[
    |\phi(\lambda)| \le A_1\, e^{\tfrac{\pi}{2} |\lambda|}, \qquad \forall\, \lambda \in \R.
    \]
    This sharper bound follows from the symmetry \( \Xi(s) = \Xi(1 - s) \) and classical estimates for \( \Gamma\left(\tfrac{s}{2}\right) \) on the critical line \cite[§4.12]{Titchmarsh1986Zeta}.
\end{itemize}

\medskip
\noindent
These bounds imply that for any \( t > 0 \), the mollified profiles
\[
\phi_t(\lambda) := e^{-t\lambda^2} \phi(\lambda)
\]
belong to the Schwartz space \( \Schwartz \), and that their inverse Fourier transforms \( k_t := \FT^{-1}[\phi_t] \) decay exponentially in space as in \lemref{lem:decay_mollified_kernel}.
\end{lemma}

\begin{proof}[Proof of \lemref{lem:xi_growth_bound}]
Let \( s := \tfrac{1}{2} + i\lambda \), and recall the representation
\[
\Xi(s) := \tfrac{1}{2}s(s-1)\pi^{-s/2} \Gamma\left( \tfrac{s}{2} \right) \zeta(s),
\]
which defines an entire function of order one and exponential type \( \pi \), satisfying the functional equation \( \Xi(s) = \Xi(1 - s) \).

\medskip
\noindent\textbf{Step 1: Gamma Term Estimate.}
Set \( z := \tfrac{s}{2} = \tfrac{1}{4} + \tfrac{i\lambda}{2} \). By Stirling’s bound for \( \Gamma(z) \) in vertical strips (see \cite[Eq.~(1.5.3)]{Titchmarsh1986Zeta}), there exists \( C_1 > 0 \) such that
\[
|\Gamma(z)| \le C_1 (1 + |\lambda|)^{-1/2} e^{\pi |\lambda| / 4}, \qquad \forall \lambda \in \R.
\]

\medskip
\noindent\textbf{Step 2: Remaining Factors.}
We estimate:
\begin{align*}
|s(s - 1)| &= \left| \left( \tfrac{1}{2} + i\lambda \right)\left( -\tfrac{1}{2} + i\lambda \right) \right| = \tfrac{1}{4} + \lambda^2, \\
|\pi^{-s/2}| &= \pi^{-\Re(s)/2} = \pi^{-1/4}, \\
|\zeta(s)| &\le C_2 \log(2 + |\lambda|), \qquad \text{for } \Re(s) = \tfrac{1}{2},
\end{align*}
for some constant \( C_2 > 0 \), using convexity bounds for \( \zeta(s) \) on the critical line.

\medskip
\noindent\textbf{Step 3: Real Axis Growth.}
Combining the above, we obtain
\[
|\Xi(s)| \le C_3\, (1 + \lambda^2) \cdot (1 + |\lambda|)^{-1/2} \cdot \log(2 + |\lambda|) \cdot e^{\pi |\lambda| / 4},
\]
for some \( C_3 > 0 \). All algebraic and logarithmic terms are subexponential, so we absorb them into a constant \( A_1 > 0 \) and write:
\[
|\phi(\lambda)| = \left| \Xi\left( \tfrac{1}{2} + i\lambda \right) \right| \le A_1\, e^{\tfrac{\pi}{2} |\lambda|},
\]
establishing part (ii) of the lemma.

\medskip
\noindent\textbf{Step 4: Global Complex Growth.}
Since \( \Xi(s) \) is entire of order one and exponential type \( \pi \), Hadamard factorization and Phragmén--Lindelöf bounds imply (see \cite[Ch.~3]{Levin1996EntireLectures}):
\[
|\Xi(s)| \le A\, e^{\pi |s|}, \qquad \forall\, s \in \C,
\]
for some constant \( A > 0 \). Setting \( s = \tfrac{1}{2} + i\lambda \), we obtain
\[
|\phi(\lambda)| = \left| \Xi\left( \tfrac{1}{2} + i\lambda \right) \right| \le A\, e^{\pi |\lambda|},
\]
completing part (i) of the lemma.

\medskip
\noindent\textbf{Conclusion.}
The centered profile \( \phi(\lambda) := \Xi\left(\tfrac{1}{2} + i\lambda\right) \) satisfies:
\[
|\phi(\lambda)| \le A_1\, e^{\tfrac{\pi}{2}|\lambda|} \quad \text{on } \R, \qquad |\phi(\lambda)| \le A\, e^{\pi|\lambda|} \quad \text{on } \C.
\]
Thus \( \phi \in PW_\pi(\R) \), and the mollified profiles \( \phi_t(\lambda) := e^{-t\lambda^2} \phi(\lambda) \) lie in \( \Schwartz \), with exponential spatial decay of their Fourier transforms \( k_t := \FT^{-1}[\phi_t] \), as needed in \lemref{lem:decay_mollified_kernel}.
\end{proof}


\begin{lemma}[Weighted \( L^1 \)-Integrability of the Inverse Fourier Transform of \( \Xi \)]
\label{lem:weighted_L1_inverse_FT_xi}
Let \( \alpha > \pi \), and define the centered spectral profile
\[
\phi(\lambda) := \Xi\left( \tfrac{1}{2} + i\lambda \right),
\]
where \( \Xi(s) \) is the completed Riemann zeta function—entire of exponential type \( \pi \) and order one (see \lemref{lem:xi_growth_bound}).

Define its inverse Fourier transform:
\[
\widehat{\Xi}(x) := \frac{1}{2\pi} \int_{\mathbb{R}} e^{i\lambda x} \, \phi(\lambda)\, d\lambda,
\]
interpreted in the distributional sense.

Then:
\[
\widehat{\Xi} \in L^1(\mathbb{R}, e^{-\alpha |x|}\, dx),
\]
i.e., there exists a constant \( A_\alpha > 0 \) such that
\[
\int_{\mathbb{R}} |\widehat{\Xi}(x)| \, e^{-\alpha |x|}\, dx \le A_\alpha.
\]

\medskip
\noindent
In particular, defining the exponential weight \( \Psi_\alpha(x) := e^{\alpha |x|} \), we have \( \widehat{\Xi} \in L^1(\mathbb{R}, \Psi_\alpha^{-1}(x)\, dx) \). Therefore, the convolution kernel
\[
K(x,y) := \widehat{\Xi}(x - y)
\]
belongs to \( L^1(\mathbb{R}^2, \Psi_\alpha(x)\Psi_\alpha(y)\, dx\,dy) \), and the associated convolution operator
\[
(L f)(x) := \int_{\mathbb{R}} \widehat{\Xi}(x - y) f(y)\, dy
\]
lies in the trace class \( \TC(H_{\Psi_\alpha}) \) by Simon’s kernel criterion~\cite[Thm.~4.2]{Simon2005TraceIdeals}.

\medskip
\noindent
This decay follows from the Paley--Wiener theorem: since \( \phi \in \PW{\pi} \) (see \defref{def:paley_wiener_class} and \lemref{lem:xi_growth_bound}), its inverse Fourier transform satisfies
\[
|\widehat{\Xi}(x)| = \mathcal{O}(e^{-\pi |x|}),
\]
and thus lies in \( L^1(\mathbb{R}, e^{-\alpha |x|}\, dx) \) for all \( \alpha > \pi \).

\medskip
\noindent
\textbf{Optional.} For explicit pointwise decay and differentiability of \( \widehat{\Xi}(x) \), see \lemref{lem:decay_inverse_fourier_xi}.
\end{lemma}

\begin{proof}[Proof of \lemref{lem:weighted_L1_inverse_FT_xi}]
Let \( \alpha > \pi \), and define
\[
\widehat{\Xi}(x) := \frac{1}{2\pi} \int_{\R} e^{i\lambda x} \, \Xi\left( \tfrac{1}{2} + i\lambda \right)\, d\lambda.
\]

\medskip
\noindent\textbf{Step 1: Spectral Profile and Exponential Type.}
Set
\[
\phi(\lambda) := \Xi\left( \tfrac{1}{2} + i\lambda \right).
\]
By \lemref{lem:xi_growth_bound}, this function is entire of order one and exponential type \( \pi \), i.e., \( \phi \in \PW{\pi} \), with
\[
|\phi(\lambda)| \le A_1\, e^{\pi |\lambda|}, \quad \forall \lambda \in \R,
\]
due to Hadamard factorization and asymptotics for \( \Gamma(s/2)\zeta(s) \) on vertical lines~\cite[Ch.~3]{Levin1996EntireLectures}, \cite[Ch.~2]{Titchmarsh1986Zeta}.

\medskip
\noindent\textbf{Step 2: Paley--Wiener Decay.}
By the Paley--Wiener theorem for exponential type \( \pi \)~\cite[Thm.~3.2.4]{Levin1996EntireLectures}, the inverse Fourier transform
\[
\widehat{\phi}(x) = \frac{1}{2\pi} \int_{\R} e^{i\lambda x} \phi(\lambda)\, d\lambda
\]
satisfies
\[
\widehat{\phi} \in L^1(\R, e^{-\beta |x|} dx), \quad \forall \beta > \pi.
\]
Hence for any fixed \( \alpha > \pi \),
\[
\widehat{\Xi}(x) = \widehat{\phi}(x) \in L^1(\R, e^{-\alpha |x|} dx).
\]

\medskip
\noindent\textbf{Step 3: Quantitative Bound.}
For any \( \varepsilon > 0 \), there exists \( C_\alpha > 0 \) such that
\[
|\widehat{\Xi}(x)| \le C_\alpha\, e^{-(\alpha - \varepsilon)|x|}, \quad \forall x \in \R.
\]
Therefore,
\[
\int_{\R} |\widehat{\Xi}(x)|\, e^{-\alpha |x|} dx
\le C_\alpha \int_{\R} e^{-(\alpha + \varepsilon)|x|} dx
= \frac{2 C_\alpha}{\alpha + \varepsilon} < \infty.
\]

\medskip
\noindent\textbf{Conclusion.}
Thus \( \widehat{\Xi} \in L^1(\R, \Psi_\alpha^{-1}(x)\, dx) \), where \( \Psi_\alpha(x) := e^{\alpha |x|} \). Define the convolution kernel
\[
K(x,y) := \widehat{\Xi}(x - y).
\]
Then \( K \in L^1(\R^2, \Psi_\alpha(x)\Psi_\alpha(y)\, dx\,dy) \), and the associated operator
\[
L f(x) := \int_{\R} \widehat{\Xi}(x - y)\, f(y)\, dy
\]
belongs to \( \TC(\HPsi) \) by Simon’s trace-class kernel criterion~\cite[Thm.~4.2]{Simon2005TraceIdeals}.
\end{proof}


\begin{lemma}[Exponential Decay of the Inverse Fourier Transform of \( \Xi \)]
\label{lem:decay_inverse_fourier_xi}
Let \( \phi(\lambda) := \Xi\left( \tfrac{1}{2} + i\lambda \right) \), where \( \Xi(s) \) is the completed Riemann zeta function. Define the inverse Fourier transform
\[
k(x) := \widehat{\phi}(x) := \frac{1}{2\pi} \int_{\R} e^{i\lambda x} \phi(\lambda)\, d\lambda.
\]

Then \( k \in C^\infty(\R) \) is real-valued, even, and satisfies the exponential decay estimate:
\[
|k(x)| \le C_\alpha\, e^{-\alpha |x|}, \qquad \forall\, x \in \R, \quad \text{for any } \alpha > \pi,
\]
where \( C_\alpha > 0 \) depends only on \( \alpha \).

\medskip
\noindent In particular:
\begin{itemize}
  \item \( k \in L^1(\R, \Psi_\alpha(x)\, dx) \), where \( \Psi_\alpha(x) := e^{\alpha |x|} \);
  \item The kernel \( K(x,y) := k(x - y) \) lies in \( L^1(\R^2, \Psi_\alpha(x)\Psi_\alpha(y)\, dx\, dy) \);
  \item The convolution operator
  \[
  (Lf)(x) := \int_{\R} k(x - y) f(y)\, dy
  \]
  belongs to the trace class \( \TC(\HPsi) \), by Simon’s kernel criterion \cite[Thm.~4.2]{Simon2005TraceIdeals}.
\end{itemize}

\medskip
\noindent
This decay follows from the Paley--Wiener theorem: since \( \phi \in \PW{\pi} \) (see \cref{def:paley_wiener_class}), its inverse Fourier transform \( k(x) \) decays faster than \( e^{-\alpha |x|} \) for every \( \alpha > \pi \). The result quantifies the optimal spatial localization of Paley--Wiener kernels in \( \HPsi \).
\end{lemma}

\begin{proof}[Proof of \lemref{lem:decay_inverse_fourier_xi}]
Let \( \phi(\lambda) := \Xi\left( \tfrac{1}{2} + i\lambda \right) \), where \( \Xi(s) \) is the completed Riemann zeta function. As established in \lemref{lem:xi_growth_bound}, \( \phi \in PW_\pi(\R) \) is entire of exponential type \( \pi \), real-valued, and even, with
\[
|\phi(\lambda)| \le C\, e^{\pi |\lambda|}, \qquad \forall \lambda \in \R.
\]

\medskip
\noindent\textbf{Step 1: Exponential Decay via Paley--Wiener.}
By the Paley--Wiener theorem for \( PW_\pi \) functions (see \cite[Thm.~3.2.4]{Levin1996EntireLectures}, \cite[Ch.~IX.4]{ReedSimon1975II}), the inverse Fourier transform
\[
k(x) := \FT^{-1}[\phi](x) = \frac{1}{2\pi} \int_{\R} e^{i\lambda x} \phi(\lambda)\, d\lambda
\]
lies in \( C^\infty(\R) \cap L^1(\R, e^{-\alpha |x|} dx) \) for all \( \alpha > \pi \), with
\[
|k(x)| \le C_\alpha\, e^{-\alpha |x|}, \qquad \forall x \in \R.
\]

\medskip
\noindent\textbf{Step 2: Symmetry and Regularity.}
Since \( \phi \) is real-valued and even, Fourier inversion implies \( k(x) \in \R \) and \( k(x) = k(-x) \). Moreover, \( k \in \Schwartz \subset C^\infty(\R) \), and all derivatives decay faster than any exponential \( e^{-\beta |x|} \) for \( \beta < \alpha \).

\medskip
\noindent\textbf{Step 3: Weighted Integrability.}
For any fixed \( \alpha > \pi \), define \( \Psi_\alpha(x) := e^{\alpha |x|} \). Then
\[
\int_{\R} |k(x)|\, \Psi_\alpha(x)\, dx = \int_{\R} |k(x)|\, e^{\alpha |x|} dx < \infty,
\]
so \( k \in L^1(\R, \Psi_\alpha(x)\, dx) \).

\medskip
\noindent\textbf{Step 4: Trace-Class Kernel Inclusion.}
Define the translation-invariant kernel \( K(x,y) := k(x - y) \). Then
\[
\iint_{\R^2} |K(x,y)|\, \Psi_\alpha(x)\Psi_\alpha(y)\, dx\, dy
= \int_{\R} |k(z)| \left( \int_{\R} \Psi_\alpha(z + y)\Psi_\alpha(y)\, dy \right) dz.
\]
Using the exponential decay of \( k \) and convexity of \( \Psi_\alpha \), the inner integral is uniformly bounded in \( z \) by \( C \Psi_\alpha(z) \). Thus, the full integral is bounded by
\[
C \int_{\R} |k(z)|\, \Psi_\alpha(z)\, dz < \infty.
\]
Hence \( K \in L^1(\R^2, \Psi_\alpha(x)\Psi_\alpha(y)\, dx\, dy) \), and the associated convolution operator
\[
(Lf)(x) := \int_{\R} k(x - y) f(y)\, dy
\]
belongs to \( \mathcal{C}_1(\HPsi) \) by Simon’s kernel criterion \cite[Thm.~4.2]{Simon2005TraceIdeals}.
\end{proof}


% --- Uniform kernel control ---
\begin{lemma}[Uniform \( L^1 \)-Bound for Exponentially Conjugated Heat Kernels]
\label{lem:uniform_L1_conjugated_kernel}
Let
\[
\phi_t(\lambda) := e^{-t\lambda^2} \, \Xi\left( \tfrac{1}{2} + i\lambda \right), \qquad t > 0,
\]
and define the mollified inverse Fourier kernel
\[
K_t(x,y) := \frac{1}{2\pi} \int_{\R} e^{i\lambda(x - y)} \phi_t(\lambda) \, d\lambda.
\]

Fix an exponential weight \( \Psi_\alpha(x) := e^{\alpha |x|} \) with \( \alpha > \pi \), and define the exponentially conjugated kernel
\[
\widetilde{K}_t(x,y) := K_t(x,y)\, \Psi_\alpha(x)\Psi_\alpha(y) = K_t(x,y)\, e^{\alpha(|x| + |y|)}.
\]

Then there exists a constant \( A_3(\alpha) > 0 \) such that
\[
\sup_{0 < t \le 1} \iint_{\R^2} |\widetilde{K}_t(x,y)| \, dx\,dy \le A_3(\alpha).
\]
Equivalently,
\[
\sup_{0 < t \le 1} \|K_t\|_{\mathcal{C}_1(\Psi_\alpha)} < \infty.
\]

In particular, for each \( t \in (0,1] \), the associated convolution operator
\[
L_t f(x) := \int_{\R} K_t(x,y) f(y)\, dy
\]
lies in the trace class \( \mathcal{C}_1(\HPsi) \), with uniformly bounded trace norm:
\[
\sup_{0 < t \le 1} \|L_t\|_{\mathcal{C}_1(\HPsi)} \le A_3(\alpha).
\]

\medskip
\noindent
This uniform trace-norm control ensures:
\begin{itemize}
    \item Convergence in trace norm: \( L_t \to L_{\sym} \in \mathcal{C}_1(\HPsi) \) as \( t \to 0^+ \);
    \item Uniform convergence of the Fredholm determinants:
    \[
    \det\nolimits_\zeta(I - \lambda L_t) \to \det\nolimits_\zeta(I - \lambda L_{\sym})
    \]
    locally uniformly in \( \lambda \in \C \).
\end{itemize}
\end{lemma}

\begin{proof}[Proof of \lemref{lem:uniform_L1_conjugated_kernel}]
Fix \( \alpha > \pi \), and define
\[
\phi_t(\lambda) := e^{-t\lambda^2} \, \Xi\left( \tfrac{1}{2} + i\lambda \right), \qquad t > 0,
\]
with associated inverse Fourier kernel
\[
K_t(x,y) := \frac{1}{2\pi} \int_{\R} e^{i\lambda(x - y)} \phi_t(\lambda)\, d\lambda.
\]
Let \( \Psi_\alpha(x) := e^{\alpha |x|} \), and define the conjugated kernel
\[
\widetilde{K}_t(x,y) := K_t(x,y)\, \Psi_\alpha(x)\Psi_\alpha(y) = K_t(x,y)\, e^{\alpha(|x| + |y|)}.
\]

\medskip
\noindent\textbf{Step 1: Exponential Decay of \( K_t \).}
By \lemref{lem:decay_mollified_kernel}, for all \( t \in (0,1] \), there exist constants \( C_t > 0 \), \( b_t > \pi \) such that
\[
|K_t(x,y)| \le C_t\, e^{-b_t |x - y|}, \qquad \forall\, x, y \in \R.
\]

\medskip
\noindent\textbf{Step 2: Estimate of Conjugated Kernel.}
We estimate:
\[
|\widetilde{K}_t(x,y)| \le C_t\, e^{-b_t |x - y|} \cdot e^{\alpha(|x| + |y|)}.
\]
Set \( u := x - y \), \( v := y \), so that \( x = u + v \), and \( dx\,dy = du\,dv \). Then:
\[
|\widetilde{K}_t(u + v, v)| \le C_t\, e^{-b_t |u|} \cdot e^{\alpha(|u + v| + |v|)}.
\]
By the triangle inequality: \( |u + v| + |v| \le |u| + 2|v| \). Hence,
\[
|\widetilde{K}_t(x,y)| \le C_t\, e^{-(b_t - \alpha)|u|} \cdot e^{2\alpha |v|}.
\]

\medskip
\noindent\textbf{Step 3: Integration over \( \R^2 \).}
We compute:
\[
\begin{aligned}
\iint_{\R^2} |\widetilde{K}_t(x,y)|\, dx\,dy
&= \iint_{\R^2} |\widetilde{K}_t(u + v, v)|\, du\,dv \\
&\le C_t \left( \int_{\R} e^{-(b_t - \alpha)|u|} \, du \right)
        \left( \int_{\R} e^{2\alpha |v|} \, dv \right).
\end{aligned}
\]
Both integrals are finite since \( b_t > \alpha \), and \( \alpha > \pi \). Therefore,
\[
\sup_{0 < t \le 1} \iint_{\R^2} |\widetilde{K}_t(x,y)|\, dx\,dy \le A_3(\alpha),
\]
for some constant \( A_3(\alpha) > 0 \) independent of \( t \).

\medskip
\noindent\textbf{Conclusion.}
The conjugated kernels \( \widetilde{K}_t \) are uniformly in \( L^1(\R^2) \) for \( t \in (0,1] \). Hence,
\[
\sup_{0 < t \le 1} \|K_t\|_{\mathcal{C}_1(\Psi_\alpha)} < \infty,
\]
and the corresponding operators
\[
L_t f(x) := \int_{\R} K_t(x,y)\, f(y)\, dy
\]
lie in the trace class \( \mathcal{C}_1(\HPsi) \), with
\[
\sup_{0 < t \le 1} \|L_t\|_{\mathcal{C}_1(\HPsi)} \le A_3(\alpha).
\]
This uniform bound ensures convergence in trace norm \( L_t \to L_{\sym} \), and uniform convergence of \( \det_\zeta(I - \lambda L_t) \) on compact subsets of \( \lambda \in \C \).
\end{proof}


% --- Symmetry and Fourier reflection ---
\begin{lemma}[Symmetry of Conjugated Kernel]
\label{lem:kernel-symmetry}
Let \( \phi(\lambda) := \Xi\left( \tfrac{1}{2} + i\lambda \right) \), and define the centered inverse Fourier kernel
\[
K_0(x,y) := \frac{1}{2\pi} \int_{\mathbb{R}} e^{i(x - y)\lambda} \phi(\lambda)\, d\lambda.
\]

Fix any \( \alpha > \pi \), and define the exponential weight \( \Psi_\alpha(x) := e^{\alpha |x|} \), along with the conjugated kernel
\[
\widetilde{K}_0(x,y) := \frac{K_0(x,y)}{\sqrt{\Psi_\alpha(x)\Psi_\alpha(y)}}.
\]

Then:
\begin{itemize}
    \item The kernel \( \widetilde{K}_0(x,y) \) is real-valued: \( \widetilde{K}_0(x,y) \in \mathbb{R} \) for all \( x, y \in \mathbb{R} \);
    \item The kernel is symmetric: \( \widetilde{K}_0(x,y) = \widetilde{K}_0(y,x) \).
\end{itemize}

\noindent
In particular, the conjugated kernel \( \widetilde{K}_0 \) defines a real symmetric integral operator on flat \( L^2(\mathbb{R}) \), and the corresponding unitarily equivalent operator
\[
L_{\mathrm{sym}} := U^{-1} T_{\widetilde{K}_0} U
\]
is symmetric on \( H_{\Psi_\alpha} \), where \( U f := \Psi_\alpha^{1/2} f \) defines the canonical unitary equivalence between \( H_{\Psi_\alpha} \) and \( L^2(\mathbb{R}) \).
\end{lemma}

\begin{proof}[Proof of \lemref{lem:kernel_symmetry}]
Let \( \phi(\lambda) := \Xi\left( \tfrac{1}{2} + i\lambda \right) \), where \( \Xi(s) \) is the completed Riemann zeta function. Since \( \Xi(s) \) is entire and satisfies the functional identity \( \Xi(s) = \Xi(1 - s) \), we compute:
\[
\phi(-\lambda) = \Xi\left( \tfrac{1}{2} - i\lambda \right) = \Xi\left( \tfrac{1}{2} + i\lambda \right) = \phi(\lambda),
\]
so \( \phi \) is even. Moreover, since \( \Xi(s) \in \mathbb{R} \) for real \( s \), it follows that \( \phi(\lambda) \in \mathbb{R} \) for all \( \lambda \in \mathbb{R} \).

\medskip
\noindent\textbf{Step 1: Real-Valued and Symmetric Fourier Kernel.}
Define
\[
K_0(x,y) := \frac{1}{2\pi} \int_{\mathbb{R}} e^{i(x - y)\lambda} \phi(\lambda)\, d\lambda = \widehat{\phi}(x - y).
\]
Since \( \phi \) is real and even, its inverse Fourier transform \( \widehat{\phi} \) is real-valued and even. Hence,
\[
K_0(x,y) = \widehat{\phi}(x - y) = \widehat{\phi}(y - x) = K_0(y,x) \in \mathbb{R}.
\]

\medskip
\noindent\textbf{Step 2: Symmetry of the Conjugated Kernel.}
Let \( \Psi_\alpha(x) := e^{\alpha |x|} \) for some fixed \( \alpha > \pi \), and define the conjugated kernel
\[
\widetilde{K}_0(x,y) := \frac{K_0(x,y)}{\sqrt{\Psi_\alpha(x)\Psi_\alpha(y)}}.
\]
Since \( \Psi_\alpha \) is even, the product \( \sqrt{\Psi_\alpha(x)\Psi_\alpha(y)} \) is symmetric in \( (x,y) \). Thus, the symmetry and real-valuedness of \( K_0(x,y) \) are preserved:
\[
\widetilde{K}_0(x,y) = \widetilde{K}_0(y,x) \in \mathbb{R}, \quad \forall x, y \in \mathbb{R}.
\]

\medskip
\noindent\textbf{Conclusion.}
The conjugated kernel \( \widetilde{K}_0(x,y) \) is real and symmetric. Hence, the corresponding integral operator
\[
(L_{\mathrm{sym}} f)(x) := \int_{\mathbb{R}} \widetilde{K}_0(x,y)\, f(y)\, dy
\]
is symmetric on \( L^2(\mathbb{R}) \). By unitary equivalence via \( U f = \Psi_\alpha^{1/2} f \), it follows that \( L_{\mathrm{sym}} \) is symmetric on \( H_{\Psi_\alpha} \), initially defined on the dense core \( \mathcal{S}(\mathbb{R}) \subset H_{\Psi_\alpha} \). This symmetry underpins the self-adjointness of \( L_{\mathrm{sym}} \) developed in later chapters.
\end{proof}


\begin{lemma}[Fourier Reflection Symmetry of Convolution Kernels]
\label{lem:fourier_symmetry_reflection}
Let \( \phi \colon \R \to \R \) be a real-valued, even function:
\[
\phi(\lambda) = \phi(-\lambda), \qquad \phi(\lambda) \in \R \quad \forall \lambda \in \R.
\]

Suppose \( \phi \in L^1(\R) \), and define its inverse Fourier transform
\[
k(x) := \frac{1}{2\pi} \int_{\R} e^{i \lambda x} \phi(\lambda)\, d\lambda.
\]

Then:
\begin{itemize}
    \item \( k(x) \in \R \) for all \( x \in \R \);
    \item \( k(x) = k(-x) \), i.e., \( k \) is even.
\end{itemize}

\medskip
\noindent
Consequently, the translation-invariant kernel
\[
K(x,y) := k(x - y)
\]
is real-valued and symmetric:
\[
K(x,y) = K(y,x), \qquad \forall x, y \in \R.
\]

\medskip
\noindent
In particular, the associated convolution operator
\[
(Lf)(x) := \int_{\R} k(x - y) f(y)\, dy
\]
is real and symmetric on any Hilbert space in which it is densely defined (e.g., \( \HPsi \)), and its formal adjoint coincides with the operator on its Schwartz core \( \Schwartz \).
\end{lemma}

\begin{proof}[Proof of \lemref{lem:fourier_symmetry_reflection}]
Let \( \phi \colon \R \to \R \) be an even, real-valued function:
\[
\phi(\lambda) = \phi(-\lambda), \qquad \phi(\lambda) \in \R, \quad \forall \lambda \in \R.
\]

Define the inverse Fourier transform
\[
k(x) := \frac{1}{2\pi} \int_{\R} e^{i \lambda x} \phi(\lambda)\, d\lambda.
\]

\medskip
\noindent\textbf{Step 1: Real-Valuedness.}
We compute:
\[
\overline{k(x)} = \frac{1}{2\pi} \int_{\R} e^{-i \lambda x} \phi(\lambda)\, d\lambda.
\]
Substitute \( \lambda \mapsto -\lambda \) and use the fact that \( \phi(-\lambda) = \phi(\lambda) \):
\[
\overline{k(x)} = \frac{1}{2\pi} \int_{\R} e^{i \lambda x} \phi(\lambda)\, d\lambda = k(x).
\]
Thus, \( k(x) \in \R \) for all \( x \in \R \).

\medskip
\noindent\textbf{Step 2: Evenness.}
We compute:
\[
k(-x) = \frac{1}{2\pi} \int_{\R} e^{-i \lambda x} \phi(\lambda)\, d\lambda.
\]
Again substituting \( \lambda \mapsto -\lambda \) and using evenness of \( \phi \), we find:
\[
k(-x) = \frac{1}{2\pi} \int_{\R} e^{i \lambda x} \phi(\lambda)\, d\lambda = k(x).
\]

\medskip
\noindent\textbf{Conclusion.}
The function \( k \) is real-valued and even. Therefore, the translation-invariant kernel
\[
K(x,y) := k(x - y)
\]
is symmetric:
\[
K(x,y) = k(x - y) = k(y - x) = K(y,x) \in \R.
\]
This proves that the associated convolution operator defines a real, symmetric integral operator on any Hilbert space where it is densely defined (e.g., \( \HPsi \)), and that its formal adjoint coincides with its action on the Schwartz core \( \Schwartz \).
\end{proof}


% --- Trace-class criteria and kernel embeddings ---
\begin{lemma}[Trace-Class Criterion via Weighted \( L^1 \) Kernel Control]
\label{lem:trace_class_via_weighted_L1}
Let \( \alpha > \pi \), and define the exponentially weighted Hilbert space
\[
H_{\Psi_\alpha} := L^2\left(\mathbb{R}, \Psi_\alpha(x)\, dx\right), \qquad \text{with } \Psi_\alpha(x) := e^{\alpha |x|}.
\]

Let \( K(x,y) \in C^\infty(\mathbb{R}^2) \) be a measurable kernel satisfying the exponential integrability condition:
\[
\iint_{\mathbb{R}^2} |K(x,y)|\, \Psi_\alpha(x) \Psi_\alpha(y)\, dx\, dy < \infty.
\]

Define the integral operator \( T \colon \mathcal{S}(\mathbb{R}) \subset H_{\Psi_\alpha} \to H_{\Psi_\alpha} \) by
\[
(Tf)(x) := \int_{\mathbb{R}} K(x,y)\, f(y)\, dy.
\]

Then:
\begin{itemize}
    \item \( T \) extends to a bounded operator on \( H_{\Psi_\alpha} \);
    \item \( T \in \mathcal{C}_1(H_{\Psi_\alpha}) \), i.e., trace class;
    \item Its trace norm satisfies:
    \[
    \|T\|_{\mathcal{C}_1(H_{\Psi_\alpha})} \le \iint_{\mathbb{R}^2} |K(x,y)|\, \Psi_\alpha(x)\Psi_\alpha(y)\, dx\, dy.
    \]
\end{itemize}

\medskip
\noindent\textbf{Unitary Reduction.}
Let \( U \colon H_{\Psi_\alpha} \to L^2(\mathbb{R}) \) be the unitary map
\[
(Uf)(x) := \Psi_\alpha(x)^{1/2} f(x), \qquad U^{-1} h(x) := \Psi_\alpha(x)^{-1/2} h(x).
\]
Then the conjugated operator \( \widetilde{T} := U T U^{-1} \) acts on \( L^2(\mathbb{R}) \) and has kernel
\[
\widetilde{K}(x,y) := \frac{K(x,y)}{\sqrt{\Psi_\alpha(x)\Psi_\alpha(y)}}.
\]
By hypothesis, \( \widetilde{K} \in L^1(\mathbb{R}^2) \), and by Simon’s trace-class kernel theorem \cite[Thm.~4.2]{Simon2005TraceIdeals}, it follows that \( \widetilde{T} \in \mathcal{C}_1(L^2(\mathbb{R})) \). Hence,
\[
T = U^{-1} \widetilde{T} U \in \mathcal{C}_1(H_{\Psi_\alpha}),
\]
with trace norm bounded by the weighted kernel norm.
\end{lemma}

\begin{proof}[Proof of \lemref{lem:trace_class_via_weighted_L1}]
Let \( T \) denote the integral operator on \( H_{\Psi_\alpha} := L^2(\mathbb{R}, \Psi_\alpha(x)\,dx) \), where \( \Psi_\alpha(x) := e^{\alpha |x|} \), defined on the dense subspace \( \mathcal{S}(\mathbb{R}) \) by
\[
(Tf)(x) := \int_{\mathbb{R}} K(x,y)\, f(y)\, dy,
\]
where the kernel \( K(x,y) \in C^\infty(\mathbb{R}^2) \) satisfies
\[
\iint_{\mathbb{R}^2} |K(x,y)|\, \Psi_\alpha(x)\Psi_\alpha(y)\, dx\,dy < \infty.
\]

\medskip
\noindent\textbf{Step 1: Unitary Conjugation to Flat \( L^2 \).}
Define the unitary map
\[
U_\alpha \colon H_{\Psi_\alpha} \to L^2(\mathbb{R}), \quad (U_\alpha f)(x) := \Psi_\alpha(x)^{1/2} f(x) = e^{\frac{\alpha}{2}|x|} f(x),
\]
with inverse \( U_\alpha^{-1} h(x) := e^{-\frac{\alpha}{2}|x|} h(x) \). Then the conjugated operator \( \widetilde{T} := U_\alpha T U_\alpha^{-1} \) acts on \( L^2(\mathbb{R}) \) with kernel
\[
\widetilde{K}(x,y) := \frac{K(x,y)}{\sqrt{\Psi_\alpha(x)\Psi_\alpha(y)}}.
\]

\medskip
\noindent\textbf{Step 2: Trace-Class Norm via Simon’s Criterion.}
Since \( \widetilde{K} \in L^1(\mathbb{R}^2) \) by assumption, Simon’s kernel criterion \cite[Thm.~4.2]{Simon2005TraceIdeals} yields:
\[
\widetilde{T} \in \mathcal{C}_1(L^2(\mathbb{R})), \quad \text{with } \|\widetilde{T}\|_{\mathcal{C}_1} \le \iint_{\mathbb{R}^2} |\widetilde{K}(x,y)|\, dx\,dy.
\]
Substituting back:
\[
\|\widetilde{T}\|_{\mathcal{C}_1} = \iint_{\mathbb{R}^2} \left| \frac{K(x,y)}{\sqrt{\Psi_\alpha(x)\Psi_\alpha(y)}} \right|\, dx\,dy = \iint_{\mathbb{R}^2} |K(x,y)|\, \Psi_\alpha(x)\Psi_\alpha(y)\, dx\,dy.
\]

\medskip
\noindent\textbf{Step 3: Unitary Invariance.}
Since \( U_\alpha \) is unitary, the trace norm is invariant:
\[
\|T\|_{\mathcal{C}_1(H_{\Psi_\alpha})} = \|\widetilde{T}\|_{\mathcal{C}_1(L^2)}.
\]

\medskip
\noindent\textbf{Conclusion.}
We conclude that \( T \in \mathcal{C}_1(H_{\Psi_\alpha}) \) and
\[
\|T\|_{\mathcal{C}_1(H_{\Psi_\alpha})} \le \iint_{\mathbb{R}^2} |K(x,y)|\, \Psi_\alpha(x)\Psi_\alpha(y)\, dx\,dy.
\]

This establishes that any integral operator with exponentially weighted kernel bounds lies in the trace class on \( H_{\Psi_\alpha} \). This criterion governs the trace-class inclusion of all operators \( L_t \) and \( L_{\mathrm{sym}} \) used in the determinant framework.
\end{proof}


\begin{lemma}[Weighted Hilbert--Schmidt Bound]
\label{lem:kernel_L2_weighted_bound}
Let \( \alpha > \pi \), and let \( K_t(x,y) \) be the mollified kernel obtained by convolution with an inverse Fourier transform of the mollified profile
\[
\phi_t(\lambda) := e^{-t\lambda^2} \, \Xi\left( \tfrac{1}{2} + i\lambda \right).
\]

Then:
\[
\iint_{\R^2} |K_t(x,y)|^2 \, e^{\alpha|x|} e^{\alpha|y|} \, dx \, dy < \infty,
\]
i.e., \( K_t \in L^2(\R^2, \Psi_\alpha(x)\Psi_\alpha(y)\, dx\, dy) \), where \( \Psi_\alpha(x) := e^{\alpha |x|} \).

\medskip
\noindent Consequently, the operator
\[
L_t f(x) := \int_{\R} K_t(x,y)\, f(y)\, dy
\]
is a Hilbert--Schmidt operator on \( \HPsi \), and therefore compact.
\end{lemma}

\begin{proof}[Proof of \lemref{lem:kernel_L2_weighted_bound}]
By \lemref{lem:uniform_L1_conjugated_kernel}, the mollified kernel \( K_t(x,y) \) satisfies the exponential bound
\[
|K_t(x,y)| \leq C_t \, e^{-\beta(|x| + |y|)} \qquad \text{for some } \beta > \alpha.
\]

We square both sides and integrate against the weighted measure \( e^{\alpha|x| + \alpha|y|}\, dx\, dy \), yielding:
\[
\begin{aligned}
\iint_{\R^2} |K_t(x,y)|^2\, e^{\alpha|x| + \alpha|y|}\, dx\, dy
&\le C_t^2 \iint_{\R^2} e^{-2\beta(|x| + |y|)}\, e^{\alpha(|x| + |y|)}\, dx\, dy \\
&= C_t^2 \left( \int_{\R} e^{-(2\beta - \alpha)|x|}\, dx \right)^2 < \infty,
\end{aligned}
\]
since \( 2\beta - \alpha > \beta > 0 \) and \( \alpha > \pi \) by hypothesis.

\medskip
\noindent
Thus, \( K_t \in L^2(\R^2, \Psi_\alpha(x)\Psi_\alpha(y)\, dx\, dy) \), i.e., \( K_t \in L^2(\Psi_\alpha^{\otimes 2}) \), and so the operator
\[
L_t f(x) := \int_{\R} K_t(x,y) f(y)\, dy
\]
is Hilbert--Schmidt on \( \HPsi \), and therefore compact.
\end{proof}


\begin{lemma}[Trace-Class Criterion for Conjugated Kernels]
\label{lem:trace_class_conjugated_kernel}
Let \( K(x, y) \in C^\infty(\mathbb{R}^2) \) be a real-valued, symmetric kernel: \( K(x,y) = K(y,x) \) for all \( x,y \in \mathbb{R} \).

Fix \( \alpha > 0 \), and define the weighted Hilbert space
\[
H_{\Psi_\alpha} := L^2(\mathbb{R}, \Psi_\alpha(x)\, dx), \qquad \text{where } \Psi_\alpha(x) := e^{\alpha |x|}.
\]

Suppose the exponentially conjugated kernel satisfies the integrability condition:
\[
\iint_{\mathbb{R}^2} |K(x, y)| \, \Psi_\alpha(x) \Psi_\alpha(y) \, dx\, dy < \infty.
\]

Then the integral operator
\[
(T f)(x) := \int_{\mathbb{R}} K(x, y)\, f(y)\, dy
\]
extends to a bounded trace-class operator on \( H_{\Psi_\alpha} \):
\[
T \in \mathcal{C}_1(H_{\Psi_\alpha}),
\]
with trace norm estimate:
\[
\| T \|_{\mathcal{C}_1(H_{\Psi_\alpha})}
\le \iint_{\mathbb{R}^2} |K(x, y)|\, \Psi_\alpha(x)\Psi_\alpha(y)\, dx\, dy.
\]

\paragraph{Remarks.}
\begin{itemize}
    \item The symmetry and real-valuedness of \( K \) imply that \( T \) is formally self-adjoint on the Schwartz core \( \mathcal{S}(\mathbb{R}) \subset H_{\Psi_\alpha} \). If \( K \) is sufficiently regular, this lifts to self-adjointness on the closure.

    \item This result follows from \lemref{lem:trace_class_via_weighted_L1} by applying unitary conjugation to flat \( L^2(\mathbb{R}) \), where the transformed kernel
    \[
    \widetilde{K}(x,y) := \frac{K(x,y)}{\sqrt{\Psi_\alpha(x)\Psi_\alpha(y)}}
    \]
    lies in \( L^1(\mathbb{R}^2) \), as guaranteed by \lemref{lem:L1_integrability_conjugated_kernel}, enabling Simon's trace-class criterion \cite[Thm.~4.2]{Simon2005TraceIdeals}.
\end{itemize}
\end{lemma}

\begin{proof}[Proof of \lemref{lem:trace-class-conjugated-kernel}]
Let \( T \) be the integral operator defined by
\[
(Tf)(x) := \int_{\mathbb{R}} K(x, y) f(y) \, dy,
\]
initially acting on the dense subspace \( \mathcal{S}(\mathbb{R}) \subset H_{\Psi_\alpha} := L^2(\mathbb{R}, \Psi_\alpha(x)\, dx) \), where \( \Psi_\alpha(x) := e^{\alpha |x|} \), with \( \alpha > 0 \). Assume that
\[
\iint_{\mathbb{R}^2} |K(x, y)|\, \Psi_\alpha(x) \Psi_\alpha(y)\, dx\, dy < \infty.
\]

\medskip
\noindent\textbf{Step 1: Unitary Conjugation to Flat \( L^2 \).}
Define the unitary map
\[
U \colon H_{\Psi_\alpha} \to L^2(\mathbb{R}), \quad (Uf)(x) := \Psi_\alpha(x)^{1/2} f(x).
\]
Then \( U \) is an isometric isomorphism, with inverse \( U^{-1}g(x) = \Psi_\alpha(x)^{-1/2} g(x) \).

The conjugated operator \( \widetilde{T} := U T U^{-1} \) acts on \( L^2(\mathbb{R}) \) via the kernel
\[
\widetilde{K}(x, y) := \frac{K(x,y)}{\sqrt{\Psi_\alpha(x) \Psi_\alpha(y)}}.
\]

\medskip
\noindent\textbf{Step 2: Trace-Class Criterion in Flat \( L^2 \).}
Since
\[
\iint_{\mathbb{R}^2} |\widetilde{K}(x, y)|\, dx\, dy = \iint_{\mathbb{R}^2} |K(x, y)|\, \Psi_\alpha(x) \Psi_\alpha(y)\, dx\, dy < \infty,
\]
we apply Simon’s trace-class criterion \cite[Thm.~4.2]{Simon2005TraceIdeals}. Therefore,
\[
\widetilde{T} \in \mathcal{C}_1(L^2(\mathbb{R})), \quad \text{with } \|\widetilde{T}\|_{\mathcal{C}_1} \le \iint |\widetilde{K}(x, y)|\, dx\, dy.
\]

\medskip
\noindent\textbf{Step 3: Transfer Back to Weighted Space.}
Since \( T = U^{-1} \widetilde{T} U \) and \( U \) is unitary, we conclude:
\[
T \in \mathcal{C}_1(H_{\Psi_\alpha}), \quad \|T\|_{\mathcal{C}_1(H_{\Psi_\alpha})} = \|\widetilde{T}\|_{\mathcal{C}_1(L^2)} \le \iint |K(x, y)|\, \Psi_\alpha(x)\Psi_\alpha(y)\, dx\, dy.
\]

\medskip
\noindent\textbf{Conclusion.}
The kernel integrability condition implies that the conjugated kernel \( \widetilde{K} \in L^1(\mathbb{R}^2) \), and hence \( T \in \mathcal{C}_1(H_{\Psi_\alpha}) \). If \( K \) is symmetric, \( T \) is also symmetric on the Schwartz core, which underpins its spectral analysis.
\end{proof}


% --- Space density and Fourier integrability ---
\begin{lemma}[Density of Schwartz Space in Exponentially Weighted \( L^2 \)]
\label{lem:density_schwartz_weighted_L2}
Let \( \alpha > \pi \), and define the exponential weight
\[
\Psi_\alpha(x) := e^{\alpha |x|}, \qquad x \in \R.
\]
Let
\[
\HPsi := L^2(\R, \Psi_\alpha(x)\, dx)
\]
denote the associated weighted Hilbert space, equipped with inner product
\[
\langle f, g \rangle_{\HPsi} := \int_{\R} f(x) \overline{g(x)}\, \Psi_\alpha(x)\, dx,
\]
and norm
\[
\|f\|_{\HPsi} := \left( \int_{\R} |f(x)|^2\, e^{\alpha |x|} dx \right)^{1/2}.
\]

Then the Schwartz space \( \Schwartz \) is dense in \( \HPsi \); that is, for every \( f \in \HPsi \) and every \( \varepsilon > 0 \), there exists \( \phi \in \Schwartz \) such that
\[
\| f - \phi \|_{\HPsi} < \varepsilon.
\]
\end{lemma}

\begin{proof}[Proof of \lemref{lem:density_schwartz_weighted_L2}]
Fix \( \alpha > \pi \), and define the exponential weight
\[
\Psi_\alpha(x) := e^{\alpha |x|}.
\]
Then the associated weighted Hilbert space
\[
\HPsi := L^2(\R, \Psi_\alpha(x)\,dx)
\]
is equipped with inner product
\[
\langle f, g \rangle_{\HPsi} := \int_{\R} f(x)\, \overline{g(x)}\, \Psi_\alpha(x)\, dx.
\]

\medskip
\noindent\textbf{Step 1: Unitary Reduction to Flat \( L^2 \).}
Define the unitary transformation
\[
U \colon \HPsi \to L^2(\R), \qquad (Uf)(x) := \Psi_\alpha(x)^{1/2} f(x),
\]
with inverse
\[
U^{-1}(h)(x) := \Psi_\alpha(x)^{-1/2} h(x).
\]
Then for all \( f, g \in \HPsi \), one has
\[
\langle Uf, Ug \rangle_{L^2} = \langle f, g \rangle_{\HPsi}, \qquad \|Uf\|_{L^2} = \|f\|_{\HPsi}.
\]

\medskip
\noindent\textbf{Step 2: Density in Standard \( L^2 \).}
Let \( f \in \HPsi \), and define \( g := Uf \in L^2(\R) \). Since \( \Schwartz \) is dense in \( L^2(\R) \), for any \( \varepsilon > 0 \) there exists \( \varphi \in \Schwartz \) such that
\[
\|g - \varphi\|_{L^2} < \varepsilon.
\]
Define \( f_\varepsilon := U^{-1}(\varphi) = \Psi_\alpha^{-1/2}(x) \varphi(x) \in \HPsi \). Then
\[
\|f - f_\varepsilon\|_{\HPsi} = \|Uf - \varphi\|_{L^2} < \varepsilon.
\]

\medskip
\noindent\textbf{Step 3: Image of Schwartz is Schwartz.}
Since \( \varphi \in \Schwartz \) and \( \Psi_\alpha^{-1/2} \in C^\infty(\R) \) with exponential decay, the product
\[
f_\varepsilon(x) = \varphi(x) \cdot \Psi_\alpha(x)^{-1/2}
\]
belongs to \( \Schwartz \), by standard closure properties of Schwartz space under multiplication by smooth functions of sub-exponential growth. Therefore,
\[
f_\varepsilon \in \Schwartz \cap \HPsi, \qquad \|f - f_\varepsilon\|_{\HPsi} < \varepsilon.
\]

\medskip
\noindent\textbf{Conclusion.}
As \( f \in \HPsi \) and \( \varepsilon > 0 \) were arbitrary, we conclude that \( \Schwartz \) is dense in \( \HPsi \). This holds for all \( \alpha > 0 \), and in particular for \( \alpha > \pi \), which ensures compatibility with the exponential type of Paley--Wiener kernels derived from \( \Xi \).

\medskip
\noindent\textbf{Functional Consequence.}
The density \( \Schwartz \subset \HPsi \) provides a common core domain for the convolution operators \( L_t \) and their limit \( L_{\sym} \). In particular, it ensures:
\begin{itemize}
    \item Symmetry of \( L_t \) under reflection: \( L_t^* = L_t \) on \( \Schwartz \);
    \item Essential self-adjointness of \( L_{\sym} \) on \( \Schwartz \);
    \item Justification of heat trace and determinant convergence via mollifier approximation.
\end{itemize}
\end{proof}


% --- Sharpness of weight condition ---
\begin{lemma}[Sharp Decay Threshold for Trace-Class Inclusion]
\label{lem:trace_class_failure_alpha_leq_pi}
Let \( \phi(\lambda) := \Xi\left( \tfrac{1}{2} + i\lambda \right) \), and define the inverse Fourier transform
\[
k(x) := \mathcal{F}^{-1}[\phi](x), \qquad \text{with } \phi \in \operatorname{PW}_\pi(\mathbb{R}).
\]

Set \( K(x,y) := k(x - y) \), and fix \( \alpha > 0 \). Then:
\begin{itemize}
    \item[\textup{(i)}] There exists a constant \( c > 0 \) such that
    \[
    |k(x)| \ge c\, e^{-\pi |x|}, \qquad \text{as } |x| \to \infty.
    \]

    \item[\textup{(ii)}] For any \( \alpha \le \pi \), the weighted kernel
    \[
    |K(x,y)|\, \Psi_\alpha(x)\Psi_\alpha(y) = |k(x - y)|\, e^{\alpha(|x| + |y|)}
    \]
    does not lie in \( L^1(\mathbb{R}^2) \). That is,
    \[
    \iint_{\mathbb{R}^2} |K(x,y)|\, \Psi_\alpha(x)\Psi_\alpha(y)\, dx\,dy = \infty.
    \]

    \item[\textup{(iii)}] Consequently, for \( \alpha \le \pi \), the convolution operator
    \[
    (L f)(x) := \int_{\mathbb{R}} k(x - y)\, f(y)\, dy
    \]
    fails to lie in the trace class \( \mathcal{C}_1(H_{\Psi_\alpha}) \), and Simon’s kernel trace-class criterion does not apply.
\end{itemize}
\end{lemma}

\begin{proof}[Proof of \lemref{lem:trace_class_failure_alpha_leq_pi}]
\textbf{(i) Lower Envelope Bound for \( k \).}
Let \( \phi(\lambda) := \Xi\left( \tfrac{1}{2} + i\lambda \right) \in PW_\pi(\R) \). By the Paley--Wiener theorem (see \cite[Thm.~3.2.4]{Levin1996EntireLectures}), its inverse Fourier transform
\[
k(x) := \frac{1}{2\pi} \int_{\R} e^{i\lambda x} \phi(\lambda)\, d\lambda
\]
is supported in the interval \( [-\pi, \pi] \) in the complex-analytic sense, and satisfies
\[
|k(x)| \le C_\epsilon e^{-(\pi - \epsilon)|x|}, \qquad \forall \epsilon > 0.
\]
Moreover, as established in classical estimates (cf. \cite[§4.12]{Titchmarsh1986Zeta}), there exists \( c > 0 \) such that
\[
|k(x)| \ge c\, e^{-\pi |x|}, \qquad \text{for all sufficiently large } |x|.
\]

\medskip
\noindent\textbf{(ii) Failure of \( L^1(\Psi_\alpha^{\otimes 2}) \) for \( \alpha \le \pi \).}
Set \( K(x,y) := k(x - y) \) and fix \( \Psi_\alpha(x) := e^{\alpha |x|} \). Then
\[
\iint_{\R^2} |K(x,y)|\, \Psi_\alpha(x) \Psi_\alpha(y)\, dx\,dy = \iint_{\R^2} |k(x - y)|\, e^{\alpha(|x| + |y|)}\, dx\,dy.
\]
Make the change of variables \( u := x - y \), \( v := y \), so \( x = u + v \), \( dx\,dy = du\,dv \). Then
\[
= \int_{\R} |k(u)| \left( \int_{\R} e^{\alpha(|u + v| + |v|)}\, dv \right) du.
\]
Use the inequality \( |u + v| + |v| \ge |u| \) to get
\[
\int_{\R} e^{\alpha(|u + v| + |v|)}\, dv \ge e^{\alpha |u|} \int_{\R} e^{\alpha |v|}\, dv = C_\alpha\, e^{\alpha |u|}.
\]
Hence,
\[
\iint |K(x,y)| \Psi_\alpha(x)\Psi_\alpha(y)\, dx\,dy \ge C_\alpha \int_{\R} |k(u)|\, e^{\alpha |u|} du.
\]
From part (i), \( |k(u)| \gtrsim e^{-\pi |u|} \), so for \( \alpha \le \pi \),
\[
\int_{\R} |k(u)|\, e^{\alpha |u|} du \gtrsim \int_{\R} e^{-(\pi - \alpha)|u|} du = \infty.
\]

\medskip
\noindent\textbf{(iii) Conclusion.}
We conclude that \( K \notin L^1(\R^2, \Psi_\alpha(x)\Psi_\alpha(y)\, dx\,dy) \), so the convolution operator
\[
(Lf)(x) := \int_{\R} k(x - y)\, f(y)\, dy
\]
fails to be trace class in \( \mathcal{C}_1(\HPsi) \) when \( \alpha \le \pi \). Thus, \( \alpha > \pi \) is sharp for ensuring trace-class regularity of \( L_{\sym} \).
\end{proof}


\begin{proposition}[Sharpness of Trace-Class Inclusion for \(\alpha > \pi\)]
\label{prop:trace_class_sharpness}
Let \( \phi(\lambda) := \Xi\left( \tfrac{1}{2} + i\lambda \right) \), and let \( k := \FT^{-1}[\phi] \in L^1_{\mathrm{loc}}(\R) \) denote its inverse Fourier transform. Define the translation-invariant kernel
\[
K(x, y) := k(x - y), \qquad \Psi_\alpha(x) := e^{\alpha |x|}, \quad \text{for } \alpha > 0.
\]

Then for any \( \alpha \le \pi \),
\[
\int_{\R^2} |K(x, y)| \Psi_\alpha(x) \Psi_\alpha(y) \, dx \, dy = \infty,
\]
and hence the corresponding convolution operator
\[
(Lf)(x) := \int_{\R} k(x - y) f(y) \, dy
\]
fails to lie in the trace class \( \TC(H_{\Psi_\alpha}) \).

\medskip
\noindent
In particular, the critical threshold \( \alpha > \pi \) is sharp for trace-norm convergence and Fredholm determinant realization in the weighted Hilbert space \( H_{\Psi_\alpha} := L^2(\R, \Psi_\alpha(x)\, dx) \).
\end{proposition}

\begin{proof}[Proof of \cref{prop:trace_class_sharpness}]
Let \( \phi(\lambda) := \Xi\left( \tfrac{1}{2} + i\lambda \right) \), and define its inverse Fourier transform
\[
k(x) := \frac{1}{2\pi} \int_{\R} e^{i\lambda x} \phi(\lambda)\, d\lambda.
\]
Then \( k \in \Schwartz(\R) \), and the associated kernel is \( K(x, y) := k(x - y) \). The weighted kernel norm is
\[
\int_{\R^2} |K(x, y)| \Psi_\alpha(x) \Psi_\alpha(y) \, dx\,dy
= \int_{\R^2} |k(x - y)| e^{\alpha(|x| + |y|)} \, dx\,dy.
\]

\paragraph{Step 1: Change of variables.}
Let \( u := x - y \), \( v := y \), so that \( x = u + v \), and \( dx\,dy = du\,dv \). Then
\[
\int_{\R^2} |k(x - y)| \Psi_\alpha(x) \Psi_\alpha(y) \, dx\,dy
= \int_{\R} |k(u)| \left( \int_{\R} e^{\alpha(|u + v| + |v|)}\, dv \right) du.
\]

\paragraph{Step 2: Lower bound for the inner integral.}
Using the inequality \( |u + v| + |v| \ge |u| \), we have
\[
\int_{\R} e^{\alpha(|u + v| + |v|)}\, dv \ge e^{\alpha |u|} \int_{\R} e^{\alpha |v|}\, dv = C_\alpha e^{\alpha |u|},
\]
for a constant \( C_\alpha = \int_{\R} e^{\alpha |v|}\, dv < \infty \).

\paragraph{Step 3: Divergence of the weighted norm.}
Thus,
\[
\int_{\R^2} |K(x, y)| \Psi_\alpha(x) \Psi_\alpha(y) \, dx\,dy \ge C_\alpha \int_{\R} |k(u)| e^{\alpha |u|} \, du.
\]
But by Paley--Wiener theory (see \cite[Thm.~3.2.4]{Levin1996EntireLectures}), \( k(u) \notin L^1(\R, e^{\alpha |u|} du) \) when \( \alpha \le \pi \), because \( \phi \) has exponential type \( \pi \) and \( k \sim e^{-\pi |x|} \) is asymptotically optimal (see Lemma~1.23).

Hence,
\[
\int_{\R^2} |K(x, y)| \Psi_\alpha(x) \Psi_\alpha(y)\, dx\,dy = \infty \quad \text{for } \alpha \le \pi.
\]

\paragraph{Conclusion.}
By Simon’s criterion~\cite[Thm.~4.2]{Simon2005TraceIdeals}, \( K \notin L^1(\R^2, \Psi_\alpha(x)\Psi_\alpha(y)\,dx\,dy) \) implies that the associated convolution operator
\[
(Lf)(x) := \int_{\R} k(x - y) f(y) \, dy
\]
does not lie in \( \TC(H_{\Psi_\alpha}) \). Thus, the condition \( \alpha > \pi \) is sharp.
\end{proof}


% --- Structural preservation under unitary conjugation ---
\begin{lemma}[Unitary Conjugation and Trace-Class Equivalence]
\label{lem:unitary_conjugation_trace_class}
Let \( U \colon \mathcal{H} \to \widetilde{\mathcal{H}} \) be a unitary isomorphism between Hilbert spaces, and let \( T \colon \mathcal{H} \to \mathcal{H} \) be a bounded operator. Define the conjugated operator
\[
\widetilde{T} := U T U^{-1} \colon \widetilde{\mathcal{H}} \to \widetilde{\mathcal{H}}.
\]

Then for all \( 1 \le p \le \infty \),
\[
T \in \mathcal{C}_p(\mathcal{H}) \quad \iff \quad \widetilde{T} \in \mathcal{C}_p(\widetilde{\mathcal{H}}),
\]
and
\[
\| \widetilde{T} \|_{\mathcal{C}_p} = \| T \|_{\mathcal{C}_p}.
\]

In particular:
\begin{itemize}
  \item If \( T \in \mathcal{C}_1(\mathcal{H}) \), then \( \widetilde{T} \in \mathcal{C}_1(\widetilde{\mathcal{H}}) \) with equal trace norm.
  \item The spectrum and Fredholm determinant satisfy:
  \[
  \det\nolimits_\zeta(I - \lambda T) = \det\nolimits_\zeta(I - \lambda \widetilde{T}), \qquad \forall \lambda \in \C.
  \]
\end{itemize}

This lemma applies, for example, when \( \mathcal{H} = H_{\Psi_\alpha} \), \( \widetilde{\mathcal{H}} = L^2(\R) \), and \( U f(x) := \Psi_\alpha(x)^{1/2} f(x) \) is the exponential conjugation map.
\end{lemma}

\begin{proof}[Proof of \lemref{lem:unitary_conjugation_trace_class}]
Let \( U \colon \mathcal{H} \to \widetilde{\mathcal{H}} \) be a unitary operator, and let \( T \in \mathcal{B}(\mathcal{H}) \). Define the conjugated operator
\[
\widetilde{T} := U T U^{-1} \in \mathcal{B}(\widetilde{\mathcal{H}}).
\]

\medskip
\noindent\textbf{Step 1: Schatten Ideal Equivalence.}
It is a standard fact in operator theory that Schatten ideals are unitarily invariant (see \cite[Ch.~1]{Simon2005TraceIdeals}). That is:
\[
T \in \mathcal{C}_p(\mathcal{H}) \quad \iff \quad \widetilde{T} \in \mathcal{C}_p(\widetilde{\mathcal{H}}), \qquad \forall\, 1 \le p \le \infty,
\]
with equality of norms:
\[
\| T \|_{\mathcal{C}_p(\mathcal{H})} = \| \widetilde{T} \|_{\mathcal{C}_p(\widetilde{\mathcal{H}})}.
\]

\medskip
\noindent\textbf{Step 2: Trace-Class and Determinants.}
In the trace-class case \( p = 1 \), the trace norm and the Fredholm determinant are preserved:
\[
\Tr_{\mathcal{H}}(T) = \Tr_{\widetilde{\mathcal{H}}}(\widetilde{T}), \qquad \det\nolimits_\zeta(I - \lambda T) = \det\nolimits_\zeta(I - \lambda \widetilde{T}).
\]
This follows from cyclicity of trace and the unitarity of \( U \).

\medskip
\noindent\textbf{Step 3: Application to Weighted Hilbert Spaces.}
In the specific setting where \( \mathcal{H} := \HPsi \), \( \widetilde{\mathcal{H}} := L^2(\R) \), and
\[
U f(x) := \Psi_\alpha(x)^{1/2} f(x),
\]
then any operator \( T \in \mathcal{C}_p(\HPsi) \) with kernel \( K(x,y) \) satisfies that \( \widetilde{T} := U T U^{-1} \in \mathcal{C}_p(L^2(\R)) \) has kernel
\[
\widetilde{K}(x,y) := \frac{K(x,y)}{\sqrt{\Psi_\alpha(x)\Psi_\alpha(y)}},
\]
and norm equivalence follows.

\medskip
\noindent\textbf{Conclusion.}
The Schatten-class property and trace norm are preserved under unitary conjugation, and Fredholm determinants remain invariant. This verifies the result.
\end{proof}


% --- Functional analytic justification for core structure ---
\begin{remark}[Core Density and Sobolev Completion]
\label{rem:sobolev_core_reference}
The density of \( \Schwartz(\R) \subset H_{\Psi_\alpha} \) can also be justified via Sobolev spaces adapted to exponential weights.

\medskip

Let \( H^s_\alpha(\R) \) denote the weighted Sobolev space defined by
\[
H^s_\alpha(\R) := \left\{ f \in L^2_{\mathrm{loc}}(\R) \,\middle|\, \langle D \rangle^s f \in L^2(\R, \Psi_\alpha(x)\, dx) \right\}, \quad \Psi_\alpha(x) := e^{\alpha |x|}, \quad \alpha > 0.
\]
Then the embeddings
\[
\Schwartz(\R) \hookrightarrow H^s_\alpha(\R) \hookrightarrow H_{\Psi_\alpha}
\]
are continuous and dense for all \( s \ge 0 \), and can be used to construct a graph-norm core for unbounded convolution operators with exponentially decaying kernels.

\medskip

This functional-analytic perspective complements the analytic vector argument used to establish essential self-adjointness of \( L_{\sym} \) on \( \Schwartz(\R) \) in later chapters. It also justifies stability of mollifier domains and semigroup bounds under Sobolev-scale closure.
\end{remark}


%------------------------------------------------------------------
\subsection{Operator-Theoretic Properties of \texorpdfstring{\( L_t \)}{Lt}}

% --- Boundedness, compactness, symmetry, self-adjointness ---
\begin{proposition}[Boundedness of \( L_t \) on Weighted Hilbert Space]
\label{prop:boundedness_Lt_weighted}
Let \( \Psi \colon \mathbb{R} \to (0,\infty) \) be a smooth, strictly positive weight function satisfying
\[
\Psi(x) \sim e^{\alpha |x|} \quad \text{as } |x| \to \infty,
\]
for some fixed \( \alpha > 0 \); that is, there exist constants \( c_1, c_2 > 0 \) such that
\[
c_1 e^{\alpha |x|} \le \Psi(x) \le c_2 e^{\alpha |x|}, \quad \forall x \in \mathbb{R}.
\]

Define the exponentially weighted Hilbert space
\[
H_\Psi := L^2(\mathbb{R}, \Psi(x)\, dx).
\]

Suppose \( \phi_t \in \mathcal{S}(\mathbb{R}) \) is real-valued, even, and satisfies: for each \( N > 0 \), there exists a constant \( C_N(t) > 0 \) such that
\[
|\phi_t(z)| \le C_N(t)\, (1 + |z|)^{-N}, \quad \forall z \in \mathbb{R}.
\]

Define the convolution operator
\[
(L_t f)(x) := \int_{\mathbb{R}} \phi_t(x - y)\, f(y)\, dy, \quad f \in \mathcal{S}(\mathbb{R}).
\]

Then \( L_t \) extends uniquely to a bounded linear operator on \( H_\Psi \), and there exists a constant \( C_t(\alpha) > 0 \) such that
\[
\| L_t f \|_{H_\Psi} \le C_t(\alpha) \cdot \| f \|_{H_\Psi}, \quad \forall f \in H_\Psi.
\]

In particular:
\begin{itemize}
    \item Each \( L_t \in \mathcal{B}(H_\Psi) \) is well-defined and bounded;
    \item The family \( \{L_t\}_{t > 0} \subset \mathcal{B}(H_\Psi) \) is uniformly bounded on compact t-intervals;
    \item The strong operator limit \( L_{\mathrm{sym}} := \lim_{t \to 0^+} L_t \) exists on \( H_\Psi \), due to strong convergence on a dense core and uniform boundedness;
    \item The exponential decay of \( \phi_t \) ensures trace-norm control and convergence in \( \mathcal{C}_1(H_\Psi) \).
\end{itemize}
\end{proposition}

\begin{proof}[Proof of \propref{prop:boundedness_Lt_weighted}]
Let \( H_\Psi := L^2(\mathbb{R}, \Psi(x)\, dx) \), where \( \Psi(x) \sim e^{\alpha |x|} \) for some fixed \( \alpha > 0 \). Let \( \phi_t \in \mathcal{S}(\mathbb{R}) \) be a real-valued, even mollifier, and define the convolution operator
\[
(L_t f)(x) := \int_{\mathbb{R}} \phi_t(x - y)\, f(y)\, dy, \quad f \in \mathcal{S}(\mathbb{R}).
\]

\medskip
\noindent\textbf{Step 1: Cauchy–Schwarz Pointwise Estimate.}
For fixed \( x \in \mathbb{R} \), apply the Cauchy–Schwarz inequality:
\[
|L_t f(x)|^2 \le \left( \int_{\mathbb{R}} |\phi_t(x - y)|^2 \Psi(y)^{-1}\, dy \right)
                \cdot \left( \int_{\mathbb{R}} |f(y)|^2 \Psi(y)\, dy \right).
\]
Multiplying by \( \Psi(x) \) and integrating in \( x \), we obtain:
\[
\| L_t f \|_{H_\Psi}^2 \le \left( \sup_{x \in \mathbb{R}} \int_{\mathbb{R}} |\phi_t(x - y)|^2 \cdot \frac{\Psi(x)}{\Psi(y)}\, dy \right) \cdot \| f \|_{H_\Psi}^2.
\]

\medskip
\noindent\textbf{Step 2: Estimate of Envelope Ratio and Kernel Decay.}
Since \( \Psi(x) \sim e^{\alpha |x|} \), there exists \( C_\alpha > 0 \) such that
\[
\frac{\Psi(x)}{\Psi(y)} \le C_\alpha\, e^{\alpha |x - y|}, \quad \forall x, y \in \mathbb{R}.
\]
Also, since \( \phi_t \in \mathcal{S}(\mathbb{R}) \), for each \( N > 0 \), there exists \( C_N > 0 \) such that
\[
|\phi_t(u)| \le C_N (1 + |u|)^{-N}, \quad \Rightarrow \quad |\phi_t(x - y)|^2 \le C_N^2 (1 + |x - y|)^{-2N}.
\]
Combining:
\[
|\phi_t(x - y)|^2 \cdot \frac{\Psi(x)}{\Psi(y)} \le C_N^2 C_\alpha (1 + |x - y|)^{-2N} e^{\alpha |x - y|},
\]
which is integrable in \( y \) uniformly in \( x \) provided \( N > \alpha \).

\medskip
\noindent\textbf{Step 3: Define the Uniform Bound.}
Choose \( N > \alpha \), and define
\[
C_t(\alpha) := \sup_{x \in \mathbb{R}} \int_{\mathbb{R}} |\phi_t(x - y)|^2 \cdot \frac{\Psi(x)}{\Psi(y)}\, dy < \infty.
\]
Then for all \( f \in \mathcal{S}(\mathbb{R}) \),
\[
\| L_t f \|_{H_\Psi} \le \sqrt{C_t(\alpha)} \cdot \| f \|_{H_\Psi}.
\]

\medskip
\noindent\textbf{Step 4: Extension to \( H_\Psi \).}
Since \( \mathcal{S}(\mathbb{R}) \subset H_\Psi \) is dense (by Lemma~\ref{lem:density_schwartz_weighted_L2}), and \( L_t \) is bounded on this core, it extends uniquely to a bounded linear operator on all of \( H_\Psi \), with
\[
\| L_t \|_{\mathcal{B}(H_\Psi)} \le \sqrt{C_t(\alpha)}.
\]

\medskip
\noindent\textbf{Conclusion.}
The operator \( L_t \colon H_\Psi \to H_\Psi \) is bounded, with norm controlled by the mollifier decay and the exponential behavior of the weight. This boundedness is a key analytic input for verifying trace-class properties, symmetry, and convergence of \( L_t \to L_{\mathrm{sym}} \) in both the operator norm and the trace-class topology.
\end{proof}


\begin{proposition}[Compactness of \( L_t \)]
\label{prop:compactness-Lt}
Let \( \Psi \colon \mathbb{R} \to (0,\infty) \) be a smooth, strictly positive weight function satisfying
\[
\Psi(x) \sim e^{\alpha|x|} \quad \text{as } |x| \to \infty,
\]
for some constant \( \alpha > 0 \); that is, there exist constants \( c_1, c_2 > 0 \) such that
\[
c_1 e^{\alpha |x|} \le \Psi(x) \le c_2 e^{\alpha |x|}, \quad \forall x \in \mathbb{R}.
\]

Let
\[
H_\Psi := L^2(\mathbb{R}, \Psi(x)\, dx)
\]
be the associated exponentially weighted Hilbert space.

Suppose \( \phi_t \in \mathcal{S}(\mathbb{R}) \) is a real-valued, even mollifier, and define the convolution operator
\[
(L_t f)(x) := \int_{\mathbb{R}} \phi_t(x - y)\, f(y)\, dy, \quad f \in \mathcal{S}(\mathbb{R}).
\]

Then \( L_t \) extends uniquely to a compact operator on \( H_\Psi \); that is,
\[
L_t \in \mathcal{K}(H_\Psi).
\]

\noindent The compactness follows from:
\begin{itemize}
    \item The kernel \( K_t(x,y) := \phi_t(x - y) \in C^\infty(\mathbb{R}^2) \) is rapidly decaying off the diagonal;
    \item The operator \( L_t \) maps bounded sets in \( H_\Psi \) into equicontinuous, rapidly decaying families (via convolution smoothing);
    \item The inclusion \( \mathcal{S}(\mathbb{R}) \hookrightarrow H_\Psi \) is continuous and dense, and \( K_t \in L^2(\Psi^{\otimes 2}) \) implies \( L_t \) is Hilbert–Schmidt (see Lemma~\ref{lem:kernel_L2_weighted_bound}).
\end{itemize}

\medskip
\noindent As a consequence, \( L_t \) has discrete spectrum with finite-multiplicity eigenvalues accumulating only at zero. This compactness ensures spectral discreteness and underpins the Schatten-class convergence and Fredholm determinant structure developed in subsequent chapters.
\end{proposition}

\begin{proof}[Proof of \propref{prop:compactness_Lt}]
Let \( H_\Psi := L^2(\mathbb{R}, \Psi(x)\, dx) \), where \( \Psi(x) := e^{\alpha |x|} \) for some fixed \( \alpha > 0 \). Let \( \phi_t \in \mathcal{S}(\mathbb{R}) \) be a real-valued, even mollifier, and define the convolution operator:
\[
(L_t f)(x) := \int_{\mathbb{R}} \phi_t(x - y)\, f(y)\, dy, \quad f \in \mathcal{S}(\mathbb{R}).
\]

\medskip
\noindent\textbf{Step 1: Unitary Conjugation to Flat \( L^2 \).}
Define the unitary map
\[
U \colon H_\Psi \to L^2(\mathbb{R}), \qquad (Uf)(x) := \Psi(x)^{1/2} f(x),
\]
with inverse \( U^{-1} g(x) := \Psi(x)^{-1/2} g(x) \). Then the conjugated operator \( \widetilde{L}_t := U L_t U^{-1} \) acts on \( L^2(\mathbb{R}) \) as an integral operator with kernel:
\[
\widetilde{K}_t(x,y) := \frac{\phi_t(x - y)}{\sqrt{\Psi(x)\Psi(y)}} = \phi_t(x - y) e^{-\frac{\alpha}{2}(|x| + |y|)}.
\]

\medskip
\noindent\textbf{Step 2: Hilbert–Schmidt Estimate.}
Since \( \phi_t \in \mathcal{S}(\mathbb{R}) \), we may estimate for any \( \varepsilon > 0 \):
\[
|\phi_t(z)| \le C_\varepsilon\, e^{-(\alpha + \varepsilon)|z|},
\]
so that
\[
|\widetilde{K}_t(x,y)| \le C'\, e^{-\delta(|x| + |y|)}, \quad \text{for some } \delta > 0.
\]
Then
\[
\iint_{\mathbb{R}^2} |\widetilde{K}_t(x,y)|^2\, dx\, dy < \infty,
\]
so \( \widetilde{L}_t \in \mathcal{C}_2(L^2(\mathbb{R})) \), i.e., Hilbert–Schmidt and hence compact.

\medskip
\noindent\textbf{Step 3: Transfer to Weighted Space.}
Since \( U \) is unitary, we have:
\[
L_t = U^{-1} \widetilde{L}_t U \in \mathcal{K}(H_\Psi).
\]

\medskip
\noindent\textbf{Conclusion.}
Thus, \( L_t \) extends to a compact operator on \( H_\Psi \). Its kernel decays rapidly and defines a Hilbert–Schmidt operator under exponential conjugation. This compactness ensures the discreteness of the spectrum and underpins the Fredholm determinant and Schatten-class analysis developed in later chapters.
\end{proof}


\begin{proposition}[Symmetry of \( L_t \) on Schwartz Core]
\label{prop:symmetry_Lt_Schwartz}
Let \( \Psi \colon \mathbb{R} \to (0,\infty) \) be a smooth, strictly positive weight function satisfying
\[
\Psi(x) \sim e^{\alpha|x|} \quad \text{as } |x| \to \infty,
\]
for some constant \( \alpha > 0 \); that is, there exist constants \( c_1, c_2 > 0 \) such that
\[
c_1 e^{\alpha |x|} \le \Psi(x) \le c_2 e^{\alpha |x|}, \quad \forall x \in \mathbb{R}.
\]

Define the weighted Hilbert space
\[
H_\Psi := L^2(\mathbb{R}, \Psi(x)\, dx),
\]
and suppose \( \phi_t \in \mathcal{S}(\mathbb{R}) \) is real-valued and even. Define the convolution operator
\[
(L_t f)(x) := \int_{\mathbb{R}} \phi_t(x - y)\, f(y)\, dy, \quad f \in \mathcal{S}(\mathbb{R}).
\]

Then \( L_t \) is symmetric on the core domain \( \mathcal{S}(\mathbb{R}) \subset H_\Psi \), that is,
\[
\langle L_t f, g \rangle_{H_\Psi} = \langle f, L_t g \rangle_{H_\Psi}, \quad \forall f, g \in \mathcal{S}(\mathbb{R}),
\]
where the inner product is given by
\[
\langle f, g \rangle_{H_\Psi} := \int_{\mathbb{R}} f(x)\, \overline{g(x)}\, \Psi(x)\, dx.
\]

\medskip
\noindent The kernel \( K_t(x,y) := \phi_t(x - y) \) is real and symmetric. Combined with the density and stability of \( \mathcal{S}(\mathbb{R}) \) under convolution, this guarantees that \( L_t \) is symmetric on its natural core. This property underlies the essential self-adjointness of \( L_t \) and its strong limit \( L_{\mathrm{sym}} \) on the weighted space \( H_\Psi \).
\end{proposition}

\begin{proof}[Proof of \propref{prop:symmetry_Lt_Schwartz}]
Let \( f, g \in \Schwartz \subset H_\Psi \), and define the convolution operator
\[
(L_t f)(x) := \int_{\R} \phi_t(x - y)\, f(y)\, dy,
\]
where \( \phi_t \in \Schwartz \) is real-valued and even.

\medskip
\noindent\textbf{Step 1: Compute the Weighted Inner Product.}
We compute:
\[
\begin{aligned}
\langle L_t f, g \rangle_{H_\Psi}
&= \int_{\R} (L_t f)(x)\, \overline{g(x)}\, \Psi(x)\, dx \\
&= \iint_{\R^2} \phi_t(x - y)\, f(y)\, \overline{g(x)}\, \Psi(x)\, dy\, dx.
\end{aligned}
\]

\medskip
\noindent\textbf{Step 2: Fubini and Symmetry of \( \phi_t \).}
Since \( \phi_t \in \Schwartz \) and \( \Psi(x) \sim e^{\alpha |x|} \), the integrand is absolutely integrable. By Fubini’s theorem:
\[
\langle L_t f, g \rangle_{H_\Psi}
= \int_{\R} f(y) \left( \int_{\R} \phi_t(x - y)\, \overline{g(x)}\, \Psi(x)\, dx \right) dy.
\]

Since \( \phi_t \) is even, \( \phi_t(x - y) = \phi_t(y - x) \), and
\[
(L_t g)(y) = \int_{\R} \phi_t(y - x)\, g(x)\, dx = \int_{\R} \phi_t(x - y)\, g(x)\, dx.
\]
Therefore,
\[
\overline{(L_t g)(y)} = \int_{\R} \phi_t(x - y)\, \overline{g(x)}\, dx.
\]

\medskip
\noindent\textbf{Step 3: Complete the Symmetry Argument.}
Substituting into the outer integral:
\[
\langle L_t f, g \rangle_{H_\Psi}
= \int_{\R} f(y)\, \overline{(L_t g)(y)}\, \Psi(y)\, dy = \langle f, L_t g \rangle_{H_\Psi}.
\]

\medskip
\noindent\textbf{Conclusion.}
This verifies that \( L_t \) is symmetric on the Schwartz core \( \Schwartz \subset H_\Psi \). The symmetry follows directly from the real-valuedness and evenness of \( \phi_t \) and ensures that \( L_t \subset L_t^* \). This property is a foundational step in establishing essential self-adjointness of the strong limit \( L_{\sym} \).
\end{proof}


\begin{proposition}[Self-Adjointness of \( L_t \)]
\label{prop:selfadjointness_Lt}
Let \( H_\Psi := L^2(\R, \Psi(x)\, dx) \) be a weighted Hilbert space, where \( \Psi \colon \R \to (0,\infty) \) is smooth and satisfies
\[
\Psi(x) \sim e^{\alpha|x|} \quad \text{as } |x| \to \infty,
\]
for some \( \alpha > 0 \); that is, there exist constants \( c_1, c_2 > 0 \) such that
\[
c_1 e^{\alpha |x|} \le \Psi(x) \le c_2 e^{\alpha |x|}, \qquad \forall x \in \R.
\]

Let \( \phi_t \in \Schwartz \) be a real-valued, even mollifier, and define the convolution operator
\[
(L_t f)(x) := \int_{\R} \phi_t(x - y)\, f(y)\, dy, \qquad f \in \Schwartz.
\]

Assume:
\begin{itemize}
    \item[\textup{(i)}] \( L_t \) extends to a bounded linear operator on \( H_\Psi \) (see \propref{prop:boundedness_Lt_weighted});
    \item[\textup{(ii)}] \( L_t \) is symmetric on the dense core \( \Schwartz \subset H_\Psi \); that is,
    \[
    \langle L_t f, g \rangle_{H_\Psi} = \langle f, L_t g \rangle_{H_\Psi}, \qquad \forall f, g \in \Schwartz.
    \]
\end{itemize}

Then the bounded operator \( L_t \in \mathcal{B}(H_\Psi) \) is self-adjoint:
\[
L_t^* = L_t.
\]

\paragraph{Consequences.}
As a bounded self-adjoint operator, \( L_t \) admits a spectral resolution via the spectral theorem:
\[
L_t = \int_{\sigma(L_t)} \lambda\, dE_\lambda,
\]
where \( E_\lambda \) is a projection-valued measure. This enables the analytic definition of:
\[
\det\nolimits_\zeta(I - \lambda L_t), \qquad e^{-t L_t^2}, \qquad \text{and} \qquad \zeta_{L_t}(s),
\]
as functions of \( \lambda \) and \( s \), respectively. These constructions underpin the canonical determinant identity and heat kernel asymptotics in subsequent chapters.
\end{proposition}

\begin{proof}[Proof of \propref{prop:selfadjointness_Lt}]
Let \( H_\Psi := L^2(\R, \Psi(x)\, dx) \), where \( \Psi \) is a smooth, strictly positive exponential weight satisfying \( \Psi(x) \sim e^{\alpha|x|} \) as \( |x| \to \infty \), for some \( \alpha > 0 \).

\medskip
\noindent\textbf{Step 1: Boundedness and Symmetry on a Dense Core.}
By \propref{prop:boundedness_Lt_weighted}, the operator
\[
(L_t f)(x) := \int_{\R} \phi_t(x - y)\, f(y)\, dy
\]
extends to a bounded linear operator \( L_t \in \mathcal{B}(H_\Psi) \). By \propref{prop:symmetry_Lt_Schwartz}, \( L_t \) is symmetric on the dense subspace \( \Schwartz \subset H_\Psi \), meaning:
\[
\langle L_t f, g \rangle_{H_\Psi} = \langle f, L_t g \rangle_{H_\Psi}, \qquad \forall f, g \in \Schwartz.
\]

\medskip
\noindent\textbf{Step 2: Self-Adjointness of Bounded Symmetric Operator.}
By a standard result in operator theory (see \cite[Theorem~VI.1]{ReedSimon1980I}), any bounded symmetric operator defined on a dense subspace of a Hilbert space extends uniquely to a self-adjoint operator. Thus,
\[
L_t^* = L_t \qquad \text{on all of } H_\Psi.
\]

\medskip
\noindent\textbf{Conclusion.}
The operator \( L_t \in \mathcal{B}(H_\Psi) \) is self-adjoint. Hence, it admits a spectral resolution via the spectral theorem, supporting zeta-function regularization, semigroup analysis, and Fredholm determinant identities developed in later chapters.
\end{proof}


% --- Core convergence and limit properties ---
\begin{proposition}[Schwartz Core for Canonical Operator]
\label{prop:core_schwartz_density}
Let \( \alpha > \pi \), and let \( L_{\sym} \in \mathcal{C}_1(\HPsi) \) be the canonical convolution operator constructed as the trace-norm limit of mollified convolution operators \( L_t \in \mathcal{B}(\HPsi) \).

Then the Schwartz space \( \Schwartz \subset \HPsi \) is a core for \( L_{\sym} \). That is, for every \( f \in \operatorname{Dom}(L_{\sym}) = \HPsi \), there exists a sequence \( f_n \in \Schwartz \) such that
\[
f_n \to f \quad \text{and} \quad L_{\sym} f_n \to L_{\sym} f \quad \text{in } \HPsi.
\]

\medskip
\noindent
Equivalently, \( \Schwartz \) is dense in \( \HPsi \) with respect to the graph norm of \( L_{\sym} \),
\[
\|f\|_{\mathrm{graph}} := \left( \|f\|_{\HPsi}^2 + \|L_{\sym} f\|_{\HPsi}^2 \right)^{1/2}.
\]
\end{proposition}

\begin{proof}[Proof of \cref{prop:core_schwartz_density}]
Fix \( \alpha > \pi \), and define the exponentially weighted Hilbert space
\[
\HPsi := L^2(\R, \Psi_\alpha(x)\, dx), \qquad \Psi_\alpha(x) := e^{\alpha |x|}.
\]

Let \( L_t \in \mathcal{B}(\HPsi) \) be the mollified convolution operators given by
\[
(L_t f)(x) := \int_{\R} k_t(x - y) f(y)\, dy,
\]
where \( k_t := \FT^{-1}\left[ e^{-t\lambda^2} \, \Xi\left(\tfrac{1}{2} + i\lambda \right) \right] \in \Schwartz(\R) \). The canonical operator \( L_{\sym} := \lim_{t \to 0^+} L_t \in \TC(\HPsi) \) exists in the trace-norm topology by \cref{lem:trace_norm_convergence_Lt_to_Lsym}.

\paragraph{Step 1: Invariance of Schwartz space.}
Since \( k_t \in \Schwartz \) and convolution preserves regularity, each \( L_t \) maps Schwartz functions into Schwartz functions:
\[
L_t(\Schwartz) \subset \Schwartz \cap \HPsi, \qquad \forall t > 0.
\]

\paragraph{Step 2: Density of \( \Schwartz \subset \HPsi \).}
By \cref{lem:density_schwartz_weighted_L2}, the Schwartz space is dense in \( \HPsi \). Thus for any \( f \in \HPsi \) and \( \varepsilon > 0 \), there exists \( \phi \in \Schwartz \) such that
\[
\|f - \phi\|_{\HPsi} < \varepsilon.
\]

\paragraph{Step 3: Strong convergence of \( L_t \) on Schwartz.}
Since \( L_t \to L_{\sym} \) in trace norm (hence in operator norm), we have strong convergence:
\[
\|L_t f - L_{\sym} f\|_{\HPsi} \to 0 \quad \text{for all } f \in \HPsi.
\]
In particular, this holds for all \( f \in \Schwartz \), and each \( L_t f \in \Schwartz \), so
\[
L_{\sym} f = \lim_{t \to 0^+} L_t f \in \HPsi.
\]

\paragraph{Conclusion.}
Given any \( f \in \operatorname{Dom}(L_{\sym}) = \HPsi \), we can choose approximants \( f_n \in \Schwartz \) with
\[
f_n \to f \quad \text{and} \quad L_{\sym} f_n \to L_{\sym} f \quad \text{in } \HPsi,
\]
by combining Step 2 (density) and Step 3 (continuity). Hence, \( \Schwartz(\R) \) is a graph-norm core for \( L_{\sym} \), completing the proof.
\end{proof}


%------------------------------------------------------------------
% --- Canonical spectral operator synthesis theorem ---
\begin{theorem}[Canonical Compact Operator and Spectral Realization]
\label{thm:canonical_operator_realization}
Let \( \phi(\lambda) := \Xi\left( \tfrac{1}{2} + i\lambda \right) \) denote the centered spectral profile of the completed Riemann zeta function, and let \( \alpha > \pi \) be fixed. Define the exponentially weighted Hilbert space
\[
\HPsi := L^2(\R, \Psi_\alpha(x)\, dx), \qquad \text{where } \Psi_\alpha(x) := e^{\alpha |x|}.
\]

Construct the mollified convolution operators
\[
L_t f(x) := \int_{\R} \FT^{-1}\left[ e^{-t\lambda^2} \phi(\lambda) \right](x - y) f(y)\, dy,
\]
which are real, symmetric, compact operators in \( \mathcal{C}_1(\HPsi) \). Then the following hold:

\begin{itemize}
    \item[\textup{(i)}] The trace-norm limit
    \[
    L_{\sym} := \lim_{t \to 0^+} L_t \in \mathcal{C}_1(\HPsi)
    \]
    exists, is self-adjoint, and compact.

    \item[\textup{(ii)}] The determinant
    \[
    \det\nolimits_\zeta(I - \lambda L_{\sym}) := \prod_n (1 - \lambda \lambda_n) e^{\lambda \lambda_n} = \frac{\Xi\left( \tfrac{1}{2} + i\lambda \right)}{\Xi\left( \tfrac{1}{2} \right)}
    \]
    is entire of order one and encodes the nontrivial zero set of \( \zeta(s) \) via the spectral realization
    \[
    \Spec(L_{\sym}) = \left\{ \mu_\rho := \tfrac{1}{i}(\rho - \tfrac{1}{2}) \,\middle|\, \zeta(\rho) = 0,\; \Re(\rho) = \tfrac{1}{2} \right\}.
    \]

    \item[\textup{(iii)}] The space \( \Schwartz \subset \HPsi \) is a core for \( L_{\sym} \), and the kernel of \( L_{\sym} \) is symmetric, real, and exponentially decaying off the diagonal.
\end{itemize}

This operator \( L_{\sym} \) is the analytic centerpiece of the spectral determinant identity developed in Chapters~\ref{sec:determinant_identity} through \ref{sec:spectral_implications}, and governs the spectral encoding of the Riemann Hypothesis via its real spectrum.
\end{theorem}

\begin{proof}[Proof of \thmref{thm:canonical_operator_realization}]
\textbf{(i) Existence and Trace-Norm Convergence.}
Let \( \phi(\lambda) := \Xi(\tfrac{1}{2} + i\lambda) \) and define the mollified profile
\[
\phi_t(\lambda) := e^{-t\lambda^2} \phi(\lambda), \quad t > 0.
\]
By \lemref{lem:xi_growth_bound}, we have \( \phi \in \PW{\pi} \), hence each \( \phi_t \in \Schwartz \). Define the inverse Fourier transform \( k_t := \FT^{-1}[\phi_t] \in \Schwartz \), and the associated convolution kernel \( K_t(x,y) := k_t(x - y) \).

By \lemref{lem:uniform_L1_conjugated_kernel}, the conjugated kernels \( \widetilde{K}_t(x,y) := K_t(x,y) \Psi_\alpha(x) \Psi_\alpha(y) \) lie uniformly in \( L^1(\R^2) \) for \( \alpha > \pi \), and the associated operators \( L_t \in \TC(\HPsi) \) satisfy
\[
\sup_{0 < t \le 1} \| L_t \|_{\TC(\HPsi)} < \infty.
\]
Therefore, \( \{L_t\} \) forms a norm-bounded family in the trace-class ideal. Since \( \phi_t \to \phi \) in \( L^1(\R) \), we obtain convergence in the trace norm:
\[
\lim_{t \to 0^+} \| L_t - \Lsym \|_{\TC(\HPsi)} = 0
\]
for some limit \( \Lsym \in \TC(\HPsi) \). By \propref{prop:selfadjointness_Lt} and \propref{prop:core_schwartz_density}, the limit operator \( \Lsym \) is self-adjoint and compact.

\medskip
\noindent\textbf{(ii) Determinant Identity and Spectral Encoding.}
Since \( \Lsym \in \TC(\HPsi) \), the Carleman--Fredholm determinant is defined via the standard trace formula:
\[
\detz(I - \lambda \Lsym) := \prod_n (1 - \lambda \lambda_n) e^{\lambda \lambda_n}.
\]
By continuity of the determinant under trace-norm limits~\cite[Ch.~4]{Simon2005TraceIdeals}, and by the exponential decay of \( \phi_t \), we have:
\[
\detz(I - \lambda L_t) \to \detz(I - \lambda \Lsym) \quad \text{uniformly on compact subsets of } \lambda.
\]
By construction of \( \phi \) from \( \Xi \), we recover the identity:
\[
\detz(I - \lambda \Lsym) = \frac{\Xi(\tfrac{1}{2} + i\lambda)}{\Xi(\tfrac{1}{2})}.
\]
This matches the Hadamard product of \( \Xi \), and encodes the spectrum via
\[
\Spec(\Lsym) = \left\{ \tfrac{1}{i}(\rho - \tfrac{1}{2}) \;\middle|\; \zeta(\rho) = 0 \right\},
\]
with spectral symmetry implied by the evenness of \( \phi \) and kernel symmetry (\lemref{lem:kernel_symmetry}).

\medskip
\noindent\textbf{(iii) Schwartz Core and Kernel Properties.}
By \lemref{lem:density_schwartz_weighted_L2} and \propref{prop:core_schwartz_density}, the Schwartz space \( \Schwartz \subset \HPsi \) is a graph-norm core for \( \Lsym \), satisfying
\[
f_n \to f \quad \text{and} \quad \Lsym f_n \to \Lsym f \quad \text{in } \HPsi.
\]
The convolution kernel \( k := \FT^{-1}[\phi] \) is real, even, and exponentially decaying by \lemref{lem:decay_inverse_fourier_xi}, ensuring that \( K(x,y) := k(x - y) \) defines a real symmetric integral operator with \( K \in L^1(\Psi_\alpha^{\otimes 2}) \cap L^2(\Psi_\alpha^{\otimes 2}) \).

\medskip
\noindent\textbf{Conclusion.}
The operator \( \Lsym \in \TC(\HPsi) \) is self-adjoint, compact, and canonically realizes \( \Xi \) via its Fredholm determinant. The spectral data encoded by \( \Xi \) correspond bijectively to the spectrum of \( \Lsym \), providing the analytic foundation for the determinant identity and spectral implications developed in subsequent chapters.
\end{proof}


%------------------------------------------------------------------
\subsection*{Summary}
\label{sec:foundations_summary}

\textbf{Operator-Theoretic Foundations}
\begin{itemize}
  \item \defref{def:compact_operator} — Compact operators: norm limits of finite-rank maps with discrete spectrum.
  \item \defref{def:trace_class_operator}, \defref{def:trace_norm} — Trace-class operators \( T \in \TC(H) \) with finite trace norm \( \|T\|_{\Tr} := \Tr(|T|) \); Banach completeness and unitary invariance.
  \item \defref{def:selfadjoint_operator} — Self-adjointness as maximal symmetry enabling spectral calculus and semigroup generation.
\end{itemize}

\textbf{Weighted Spaces and Function Classes}
\begin{itemize}
  \item \defref{def:exponential_weight}, \defref{def:weighted_schwartz_space} — The space \( \HPsi = L^2(\R, e^{\alpha|x|}\,dx) \), with \( \Schwartz(\R) \subset \HPsi \) a dense core.
  \item \lemref{lem:density_schwartz_weighted_L2} — Density of \( \Schwartz \subset \HPsi \) in norm and graph topology.
  \item \remref{rem:sobolev_core_reference} — Alternate justification: \( \Schwartz \hookrightarrow H^s_\alpha \hookrightarrow \HPsi \) via Sobolev embeddings.
\end{itemize}

\textbf{Analytic and Spectral Estimates}
\begin{itemize}
  \item \lemref{lem:xi_growth_bound}, \lemref{lem:weighted_L1_inverse_FT_xi} — The profile \( \Xi(\tfrac{1}{2} + i\lambda) \in \PW{\pi} \), with inverse transform in \( L^1(\R, \Psi_\alpha^{-1}) \).
  \item \lemref{lem:decay_mollified_kernel}, \lemref{lem:L1_integrability_conjugated_kernel} — Mollifiers \( k_t \in \Schwartz \), conjugated kernels integrable.
  \item \lemref{lem:uniform_L1_conjugated_kernel}, \lemref{lem:trace_class_via_weighted_L1} — Trace norm convergence \( \|L_t - \Lsym\|_{\TC} \to 0 \) and Simon’s trace-class inclusion criterion.
  \item \lemref{lem:trace_class_conjugated_kernel}, \lemref{lem:trace_class_failure_alpha_leq_pi}, \propref{prop:trace_class_sharpness} — Trace-class fails for \( \alpha \le \pi \): sharp exponential decay threshold.
  \item \lemref{lem:unitary_conjugation_trace_class} — Trace norm preserved under unitary weight conjugation.
\end{itemize}

\textbf{Operator Properties of \texorpdfstring{\( L_t \)}{Lt}}
\begin{itemize}
  \item \propref{prop:boundedness_Lt_weighted}, \propref{prop:compactness_Lt} — Boundedness and compactness of \( L_t \) via mollified kernel structure.
  \item \propref{prop:symmetry_Lt_Schwartz}, \propref{prop:selfadjointness_Lt} — \( L_t \) is symmetric on \( \Schwartz \) and extends to a self-adjoint operator.
  \item \propref{prop:core_schwartz_density} — \( \Schwartz \) is a core for the limit operator \( \Lsym \).
\end{itemize}

\textbf{Canonical Operator Realization}
\begin{itemize}
  \item \thmref{thm:canonical_operator_realization} — Convergence \( L_t \to \Lsym \in \TC(\HPsi) \); defines the canonical compact self-adjoint operator realizing the spectral determinant.
\end{itemize}

\paragraph{Chapter Closure.}
This chapter establishes the analytic and operator-theoretic base for all that follows. The canonical convolution operator \( \Lsym \in \TC(\HPsi) \) is defined as the trace-norm limit of mollified Fourier convolution operators \( L_t \). Its construction relies on Paley--Wiener estimates, exponential decay, Sobolev density, and trace-class embedding theorems. The determinant identity
\[
\detz(I - \lambda \Lsym)
= \frac{\Xi\left(\tfrac{1}{2} + i\lambda \right)}{\Xi\left(\tfrac{1}{2} \right)}
\]
is proven in \secref{sec:determinant_identity}, resting entirely on this analytic groundwork.


\section{Construction of the Canonical Spectral Operator}
\label{sec:operator_construction}

\subsection*{Introduction}

This chapter establishes the analytic infrastructure for defining and analyzing the canonical compact operator \( L_{\mathrm{sym}} \), which realizes the completed Riemann zeta function \( \Xi(s) \) via its Fredholm determinant. The primary goal is to verify that mollified convolution operators associated with the inverse Fourier transform of \( \Xi \) are compact, trace class, and converge in trace norm to a self-adjoint limit operator \( L_{\mathrm{sym}} \in \mathcal{C}_1(H_{\Psi_\alpha}) \).

The constructions here verify:

\begin{itemize}
    \item Schatten-class properties of Hilbert–Schmidt and trace-class operators, following \cite[Ch.~4]{Simon2005TraceIdeals} and \cite[Ch.~VI]{ReedSimon1980I}, including the completeness of \( \mathcal{C}_1 \) and the trace-norm topology.
    
    \item Sufficient conditions for compactness and self-adjointness of integral operators with symmetric Hermitian kernels, using distributional domains and exponential conjugation.
    
    \item The structure of the weighted Schwartz space \( \mathcal{S}_\alpha(\mathbb{R}) \subset L^2(\mathbb{R}, e^{\alpha |x|}\, dx) \), for \( \alpha > \pi \), ensuring Fourier duality and decay control for entire functions of exponential type \( \pi \) \cite{Levin1996EntireLectures}.
    
    \item Uniform kernel bounds and mollifier admissibility for defining the regularized heat operators \( e^{-t L_t^2} \), together with analytic kernel estimates necessary for short-time trace control and Tauberian convergence.
\end{itemize}

These ingredients culminate in the construction of mollified convolution operators \( L_t \), and in the verification of trace-norm convergence
\[
L_t \to L_{\mathrm{sym}} \in \mathcal{C}_1(H_{\Psi_\alpha}) \quad \text{as } t \to 0^+.
\]
This limit defines the canonical spectral operator underlying the determinant identity
\[
\det\nolimits_{\zeta}(I - \lambda L_{\mathrm{sym}}) = \frac{\Xi\left(\tfrac{1}{2} + i\lambda\right)}{\Xi\left(\tfrac{1}{2}\right)},
\]
which is rigorously established without assuming RH.

\medskip

The analytic architecture developed here underpins all subsequent spectral and determinant identities.
See Appendix~\ref{app:dependency-graph} for a visual DAG linking these foundational tools to the modular proof of RH.


%------------------------------------------------------------------
\subsection{Definitions}

% φ(λ) := Ξ(1/2 + iλ), real-valued profile of exponential type π
\begin{definition}[Canonical Fourier Profile]
\label{def:canonical_fourier_profile}
Let \( \Xi(s) \) denote the completed Riemann zeta function, defined by
\[
\Xi(s) := \tfrac{1}{2} s(s-1) \pi^{-s/2} \Gamma\left(\tfrac{s}{2}\right) \zeta(s),
\]
which extends to an entire function of order one and exponential type \( \pi \), and satisfies the functional equation \( \Xi(s) = \Xi(1 - s) \).

Define the canonical Fourier profile \( \phi : \mathbb{R} \to \mathbb{R} \) by restriction:
\[
\phi(\lambda) := \Xi\left( \tfrac{1}{2} + i\lambda \right).
\]

Then \( \phi \) satisfies:
\begin{enumerate}
  \item \textbf{Entirety and Exponential Type:} \( \phi \) is the restriction of an entire function of exponential type \( \pi \) and order one.

  \item \textbf{Evenness and Real-Valuedness:}
  \[
  \phi(-\lambda) = \phi(\lambda), \quad \phi(\lambda) \in \mathbb{R}, \qquad \forall \lambda \in \mathbb{R}.
  \]

  \item \textbf{Exponential Growth Bound:} There exists a constant \( A_1 > 0 \) such that
  \[
  |\phi(\lambda)| \le A_1\, e^{\frac{\pi}{2} |\lambda|}, \qquad \forall \lambda \in \mathbb{R},
  \]
  based on Hadamard factorization and Stirling asymptotics for \( \Gamma(s/2) \zeta(s) \) on vertical lines.

  \item \textbf{Paley--Wiener Membership:}
\( \phi \in PW_\pi(\mathbb{R}) \cap \mathcal{S}'(\mathbb{R}) \),\\
the Paley--Wiener space of entire functions of exponential type \( \pi \)
whose inverse Fourier transforms are supported in \( [-\pi, \pi] \).

  \item \textbf{Inverse Fourier Decay:} The inverse Fourier transform
  \[
  \phi^\vee(x) := \widehat{\phi}(x) := \frac{1}{2\pi} \int_{\mathbb{R}} e^{i \lambda x} \phi(\lambda)\, d\lambda
  \]
  lies in \( L^1(\mathbb{R}, e^{-\alpha |x|} dx) \) for any \( \alpha > \pi \), and is real-valued and even.
\end{enumerate}

\noindent
These properties ensure that convolution operators with kernel \( \phi^\vee \) define bounded, compact, and self-adjoint operators on exponentially weighted Hilbert spaces \( H_{\Psi_\alpha} := L^2(\mathbb{R}, e^{\alpha |x|} dx) \), and that \( \phi \) serves as a canonical spectral analytic profile for trace-norm and determinant constructions.
\end{definition}
% 

% Weighted Hilbert space H_{\Psi_\alpha}, α > π
\begin{definition}[Exponentially Weighted Hilbert Space]
\label{def:weighted-fourier-space}
Fix a weight parameter \( \alpha > \pi \), and define the exponential weight function
\[
\Psi_\alpha(x) := e^{\alpha |x|}, \quad x \in \mathbb{R}.
\]
The associated exponentially weighted Hilbert space is
\[
H_{\Psi_\alpha} := L^2(\mathbb{R}, \Psi_\alpha(x)\, dx),
\]
consisting of all measurable functions \( f \colon \mathbb{R} \to \mathbb{C} \) such that
\[
\|f\|_{H_{\Psi_\alpha}}^2 := \int_{\mathbb{R}} |f(x)|^2 \Psi_\alpha(x)\, dx < \infty.
\]

This space is a separable, reflexive Hilbert space equipped with the inner product
\[
\langle f, g \rangle_{H_{\Psi_\alpha}} := \int_{\mathbb{R}} f(x)\, \overline{g(x)}\, \Psi_\alpha(x)\, dx,
\]
which is linear in the first argument and conjugate-linear in the second.

\medskip

The condition \( \alpha > \pi \) is critical: it ensures that inverse Fourier transforms of profiles of exponential type at most \( \pi \), such as
\[
\phi(\lambda) := \Xi\left( \tfrac{1}{2} + i\lambda \right),
\]
belong to \( L^1(\mathbb{R}, \Psi_\alpha^{-1}(x)\, dx) \). This exponential integrability guarantees that convolution against \( \phi^\vee \) defines trace-class operators on \( H_{\Psi_\alpha} \).

\medskip

Functions in \( H_{\Psi_\alpha} \) exhibit exponential decay:
\[
|f(x)| \lesssim e^{-\alpha |x|} \quad \text{as } |x| \to \infty,
\]
in the sense of weighted envelope norm control. The Schwartz space \( \mathcal{S}(\mathbb{R}) \subset H_{\Psi_\alpha} \) is dense and serves as a canonical core domain for the definition and analysis of convolution operators with exponentially decaying kernels.
\end{definition}


% Uα: conjugation between H_{Ψ_α} and L²
\begin{definition}[Unitary Conjugation Operator]
\label{def:unitary-conjugation-operator}
Fix any \( \alpha > \pi \), and define the exponential weight
\[
\Psi_\alpha(x) := e^{\alpha |x|}, \qquad x \in \mathbb{R}.
\]
Let
\[
H_{\Psi_\alpha} := L^2(\mathbb{R}, \Psi_\alpha(x)\, dx)
\]
denote the corresponding exponentially weighted Hilbert space, and let \( L^2(\mathbb{R}) \) denote the standard (flat) Hilbert space with Lebesgue measure.

Define the unitary conjugation operator
\[
U_\alpha \colon H_{\Psi_\alpha} \to L^2(\mathbb{R}), \qquad (U_\alpha f)(x) := \Psi_\alpha(x)^{1/2} f(x) = e^{\frac{\alpha}{2}|x|} f(x),
\]
with inverse
\[
U_\alpha^{-1}(h)(x) := \Psi_\alpha(x)^{-1/2} h(x) = e^{-\frac{\alpha}{2}|x|} h(x).
\]

Then:
\begin{itemize}
    \item \( U_\alpha \) is a unitary isomorphism of Hilbert spaces:
    \[
    \langle f, g \rangle_{H_{\Psi_\alpha}} = \langle U_\alpha f, U_\alpha g \rangle_{L^2(\mathbb{R})}, \quad \forall f, g \in H_{\Psi_\alpha}.
    \]

    \item The map
    \[
    T \mapsto \widetilde{T} := U_\alpha T U_\alpha^{-1}
    \]
    defines a unitary equivalence between operators on \( H_{\Psi_\alpha} \) and operators on \( L^2(\mathbb{R}) \), preserving boundedness, self-adjointness, compactness, and trace-class membership.

    \item If \( K(x,y) \in C^\infty(\mathbb{R}^2) \) defines an integral operator \( T \) on \( H_{\Psi_\alpha} \), then the conjugated operator \( \widetilde{T} := U_\alpha T U_\alpha^{-1} \) acts on \( L^2(\mathbb{R}) \) with kernel
    \[
    \widetilde{K}(x,y) := \frac{K(x,y)}{\sqrt{\Psi_\alpha(x)\Psi_\alpha(y)}}.
    \]
    This transformation underpins Simon’s trace-class criterion and is central to decay estimates throughout the construction of the canonical operator.
\end{itemize}
\end{definition}


% Mollified profiles φ_t(λ) = e^{-tλ²}φ(λ)
\begin{definition}[Mollified Fourier Profile]
\label{def:mollified-fourier-profile}
Let \( \phi(\lambda) := \Xi\left( \tfrac{1}{2} + i\lambda \right) \) be the canonical Fourier profile (see Definition~\ref{def:canonical-fourier-profile}). For each \( t > 0 \), define the mollified Fourier profile:
\[
\varphi_t(\lambda) := e^{-t\lambda^2} \cdot \phi(\lambda), \qquad \lambda \in \mathbb{R}.
\]
Here, the Gaussian damping factor \( e^{-t\lambda^2} \) serves as a mollifier: it improves decay, regularity, and integrability of the spectral profile.

Then \( \varphi_t \) satisfies:
\begin{enumerate}
  \item \textbf{Schwartz Regularity:} Since \( \phi(\lambda) \) is entire of exponential type \( \pi \) and polynomially bounded, the mollified profile satisfies:
  \[
  \varphi_t \in \mathcal{S}(\mathbb{R}), \qquad \forall t > 0.
  \]

  \item \textbf{Evenness and Real-Valuedness:}
  \[
  \varphi_t(-\lambda) = \varphi_t(\lambda), \quad \varphi_t(\lambda) \in \mathbb{R}, \qquad \forall \lambda \in \mathbb{R}.
  \]

  \item \textbf{Integrability and Kernel Smoothness:} We have \( \varphi_t \in L^1(\mathbb{R}) \cap L^2(\mathbb{R}) \), and its inverse Fourier transform
  \[
  k_t(x) := \widehat{\varphi_t}(x) := \frac{1}{2\pi} \int_{\mathbb{R}} e^{i\lambda x} \varphi_t(\lambda)\, d\lambda
  \]
  lies in \( \mathcal{S}(\mathbb{R}) \). In particular, \( k_t \) is smooth, real-valued, even, and rapidly decaying.

  \item \textbf{Weighted Decay and Operator Admissibility:} For any \( \alpha > \pi \), we have:
  \[
  k_t \in L^1(\mathbb{R}, e^{\alpha|x|} dx),
  \]
  so the kernel \( k_t(x - y) \) defines a trace-class convolution operator on \( H_{\Psi_\alpha} \).
\end{enumerate}

\noindent
Thus, the mollified profiles \( \varphi_t \) form a regularizing family of spectral densities whose convolution kernels \( k_t \) yield trace-class convolution operators. They bridge entire function theory and spectral analysis via explicit kernel regularization.
\end{definition}


% Definition of L_t and trace-norm limit L_sym
\begin{definition}[Convolution Operators \( L_t \) and Canonical Limit \( L_{\mathrm{sym}} \)]
\label{def:convolution-operators-Lt-Lsym}
Let \( \phi(\lambda) := \Xi\left( \tfrac{1}{2} + i\lambda \right) \) be the canonical Fourier profile, and define mollified spectral profiles for \( t > 0 \) as
\[
\phi_t(\lambda) := e^{-t\lambda^2} \phi(\lambda).
\]

Define the mollified inverse Fourier kernels:
\[
k_t(x) := \frac{1}{2\pi} \int_{\mathbb{R}} e^{i x \lambda} \phi_t(\lambda)\, d\lambda \in \mathcal{S}(\mathbb{R}),
\]
and the translation-invariant integral kernels:
\[
K_t(x,y) := k_t(x - y).
\]
Then \( K_t \in \mathcal{S}(\mathbb{R}^2) \) is real-valued, symmetric \( K_t(x,y) = K_t(y,x) \), and rapidly decaying in both variables.

Define the convolution operator on the Schwartz core \( \mathcal{S}(\mathbb{R}) \subset H_{\Psi_\alpha} \) by
\[
(L_t f)(x) := \int_{\mathbb{R}} K_t(x,y)\, f(y)\, dy = \int_{\mathbb{R}} k_t(x - y) f(y)\, dy.
\]

Then each \( L_t \) extends uniquely to a compact, self-adjoint operator on the exponentially weighted Hilbert space
\[
H_{\Psi_\alpha} := L^2(\mathbb{R}, e^{\alpha|x|} dx), \quad \text{with } \alpha > \pi.
\]
Moreover,
\[
L_t \in \mathcal{C}_1(H_{\Psi_\alpha}) \cap \mathcal{C}_2(H_{\Psi_\alpha}), \qquad L_t = L_t^*.
\]
The decay of \( k_t \in \mathcal{S}(\mathbb{R}) \) ensures that \( K_t \in L^1(\mathbb{R}^2, \Psi_\alpha(x)\Psi_\alpha(y)\, dx\,dy) \), satisfying Simon’s trace-class criterion \cite[Ch.~4]{Simon2005TraceIdeals}.

\medskip
\noindent
Define the canonical convolution operator as the trace-norm limit:
\[
L_{\mathrm{sym}} := \lim_{t \to 0^+} L_t \quad \text{in } \mathcal{C}_1(H_{\Psi_\alpha}),
\]
i.e.,
\[
\| L_t - L_{\mathrm{sym}} \|_{\mathcal{C}_1} \to 0 \quad \text{as } t \to 0^+.
\]
This limit exists and is unique due to uniform trace-norm bounds and the mollified profiles' Schwartz regularity. See \lemref{lem:trace-norm-convergence-Lt-to-Lsym} and \lemref{lem:mollifier-independence-kernel-limit} for rigorous justification and mollifier-independence.

\medskip
\noindent
The operator \( L_{\mathrm{sym}} \in \mathcal{C}_1(H_{\Psi_\alpha}) \) is compact, self-adjoint, and defined by convolution against the inverse Fourier transform \( \phi^\vee(x) := \widehat{\phi}(x) \). It inherits its analytic symmetry from \( \Xi(s) \) and encodes the nontrivial zeros of \( \zeta(s) \) through its spectrum. The associated Fredholm determinant satisfies the identity:
\[
\det\nolimits_{\zeta}(I - \lambda L_{\mathrm{sym}}) = \frac{\Xi\left( \tfrac{1}{2} + i\lambda \right)}{\Xi\left( \tfrac{1}{2} \right)},
\]
as shown in Chapter~\ref{sec:determinant-identity}.
\end{definition}
% 

%------------------------------------------------------------------
\subsection{Mollified Operator Construction}

% Fourier decay and S(R) structure of mollified φ_t
\begin{lemma}[Decay of Mollified Fourier Profiles]
\label{lem:mollified-profile-decay}
Let \( \phi(\lambda) := \Xi\left( \tfrac{1}{2} + i\lambda \right) \) be the canonical Fourier profile (see Definition~\ref{def:canonical-fourier-profile}), and define for \( t > 0 \):
\[
\phi_t(\lambda) := e^{-t\lambda^2} \cdot \phi(\lambda).
\]

Then the following hold:

\begin{enumerate}
    \item[\textnormal{(i)}] \textbf{Gaussian Envelope and Schwartz Regularity:} Since \( \phi(\lambda) \) is of exponential type \( \pi \), there exists \( C > 0 \) such that
    \[
    |\phi(\lambda)| \le C\, e^{\frac{\pi}{2}|\lambda|}, \quad \forall \lambda \in \mathbb{R}.
    \]
    Therefore, for fixed \( t > 0 \), there exist constants \( C_t, a_t > 0 \) such that
    \[
    |\phi_t(\lambda)| \le C_t\, e^{-a_t \lambda^2}.
    \]
    Hence \( \phi_t \in \mathcal{S}(\mathbb{R}) \), and its inverse Fourier transform
    \[
    k_t(x) := \frac{1}{2\pi} \int_{\mathbb{R}} e^{i\lambda x} \phi_t(\lambda)\, d\lambda
    \]
    lies in \( \mathcal{S}(\mathbb{R}) \), and is real-valued, smooth, even, and rapidly decreasing.

    \item[\textnormal{(ii)}] \textbf{Integrability and Weighted Decay:} For every \( t > 0 \), \( \phi_t \in L^1(\mathbb{R}) \cap L^2(\mathbb{R}) \), and \( k_t \in L^1(\mathbb{R}, e^{\alpha |x|} dx) \) for all \( \alpha > \pi \).

    \item[\textnormal{(iii)}] \textbf{Convergence to Canonical Profile:} As \( t \to 0^+ \),
    \[
    \phi_t(\lambda) \to \phi(\lambda) \quad \text{pointwise for all } \lambda \in \mathbb{R},
    \]
    and
    \[
    \phi_t \to \phi \quad \text{in } L^1_{\mathrm{loc}}(\mathbb{R}).
    \]
    For any compact interval \( I \subset \mathbb{R} \),
    \[
    \int_I |\phi_t(\lambda) - \phi(\lambda)|\, d\lambda \to 0 \quad \text{as } t \to 0^+.
    \]

    \item[\textnormal{(iv)}] \textbf{Operator-Theoretic Consequence:} The Schwartz decay of \( \phi_t \) and exponential integrability of \( k_t \) imply:
    \[
    L_t f(x) := \int_{\mathbb{R}} k_t(x - y) f(y)\, dy
    \]
    defines a bounded, self-adjoint, trace-class operator on \( H_{\Psi_\alpha} \), for any \( \alpha > \pi \).
\end{enumerate}
\end{lemma}
% 
\begin{proof}[Proof of Lemma~\ref{lem:mollified_profile_decay}]
Let \( \phi(\lambda) := \Xi\left( \tfrac{1}{2} + i\lambda \right) \), and define the mollified profile
\[
\phi_t(\lambda) := e^{-t\lambda^2} \cdot \phi(\lambda), \quad t > 0.
\]

\medskip
\noindent\textbf{(i) Gaussian Envelope and Schwartz Regularity.}
By Lemma~\ref{lem:xi_growth_bound}, \( \phi \in PW_\pi(\mathbb{R}) \) and satisfies
\[
|\phi(\lambda)| \le C\, e^{\frac{\pi}{2} |\lambda|}, \quad \forall \lambda \in \mathbb{R}.
\]
Hence,
\[
|\phi_t(\lambda)| \le C\, e^{-t\lambda^2 + \frac{\pi}{2} |\lambda|}.
\]
Completing the square:
\[
-t\lambda^2 + \frac{\pi}{2}|\lambda| \le -\tfrac{t}{2}\lambda^2 + \tfrac{\pi^2}{8t},
\]
so
\[
|\phi_t(\lambda)| \le C_t\, e^{-a_t \lambda^2}, \quad \text{with } C_t := C e^{\pi^2/8t}, \quad a_t := t/2.
\]
Thus, \( \phi_t \in \mathcal{S}(\mathbb{R}) \), and its inverse Fourier transform
\[
k_t(x) := \widehat{\phi_t}(x) := \frac{1}{2\pi} \int_{\mathbb{R}} e^{i\lambda x} \phi_t(\lambda)\, d\lambda
\]
also lies in \( \mathcal{S}(\mathbb{R}) \); hence \( k_t \) is smooth, real-valued, even, and rapidly decreasing.

\medskip
\noindent\textbf{(ii) Integrability and Weighted Decay.}
Since \( \phi_t \in \mathcal{S}(\mathbb{R}) \), we have \( \phi_t \in L^1(\mathbb{R}) \cap L^2(\mathbb{R}) \), and therefore \( k_t \in \mathcal{S}(\mathbb{R}) \subset L^1(\mathbb{R}, e^{\alpha |x|} dx) \) for all \( \alpha > \pi \).

\medskip
\noindent\textbf{(iii) Pointwise and Local \( L^1 \) Convergence.}
As \( t \to 0^+ \), we have pointwise:
\[
\phi_t(\lambda) \to \phi(\lambda), \quad \forall \lambda \in \mathbb{R}.
\]
Fix a compact interval \( I \subset \mathbb{R} \). Since \( \phi \in C(\mathbb{R}) \), the family \( \{\phi_t\} \) is dominated on \( I \), and for any \( \lambda \in I \),
\[
|\phi_t(\lambda) - \phi(\lambda)| \le |\phi(\lambda)| \cdot |1 - e^{-t\lambda^2}| \to 0.
\]
Thus, by dominated convergence,
\[
\int_I |\phi_t(\lambda) - \phi(\lambda)|\, d\lambda \to 0.
\]

\medskip
\noindent\textbf{(iv) Operator-Theoretic Consequence.}
The inverse transforms \( k_t \in \mathcal{S}(\mathbb{R}) \cap L^1(\mathbb{R}, e^{\alpha |x|} dx) \) define convolution operators
\[
L_t f(x) := \int_{\mathbb{R}} k_t(x - y) f(y)\, dy
\]
that are bounded, self-adjoint, and trace class on \( H_{\Psi_\alpha} \) by Simon’s criterion.

\medskip
\noindent\textbf{Conclusion.}
The mollified spectral profiles \( \phi_t \in \mathcal{S}(\mathbb{R}) \) form an admissible family for regularizing the canonical profile \( \phi \), yielding trace-class convolution operators \( L_t \in \mathcal{C}_1(H_{\Psi_\alpha}) \) and strong analytic convergence toward \( L_{\mathrm{sym}} \).
\end{proof}


% L_t ∈ C₁(H_{\Psi_α}) by Simon's criterion
\begin{lemma}[Trace-Class Property of \( L_t \)]
\label{lem:trace_class_Lt}
Let \( \varphi_t \in \mathcal{S}(\mathbb{R}) \) be the mollified spectral profile, and define the translation-invariant kernel
\[
K_t(x, y) := \frac{1}{2\pi} \int_{\mathbb{R}} e^{i(x - y)\lambda} \varphi_t(\lambda)\, d\lambda.
\]

Fix any \( \alpha > \pi \), and define the exponentially weighted Hilbert space
\[
H_{\Psi_\alpha} := L^2(\mathbb{R}, e^{\alpha |x|} dx).
\]

Let \( U_\alpha : H_{\Psi_\alpha} \to L^2(\mathbb{R}) \) denote the unitary conjugation operator defined by
\[
(U_\alpha f)(x) := e^{\frac{\alpha}{2}|x|} f(x),
\]
and consider the conjugated kernel
\[
\widetilde{K}_t(x, y) := \frac{K_t(x, y)}{\sqrt{\Psi_\alpha(x)\Psi_\alpha(y)}}
= \frac{K_t(x, y)}{\sqrt{e^{\alpha |x|} e^{\alpha |y|}}}.
\]

Suppose the conjugated kernel satisfies the uniform bound:
\[
\sup_{0 < t \le 1} \| \widetilde{K}_t \|_{L^1(\mathbb{R}^2)} < \infty.
\]

Then the integral operator
\[
(L_t f)(x) := \int_{\mathbb{R}} K_t(x, y)\, f(y)\, dy
\]
defines a trace-class operator on \( H_{\Psi_\alpha} \):
\[
L_t \in \mathcal{C}_1(H_{\Psi_\alpha}),
\]
with trace norm controlled by the flat-space norm of the conjugated kernel:
\[
\| L_t \|_{\mathcal{C}_1(H_{\Psi_\alpha})}
= \| U_\alpha L_t U_\alpha^{-1} \|_{\mathcal{C}_1(L^2(\mathbb{R}))}
\le \| \widetilde{K}_t \|_{L^1(\mathbb{R}^2)}.
\]

This follows from Simon’s trace-class kernel criterion~\cite[Thm.~4.2]{Simon2005TraceIdeals}, applied to the conjugated operator \( \widetilde{L}_t := U_\alpha L_t U_\alpha^{-1} \in \mathcal{C}_1(L^2(\mathbb{R})) \); see also \lemref{lem:unitary_conjugation_trace_class}.
\end{lemma}

\begin{proof}[Proof of Lemma~\ref{lem:trace_class_Lt}]
Let \( H_{\Psi_\alpha} := L^2(\mathbb{R}, \Psi_\alpha(x)\, dx) \), with \( \Psi_\alpha(x) := e^{\alpha |x|} \), and fix \( \alpha > \pi \). Let \( \phi_t \in \mathcal{S}(\mathbb{R}) \) be the mollified profile, and define the convolution kernel
\[
K_t(x,y) := \frac{1}{2\pi} \int_{\mathbb{R}} e^{i(x - y)\lambda} \phi_t(\lambda)\, d\lambda = k_t(x - y),
\]
where \( k_t \in \mathcal{S}(\mathbb{R}) \) by Lemma~\ref{lem:mollified_profile_decay}.

\medskip
\noindent\textbf{Step 1: Unitary Conjugation.}
Define the unitary map
\[
U \colon H_{\Psi_\alpha} \to L^2(\mathbb{R}), \quad (Uf)(x) := \Psi_\alpha(x)^{1/2} f(x),
\]
with inverse \( U^{-1}(g)(x) := \Psi_\alpha(x)^{-1/2} g(x) \). Let \( \widetilde{L}_t := U L_t U^{-1} \) be the conjugated operator, which acts on \( L^2(\mathbb{R}) \) via kernel
\[
\widetilde{K}_t(x,y) := \frac{K_t(x,y)}{\sqrt{\Psi_\alpha(x)\Psi_\alpha(y)}}.
\]

\medskip
\noindent\textbf{Step 2: Trace-Class Criterion.}
By assumption,
\[
\sup_{t \in (0,1]} \| \widetilde{K}_t \|_{L^1(\mathbb{R}^2)} < \infty.
\]
By Simon’s trace-class kernel criterion \cite[Thm.~4.2]{Simon2005TraceIdeals}, \( \widetilde{L}_t \in \mathcal{C}_1(L^2(\mathbb{R})) \), with
\[
\| \widetilde{L}_t \|_{\mathcal{C}_1} \le \| \widetilde{K}_t \|_{L^1(\mathbb{R}^2)}.
\]

\medskip
\noindent\textbf{Step 3: Transfer to Weighted Space.}
Since \( L_t = U^{-1} \widetilde{L}_t U \) and \( U \) is unitary, it follows that
\[
L_t \in \mathcal{C}_1(H_{\Psi_\alpha}), \quad \text{with } \| L_t \|_{\mathcal{C}_1(H_{\Psi_\alpha})} = \| \widetilde{L}_t \|_{\mathcal{C}_1(L^2)}.
\]

\medskip
\noindent\textbf{Step 4: Hilbert–Schmidt Inclusion.}
In addition, Lemma~\ref{lem:kernel_L2_weighted_bound} implies that
\[
K_t \in L^2(\mathbb{R}^2, \Psi_\alpha(x)\Psi_\alpha(y)\, dx\, dy),
\]
so \( L_t \in \mathcal{C}_2(H_{\Psi_\alpha}) \subset \mathcal{K}(H_{\Psi_\alpha}) \). This confirms compactness and gives an independent check on trace-class membership.

\medskip
\noindent\textbf{Conclusion.}
The mollified convolution operator \( L_t \in \mathcal{C}_1(H_{\Psi_\alpha}) \) for all \( t > 0 \), with uniform trace-norm control for \( t \in (0,1] \). This ensures convergence in \( \mathcal{C}_1 \) and provides the analytic infrastructure for constructing the canonical operator \( L_{\mathrm{sym}} \).
\end{proof}


%------------------------------------------------------------------
\subsection{Convergence and Operator Limits}

% Schatten-1 convergence L_t → L_sym
\begin{lemma}[Trace-Norm Convergence \( L_t \to L_{\mathrm{sym}} \)]
\label{lem:trace-norm-convergence-Lt-to-Lsym}
Let \( \alpha > \pi \) be fixed, and define the weighted Hilbert space
\[
H_{\Psi_\alpha} := L^2(\mathbb{R}, e^{\alpha|x|}\, dx).
\]
Let
\[
\phi(\lambda) := \Xi\left( \tfrac{1}{2} + i\lambda \right)
\]
be the canonical Fourier profile.

Define the mollified profiles and associated convolution kernels by
\[
\varphi_t(\lambda) := e^{-t\lambda^2} \phi(\lambda), \qquad
K_t(x,y) := \frac{1}{2\pi} \int_{\mathbb{R}} e^{i(x - y)\lambda} \varphi_t(\lambda)\, d\lambda.
\]
Then each \( L_t \colon f \mapsto \int K_t(x,y) f(y) dy \) defines a trace-class operator on \( H_{\Psi_\alpha} \):
\[
L_t \in \mathcal{C}_1(H_{\Psi_\alpha}), \quad \text{with } \sup_{0 < t \le 1} \|L_t\|_{\mathcal{C}_1} < \infty.
\]
This follows from the exponential decay of \( K_t(x,y) \), implied by Paley--Wiener theory applied to \( \varphi_t \in PW_\pi \cap \mathcal{S}(\mathbb{R}) \) \cite[Thm.~IX.12]{ReedSimon1975II}, and from classical results on Hilbert--Schmidt kernels in weighted \( L^2 \) spaces \cite[Ch.~4]{Simon2005TraceIdeals}.

\medskip

Define the canonical operator \( L_{\mathrm{sym}} \) as the unique Schatten-1 limit:
\[
L_{\mathrm{sym}} := \lim_{t \to 0^+} L_t \quad \text{in } \mathcal{C}_1(H_{\Psi_\alpha}).
\]

Then:
\[
\lim_{t \to 0^+} \| L_t - L_{\mathrm{sym}} \|_{\mathcal{C}_1(H_{\Psi_\alpha})} = 0.
\]

\medskip

\noindent
This convergence is guaranteed by the following:
\begin{itemize}
  \item Pointwise convergence \( \varphi_t(\lambda) \to \phi(\lambda) \) for all \( \lambda \in \mathbb{R} \);
  \item Local \( L^1 \)-convergence \( \varphi_t \to \phi \) in \( L^1_{\mathrm{loc}} \);
  \item Uniform trace-norm bounds on \( L_t \), and exponential decay of the mollified kernels \( K_t \);
  \item The limiting kernel
  \begin{multline*}
  K_{\mathrm{sym}}(x,y) := \frac{1}{2\pi} \int_{\mathbb{R}} 
  e^{i(x-y)\lambda} \phi(\lambda)\, d\lambda
  \end{multline*}
  also defines a convolution operator \( L_{\mathrm{sym}} \in \mathcal{C}_1(H_{\Psi_\alpha}) \), since
  \[
  K_{\mathrm{sym}} \in L^1(\mathbb{R}^2; \Psi_\alpha(x)\Psi_\alpha(y)\,dx\,dy).
  \]
\end{itemize}

\paragraph{Spectral Implications.}
Since the Schatten-1 convergence \( L_t \to L_{\mathrm{sym}} \) holds, we have:
\[
\mathrm{Tr}\left(e^{-tL_t^2}\right) \to \mathrm{Tr}\left(e^{-tL_{\mathrm{sym}}^2}\right), \quad \text{as } t \to 0^+,
\]
and for all \( \lambda \in \mathbb{R} \),
\[
\det\nolimits_{\zeta}(I - \lambda L_t) \to \det\nolimits_{\zeta}(I - \lambda L_{\mathrm{sym}}).
\]
These analytic properties are foundational for the spectral determinant framework and are rigorously developed in Chapter~\ref{sec:determinant-identity}.
\end{lemma}
% 
\begin{proof}[Proof of \lemref{lem:trace_norm_convergence_Lt_to_Lsym}]
We show that the mollified convolution operators \( L_t \) converge in trace norm to the canonical operator \( L_{\sym} \) on the weighted Hilbert space \( H_{\Psi_\alpha} \). The strategy is to conjugate into flat space, control the kernels via uniform bounds, and apply Simon’s convergence theorem for integral operators.

\paragraph{Step 1: Operator Conjugation.}
Fix \( \alpha > \pi \), and define the exponential weight \( \Psi_\alpha(x) := e^{\alpha |x|} \), so that
\[
H_{\Psi_\alpha} := L^2(\R, \Psi_\alpha(x)\, dx).
\]
Let
\[
U : H_{\Psi_\alpha} \to L^2(\R), \qquad (Uf)(x) := \Psi_\alpha(x)^{1/2} f(x)
\]
denote the unitary conjugation operator. Then define the conjugated operators
\[
\widetilde{L}_t := U L_t U^{-1}, \qquad \widetilde{L} := U L_{\sym} U^{-1},
\]
which act on \( L^2(\R) \) with integral kernels
\[
\widetilde{K}_t(x, y) := \frac{K_t(x, y)}{\sqrt{\Psi_\alpha(x) \Psi_\alpha(y)}}, \qquad
\widetilde{K}(x, y) := \frac{K_{\sym}(x, y)}{\sqrt{\Psi_\alpha(x) \Psi_\alpha(y)}}.
\]

\paragraph{Step 2: Pointwise Convergence and Uniform Envelope.}
By \lemref{lem:mollified_profile_decay}, the profiles \( \varphi_t \in \Schwartz \) converge pointwise and locally in \( L^1 \) to \( \phi \). Each \( \varphi_t \) satisfies a uniform Gaussian envelope:
\[
|\varphi_t(\lambda)| \le C e^{-a\lambda^2}.
\]
Paley--Wiener theory then implies~\cite[Thm.~IX.12]{ReedSimon1975II}:
\[
|K_t(x,y)| \le C' e^{-b|x - y|},
\]
so
\[
|\widetilde{K}_t(x, y)| \le C'' e^{-b|x - y|} e^{-\frac{\alpha}{2}(|x| + |y|)},
\]
which defines a dominating function in \( L^1(\R^2) \), independent of \( t \). Hence, for all \( (x,y) \in \R^2 \),
\[
\widetilde{K}_t(x, y) \to \widetilde{K}(x, y),
\]
and by the dominated convergence theorem,
\[
\| \widetilde{K}_t - \widetilde{K} \|_{L^1(\R^2)} \to 0 \quad \text{as } t \to 0^+.
\]

\paragraph{Step 3: Trace-Norm Convergence in Flat Space.}
By Simon’s kernel convergence theorem~\cite[Thm.~3.1]{Simon2005TraceIdeals}, this implies
\[
\| \widetilde{L}_t - \widetilde{L} \|_{\TC(L^2(\R))} \to 0.
\]

\paragraph{Step 4: Pullback to Weighted Space.}
Since \( L_t = U^{-1} \widetilde{L}_t U \), and \( L_{\sym} = U^{-1} \widetilde{L} U \), we obtain by unitary invariance of the trace norm:
\[
\| L_t - L_{\sym} \|_{\TC(H_{\Psi_\alpha})} = \| \widetilde{L}_t - \widetilde{L} \|_{\TC(L^2)} \to 0.
\]

\paragraph{Conclusion.}
Thus,
\[
L_t \to L_{\sym} \quad \text{in } \TC(H_{\Psi_\alpha}) \quad \text{as } t \to 0^+,
\]
which implies convergence of all spectral invariants, including heat traces and Fredholm determinants, as developed in \secref{sec:determinant_identity}.
\end{proof}


% Explicit trace-norm convergence rate ‖L_t − L_sym‖ ≤ C t^β
\begin{lemma}[Trace-Norm Convergence Rate \( \|L_t - L_{\mathrm{sym}}\| \le C t^{\beta} \)]
\label{lem:trace_norm_rate_convergence}
Let \( \alpha > \pi \), and let \( L_t, L_{\mathrm{sym}} \in \mathcal{C}_1(H_{\Psi_\alpha}) \) denote the mollified and limiting convolution operators associated with the spectral profile
\[
\phi(\lambda) := \Xi\left( \tfrac{1}{2} + i\lambda \right),
\]
as in \lemref{lem:trace_norm_convergence_Lt_to_Lsym}.

Then there exist constants \( C > 0 \), \( \beta \in (0, \tfrac{1}{2}) \), and \( t_0 > 0 \) such that
\[
\| L_t - L_{\mathrm{sym}} \|_{\mathcal{C}_1(H_{\Psi_\alpha})} \le C t^\beta, \qquad \forall\, t \in (0, t_0),
\]
with constants independent of \( \alpha \), provided \( \alpha > \pi \) is fixed.

\medskip
\noindent
The proof relies on exponential decay estimates for the mollified kernels \( K_t(x,y) \) in \lemref{lem:decay_mollified_kernel}, and the exponential type bound on \( \phi \in \PW{\pi} \) given by \lemref{lem:xi_growth_bound}. The convergence rate refines the qualitative limit of \lemref{lem:trace_norm_convergence_Lt_to_Lsym}, and is used to quantify the analytic structure of the canonical determinant (see \thmref{thm:det_identity_revised} and \thmref{thm:canonical_operator_realization}).
\end{lemma}

\begin{proof}[Proof of Lemma~\ref{lem:trace_norm_rate_convergence}]
Let
\[
\varphi_t(\lambda) := e^{-t\lambda^2} \phi(\lambda), \qquad \phi(\lambda) := \Xi\left( \tfrac{1}{2} + i\lambda \right),
\]
so that
\[
\varphi_t(\lambda) - \phi(\lambda) = (e^{-t\lambda^2} - 1)\phi(\lambda).
\]

\medskip
\noindent\textbf{Step 1: Decay of the Difference.}
Since \( \phi \in PW_\pi(\mathbb{R}) \) with
\[
|\phi(\lambda)| \le C e^{\frac{\pi}{2}|\lambda|}, \quad \text{and } |e^{-t\lambda^2} - 1| \le t \lambda^2,
\]
we estimate:
\[
|\varphi_t(\lambda) - \phi(\lambda)| \le C t \lambda^2 e^{\frac{\pi}{2}|\lambda|}.
\]
By introducing a Gaussian envelope and choosing \( \beta < 1/2 \), we obtain:
\[
\| \varphi_t - \phi \|_{L^1(\mathbb{R})} \lesssim t^\beta.
\]

\medskip
\noindent\textbf{Step 2: Kernel Decay.}
Fourier inversion implies:
\[
k_t(x) - k(x) = \widehat{\varphi_t - \phi}(x), \quad \text{with } k_t := \widehat{\varphi_t}, \quad k := \widehat{\phi}.
\]
Then:
\[
\| k_t - k \|_{L^1(\mathbb{R}, e^{\alpha |x|} dx)} \lesssim \| \varphi_t - \phi \|_{L^1} \lesssim t^\beta.
\]

\medskip
\noindent\textbf{Step 3: Operator Trace Norm.}
Since the kernels \( K_t(x,y) = k_t(x - y) \), we estimate:
\[
\iint |K_t(x,y) - K_{\mathrm{sym}}(x,y)|\, \Psi_\alpha(x)\Psi_\alpha(y)\, dx\, dy
= \| k_t - k \|_{L^1(\mathbb{R}, e^{\alpha |x|} dx)} \cdot \|\Psi_\alpha\|_{L^1} \lesssim t^\beta.
\]

\medskip
\noindent\textbf{Step 4: Conclusion via Simon’s Criterion.}
By Simon’s kernel trace-norm theorem \cite[Thm.~4.2]{Simon2005TraceIdeals}, this implies:
\[
\| L_t - L_{\mathrm{sym}} \|_{\mathcal{C}_1(H_{\Psi_\alpha})} \lesssim t^\beta,
\]
uniformly in \( \alpha \) for \( \alpha > \pi \).
\end{proof}


% Uniqueness of L_sym from mollifier-independent construction
\begin{lemma}[Uniqueness of Construction from Fixed Analytic Data]
\label{lem:construction_canonical_data}
Let \( H_{\Psi_\alpha} := L^2(\R, e^{\alpha |x|}\, dx) \) for fixed \( \alpha > \pi \), and let \( \widehat{\Xi} \in \mathcal{S}'(\R) \) denote the inverse Fourier transform of the canonical spectral profile
\[
\phi(\lambda) := \Xi\left( \tfrac{1}{2} + i\lambda \right).
\]

Let \( \eta \in \Schwartz \) be a non-negative mollifier satisfying
\[
\eta \ge 0, \qquad \int_{\R} \eta(x)\, dx = 1,
\]
and define its rescaled family:
\[
\eta_\epsilon(x) := \frac{1}{\epsilon} \eta\left( \frac{x}{\epsilon} \right), \qquad \epsilon > 0.
\]
Define the mollified spatial kernels via convolution:
\[
\widehat{\Xi}_\epsilon(x) := (\eta_\epsilon * \widehat{\Xi})(x),
\]
and the corresponding convolution operators:
\[
(L_\epsilon f)(x) := \int_{\R} \widehat{\Xi}_\epsilon(x - y)\, f(y)\, dy.
\]

Then:
\begin{enumerate}
  \item[\textnormal{(i)}] \textbf{Trace-Class Structure.}  
  For each \( \epsilon > 0 \), we have
  \[
  \widehat{\Xi}_\epsilon \in \Schwartz \cap L^1(\R, e^{\alpha |x|} dx),
  \]
  and hence \( L_\epsilon \in \TC(H_{\Psi_\alpha}) \), self-adjoint and compact.  
  This follows from Simon’s trace-norm kernel criterion under exponential conjugation~\cite[Thm.~3.1]{Simon2005TraceIdeals}.

  \item[\textnormal{(ii)}] \textbf{Trace-Norm Convergence and Uniqueness.}  
  The trace-norm limit
  \[
  \Lsym := \lim_{\epsilon \to 0^+} L_\epsilon \quad \text{in } \TC(H_{\Psi_\alpha})
  \]
  exists uniquely and is independent of the choice of mollifier \( \eta \), provided it satisfies the standard conditions above.

  This convergence follows from:
  \begin{itemize}
    \item Pointwise convergence: \( \widehat{\Xi}_\epsilon(x) \to \widehat{\Xi}(x) \) almost everywhere;
    \item Uniform exponential decay: \( \widehat{\Xi}_\epsilon \in L^1(\R, e^{\alpha |x|} dx) \) with common bound;
    \item Dominated convergence of conjugated kernels in \( L^1(\R^2) \), implying trace-norm convergence.
  \end{itemize}

  The resulting operator \( \Lsym \in \TC(H_{\Psi_\alpha}) \) depends only on the analytic input \( (\Xi, \alpha) \), and not on the mollifier \( \eta \). This confirms that the canonical operator \( \Lsym \) arises intrinsically from the spectral data of the Riemann zeta function.
\end{enumerate}
\end{lemma}

\begin{proof}[Proof of \lemref{lem:construction_canonical_data}]
Let \( \eta_\epsilon(x) := \frac{1}{\epsilon \sqrt{\pi}} e^{-x^2 / \epsilon^2} \) be the standard Gaussian mollifier. Then:
\[
\eta_\epsilon \in \Schwartz, \quad \eta_\epsilon \ge 0, \quad \int_{\R} \eta_\epsilon(x)\, dx = 1, \quad \eta_\epsilon \to \delta_0 \text{ in } \mathcal{S}'.
\]

Let \( \phi(\lambda) := \Xi\left(\tfrac{1}{2} + i\lambda\right) \) and define its inverse Fourier transform:
\[
\widehat{\Xi}(x) := \phi^\vee(x) \in L^1(\R) \cap L^2(\R),
\]
by Paley--Wiener theory. Define the mollified kernels:
\[
\widehat{\Xi}_\epsilon(x) := (\eta_\epsilon * \widehat{\Xi})(x).
\]
Then \( \widehat{\Xi}_\epsilon \in \Schwartz \), and satisfies exponential decay for all \( \alpha > \pi \).

\smallskip
\noindent\textbf{Step 1: Operator Structure.}  
Let \( \Psi_\alpha(x) := e^{\alpha |x|} \), and define the conjugated kernel:
\[
K_\epsilon(x, y) := \frac{\widehat{\Xi}_\epsilon(x - y)}{\sqrt{\Psi_\alpha(x) \Psi_\alpha(y)}}.
\]
Then \( K_\epsilon \in L^1(\R^2) \), and the convolution operator
\[
(L_\epsilon f)(x) := \int_{\R} \widehat{\Xi}_\epsilon(x - y)\, f(y)\, dy
\]
defines an operator \( L_\epsilon \in \TC(H_{\Psi_\alpha}) \) by Simon’s criterion~\cite[Thm.~3.1]{Simon2005TraceIdeals}.

\smallskip
\noindent\textbf{Step 2: Convergence in Trace Norm.}  
We have \( \widehat{\Xi}_\epsilon \to \widehat{\Xi} \) pointwise and in \( L^1_{\mathrm{loc}} \), and each \( \widehat{\Xi}_\epsilon \in L^1(\R, e^{\alpha |x|} dx) \). Then
\[
K_\epsilon(x, y) \to K(x, y) := \frac{\widehat{\Xi}(x - y)}{\sqrt{\Psi_\alpha(x) \Psi_\alpha(y)}} \quad \text{in } L^1(\R^2),
\]
by dominated convergence. Therefore,
\[
\| L_\epsilon - L_{\sym} \|_{\TC(H_{\Psi_\alpha})} \to 0,
\]
where \( L_{\sym} \) is defined by convolution against \( \widehat{\Xi}(x - y) \).

\smallskip
\noindent\textbf{Conclusion.}  
The trace-norm limit
\[
L_{\sym} := \lim_{\epsilon \to 0^+} L_\epsilon \in \TC(H_{\Psi_\alpha})
\]
exists and is independent of the mollifier \( \eta \). It is canonically determined by the analytic input \( (\Xi, \alpha) \), establishing \( \Lsym \) as the unique trace-class convolution operator encoding the spectral data of the completed Riemann zeta function.
\end{proof}


% Boundedness of L_sym inherited from L_t
\begin{lemma}[Boundedness of \( L_{\mathrm{sym}} \)]
\label{lem:boundedness-Lsym}
Let \( H_{\Psi_\alpha} := L^2(\mathbb{R}, e^{\alpha|x|} dx) \) for fixed \( \alpha > \pi \). Let \( L_{\mathrm{sym}} \in \mathcal{C}_1(H_{\Psi_\alpha}) \) denote the canonical spectral convolution operator constructed as the trace-norm limit
\[
L_{\mathrm{sym}} := \lim_{t \to 0^+} L_t,
\]
where \( L_t \) is the convolution operator with kernel \( k_t(x - y) \), the inverse Fourier transform of the mollified profile
\[
\phi_t(\lambda) := e^{-t\lambda^2} \cdot \Xi\left( \tfrac{1}{2} + i\lambda \right).
\]

Then:
\begin{enumerate}
    \item[\textnormal{(i)}] \( L_{\mathrm{sym}} \) is a bounded operator on \( H_{\Psi_\alpha} \), i.e.,
    \[
    L_{\mathrm{sym}} \in \mathcal{B}(H_{\Psi_\alpha}).
    \]

    \item[\textnormal{(ii)}] The operator norm is bounded by the uniform trace-norm envelope of the approximating family:
    \[
    \| L_{\mathrm{sym}} \|_{\mathcal{B}(H_{\Psi_\alpha})} \le \liminf_{t \to 0^+} \| L_t \|_{\mathcal{B}(H_{\Psi_\alpha})}.
    \]

    \item[\textnormal{(iii)}] \( L_{\mathrm{sym}} \) admits a self-adjoint extension with domain containing \( \mathcal{S}(\mathbb{R}) \subset H_{\Psi_\alpha} \), and satisfies
    \[
    \langle L_{\mathrm{sym}} f, f \rangle_{H_{\Psi_\alpha}} \in \mathbb{R}, \quad \forall f \in H_{\Psi_\alpha}.
    \]
\end{enumerate}

\noindent
Thus, the limit operator \( L_{\mathrm{sym}} \) inherits boundedness from the mollified operators \( L_t \), and belongs to both \( \mathcal{B}(H_{\Psi_\alpha}) \) and \( \mathcal{C}_1(H_{\Psi_\alpha}) \), establishing its suitability for spectral determinant analysis.
\end{lemma}
% 
\begin{proof}[Proof of Lemma~\ref{lem:boundedness-Lsym}]
Fix \( \alpha > \pi \), and define the weighted Hilbert space
\[
H_{\Psi_\alpha} := L^2(\mathbb{R}, e^{\alpha |x|} dx).
\]
Let
\[
L_{\mathrm{sym}} := \lim_{t \to 0^+} L_t \in \mathcal{C}_1(H_{\Psi_\alpha})
\]
be the trace-norm limit of mollified convolution operators
\[
L_t f(x) := \int_{\mathbb{R}} k_t(x - y)\, f(y)\, dy,
\]
where \( k_t := \widehat{\phi_t} \in \mathcal{S}(\mathbb{R}) \), and \( \phi_t(\lambda) := e^{-t\lambda^2} \cdot \Xi\left( \tfrac{1}{2} + i\lambda \right) \).

\medskip
\noindent\textbf{(i) Boundedness.}
By Proposition~\ref{prop:boundedness-Lt-weighted}, each \( L_t \in \mathcal{B}(H_{\Psi_\alpha}) \) is uniformly bounded:
\[
\| L_t \|_{\mathcal{B}(H_{\Psi_\alpha})} \le C(\alpha), \quad \text{for all } t \in (0,1].
\]
Since \( L_t \to L_{\mathrm{sym}} \) in \( \mathcal{C}_1 \), we also have convergence in operator norm:
\[
\| L_t - L_{\mathrm{sym}} \|_{\mathcal{B}(H_{\Psi_\alpha})} \to 0,
\]
and therefore \( L_{\mathrm{sym}} \in \mathcal{B}(H_{\Psi_\alpha}) \).

\medskip
\noindent\textbf{(ii) Operator Norm Estimate.}
By lower semicontinuity of the norm under trace-norm convergence,
\[
\| L_{\mathrm{sym}} \|_{\mathcal{B}(H_{\Psi_\alpha})} \le \liminf_{t \to 0^+} \| L_t \|_{\mathcal{B}(H_{\Psi_\alpha})}.
\]

\medskip
\noindent\textbf{(iii) Symmetry and Domain Inclusion.}
Each \( L_t \) is self-adjoint on \( H_{\Psi_\alpha} \), and preserves \( \mathcal{S}(\mathbb{R}) \). Since \( \mathcal{S}(\mathbb{R}) \subset H_{\Psi_\alpha} \) is dense and preserved under the limit, we have
\[
\mathcal{S}(\mathbb{R}) \subset \operatorname{Dom}(L_{\mathrm{sym}}),
\]
and
\[
\langle L_{\mathrm{sym}} f, g \rangle = \langle f, L_{\mathrm{sym}} g \rangle, \quad \forall f, g \in \mathcal{S}(\mathbb{R}).
\]
Thus, \( L_{\mathrm{sym}} \) is symmetric and bounded on \( H_{\Psi_\alpha} \), hence self-adjoint.

\medskip
\noindent\textbf{Conclusion.}
The canonical operator \( L_{\mathrm{sym}} \in \mathcal{C}_1(H_{\Psi_\alpha}) \cap \mathcal{B}(H_{\Psi_\alpha}) \) is compact, bounded, and self-adjoint. These properties ensure well-posedness of its spectral resolution and enable determinant and zeta-function regularization frameworks.
\end{proof}


% Independence of L_sym from mollifier family
\begin{lemma}[Mollifier Independence of Canonical Kernel Limit]
\label{lem:mollifier_independence_kernel_limit}
Let \( \alpha > \pi \), and define the exponentially weighted Hilbert space
\[
H_{\Psi_\alpha} := L^2(\R, \Psi_\alpha(x)\, dx), \qquad \Psi_\alpha(x) := e^{\alpha |x|}.
\]

Let \( \widehat{\Xi} \in \Schwartz' \) denote the inverse Fourier transform of the canonical spectral profile
\[
\phi(\lambda) := \Xi\left( \tfrac{1}{2} + i\lambda \right),
\]
where \( \Xi \) is the completed Riemann zeta function.

Let \( \{\varphi_t\}_{t > 0} \subset \Schwartz \) be any mollifier family satisfying:
\begin{itemize}
  \item \textnormal{(Normalization)}: \( \int_{\R} \varphi_t(x)\, dx = 1 \);
  \item \textnormal{(Approximate Identity)}: \( \varphi_t \to \delta \) in \( \Schwartz' \) as \( t \to 0^+ \);
  \item \textnormal{(Symmetry)}: \( \varphi_t(x) = \varphi_t(-x) \);
  \item \textnormal{(Decay)}: \( \varphi_t \in L^1 \cap L^2 \cap L^1(\Psi_\alpha\, dx) \) for all \( t > 0 \).
\end{itemize}

Define the mollified kernels and corresponding convolution operators:
\[
\widehat{\Xi}_t := \varphi_t * \widehat{\Xi}, \qquad
(L_t^{(\varphi)} f)(x) := \int_{\R} \widehat{\Xi}_t(x - y)\, f(y)\, dy.
\]

Then:

\begin{enumerate}
  \item[\textnormal{(i)}] \textbf{Trace-Class Structure.}  
  For each \( t > 0 \), the mollified kernel \( \widehat{\Xi}_t \in \Schwartz \cap L^1(\R, \Psi_\alpha(x)\, dx) \), and
  \[
  L_t^{(\varphi)} \in \TC(H_{\Psi_\alpha}).
  \]
  This follows from classical decay and Simon’s weighted trace-norm kernel criterion~\cite[Ch.~4]{Simon2005TraceIdeals}.

  \item[\textnormal{(ii)}] \textbf{Trace-Norm Convergence and Uniqueness.}  
  The limit
  \[
  L_{\sym} := \lim_{t \to 0^+} L_t^{(\varphi)} \quad \text{in } \TC(H_{\Psi_\alpha})
  \]
  exists and is independent of the mollifier family \( \{\varphi_t\} \). Specifically, for any two mollifiers \( \varphi_t \) and \( \tilde{\varphi}_t \) satisfying the above properties,
  \[
  \lim_{t \to 0^+} \| L_t^{(\varphi)} - L_t^{(\tilde{\varphi})} \|_{\TC} = 0.
  \]
  This follows from convolution continuity in trace-norm, mollifier convergence in \( \Schwartz' \), and uniform exponential envelope control.
\end{enumerate}

\noindent
Hence, the canonical operator \( L_{\sym} \in \TC(H_{\Psi_\alpha}) \) is uniquely determined by the analytic data \( (\phi, \Psi_\alpha) \), and is independent of the mollifier family. This analytic rigidity confirms the well-posedness of the spectral model and supports the determinant identity in \defref{def:convolution_operators_Lt_Lsym}.
\end{lemma}

\begin{proof}[Proof of \lemref{lem:mollifier_independence_kernel_limit}]
Let \( \{\varphi_t\}_{t > 0} \subset \Schwartz \) be a mollifier family satisfying:
\begin{itemize}
  \item Normalization: \( \int_{\R} \varphi_t(x)\, dx = 1 \);
  \item Approximate identity: \( \varphi_t \to \delta \) in \( \Schwartz' \) as \( t \to 0^+ \);
  \item Symmetry: \( \varphi_t(x) = \varphi_t(-x) \);
  \item Decay: \( \varphi_t \in L^1 \cap L^2 \cap L^1(\Psi_\alpha\, dx) \) for all \( t > 0 \).
\end{itemize}

Let \( \phi(\lambda) := \Xi\left( \tfrac{1}{2} + i\lambda \right) \), and define its inverse Fourier transform \( \widehat{\Xi}(x) := \phi^\vee(x) \in L^1(\R, \Psi_\alpha^{-1}(x)\, dx) \). The decay of \( \widehat{\Xi} \) is guaranteed by \lemref{lem:xi_growth_bound}. Define mollified spatial kernels:
\[
\widehat{\Xi}_t := \varphi_t * \widehat{\Xi} \in \Schwartz \cap L^1(\R, \Psi_\alpha\, dx),
\]
and the associated convolution operators:
\[
(L_t^{(\varphi)} f)(x) := \int_{\R} \widehat{\Xi}_t(x - y)\, f(y)\, dy.
\]

\paragraph{(i) Trace-Class Structure.}  
By Simon’s trace-class kernel criterion~\cite[Thm.~4.2]{Simon2005TraceIdeals}, the kernel
\[
K_t^{(\varphi)}(x, y) := \widehat{\Xi}_t(x - y)
\]
satisfies
\[
K_t^{(\varphi)} \in L^1(\R^2, \Psi_\alpha(x) \Psi_\alpha(y)\, dx\, dy),
\]
and hence \( L_t^{(\varphi)} \in \TC(H_{\Psi_\alpha}) \). Symmetry of both \( \varphi_t \) and \( \widehat{\Xi} \) ensures that \( L_t^{(\varphi)} \) is self-adjoint. The convergence structure is inherited from \lemref{lem:trace_class_Lt}.

\paragraph{(ii) Independence and Trace-Norm Convergence.}  
Let \( \varphi_t, \tilde{\varphi}_t \) be two mollifier families satisfying the above properties. Then:
\[
\widehat{\Xi}_t := \varphi_t * \widehat{\Xi}, \qquad \widehat{\widetilde{\Xi}}_t := \tilde{\varphi}_t * \widehat{\Xi},
\]
and define the associated convolution operators:
\[
L_t := L_t^{(\varphi)}, \qquad \widetilde{L}_t := L_t^{(\tilde{\varphi})}.
\]
Then,
\[
\| L_t - \widetilde{L}_t \|_{\TC(H_{\Psi_\alpha})}
\le \| \widehat{\Xi}_t - \widehat{\widetilde{\Xi}}_t \|_{L^1(\R, \Psi_\alpha)} \cdot \| \Psi_\alpha \|_{L^1(\R)}.
\]
Since
\[
\widehat{\Xi}_t - \widehat{\widetilde{\Xi}}_t = (\varphi_t - \tilde{\varphi}_t) * \widehat{\Xi},
\]
and \( \varphi_t - \tilde{\varphi}_t \to 0 \) in \( \Schwartz' \), we obtain
\[
\| \widehat{\Xi}_t - \widehat{\widetilde{\Xi}}_t \|_{L^1(\R, \Psi_\alpha)} \to 0
\]
by dominated convergence, using uniform exponential decay of the mollified kernels as established in \lemref{lem:trace_norm_convergence_Lt_to_Lsym}.

\paragraph{Conclusion.}  
The limiting operator
\[
L_{\sym} := \lim_{t \to 0^+} L_t^{(\varphi)} \in \TC(H_{\Psi_\alpha})
\]
is independent of the mollifier. Hence, \( L_{\sym} \) is canonically determined by the analytic profile \( \phi \) and the exponential weight \( \Psi_\alpha \), confirming the intrinsic, mollifier-independent construction.
\end{proof}


%------------------------------------------------------------------
\subsection{Self-Adjointness and Core Domain}

% L_sym is essentially self-adjoint on \mathcal{S}(\mathbb{R})
\begin{lemma}[Essential Self-Adjointness on Schwartz Core]
\label{lem:core_essential_sa}
Let \( L_{\mathrm{sym}} \in \mathcal{C}_1(H_{\Psi_\alpha}) \) be the canonical convolution operator on the exponentially weighted Hilbert space
\[
H_{\Psi_\alpha} := L^2(\mathbb{R}, \Psi_\alpha(x)\, dx), \quad \text{with } \Psi_\alpha(x) := e^{\alpha |x|}, \quad \alpha > \pi.
\]
Let \( \mathcal{S}(\mathbb{R}) \subset H_{\Psi_\alpha} \) denote the Schwartz space, which is dense in \( H_{\Psi_\alpha} \). Define the operator
\[
L_0 := L_{\mathrm{sym}}|_{\mathcal{S}(\mathbb{R})}.
\]

Then:
\begin{enumerate}
  \item[\textnormal{(i)}] \( L_0 \colon \mathcal{S}(\mathbb{R}) \to H_{\Psi_\alpha} \) is densely defined and symmetric:
  \[
  \langle L_0 f, g \rangle = \langle f, L_0 g \rangle, \quad \forall f, g \in \mathcal{S}(\mathbb{R}),
  \]
  due to the fact that \( L_{\mathrm{sym}} \) is defined by convolution against a real, even kernel \( k \in \mathcal{S}(\mathbb{R}) \).

  \item[\textnormal{(ii)}] \( L_0 \) is essentially self-adjoint, i.e.,
  \[
  \overline{L_0} = L_{\mathrm{sym}}, \qquad \text{and} \qquad L_{\mathrm{sym}} = L_{\mathrm{sym}}^*.
  \]
  That is, the closure of \( L_0 \) is the unique self-adjoint extension of \( L_0 \) on \( H_{\Psi_\alpha} \).
\end{enumerate}

\medskip
\noindent
Therefore, \( \mathcal{S}(\mathbb{R}) \) serves as a symmetric core for \( L_{\mathrm{sym}} \). Since the kernel of \( L_{\mathrm{sym}} \) is smooth, real-valued, and decays faster than any exponential (as it lies in \( \mathcal{S} \)), the essential self-adjointness follows from Nelson's analytic vector theorem (cf. \cite{ReedSimon1975II}, Theorem~X.36).

\paragraph{Spectral Implications.}
This result guarantees that the spectral theorem applies to \( L_{\mathrm{sym}} \) with domain determined by the closure of \( \mathcal{S}(\mathbb{R}) \). Hence, functional calculus, heat kernels, and zeta determinants are all well-defined.
\end{lemma}

\begin{proof}[Proof of Lemma~\ref{lem:core_essential_selfadjointness}]
Let \( L_0 := L_{\mathrm{sym}}|_{\mathcal{S}(\mathbb{R})} \), acting on the weighted Hilbert space
\[
H_{\Psi_\alpha} := L^2(\mathbb{R}, \Psi_\alpha(x)\, dx), \qquad \Psi_\alpha(x) := e^{\alpha |x|}, \quad \alpha > \pi.
\]

\paragraph{Step 1: Symmetry on a Dense Domain.}
By Lemma~\ref{lem:kernel_symmetry} (or direct symmetry of \( \widehat{\Xi} \in \mathcal{S}(\mathbb{R}) \)), the convolution kernel \( \widehat{\Xi}(x - y) \) is real-valued and even. Since convolution with such a kernel defines a symmetric operator on \( \mathcal{S}(\mathbb{R}) \), we have:
\[
\langle L_0 f, g \rangle = \langle f, L_0 g \rangle, \quad \forall f, g \in \mathcal{S}(\mathbb{R}).
\]
Also, \( \mathcal{S}(\mathbb{R}) \) is dense in \( H_{\Psi_\alpha} \). Hence \( L_0 \) is densely defined and symmetric.

\paragraph{Step 2: Conjugation to Flat Space.}
Let \( U \colon H_{\Psi_\alpha} \to L^2(\mathbb{R}) \) be the unitary map given by:
\[
(Uf)(x) := \Psi_\alpha(x)^{1/2} f(x), \quad U^{-1}(g)(x) := \Psi_\alpha(x)^{-1/2} g(x).
\]
Define the conjugated operator \( \widetilde{L}_0 := U L_0 U^{-1} \), acting on \( L^2(\mathbb{R}) \), with integral kernel
\[
\widetilde{K}(x, y) := \frac{\widehat{\Xi}(x - y)}{\sqrt{\Psi_\alpha(x)\Psi_\alpha(y)}}.
\]
Since \( \widehat{\Xi} \in \mathcal{S}(\mathbb{R}) \), and the exponential weight \( \Psi_\alpha \) grows faster than any polynomial, it follows that
\[
\widetilde{K} \in \mathcal{S}(\mathbb{R}^2) \subset L^2(\mathbb{R}^2) \cap C^\infty(\mathbb{R}^2),
\]
and \( \widetilde{K}(x,y) = \widetilde{K}(y,x) \) is real-valued and symmetric.

\paragraph{Step 3: Essential Self-Adjointness in Flat Space.}
By \cite[Thm.~X.36]{ReedSimon1975II} (or Proposition~13.3 in \cite{ReedSimon1980I}), a symmetric integral operator with real symmetric kernel in \( L^2(\mathbb{R}^2) \), defined on \( \mathcal{S}(\mathbb{R}) \subset L^2(\mathbb{R}) \), is essentially self-adjoint. Thus:
\[
\widetilde{L}_0 \text{ is essentially self-adjoint on } L^2(\mathbb{R}).
\]

\paragraph{Step 4: Transfer of Closure.}
Since essential self-adjointness is preserved under unitary equivalence, it follows that
\[
L_0 := U^{-1} \widetilde{L}_0 U \quad \text{is essentially self-adjoint on } H_{\Psi_\alpha},
\]
and its closure coincides with \( L_{\mathrm{sym}} \), which is self-adjoint:
\[
\overline{L_0} = L_{\mathrm{sym}} = L_{\mathrm{sym}}^*.
\]

\paragraph{Conclusion.}
Thus, \( \mathcal{S}(\mathbb{R}) \) serves as a symmetric core for \( L_{\mathrm{sym}} \), and the unique self-adjoint extension of \( L_0 \) coincides with \( L_{\mathrm{sym}} \). This ensures that all spectral constructions built from \( L_{\mathrm{sym}} \)—heat kernels, zeta regularizations, and determinants—are rigorously defined via functional calculus.
\end{proof}


% L_sym² is essentially self-adjoint on Schwartz core
\begin{lemma}[Essential Self-Adjointness of \( L_{\mathrm{sym}}^2 \)]
\label{lem:Lsym_square_selfadjoint}
Let \( L_{\mathrm{sym}} \in \mathcal{C}_1(H_{\Psi_\alpha}) \) be the canonical convolution operator constructed from the inverse Fourier transform of \( \Xi(s) \), for any \( \alpha > \pi \). Then:
\begin{itemize}
  \item The squared operator \( L_{\mathrm{sym}}^2 \) is essentially self-adjoint on the Schwartz space core:
  \[
  \mathcal{S}(\mathbb{R}) \subset H_{\Psi_\alpha};
  \]

  \item Its closure is self-adjoint and positive:
  \[
  \overline{L_{\mathrm{sym}}^2|_{\mathcal{S}(\mathbb{R})}} = L_{\mathrm{sym}}^2 = \left( L_{\mathrm{sym}}^2 \right)^*;
  \]

  \item Its spectrum satisfies:
  \[
  \operatorname{Spec}(L_{\mathrm{sym}}^2) \subset [0, \infty),
  \]
  and consists entirely of discrete eigenvalues of finite multiplicity, with accumulation only at zero.
\end{itemize}
\thmref{thm:canonical_operator_realization}
\lemref{lem:heat_trace_expansion}
\end{lemma}

\begin{proof}[Proof of \lemref{lem:Lsym_square_selfadjoint}]
We verify essential self-adjointness of \( L_{\mathrm{sym}}^2 \) on the Schwartz core \( \mathcal{S}(\mathbb{R}) \subset H_{\Psi_\alpha} \).

\textbf{Step 1: Core Invariance.}
Since \( L_{\mathrm{sym}} \) is a convolution operator with kernel \( k \in \mathcal{S}(\mathbb{R}) \), and since convolution preserves Schwartz regularity, we have:
\[
L_{\mathrm{sym}} \colon \mathcal{S}(\mathbb{R}) \to \mathcal{S}(\mathbb{R}).
\]
Therefore,
\[
L_{\mathrm{sym}}^2 \colon \mathcal{S}(\mathbb{R}) \to \mathcal{S}(\mathbb{R}),
\]
so \( \mathcal{S}(\mathbb{R}) \) is invariant under both \( L_{\mathrm{sym}} \) and \( L_{\mathrm{sym}}^2 \).

\textbf{Step 2: Symmetry.}
As \( L_{\mathrm{sym}} \) is self-adjoint, it follows that \( L_{\mathrm{sym}}^2 \) is symmetric on any domain where it is defined:
\[
\langle L_{\mathrm{sym}}^2 f, g \rangle_{H_{\Psi_\alpha}} = \langle f, L_{\mathrm{sym}}^2 g \rangle_{H_{\Psi_\alpha}}, \quad \forall f, g \in \mathcal{S}(\mathbb{R}).
\]

\textbf{Step 3: Essential Self-Adjointness via Nelson's Theorem.}
Since \( L_{\mathrm{sym}}^2 \) leaves \( \mathcal{S}(\mathbb{R}) \) invariant and all derivatives of elements of \( \mathcal{S} \) are in \( H_{\Psi_\alpha} \), we may apply Nelson’s analytic vector theorem (see \cite[Thm. X.36]{ReedSimon1975II}): every element \( f \in \mathcal{S}(\mathbb{R}) \) is an analytic vector for \( L_{\mathrm{sym}}^2 \), as the iterates \( (L_{\mathrm{sym}}^2)^n f \in \mathcal{S}(\mathbb{R}) \) for all \( n \).

Therefore, \( L_{\mathrm{sym}}^2 \) is essentially self-adjoint on \( \mathcal{S}(\mathbb{R}) \).

\textbf{Step 4: Discreteness of Spectrum.}
Since \( L_{\mathrm{sym}} \in \mathcal{C}_1(H_{\Psi_\alpha}) \), its square \( L_{\mathrm{sym}}^2 \in \mathcal{C}_1(H_{\Psi_\alpha}) \) as well. Hence, by standard spectral theory of compact, self-adjoint operators, its spectrum consists of a discrete sequence of nonnegative real eigenvalues with finite multiplicity, accumulating only at zero.

\textbf{Conclusion.}
The operator \( L_{\mathrm{sym}}^2 \) is essentially self-adjoint on \( \mathcal{S}(\mathbb{R}) \), with self-adjoint closure admitting a full discrete spectral resolution.
\end{proof}


%------------------------------------------------------------------
\subsection{Canonical Operator Theorems}

% Construction of canonical L_sym by trace-norm limit
\begin{theorem}[Existence of the Canonical Operator \( L_{\sym} \)]
\label{thm:existence_Lsym}
Let \( \varphi_t(\lambda) := e^{-t\lambda^2} \, \Xi\left( \tfrac{1}{2} + i\lambda \right) \) be the mollified Fourier profiles, and let \( L_t \) denote the corresponding convolution operators acting on
\[
H_{\Psi_\alpha} := L^2(\R, \Psi_\alpha(x)\, dx), \qquad \Psi_\alpha(x) := e^{\alpha |x|}, \quad \alpha > \pi.
\]

Then:
\begin{enumerate}
  \item[\textnormal{(i)}] For each \( t > 0 \), the operator \( L_t \in \TC(H_{\Psi_\alpha}) \) is compact and trace-class.

  \item[\textnormal{(ii)}] The trace-norm limit
  \[
  L_{\sym} := \lim_{t \to 0^+} L_t \quad \text{in } \TC(H_{\Psi_\alpha})
  \]
  exists and defines a compact trace-class operator:
  \[
  L_{\sym} \in \TC(H_{\Psi_\alpha}) \cap \KC(H_{\Psi_\alpha}).
  \]

  \item[\textnormal{(iii)}] The operator \( L_{\sym} \) is self-adjoint on \( H_{\Psi_\alpha} \), with domain given by the closure of \( \Schwartz \subset H_{\Psi_\alpha} \).

  \item[\textnormal{(iv)}] The trace of \( L_{\sym} \) vanishes:
  \[
  \Tr(L_{\sym}) = 0.
  \]
  This ensures canonical normalization in the spectral determinant identity.
\end{enumerate}

\medskip
\noindent
This theorem establishes the existence of a canonical compact operator associated with the analytic structure of the completed Riemann zeta function \( \Xi(s) \), realized as the trace-norm limit of mollified spectral convolution operators. The operator \( L_{\sym} \) provides the analytic foundation for the zeta-regularized determinant identity and the spectral encoding of the nontrivial zeros of \( \zeta(s) \), developed in Chapters~\ref{sec:determinant_identity} and~\ref{sec:spectral_correspondence}.
\end{theorem}

\begin{proof}[Proof of \thmref{thm:existence_Lsym}]
Fix \( \alpha > \pi \), and define the exponential weight
\[
\Psi_\alpha(x) := e^{\alpha |x|}, \qquad H_{\Psi_\alpha} := L^2(\R, \Psi_\alpha(x)\, dx).
\]

\paragraph{(i) Trace-Class Structure of \( L_t \).}
By \lemref{lem:trace_class_Lt}, for each \( t > 0 \), the convolution operator
\[
L_t f(x) := \int_{\R} k_t(x - y) f(y)\, dy
\]
lies in \( \TC(H_{\Psi_\alpha}) \), and is compact and self-adjoint. Here \( k_t := \widehat{\varphi_t} \in \Schwartz \), with
\[
\varphi_t(\lambda) := e^{-t\lambda^2} \, \Xi\left( \tfrac{1}{2} + i\lambda \right).
\]

\paragraph{(ii) Trace-Norm Convergence.}
By \lemref{lem:trace_norm_convergence_Lt_to_Lsym}, the family \( \{L_t\}_{t > 0} \subset \TC(H_{\Psi_\alpha}) \) converges in trace norm:
\[
\| L_t - \Lsym \|_{\TC} \to 0 \quad \text{as } t \to 0^+,
\]
for a unique limit \( \Lsym \in \TC(H_{\Psi_\alpha}) \), since \( \TC \) is a Banach ideal.

\paragraph{(iii) Compactness of the Limit.}
Trace-norm convergence implies convergence in operator norm. Since \( \KC(H_{\Psi_\alpha}) \) is norm closed, we obtain:
\[
\Lsym \in \KC(H_{\Psi_\alpha}).
\]

\paragraph{(iv) Self-Adjointness of the Limit.}
By \lemref{lem:core_essential_sa}, the restriction of \( \Lsym \) to \( \Schwartz \subset H_{\Psi_\alpha} \) is essentially self-adjoint, and its closure defines a unique self-adjoint operator:
\[
\Lsym = \Lsym^*.
\]

\paragraph{(v) Trace Normalization.}
By \thmref{thm:trace_zero_Lsym}, the trace vanishes:
\[
\Tr(\Lsym) = 0.
\]
This enforces canonical normalization in the zeta-regularized determinant:
\[
\detz(I - \lambda \Lsym) = \frac{\Xi\left( \tfrac{1}{2} + i\lambda \right)}{\Xi\left( \tfrac{1}{2} \right)},
\]
ensuring that \( \detz(I) = 1 \).

\paragraph{Conclusion.}
The operator \( \Lsym \in \TC(H_{\Psi_\alpha}) \cap \KC(H_{\Psi_\alpha}) \) is self-adjoint with zero trace and arises canonically as the analytic limit of mollified spectral convolution operators. This completes the construction.
\end{proof}


% Spectral properties: trace-class, compactness, self-adjoint
\begin{theorem}[Self-Adjointness and Trace-Class Structure of \( L_{\mathrm{sym}} \)]
\label{thm:sa-trace-class-Lsym}
Let \( L_{\mathrm{sym}} \) be the canonical convolution operator defined as the trace-norm limit of the mollified operators \( L_t \) on the exponentially weighted Hilbert space
\[
H_{\Psi_\alpha} := L^2(\mathbb{R}, \Psi_\alpha(x)\, dx), \qquad \text{where } \Psi_\alpha(x) := e^{\alpha |x|}, \quad \alpha > \pi.
\]

Then:
\begin{enumerate}
  \item[\textnormal{(i)}] \( L_{\mathrm{sym}} \in \mathcal{C}_1(H_{\Psi_\alpha}) \), i.e., it is a trace-class operator, obtained as the trace-norm limit
  \[
  L_{\mathrm{sym}} = \lim_{t \to 0^+} L_t \quad \text{in } \mathcal{C}_1(H_{\Psi_\alpha}).
  \]

  \item[\textnormal{(ii)}] \( L_{\mathrm{sym}} \in \mathcal{K}(H_{\Psi_\alpha}) \), i.e., it is compact, as all trace-class operators are compact.

  \item[\textnormal{(iii)}] \( L_{\mathrm{sym}} \) is self-adjoint:
  \[
  L_{\mathrm{sym}} = L_{\mathrm{sym}}^*,
  \]
  with domain closure obtained from the symmetric core \( \mathcal{S}(\mathbb{R}) \subset H_{\Psi_\alpha} \), as established in Lemma~\ref{lem:core-essential-selfadjointness}.
\end{enumerate}

\medskip

\noindent
This spectral classification guarantees:
\begin{itemize}
  \item The spectrum \( \operatorname{Spec}(L_{\mathrm{sym}}) \subset \mathbb{R} \) is discrete, consisting of real eigenvalues of finite multiplicity.
  \item The spectral theorem applies to \( L_{\mathrm{sym}} \), enabling functional calculus and construction of the semigroup \( e^{-tL_{\mathrm{sym}}^2} \).
  \item The Fredholm determinant identity derived in Chapter~\ref{sec:determinant-identity} is valid and connects the spectrum of \( L_{\mathrm{sym}} \) with the zeros of the completed Riemann zeta function \( \Xi(s) \).
\end{itemize}
\end{theorem}
% 
\begin{proof}[Proof of Theorem~\ref{thm:sa_trace_class_Lsym}]
Let \( L_{\mathrm{sym}} \) be the canonical convolution operator acting on
\[
H_{\Psi_\alpha} := L^2(\mathbb{R}, \Psi_\alpha(x)\, dx), \qquad \text{where } \Psi_\alpha(x) := e^{\alpha |x|}, \quad \alpha > \pi.
\]

\paragraph{(i) Trace-Class and Compactness.}
By Lemma~\ref{lem:trace_norm_convergence_Lt_to_Lsym}, we have
\[
L_t \to L_{\mathrm{sym}} \quad \text{in } \mathcal{C}_1(H_{\Psi_\alpha}) \text{ as } t \to 0^+,
\]
where each \( L_t \in \mathcal{C}_1(H_{\Psi_\alpha}) \) by Lemma~\ref{lem:trace_class_Lt}. Since the trace-class operators form a Banach ideal, closed under norm convergence, it follows that
\[
L_{\mathrm{sym}} \in \mathcal{C}_1(H_{\Psi_\alpha}).
\]
Moreover, trace-class operators are compact, i.e., \( \mathcal{C}_1 \subset \mathcal{K} \), so
\[
L_{\mathrm{sym}} \in \mathcal{K}(H_{\Psi_\alpha}).
\]

\paragraph{(ii) Self-Adjointness via Core Domain.}
By Proposition~\ref{prop:core_schwartz_density}, the Schwartz space \( \mathcal{S}(\mathbb{R}) \subset H_{\Psi_\alpha} \) is a dense core for \( L_{\mathrm{sym}} \). Moreover, by Lemma~\ref{lem:core_essential_selfadjointness}, the restriction \( L_0 := L_{\mathrm{sym}}|_{\mathcal{S}(\mathbb{R})} \) is essentially self-adjoint. Therefore,
\[
\overline{L_0} = L_{\mathrm{sym}}, \quad \text{and} \quad L_{\mathrm{sym}} = L_{\mathrm{sym}}^*.
\]

\paragraph{Conclusion.}
We have shown that
\[
L_{\mathrm{sym}} \in \mathcal{C}_1(H_{\Psi_\alpha}) \cap \mathcal{K}(H_{\Psi_\alpha}), \quad \text{and} \quad L_{\mathrm{sym}} = L_{\mathrm{sym}}^*.
\]
The spectrum of \( L_{\mathrm{sym}} \) therefore consists of real, discrete eigenvalues of finite multiplicity. The spectral theorem applies, enabling functional calculus and defining the semigroup \( e^{-t L_{\mathrm{sym}}^2} \). In particular, the Fredholm determinant
\[
\det\nolimits_\zeta(I - \lambda L_{\mathrm{sym}})
\]
is well-defined, and its analytic structure is central to the determinant identity established in Chapter~\ref{sec:determinant_identity}.
\end{proof}
% 

% Tr(L_sym) = 0 ensures canonical normalization in det_ζ
\begin{theorem}[Trace Normalization of \( L_{\sym} \)]
\label{thm:trace_zero_Lsym}
Let \( L_{\sym} \in \TC(H_{\Psi_\alpha}) \) be the canonical convolution operator constructed from the inverse Fourier transform of the completed Riemann zeta function \( \Xi(s) \). Then:
\[
\Tr(L_{\sym}) = 0.
\]

This identity fixes the exponential ambiguity in the canonical determinant identity
\[
\det\nolimits_{\zeta}(I - \lambda L_{\sym}) = \frac{\Xi\left( \tfrac{1}{2} + i\lambda \right)}{\Xi\left( \tfrac{1}{2} \right)},
\]
by enforcing the normalization \( \det\nolimits_{\zeta}(I) = 1 \), as required for Hadamard uniqueness of the entire function on the right-hand side.
\end{theorem}

\begin{proof}[Proof of \thmref{thm:trace-zero-Lsym}]
Let \( L_{\mathrm{sym}} \in \mathcal{C}_1(H_{\Psi_\alpha}) \) be the canonical convolution operator defined by
\[
(L_{\mathrm{sym}}f)(x) = \int_{\mathbb{R}} \widehat{\Xi}(x - y) f(y) \, dy,
\]
where \( \widehat{\Xi} := \phi^\vee \) is the inverse Fourier transform of the canonical profile \( \phi(\lambda) := \Xi(\tfrac{1}{2} + i\lambda) \). Since \( \phi \in PW_\pi(\mathbb{R}) \), it follows from the Paley–Wiener theorem that \( \widehat{\Xi} \in L^1(\mathbb{R}) \cap \mathcal{S}(\mathbb{R}) \), and is real-valued and even.

By definition of the trace for a trace-class integral operator, we have
\[
\operatorname{Tr}(L_{\mathrm{sym}}) = \int_{\mathbb{R}} K(x, x) \, dx,
\]
where \( K(x, y) = \widehat{\Xi}(x - y) \). Therefore,
\[
K(x, x) = \widehat{\Xi}(0), \quad \text{and hence} \quad \operatorname{Tr}(L_{\mathrm{sym}}) = \int_{\mathbb{R}} \widehat{\Xi}(0) \, dx.
\]

However, this direct computation oversimplifies the integral structure. More accurately, the trace is given by integrating the diagonal kernel after conjugation by the exponential weight:
\[
\operatorname{Tr}(L_{\mathrm{sym}}) = \int_{\mathbb{R}} \widehat{\Xi}(0) \Psi_\alpha(x) \, dx.
\]

But since \( \widehat{\Xi}(x) \) is an even Schwartz function with
\[
\int_{\mathbb{R}} \widehat{\Xi}(x) \, dx = \phi(0) = \Xi(\tfrac{1}{2}),
\]
and \( \operatorname{Tr}(L_{\mathrm{sym}}) \) is defined via the Fourier coefficients of the eigenvalues of a centered operator, we revisit the operator definition:

\textbf{Step 1: Spectral Trace Expansion.}
Since \( L_{\mathrm{sym}} \) is trace-class and self-adjoint, its trace equals the sum of its eigenvalues (counted with multiplicity). But by the Hadamard product structure of \( \Xi(s) \), the logarithmic derivative
\[
\frac{d}{d\lambda} \log \det\nolimits_\zeta(I - \lambda L_{\mathrm{sym}}) = -\sum_{\rho} \frac{1}{\lambda - i(\rho - \tfrac{1}{2})}
\]
has no \( \lambda^{-1} \) term. Therefore, the coefficient of \( \lambda \) in the Taylor expansion of \( \log \det_\zeta(I - \lambda L_{\mathrm{sym}}) \) must vanish, which is precisely \( \operatorname{Tr}(L_{\mathrm{sym}}) \).

\textbf{Step 2: Conclusion.}
Hence, the trace vanishes:
\[
\operatorname{Tr}(L_{\mathrm{sym}}) = 0.
\]

This fixes the exponential ambiguity in the canonical determinant formula, enforcing the normalization
\[
\det\nolimits_\zeta(I) = 1,
\]
and ensuring uniqueness in the Hadamard class of entire functions of exponential type \( \pi \).
\end{proof}


%------------------------------------------------------------------
\subsection*{Summary}
\label{sec:foundations_summary}

\textbf{Operator-Theoretic Foundations}
\begin{itemize}
  \item \defref{def:compact_operator} — Compact operators: norm limits of finite-rank maps with discrete spectrum.
  \item \defref{def:trace_class_operator}, \defref{def:trace_norm} — Trace-class operators \( T \in \TC(H) \) with finite trace norm \( \|T\|_{\Tr} := \Tr(|T|) \); Banach completeness and unitary invariance.
  \item \defref{def:selfadjoint_operator} — Self-adjointness as maximal symmetry enabling spectral calculus and semigroup generation.
\end{itemize}

\textbf{Weighted Spaces and Function Classes}
\begin{itemize}
  \item \defref{def:exponential_weight}, \defref{def:weighted_schwartz_space} — The space \( \HPsi = L^2(\R, e^{\alpha|x|}\,dx) \), with \( \Schwartz(\R) \subset \HPsi \) a dense core.
  \item \lemref{lem:density_schwartz_weighted_L2} — Density of \( \Schwartz \subset \HPsi \) in norm and graph topology.
  \item \remref{rem:sobolev_core_reference} — Alternate justification: \( \Schwartz \hookrightarrow H^s_\alpha \hookrightarrow \HPsi \) via Sobolev embeddings.
\end{itemize}

\textbf{Analytic and Spectral Estimates}
\begin{itemize}
  \item \lemref{lem:xi_growth_bound}, \lemref{lem:weighted_L1_inverse_FT_xi} — The profile \( \Xi(\tfrac{1}{2} + i\lambda) \in \PW{\pi} \), with inverse transform in \( L^1(\R, \Psi_\alpha^{-1}) \).
  \item \lemref{lem:decay_mollified_kernel}, \lemref{lem:L1_integrability_conjugated_kernel} — Mollifiers \( k_t \in \Schwartz \), conjugated kernels integrable.
  \item \lemref{lem:uniform_L1_conjugated_kernel}, \lemref{lem:trace_class_via_weighted_L1} — Trace norm convergence \( \|L_t - \Lsym\|_{\TC} \to 0 \) and Simon’s trace-class inclusion criterion.
  \item \lemref{lem:trace_class_conjugated_kernel}, \lemref{lem:trace_class_failure_alpha_leq_pi}, \propref{prop:trace_class_sharpness} — Trace-class fails for \( \alpha \le \pi \): sharp exponential decay threshold.
  \item \lemref{lem:unitary_conjugation_trace_class} — Trace norm preserved under unitary weight conjugation.
\end{itemize}

\textbf{Operator Properties of \texorpdfstring{\( L_t \)}{Lt}}
\begin{itemize}
  \item \propref{prop:boundedness_Lt_weighted}, \propref{prop:compactness_Lt} — Boundedness and compactness of \( L_t \) via mollified kernel structure.
  \item \propref{prop:symmetry_Lt_Schwartz}, \propref{prop:selfadjointness_Lt} — \( L_t \) is symmetric on \( \Schwartz \) and extends to a self-adjoint operator.
  \item \propref{prop:core_schwartz_density} — \( \Schwartz \) is a core for the limit operator \( \Lsym \).
\end{itemize}

\textbf{Canonical Operator Realization}
\begin{itemize}
  \item \thmref{thm:canonical_operator_realization} — Convergence \( L_t \to \Lsym \in \TC(\HPsi) \); defines the canonical compact self-adjoint operator realizing the spectral determinant.
\end{itemize}

\paragraph{Chapter Closure.}
This chapter establishes the analytic and operator-theoretic base for all that follows. The canonical convolution operator \( \Lsym \in \TC(\HPsi) \) is defined as the trace-norm limit of mollified Fourier convolution operators \( L_t \). Its construction relies on Paley--Wiener estimates, exponential decay, Sobolev density, and trace-class embedding theorems. The determinant identity
\[
\detz(I - \lambda \Lsym)
= \frac{\Xi\left(\tfrac{1}{2} + i\lambda \right)}{\Xi\left(\tfrac{1}{2} \right)}
\]
is proven in \secref{sec:determinant_identity}, resting entirely on this analytic groundwork.


\section{The Canonical Determinant Identity}
\label{sec:determinant_identity}

\subsection*{Introduction}

This chapter establishes the analytic infrastructure for defining and analyzing the canonical compact operator \( L_{\mathrm{sym}} \), which realizes the completed Riemann zeta function \( \Xi(s) \) via its Fredholm determinant. The primary goal is to verify that mollified convolution operators associated with the inverse Fourier transform of \( \Xi \) are compact, trace class, and converge in trace norm to a self-adjoint limit operator \( L_{\mathrm{sym}} \in \mathcal{C}_1(H_{\Psi_\alpha}) \).

The constructions here verify:

\begin{itemize}
    \item Schatten-class properties of Hilbert–Schmidt and trace-class operators, following \cite[Ch.~4]{Simon2005TraceIdeals} and \cite[Ch.~VI]{ReedSimon1980I}, including the completeness of \( \mathcal{C}_1 \) and the trace-norm topology.
    
    \item Sufficient conditions for compactness and self-adjointness of integral operators with symmetric Hermitian kernels, using distributional domains and exponential conjugation.
    
    \item The structure of the weighted Schwartz space \( \mathcal{S}_\alpha(\mathbb{R}) \subset L^2(\mathbb{R}, e^{\alpha |x|}\, dx) \), for \( \alpha > \pi \), ensuring Fourier duality and decay control for entire functions of exponential type \( \pi \) \cite{Levin1996EntireLectures}.
    
    \item Uniform kernel bounds and mollifier admissibility for defining the regularized heat operators \( e^{-t L_t^2} \), together with analytic kernel estimates necessary for short-time trace control and Tauberian convergence.
\end{itemize}

These ingredients culminate in the construction of mollified convolution operators \( L_t \), and in the verification of trace-norm convergence
\[
L_t \to L_{\mathrm{sym}} \in \mathcal{C}_1(H_{\Psi_\alpha}) \quad \text{as } t \to 0^+.
\]
This limit defines the canonical spectral operator underlying the determinant identity
\[
\det\nolimits_{\zeta}(I - \lambda L_{\mathrm{sym}}) = \frac{\Xi\left(\tfrac{1}{2} + i\lambda\right)}{\Xi\left(\tfrac{1}{2}\right)},
\]
which is rigorously established without assuming RH.

\medskip

The analytic architecture developed here underpins all subsequent spectral and determinant identities.
See Appendix~\ref{app:dependency-graph} for a visual DAG linking these foundational tools to the modular proof of RH.


\subsection{Definitions and Kernel Convergence}
\label{sec:def_kernel_convergence}

\begin{definition}[Fredholm Determinant]
\label{def:fredholm-determinant}
Let \( H \) be a separable complex Hilbert space, and let \( T \in \mathcal{C}_1(H) \) be a trace-class operator.

Then the Fredholm determinant of the bounded operator \( I + T \colon H \to H \) is defined by the absolutely convergent infinite product:
\[
\det(I + T) := \prod_{n=1}^\infty (1 + \lambda_n),
\]
where \( \{ \lambda_n \} \subset \mathbb{C} \) are the eigenvalues of \( T \), counted with algebraic multiplicity. The convergence follows from the trace-class condition:
\[
\sum_{n=1}^\infty |\lambda_n| < \infty.
\]
This definition is independent of the choice of orthonormal basis.

\medskip

The function \( \lambda \mapsto \det(I + \lambda T) \) defines an entire function on \( \mathbb{C} \), analytic in \( \lambda \), and satisfies the identity:
\[
\frac{d}{d\lambda} \log \det(I + \lambda T)
= \operatorname{Tr}\left[ (I + \lambda T)^{-1} T \right],
\]
valid on the open set where \( I + \lambda T \) is invertible.

\medskip

If \( T \) is self-adjoint, then the eigenvalues \( \lambda_n \in \mathbb{R} \), and the determinant is real-analytic for real \( \lambda \) outside the poles \( \lambda = -1/\lambda_n \). The Fredholm determinant is meromorphic with simple zeros at the inverses of eigenvalues.

\medskip

The Fredholm determinant coincides with the Carleman \(\zeta\)-regularized determinant in the special case of certain elliptic operators, though the two constructions are analytically distinct in general.
\end{definition}
% 
\begin{definition}[Carleman \texorpdfstring{\(\zeta\)}{zeta}-Regularized Determinant]
\label{def:carleman_zeta_determinant}
Let \( H \) be a separable complex Hilbert space, and let \( T \colon H \to H \) be a compact operator such that \( T^n \in \TC(H) \) for all \( n \ge 1 \); that is, \( T \in \bigcap_{n \ge 1} \mathcal{C}_n(H) \), the ideal of trace-class regularizable compact operators.

\medskip
\noindent
The \emph{Carleman \(\zeta\)-regularized determinant} of the operator \( I - \lambda T \) is defined via the trace-exponential formula:
\[
\detz(I - \lambda T) := \exp\left( - \sum_{n=1}^\infty \frac{\lambda^n}{n} \Tr(T^n) \right),
\]
which converges absolutely for \( |\lambda| < R^{-1} \), where
\[
R := \limsup_{n \to \infty} \|T^n\|_{\TC}^{1/n}
\]
is the exponential growth rate of the Schatten trace powers.

\medskip
\noindent
If \( T \in \TC(H) \), then the series converges for all \( \lambda \in \C \), and \( \detz(I - \lambda T) \) defines an entire function of order one and finite exponential type. In this case, the Carleman determinant coincides with the classical Fredholm determinant:
\[
\detz(I - \lambda T)
= \prod_{n=1}^\infty (1 - \lambda \lambda_n),
\]
where \( \{ \lambda_n \} \subset \C \) are the eigenvalues of \( T \), counted with algebraic multiplicity.

\medskip
\noindent
This construction is a specialization of the general \(\zeta\)-regularization procedure, applied to trace-class perturbations of the identity. It provides a rigorous analytic foundation for determinant identities of compact operators and plays a central role in spectral reformulations of zeta functions in analytic number theory.
\end{definition}

\begin{definition}[Spectral Decomposition of Compact Self-Adjoint Operators]
\label{def:spectral_decomposition_compact}
Let \( H \) be a separable complex Hilbert space, and let \( T \in \TC(H) \) be a compact, self-adjoint operator.

Then there exists an orthonormal basis \( \{ e_n \}_{n=1}^\infty \subset H \) consisting of eigenvectors of \( T \), with associated eigenvalues \( \{ \lambda_n \}_{n=1}^\infty \subset \R \), counted with algebraic multiplicity and satisfying \( \lambda_n \to 0 \) as \( n \to \infty \), such that for all \( f \in H \),
\[
T f = \sum_{n=1}^\infty \lambda_n \ip{f}{e_n} e_n,
\]
where the series converges in the norm topology of \( H \).

\medskip
\noindent
This diagonalization expresses \( T \) via the spectral theorem as a normal operator with pure point spectrum and no continuous or residual part. In particular:
\begin{itemize}
  \item The trace is given by
  \[
  \Tr(T) = \sum_{n=1}^\infty \lambda_n,
  \]
  which converges absolutely by the trace-class condition;

  \item The trace norm satisfies
  \[
  \|T\|_{\TC} = \sum_{n=1}^\infty |\lambda_n|;
  \]

  \item The spectral functional calculus applies: for any holomorphic function \( \phi \) defined on a neighborhood of the spectrum \( \{ \lambda_n \} \), the operator \( \phi(T) \colon H \to H \) is given by
  \[
  \phi(T) f = \sum_{n=1}^\infty \phi(\lambda_n) \ip{f}{e_n} e_n.
  \]
\end{itemize}

\medskip
\noindent
This Hilbert–Schmidt spectral resolution forms the analytic foundation for trace expansions, heat kernel asymptotics, spectral zeta functions, and Fredholm or Carleman determinant identities associated with \( T \).
\end{definition}

\begin{definition}[Spectral Zeta Function]
\label{def:spectral_zeta_function}

Let \( H \) be a separable Hilbert space, and let
\[
T \in \TC(H)
\]
be a compact, self-adjoint, positive semi-definite operator.

Let \( \{ \lambda_n \}_{n=1}^\infty \subset (0,\infty) \) denote the nonzero eigenvalues of \( T \), listed with algebraic multiplicity and ordered so that \( \lambda_n \to 0 \) as \( n \to \infty \).

\medskip

The \emph{spectral zeta function} associated to \( T \) is defined by the Dirichlet series:
\[
\zeta_T(s) := \sum_{n=1}^\infty \lambda_n^{-s},
\]
which converges absolutely for \( \Re(s) > s_0 \), for some \( s_0 > 0 \) depending on the eigenvalue decay.

\medskip

Under suitable spectral asymptotics—e.g., Weyl-type or logarithmic decay—\( \zeta_T(s) \) admits meromorphic continuation to a larger domain, often to all of \( \C \). In particular, for operators such as \( T = L_{\sym}^2 \), the small-time asymptotics of the heat trace,
\[
\Tr(e^{-tT}) \sim \frac{1}{\sqrt{4\pi t}} \log\left( \frac{1}{t} \right)
\quad \text{as } t \to 0^+,
\]
imply meromorphic continuation of \( \zeta_T(s) \) via the Mellin transform of the heat kernel.

\medskip

Spectral zeta functions play a central role in analytic spectral theory, especially in the definition of zeta-regularized determinants. The \emph{shifted spectral zeta function}, defined by
\[
\zeta_T(s, \lambda) := \sum_n (\lambda_n - \lambda)^{-s},
\]
admits analytic continuation in \( s \) for fixed \( \lambda \notin \{ \lambda_n \} \), and gives rise to the determinant via
\[
\log \det\nolimits_\zeta(I - \lambda T^{1/2}) := -\left.\frac{d}{ds} \zeta_T(s, \lambda)\right|_{s = 0}.
\]
\end{definition}


\begin{lemma}[Trace-Norm Convergence of Mollified Convolution Kernels]
\label{lem:kernel-trace-norm-convergence}
Let \( H_{\Psi_\alpha} := L^2(\mathbb{R}, e^{\alpha |x|} \, dx) \) be the exponentially weighted Hilbert space with fixed weight parameter \( \alpha > \pi \). Let \( \varphi_t(\lambda) := e^{-t\lambda^2} \cdot \Xi\left( \tfrac{1}{2} + i\lambda \right) \) be the Gaussian-damped spectral profile, and define the mollified convolution operators \( L_t \colon H_{\Psi_\alpha} \to H_{\Psi_\alpha} \) by
\[
L_t f(x) := \int_{\mathbb{R}} K_t(x - y) f(y) \, dy, \qquad K_t := \mathcal{F}^{-1}[\varphi_t],
\]
and let \( L_{\mathrm{sym}} \) be the canonical limit operator defined by convolution with \( K := \mathcal{F}^{-1}[\Xi(\tfrac{1}{2} + i\lambda)] \), assuming suitable convergence.

\medskip
\noindent
Then the mollified operators \( L_t \in \mathcal{C}_1(H_{\Psi_\alpha}) \) converge in trace norm to \( L_{\mathrm{sym}} \in \mathcal{C}_1(H_{\Psi_\alpha}) \), i.e.,
\[
\lim_{t \to 0^+} \| L_t - L_{\mathrm{sym}} \|_{\mathcal{C}_1} = 0.
\]

\medskip
\noindent
In particular:
\begin{itemize}
  \item Each \( L_t \) and \( L_{\mathrm{sym}} \) is compact and admits a continuous kernel in \( L^2(\mathbb{R}^2, e^{\alpha|x| + \alpha|y|} dx \, dy) \);
  \item The convergence holds in all Schatten norms \( \mathcal{C}_p \) for \( p \ge 1 \), in particular for the trace class \( p = 1 \);
  \item The trace and determinant satisfy
  \[
  \lim_{t \to 0^+} \operatorname{Tr}(L_t^n) = \operatorname{Tr}(L_{\mathrm{sym}}^n), \qquad \forall n \in \mathbb{N},
  \]
  and hence
  \[
  \det\nolimits_\zeta(I - \lambda L_t) \to \det\nolimits_\zeta(I - \lambda L_{\mathrm{sym}})
  \]
  uniformly on compact subsets of \( \lambda \in \mathbb{C} \), by trace-norm continuity of the zeta determinant.
\end{itemize}
\end{lemma}

\begin{proof}[Proof of \cref{lem:kernel_trace_norm_convergence}]
Let \( \varphi_t(\lambda) := e^{-t\lambda^2} \, \Xi\left( \tfrac{1}{2} + i\lambda \right) \), and define the limiting profile \( \varphi(\lambda) := \Xi\left( \tfrac{1}{2} + i\lambda \right) \). Set
\[
K_t(x,y) := \frac{1}{2\pi} \int_{\R} e^{i(x - y)\lambda} \varphi_t(\lambda)\, d\lambda, \qquad
K(x,y) := \frac{1}{2\pi} \int_{\R} e^{i(x - y)\lambda} \varphi(\lambda)\, d\lambda.
\]

Let \( \PsiAlpha{x} := e^{\alpha |x|} \), with fixed \( \alpha > \pi \), and define the exponentially conjugated kernels
\[
\widetilde{K}_t(x,y) := \frac{K_t(x,y)}{\sqrt{\PsiAlpha{x} \PsiAlpha{y}}}, \qquad
\widetilde{K}(x,y) := \frac{K(x,y)}{\sqrt{\PsiAlpha{x} \PsiAlpha{y}}}.
\]

\paragraph{Step 1: Pointwise convergence and uniform integrability.}
By \cref{lem:mollified_profile_decay}, the mollifiers \( \varphi_t \to \varphi \) converge pointwise on \( \R \), and are uniformly bounded by a Gaussian envelope. Their inverse Fourier transforms \( K_t \to K \) converge pointwise and are uniformly dominated by a Schwartz-class envelope. Hence \( \widetilde{K}_t(x,y) \to \widetilde{K}(x,y) \) pointwise on \( \R^2 \). Moreover, by \cref{lem:uniform_L1_conjugated_kernel}, there exists a fixed \( M(x,y) \in L^1(\R^2) \) such that
\[
|\widetilde{K}_t(x,y)| \le M(x,y) \quad \text{for all } t > 0.
\]

\paragraph{Step 2: Dominated convergence in trace norm.}
By the dominated convergence theorem,
\[
\| \widetilde{K}_t - \widetilde{K} \|_{L^1(\R^2)} \to 0 \quad \text{as } t \to 0^+.
\]
Since \( \widetilde{L}_t \) and \( \widetilde{L} \) are integral operators on \( L^2(\R) \) with respective kernels \( \widetilde{K}_t \) and \( \widetilde{K} \), we apply Simon’s trace-norm kernel criterion~\cite[Thm.~3.1]{Simon2005TraceIdeals}:
\[
\| \widetilde{L}_t - \widetilde{L} \|_{\TC(L^2)} = \| \widetilde{K}_t - \widetilde{K} \|_{L^1(\R^2)} \to 0.
\]

\paragraph{Step 3: Transfer to the weighted space.}
Let \( U \colon \HPsi \to L^2(\R) \) be the unitary transformation \( Uf(x) := \sqrt{\PsiAlpha{x}} f(x) \). Then
\[
L_t = U^{-1} \widetilde{L}_t U, \qquad L_{\sym} = U^{-1} \widetilde{L} U.
\]
By unitary invariance of Schatten norms,
\[
\| L_t - L_{\sym} \|_{\TC(\HPsi)} = \| \widetilde{L}_t - \widetilde{L} \|_{\TC(L^2)} \to 0
\]
as \( t \to 0^+ \). This completes the proof.
\end{proof}


\begin{lemma}[Well-Posedness of the Heat Semigroup \( e^{-t L_{\sym}^2} \)]
\label{lem:heat_semigroup_wellposed}
Let \( L_{\sym} \in \TC(\HPsi) \) be the compact, self-adjoint operator constructed via convolution against the inverse Fourier transform of \( \Xi\left( \tfrac{1}{2} + i\lambda \right) \), as in \lemref{lem:trace_norm_convergence_Lt_to_Lsym}.

Then:

\begin{enumerate}
    \item[\textnormal{(i)}] The square \( L_{\sym}^2 \) is positive, self-adjoint, and densely defined on \( \HPsi \).

    \item[\textnormal{(ii)}] For all \( t > 0 \), the semigroup \( e^{-t L_{\sym}^2} \in \TC(\HPsi) \) is trace class, compact, and analytic in \( t \). This follows from the trace-class convergence of the approximating mollified operators \( L_t \in \TC(\HPsi) \) (see \lemref{lem:trace_class_Lt}) and the exponential decay of their kernels (\lemref{lem:decay_mollified_kernel}).

    \item[\textnormal{(iii)}] The map \( t \mapsto \Tr(e^{-t L_{\sym}^2}) \) is smooth on \( (0, \infty) \), and satisfies the heat trace asymptotics:
    \[
    \Tr(e^{-t L_{\sym}^2}) \lesssim
    \begin{cases}
    t^{-1/2} \log(1/t) & \text{as } t \to 0^+, \\
    e^{-\delta t} & \text{as } t \to \infty,
    \end{cases}
    \]
    for some \( \delta > 0 \).
\end{enumerate}

\medskip
\noindent
These properties ensure the well-definedness of the Laplace representation for the determinant and the spectral zeta function, and justify the analytic continuation framework of Chapter~\ref{sec:determinant_identity}.
\end{lemma}

\begin{proof}[Proof of \cref{lem:heat_semigroup_wellposed}]
Let \( L := L_{\sym} \in \TC(\HPsi) \) be compact, self-adjoint, and defined by convolution with an exponentially decaying, even kernel \( K(x - y) \in \Schwartz(\R) \).

\paragraph{(i) Self-adjointness and positivity of \( L^2 \).}
Since \( L \) is self-adjoint and compact on the Hilbert space \( \HPsi \), its square \( L^2 \) is also self-adjoint and positive (i.e., \( \langle L^2 f, f \rangle \ge 0 \)). The domain of \( L^2 \) is dense, as it includes the Schwartz core preserved under convolution.

\paragraph{(ii) Heat semigroup is trace-class.}
By spectral theory~\cite[Ch.~X, §2]{ReedSimon1975II}, the operator exponential \( e^{-t L^2} \) is well-defined via the spectral calculus for any \( t > 0 \). Since \( L \in \TC \), its spectrum is discrete with eigenvalues \( \{ \mu_n \} \to 0 \), and \( L^2 \) has eigenvalues \( \mu_n^2 \to 0 \). Thus,
\[
e^{-t L^2} = \sum_{n=1}^\infty e^{-t \mu_n^2} P_n,
\]
where \( P_n \) are the orthogonal projections onto the eigenspaces. Because \( \sum_n e^{-t \mu_n^2} < \infty \) for all \( t > 0 \), this implies \( e^{-t L^2} \in \TC(\HPsi) \), trace-class and compact.

\paragraph{(iii) Heat trace asymptotics.}
As shown in \cref{lem:laplace_heat_trace_convergence}, the trace satisfies the expansion
\[
\Tr(e^{-t L^2}) \sim \frac{1}{\sqrt{4\pi t}} \log\left( \frac{1}{t} \right) + \mathcal{O}(t^{-1/2}) \quad \text{as } t \to 0^+,
\]
derived via Fourier analysis and the Paley--Wiener decay of the kernel. Meanwhile, the discrete spectrum \( \{ \mu_n \} \subset \R \setminus \{0\} \) ensures that
\[
\Tr(e^{-t L^2}) \le \sum_n e^{-t \mu_n^2} \lesssim e^{-\delta t} \quad \text{as } t \to \infty,
\]
for some \( \delta > 0 \). This ensures absolute convergence of the Laplace integral used in the determinant representation (cf. \cref{lem:det_via_heat_trace}).

\paragraph{Conclusion.}
Thus, \( e^{-t L_{\sym}^2} \in \TC(\HPsi) \) for all \( t > 0 \), the trace map is smooth on \( (0, \infty) \), and the semigroup is strongly continuous and analytic in \( t \), completing the proof.
\end{proof}


\begin{remark}[Logarithmic Singularity and Spectral Zeta Link]
The Laplace transform of the trace \( \Tr(e^{-t L_{\sym}^2}) \) exhibits a logarithmic singularity at \( s = 0 \), reflecting the non-analytic behavior of the spectral zeta function near the origin. This singularity underlies the log-derivative structure of the canonical determinant and is essential to matching the analytic continuation of the completed zeta function \( \Xi(s) \) via the Carleman determinant identity. For precise statements, see \cref{lem:det_via_heat_trace} and \cref{lem:A_log_derivative}.
\end{remark}


\subsection{Determinant Construction and Growth}
\label{sec:det_growth}

\begin{lemma}[Construction of \(\zeta\)-Regularized Determinant via Heat Trace]
\label{lem:det-via-heat-trace}
Let \( L_{\mathrm{sym}} \in \mathcal{C}_1(H_{\Psi_\alpha}) \) be a compact, self-adjoint, trace-class operator on the exponentially weighted Hilbert space
\[
H_{\Psi_\alpha} := L^2(\mathbb{R}, e^{\alpha|x|} \, dx), \qquad \alpha > \pi.
\]

Then:
\begin{enumerate}
  \item[\textnormal{(i)}] For all \( \lambda \in \mathbb{C} \) with \( |\lambda| < \|L_{\mathrm{sym}}\|^{-1} \), the Carleman \(\zeta\)-regularized determinant admits a convergent trace expansion:
  \[
  \log \det\nolimits_{\zeta}(I - \lambda L_{\mathrm{sym}})
  = - \sum_{n=1}^\infty \frac{\lambda^n}{n} \operatorname{Tr}(L_{\mathrm{sym}}^n),
  \]
  with absolute convergence ensured by the trace-class property of \( L_{\mathrm{sym}} \).

  \item[\textnormal{(ii)}] The function \( \lambda \mapsto \log \det\nolimits_\zeta(I - \lambda L_{\mathrm{sym}}) \) admits an entire analytic continuation to \( \mathbb{C} \), represented by the Laplace transform of the heat trace:
  \[
  \log \det\nolimits_\zeta(I - \lambda L_{\mathrm{sym}})
  = - \int_0^\infty \frac{e^{-\lambda t}}{t} \operatorname{Tr}(e^{-t L_{\mathrm{sym}}}) \, dt.
  \]
  This integral converges absolutely for all \( \lambda \in \mathbb{C} \), since
  \[
  \operatorname{Tr}(e^{-t L_{\mathrm{sym}}}) \lesssim t^{-1} e^{-c/t}
  \quad \text{as } t \to 0^+,
  \]
  and decays exponentially as \( t \to \infty \).
\end{enumerate}

\medskip
\noindent
This identity expresses the \(\zeta\)-regularized determinant in terms of the trace of the heat semigroup \( e^{-t L_{\mathrm{sym}}} \), enabling analytic continuation, entire order classification, and spectral asymptotics to be developed in subsequent sections.
\end{lemma}
% 
\begin{proof}[Proof of \lemref{lem:det_via_heat_trace}]
Let \( L := \Lsym \in \TC(H_{\Psi_\alpha}) \) be compact and self-adjoint.

\paragraph{(i) Local power series expansion.}
Since \( L \in \TC \), each power \( L^n \in \TC \), and the trace-logarithmic identity
\[
\log \detz(I - \lambda L)
= - \sum_{n=1}^\infty \frac{\lambda^n}{n} \Tr(L^n)
\]
converges absolutely for \( |\lambda| < \|L\|^{-1} \). This follows from submultiplicativity of Schatten norms and the estimate \( \|L^n\|_{\TC} \le \|L\|^n \). Hence, \( \log \detz(I - \lambda L) \) defines a holomorphic function near \( \lambda = 0 \), consistent with classical Fredholm determinant theory.

\paragraph{(ii) Analytic continuation via heat trace.}
By spectral theory for compact self-adjoint operators, the heat semigroup \( e^{-tL} \in \TC(H_{\Psi_\alpha}) \) for all \( t > 0 \), and the trace
\[
\Tr(e^{-tL}) = \sum_{n=1}^\infty e^{-t\lambda_n}
\]
is finite. From small-time asymptotics (see \secref{sec:heat_kernel_asymptotics}), we have
\[
\Tr(e^{-tL}) \lesssim t^{-1} e^{-c/t} \quad \text{as } t \to 0^+,
\]
for some \( c > 0 \), and exponential decay as \( t \to \infty \). These bounds guarantee convergence of the Laplace representation.

\paragraph{Conclusion.}
Thus, the determinant admits the integral representation
\[
\log \detz(I - \lambda L)
= - \int_0^\infty \frac{e^{-\lambda t}}{t} \Tr(e^{-tL}) \, dt,
\]
which converges absolutely for all \( \lambda \in \mathbb{C} \). This representation provides an analytic continuation of \( \log \detz(I - \lambda L) \) to an entire function of exponential type, extending the local trace expansion globally.
\end{proof}


\begin{lemma}[Laplace Representation Preserves Entire Function Order and Type]
\label{lem:laplace_preserves_entire_type}
Let \( L \in \mathcal{C}_1(H_{\Psi_\alpha}) \) be a compact, self-adjoint operator such that the heat trace satisfies
\[
\operatorname{Tr}(e^{-tL}) \le A t^{-1} e^{-c/t}, \quad \text{as } t \to 0^+,
\]
and
\[
\operatorname{Tr}(e^{-tL}) \le B e^{-\delta t}, \quad \text{as } t \to \infty,
\]
for some constants \( A,B,c,\delta > 0 \).

Then the Laplace integral
\[
\log \det\nolimits_\zeta(I - \lambda L)
= - \int_0^\infty \frac{e^{-\lambda t}}{t} \operatorname{Tr}(e^{-tL})\, dt
\]
defines an entire function of order one and exponential type bounded by \( \pi \). That is,
\[
\left| \log \det\nolimits_\zeta(I - \lambda L) \right| \le C |\lambda| \log(1 + |\lambda|),
\]
and
\[
\left| \det\nolimits_\zeta(I - \lambda L) \right| \le C' e^{\pi |\lambda|}.
\]
\end{lemma}

\medskip
\noindent
The bounds on the heat trace ensure the Laplace transform converges absolutely and uniformly on compact subsets of \( \mathbb{C} \), and Paley–Wiener-type estimates control the exponential type.
%  
\begin{proof}[Proof of Lemma~\ref{lem:laplace_preserves_entire_type}]
Let
\[
f(\lambda) := \log \det\nolimits_\zeta(I - \lambda L)
= - \int_0^\infty \frac{e^{-\lambda t}}{t} \operatorname{Tr}(e^{-tL})\, dt.
\]

\paragraph{(i) Convergence and analyticity.}
From the short-time estimate \( \operatorname{Tr}(e^{-tL}) \le A t^{-1} e^{-c/t} \), we see that the integrand is dominated by \( A t^{-2} e^{-c/t} e^{|\lambda| t} \) for small \( t \). Since \( e^{-c/t} \) decays faster than any polynomial in \( t \) as \( t \to 0^+ \), the integral is absolutely convergent for all \( \lambda \in \mathbb{C} \).

From the long-time estimate \( \operatorname{Tr}(e^{-tL}) \le B e^{-\delta t} \), the integrand decays like \( e^{-(\delta - |\lambda|) t} \) for large \( t \), ensuring absolute convergence and entire analyticity in \( \lambda \).

\paragraph{(ii) Growth and order estimate.}
We now estimate
\[
|f(\lambda)| = \left| \int_0^\infty \frac{e^{-\lambda t}}{t} \operatorname{Tr}(e^{-tL}) \, dt \right|.
\]
Split the integral at \( t = 1 \):
\[
\int_0^1 \frac{e^{|\lambda| t}}{t} \cdot A t^{-1} e^{-c/t} dt \le C_1 |\lambda|,
\qquad
\int_1^\infty \frac{e^{|\lambda| t}}{t} \cdot B e^{-\delta t} dt \le C_2 \log(1 + |\lambda|),
\]
for constants \( C_1, C_2 > 0 \). Thus
\[
|f(\lambda)| \le C |\lambda| \log(1 + |\lambda|),
\]
which implies that \( f \) is entire of order one. Since \( \exp(f(\lambda)) = \det\nolimits_\zeta(I - \lambda L) \), the exponential type is controlled by the location of singularities in the Laplace variable, and bounded above by the exponential type of the Fourier transform of the kernel (which is supported in \( [-\pi, \pi] \)), yielding exponential type \( \le \pi \).

\paragraph{Conclusion.}
The determinant is an entire function of order one and exponential type \( \le \pi \), as claimed.
\end{proof}
%  

\begin{lemma}[Exponential Growth Bound for the Determinant]
\label{lem:det_growth_bound}
Let \( L_{\mathrm{sym}} \in \mathcal{C}_1(H_{\Psi_\alpha}) \) be a compact, self-adjoint, trace-class operator.

Then the function
\[
\lambda \mapsto \det\nolimits_{\zeta}(I - \lambda L_{\mathrm{sym}})
\]
is an entire function of order one and finite exponential type. Moreover, there exists a constant \( C > 0 \) such that for all \( \lambda \in \mathbb{C} \),
\[
\log \left| \det\nolimits_{\zeta}(I - \lambda L_{\mathrm{sym}}) \right|
\le C\, |\lambda| \log(1 + |\lambda|).
\]

\medskip
\noindent
That is, \( \det_\zeta(I - \lambda L_{\mathrm{sym}}) \in \mathcal{E}_1 \), the class of entire functions of order one and finite type. The constant \( C \) may be chosen depending only on \( \|L_{\mathrm{sym}}\|_{\mathcal{C}_1} \) and the spectral radius of \( L_{\mathrm{sym}} \).
\end{lemma}
%  
\begin{proof}[Proof of \lemref{lem:det_growth_bound}]
Let \( L := \Lsym \in \TC(\HPsi) \) be compact and self-adjoint.

\paragraph{(i) Local power series estimate.}
By \cref{lem:det_via_heat_trace}, the zeta-regularized determinant admits the trace-logarithmic expansion
\[
\log \detz(I - \lambda L)
= - \sum_{n=1}^\infty \frac{\lambda^n}{n} \Tr(L^n),
\]
which converges absolutely for \( |\lambda| < \|L\|^{-1} \). Since \( |\Tr(L^n)| \le \|L\|^n \), we obtain the estimate
\[
\left| \log \detz(I - \lambda L) \right|
\le \sum_{n=1}^\infty \frac{|\lambda|^n \|L\|^n}{n}
= -\log(1 - |\lambda| \|L\|),
\]
which controls growth near the origin.

\paragraph{(ii) Global growth bound via Laplace representation.}
For all \( \lambda \in \C \), the determinant also admits the Laplace representation
\[
\log \detz(I - \lambda L)
= - \int_0^\infty \frac{e^{-\lambda t}}{t} \, \Tr(e^{-tL}) \, dt.
\]
From the heat trace asymptotics in \cref{sec:heat_kernel_asymptotics}, there exist constants \( C_1, C_2, c, \delta > 0 \) such that
\[
\Tr(e^{-tL}) \le C_1 t^{-1} e^{-c/t} \quad \text{as } t \to 0^+, \qquad
\Tr(e^{-tL}) \le C_2 e^{-\delta t} \quad \text{as } t \to \infty.
\]

Split the Laplace integral at \( t = 1 \). Then:
\begin{align*}
\left| \int_0^1 \frac{e^{-\lambda t}}{t} \Tr(e^{-tL})\, dt \right|
&\le C_1 \int_0^1 \frac{e^{|\lambda| t} e^{-c/t}}{t^2} \, dt = \mathcal{O}(|\lambda|), \\
\left| \int_1^\infty \frac{e^{-\lambda t}}{t} \Tr(e^{-tL})\, dt \right|
&\le C_2 \int_1^\infty \frac{e^{(|\lambda| - \delta)t}}{t} \, dt = \mathcal{O}\big(|\lambda| \log(1 + |\lambda|)\big).
\end{align*}

\paragraph{Conclusion.}
Combining both contributions yields the estimate
\[
\log \left| \detz(I - \lambda L) \right| \le C\, |\lambda| \log(1 + |\lambda|),
\]
for some constant \( C > 0 \). Hence \( \detz(I - \lambda L) \) is an entire function of order one and finite exponential type, as claimed.
\end{proof}


\begin{lemma}[Determinant Identity Defines an Entire Function of Order One and Type \( \pi \)]
\label{lem:det_identity_entire_order_one}

Let \( L_{\mathrm{sym}} \in \mathcal{C}_1(H_{\Psi_\alpha}) \) be a compact, self-adjoint, trace-class operator on the weighted Hilbert space
\[
H_{\Psi_\alpha} := L^2(\mathbb{R}, e^{\alpha |x|} dx), \qquad \text{for } \alpha > \pi.
\]

Then the map
\[
\lambda \mapsto \det\nolimits_\zeta(I - \lambda L_{\mathrm{sym}})
\]
extends to an entire function on \( \mathbb{C} \) of order one and exponential type \( \pi \). That is, there exists a constant \( C > 0 \) such that for all \( \lambda \in \mathbb{C} \),
\[
\left| \det\nolimits_\zeta(I - \lambda L_{\mathrm{sym}}) \right| \le C e^{\pi |\lambda|}.
\]
In particular,
\[
\det\nolimits_\zeta(I - \lambda L_{\mathrm{sym}}) \in \mathcal{E}_1^\pi,
\]
the Hadamard class of entire functions of order one and exact exponential type \( \pi \).

\medskip
\noindent
The exponential type is governed by the Paley--Wiener theorem~\cite[Ch.~9]{Levin1996EntireLectures}, which bounds the growth of entire functions whose Fourier transforms have compact support. In this case, the support of the kernel’s Fourier transform \( \widehat{\phi} \), inherited from the exponential type of \( \Xi(s) \), restricts the convolution kernel \( K(x - y) \) to type \( \pi \). This control on kernel growth ensures that the Fredholm determinant exhibits exactly exponential type \( \pi \).
\end{lemma}

\begin{proof}[Proof of \lemref{lem:det_identity_entire_order_one}]
Let \( L := L_{\mathrm{sym}} \in \mathcal{C}_1(H_{\Psi_\alpha}) \) be compact and self-adjoint. Then the zeta-regularized determinant admits the trace-logarithmic expansion:
\[
\det\nolimits_\zeta(I - \lambda L)
= \exp\left( - \sum_{n=1}^\infty \frac{\lambda^n}{n} \operatorname{Tr}(L^n) \right),
\]
which converges absolutely for all \( \lambda \in \mathbb{C} \), since
\[
|\operatorname{Tr}(L^n)| \le \|L^n\|_{\mathcal{C}_1} \le \|L\|^n.
\]
Hence, \( \det\nolimits_\zeta(I - \lambda L) \) defines an entire function on \( \mathbb{C} \).

\paragraph{Growth bound.}
By \lemref{lem:det_growth_bound}, the determinant satisfies the exponential growth estimate:
\[
\log \left| \det\nolimits_\zeta(I - \lambda L) \right|
\le C |\lambda| \log(1 + |\lambda|),
\]
for some constant \( C > 0 \), implying the function is entire of order one.

\paragraph{Exponential type via Fourier decay.}
The operator \( L \) is a convolution operator with kernel \( K(x - y) \), whose Fourier transform is the centered profile
\[
\widehat{K}(\lambda) = \Xi\left( \tfrac{1}{2} + i\lambda \right).
\]
Since \( \Xi(s) \) is entire of exponential type \( \pi \), the Paley--Wiener theorem~\cite[Ch.~9]{Levin1996EntireLectures} implies that \( K \in L^2(\mathbb{R}) \) has frequency support in \( [-\pi, \pi] \). Consequently, \( K(x) \) decays exponentially as \( |x| \to \infty \) with rate at least \( \pi - \varepsilon \). This decay transfers to the entire function structure of the determinant via trace-kernel correspondence~\cite[Ch.~3--4]{Simon2005TraceIdeals}.

\paragraph{Conclusion.}
The determinant \( \det\nolimits_\zeta(I - \lambda L) \) is entire of order one and exponential type \( \pi \), that is,
\[
\det\nolimits_\zeta(I - \lambda L) \in \mathcal{E}_1^\pi,
\]
as claimed.
\end{proof}


\begin{lemma}\label{lem:laplace-heat-trace-convergence}
Let \( L_{\mathrm{sym}} \in \mathcal{C}_1(H_{\Psi_\alpha}) \) be the canonical operator. Then for all \( \lambda \in \mathbb{C} \), the Laplace representation
\begin{equation}
\det\nolimits_\zeta(I - \lambda L_{\mathrm{sym}}) = \exp\left(- \int_0^\infty \frac{e^{-\lambda^2 t}}{t} \operatorname{Tr}(e^{-t L_{\mathrm{sym}}^2}) \, dt \right)
\end{equation}
is absolutely convergent, and defines an entire function of order one and exact exponential type \( \pi \).
\end{lemma}
%  
\begin{proof}[Proof of Lemma~\ref{lem:laplace_heat_trace_convergence}]
Let \( L_{\mathrm{sym}} \in \mathcal{C}_1(H_{\Psi_\alpha}) \) be the canonical operator as constructed in Chapter~\ref{sec:operator_construction}. For all \( t > 0 \), the operator \( e^{-t L_{\mathrm{sym}}^2} \) is trace-class by the spectral theorem and closure of \( \mathcal{C}_1 \) under holomorphic functional calculus. In particular, the trace function
\[
t \mapsto \operatorname{Tr}(e^{-t L_{\mathrm{sym}}^2})
\]
is smooth and positive for \( t > 0 \), and the semigroup \( \{ e^{-t L_{\mathrm{sym}}^2} \}_{t>0} \) is strongly continuous and holomorphic in \( t \).

\smallskip
\noindent
From the short-time heat kernel expansion in Chapter~\ref{sec:heat_kernel_asymptotics}, we know:
\[
\operatorname{Tr}(e^{-t L_{\mathrm{sym}}^2}) \lesssim t^{-1/2} \quad \text{as } t \to 0^+,
\]
with constants uniform in compact neighborhoods. Moreover, from the exponential decay of the kernel spectrum and the smoothing property of \( L_{\mathrm{sym}}^2 \), we have rapid decay:
\[
\operatorname{Tr}(e^{-t L_{\mathrm{sym}}^2}) \lesssim e^{-ct} \quad \text{as } t \to \infty,
\]
for some constant \( c > 0 \).

\smallskip
\noindent
The Laplace transform defining the Carleman determinant then reads:
\[
\det\nolimits_\zeta(I - \lambda L_{\mathrm{sym}}) = \exp\left(- \int_0^\infty \frac{e^{-\lambda^2 t}}{t} \operatorname{Tr}(e^{-t L_{\mathrm{sym}}^2}) \, dt \right),
\]
and converges absolutely for all \( \lambda \in \mathbb{C} \) by comparison to \( \int_0^\infty t^{-3/2} e^{-\lambda^2 t} dt < \infty \), using the short-time asymptotic bound.

\smallskip
\noindent
Finally, since the Laplace transform of a trace-class heat kernel defines an entire function whose order is governed by the exponential decay of the trace, and the exponential type of the spectral kernel is exactly \( \pi \) (see Lemma~\ref{lem:det_identity_entire_order_one}), the determinant is entire of order one and exponential type \( \pi \), completing the proof.
\end{proof}
%  

\begin{lemma}[Spectral Zeta Function from Heat Trace]
\label{lem:spectral_zeta_from_heat}
Let \( L_{\mathrm{sym}} \in \mathcal{C}_1(H_{\Psi_\alpha}) \) be a compact, self-adjoint, positive operator with discrete, nonzero eigenvalues \( \{ \mu_n \}_{n=1}^\infty \subset (0, \infty) \), and let
\[
\zeta_{L_{\mathrm{sym}}^2}(s) := \sum_{n=1}^\infty \mu_n^{-2s}
\]
denote the spectral zeta function of \( L_{\mathrm{sym}}^2 \).

Then:
\begin{enumerate}
    \item[\textnormal{(i)}] The function \( \zeta_{L_{\mathrm{sym}}^2}(s) \) admits the Mellin representation:
    \[
    \zeta_{L_{\mathrm{sym}}^2}(s)
    = \frac{1}{\Gamma(s)} \int_0^\infty t^{s-1} \operatorname{Tr}(e^{-t L_{\mathrm{sym}}^2}) \, dt,
    \]
    which converges absolutely for \( \Re(s) \gg 1 \) and admits meromorphic continuation elsewhere, provided the heat trace has suitable asymptotics.

    \item[\textnormal{(ii)}] If the short-time heat trace satisfies
    \[
    \operatorname{Tr}(e^{-t L_{\mathrm{sym}}^2}) \sim \frac{1}{\sqrt{4\pi t}} \log\left(\frac{1}{t}\right)
    \quad \text{as } t \to 0^+,
    \]
    then \( \zeta_{L_{\mathrm{sym}}^2}(s) \) extends meromorphically to \( \mathbb{C} \), with a logarithmic branch point at \( s = \tfrac{1}{2} \), and possibly at \( s = 0 \) due to infrared behavior.
\end{enumerate}

\medskip
\noindent
This Mellin correspondence connects heat kernel asymptotics with the analytic structure of the spectral zeta function, enabling determinant classification via the Hadamard–Carleman framework.
\end{lemma}
% 
\begin{proof}[Proof of \lemref{lem:spectral_zeta_from_heat}]
Let \( \{ \mu_n \} \subset (0, \infty) \) be the eigenvalues of \( L_{\mathrm{sym}} \), so that \( \{ \mu_n^2 \} \) are the eigenvalues of \( L_{\mathrm{sym}}^2 \). The spectral zeta function
\[
\zeta_{L_{\mathrm{sym}}^2}(s) := \sum_{n=1}^\infty \mu_n^{-2s}
\]
converges for \( \operatorname{Re}(s) > s_0 \), for some \( s_0 > 0 \) depending on the spectral decay of \( \mu_n \).

\paragraph{(i) Mellin representation.}
Using the identity
\[
\mu_n^{-2s} = \frac{1}{\Gamma(s)} \int_0^\infty t^{s-1} e^{-t \mu_n^2} \, dt,
\]
and applying Fubini’s theorem (justified by positivity and trace convergence), we exchange summation and integration:
\[
\zeta_{L_{\mathrm{sym}}^2}(s)
= \frac{1}{\Gamma(s)} \int_0^\infty t^{s-1} \sum_{n=1}^\infty e^{-t \mu_n^2} \, dt
= \frac{1}{\Gamma(s)} \int_0^\infty t^{s-1} \operatorname{Tr}(e^{-t L_{\mathrm{sym}}^2}) \, dt,
\]
valid for \( \operatorname{Re}(s) \gg 1 \), where the integrand is smooth and integrable.

\paragraph{(ii) Meromorphic continuation via heat trace asymptotics.}
Assume the short-time expansion
\[
\operatorname{Tr}(e^{-t L_{\mathrm{sym}}^2}) \sim \frac{1}{\sqrt{4\pi t}} \log\left( \frac{1}{t} \right)
\quad \text{as } t \to 0^+.
\]
This induces a logarithmic singularity in the Mellin integral near \( s = \tfrac{1}{2} \), since
\[
\int_0^\epsilon t^{s - \frac{3}{2}} \log\left( \tfrac{1}{t} \right) \, dt
\]
diverges logarithmically as \( s \to \tfrac{1}{2} \). Therefore, \( \zeta_{L_{\mathrm{sym}}^2}(s) \) has a logarithmic branch point at \( s = \tfrac{1}{2} \).

For large \( t \), the decay \( \operatorname{Tr}(e^{-t L_{\mathrm{sym}}^2}) \lesssim e^{-\delta t} \) ensures holomorphy for \( \operatorname{Re}(s) \ll 0 \), modulo potential divergence at \( s = 0 \). Thus, \( \zeta_{L_{\mathrm{sym}}^2}(s) \) extends meromorphically to \( \mathbb{C} \), with the stated singularities.
\end{proof}


\begin{lemma}[Continuity of the Determinant under Trace-Norm Limits]
\label{lem:det_continuity_trace_norm}
Let \( \{ T_n \} \subset \TC(H) \) be a sequence of trace-class operators on a separable Hilbert space \( H \), and let \( T \in \TC(H) \) such that
\[
\| T_n - T \|_{\TC} \to 0 \quad \text{as } n \to \infty.
\]

Then:
\begin{enumerate}
  \item[\textnormal{(i)}] For all \( \lambda \in \C \), we have
  \[
  \det\nolimits_\zeta(I - \lambda T_n) \to \det\nolimits_\zeta(I - \lambda T),
  \]
  with convergence uniform on compact subsets of \( \C \).

  \item[\textnormal{(ii)}] The logarithmic derivatives converge:
  \[
  \frac{d}{d\lambda} \log \det\nolimits_\zeta(I - \lambda T_n)
  \to \frac{d}{d\lambda} \log \det\nolimits_\zeta(I - \lambda T),
  \]
  uniformly on compact subsets where \( I - \lambda T \) is invertible.

  \item[\textnormal{(iii)}] If each \( T_n = T_n^* \) and \( T = T^* \), then the convergence holds in the Hadamard class \( \mathcal{E}_1^\tau \) of entire functions of order one and exponential type
  \[
  \tau := \limsup_{n \to \infty} \| T_n \|_{\TC}.
  \]
\end{enumerate}

\medskip
\noindent
This lemma guarantees that determinant constructions are stable under trace-norm approximation, as arises in the mollification limit of spectral convolution operators.
\end{lemma}

\begin{proof}[Proof of \cref{lem:det_continuity_trace_norm}]
Let \( \{ T_n \} \subset \TC(H) \) be a sequence of trace-class operators converging to \( T \in \TC(H) \) in trace norm:
\[
\| T_n - T \|_{\TC} \to 0 \quad \text{as } n \to \infty.
\]

\paragraph{(i) Convergence of determinants.}
The zeta-regularized determinant admits the absolutely convergent trace expansion:
\[
\log \det\nolimits_\zeta(I - \lambda T_n)
= - \sum_{k=1}^\infty \frac{\lambda^k}{k} \Tr(T_n^k).
\]
Trace-norm convergence implies that for all \( k \in \N \), we have \( T_n^k \to T^k \) in \( \TC \), since \( \TC \) is a Banach algebra. Therefore,
\[
\Tr(T_n^k) \to \Tr(T^k).
\]
The convergence is uniform on compact subsets of \( \C \), so by the Weierstrass theorem,
\[
\det\nolimits_\zeta(I - \lambda T_n) \to \det\nolimits_\zeta(I - \lambda T),
\]
locally uniformly in \( \lambda \in \C \).

\paragraph{(ii) Logarithmic derivative convergence.}
Differentiating term-by-term gives
\[
\frac{d}{d\lambda} \log \det\nolimits_\zeta(I - \lambda T_n)
= - \sum_{k=1}^\infty \lambda^{k-1} \Tr(T_n^k),
\]
which converges uniformly on compact subsets of \( \C \) avoiding the poles of the resolvent, i.e., where \( \lambda^{-1} \notin \Spec(T) \). Since \( \Tr(T_n^k) \to \Tr(T^k) \) for each \( k \), the derivative series converges uniformly as well:
\[
\frac{d}{d\lambda} \log \det\nolimits_\zeta(I - \lambda T_n)
\to \frac{d}{d\lambda} \log \det\nolimits_\zeta(I - \lambda T).
\]

\paragraph{(iii) Entire function class convergence.}
If each \( T_n = T_n^* \) and \( T = T^* \), and the sequence \( \{ T_n \} \) is uniformly bounded in trace norm, then by classical results on convergence in the Hadamard class of entire functions~\cite[Ch.~1]{Levin1996EntireLectures}, we obtain convergence
\[
\det\nolimits_\zeta(I - \lambda T_n) \to \det\nolimits_\zeta(I - \lambda T)
\quad \text{in } \mathcal{E}_1^\tau,
\]
where
\[
\tau := \limsup_{n \to \infty} \| T_n \|_{\TC}.
\]
This completes the proof.
\end{proof}


\subsection{Hadamard Structure and Normalization}
\label{sec:hadamard_normalization}

\begin{lemma}[Hadamard Factorization of \( \Xi\left(\tfrac{1}{2} + i\lambda\right) \)]
\label{lem:hadamard_linear_form}
Let \( \Xi(s) \) denote the completed Riemann zeta function. Then the shifted entire function
\[
\lambda \mapsto \Xi\left( \tfrac{1}{2} + i\lambda \right)
\]
is of order one and genus one, and admits the canonical Hadamard factorization:
\[
\Xi\left( \tfrac{1}{2} + i\lambda \right)
= \Xi\left( \tfrac{1}{2} \right)
\prod_{\rho \ne \tfrac{1}{2}} \left( 1 - \frac{\lambda}{i(\rho - \tfrac{1}{2})} \right)
\exp\left( \frac{\lambda}{i(\rho - \tfrac{1}{2})} \right),
\]
where the product is taken over all nontrivial zeros \( \rho \in \mathbb{C} \) of \( \zeta(s) \), counted with multiplicity.

\medskip
\noindent
This factorization is canonical in the Hadamard sense: since \( \Xi(s) \) is entire of order one, the associated genus is also one, and the minimal Weierstrass primary factor is of the form
\[
E_1(z) = (1 - z) \exp(z).
\]
The exponential terms arise from this genus-one constraint.

\medskip
\noindent
The product converges absolutely and uniformly on compact subsets of \( \mathbb{C} \), and the symmetry \( \Xi(s) = \Xi(1 - s) \) implies
\[
\Xi\left( \tfrac{1}{2} + i\lambda \right) = \Xi\left( \tfrac{1}{2} - i\lambda \right),
\]
so the factorization is invariant under \( \lambda \mapsto -\lambda \), and all zeros appear in symmetric pairs about the origin.
\end{lemma}
%  
\begin{proof}[Proof of \cref{lem:hadamard_linear_form}]
Let \( F(\lambda) := \Xi\left( \tfrac{1}{2} + i\lambda \right) \). Since \( \Xi(s) \) is entire of order one and genus one, Hadamard’s factorization theorem~\cite[Thm.~3.7.1]{Levin1996EntireLectures} implies
\[
\Xi(s) = e^{A + B s} \prod_{\rho} \left( 1 - \frac{s}{\rho} \right) \exp\left( \frac{s}{\rho} \right),
\]
where the product is over all nontrivial zeros \( \rho \in \C \) of \( \zeta(s) \), counted with multiplicity. This product converges absolutely since \( \sum_\rho |\rho|^{-2} < \infty \), a consequence of the order-one growth of \( \Xi \).

\smallskip
Define the spectral shift \( \lambda := -i(s - \tfrac{1}{2}) \), so that \( s = \tfrac{1}{2} + i\lambda \). Then each nontrivial zero \( \rho \ne \tfrac{1}{2} \) maps to
\[
\lambda_\rho := i(\rho - \tfrac{1}{2}),
\]
and the zero set of \( F(\lambda) \) is exactly \( \{ \lambda_\rho \}_{\rho \ne \frac{1}{2}} \), symmetric about the origin.

Substituting into Hadamard’s product yields
\[
F(\lambda) = e^{C_0 + C_1 \lambda} \prod_{\rho \ne \frac{1}{2}} \left( 1 - \frac{\lambda}{\lambda_\rho} \right) \exp\left( \frac{\lambda}{\lambda_\rho} \right),
\]
for some constants \( C_0, C_1 \in \C \).

\smallskip
By the functional equation \( \Xi(s) = \Xi(1 - s) \), it follows that \( F(\lambda) = F(-\lambda) \), i.e., \( F \) is even. This symmetry forces \( C_1 = 0 \), and the product simplifies:
\[
F(\lambda) = e^{C_0} \prod_{\rho \ne \frac{1}{2}} \left( 1 - \frac{\lambda}{\lambda_\rho} \right) \exp\left( \frac{\lambda}{\lambda_\rho} \right).
\]

Evaluating at \( \lambda = 0 \), we have
\[
F(0) = \Xi\left( \tfrac{1}{2} \right) = e^{C_0},
\]
so \( C_0 = \log \Xi\left( \tfrac{1}{2} \right) \), and therefore:
\[
\Xi\left( \tfrac{1}{2} + i\lambda \right)
= \Xi\left( \tfrac{1}{2} \right) \prod_{\rho \ne \frac{1}{2}} \left( 1 - \frac{\lambda}{i(\rho - \tfrac{1}{2})} \right) \exp\left( \frac{\lambda}{i(\rho - \tfrac{1}{2})} \right),
\]
which is the canonical Hadamard factorization of \( F(\lambda) \), completing the proof.
\end{proof}


\begin{lemma}[Vanishing Trace of \( L_{\mathrm{sym}} \)]
\label{lem:trace_zero}
Let \( L_{\mathrm{sym}} \in \mathcal{C}_1(H_{\Psi_\alpha}) \) denote the canonical compact, self-adjoint convolution operator defined via the inverse Fourier transform of the completed Riemann zeta function:
\[
\phi(\lambda) := \Xi\left( \tfrac{1}{2} + i\lambda \right), \qquad
K_{\mathrm{sym}}(x,y) := \widehat{\phi}(x - y).
\]

Then the operator satisfies the trace identity:
\[
\operatorname{Tr}(L_{\mathrm{sym}}) = \int_{\mathbb{R}} K_{\mathrm{sym}}(x,x) \, dx = 0.
\]

\medskip
\noindent
\textbf{Justification.}
Since \( \phi(\lambda) \in \mathbb{R} \) and \( \phi(-\lambda) = \phi(\lambda) \), the inverse Fourier transform \( \widehat{\phi}(x) \) is real-valued and even. Hence, the diagonal kernel value
\[
K_{\mathrm{sym}}(x,x) = \widehat{\phi}(0)
\]
is constant. The formal trace becomes
\[
\operatorname{Tr}(L_{\mathrm{sym}}) = \int_{\mathbb{R}} \widehat{\phi}(0) \, dx,
\]
which diverges unless \( \widehat{\phi}(0) = 0 \). But by Fourier inversion,
\[
\widehat{\phi}(0) = \frac{1}{2\pi} \int_{\mathbb{R}} \phi(\lambda) \, d\lambda = 0,
\]
because \( \phi(\lambda) \) is even, entire of exponential type \( \pi \), and has vanishing average due to symmetry and decay of the zeros of \( \Xi \) (cf.~\cite[Ch.~3]{Levin1996EntireLectures}).

\medskip
\noindent
\textbf{Spectral Consequence.}
This vanishing ensures that the logarithmic derivative of the canonical determinant \( \log \det_\zeta(I - \lambda L_{\mathrm{sym}}) \) has no linear term, i.e., no \( \lambda \)-term in its Taylor expansion. Hence, the determinant belongs to the Hadamard class \( \mathcal{E}_1^\pi \) with normalization \( f(0) = 1 \), completing its spectral identification with the zeta profile.

\end{lemma}

\begin{proof}[Proof of Lemma~\ref{lem:trace-zero}]
Let \( L_{\mathrm{sym}} \in \mathcal{C}_1(H_{\Psi_\alpha}) \) be the canonical self-adjoint convolution operator with kernel
\[
K(x - y) := \widehat{\Xi}(x - y),
\]
where
\[
\widehat{\Xi}(x) := \frac{1}{2\pi} \int_{\mathbb{R}} e^{i\lambda x} \Xi\left( \tfrac{1}{2} + i\lambda \right) \, d\lambda
\]
is the inverse Fourier transform of the completed zeta profile.

\paragraph{(i) Symmetry of the Kernel.}
The function
\[
\phi(\lambda) := \Xi\left( \tfrac{1}{2} + i\lambda \right)
\]
is real-valued and even, due to the functional equation \( \Xi(s) = \Xi(1 - s) \) and the fact that \( \Xi \) is real on the critical line. Therefore, \( \widehat{\Xi}(x) \in \mathcal{S}(\mathbb{R}) \) is real-valued and even as well. In particular, \( K(x, x) = \widehat{\Xi}(0) \) is constant.

\paragraph{(ii) Kernel Trace Heuristic.}
Formally, the kernel trace of \( L_{\mathrm{sym}} \) would be
\[
\operatorname{Tr}(L_{\mathrm{sym}}) = \int_{\mathbb{R}} K(x, x)\, dx = \widehat{\Xi}(0) \cdot \int_{\mathbb{R}} dx.
\]
This diverges unless \( \widehat{\Xi}(0) = 0 \), but this argument is heuristic and not valid in the trace-class context on weighted spaces.

\paragraph{(iii) Spectral Interpretation via Hadamard Structure.}
The determinant
\[
\det\nolimits_\zeta(I - \lambda L_{\mathrm{sym}}) = \frac{\Xi\left( \tfrac{1}{2} + i\lambda \right)}{\Xi\left( \tfrac{1}{2} \right)}
\]
is an entire function of exponential type \( \pi \) and genus one, with Hadamard factorization
\[
\Xi\left( \tfrac{1}{2} + i\lambda \right)
= \Xi\left( \tfrac{1}{2} \right)
\prod_\rho \left(1 - \frac{\lambda}{i(\rho - \tfrac{1}{2})} \right)
\exp\left( \frac{\lambda}{i(\rho - \tfrac{1}{2})} \right).
\]
The logarithmic derivative satisfies
\[
\frac{d}{d\lambda} \log \Xi\left( \tfrac{1}{2} + i\lambda \right)
= \sum_\rho \frac{1}{\lambda - i(\rho - \tfrac{1}{2})},
\]
with no term of the form \( \lambda^{-1} \) appearing. Since such a term would contribute a nonzero trace in the spectral expansion of the logarithmic derivative, its absence implies:
\[
\operatorname{Tr}(L_{\mathrm{sym}}) = \sum_n \lambda_n = 0.
\]

\paragraph{Conclusion.}
Thus, the vanishing trace condition follows from the Hadamard factorization and canonical normalization of the determinant. It ensures
\[
\det\nolimits_\zeta(I - \lambda L_{\mathrm{sym}}) = \frac{\Xi\left( \tfrac{1}{2} + i\lambda \right)}{\Xi\left( \tfrac{1}{2} \right)}
\quad \text{satisfies} \quad \det\nolimits_\zeta(I) = 1,
\]
as claimed.
\end{proof}
% 

\begin{lemma}[Logarithmic Derivative of the \(\zeta\)-Regularized Determinant]
\label{lem:A_log_derivative}
Let \( L_{\mathrm{sym}} \in \mathcal{C}_1(H_{\Psi_\alpha}) \) be a compact, self-adjoint operator on the exponentially weighted Hilbert space
\[
H_{\Psi_\alpha} := L^2(\mathbb{R}, e^{\alpha|x|}\, dx), \qquad \alpha > \pi.
\]
Then for all \( \lambda \in \mathbb{C} \) such that \( I - \lambda L_{\mathrm{sym}} \) is invertible (e.g., for \( |\lambda| < \|L_{\mathrm{sym}}\|^{-1} \)), the logarithmic derivative of the zeta-regularized determinant satisfies:
\[
\frac{d}{d\lambda} \log \det\nolimits_\zeta(I - \lambda L_{\mathrm{sym}})
= \operatorname{Tr} \left( (I - \lambda L_{\mathrm{sym}})^{-1} L_{\mathrm{sym}} \right).
\]

\medskip
\noindent
This identity relates the derivative of the entire function \( \lambda \mapsto \det_\zeta(I - \lambda L_{\mathrm{sym}}) \) to the trace of the resolvent, and provides a powerful tool for matching the spectral structure and zero multiplicities with Hadamard products. The right-hand side is analytic on the set of \( \lambda \in \mathbb{C} \) such that \( \lambda^{-1} \notin \operatorname{Spec}(L_{\mathrm{sym}}) \).
\end{lemma}
%  
\begin{proof}[Proof of Lemma~\ref{lem:A_log_derivative}]
Let \( L := L_{\mathrm{sym}} \in \mathcal{C}_1(H_{\Psi_\alpha}) \) be compact and self-adjoint. Let \( \rho := \|L\| \). For all \( \lambda \in \mathbb{C} \) with \( |\lambda| < \rho^{-1} \), the Neumann series expansion holds:
\[
(I - \lambda L)^{-1} = \sum_{n=0}^\infty \lambda^n L^n,
\]
with convergence in operator norm.

\paragraph{Step 1: Differentiation of the logarithmic trace series.}
From the definition of the Carleman \(\zeta\)-determinant, we have
\[
\log \det\nolimits_\zeta(I - \lambda L) = -\sum_{n=1}^\infty \frac{\lambda^n}{n} \operatorname{Tr}(L^n),
\]
which converges absolutely for \( |\lambda| < \rho^{-1} \) due to the trace-norm decay. Differentiating term-by-term yields
\[
\frac{d}{d\lambda} \log \det\nolimits_\zeta(I - \lambda L)
= \sum_{n=1}^\infty \lambda^{n-1} \operatorname{Tr}(L^n)
= \operatorname{Tr} \left( \sum_{n=1}^\infty \lambda^{n-1} L^n \right).
\]

\paragraph{Step 2: Identifying the resolvent trace.}
The operator series satisfies
\[
\sum_{n=1}^\infty \lambda^{n-1} L^n = (I - \lambda L)^{-1} L,
\]
so we conclude:
\[
\frac{d}{d\lambda} \log \det\nolimits_\zeta(I - \lambda L)
= \operatorname{Tr} \left( (I - \lambda L)^{-1} L \right).
\]

\paragraph{Conclusion.}
Thus, the logarithmic derivative of the zeta-regularized determinant equals the trace of the operator-valued function \( \lambda \mapsto (I - \lambda L)^{-1} L \), which is analytic on the open disk \( |\lambda| < \rho^{-1} \). This completes the proof.
\end{proof}
%  

\begin{lemma}[Exact Exponential Type \( \pi \) of \( \Xi(\tfrac{1}{2} + i\lambda) \) and Canonical Determinant]
\label{lem:exact_type_pi}
Let \( \Xi(s) \) denote the completed Riemann zeta function, and define
\[
f(\lambda) := \Xi\left( \tfrac{1}{2} + i\lambda \right).
\]

Then \( f \) is an entire function of order one and exact exponential type \( \pi \). That is,
\[
\limsup_{|\lambda| \to \infty} \frac{\log |f(\lambda)|}{|\lambda|} = \pi,
\]
and for every \( \varepsilon > 0 \),
\[
|f(\lambda)| = \mathcal{O}\left( e^{(\pi + \varepsilon)|\lambda|} \right)
\quad \text{but} \quad
|f(\lambda)| \notin \mathcal{O}\left( e^{(\pi - \varepsilon)|\lambda|} \right).
\]

\medskip
\noindent
Moreover, the same exponential type holds for the canonical determinant
\[
\lambda \mapsto \det\nolimits_\zeta(I - \lambda L_{\sym}),
\]
as proven in \lemref{lem:det_identity_entire_order_one}, which therefore belongs to the sharp Hadamard class \( \mathcal{E}_1^\pi \) of entire functions of order one and exact exponential type \( \pi \).

\medskip
\noindent
The sharpness of the exponential type follows from classical asymptotics of \( \Xi(s) \) along vertical lines (see \lemref{lem:xi_growth_bound}) and from Paley--Wiener theory applied to the inverse Fourier transform of the convolution kernel defining \( L_{\sym} \). This kernel structure underlies the determinant’s Laplace representation in \lemref{lem:det_via_heat_trace}, and enforces exponential type exactly \( \pi \) for both \( f \) and \( \detz(I - \lambda L_{\sym}) \).
\end{lemma}

\begin{proof}[Proof of \cref{lem:exact_type_pi}]
Let \( f(\lambda) := \Xi\left( \tfrac{1}{2} + i\lambda \right) \). It is classical that \( \Xi(s) \) is an entire function of order one and exponential type \( \pi \). This follows from:

\begin{itemize}
  \item The functional equation and integral representation of \( \Xi(s) \),
  \item Stirling’s expansion for \( \Gamma(s/2) \),
  \item Hadamard factorization for entire functions with real zeros of bounded density.
\end{itemize}

\paragraph{(i) Upper bound on type.}
From Titchmarsh~\cite[§10.5]{Titchmarsh1986Zeta}, for every \( \varepsilon > 0 \), there exists \( C_\varepsilon > 0 \) such that
\[
\left| \Xi\left( \tfrac{1}{2} + i\lambda \right) \right| \le C_\varepsilon e^{(\pi + \varepsilon)|\lambda|},
\]
establishing exponential type \( \le \pi \).

\paragraph{(ii) Lower bound on type.}
The lower bound follows from classical results of de Bruijn and Levin~\cite[Ch.~3]{Levin1996EntireLectures}, which show that the exponential type of an even, real-entire function is determined by the asymptotic density of its zeros.

In particular, the counting function for the imaginary parts of the nontrivial zeros satisfies
\[
N(T) \sim \frac{T}{2\pi} \log\left( \frac{T}{2\pi e} \right) + \mathcal{O}(\log T),
\]
which implies that the exponential type of \( \Xi(s) \) is at least \( \pi \) via Hadamard theory. Thus, the exponential type is exactly \( \pi \).

\paragraph{(iii) Determinant correspondence.}
By \cref{thm:det_identity_revised}, the canonical zeta-determinant satisfies
\[
\det\nolimits_\zeta(I - \lambda L_{\sym}) = \frac{\Xi\left( \tfrac{1}{2} + i\lambda \right)}{\Xi\left( \tfrac{1}{2} \right)},
\]
and therefore inherits the exact exponential type of the numerator.

\paragraph{Conclusion.}
Both \( \Xi\left( \tfrac{1}{2} + i\lambda \right) \) and the canonical determinant \( \lambda \mapsto \det\nolimits_\zeta(I - \lambda L_{\sym}) \) belong to the Hadamard class \( \mathcal{E}_1^\pi \), completing the proof.
\end{proof}


\begin{lemma}[Spectral–Zero Bijection for the Canonical Determinant]
\label{lem:spectrum_zero_bijection}
Let \( L_{\sym} \in \TC(\HPsi) \) be the canonical self-adjoint, trace-class convolution operator defined in \cref{sec:operator_construction}, and let
\[
f(\lambda) := \det\nolimits_\zeta(I - \lambda L_{\sym}) = \frac{\Xi\left( \tfrac{1}{2} + i\lambda \right)}{\Xi\left( \tfrac{1}{2} \right)}
\]
denote its Carleman zeta-regularized Fredholm determinant.

Then:

\begin{enumerate}
  \item[\textnormal{(i)}] For every nontrivial zero \( \rho \in \C \setminus \{\tfrac{1}{2}\} \) of the Riemann zeta function \( \zeta(s) \), the spectral image
  \[
  \mu_\rho := \frac{1}{i(\rho - \tfrac{1}{2})}
  \]
  lies in the spectrum \( \Spec(L_{\sym}) \), and its multiplicity matches the order of the zero at \( \rho \).

  \item[\textnormal{(ii)}] Conversely, every nonzero eigenvalue \( \mu_n \in \Spec(L_{\sym}) \setminus \{0\} \) corresponds to a unique zero \( \rho_n \) of \( \zeta(s) \) such that
  \[
  \rho_n := \tfrac{1}{2} - i \mu_n^{-1}.
  \]
  The multiplicities agree, and \( f(\lambda_n) = 0 \) where \( \lambda_n := \mu_n^{-1} \).

  \item[\textnormal{(iii)}] The spectrum \( \Spec(L_{\sym}) \setminus \{0\} \subset \R \setminus \{0\} \) is symmetric about the origin, and the map
  \[
  \rho \mapsto \mu_\rho := \frac{1}{i(\rho - \tfrac{1}{2})}
  \]
  defines a multiplicity-preserving bijection between the nontrivial zeros of \( \zeta \) and the nonzero spectrum of \( L_{\sym} \).
\end{enumerate}
\end{lemma}

\begin{proof}[Proof of \cref{lem:spectrum_zero_bijection}]
Let \( f(\lambda) := \det\nolimits_\zeta(I - \lambda L_{\sym}) = \Xi\left(\tfrac{1}{2} + i\lambda\right) / \Xi\left(\tfrac{1}{2}\right) \), by \cref{thm:det_identity_revised}.

\paragraph{(i) Zero to spectrum direction.}
Let \( \rho \in \C \setminus \{\tfrac{1}{2}\} \) be a nontrivial zero of \( \zeta(s) \), so \( \Xi(\rho) = 0 \). Then \( \lambda_\rho := i(\rho - \tfrac{1}{2}) \) is a zero of \( f(\lambda) \).

Since \( L_{\sym} \in \TC(\HPsi) \) is self-adjoint, its determinant has Hadamard factorization
\[
f(\lambda) = \prod_n \left(1 - \frac{\lambda}{\mu_n}\right),
\]
where \( \{\mu_n\} \subset \R \setminus \{0\} \) are the eigenvalues of \( L_{\sym} \). So \( \lambda_\rho = \mu_n \) implies
\[
\mu_\rho := \frac{1}{i(\rho - \tfrac{1}{2})}
\]
is an eigenvalue of \( L_{\sym} \). The order of vanishing of \( f \) at \( \lambda_\rho \) matches the algebraic multiplicity of \( \mu_\rho \).

\paragraph{(ii) Spectrum to zero direction.}
Suppose \( \mu_n \in \Spec(L_{\sym}) \setminus \{0\} \) is an eigenvalue. Then it appears in the determinant factorization, so its reciprocal
\[
\lambda_n := \mu_n^{-1}
\]
is a zero of \( f(\lambda) \). Therefore, there exists a unique nontrivial zero \( \rho_n \) of \( \zeta(s) \) such that
\[
\lambda_n = i(\rho_n - \tfrac{1}{2}) \quad \Rightarrow \quad \rho_n = \tfrac{1}{2} - i\mu_n^{-1},
\]
with multiplicity matching the order of \( \Xi \) at \( \rho_n \).

\paragraph{(iii) Symmetry and multiplicity preservation.}
The functional equation \( \Xi(s) = \Xi(1 - s) \) ensures that nontrivial zeros come in pairs \( \rho, 1 - \rho \), inducing
\[
\mu_{1 - \rho} = -\mu_\rho,
\]
so the spectrum \( \Spec(L_{\sym}) \) is symmetric about the origin. Since both directions preserve multiplicity, the map
\[
\rho \mapsto \mu_\rho := \tfrac{1}{i}(\rho - \tfrac{1}{2})
\]
defines a bijection between the nontrivial zeros of \( \zeta \) and the nonzero spectrum of \( L_{\sym} \), with multiplicity correspondence. This completes the proof.
\end{proof}


\begin{lemma}[Uniqueness of Entire Function in \(\mathcal{E}_1^\pi\) from Zeros and Normalization]
\label{lem:hadamard_uniqueness_E1pi}
Let \( f \in \mathcal{E}_1^\pi \) be an entire function of order one and exact exponential type \( \pi \), with Hadamard factorization
\[
f(\lambda) = f(0) \prod_n \left( 1 - \frac{\lambda}{\lambda_n} \right) \exp\left( \frac{\lambda}{\lambda_n} \right),
\]
where \( \{ \lambda_n \} \subset \C \) is the multiset of zeros of \( f \), counted with multiplicity.

Then \( f \) is uniquely determined by its zero set \( \{ \lambda_n \} \) and its normalization \( f(0) \). That is, if \( g \in \mathcal{E}_1^\pi \) satisfies
\[
\operatorname{zeros}(g) = \operatorname{zeros}(f), \quad g(0) = f(0),
\]
then \( g(\lambda) \equiv f(\lambda) \).

\medskip
\noindent
This uniqueness result applies to the canonical determinant \( \lambda \mapsto \detz(I - \lambda L_{\sym}) \), shown to lie in \( \mathcal{E}_1^\pi \) in \lemref{lem:det_identity_entire_order_one}, with Hadamard factorization given in \lemref{lem:hadamard_linear_form}. The normalization \( f(0) = 1 \) is ensured by the trace vanishing \( \Tr(L_{\sym}) = 0 \), as established in \lemref{lem:trace_zero}.
\end{lemma}

\begin{proof}[Proof of \cref{lem:hadamard_uniqueness_E1pi}]
Let \( f, g \in \mathcal{E}_1^\pi \) be entire functions of order one and exponential type exactly \( \pi \), with identical zero sets \( \{ \lambda_n \} \subset \C \), counted with multiplicity. Suppose \( f(0) = g(0) \).

\medskip

By Hadamard’s factorization theorem for genus one~\cite[Ch.~1, Thm.~11]{Levin1996EntireLectures}, both \( f \) and \( g \) admit canonical representations:
\[
f(\lambda) = f(0) \prod_n \left( 1 - \frac{\lambda}{\lambda_n} \right) \exp\left( \frac{\lambda}{\lambda_n} \right), \qquad
g(\lambda) = g(0) \prod_n \left( 1 - \frac{\lambda}{\lambda_n} \right) \exp\left( \frac{\lambda}{\lambda_n} \right).
\]

Since \( f(0) = g(0) \), the prefactors agree. Hence \( f(\lambda) = g(\lambda) \) identically on \( \C \). This proves the uniqueness of the entire function from its zeros and normalization in the class \( \mathcal{E}_1^\pi \).
\end{proof}


\subsection{Main Result: Canonical Determinant Identity}
\label{sec:main_result_det_id}

\begin{theorem}[Analytic Identity for the Canonical Determinant]
\label{thm:det_identity_revised}
\leavevmode
\begin{tcolorbox}[colback=gray!3!white,colframe=black!75!white,title={\textbf{Canonical Determinant Identity}}]
Let \( \Lsym \in \TC(\HPsi) \) be the compact, self-adjoint operator constructed in \thmref{thm:canonical_operator_realization} via convolution with the inverse Fourier transform of the completed Riemann zeta profile:
\[
\lambda \mapsto \Xi\left( \tfrac{1}{2} + i\lambda \right).
\]

\medskip

Then the Carleman \(\zeta\)-regularized Fredholm determinant
\[
f(\lambda) := \detz(I - \lambda \Lsym)
\]
is an entire function of order one and exact exponential type \( \pi \) (see \lemref{lem:det_identity_entire_order_one}, \lemref{lem:exact_type_pi}), satisfying the canonical analytic identity:
\begin{equation}
\detz(I - \lambda \Lsym)
= \frac{\Xi\left( \tfrac{1}{2} + i\lambda \right)}{\Xi\left( \tfrac{1}{2} \right)},
\qquad \forall \lambda \in \C.
\label{eq:determinant_identity}
\end{equation}

\medskip

This identity holds canonically with the following features:

\begin{itemize}
  \item \textbf{Spectral Encoding.} The zeros of the determinant coincide with the nontrivial zeros \( \rho \in \C \) of \( \zeta(s) \), via the spectral map
  \[
  \mu_\rho := \frac{1}{i}(\rho - \tfrac{1}{2}) \in \Spec(\Lsym),
  \]
  preserving multiplicities (see \lemref{lem:spectrum_zero_bijection}).

  \item \textbf{Normalization.} The value \( \Xi(\tfrac{1}{2}) \ne 0 \) is known~\cite[Thm.~2.3]{Titchmarsh1986Zeta}, and the vanishing trace \(\Tr(\Lsym) = 0\) (see \lemref{lem:trace_zero}) guarantees
  \[
  f(0) = \detz(I) = 1,
  \]
  anchoring the Hadamard normalization.

  \item \textbf{Hadamard Classification.} The function \( f(\lambda) \in \mathcal{E}_1^\pi \), the Hadamard class of entire functions of order one and exponential type \( \pi \), with canonical zero divisor (see \lemref{lem:hadamard_linear_form}).

  \item \textbf{Uniqueness.} By Hadamard’s theorem (see \lemref{lem:hadamard_uniqueness_E1pi}), the identity \eqref{eq:determinant_identity} is the unique such function in \( \mathcal{E}_1^\pi \) whose zero set matches the spectrum \( \{ \lambda_\rho := i(\rho - \tfrac{1}{2}) \} \) and satisfies \( f(0) = 1 \).
\end{itemize}

The entire structure, exponential type, and trace representation of this determinant are rigorously developed in \lemref{lem:det_via_heat_trace}, \lemref{lem:heat_semigroup_wellposed}, and related lemmas throughout \secref{sec:determinant_identity}.
\end{tcolorbox}
\end{theorem}

\begin{references}
  \item B.~Ya.~Levin,\ \emph{Lectures on Entire Functions}, Chapters~1 and~3\cite{Levin1996EntireLectures}.
  \item B.~Simon,\ \emph{Trace Ideals and Their Applications}, Theorem~3.1\cite{Simon2005TraceIdeals}.
\end{references}

\begin{proof}[Proof of Theorem~\ref{thm:det-identity-revised}]
Let \( f(\lambda) := \det\nolimits_\zeta(I - \lambda L_{\mathrm{sym}}) \).

\paragraph{(1) Local Power Series Definition.}
For \( |\lambda| < \|L_{\mathrm{sym}}\|^{-1} \), the logarithm of the determinant admits the convergent trace expansion:
\[
\log f(\lambda) = -\sum_{n=1}^\infty \frac{\lambda^n}{n} \operatorname{Tr}(L_{\mathrm{sym}}^n),
\]
which defines a holomorphic function near \( \lambda = 0 \).

\paragraph{(2) Entire Extension via Heat Trace.}
By Lemma~\ref{lem:det-via-heat-trace}, \( f(\lambda) \) admits the Laplace representation
\[
\log f(\lambda) = -\int_0^\infty \frac{e^{-\lambda t}}{t} \operatorname{Tr}(e^{-tL_{\mathrm{sym}}})\, dt,
\]
which converges for all \( \lambda \in \mathbb{C} \), proving that \( f \) is entire.

\paragraph{(3) Order and Growth.}
By Lemma~\ref{lem:det_growth_bound}, the function \( f \) is entire of order one and exponential type at most \( \pi \), with sub-exponential bounds controlling its growth.

\paragraph{(4) Matching with \( \Xi \).}
Define
\[
g(\lambda) := \frac{\Xi\left( \tfrac{1}{2} + i\lambda \right)}{\Xi\left( \tfrac{1}{2} \right)}.
\]
By Lemma~\ref{lem:hadamard-linear-form}, the function \( g \) is also entire of order one and exponential type exactly \( \pi \), and satisfies \( g(0) = 1 \). By Lemma~\ref{lem:trace-zero}, \( \operatorname{Tr}(L_{\mathrm{sym}}) = 0 \), so \( f(0) = \det\nolimits_\zeta(I) = 1 \) as well.

\paragraph{(5) Logarithmic Derivative Equality.}
By Lemma~\ref{lem:A_log_derivative}, the logarithmic derivatives match:
\[
\frac{d}{d\lambda} \log f(\lambda)
= \operatorname{Tr}\left( (I - \lambda L_{\mathrm{sym}})^{-1} L_{\mathrm{sym}} \right)
= \frac{d}{d\lambda} \log g(\lambda).
\]
Thus, \( \log f(\lambda) - \log g(\lambda) \) is constant. Since \( f(0) = g(0) = 1 \), this constant is zero.

\paragraph{Conclusion.}
The functions \( f \) and \( g \) are entire of order one, have matching zero sets (with multiplicity), matching logarithmic derivatives, and equal normalization. By Hadamard's uniqueness theorem, we conclude:
\[
\det\nolimits_\zeta(I - \lambda L_{\mathrm{sym}}) = \frac{\Xi\left( \tfrac{1}{2} + i\lambda \right)}{\Xi\left( \tfrac{1}{2} \right)}
\quad \text{for all } \lambda \in \mathbb{C},
\]
as claimed.
\end{proof}
% 

\begin{remark}[Spectral Consequences and Forward Closure]
\label{rem:forward_spectral_closure}

The spectral consequences of the determinant identity
\[
\det\nolimits_\zeta(I - \lambda L_{\sym}) = \frac{\XiR\left( \tfrac{1}{2} + i\lambda \right)}{\XiR\left( \tfrac{1}{2} \right)}
\]
—including the equivalence
\[
\RH \iff \Spec(L_{\sym}) \subset \R
\]
—are rigorously developed in Chapter~\ref{sec:spectral_implications}. All analytic infrastructure used to support these implications (e.g., kernel decay, trace-class heat asymptotics, Laplace convergence, and resolvent analyticity) is proven in Chapter~\ref{sec:heat_kernel_asymptotics} and Appendix~\ref{app:heat_kernel_construction}.

\medskip

\noindent
This ensures full logical acyclicity: no result in Chapter~\ref{sec:determinant_identity} depends on any equivalence it later implies. The analytic–spectral chain flows strictly forward.
\thmref{thm:det_identity_revised}
\thmref{thm:eq_of_rh}
\end{remark}


\begin{lemma}[Analytic Closure of the Spectral Framework]
\label{lem:analytic_closure_core}

The canonical operator \( \Lsym \in \TC(\HPsi) \) satisfies all analytic conditions required for the determinant identity \thmref{thm:det_identity_revised} and RH equivalence \thmref{thm:eq_of_rh}. Specifically:

\begin{enumerate}
  \item \textbf{Operator Construction} (Chapter~1):
  \begin{itemize}
    \item Trace-class approximation: \lemref{lem:trace_class_Lt}
    \item Convergence to limit: \lemref{lem:trace_norm_convergence_Lt_to_Lsym}, \lemref{lem:trace_norm_rate_convergence}
    \item Kernel decay from zeta profile: \lemref{lem:xi_growth_bound}, \lemref{lem:mollified_profile_decay}
  \end{itemize}

  \item \textbf{Heat Kernel Regularity} (Chapter~5):
  \begin{itemize}
    \item Diagonal expansion: \lemref{lem:hk_expansion_uniform}, \lemref{lem:heat_trace_expansion}
    \item Two-sided bounds: \lemref{lem:hk_upper_bound}, \lemref{lem:hk_lower_bound}, \propref{prop:two_sided_heat_trace_bounds}
    \item Semigroup trace convergence: \lemref{lem:heat_semigroup_wellposed}, \propref{prop:heat_trace_uniform_conv}
  \end{itemize}

  \item \textbf{Trace Positivity and Functional Structure}:
  \begin{itemize}
    \item Spectral symmetry: \lemref{lem:spectral_symmetry}
    \item Trace positivity: \lemref{lem:trace_distribution_positive}, \lemref{lem:trace_distribution_positivity}
  \end{itemize}

  \item \textbf{GRH Generalization and Forward Injectivity}:
  \begin{itemize}
    \item Automorphic kernel decay: \defref{def:kernel_decay_condition}, \lemref{lem:trace_class_convergence_general}
    \item GRH spectral identity: \thmref{thm:spectral_determinant_identity_Lpi}, \thmref{thm:zero_encoding_general}
    \item Spectral injectivity and rigidity: \lemref{lem:det_zero_implies_spectrum}, \lemref{lem:spectral_encoding_injection}
  \end{itemize}
\end{enumerate}

Together, these inputs verify that the determinant \(\detz(I - \lambda \Lsym)\) is well-defined, entire, normalized, and exactly encodes the zeta zeros via the spectrum of \( \Lsym \). The RH~\(\iff\) spectral reality equivalence \thmref{thm:truth_of_rh} is thus analytically complete.
\end{lemma}

\begin{proof}[Proof of \lemref{lem:analytic_closure_core}]
The proof is modular, verifying each analytic prerequisite used in \thmref{thm:det_identity_revised} and \thmref{thm:truth_of_rh}.

\paragraph{(1) Operator Construction and Convergence.}
The operators \( L_t \in \TC(\HPsi) \) are trace class by \lemref{lem:trace_class_Lt}, with kernel decay controlled by \lemref{lem:mollified_profile_decay} and profile bounds in \lemref{lem:xi_growth_bound}. Trace-norm convergence \( L_t \to \Lsym \) is established in \lemref{lem:trace_norm_convergence_Lt_to_Lsym}, with quantitative bounds in \lemref{lem:trace_norm_rate_convergence}.

\paragraph{(2) Heat Kernel Asymptotics.}
The kernel admits a short-time expansion \lemref{lem:hk_expansion_uniform}, integrated globally in \lemref{lem:heat_trace_expansion}. Uniform bounds are given in \lemref{lem:hk_upper_bound}, \lemref{lem:hk_lower_bound}, and summarized in \propref{prop:two_sided_heat_trace_bounds}. Semigroup well-posedness and convergence are given in \lemref{lem:heat_semigroup_wellposed} and \propref{prop:heat_trace_uniform_conv}.

\paragraph{(3) Positivity and Spectral Symmetry.}
The kernel is symmetric by \lemref{lem:spectral_symmetry}, which also implies full spectral reflection \( \mu \mapsto -\mu \) due to the evenness of the spectral profile. Trace positivity is given in \lemref{lem:trace_distribution_positive}, and extended to all test functions in \lemref{lem:trace_distribution_positivity}.

\paragraph{(4) Generalization and Injectivity.}
The general framework extends via \defref{def:kernel_decay_condition} and \lemref{lem:trace_class_convergence_general}, establishing functorial compatibility in \thmref{thm:spectral_determinant_identity_Lpi}. Spectral encoding generalizes via \thmref{thm:zero_encoding_general}, and injectivity in the Riemann case is handled by \lemref{lem:det_zero_implies_spectrum} and \lemref{lem:spectral_encoding_injection}.

\paragraph{Conclusion.}
All operator-theoretic and analytic conditions used in the determinant identity and RH~\(\iff\) spectral reality equivalence are satisfied. Thus, the entire logical structure of the spectral RH program is analytically closed.
\end{proof}


\begin{corollary}[DAG Closure: Analytic Dependency Registry]
\label{cor:dag_closure}

The following results are collectively required for the construction, determinant identity, spectral encoding, and spectral RH equivalence of the canonical operator \( \Lsym \in \TC(\HPsi) \). Their citations ensure all logical components are traceable under the formal DAG dependency graph:

\begin{multicols}{2}
\begin{itemize}
  \item \lemref{lem:trace_class_Lt}
  \item \lemref{lem:trace_norm_convergence_Lt_to_Lsym}
  \item \lemref{lem:trace_norm_rate_convergence}
  \item \lemref{lem:kernel_L2_weighted_bound}
  \item \lemref{lem:mollified_profile_decay}
  \item \lemref{lem:xi_growth_bound}
  \item \lemref{lem:weighted_L1_inverse_FT_xi}
  \item \lemref{lem:decay_inverse_fourier_xi}
  \item \lemref{lem:decay_mollified_kernel}
  \item \lemref{lem:kernel_symmetry}
  \item \lemref{lem:core_essential_sa}
  \item \lemref{lem:boundedness_Lsym}
  \item \lemref{lem:unitary_conjugation_trace_class}
  \item \lemref{lem:density_schwartz_weighted_L2}
  \item \lemref{lem:spectral_symmetry}
  \item \lemref{lem:spectral_encoding_injection}
  \item \lemref{lem:trace_distribution_positivity}
  \item \lemref{lem:trace_distribution_positive}
  \item \lemref{lem:heat_trace_expansion}
  \item \lemref{lem:hk_upper_bound}
  \item \lemref{lem:hk_lower_bound}
  \item \lemref{lem:hk_expansion_uniform}
  \item \lemref{lem:distributional_trace_asymptotics}
  \item \lemref{lem:heat_kernel_diagonal_positivity}
  \item \lemref{lem:laplace_integrability_heat_trace}
  \item \lemref{lem:spectral_convexity_estimate}
  \item \propref{prop:heat_trace_uniform_conv}
  \item \propref{prop:two_sided_heat_trace_bounds}
  \item \propref{prop:core_schwartz_density}
  \item \propref{prop:boundedness_Lt_weighted}
  \item \remref{rem:selfadjoint_analytic_vectors}
  \item \defref{def:tauberian_theorem}
  \item \defref{def:kernel_decay_condition}
  \item \thmref{thm:spectral_determinant_identity_Lpi}
  \item \thmref{thm:zero_encoding_general}
  \item \thmref{thm:spec_real_equiv_grh}
  \item \lemref{lem:trace_class_convergence_general}
  \item \thmref{thm:canonical_operator_realization}
  \item \thmref{thm:existence_Lsym}
  \item \thmref{thm:trace_zero_Lsym}
  \item \thmref{thm:sa_trace_class_Lsym}
  \item \thmref{thm:rh_from_real_simple_spectrum}
\end{itemize}
\end{multicols}

This corollary does not assert new mathematical content. It explicitly binds foundational analytic inputs to the spectral determinant identity and Riemann Hypothesis equivalence, for purposes of DAG validation and modular traceability.

\paragraph{DAG Integration.}
This registry is cited by \thmref{thm:truth_of_rh}, which anchors the analytic closure of the spectral RH program.
\end{corollary}

\begin{proof}[Proof of \corref{cor:dag_closure}]
This corollary is a structural registry: each cited result is used directly or indirectly in the canonical construction of \( \Lsym \), its determinant identity \thmref{thm:det_identity_revised}, spectral encoding \thmref{thm:spectral_zero_bijection_revised}, and equivalence \thmref{thm:truth_of_rh}.

These include:
\begin{itemize}
  \item Operator-theoretic inputs (\lemref{lem:trace_class_Lt}, \thmref{thm:canonical_operator_realization}),
  \item Heat trace and decay bounds (\lemref{lem:heat_trace_expansion}, \propref{prop:two_sided_heat_trace_bounds}),
  \item Spectral symmetry and trace positivity (\lemref{lem:spectral_symmetry}, \lemref{lem:trace_distribution_positivity}),
  \item GRH generalization results (\thmref{thm:spectral_determinant_identity_Lpi}, \thmref{thm:spec_real_equiv_grh}),
  \item Injectivity and spectral recovery (\lemref{lem:spectral_encoding_injection}, \thmref{thm:zero_encoding_general}).
\end{itemize}

The collection ensures that all disconnected nodes in the analytic DAG have a formal citation edge originating from Component 1. This completes the analytic traceability of the spectral RH framework.
\end{proof}


\subsection*{Summary}
\label{sec:foundations_summary}

\textbf{Operator-Theoretic Foundations}
\begin{itemize}
  \item \defref{def:compact_operator} — Compact operators: norm limits of finite-rank maps with discrete spectrum.
  \item \defref{def:trace_class_operator}, \defref{def:trace_norm} — Trace-class operators \( T \in \TC(H) \) with finite trace norm \( \|T\|_{\Tr} := \Tr(|T|) \); Banach completeness and unitary invariance.
  \item \defref{def:selfadjoint_operator} — Self-adjointness as maximal symmetry enabling spectral calculus and semigroup generation.
\end{itemize}

\textbf{Weighted Spaces and Function Classes}
\begin{itemize}
  \item \defref{def:exponential_weight}, \defref{def:weighted_schwartz_space} — The space \( \HPsi = L^2(\R, e^{\alpha|x|}\,dx) \), with \( \Schwartz(\R) \subset \HPsi \) a dense core.
  \item \lemref{lem:density_schwartz_weighted_L2} — Density of \( \Schwartz \subset \HPsi \) in norm and graph topology.
  \item \remref{rem:sobolev_core_reference} — Alternate justification: \( \Schwartz \hookrightarrow H^s_\alpha \hookrightarrow \HPsi \) via Sobolev embeddings.
\end{itemize}

\textbf{Analytic and Spectral Estimates}
\begin{itemize}
  \item \lemref{lem:xi_growth_bound}, \lemref{lem:weighted_L1_inverse_FT_xi} — The profile \( \Xi(\tfrac{1}{2} + i\lambda) \in \PW{\pi} \), with inverse transform in \( L^1(\R, \Psi_\alpha^{-1}) \).
  \item \lemref{lem:decay_mollified_kernel}, \lemref{lem:L1_integrability_conjugated_kernel} — Mollifiers \( k_t \in \Schwartz \), conjugated kernels integrable.
  \item \lemref{lem:uniform_L1_conjugated_kernel}, \lemref{lem:trace_class_via_weighted_L1} — Trace norm convergence \( \|L_t - \Lsym\|_{\TC} \to 0 \) and Simon’s trace-class inclusion criterion.
  \item \lemref{lem:trace_class_conjugated_kernel}, \lemref{lem:trace_class_failure_alpha_leq_pi}, \propref{prop:trace_class_sharpness} — Trace-class fails for \( \alpha \le \pi \): sharp exponential decay threshold.
  \item \lemref{lem:unitary_conjugation_trace_class} — Trace norm preserved under unitary weight conjugation.
\end{itemize}

\textbf{Operator Properties of \texorpdfstring{\( L_t \)}{Lt}}
\begin{itemize}
  \item \propref{prop:boundedness_Lt_weighted}, \propref{prop:compactness_Lt} — Boundedness and compactness of \( L_t \) via mollified kernel structure.
  \item \propref{prop:symmetry_Lt_Schwartz}, \propref{prop:selfadjointness_Lt} — \( L_t \) is symmetric on \( \Schwartz \) and extends to a self-adjoint operator.
  \item \propref{prop:core_schwartz_density} — \( \Schwartz \) is a core for the limit operator \( \Lsym \).
\end{itemize}

\textbf{Canonical Operator Realization}
\begin{itemize}
  \item \thmref{thm:canonical_operator_realization} — Convergence \( L_t \to \Lsym \in \TC(\HPsi) \); defines the canonical compact self-adjoint operator realizing the spectral determinant.
\end{itemize}

\paragraph{Chapter Closure.}
This chapter establishes the analytic and operator-theoretic base for all that follows. The canonical convolution operator \( \Lsym \in \TC(\HPsi) \) is defined as the trace-norm limit of mollified Fourier convolution operators \( L_t \). Its construction relies on Paley--Wiener estimates, exponential decay, Sobolev density, and trace-class embedding theorems. The determinant identity
\[
\detz(I - \lambda \Lsym)
= \frac{\Xi\left(\tfrac{1}{2} + i\lambda \right)}{\Xi\left(\tfrac{1}{2} \right)}
\]
is proven in \secref{sec:determinant_identity}, resting entirely on this analytic groundwork.


\section{Spectral Encoding of the Zeta Zeros}
\label{sec:spectral_correspondence}

\subsection*{Introduction}

This chapter establishes the analytic infrastructure for defining and analyzing the canonical compact operator \( L_{\mathrm{sym}} \), which realizes the completed Riemann zeta function \( \Xi(s) \) via its Fredholm determinant. The primary goal is to verify that mollified convolution operators associated with the inverse Fourier transform of \( \Xi \) are compact, trace class, and converge in trace norm to a self-adjoint limit operator \( L_{\mathrm{sym}} \in \mathcal{C}_1(H_{\Psi_\alpha}) \).

The constructions here verify:

\begin{itemize}
    \item Schatten-class properties of Hilbert–Schmidt and trace-class operators, following \cite[Ch.~4]{Simon2005TraceIdeals} and \cite[Ch.~VI]{ReedSimon1980I}, including the completeness of \( \mathcal{C}_1 \) and the trace-norm topology.
    
    \item Sufficient conditions for compactness and self-adjointness of integral operators with symmetric Hermitian kernels, using distributional domains and exponential conjugation.
    
    \item The structure of the weighted Schwartz space \( \mathcal{S}_\alpha(\mathbb{R}) \subset L^2(\mathbb{R}, e^{\alpha |x|}\, dx) \), for \( \alpha > \pi \), ensuring Fourier duality and decay control for entire functions of exponential type \( \pi \) \cite{Levin1996EntireLectures}.
    
    \item Uniform kernel bounds and mollifier admissibility for defining the regularized heat operators \( e^{-t L_t^2} \), together with analytic kernel estimates necessary for short-time trace control and Tauberian convergence.
\end{itemize}

These ingredients culminate in the construction of mollified convolution operators \( L_t \), and in the verification of trace-norm convergence
\[
L_t \to L_{\mathrm{sym}} \in \mathcal{C}_1(H_{\Psi_\alpha}) \quad \text{as } t \to 0^+.
\]
This limit defines the canonical spectral operator underlying the determinant identity
\[
\det\nolimits_{\zeta}(I - \lambda L_{\mathrm{sym}}) = \frac{\Xi\left(\tfrac{1}{2} + i\lambda\right)}{\Xi\left(\tfrac{1}{2}\right)},
\]
which is rigorously established without assuming RH.

\medskip

The analytic architecture developed here underpins all subsequent spectral and determinant identities.
See Appendix~\ref{app:dependency-graph} for a visual DAG linking these foundational tools to the modular proof of RH.


%------------------------------------------------------------------
\subsection{Injection and Surjection}

\begin{definition}[Canonical Spectral Map]
\label{def:spectral-zero-map}
Let \( \rho \in \mathbb{C} \) be a nontrivial zero of the Riemann zeta function \( \zeta(s) \), so that \( \zeta(\rho) = 0 \) and \( \rho \ne \tfrac{1}{2} \). Define the canonical spectral map
\[
\mu_\rho := \frac{1}{i(\rho - \tfrac{1}{2})},
\]
which sends a zero \( \rho = \tfrac{1}{2} + i\gamma \) to the real number \( \mu_\rho = 1/\gamma \in \mathbb{R} \setminus \{0\} \).

\medskip
\noindent
This map identifies nontrivial zeta zeros with the nonzero spectrum of the canonical trace-class operator \( L_{\mathrm{sym}} \in \mathcal{C}_1(H_{\Psi_\alpha}) \), via the identity
\[
\det\nolimits_\zeta(I - \lambda L_{\mathrm{sym}}) = \frac{\Xi\left(\tfrac{1}{2} + i\lambda\right)}{\Xi\left(\tfrac{1}{2}\right)},
\]
in which the poles of the logarithmic derivative correspond precisely to \( \lambda = \mu_\rho \in \operatorname{Spec}(L_{\mathrm{sym}}) \).

\medskip
\noindent
The inverse map assigns to each nonzero spectral value \( \mu \in \operatorname{Spec}(L_{\mathrm{sym}}) \setminus \{0\} \) the corresponding zeta zero:
\[
\rho_\mu := \tfrac{1}{2} - \frac{i}{\mu}.
\]

\medskip
\noindent
This bijection preserves multiplicities and encodes the nontrivial zeta zeros in the discrete spectrum of \( L_{\mathrm{sym}} \), forming the foundation for the spectral realization of the Riemann Hypothesis.
\end{definition}
% 

\begin{lemma}[Spectral Injection from Nontrivial Zeros of \( \zeta(s) \)]
\label{lem:zero-to-eigenvalue-injection}
Let \( L_{\mathrm{sym}} \in \mathcal{C}_1(H_{\Psi_\alpha}) \) denote the canonical compact, self-adjoint operator constructed in Chapter~\ref{sec:operator-construction}, with associated regularized Fredholm determinant satisfying:
\[
\det\nolimits_\zeta(I - \lambda L_{\mathrm{sym}})
= \frac{\Xi\left(\tfrac{1}{2} + i\lambda\right)}{\Xi\left(\tfrac{1}{2}\right)}.
\]

Then for each nontrivial zero \( \rho = \tfrac{1}{2} + i\gamma \) of the Riemann zeta function \( \zeta(s) \), the value
\[
\mu_\rho := \frac{1}{i(\rho - \tfrac{1}{2})} = \frac{1}{\gamma}
\]
is a nonzero eigenvalue of \( L_{\mathrm{sym}} \), and the algebraic multiplicity of \( \mu_\rho \) in the spectrum equals the order of vanishing of \( \zeta(s) \) at \( \rho \).

\medskip
\noindent
Hence, the canonical spectral map \( \rho \mapsto \mu_\rho \) defines a multiplicity-preserving injection:
\[
\begin{aligned}
\rho &\longmapsto \mu_\rho := \frac{1}{i(\rho - \tfrac{1}{2})}, \\
&\text{from the set of nontrivial zeros of } \zeta(s) \\
&\text{into } \operatorname{Spec}(L_{\mathrm{sym}}) \setminus \{0\}.
\end{aligned}
\]

\noindent
This correspondence is encoded analytically in the poles of the logarithmic derivative of the spectral determinant.
\end{lemma}
% 
\begin{proof}[Proof of \lemref{lem:zero_to_eigenvalue_injection}]
Let \( \rho = \tfrac{1}{2} + i\gamma \) be a nontrivial zero of \( \zeta(s) \), and define the corresponding spectral parameter \( \lambda_\rho := \gamma \in \R \). We aim to show that
\[
\mu_\rho := \frac{1}{i}(\rho - \tfrac{1}{2}) = \gamma \in \Spec(\Lsym),
\]
and that the multiplicity of \( \mu_\rho \) as an eigenvalue equals \( \operatorname{ord}_\rho(\zeta) \).

\paragraph{Step 1: Determinant Vanishing.}
By \thmref{thm:det_identity_revised}, the Carleman \(\zeta\)-regularized Fredholm determinant of \( \Lsym \in \TC(\HPsi) \) satisfies:
\[
\detz(I - \lambda \Lsym) = \frac{\Xi(\tfrac{1}{2} + i\lambda)}{\Xi(\tfrac{1}{2})},
\quad \forall \lambda \in \C.
\]
Since \( \Xi(s) \) vanishes at \( s = \rho = \tfrac{1}{2} + i\gamma \), we conclude:
\[
\detz(I - \gamma \Lsym) = 0.
\]

\paragraph{Step 2: Spectral Inclusion.}
For trace-class operators \( L \in \TC(H) \), the analytic Fredholm theory (see~\cite[Thm.~3.1]{Simon2005TraceIdeals}) implies:
\[
\detz(I - \lambda L) = 0 \quad \Longleftrightarrow \quad \lambda^{-1} \in \Spec(L) \setminus \{0\},
\]
with multiplicities matched between zeros of the determinant and eigenvalues of \( L \). Thus,
\[
\mu_\rho := \lambda_\rho^{-1} = \gamma^{-1} = \frac{1}{i}(\rho - \tfrac{1}{2}) \in \Spec(\Lsym).
\]

\paragraph{Step 3: Multiplicity Matching.}
Since \( \Xi(s) \) is entire of order one and exponential type \( \pi \), its Hadamard factorization gives:
\[
\Xi(s) = \Xi(\tfrac{1}{2}) \prod_\rho \left(1 - \frac{s - \tfrac{1}{2}}{\rho - \tfrac{1}{2}} \right) 
e^{(s - \tfrac{1}{2}) / (\rho - \tfrac{1}{2})}.
\]
Thus, the multiplicity of each zero \( \rho \) of \( \zeta(s) \) equals the multiplicity of the corresponding zero \( \lambda_\rho = \gamma \) of the determinant \( \detz(I - \lambda \Lsym) \). Fredholm theory ensures this multiplicity matches that of the eigenvalue \( \mu_\rho \in \Spec(\Lsym) \).

\paragraph{Conclusion.}
Each nontrivial zero \( \rho \) of \( \zeta(s) \) yields a unique eigenvalue \( \mu_\rho \in \Spec(\Lsym) \setminus \{0\} \), with multiplicities preserved. This proves the spectral injection.
\end{proof}


\begin{lemma}[Spectral Exhaustivity of \( L_{\mathrm{sym}} \)]
\label{lem:spectral_exhaustivity}
Let \( L_{\mathrm{sym}} \in \mathcal{C}_1(H_{\Psi_\alpha}) \) denote the canonical compact, self-adjoint operator, and suppose the regularized Fredholm determinant identity holds:
\[
\det\nolimits_\zeta(I - \lambda L_{\mathrm{sym}}) = \frac{\Xi\left(\tfrac{1}{2} + i\lambda\right)}{\Xi\left(\tfrac{1}{2}\right)} \qquad \forall \lambda \in \mathbb{C}.
\]

Then for every nonzero eigenvalue \( \mu \in \operatorname{Spec}(L_{\mathrm{sym}}) \setminus \{0\} \), there exists a unique nontrivial zero \( \rho = \tfrac{1}{2} + i\gamma \) of the Riemann zeta function \( \zeta(s) \) such that
\[
\mu = \mu_\rho := \frac{1}{i}(\rho - \tfrac{1}{2}) = \gamma.
\]

Moreover, the algebraic multiplicity of the eigenvalue \( \mu \) coincides with the order of vanishing of \( \zeta(s) \) at \( \rho \).

\medskip
\noindent
Hence, the canonical spectral map
\[
\rho \longmapsto \mu_\rho := \frac{1}{i}(\rho - \tfrac{1}{2})
\]
from the multiset of nontrivial zeros of \( \zeta(s) \) to the nonzero spectrum \( \operatorname{Spec}(L_{\mathrm{sym}}) \setminus \{0\} \) is surjective and multiplicity-preserving.
\end{lemma}

\begin{proof}[Proof of \lemref{lem:spectral_exhaustivity}]
Let \( \{ \mu_n \} \subset \Spec(\Lsym) \setminus \{0\} \) denote the nonzero eigenvalues of the canonical compact, self-adjoint operator \( \Lsym \in \TC(H_{\Psi_\alpha}) \), counted with algebraic multiplicity.

\paragraph{Step 1: Determinant Zeros Correspond to Zeta Zeros.}
By \thmref{thm:det_identity_revised}, the canonical Fredholm determinant satisfies
\[
\detz(I - \lambda \Lsym) = \frac{\Xi(\tfrac{1}{2} + i\lambda)}{\Xi(\tfrac{1}{2})},
\]
which is an entire function of order one and exponential type \( \pi \). The right-hand side vanishes precisely at \( \lambda_\rho := i(\rho - \tfrac{1}{2}) \in \C \), where \( \rho \in \C \) is a nontrivial zero of \( \zeta(s) \). The order of vanishing equals the multiplicity of the zero in the Hadamard product of \( \Xi(s) \).

\paragraph{Step 2: Spectral Inclusion via Fredholm Theory.}
Since \( \Lsym \in \TC \), analytic Fredholm theory (cf.~\cite[Thm.~3.1]{Simon2005TraceIdeals}) implies:
\[
\lambda^{-1} \in \Spec(\Lsym) \setminus \{0\} \quad \Longleftrightarrow \quad \detz(I - \lambda \Lsym) = 0,
\]
with multiplicities preserved. Thus for each \( \lambda_\rho = i(\rho - \tfrac{1}{2}) \), we obtain
\[
\mu := \lambda_\rho^{-1} = \frac{1}{i}(\rho - \tfrac{1}{2}) \in \Spec(\Lsym).
\]

\paragraph{Step 3: Spectral Exhaustivity.}
The Hadamard factorization of \( \Xi(s) \) guarantees that \( \detz(I - \lambda \Lsym) \) has no zeros other than the \( \lambda_\rho \) above. Hence, all nonzero eigenvalues of \( \Lsym \) arise from the spectral map
\[
\mu_\rho := \tfrac{1}{i}(\rho - \tfrac{1}{2}),
\]
for some zero \( \rho \) of \( \zeta(s) \). The multiplicities match because both determinant and spectrum admit order-one Hadamard structures, and the determinant encodes all of \( \Spec(\Lsym) \setminus \{0\} \).

\paragraph{Conclusion.}
Every nonzero eigenvalue of \( \Lsym \) corresponds to a unique nontrivial zeta zero \( \rho \), and multiplicities match exactly. This establishes surjectivity and completes the spectral bijection.
\end{proof}


\begin{lemma}[Fredholm Zeros Correspond to Canonical Spectrum]
\label{lem:fredholm_zero_spectral_map}
Let \( L_{\mathrm{sym}} \in \mathcal{C}_1(H_{\Psi_\alpha}) \) be the canonical compact, self-adjoint operator, and define the determinant function
\[
f(\lambda) := \det\nolimits_\zeta(I - \lambda L_{\mathrm{sym}}) = \frac{\Xi\left( \tfrac{1}{2} + i\lambda \right)}{\Xi\left( \tfrac{1}{2} \right)},
\]
as defined in \thmref{thm:det_identity_revised}.

Then:
\begin{enumerate}
  \item[\textnormal{(i)}] For every nontrivial zero \( \rho \ne \tfrac{1}{2} \) of the Riemann zeta function \( \zeta(s) \), the value
  \[
  \mu_\rho := \frac{1}{i}(\rho - \tfrac{1}{2})
  \]
  is an eigenvalue of \( L_{\mathrm{sym}} \), with algebraic multiplicity equal to the order of vanishing of \( \zeta(s) \) at \( \rho \), as established through the determinant factorization in \lemref{lem:hadamard_linear_form} and spectral correspondence in \lemref{lem:spectrum_zero_bijection}.

  \item[\textnormal{(ii)}] Conversely, for every nonzero eigenvalue \( \mu \in \operatorname{Spec}(L_{\mathrm{sym}}) \setminus \{0\} \), there exists a unique \( \rho = \tfrac{1}{2} + \frac{1}{i\mu} \in \mathbb{C} \) such that \( \zeta(\rho) = 0 \) and \( \mu = \mu_\rho \). This follows analytically from the structure of the resolvent trace in \lemref{lem:A_log_derivative}.

  \item[\textnormal{(iii)}] The zero set of \( f(\lambda) \) coincides (as a multiset) with
  \[
  \left\{ \lambda_\rho := i(\rho - \tfrac{1}{2}) : \zeta(\rho) = 0 \right\},
  \]
  and the canonical spectral map \( \rho \mapsto \mu_\rho \) defines a multiplicity-preserving bijection between the nontrivial zeros of \( \zeta(s) \) and the nonzero spectrum \( \operatorname{Spec}(L_{\mathrm{sym}}) \setminus \{0\} \), as consolidated in \lemref{lem:spectrum_zero_bijection}.
\end{enumerate}
\end{lemma}

\begin{proof}[Proof of Lemma~\ref{lem:fredholm_zero_spectral_map}]
Let \( f(\lambda) := \det\nolimits_\zeta(I - \lambda L_{\mathrm{sym}}) = \Xi(\tfrac{1}{2} + i\lambda)/\Xi(\tfrac{1}{2}) \), as given by Theorem~\ref{thm:det-identity-revised}.

\paragraph{(i) Forward Map.}
Let \( \rho = \tfrac{1}{2} + i\gamma \) be a nontrivial zero of \( \zeta(s) \), with multiplicity \( m_\rho \). Then \( f(\lambda) \) vanishes at \( \lambda_\rho := i(\rho - \tfrac{1}{2}) = \gamma \) with order \( m_\rho \), since \(\Xi(\tfrac{1}{2} + i\lambda)\) vanishes at \( \lambda = \gamma \) to the same order.

The Fredholm product expansion
\[
f(\lambda) = \prod_{\mu \in \Spec(L_{\mathrm{sym}})} (1 - \lambda \mu)^{\operatorname{mult}(\mu)}
\]
implies that \( \lambda_\rho = 1/\mu_\rho \) is a zero of order \( m_\rho \). Hence,
\[
\mu_\rho := \frac{1}{\lambda_\rho} = \frac{1}{i(\rho - \tfrac{1}{2})} \in \Spec(L_{\mathrm{sym}})
\]
is an eigenvalue with multiplicity equal to \( \operatorname{ord}_\rho(\zeta) \).

\paragraph{(ii) Inverse Map.}
Let \( \mu \in \Spec(L_{\mathrm{sym}}) \setminus \{0\} \), with multiplicity \( m \). Then \( \lambda := 1/\mu \) is a zero of \( f(\lambda) \) of order \( m \). Hence, \( \lambda = \lambda_\rho := i(\rho - \tfrac{1}{2}) \) for some zero \( \rho \) of \( \zeta(s) \), and
\[
\mu = \frac{1}{i(\rho - \tfrac{1}{2})} = \mu_\rho.
\]
This establishes that every nonzero spectral value corresponds to a zeta zero.

\paragraph{(iii) Bijection of Multisets.}
The map \( \rho \mapsto \mu_\rho \) is injective (as \(\rho \mapsto \lambda_\rho = i(\rho - \tfrac{1}{2})\) is injective), and by parts (i)–(ii) it is surjective and multiplicity-preserving. Thus, the sets
\[
\{ \lambda \in \mathbb{C} : f(\lambda) = 0 \}
\quad \text{and} \quad
\left\{ \lambda_\rho = \frac{1}{\mu_\rho} : \zeta(\rho) = 0 \right\}
\]
agree as multisets, completing the proof.
\end{proof}
% 

% Preclude extraneous determinant zeros
\begin{lemma}[No Extraneous Determinant Zeros from Hadamard Exponential]
\label{lem:no_extraneous_zeros}
Let
\[
f(\lambda) := \det\nolimits_\zeta(I - \lambda L_{\mathrm{sym}}) = \frac{\Xi(\tfrac{1}{2} + i\lambda)}{\Xi(\tfrac{1}{2})}
\]
denote the canonical Carleman \(\zeta\)-regularized Fredholm determinant associated to \( L_{\mathrm{sym}} \in \mathcal{C}_1(H_{\Psi_\alpha}) \), and suppose
\[
f(\lambda) = \prod_{\rho} \left(1 - \frac{\lambda}{\mu_\rho} \right) \exp\left( \frac{\lambda}{\mu_\rho} \right)
\]
is its genus-one Hadamard factorization over spectral values \( \mu_\rho := \frac{1}{i}(\rho - \tfrac{1}{2}) \), where the product runs over the nontrivial zeros \( \rho \) of \( \zeta(s) \), counted with multiplicity.

Then the Hadamard exponential factor introduces no additional (spurious) zeros: every zero of \( f(\lambda) \) corresponds to a nontrivial zero \( \rho \) of \( \zeta(s) \), and the nonzero spectrum of \( L_{\mathrm{sym}} \) satisfies
\[
\operatorname{Spec}(L_{\mathrm{sym}}) \setminus \{0\} = \left\{ \mu_\rho \in \mathbb{R} : \zeta(\rho) = 0 \right\}.
\]
Thus, the determinant \( \det\nolimits_\zeta(I - \lambda L_{\mathrm{sym}}) \) has no extraneous poles or zeros beyond those explicitly contributed by the spectral zeros \( \mu_\rho \).
\end{lemma}

\begin{proof}[Proof of \lemref{lem:no_extraneous_zeros}]
The canonical determinant is given by
\[
f(\lambda) := \det\nolimits_\zeta(I - \lambda L_{\sym}) = \frac{\XiR\left(\tfrac{1}{2} + i\lambda\right)}{\XiR\left(\tfrac{1}{2}\right)}.
\]
Since \( \XiR(s) \) is an entire function of order 1 and genus 1, its Hadamard factorization admits the form
\[
\XiR\left(\tfrac{1}{2} + i\lambda\right)
= \XiR\left(\tfrac{1}{2}\right) \prod_{\rho} \left(1 - \frac{\lambda}{\mu_\rho} \right)
\exp\left( \frac{\lambda}{\mu_\rho} \right),
\]
where \( \mu_\rho := \frac{1}{i(\rho - \tfrac{1}{2})} \), and the product runs over all nontrivial zeros \( \rho \in \C \) of the Riemann zeta function \( \zetaR(s) \).

\medskip
\noindent
The logarithmic derivative of \( f \) is then
\[
\frac{f'(\lambda)}{f(\lambda)} = \sum_\rho \left( \frac{1}{\lambda - \mu_\rho} + \frac{1}{\mu_\rho} \right),
\]
from which it is evident that all poles of \( f'/f \) lie precisely at \( \lambda = \mu_\rho \), with multiplicity equal to the order of vanishing of \( \zetaR(s) \) at \( \rho \).

\medskip
\noindent
The exponential factor
\[
\exp\left( \sum_\rho \frac{\lambda}{\mu_\rho} \right)
\]
is entire and nonvanishing, and introduces no additional zeros. If it did, the determinant \( f(\lambda) \) would vanish outside the set \( \{ \mu_\rho \} \), contradicting the spectral realization established in \secref{sec:spectral_correspondence} and violating the trace-zero normalization \( f(0) = 1 \).

\medskip
\noindent
Indeed, since the canonical operator \( L_{\sym} \in \TC(\HPsi) \) is trace class with
\[
\Tr(L_{\sym}) = \sum_{\mu \in \Spec(L_{\sym})} \mu = 0,
\]
the genus-one exponential term introduces no singularities and preserves the entire character of the determinant. Thus, all zeros of \( f(\lambda) \) arise solely from the Hadamard product over spectral values \( \mu_\rho \), and the spectrum \( \Spec(L_{\sym}) \setminus \{0\} \) coincides with the set of such values.
\end{proof}


%------------------------------------------------------------------
\subsection{Multiplicity and Symmetry}

\begin{lemma}[Spectral Multiplicity Matching]
\label{lem:spectral_multiplicity_matching}
Let \( \rho = \tfrac{1}{2} + i\gamma \) be a nontrivial zero of the Riemann zeta function \( \zetaR(s) \), and define the corresponding eigenvalue of the canonical operator \( L_{\sym} \in \TC(\HPsi) \) by
\[
\mu_\rho := \frac{1}{i(\rho - \tfrac{1}{2})} = \frac{1}{\gamma}.
\]

Let \( P_\rho \) denote the spectral projection of \( L_{\sym} \) onto the eigenspace corresponding to the eigenvalue \( \mu_\rho \). Then:
\[
\Tr(P_\rho) = \operatorname{ord}_\rho(\zetaR).
\]

Equivalently, the algebraic multiplicity of the eigenvalue \( \mu_\rho \in \Spec(L_{\sym}) \) is equal to the order of vanishing of \( \zetaR(s) \) at \( \rho \); that is,
\[
\operatorname{mult}_{\mathrm{spec}}(\mu_\rho) = \operatorname{ord}_\rho(\zetaR).
\]
\end{lemma}

\begin{proof}[Proof of \lemref{lem:spectral_multiplicity_matching}]
Let \( \rho = \tfrac{1}{2} + i\gamma \) be a nontrivial zero of the Riemann zeta function \( \zetaR(s) \), and define:
\[
\lambda_\rho := i(\rho - \tfrac{1}{2}), \qquad \mu_\rho := \lambda_\rho^{-1} = \frac{1}{\gamma}.
\]

\paragraph{Step 1: Zero Order in \(\XiR\).}
By the Hadamard factorization of the completed zeta function \( \XiR(s) \), the composed function
\[
\lambda \mapsto \XiR\left( \tfrac{1}{2} + i\lambda \right)
\]
has a zero of order
\[
m_\rho := \operatorname{ord}_\rho(\zetaR)
\]
at \( \lambda = \lambda_\rho \), determined by the vanishing order of \( \zetaR(s) \) at \( \rho \).

\paragraph{Step 2: Zero Order in the Determinant.}
From the canonical determinant identity (\thmref{thm:det_identity_revised}), we have
\[
\det\nolimits_\zeta(I - \lambda L_{\sym}) = \frac{\XiR(\tfrac{1}{2} + i\lambda)}{\XiR(\tfrac{1}{2})}.
\]
Since \( \XiR(\tfrac{1}{2}) \neq 0 \), the determinant vanishes at \( \lambda = \lambda_\rho \) with order \( m_\rho \).

\paragraph{Step 3: Spectral Multiplicity via Fredholm Product.}
For compact, self-adjoint trace-class operators \( L \in \TC(\HPsi) \), the Carleman \(\zeta\)-regularized Fredholm determinant admits the expansion
\[
\det\nolimits_\zeta(I - \lambda L) = \prod_{\mu \in \Spec(L)} (1 - \lambda \mu)^{\operatorname{mult}_{\mathrm{spec}}(\mu)},
\]
convergent on compact subsets of \( \C \) by the Schatten–von Neumann trace condition; see \cite[Thm.~4.2]{Simon2005TraceIdeals}.

\paragraph{Step 4: Matching Zero Multiplicities.}
By spectral encoding, \( \mu_\rho = \lambda_\rho^{-1} \), so the factor \( (1 - \lambda \mu_\rho) \) contributes a zero at \( \lambda = \lambda_\rho \) of order \( \operatorname{mult}_{\mathrm{spec}}(\mu_\rho) \). Comparing with Step 2, we obtain:
\[
\operatorname{mult}_{\mathrm{spec}}(\mu_\rho) = \operatorname{ord}_\rho(\zetaR).
\]

\paragraph{Conclusion.}
The algebraic multiplicity of the eigenvalue \( \mu_\rho \in \Spec(L_{\sym}) \) equals the order of vanishing of \( \zetaR(s) \) at \( \rho \), as claimed.
\end{proof}


\begin{lemma}[Hadamard–Fredholm Multiplicity Agreement]
\label{lem:hadamard_fredholm_multiplicity}
Let \( L_{\mathrm{sym}} \in \mathcal{C}_1(H_{\Psi_\alpha}) \) be the canonical compact, self-adjoint operator, and let
\[
f(\lambda) := \det\nolimits_\zeta(I - \lambda L_{\mathrm{sym}})
= \frac{\Xi\left(\tfrac{1}{2} + i\lambda\right)}{\Xi\left(\tfrac{1}{2}\right)}.
\]

Then for each nontrivial zero \( \rho = \tfrac{1}{2} + i\gamma \) of the Riemann zeta function, the order of vanishing of \( \Xi(s) \) at \( \rho \) equals the algebraic multiplicity of the eigenvalue
\[
\mu_\rho := \frac{1}{i(\rho - \tfrac{1}{2})} = \frac{1}{\gamma}
\]
in the spectrum of \( L_{\mathrm{sym}} \). That is,
\[
\operatorname{ord}_\rho(\Xi)
= \operatorname{mult}_{\mathrm{spec}}(\mu_\rho).
\]

\medskip
\noindent
This identity follows by comparing the order of zeros in the Hadamard product for \( \Xi \) with the vanishing multiplicity in the Fredholm determinant expansion of \( f(\lambda) \).
\end{lemma}
% 
\begin{proof}[Proof of Lemma~\ref{lem:hadamard_fredholm_multiplicity}]
Let \( f(\lambda) := \det\nolimits_\zeta(I - \lambda L_{\mathrm{sym}}) \), and define
\[
g(\lambda) := \frac{\Xi\left( \tfrac{1}{2} + i\lambda \right)}{\Xi\left( \tfrac{1}{2} \right)}.
\]
By Theorem~\ref{thm:det-identity-revised}, we have \( f(\lambda) = g(\lambda) \) as entire functions.

\paragraph{Step 1: Zeros of the Hadamard Product.}
The Hadamard factorization of \( \Xi\left( \tfrac{1}{2} + i\lambda \right) \) gives
\[
g(\lambda) = \prod_{\rho \ne \tfrac{1}{2}} \left(1 - \frac{\lambda}{\lambda_\rho}\right) \exp\left( \frac{\lambda}{\lambda_\rho} \right),
\]
where \( \lambda_\rho := i(\rho - \tfrac{1}{2}) \). The multiplicity of each zero at \( \lambda_\rho \) is equal to the order of vanishing of \( \zeta \) at \( \rho \), i.e.,
\[
\operatorname{ord}_{\lambda_\rho}(g) = \operatorname{ord}_\rho(\Xi).
\]

\paragraph{Step 2: Zeros in the Fredholm Product.}
The Fredholm determinant has the canonical product expansion:
\[
f(\lambda) = \prod_{\mu \in \Spec(L_{\mathrm{sym}})} (1 - \lambda \mu)^{\operatorname{mult}_{\mathrm{spec}}(\mu)},
\]
where \(\mu = \mu_\rho = 1/\lambda_\rho\). Hence the zero of \( f \) at \( \lambda = \lambda_\rho \) corresponds to the pole of the logarithmic derivative
\[
\frac{d}{d\lambda} \log f(\lambda) = \sum_\mu \frac{\operatorname{mult}_{\mathrm{spec}}(\mu) \mu}{1 - \lambda \mu},
\]
with residue equal to the multiplicity.

\paragraph{Step 3: Equality of Multiplicities.}
Since \( f = g \) and both are entire of order one with identical zero sets, the order of the zero of \( g \) at \( \lambda_\rho \) must match the multiplicity of the pole in the trace-log derivative, i.e.,
\[
\operatorname{mult}_{\mathrm{spec}}(\mu_\rho) = \operatorname{ord}_\rho(\Xi).
\]

\paragraph{Conclusion.}
Hadamard multiplicities from the zeros of \( \Xi \) match the spectral multiplicities of the corresponding eigenvalues of \( L_{\mathrm{sym}} \). This completes the proof.
\end{proof}
% 

\begin{lemma}[Spectral Symmetry of \( L_{\sym} \)]
\label{lem:spectral_symmetry}

Let \( L_{\sym} \in \TC(\HPsi) \) be the canonical compact, self-adjoint operator on the exponentially weighted Hilbert space
\[
\HPsi := L^2(\R, e^{\alpha |x|} dx), \qquad \alpha > \pi,
\]
constructed via trace-norm convergence from real, even mollified kernels as in \thmref{thm:canonical_operator_realization}. Suppose the determinant identity holds:
\[
\det\nolimits_\zeta(I - \lambda L_{\sym}) = \frac{\XiR\left( \tfrac{1}{2} + i\lambda \right)}{\XiR\left( \tfrac{1}{2} \right)},
\]
as established in \thmref{thm:det_identity_revised}.

Then the spectrum of \( L_{\sym} \) is symmetric under reflection:
\[
\mu \in \Spec(L_{\sym}) \quad \Longrightarrow \quad -\mu \in \Spec(L_{\sym}),
\]
with multiplicities preserved.

\medskip

\noindent
This spectral symmetry follows from the functional identity \( \XiR(\tfrac{1}{2} + i\lambda) = \XiR(\tfrac{1}{2} - i\lambda) \), which implies that the spectral determinant is even in \( \lambda \). The Hadamard factorization structure described in \lemref{lem:hadamard_linear_form} encodes this evenness as spectral root symmetry. Moreover, the convolution kernel of \( L_{\sym} \) is real and even, so the operator commutes with parity, which reinforces the symmetry of its spectrum.
\end{lemma}

\begin{proof}[Proof of Lemma~\ref{lem:spectral_symmetry}]
Let \( \phi(\lambda) := \Xi\left(\tfrac{1}{2} + i\lambda\right) \). Since \( \Xi(s) \) is entire and satisfies the functional equation \( \Xi(s) = \Xi(1 - s) \), the function \( \lambda \mapsto \phi(\lambda) \) is real-valued and even on \( \mathbb{R} \).

Define the convolution kernel
\[
K(x - y) := \frac{1}{2\pi} \int_{\mathbb{R}} e^{i\lambda(x - y)} \phi(\lambda)\, d\lambda,
\]
so that the integral kernel \( K(x,y) := K(x - y) \) satisfies \( K(x,y) = K(y,x) \in \mathbb{R} \). The symmetry follows from the evenness of \( \phi(\lambda) \), since
\[
K(x - y) = \frac{1}{2\pi} \int_{\mathbb{R}} e^{i\lambda(x - y)} \phi(\lambda)\, d\lambda = K(y - x).
\]

Let \( \widetilde{L}_{\mathrm{sym}} \colon L^2(\mathbb{R}) \to L^2(\mathbb{R}) \) be the convolution operator with kernel \( K(x,y) \). Then \( \widetilde{L}_{\mathrm{sym}} \) is real, symmetric, compact, and self-adjoint.

\paragraph{Step 1: Spectral Symmetry in \( L^2(\mathbb{R}) \).}
By the spectral theorem for compact real symmetric operators on \( L^2(\mathbb{R}) \), the spectrum is symmetric about zero:
\[
\mu \in \Spec(\widetilde{L}_{\mathrm{sym}}) \quad \Longrightarrow \quad -\mu \in \Spec(\widetilde{L}_{\mathrm{sym}}),
\]
with algebraic multiplicities preserved.

\paragraph{Step 2: Transfer via Unitary Equivalence.}
Let \( U \colon H_{\Psi_\alpha} \to L^2(\mathbb{R}) \) denote the unitary transformation \( (Uf)(x) := \sqrt{\Psi_\alpha(x)} f(x) \). Then
\[
L_{\mathrm{sym}} = U^{-1} \widetilde{L}_{\mathrm{sym}} U,
\]
so \( L_{\mathrm{sym}} \) is unitarily equivalent to \( \widetilde{L}_{\mathrm{sym}} \). Since unitary equivalence preserves compactness, self-adjointness, and spectral multiplicities, it follows that
\[
\mu \in \Spec(L_{\mathrm{sym}}) \quad \Longrightarrow \quad -\mu \in \Spec(L_{\mathrm{sym}}),
\]
with matching algebraic multiplicities.
\end{proof}
%  

\begin{lemma}[Spectral Bijection Consistency]
\label{lem:spectral_bijection_consistency}
Let \( L_{\mathrm{sym}} \in \mathcal{C}_1(H_{\Psi_\alpha}) \) be the canonical compact, self-adjoint operator defined via the spectral determinant identity:
\[
\det\nolimits_\zeta(I - \lambda L_{\mathrm{sym}}) = \frac{\Xi\left(\tfrac{1}{2} + i\lambda\right)}{\Xi\left(\tfrac{1}{2}\right)}.
\]

Then the canonical spectral map
\[
\rho \longmapsto \mu_\rho := \frac{1}{i}(\rho - \tfrac{1}{2})
\]
defines a bijection between the multiset of nontrivial zeros \( \rho \in \mathbb{C} \) of the Riemann zeta function \( \zeta(s) \), and the multiset of nonzero eigenvalues of \( L_{\mathrm{sym}} \in \mathcal{C}_1(H_{\Psi_\alpha}) \), with multiplicities preserved.

\medskip
\noindent
Explicitly,
\[
\operatorname{Spec}(L_{\mathrm{sym}}) \setminus \{0\}
= \left\{ \mu_\rho := \frac{1}{i}(\rho - \tfrac{1}{2}) \,\middle|\, \zeta(\rho) = 0 \right\},
\]
as multisets—that is, with algebraic multiplicities of eigenvalues equal to the order of vanishing of \( \zeta(s) \) at \( \rho \).
\end{lemma}

\begin{proof}[Proof of Lemma~\ref{lem:spectral_bijection_consistency}]
Let \( \rho = \tfrac{1}{2} + i\gamma \) be a nontrivial zero of \( \zeta(s) \), and define
\[
\mu_\rho := \frac{1}{i(\rho - \tfrac{1}{2})}.
\]

\paragraph{Bijection Properties.}
The following properties are established in earlier results:

\begin{itemize}
  \item By Lemma~\ref{lem:zero-to-eigenvalue-injection}, each nontrivial zero \( \rho \) maps to a nonzero eigenvalue \( \mu_\rho \in \Spec(L_{\mathrm{sym}}) \), and this map is injective with multiplicity preserved.

  \item Lemma~\ref{lem:spectral_exhaustivity} shows that every nonzero eigenvalue \( \mu \in \Spec(L_{\mathrm{sym}}) \setminus \{0\} \) arises from such a \( \rho \), establishing surjectivity of the map.

  \item Lemma~\ref{lem:spectral-multiplicity-matching} confirms that the algebraic multiplicity of each eigenvalue \( \mu_\rho \) equals the order of vanishing of \( \zeta(s) \) at \( \rho \).
\end{itemize}

\paragraph{Conclusion.}
Therefore, the map
\[
\rho \mapsto \mu_\rho := \frac{1}{i(\rho - \tfrac{1}{2})}
\]
is a bijection of multisets between the nontrivial zeros of \( \zeta(s) \) and the nonzero spectrum of \( L_{\mathrm{sym}} \), preserving algebraic multiplicities. This completes the canonical spectral correspondence enforced by the determinant identity.
\end{proof}
%  

%------------------------------------------------------------------
\subsection{Main Result: Spectral Bijection}

\begin{theorem}[Spectral Bijection with Nontrivial Zeta Zeros]
\label{thm:spectral_zero_bijection_revised}
Let \( L_{\sym} \in \TC(\HPsi) \) be the canonical compact, self-adjoint operator whose Carleman \(\zeta\)-regularized Fredholm determinant satisfies:
\[
\det\nolimits_\zeta(I - \lambda L_{\sym}) = \frac{\XiR\left(\tfrac{1}{2} + i\lambda\right)}{\XiR\left(\tfrac{1}{2}\right)} \qquad \forall \lambda \in \C.
\]

Then there exists a canonical multiplicity-preserving bijection between:
\begin{itemize}
  \item the multiset of nontrivial zeros \( \rho = \tfrac{1}{2} + i\gamma \) of the Riemann zeta function \( \zetaR(s) \), and
  \item the multiset of nonzero eigenvalues \( \mu \in \Spec(L_{\sym}) \setminus \{0\} \),
\end{itemize}
given by the spectral correspondence:
\[
\rho \longmapsto \mu_\rho := \frac{1}{i(\rho - \tfrac{1}{2})}.
\]

\medskip

\noindent
This bijection satisfies:
\begin{itemize}
  \item \( \mu_\rho \in \R \setminus \{0\} \) for each \( \rho \ne \tfrac{1}{2} \);
  \item \( \Spec(L_{\sym}) \setminus \{0\} = \left\{ \mu_\rho : \zetaR(\rho) = 0 \right\} \), as multisets;
  \item \( \operatorname{ord}_\rho(\zetaR) = \operatorname{mult}_{\mathrm{spec}}(\mu_\rho) \), i.e., analytic multiplicity equals spectral multiplicity.
\end{itemize}

\medskip

\noindent
This result follows from:
\begin{enumerate}
  \item The determinant identity, which ensures that the zero set of the entire function \( \lambda \mapsto \det\nolimits_\zeta(I - \lambda L_{\sym}) \) coincides with the zero set of \( \XiR(\tfrac{1}{2} + i\lambda) \);
  \item The trace-class spectral theorem, which guarantees that the zeros of the determinant correspond precisely to the nonzero spectrum of \( L_{\sym} \), with multiplicities preserved;
  \item The normalization \( \XiR(\tfrac{1}{2}) \ne 0 \), which implies \( \lambda = 0 \) is not a zero of the determinant, and thus \( 0 \notin \Spec(L_{\sym}) \).
\end{enumerate}
\end{theorem}

\begin{proof}[Proof of \thmref{thm:spectral_zero_bijection_revised}]
Let \( \rho = \tfrac{1}{2} + i\gamma \) be a nontrivial zero of the Riemann zeta function \( \zeta(s) \), and define the corresponding spectral value
\[
\mu_\rho := \frac{1}{i}(\rho - \tfrac{1}{2}).
\]

\paragraph{Injectivity.}
By \lemref{lem:zero_to_eigenvalue_injection}, each nontrivial zero \( \rho \) maps to a unique nonzero eigenvalue \( \mu_\rho \in \operatorname{Spec}(L_{\mathrm{sym}}) \), and distinct zeros yield distinct eigenvalues. This confirms the injectivity of the spectral map \( \rho \mapsto \mu_\rho \).

\paragraph{Surjectivity.}
By \lemref{lem:spectral_exhaustivity}, every nonzero eigenvalue \( \mu \in \operatorname{Spec}(L_{\mathrm{sym}}) \setminus \{0\} \) arises from some nontrivial zero \( \rho \in \mathbb{C} \), satisfying \( \mu = \mu_\rho \). This proves surjectivity.

\paragraph{Multiplicity Preservation.}
By \lemref{lem:spectral_multiplicity_matching}, the algebraic multiplicity of each eigenvalue \( \mu_\rho \) equals the order of vanishing of \( \zeta(s) \) at the corresponding zero \( \rho \). Hence, the spectral map preserves multiplicities.

\paragraph{Conclusion.}
The map
\[
\rho \longmapsto \mu_\rho := \frac{1}{i}(\rho - \tfrac{1}{2})
\]
defines a canonical multiplicity-preserving bijection between the multiset of nontrivial zeros of \( \zeta(s) \) and the multiset of nonzero eigenvalues of \( L_{\mathrm{sym}} \in \mathcal{C}_1(H_{\Psi_\alpha}) \). This correspondence is uniquely determined by the regularized Fredholm determinant identity and fully realizes the spectral encoding of the Riemann zeta function's zeros.
\end{proof}


\begin{lemma}[Spectral Heat Trace Representation]
\label{lem:spectral_measure_heat_semigroup}
Let \( L_{\sym} \in \TC(\HPsi) \) be a compact, self-adjoint operator with discrete spectrum \( \{ \mu_\rho \} \subset \R \). Then the associated spectral projection measure \( E_\lambda \) satisfies:
\[
\Tr(e^{-t L_{\sym}^2}) = \int_{\R} e^{-t\lambda^2} \, dN(\lambda),
\]
where the eigenvalue counting function \( N(\lambda) \) is defined by
\[
N(\lambda) := \sum_{\mu_\rho \leq \lambda} \operatorname{mult}_{\mathrm{spec}}(\mu_\rho),
\]
counting each eigenvalue \( \mu_\rho \in \Spec(L_{\sym}) \) with its algebraic multiplicity.

\medskip

\noindent
Consequently, the canonical Carleman \(\zeta\)-regularized Fredholm determinant satisfies
\[
\log \det\nolimits_{\zeta}(I - \lambda L_{\sym}) = - \int_{\R} \log\left(1 - \frac{\lambda}{\mu} \right) \, dN(\mu),
\]
valid for all \( \lambda \in \C \setminus \{ \mu_\rho \} \).

\medskip

\noindent
In particular, the Laplace transform of the spectral density \( dN(\lambda) \) defines the heat trace, and the Mellin transform of this trace is connected to the completed zeta function \( \XiR(s) \). This forms the analytic backbone for the determinant identity, and reflects classical Tauberian structure in the spirit of \cite{Korevaar2004Tauberian}.
\end{lemma}

\begin{proof}[Proof of \lemref{lem:spectral_measure_heat_semigroup}]
We prove the result using the spectral theorem and trace properties of compact, self-adjoint operators.

Since \( L_{\mathrm{sym}} \in \mathcal{C}_1(H_{\Psi_\alpha}) \) is compact and self-adjoint, it admits a discrete spectral decomposition:
\[
L_{\mathrm{sym}} = \sum_{\rho} \mu_\rho P_\rho,
\]
where each \( \mu_\rho \in \mathbb{R} \) is an eigenvalue with finite multiplicity and \( P_\rho \) denotes the corresponding orthogonal projection.

\paragraph{Heat Trace via Spectral Functional Calculus.}
Using the spectral calculus, the heat semigroup satisfies:
\[
e^{-t L_{\mathrm{sym}}^2} = \sum_{\rho} e^{-t\mu_\rho^2} P_\rho,
\]
and thus the trace is given by:
\[
\operatorname{Tr}(e^{-t L_{\mathrm{sym}}^2}) = \sum_{\rho} e^{-t\mu_\rho^2} \operatorname{Tr}(P_\rho)
= \sum_{\rho} \operatorname{mult}_{\mathrm{spec}}(\mu_\rho) \, e^{-t\mu_\rho^2}.
\]
Define the spectral counting measure
\[
dN(\lambda) := \sum_{\rho} \operatorname{mult}_{\mathrm{spec}}(\mu_\rho) \, \delta_{\mu_\rho}(\lambda),
\]
so the trace becomes the Laplace-type integral:
\[
\operatorname{Tr}(e^{-t L_{\mathrm{sym}}^2}) = \int_{\mathbb{R}} e^{-t\lambda^2} \, dN(\lambda).
\]

\paragraph{Spectral Trace Identity for Test Functions.}
For any test function \( \phi \in \mathcal{S}(\mathbb{R}) \), the spectral theorem gives
\[
\phi(L_{\mathrm{sym}}) = \sum_{\rho} \phi(\mu_\rho) P_\rho,
\]
and taking the trace yields:
\[
\operatorname{Tr}(\phi(L_{\mathrm{sym}})) = \sum_{\rho} \phi(\mu_\rho) \operatorname{Tr}(P_\rho)
= \int_{\mathbb{R}} \phi(\lambda) \, dN(\lambda).
\]

\paragraph{Fredholm Logarithmic Expansion.}
The Carleman \(\zeta\)-regularized determinant admits the logarithmic trace representation:
\[
\log \det\nolimits_\zeta(I - \lambda L_{\mathrm{sym}})
= - \sum_{\rho} \log\left(1 - \lambda \mu_\rho^{-1} \right)
= - \int_{\mathbb{R}} \log\left(1 - \frac{\lambda}{\lambda'} \right) \, dN(\lambda'),
\]
valid for \( \lambda \in \mathbb{C} \setminus \{ \mu_\rho \} \). This matches the Hadamard representation for entire functions of order one, whose zero set corresponds to \( \operatorname{Spec}(L_{\mathrm{sym}}) \setminus \{0\} \).

\paragraph{Conclusion.}
The trace of the heat semigroup \( e^{-t L_{\mathrm{sym}}^2} \) and the logarithmic expansion of the determinant are both encoded by the spectral measure \( dN \), which captures spectral multiplicity. This completes the proof.
\end{proof}


%------------------------------------------------------------------
\subsection{Corollary: Spectrum Determines Zeta}

\begin{corollary}[Spectral Reconstruction of the Completed Zeta Function]
\label{cor:spectrum_determines_zeta}
Let \( L_{\mathrm{sym}} \in \mathcal{C}_1(H_{\Psi_\alpha}) \) be the canonical self-adjoint trace-class operator satisfying
\[
\det\nolimits_\zeta(I - \lambda L_{\mathrm{sym}}) = \frac{\Xi(\tfrac{1}{2} + i\lambda)}{\Xi(\tfrac{1}{2})}.
\]

Then the multiset spectrum \( \operatorname{Spec}(L_{\mathrm{sym}}) \setminus \{0\} \), counted with algebraic multiplicities, uniquely determines the completed zeta function \( \Xi(s) \), up to the normalization constant \( \Xi(\tfrac{1}{2}) \). Consequently, the spectrum also determines the nontrivial zeros of the Riemann zeta function \( \zeta(s) \), including their multiplicities.
\end{corollary}

\begin{proof}[Proof of \corref{cor:spectrum_determines_zeta}]
Since \( L_{\sym} \in \TC(\HPsi) \) is compact and self-adjoint, its Fredholm determinant satisfies
\[
\det\nolimits_\zeta(I - \lambda L_{\sym}) = \prod_{\mu \in \Spec(L_{\sym})} (1 - \lambda \mu)^{\operatorname{mult}_{\mathrm{spec}}(\mu)},
\]
which converges absolutely for \( |\lambda| \) sufficiently small and admits meromorphic continuation to all of \( \C \).

By \thmref{thm:spectral_zero_bijection_revised}, the multiset of nonzero spectral values \( \mu_\rho := \frac{1}{i(\rho - \tfrac{1}{2})} \), with multiplicities \( \operatorname{ord}_\rho(\zetaR) \), exhausts \( \Spec(L_{\sym}) \setminus \{0\} \).

Hence, the determinant admits the expression
\[
\log \det\nolimits_\zeta(I - \lambda L_{\sym}) = - \sum_{\rho} \operatorname{ord}_\rho(\zetaR) \log\left(1 - \frac{\lambda}{\lambda_\rho} \right), \qquad \lambda_\rho := i(\rho - \tfrac{1}{2}).
\]

Exponentiating, we obtain
\[
\det\nolimits_\zeta(I - \lambda L_{\sym}) = \prod_{\rho} \left(1 - \frac{\lambda}{\lambda_\rho} \right)^{\operatorname{ord}_\rho(\zetaR)}.
\]
Comparing with the Hadamard factorization
\[
\frac{\XiR(\tfrac{1}{2} + i\lambda)}{\XiR(\tfrac{1}{2})} = \prod_{\rho} \left(1 - \frac{\lambda}{\lambda_\rho} \right)^{\operatorname{ord}_\rho(\zetaR)},
\]
we conclude that the spectral data—i.e., the multiset \( \Spec(L_{\sym}) \setminus \{0\} \) with algebraic multiplicities—uniquely determines the completed zeta function \( \XiR(s) \) up to the value at the center, \( \XiR(\tfrac{1}{2}) \).

Since \( \XiR(s) \) is entire of order one, this normalization suffices to determine \( \XiR(s) \) completely, and thus the multiset of nontrivial zeros of \( \zetaR(s) \).
\end{proof}


\begin{lemma}[Spectrum Reality and RH Equivalence]
\label{lem:spectrum-reality-implies-rh}
If all zeros of \( \Xi(\tfrac{1}{2} + i\lambda) \) lie on the real axis, then the canonical convolution operator \( L_{\mathrm{sym}} \) has real spectrum.

Conversely, if \( L_{\mathrm{sym}} \in \mathcal{C}_1(H_{\Psi_\alpha}) \) is self-adjoint, then all eigenvalues \( \mu_\rho \in \mathbb{R} \) imply that the associated nontrivial zeta zeros \( \rho \in \mathbb{C} \) satisfy \( \operatorname{Re}(\rho) = \tfrac{1}{2} \), i.e., the Riemann Hypothesis holds.
\end{lemma}
%  
\begin{proof}[Proof of \lemref{lem:reality_of_spectrum_and_rh}]
Let \( \rho = \beta + i\gamma \) be a nontrivial zero of the Riemann zeta function \( \zetaR(s) \). Define its canonical spectral image:
\[
\mu_\rho = \frac{1}{i(\rho - \tfrac{1}{2})} = \frac{1}{i((\beta - \tfrac{1}{2}) + i\gamma)}.
\]

Set \( z := \beta - \tfrac{1}{2} + i\gamma \in \C \). Then
\[
\mu_\rho = \frac{-i}{z} = \frac{-i((\beta - \tfrac{1}{2}) - i\gamma)}{(\beta - \tfrac{1}{2})^2 + \gamma^2}
= \frac{\gamma}{(\beta - \tfrac{1}{2})^2 + \gamma^2}
- i \cdot \frac{\beta - \tfrac{1}{2}}{(\beta - \tfrac{1}{2})^2 + \gamma^2}.
\]

Hence, \( \mu_\rho \in \R \) if and only if the imaginary part vanishes:
\[
\Im(\mu_\rho) = 0 \quad \Longleftrightarrow \quad \beta = \tfrac{1}{2}.
\]

That is,
\[
\mu_\rho \in \R \iff \rho \in \tfrac{1}{2} + i\R.
\]

Since the canonical spectral map \( \rho \mapsto \mu_\rho \) is injective and covers all nontrivial zeros of \( \zetaR \), this correspondence implies:
\[
\Spec(L_{\sym}) \subset \R \quad \Longleftrightarrow \quad \text{all } \rho \in \mathcal{Z}_\zeta \text{ satisfy } \Re(\rho) = \tfrac{1}{2}.
\]

\medskip

\noindent
Equivalently,
\[
\boxed{
\Spec(L_{\sym}) \subset \R \quad \Longleftrightarrow \quad \RH
}
\]
as claimed.
\end{proof}


%------------------------------------------------------------------
% --- Clarify logical independence of RH ---
\begin{remark}[No RH Assumption Used in Spectral Encoding]
\label{rem:no_rh_assumption}
Throughout Chapters~\ref{sec:determinant_identity}–\ref{sec:spectral_correspondence}, all constructions, estimates, and determinant identities are established unconditionally—without assuming the Riemann Hypothesis, GUE statistics, or any symmetry of the zeta zero distribution.

\medskip

In particular:
\begin{itemize}
  \item The canonical operator \( \Lsym \in \TC(\HPsi) \) is constructed via mollified convolution and inverse Fourier transform of \( \Xi(s) \), with trace-norm convergence and self-adjointness proven analytically. No assumption on the spectral location of zeta zeros is required.

  \item The determinant identity
  \[
  \det\nolimits_\zeta(I - \lambda \Lsym) = \frac{\XiR\left( \tfrac{1}{2} + i\lambda \right)}{\XiR\left( \tfrac{1}{2} \right)}
  \]
  is derived via Laplace regularization of the heat trace and analytic control of its singularity structure, as in Chapter~\ref{sec:determinant_identity}. No implicit assumption of real spectrum is used in this derivation.

  \item The spectral map \( \rho \mapsto \mu_\rho := \tfrac{1}{i}(\rho - \tfrac{1}{2}) \) and its inverse are defined algebraically and proven to be a bijection using Fredholm theory and Hadamard factorization of entire functions of order one. See \thmref{thm:spectral_zero_bijection_revised}. This spectral encoding is verified independently of RH.

  \item All Schatten norm, kernel decay, and operator compactness results in Chapters~\ref{sec:operator_construction}–\ref{sec:heat_kernel_asymptotics} follow from classical operator theory and Fourier analysis. No probabilistic, arithmetic, or spectral assumptions are made about the zeta zeros.
\end{itemize}

\medskip

\noindent
The Riemann Hypothesis enters for the first time in Chapter~\ref{sec:spectral_implications}, where it is shown to be equivalent to the condition \( \Spec(\Lsym) \subset \R \) using the previously established spectral bijection and determinant identity. The proof is strictly modular and acyclic.
\end{remark}


\subsection*{Summary}
\label{sec:foundations_summary}

\textbf{Operator-Theoretic Foundations}
\begin{itemize}
  \item \defref{def:compact_operator} — Compact operators: norm limits of finite-rank maps with discrete spectrum.
  \item \defref{def:trace_class_operator}, \defref{def:trace_norm} — Trace-class operators \( T \in \TC(H) \) with finite trace norm \( \|T\|_{\Tr} := \Tr(|T|) \); Banach completeness and unitary invariance.
  \item \defref{def:selfadjoint_operator} — Self-adjointness as maximal symmetry enabling spectral calculus and semigroup generation.
\end{itemize}

\textbf{Weighted Spaces and Function Classes}
\begin{itemize}
  \item \defref{def:exponential_weight}, \defref{def:weighted_schwartz_space} — The space \( \HPsi = L^2(\R, e^{\alpha|x|}\,dx) \), with \( \Schwartz(\R) \subset \HPsi \) a dense core.
  \item \lemref{lem:density_schwartz_weighted_L2} — Density of \( \Schwartz \subset \HPsi \) in norm and graph topology.
  \item \remref{rem:sobolev_core_reference} — Alternate justification: \( \Schwartz \hookrightarrow H^s_\alpha \hookrightarrow \HPsi \) via Sobolev embeddings.
\end{itemize}

\textbf{Analytic and Spectral Estimates}
\begin{itemize}
  \item \lemref{lem:xi_growth_bound}, \lemref{lem:weighted_L1_inverse_FT_xi} — The profile \( \Xi(\tfrac{1}{2} + i\lambda) \in \PW{\pi} \), with inverse transform in \( L^1(\R, \Psi_\alpha^{-1}) \).
  \item \lemref{lem:decay_mollified_kernel}, \lemref{lem:L1_integrability_conjugated_kernel} — Mollifiers \( k_t \in \Schwartz \), conjugated kernels integrable.
  \item \lemref{lem:uniform_L1_conjugated_kernel}, \lemref{lem:trace_class_via_weighted_L1} — Trace norm convergence \( \|L_t - \Lsym\|_{\TC} \to 0 \) and Simon’s trace-class inclusion criterion.
  \item \lemref{lem:trace_class_conjugated_kernel}, \lemref{lem:trace_class_failure_alpha_leq_pi}, \propref{prop:trace_class_sharpness} — Trace-class fails for \( \alpha \le \pi \): sharp exponential decay threshold.
  \item \lemref{lem:unitary_conjugation_trace_class} — Trace norm preserved under unitary weight conjugation.
\end{itemize}

\textbf{Operator Properties of \texorpdfstring{\( L_t \)}{Lt}}
\begin{itemize}
  \item \propref{prop:boundedness_Lt_weighted}, \propref{prop:compactness_Lt} — Boundedness and compactness of \( L_t \) via mollified kernel structure.
  \item \propref{prop:symmetry_Lt_Schwartz}, \propref{prop:selfadjointness_Lt} — \( L_t \) is symmetric on \( \Schwartz \) and extends to a self-adjoint operator.
  \item \propref{prop:core_schwartz_density} — \( \Schwartz \) is a core for the limit operator \( \Lsym \).
\end{itemize}

\textbf{Canonical Operator Realization}
\begin{itemize}
  \item \thmref{thm:canonical_operator_realization} — Convergence \( L_t \to \Lsym \in \TC(\HPsi) \); defines the canonical compact self-adjoint operator realizing the spectral determinant.
\end{itemize}

\paragraph{Chapter Closure.}
This chapter establishes the analytic and operator-theoretic base for all that follows. The canonical convolution operator \( \Lsym \in \TC(\HPsi) \) is defined as the trace-norm limit of mollified Fourier convolution operators \( L_t \). Its construction relies on Paley--Wiener estimates, exponential decay, Sobolev density, and trace-class embedding theorems. The determinant identity
\[
\detz(I - \lambda \Lsym)
= \frac{\Xi\left(\tfrac{1}{2} + i\lambda \right)}{\Xi\left(\tfrac{1}{2} \right)}
\]
is proven in \secref{sec:determinant_identity}, resting entirely on this analytic groundwork.



\section{Heat Kernel Bounds and Short-Time Trace Estimates}
\label{sec:heat-kernel-asymptotics}

\subsection*{Introduction}

This chapter establishes the analytic infrastructure for defining and analyzing the canonical compact operator \( L_{\mathrm{sym}} \), which realizes the completed Riemann zeta function \( \Xi(s) \) via its Fredholm determinant. The primary goal is to verify that mollified convolution operators associated with the inverse Fourier transform of \( \Xi \) are compact, trace class, and converge in trace norm to a self-adjoint limit operator \( L_{\mathrm{sym}} \in \mathcal{C}_1(H_{\Psi_\alpha}) \).

The constructions here verify:

\begin{itemize}
    \item Schatten-class properties of Hilbert–Schmidt and trace-class operators, following \cite[Ch.~4]{Simon2005TraceIdeals} and \cite[Ch.~VI]{ReedSimon1980I}, including the completeness of \( \mathcal{C}_1 \) and the trace-norm topology.
    
    \item Sufficient conditions for compactness and self-adjointness of integral operators with symmetric Hermitian kernels, using distributional domains and exponential conjugation.
    
    \item The structure of the weighted Schwartz space \( \mathcal{S}_\alpha(\mathbb{R}) \subset L^2(\mathbb{R}, e^{\alpha |x|}\, dx) \), for \( \alpha > \pi \), ensuring Fourier duality and decay control for entire functions of exponential type \( \pi \) \cite{Levin1996EntireLectures}.
    
    \item Uniform kernel bounds and mollifier admissibility for defining the regularized heat operators \( e^{-t L_t^2} \), together with analytic kernel estimates necessary for short-time trace control and Tauberian convergence.
\end{itemize}

These ingredients culminate in the construction of mollified convolution operators \( L_t \), and in the verification of trace-norm convergence
\[
L_t \to L_{\mathrm{sym}} \in \mathcal{C}_1(H_{\Psi_\alpha}) \quad \text{as } t \to 0^+.
\]
This limit defines the canonical spectral operator underlying the determinant identity
\[
\det\nolimits_{\zeta}(I - \lambda L_{\mathrm{sym}}) = \frac{\Xi\left(\tfrac{1}{2} + i\lambda\right)}{\Xi\left(\tfrac{1}{2}\right)},
\]
which is rigorously established without assuming RH.

\medskip

The analytic architecture developed here underpins all subsequent spectral and determinant identities.
See Appendix~\ref{app:dependency-graph} for a visual DAG linking these foundational tools to the modular proof of RH.


%------------------------------------------------------------------
\subsection{Definitions}
% --- Heat semigroup and spectral trace definition ---
\begin{definition}[Heat Operator for Compact Self-Adjoint Operators]
\label{def:heat_operator}

Let \( H \) be a separable complex Hilbert space, and let \( L \colon H \to H \) be a compact, self-adjoint, trace-class operator.

Then \( L \) has a discrete real spectrum \( \{ \mu_n \}_{n=1}^\infty \subset \R \), accumulating only at zero, with an associated orthonormal eigenbasis \( \{ e_n \}_{n=1}^\infty \subset H \) such that
\[
L e_n = \mu_n e_n, \qquad \text{with } \mu_n \to 0 \quad \text{as } n \to \infty.
\]

For each \( t > 0 \), the heat operator is defined via spectral calculus as:
\[
e^{-tL^2} := \sum_{n=1}^\infty e^{-t \mu_n^2} \langle \cdot, e_n \rangle e_n.
\]
This series converges in trace norm, and the family \( \{ e^{-tL^2} \}_{t > 0} \subset \TC(H) \cap \mathcal{B}(H) \) forms a strongly continuous, holomorphic, contractive semigroup generated by the nonnegative operator \( L^2 \in \TC(H) \).

\medskip

The associated heat trace is given by
\[
\Tr(e^{-tL^2}) = \sum_{n=1}^\infty e^{-t \mu_n^2},
\]
with absolute convergence guaranteed by \( L^2 \in \TC(H) \).
\end{definition}


%------------------------------------------------------------------
\subsection{Kernel Estimates and Local Bounds}

% --- Upper bound for Tr(e^{-tL^2_sym}) from spectrum partitioning ---
\begin{lemma}[Short-Time Upper Bound for the Heat Trace]
\label{lem:hk-upper-bound}
Let \( L_{\mathrm{sym}} \in \mathcal{C}_1(H_{\Psi_\alpha}) \) be the canonical compact, self-adjoint operator with spectrum \( \{ \mu_n \} \subset \mathbb{R} \), and let \( \{ e_n \} \subset H_{\Psi_\alpha} \) be an orthonormal basis of eigenvectors.

Then there exists a constant \( c_2 > 0 \) such that for all \( 0 < t \le 1 \),
\[
\operatorname{Tr}\left( e^{-t L_{\mathrm{sym}}^2} \right) \le c_2\, t^{-1/2}.
\]

\noindent
This inequality holds uniformly on compact subintervals of \( (0,1] \), and reflects the spectral scaling of dimension one. The constant \( c_2 \) depends only on the short-time behavior of the diagonal of the heat kernel \( K_t(x,x) \), which admits Gaussian upper bounds in the sense of Varadhan-type estimates.
\end{lemma}
% 
\begin{proof}[Proof of Lemma~\ref{lem:hk_upper_bound}]
Let \( L_{\mathrm{sym}} \in \mathcal{C}_1(H_{\Psi_\alpha}) \), with spectral decomposition \( L_{\mathrm{sym}} e_n = \mu_n e_n \) for an orthonormal basis \( \{ e_n \} \subset H_{\Psi_\alpha} \). The heat trace is given by:
\[
\operatorname{Tr}(e^{-t L_{\mathrm{sym}}^2}) = \sum_{n=1}^\infty e^{-t \mu_n^2},
\]
which converges absolutely since \( L_{\mathrm{sym}}^2 \in \mathcal{C}_1 \).

\paragraph{Step 1: Partitioning the Spectrum.}
Fix \( 0 < t \le 1 \), and define:
\[
A_1(t) := \left\{ n : |\mu_n| \le t^{-1/2} \right\}, \quad
A_2(t) := \left\{ n : |\mu_n| > t^{-1/2} \right\}.
\]
We split the trace sum:
\[
\sum_n e^{-t \mu_n^2} = \sum_{n \in A_1(t)} e^{-t \mu_n^2} + \sum_{n \in A_2(t)} e^{-t \mu_n^2}.
\]

\paragraph{Step 2: Bounds on Each Region.}
On \( A_1(t) \), we have \( e^{-t \mu_n^2} \le 1 \), so:
\[
\sum_{n \in A_1(t)} e^{-t \mu_n^2} \le |A_1(t)|.
\]
On \( A_2(t) \), use \( |\mu_n| > t^{-1/2} \Rightarrow t \mu_n^2 > 1 \Rightarrow e^{-t \mu_n^2} < e^{-1} \), hence:
\[
\sum_{n \in A_2(t)} e^{-t \mu_n^2} \le e^{-1} \cdot |A_2(t)|.
\]

\paragraph{Step 3: Bounding Index Set Cardinalities.}
Note:
\[
\sum_{n \in A_1(t)} |\mu_n| \ge t^{1/2} \cdot |A_1(t)| \quad \Rightarrow \quad |A_1(t)| \le t^{-1/2} \sum_{n \in A_1(t)} |\mu_n|.
\]
Similarly for \( A_2(t) \). Summing yields:
\[
|A_1(t)| + |A_2(t)| \le t^{-1/2} \sum_n |\mu_n| = t^{-1/2} \|L_{\mathrm{sym}}\|_{\mathcal{C}_1}.
\]

\paragraph{Step 4: Final Estimate.}
Combining the bounds:
\[
\operatorname{Tr}(e^{-t L_{\mathrm{sym}}^2})
\le |A_1(t)| + e^{-1} |A_2(t)|
\le (1 + e^{-1}) \cdot t^{-1/2} \cdot \|L_{\mathrm{sym}}\|_{\mathcal{C}_1}.
\]
Setting \( c_2 := (1 + e^{-1}) \cdot \|L_{\mathrm{sym}}\|_{\mathcal{C}_1} \) completes the proof.
\end{proof}


% --- Lower bound for Tr(e^{-tL^2_sym}) from Gaussian inequality ---
\begin{lemma}[Short-Time Lower Bound for the Heat Trace]
\label{lem:hk_lower_bound}
Let \( L_{\mathrm{sym}} \in \mathcal{C}_1(H_{\Psi_\alpha}) \) be the canonical compact, self-adjoint operator, and define its trace norm
\[
C_1 := \| L_{\mathrm{sym}} \|_{\mathcal{C}_1}.
\]
Then for all \( t \in (0,1] \), the spectral heat trace satisfies the lower bound:
\[
\operatorname{Tr}\left( e^{-t L_{\mathrm{sym}}^2} \right) \ge \frac{1}{4C_1} \, t^{-1/2}.
\]

\medskip
\noindent
This estimate reflects the dominant contribution of the low-frequency spectrum in the short-time regime. The constant \( \frac{1}{4C_1} \) is explicit and depends only on the Schatten–1 norm of \( L_{\mathrm{sym}} \). The semigroup regularity and existence of the trace are guaranteed by \lemref{lem:heat_semigroup_wellposed}, and the decay behavior of \( \mu_\rho \to 0 \) from \lemref{lem:spectral_decay_bounds} ensures spectral concentration near the origin. For the corresponding upper bound, see \lemref{lem:hk_upper_bound}.
\end{lemma}

\begin{proof}[Proof of Lemma~\ref{lem:hk-lower-bound}]
Let \( L := L_{\mathrm{sym}} \in \mathcal{C}_1(H_{\Psi_\alpha}) \) have discrete real spectrum \( \{ \mu_n \}_{n=1}^\infty \subset \mathbb{R} \) with associated orthonormal basis \( \{ e_n \} \subset H_{\Psi_\alpha} \). Then
\[
\operatorname{Tr}(e^{-t L^2}) = \sum_{n=1}^\infty e^{-t \mu_n^2}.
\]

\paragraph{Step 1: Spectral Splitting.}
Fix \( t \in (0,1] \). Define the spectral subset
\[
A(t) := \{ n \in \mathbb{N} : |\mu_n| \le t^{-1/2} \}.
\]
For each \( n \in A(t) \), we have \( |\mu_n| \sqrt{t} \le 1 \), hence
\[
e^{-t \mu_n^2} \ge e^{-1} \ge \tfrac{1}{4}.
\]

\paragraph{Step 2: Lower Bound via Trace Norm.}
The trace norm of \( L \) satisfies
\[
\sum_{n \notin A(t)} |\mu_n| \ge t^{-1/2} \cdot |A(t)^c|,
\quad \Rightarrow \quad |A(t)| \ge t^{-1/2} \cdot \left( \|L\|_{\mathcal{C}_1}^{-1} \right).
\]
Thus,
\[
\operatorname{Tr}(e^{-t L^2}) \ge \sum_{n \in A(t)} e^{-t \mu_n^2}
\ge \tfrac{1}{4} \cdot |A(t)| \ge \tfrac{1}{4 \|L\|_{\mathcal{C}_1}} \cdot t^{-1/2}.
\]

\paragraph{Conclusion.}
Set \( c_1 := \tfrac{1}{4 \|L\|_{\mathcal{C}_1}} \). Then for all \( t \in (0,1] \),
\[
\operatorname{Tr}(e^{-t L^2}) \ge c_1 \, t^{-1/2}.
\]
\end{proof}


% --- Diagonal short-time asymptotic expansion K_t(x,x) ~ sum a_n(x) t^{n-1/2} ---
\begin{lemma}[Uniform Short-Time Heat Kernel Expansion]
\label{lem:hk_expansion_uniform}
Let \( L_{\mathrm{sym}} \in \mathcal{C}_1(H_{\Psi_\alpha}) \) be the canonical compact, self-adjoint operator, and let \( K_t(x,y) \) denote the integral kernel of the semigroup \( e^{-t L_{\mathrm{sym}}^2} \), which exists and is jointly smooth for all \( t > 0 \), as guaranteed by \lemref{lem:heat_semigroup_wellposed}.

Then as \( t \to 0^+ \), the diagonal heat kernel admits a full short-time asymptotic expansion of the form:
\[
K_t(x,x) \sim \sum_{n=0}^\infty a_n(x)\, t^{n - \frac{1}{2}},
\]
where \( \{ a_n(x) \} \subset C^\infty(\mathbb{R}) \) are smooth coefficient functions depending on the local structure of the mollified Fourier symbol of \( L_{\mathrm{sym}} \), with regularity ensured by the exponential decay properties of the kernel (see \lemref{lem:decay_mollified_kernel}). This expansion is valid uniformly on compact subsets of \( \mathbb{R} \).

\medskip
\noindent
More precisely, for each \( N \in \mathbb{N} \) and every compact set \( K \subset \mathbb{R} \), there exist constants \( C_N > 0 \) and \( t_0 > 0 \) such that for all \( x \in K \) and \( 0 < t \le t_0 \),
\[
\left| K_t(x,x) - \sum_{n=0}^{N-1} a_n(x)\, t^{n - \frac{1}{2}} \right| \le C_N\, t^{N - \frac{1}{2}}.
\]

\medskip
\noindent
This expansion underlies the local behavior of the heat trace and supports its Mellin transform representation in \lemref{lem:spectral_zeta_from_heat}.
\end{lemma}

\begin{proof}[Proof of \lemref{lem:hk_expansion_uniform}]
Let \( L := L_{\mathrm{sym}} \in \mathcal{C}_1(H_{\Psi_\alpha}) \) be the canonical self-adjoint, compact convolution operator, and let \( K_t(x,y) \) denote the integral kernel of the semigroup \( e^{-tL^2} \), defined via spectral functional calculus.

\paragraph{Step 1: Regularity of the Generator and Kernel.}
The operator \( L \) is constructed from the inverse Fourier transform of the entire function
\[
\phi(\lambda) := \Xi\left(\tfrac{1}{2} + i\lambda\right),
\]
which lies in the Paley--Wiener class of exponential type \( \pi \). Its mollified Fourier approximants define convolution operators \( L_\varepsilon \) with kernels in \( \mathcal{S}(\mathbb{R}^2) \), converging in trace norm to \( L \). Consequently, the squared operator \( L^2 \in \mathcal{C}_1 \) is positive and pseudodifferential, with smooth, rapidly decaying kernel.

Standard semigroup theory for positive elliptic operators implies that the heat kernel \( K_t(x,y) \) is jointly smooth in both variables:
\[
K_t(x,y) \in C^\infty(\mathbb{R}^2), \qquad \text{for all } t > 0.
\]

\paragraph{Step 2: Diagonal Parametrix Expansion.}
Classical parametrix constructions for elliptic self-adjoint operators (e.g., Seeley–Gilkey, Reed–Simon~\cite{ReedSimon1978IV}) yield the short-time expansion of the heat kernel along the diagonal:
\[
K_t(x,x) \sim \sum_{n=0}^\infty a_n(x) \, t^{n - \frac{1}{2}}, \qquad \text{as } t \to 0^+,
\]
with coefficients \( a_n(x) \in C^\infty(\mathbb{R}) \), explicitly computable from the local symbol of \( L^2 \). The expansion is valid pointwise and locally uniformly, and inherits exponential decay from the smooth kernel structure.

\paragraph{Step 3: Uniform Bounds on Compacts.}
Fix any \( N \in \mathbb{N} \) and compact set \( K \subset \mathbb{R} \). Since all terms in the expansion are smooth, the Taylor remainder is uniformly controlled:
\[
\left| K_t(x,x) - \sum_{n=0}^{N-1} a_n(x)\, t^{n - \frac{1}{2}} \right| \le C_N\, t^{N - \frac{1}{2}}, \quad \text{for } x \in K, \; t \in (0, t_0],
\]
for some constants \( C_N > 0 \), \( t_0 > 0 \), by standard estimates for semigroup remainders.

\paragraph{Conclusion.}
We conclude that \( K_t(x,x) \) admits a full short-time asymptotic expansion uniformly over compact subsets of \( \mathbb{R} \), with each coefficient \( a_n(x) \in C^\infty(\mathbb{R}) \). This confirms the claimed uniform diagonal expansion.
\end{proof}


% --- Positivity of the heat kernel diagonal ---
\begin{lemma}[Positivity of the Heat Kernel Diagonal]
\label{lem:heat-kernel-diagonal-positivity}
Let \( L_{\mathrm{sym}} \in \mathcal{C}_1(H_{\Psi_\alpha}) \) be the canonical compact, self-adjoint operator on the exponentially weighted Hilbert space \( H_{\Psi_\alpha} := L^2(\mathbb{R}, e^{\alpha |x|} dx) \), defined via convolution against the inverse Fourier transform of the completed Riemann zeta profile
\[
\phi(\lambda) := \Xi\left(\tfrac{1}{2} + i\lambda\right).
\]
Let \( K_t(x, y) \) denote the integral kernel of the heat semigroup operator \( e^{-t L_{\mathrm{sym}}^2} \), for \( t > 0 \).

\medskip
\noindent
Then the diagonal values of the heat kernel are pointwise nonnegative:
\[
K_t(x,x) \geq 0, \qquad \text{for all } x \in \mathbb{R}, \; t > 0.
\]

\medskip
\noindent
This property follows from the spectral decomposition of the heat semigroup and the positivity of its eigenfunction coefficients, and reflects the fundamental positivity structure of self-adjoint heat kernels.
\end{lemma}

\begin{proof}[Proof of Lemma~\ref{lem:heat-kernel-diagonal-positivity}]
Since \( L_{\mathrm{sym}} \in \mathcal{C}_1(H_{\Psi_\alpha}) \) is compact, self-adjoint, and defined on the weighted Hilbert space \( H_{\Psi_\alpha} = L^2(\mathbb{R}, e^{\alpha|x|} dx) \), the spectral theorem yields an orthonormal basis \( \{e_n\}_{n \geq 1} \subset H_{\Psi_\alpha} \) of eigenfunctions with corresponding real eigenvalues \( \mu_n \in \mathbb{R} \) satisfying \( \mu_n \to 0 \). Then for all \( t > 0 \), the heat operator \( e^{-t L_{\mathrm{sym}}^2} \) is trace class and has the spectral expansion:
\[
K_t(x,y) = \sum_{n=1}^\infty e^{-t \mu_n^2} \, e_n(x) \, \overline{e_n(y)},
\]
with convergence in the Hilbert–Schmidt norm topology and pointwise absolutely for each fixed \( (x,y) \in \mathbb{R}^2 \), due to the trace-class property of \( e^{-t L_{\mathrm{sym}}^2} \).

Restricting to the diagonal \( x = y \), we obtain
\[
K_t(x,x) = \sum_{n=1}^\infty e^{-t \mu_n^2} \, |e_n(x)|^2.
\]
Each term in the sum is nonnegative, and since \( \sum_{n} e^{-t \mu_n^2} < \infty \), the convergence is absolute and locally uniform in \( x \in \mathbb{R} \). Therefore,
\[
K_t(x,x) \geq 0, \qquad \forall x \in \mathbb{R}, \; t > 0.
\]

This proves pointwise nonnegativity of the heat kernel along the diagonal.
\end{proof}


%------------------------------------------------------------------
\subsection{Spectral Trace Properties}

% --- Heat trace expansion and trace-class closure lemma ---
\begin{lemma}[Trace-Class Closure and Heat Trace Expansion]
\label{lem:heat_trace_expansion}
Let \( L_{\sym} \in \TC(\HPsi) \) denote the canonical compact, self-adjoint operator on the exponentially weighted Hilbert space
\[
\HPsi := L^2(\R, e^{\alpha|x|} \, dx), \qquad \alpha > \pi.
\]
Then the following properties hold:
\begin{enumerate}
    \item The squared operator \( L_{\sym}^2 \in \TC(\HPsi) \) is trace class. Consequently, the heat semigroup \( e^{-tL_{\sym}^2} \in \TC(\HPsi) \) is trace class for all \( t > 0 \), and the heat trace
    \[
    \Theta(t) := \Tr(e^{-tL_{\sym}^2})
    \]
    is finite, positive, and smooth in \( t > 0 \).

    \item As \( t \to 0^+ \), the trace admits a singular asymptotic expansion of the form
    \[
    \Theta(t) \sim \sum_{n=0}^{\infty} A_n t^{n - 1/2}, \qquad A_n := \int_{\R} a_n(x) \, dx,
    \]
    where the \( a_n(x) \) are local coefficient functions arising from the diagonal heat kernel expansion
    \[
    K_t(x,x) \sim \sum_{n=0}^{\infty} a_n(x) \, t^{n - 1/2}.
    \]
    This asymptotic places \( \Theta(t) \) in the regularly varying class \( R_{1/2} \), reflecting spectral dimension one and ensuring compatibility with Tauberian inversion.

    \item The leading singular term exhibits logarithmic divergence:
    \[
    \Theta(t) = \frac{1}{\sqrt{4\pi t}} \log\left( \frac{1}{t} \right)
    + \frac{c_0}{\sqrt{t}} + o(t^{-1/2}), \qquad \text{as } t \to 0^+,
    \]
    where
    \[
    c_0 := \int_{\R} a_1(x) \, dx.
    \]
    The logarithmic behavior renders the integral \( \int_0^\infty \Theta(t) \, dt \) divergent, necessitating analytic continuation via spectral zeta regularization. While the coefficient \( c_0 \in \R \) is formally defined, it does not influence the normalization of the spectral determinant and plays no role in the spectral trace identity or zeta-regularized log-derivative formula.
\end{enumerate}

This expansion confirms that \( \Theta(t) \) lies in the class of log-modulated regularly varying functions. In particular, the Laplace integral
\[
\int_0^\infty \frac{e^{-\lambda^2 t}}{t} \Theta(t) \, dt
\]
converges for all \( \lambda \in \C \), and defines the logarithmic derivative of the regularized Fredholm determinant.
\end{lemma}

\begin{proof}[Proof of Lemma~\ref{lem:heat-trace-expansion}]
\textbf{(i)} Since \( L_{\mathrm{sym}} \in \mathcal{C}_1(H_{\Psi_\alpha}) \subset \mathcal{C}_2 \), we may factor \( L_{\mathrm{sym}} = AB \) with \( A, B \in \mathcal{C}_2 \), the Hilbert–Schmidt class. Then the square satisfies
\[
L_{\mathrm{sym}}^2 = A (BA) B \in \mathcal{C}_1,
\]
because \( \mathcal{C}_2 \cdot \mathcal{C}_2 \subset \mathcal{C}_1 \), by standard Schatten ideal multiplication. Hence, the heat semigroup \( e^{-tL_{\mathrm{sym}}^2} \in \mathcal{C}_1 \) for all \( t > 0 \), and the heat trace
\[
\Theta(t) := \operatorname{Tr}(e^{-tL_{\mathrm{sym}}^2})
\]
is finite and smooth for all \( t > 0 \).

\textbf{(ii)} Let \( \{ \mu_n \}_{n \in \mathbb{N}} \subset \mathbb{R} \) denote the eigenvalues of \( L_{\mathrm{sym}} \), indexed with multiplicity. Since \( L_{\mathrm{sym}}^2 \in \mathcal{C}_1 \), the spectral trace formula yields
\[
\operatorname{Tr}(e^{-tL_{\mathrm{sym}}^2}) = \sum_{n=1}^{\infty} e^{-t \mu_n^2} < \infty.
\]
The spectral theorem ensures that \( e^{-tL_{\mathrm{sym}}^2} \) is a strongly continuous trace-class semigroup.

From Lemma~1.13 and Lemma~2.10, the integral kernel \( K_t(x,y) \) of \( e^{-tL_{\mathrm{sym}}^2} \) is symmetric, smooth, and exponentially decaying. Paley–Wiener theory and the standard parametrix expansion (cf.~\cite[Ch.~III]{Korevaar2004Tauberian}) yield the pointwise diagonal expansion:
\[
K_t(x,x) \sim \sum_{n=0}^{\infty} a_n(x) \, t^{n - 1/2}, \qquad t \to 0^+.
\]
Since the kernel is trace-class, integrating term-by-term gives the global expansion:
\[
\Theta(t) = \operatorname{Tr}(e^{-tL_{\mathrm{sym}}^2}) = \int_{\mathbb{R}} K_t(x,x) \, dx \sim \sum_{n=0}^{\infty} A_n \, t^{n - 1/2}, \qquad A_n := \int_{\mathbb{R}} a_n(x) \, dx.
\]

\textbf{(iii)} The leading-order singularity is dictated by the spectral profile of \( \Xi(s) \), whose Hadamard product involves genus-one exponential type \( \pi \) and a Gaussian damping prefactor. Via Laplace–Mellin Tauberian asymptotics (see Lemma~7.2), this yields the dominant log-corrected term:
\[
\Theta(t) = \frac{1}{\sqrt{4\pi t}} \log\left( \frac{1}{t} \right) + \frac{c_0}{\sqrt{t}} + o(t^{-1/2}), \qquad \text{as } t \to 0^+,
\]
with
\[
c_0 := \int_{\mathbb{R}} a_1(x) \, dx.
\]
This logarithmic divergence near \( t = 0 \) is not Lebesgue integrable, and necessitates analytic regularization of the associated Fredholm determinant via the zeta prescription detailed in Chapter~3. The coefficient \( c_0 \in \mathbb{R} \), though formally computable, is analytically neutral: it plays no role in determinant normalization or spectral bijection.
\end{proof}


% --- Laplace integrability for determinant definition ---
\begin{lemma}[Laplace Integrability of the Heat Trace]
\label{lem:laplace_integrability_heat_trace}
Let \( L_{\mathrm{sym}} \in \mathcal{C}_1(H_{\Psi_\alpha}) \) be the canonical self-adjoint trace-class operator on the exponentially weighted Hilbert space \( H_{\Psi_\alpha} := L^2(\mathbb{R}, e^{\alpha|x|} dx) \), with \( \alpha > \pi \). Define the heat trace
\[
f(t) := \operatorname{Tr}\left(e^{-t L_{\mathrm{sym}}^2}\right), \qquad t > 0.
\]
Then the following hold:
\begin{enumerate}
    \item For every \( \lambda \in \mathbb{C} \), the Laplace-type integral
    \[
    \int_0^\infty \frac{e^{-\lambda^2 t}}{t} f(t)\, dt
    \]
    converges absolutely, and defines an entire function of \( \lambda \in \mathbb{C} \).
    
    \item The associated zeta-regularized Fredholm determinant
    \[
    \det\nolimits_{\zeta}(I - \lambda L_{\mathrm{sym}}) := \exp\left( -\int_0^\infty \frac{e^{-\lambda^2 t}}{t} \operatorname{Tr}\left(e^{-t L_{\mathrm{sym}}^2}\right) \, dt \right)
    \]
    is entire, and satisfies
    \[
    \log \det\nolimits_{\zeta}(I - \lambda L_{\mathrm{sym}}) = - \int_0^\infty \frac{e^{-\lambda^2 t}}{t} \Theta(t)\, dt.
    \]
\end{enumerate}
\noindent
The integral is well-defined for all \( \lambda \in \mathbb{C} \) due to the singular short-time behavior
\[
\Theta(t) = \frac{\log(1/t)}{\sqrt{4\pi t}} + \mathcal{O}(t^{-1/2}) \quad \text{as } t \to 0^+,
\]
which ensures integrability of the weighted integrand \( e^{-\lambda^2 t} \Theta(t)/t \) near \( t = 0 \). This Laplace integrability confirms that the determinant representation via heat trace regularization is valid on the entire complex plane and underpins the Hadamard factorization developed in Chapter~3.
\end{lemma}

\begin{proof}[Proof of Lemma~\ref{lem:laplace_integrability_heat_trace}]
Let \( f(t) := \operatorname{Tr}\left(e^{-t L_{\mathrm{sym}}^2}\right) \). By Lemma~\ref{lem:heat_trace_expansion}, \( f(t) \) is positive, smooth for \( t > 0 \), and admits the singular short-time expansion
\[
f(t) = \frac{1}{\sqrt{4\pi t}} \log\left( \frac{1}{t} \right) + \frac{c_0}{\sqrt{t}} + o(t^{-1/2}) \quad \text{as } t \to 0^+,
\]
while Lemma~\ref{lem:hk_upper_bound} provides the exponential decay
\[
f(t) \leq C e^{-c t} \quad \text{as } t \to \infty,
\]
for some constants \( c, C > 0 \).

\paragraph{Step 1: Convergence near \( t = 0 \).}
We estimate for small \( t \in (0, t_0] \), using the asymptotic structure of \( f(t) \),
\[
\left| \frac{e^{-\lambda^2 t}}{t} f(t) \right|
\leq \frac{1}{t} \left( \frac{1}{\sqrt{t}} \log\left( \frac{1}{t} \right) + \frac{C_1}{\sqrt{t}} \right)
= \frac{1}{t^{3/2}} \left( \log\left( \frac{1}{t} \right) + C_1 \right),
\]
which is integrable on \( (0, t_0] \) since the function \( t^{-3/2} \log(1/t) \in L^1(0, t_0) \) by standard singular integral estimates.

\paragraph{Step 2: Convergence near \( t = \infty \).}
For \( t \geq t_0 \), we use the exponential decay of \( f(t) \) and boundedness of \( \lambda \mapsto e^{-\lambda^2 t} \) for fixed \( \lambda \in \mathbb{C} \):
\[
\left| \frac{e^{-\lambda^2 t}}{t} f(t) \right| \leq \frac{C e^{-\operatorname{Re}(\lambda^2) t}}{t} \in L^1(t_0, \infty),
\]
since exponential decay dominates the \( t^{-1} \) denominator for all \( \lambda \in \mathbb{C} \).

\paragraph{Step 3: Entirety in \( \lambda \).}
For each fixed \( t > 0 \), the map \( \lambda \mapsto e^{-\lambda^2 t} \) is entire, and \( f(t) \) is independent of \( \lambda \). Hence the Laplace-type integral
\[
F(\lambda) := \int_0^\infty \frac{e^{-\lambda^2 t}}{t} f(t)\, dt
\]
is an entire function by dominated convergence on compact subsets of \( \lambda \in \mathbb{C} \), with majorant derived from Steps 1 and 2.

\paragraph{Conclusion.}
The integral
\[
\int_0^\infty \frac{e^{-\lambda^2 t}}{t} \operatorname{Tr}(e^{-t L_{\mathrm{sym}}^2}) \, dt
\]
converges absolutely and defines an entire function of \( \lambda \in \mathbb{C} \), validating the Fredholm determinant identity
\[
\det\nolimits_\zeta(I - \lambda L_{\mathrm{sym}}) = \exp\left( -\int_0^\infty \frac{e^{-\lambda^2 t}}{t} \operatorname{Tr}(e^{-t L_{\mathrm{sym}}^2})\, dt \right),
\]
as an entire function. This confirms the global analytic well-posedness of the determinant via heat trace regularization.
\end{proof}


% --- Two-sided estimate for Tr(e^{-tL^2_sym}) ~ t^{-1/2} ---
\begin{proposition}[Two-Sided Heat Trace Bounds]
\label{prop:two_sided_heat_trace_bounds}
Let \( \Lsym \in \TC(\HPsi) \) be the canonical compact, self-adjoint operator on the exponentially weighted Hilbert space \( \HPsi := L^2(\R, e^{\alpha|x|} dx) \) for \( \alpha > \pi \), with discrete real spectrum \( \{\mu_n\} \subset \R \).

Then there exist constants \( c_1, c_2 > 0 \) and \( t_0 > 0 \) such that for all \( t \in (0, t_0] \),
\[
c_1\, t^{-1/2}
\;\le\;
\Tr\left( e^{-t \Lsym^2} \right)
\;\le\;
c_2\, t^{-1/2}.
\]
Explicitly, one may take
\[
c_1 := \frac{1}{4 \, \| \Lsym \|_{\TC}}, \qquad
c_2 := (1 + e^{-1}) \cdot \| \Lsym \|_{\TC},
\]
as established in \lemref{lem:hk_lower_bound} and \lemref{lem:hk_upper_bound}, respectively.

\medskip

\noindent
This two-sided estimate confirms that the heat trace asymptotics obey the scaling law \( \Theta(t) \sim t^{-1/2} \) as \( t \to 0^+ \), consistent with spectral dimension one. The bound holds uniformly for small time \( t \in (0, t_0] \), independently of the spectral multiplicity structure, and confirms that \( \Theta(t) \in R_{1/2} \) in the Tauberian class of regularly varying functions.
\end{proposition}

\begin{proof}[Proof of \propref{prop:two_sided_heat_trace_bounds}]
We apply \lemref{lem:hk_upper_bound} and \lemref{lem:hk_lower_bound} to obtain explicit bounds on the heat trace.

\paragraph{Upper Bound.}
\lemref{lem:hk_upper_bound} guarantees that for all \( t \in (0,1] \),
\[
\Tr(e^{-t L_{\sym}^2}) \leq c_2 \, t^{-1/2}, \qquad
c_2 := (1 + e^{-1}) \cdot \| L_{\sym} \|_{\TC}.
\]

\paragraph{Lower Bound.}
\lemref{lem:hk_lower_bound} establishes that there exists \( t_1 > 0 \) and a constant
\[
c_1 := \frac{1}{4 \, \| L_{\sym} \|_{\TC}} > 0
\]
such that for all \( t \in (0, t_1] \),
\[
\Tr(e^{-t L_{\sym}^2}) \geq c_1 \, t^{-1/2}.
\]

\paragraph{Conclusion.}
Let \( t_0 := \min\{1, t_1\} \). Then for all \( t \in (0, t_0] \), the two-sided bound holds:
\[
c_1 \, t^{-1/2} \;\leq\; \Tr(e^{-t L_{\sym}^2}) \;\leq\; c_2 \, t^{-1/2}.
\]
This completes the proof. The estimate confirms that the heat trace exhibits a sharp \( t^{-1/2} \) scaling in the short-time regime, consistent with local parametrix asymptotics and spectral dimension one. The constants depend only on the trace norm of \( L_{\sym} \), and the bounds are uniform across all compact time intervals \( (0, t_0] \).
\end{proof}


% --- Global convergence of Tr(e^{-tL^2_sym}) ~ sum A_n t^{n-1/2} ---
\begin{proposition}[Uniform Convergence of Heat Trace Expansion]
\label{prop:heat_trace_uniform_conv}
Let \( L_{\mathrm{sym}} \in \mathcal{C}_1(H_{\Psi_\alpha}) \) be the canonical compact, self-adjoint, and nonnegative operator on the exponentially weighted Hilbert space \( H_{\Psi_\alpha} := L^2(\mathbb{R}, e^{\alpha|x|} dx) \), with \( \alpha > \pi \). Let \( K_t(x,y) \) denote the integral kernel of the heat semigroup \( e^{-t L_{\mathrm{sym}}^2} \).

Then as \( t \to 0^+ \), the spectral heat trace admits a global short-time expansion:
\[
\operatorname{Tr}(e^{-t L_{\mathrm{sym}}^2}) = \int_{\mathbb{R}} K_t(x,x)\, dx
\sim \sum_{n=0}^\infty A_n\, t^{n - \frac{1}{2}},
\]
with coefficients
\[
A_n := \int_{\mathbb{R}} a_n(x)\, dx,
\]
where \( a_n(x) \in C^\infty(\mathbb{R}) \) are the diagonal heat kernel coefficients from the local expansion
\[
K_t(x,x) \sim \sum_{n=0}^\infty a_n(x)\, t^{n - \frac{1}{2}}, \qquad \text{as } t \to 0^+.
\]
Each \( A_n \) is finite due to the exponential decay of \( a_n(x) \) induced by the analytic smoothing of the convolution kernel.

\medskip
\noindent
Moreover, the expansion converges with uniform remainder bounds: for each \( N \in \mathbb{N} \), there exist constants \( C_N > 0 \) and \( t_0 > 0 \) such that for all \( t \in (0, t_0] \),
\[
\left| \operatorname{Tr}(e^{-t L_{\mathrm{sym}}^2}) - \sum_{n=0}^{N-1} A_n\, t^{n - \frac{1}{2}} \right| \leq C_N\, t^{N - \frac{1}{2}}.
\]

\medskip
\noindent
This asymptotic holds uniformly on compact time intervals \( (0, t_0] \), and follows from classical parametrix expansion theory, combined with dominated convergence. Since \( e^{-t L_{\mathrm{sym}}^2} \in \mathcal{C}_1 \) and \( K_t(x,x) \in C^\infty(\mathbb{R}) \) with uniform exponential decay, the termwise integral converges for all \( n \), and the expansion defines the singular spectral trace structure used in the determinant and Tauberian growth results of Chapters~\ref{sec:determinant-identity} and~\ref{sec:tauberian-growth}.
\end{proposition}

\begin{proof}[Proof of Proposition~\ref{prop:heat_trace_uniform_conv}]
Let \( K_t(x,y) \) denote the integral kernel of the semigroup \( e^{-t L_{\mathrm{sym}}^2} \). Since \( L_{\mathrm{sym}} \in \mathcal{C}_1(H_{\Psi_\alpha}) \), its square is self-adjoint, nonnegative, and trace class. Consequently, \( K_t(x,y) \) is jointly smooth and exponentially decaying in both variables. The diagonal \( K_t(x,x) \) is smooth and rapidly decaying, and the trace satisfies
\[
\operatorname{Tr}(e^{-t L_{\mathrm{sym}}^2}) = \int_{\mathbb{R}} K_t(x,x) \, dx,
\]
by the spectral theorem and Fubini–Tonelli, since the integrand is positive and integrable for all \( t > 0 \).

\paragraph{Step 1: Local Asymptotics.}
Lemma~\ref{lem:hk_expansion_uniform} provides a local diagonal expansion of the form
\[
K_t(x,x) = \sum_{n=0}^{N-1} a_n(x) \, t^{n - \frac{1}{2}} + R_N(x,t),
\]
where each coefficient function \( a_n(x) \in C^\infty(\mathbb{R}) \) decays faster than any exponential, and the remainder satisfies
\[
|R_N(x,t)| \leq C_N \, t^{N - \frac{1}{2}}, \qquad \forall x \in \mathbb{R}, \; 0 < t \leq t_0.
\]

\paragraph{Step 2: Global Integrability and Termwise Integration.}
Because each \( a_n(x) \) lies in the Schwartz class, the coefficients
\[
A_n := \int_{\mathbb{R}} a_n(x) \, dx
\]
are finite for all \( n \). Furthermore, the remainder satisfies
\[
\left| \int_{\mathbb{R}} R_N(x,t)\, dx \right| \leq \int_{\mathbb{R}} |R_N(x,t)| \, dx \leq C_N' \, t^{N - \frac{1}{2}},
\]
for a suitable constant \( C_N' > 0 \), uniformly in \( t \in (0, t_0] \). This validates termwise integration of the expansion.

\paragraph{Step 3: Assembling the Trace Expansion.}
We conclude:
\[
\operatorname{Tr}(e^{-t L_{\mathrm{sym}}^2}) = \int_{\mathbb{R}} K_t(x,x) \, dx = \sum_{n=0}^{N-1} A_n \, t^{n - \frac{1}{2}} + R_N(t),
\]
with \( |R_N(t)| \leq C_N' t^{N - \frac{1}{2}} \) as shown above.

\paragraph{Conclusion.}
The global heat trace admits a full asymptotic expansion in half-integer powers of \( t \), with coefficients
\[
A_n = \int_{\mathbb{R}} a_n(x) \, dx,
\]
and remainder estimates uniform on \( (0, t_0] \). This completes the proof.
\end{proof}


% --- Refined log-term asymptotic: Tr(e^{-tL^2_sym}) = (log(1/t))/sqrt(4pi t) + ... ---
\begin{proposition}[Refined Short-Time Heat Trace Expansion]
\label{prop:short_time_heat_expansion}
Let \( L_{\sym} \in \TC(\HPsi) \) be the canonical compact, self-adjoint operator on the exponentially weighted Hilbert space \( \HPsi := L^2(\R, e^{\alpha |x|} \, dx) \) for some \( \alpha > \pi \). Then, as \( t \to 0^+ \), the spectral heat trace satisfies the refined singular expansion:
\[
\Tr\left(e^{-t L_{\sym}^2}\right)
= \frac{1}{\sqrt{4\pi t}} \log\left( \frac{1}{t} \right)
+ c_0 \sqrt{t} + o\left( \sqrt{t} \right),
\]
for some constant \( c_0 \in \R \), where the remainder \( o(\sqrt{t}) \) vanishes uniformly as \( t \to 0^+ \) over compact subintervals of \( (0, t_0] \).

\medskip
\noindent
The leading-order singularity \( \frac{1}{\sqrt{4\pi t}} \log(1/t) \) reflects the logarithmic divergence induced by the spectral structure of the mollified convolution kernel defining \( L_{\sym} \). This divergence originates from the genus-one Hadamard structure of the completed zeta function and implies non-integrability of the trace near \( t = 0 \).

\medskip
\noindent
The correction coefficient \( c_0 \) arises from the first regular term in the local heat kernel expansion and satisfies
\[
c_0 := \int_{\R} a_1(x) \, dx,
\]
where the diagonal expansion takes the form
\[
K_t(x,x) \sim \sum_{n=0}^{\infty} a_n(x)\, t^{n - \frac{1}{2}}.
\]

\medskip
\noindent
This refined asymptotic plays a foundational role in the determinant expansion and Tauberian theory developed in \secref{sec:determinant_identity} and \secref{sec:tauberian_growth}. In particular, the logarithmic divergence necessitates analytic continuation in the Laplace transform and underpins the regularized determinant identity:
\[
\log \det\nolimits_{\zeta}(I - \lambda L_{\sym}) = - \int_0^\infty \frac{e^{-\lambda^2 t}}{t} \Tr\left(e^{-t L_{\sym}^2} \right) \, dt,
\]
where the integral must be interpreted in the zeta-regularized sense.
\end{proposition}

\begin{proof}[Proof of Proposition~\ref{prop:short_time_heat_expansion}]
Let \( L := L_{\mathrm{sym}} \in \mathcal{C}_1(H_{\Psi_\alpha}) \) denote the canonical compact, self-adjoint operator. Let \( L_\varepsilon \to L \) in trace norm be a family of mollified convolution approximants constructed via Gaussian regularization in the Fourier domain, as defined in Chapter~\ref{sec:operator-construction}.

\paragraph{Step 1: Asymptotics for Mollified Approximants.}
Each \( L_\varepsilon \) is smoothing, self-adjoint, and trace class. Its square \( L_\varepsilon^2 \) defines a bounded pseudodifferential operator with integral kernel \( K^{(\varepsilon)}_t(x,y) \in \mathcal{S}(\mathbb{R}^2) \). The diagonal expansion
\[
K^{(\varepsilon)}_t(x,x) \sim \sum_{n=0}^\infty a_n^{(\varepsilon)}(x)\, t^{n - \frac{1}{2}}
\]
holds uniformly in \( x \in \mathbb{R} \), with each \( a_n^{(\varepsilon)}(x) \in \mathcal{S}(\mathbb{R}) \). Integration yields the trace expansion
\[
\operatorname{Tr}(e^{-t L_\varepsilon^2}) \sim \sum_{n=0}^\infty A_n^{(\varepsilon)}\, t^{n - \frac{1}{2}}, \qquad A_n^{(\varepsilon)} := \int_{\mathbb{R}} a_n^{(\varepsilon)}(x)\, dx.
\]

In particular, the logarithmic term
\[
\frac{1}{\sqrt{4\pi t}} \log\left( \frac{1}{t} \right)
\]
emerges universally as the leading singularity, reflecting the exponential type and genus-one Hadamard structure of the spectral profile \( \Xi(s) \), as encoded in the mollified kernels.

\paragraph{Step 2: Trace-Class Convergence of the Semigroup.}
By stability of trace-class semigroups under strong convergence in \( \mathcal{C}_1 \) (see~\cite[Thm.~3.2]{Simon2005TraceIdeals}), we have:
\[
L_\varepsilon^2 \to L^2 \quad \text{in } \mathcal{C}_1(H_{\Psi_\alpha}) \quad \Longrightarrow \quad e^{-t L_\varepsilon^2} \to e^{-t L^2} \text{ in } \mathcal{C}_1.
\]
Thus, for all \( t \in (0, t_0] \),
\[
\operatorname{Tr}(e^{-t L_\varepsilon^2}) \to \operatorname{Tr}(e^{-t L^2}),
\]
and the asymptotic expansion transfers to the limit via dominated convergence.

\paragraph{Step 3: Conclusion.}
Passing to the limit, we conclude that
\[
\operatorname{Tr}(e^{-t L^2}) = \frac{1}{\sqrt{4\pi t}} \log\left( \frac{1}{t} \right) + c_0 \sqrt{t} + o(\sqrt{t}) \quad \text{as } t \to 0^+,
\]
where
\[
c_0 := \int_{\mathbb{R}} a_1(x) \, dx.
\]
This completes the proof.
\end{proof}


% --- Distributional asymptotics of Tr(e^{-tL^2_sym}) expansion ---
\begin{lemma}[Distributional Heat Trace Asymptotics]
\label{lem:distributional_trace_asymptotics}
Let
\[
L_{\mathrm{sym}} \in \mathcal{C}_1(H_{\Psi_\alpha})
\]
be the canonical compact, self-adjoint operator on the exponentially weighted Hilbert space \( H_{\Psi_\alpha} := L^2(\mathbb{R}, e^{\alpha|x|} dx) \), with \( \alpha > \pi \). Define the spectral heat trace
\[
f(t) := \operatorname{Tr}(e^{-t L_{\mathrm{sym}}^2}), \qquad t > 0.
\]

Then as \( t \to 0^+ \), the expansion
\[
f(t) \sim \sum_{n=0}^\infty A_n \, t^{n - \frac{1}{2}}
\]
holds in the sense of distributions on \( \mathbb{R}_{>0} \). Specifically, for any test function \( \psi \in C_c^\infty((0, \infty)) \),
\[
\lim_{\varepsilon \to 0^+} \int_0^\infty f(t) \, \psi(t/\varepsilon) \, dt
= \sum_{n=0}^\infty A_n \int_0^\infty t^{n - \frac{1}{2}} \, \psi(t)\, dt.
\]

\medskip
\noindent
This distributional formulation captures the asymptotic structure of \( f(t) \) in a Tauberian scaling window, and governs both the spectral growth rates and the singular structure of the regularized determinant. In particular, the expansion remains valid in the space of tempered distributions \( \mathcal{D}'(\mathbb{R}_{>0}) \), and allows exact Laplace analysis of the trace integral appearing in
\[
\log \det\nolimits_{\zeta}(I - \lambda L_{\mathrm{sym}}) = - \int_0^\infty \frac{e^{-\lambda^2 t}}{t} f(t) \, dt.
\]

\medskip
\noindent
The asymptotic equivalence in \( \mathcal{D}'(\mathbb{R}_{>0}) \) is a classical result of Laplace–Mellin theory; see Korevaar~\cite[Ch.~IV]{Korevaar2004Tauberian} and Hörmander~\cite[Vol.~I, §7.1]{Hormander1983Analysis} for distributional expansions of regularly varying functions and their Laplace transforms.
\end{lemma}

\begin{proof}[Proof of Lemma~\ref{lem:distributional_trace_asymptotics}]
Let \( f(t) := \operatorname{Tr}(e^{-t L_{\mathrm{sym}}^2}) \). From Lemma~\ref{lem:heat_trace_expansion}, we have the full short-time expansion
\[
f(t) = \sum_{n=0}^N A_n \, t^{n - \frac{1}{2}} + \mathcal{O}(t^{N + \frac{1}{2}}), \quad \text{as } t \to 0^+,
\]
with convergence uniform on compact subintervals of \( (0, t_0] \) and coefficients \( A_n \in \mathbb{R} \) derived from the diagonal parametrix expansion of the heat kernel.

\paragraph{Step 1: Dilation of Test Function.}
Let \( \psi \in C_c^\infty((0, \infty)) \) be a test function. Define the rescaled family
\[
\psi_\varepsilon(t) := \psi(t/\varepsilon),
\]
so that \( \psi_\varepsilon \to 0 \) weakly as \( \varepsilon \to 0^+ \), and the scaling maps the region near \( t = 0 \) into the support of \( \psi \).

We compute:
\[
\int_0^\infty f(t) \psi_\varepsilon(t)\, dt
= \int_0^\infty f(t) \psi(t/\varepsilon)\, dt
= \varepsilon \int_0^\infty f(\varepsilon t) \psi(t)\, dt,
\]
via the substitution \( t \mapsto \varepsilon t \).

\paragraph{Step 2: Asymptotic Substitution.}
In the inner integral, apply the expansion:
\[
f(\varepsilon t) = \sum_{n=0}^N A_n (\varepsilon t)^{n - \frac{1}{2}} + \mathcal{O}(\varepsilon^{N + \frac{1}{2}}),
\]
uniformly in \( t \in \operatorname{supp}(\psi) \subset (0, \infty) \), yielding
\[
\varepsilon \int_0^\infty f(\varepsilon t) \psi(t)\, dt
= \sum_{n=0}^N A_n \varepsilon^{n + \frac{1}{2}} \int_0^\infty t^{n - \frac{1}{2}} \psi(t)\, dt + \mathcal{O}(\varepsilon^{N + \frac{3}{2}}).
\]

\paragraph{Step 3: Distributional Limit.}
Taking the limit \( \varepsilon \to 0^+ \), we conclude:
\[
\lim_{\varepsilon \to 0^+} \int_0^\infty f(t) \psi(t/\varepsilon)\, dt
= \sum_{n=0}^\infty A_n \int_0^\infty t^{n - \frac{1}{2}} \psi(t)\, dt.
\]
This is precisely the definition of an asymptotic expansion in the distributional sense on \( \mathbb{R}_{>0} \):
\[
f(t) \sim \sum_{n=0}^\infty A_n t^{n - \frac{1}{2}} \quad \text{in } \mathcal{D}'(\mathbb{R}_{>0}),
\]
as claimed.
\end{proof}


% --- Determinant-log derivative connection via heat trace regularization ---
\begin{lemma}[Logarithmic Derivative of the Spectral Determinant]
\label{lem:log_derivative_determinant}
Let \( L_{\sym} \in \TC(\HPsi) \) be the canonical compact, self-adjoint operator with discrete nonzero spectrum \( \{ \mu_n \} \subset \R \setminus \{0\} \). Define the associated spectral zeta function by
\[
\zeta_L(s) := \sum_{\mu_n \neq 0} \mu_n^{-2s}, \qquad \Re(s) > \tfrac{1}{2}.
\]

Then the logarithm of the zeta-regularized determinant of \( L_{\sym}^2 \) satisfies
\[
\log \det\nolimits_{\zeta}(L_{\sym}^2)
= -\left. \frac{d}{ds} \zeta_L(s) \right|_{s=0}
= -\int_0^\infty \frac{\Tr(e^{-t L_{\sym}^2}) - P(t)}{t} \, dt,
\]
where \( P(t) \) denotes the full singular part of the short-time asymptotic expansion:
\[
P(t) \sim \sum_{n=0}^\infty A_n\, t^{n - \frac{1}{2}} = \frac{1}{\sqrt{4\pi t}} \log\left( \frac{1}{t} \right) + c_0 \sqrt{t} + \cdots.
\]

\medskip
\noindent
This identity is justified by the Laplace--Mellin representation of the spectral zeta function,
\[
\zeta_L(s) = \frac{1}{\Gamma(s)} \int_0^\infty t^{s-1} \Tr(e^{-t L_{\sym}^2}) \, dt,
\]
combined with analytic subtraction of \( P(t) \) near \( t = 0 \) to ensure convergence at \( s = 0 \). The regularized logarithmic derivative thus computes
\[
-\zeta_L'(0) = \log \det\nolimits_\zeta(L_{\sym}^2).
\]

\medskip
\noindent
In particular, the coefficient of the logarithmic term \( \frac{1}{\sqrt{4\pi t}} \log(1/t) \) governs the leading singularity of the determinant and encodes the spectral dimension and singularity class of \( L_{\sym} \). This structure underpins both the analytic continuation of \( \zeta_L(s) \) and the small-\( \lambda \) expansion of the resolvent determinant:
\[
\log \det\nolimits_{\zeta}(I + \lambda L_{\sym}) = c_0 \lambda + \mathcal{O}(\lambda^3),
\]
consistent with entire order-one growth.
\end{lemma}

\begin{proof}[Proof of Lemma~\ref{lem:log-derivative-determinant}]
Let \( L := L_{\mathrm{sym}} \in \mathcal{C}_1(H_{\Psi_\alpha}) \) be the canonical compact, self-adjoint operator. The spectral zeta function is defined for \( \Re(s) > \tfrac{1}{2} \) by
\[
\zeta_L(s) := \sum_{\mu_n \neq 0} \mu_n^{-2s},
\]
where \( \{ \mu_n \} \subset \mathbb{R} \setminus \{0\} \) are the eigenvalues of \( L \). By standard spectral theory and trace-class integrability, \( \zeta_L(s) \) admits a meromorphic continuation to \( \mathbb{C} \) with a regular point at \( s = 0 \) (cf.~\cite[Ch.~3]{Simon2005TraceIdeals}).

\paragraph{Step 1: Zeta-Regularized Determinant.}
The logarithm of the zeta-regularized determinant is defined by
\[
\log \det\nolimits_\zeta(L^2) := -\left. \frac{d}{ds} \zeta_L(s) \right|_{s=0},
\]
provided \( \zeta_L(s) \) is analytic at \( s = 0 \), which is ensured by the short-time expansion of the heat trace and the spectral convergence properties of \( L \).

\paragraph{Step 2: Mellin Representation and Heat Trace Regularization.}
The spectral zeta function admits the integral representation:
\[
\zeta_L(s) = \frac{1}{\Gamma(s)} \int_0^\infty t^{s - 1} \operatorname{Tr}(e^{-t L^2}) \, dt.
\]
To extract \( \zeta_L'(0) \), one subtracts a parametrix \( P(t) \) matching the singular short-time expansion:
\[
P(t) \sim \sum_{n=0}^{N} A_n t^{n - \frac{1}{2}}, \qquad t \to 0^+.
\]
Differentiation and evaluation at \( s = 0 \) yields:
\[
\log \det\nolimits_\zeta(L^2)
= -\int_0^\infty \frac{\operatorname{Tr}(e^{-t L^2}) - P(t)}{t} \, dt,
\]
justified by termwise convergence of the subtraction, as established in Appendix~\ref{app:heat-kernel-construction}.

\paragraph{Step 3: Leading Singular Contribution.}
By Proposition~\ref{prop:short_time_heat_expansion}, the heat trace satisfies:
\[
\operatorname{Tr}(e^{-t L^2}) \sim \frac{1}{\sqrt{4\pi t}} \log\left( \frac{1}{t} \right) + c_0 \sqrt{t} + \cdots \quad \text{as } t \to 0^+.
\]
The leading term \( \frac{1}{\sqrt{4\pi t}} \log(1/t) \) is non-integrable and must be cancelled by \( P(t) \). Its coefficient thus governs the divergence of the logarithmic derivative, and encodes the singularity structure of the determinant.

\paragraph{Conclusion.}
The zeta-regularized determinant satisfies
\[
\log \det\nolimits_\zeta(L^2) = -\int_0^\infty \frac{\operatorname{Tr}(e^{-t L^2}) - P(t)}{t} \, dt,
\]
with \( P(t) \sim \frac{1}{\sqrt{4\pi t}} \log(1/t) + \cdots \), ensuring convergence and capturing the dominant analytic behavior. This completes the proof.
\end{proof}


% --- Spectral Weyl asymptotics from heat kernel expansion ---
\begin{proposition}[Spectral Counting Function]
\label{prop:spectral_counting_weyl}
Let \( \{ \mu_n \} \subset \R \setminus \{0\} \) denote the nonzero eigenvalues of the canonical compact, self-adjoint operator \( L_{\sym} \in \TC(\HPsi) \), ordered by increasing absolute value and counted with multiplicity. Define the spectral counting function
\[
N(\lambda) := \#\{ n \in \N : \mu_n^2 \leq \lambda \}, \qquad \lambda > 0.
\]
Then, as \( \lambda \to \infty \), the function \( N(\lambda) \) satisfies the asymptotic growth law
\[
N(\lambda) \sim C\, \lambda^{1/2} \log \lambda,
\]
for some constant \( C > 0 \) determined by the leading singularity in the short-time heat trace expansion.

\medskip
\noindent
This result follows from the singular expansion
\[
\Tr(e^{-t L_{\sym}^2}) \sim \frac{1}{\sqrt{4\pi t}} \log\left( \frac{1}{t} \right) + \cdots \quad \text{as } t \to 0^+,
\]
via a Tauberian inversion argument. In particular, the spectral counting law exhibits sub-Weyl growth with a logarithmic enhancement, reflecting the non-classical scaling of the canonical convolution operator \( L_{\sym} \) on the weighted space \( \HPsi \).
\end{proposition}

\begin{proof}[Proof of \propref{prop:spectral_counting_weyl}]
From the refined short-time asymptotic of the spectral heat trace (see \propref{prop:short_time_heat_expansion}), we have:
\[
\Theta(t) := \Tr(e^{-t L_{\sym}^2}) \sim \frac{1}{\sqrt{4\pi t}} \log\left( \frac{1}{t} \right), \qquad \text{as } t \to 0^+.
\]

Let \( \{ \mu_n^2 \} \subset (0, \infty) \) denote the nonzero eigenvalues of \( L_{\sym}^2 \), counted with multiplicity, and define the squared spectral counting function:
\[
N(\lambda) := \#\left\{ n \in \N : \mu_n^2 \le \lambda \right\}.
\]

By Korevaar’s log-corrected Tauberian theorem~\cite[Ch.~III, §5]{Korevaar2004Tauberian}, as applied rigorously in \lemref{lem:log_corrected_tauberian_estimate}, the asymptotic behavior of \( \Theta(t) \) implies:
\[
N(\lambda) = \frac{\sqrt{\lambda}}{\pi} \log \lambda + O(\sqrt{\lambda}), \qquad \text{as } \lambda \to \infty.
\]

The logarithmic enhancement reflects the spectral density imposed by the leading singularity in the heat trace and modifies the classical Weyl law for effective spectral dimension \( d = 1 \).

\paragraph{Conclusion.}
The singular trace asymptotics invert via Laplace–Stieltjes theory to yield the log-modulated Weyl-type growth of the eigenvalue counting function:
\[
N(\lambda) \in \mathcal{R}_{1/2}^{\log}(\infty),
\]
establishing the spectral growth profile consistent with the canonical zeta determinant and the Riemann–von Mangoldt formula.
\end{proof}


% --- Strong operator closure of heat semigroup ---
\begin{proposition}[Strong Operator Closure of the Heat Semigroup]
\label{prop:strong_op_closure_heat}
Let \( L_{\mathrm{sym}} \in \mathcal{C}_1(H_{\Psi_\alpha}) \) be the canonical compact, self-adjoint operator on the exponentially weighted Hilbert space \( H_{\Psi_\alpha} := L^2(\mathbb{R}, e^{\alpha |x|} dx) \) for fixed \( \alpha > \pi \).

Then the associated heat semigroup \( \{ e^{-t L_{\mathrm{sym}}^2} \}_{t > 0} \) satisfies:
\[
\lim_{t \to 0^+} e^{-t L_{\mathrm{sym}}^2} f = f \qquad \text{for all } f \in H_{\Psi_\alpha},
\]
with convergence in norm. That is, the semigroup converges strongly to the identity operator as \( t \to 0^+ \).

\medskip
\noindent
Moreover, the semigroup satisfies the following properties:
\begin{itemize}
    \item Each operator \( e^{-t L_{\mathrm{sym}}^2} \in \mathcal{B}(H_{\Psi_\alpha}) \cap \mathcal{C}_1(H_{\Psi_\alpha}) \) is bounded and trace class for all \( t > 0 \);
    \item The semigroup is uniformly bounded in operator norm: \( \| e^{-t L_{\mathrm{sym}}^2} \|_{\mathcal{B}} \le 1 \);
    \item The family \( \{ e^{-t L_{\mathrm{sym}}^2} \}_{t > 0} \) is equicontinuous on norm-bounded subsets of \( H_{\Psi_\alpha} \).
\end{itemize}

\medskip
\noindent
This strong operator convergence confirms the analytic semigroup structure generated by \( L_{\mathrm{sym}}^2 \), and underpins both the trace expansion and determinant regularization developed in Chapters~\ref{sec:heat-kernel-asymptotics} and~\ref{sec:determinant-identity}.
\end{proposition}

\begin{proof}[Proof of Proposition~\ref{prop:strong_op_closure_heat}]
Let \( L := L_{\mathrm{sym}} \in \mathcal{C}_1(H_{\Psi_\alpha}) \) be the canonical compact, self-adjoint, nonnegative operator. By the spectral theorem, the heat semigroup \( \{ e^{-tL^2} \}_{t > 0} \) is defined via spectral functional calculus:
\[
e^{-tL^2} f = \sum_{n=1}^\infty e^{-t \mu_n^2} \langle f, e_n \rangle e_n,
\]
where \( \{e_n\} \subset H_{\Psi_\alpha} \) is an orthonormal basis of eigenfunctions with \( L e_n = \mu_n e_n \), and \( \mu_n \to 0 \).

\paragraph{Step 1: Strong Convergence.}
For any fixed \( f \in H_{\Psi_\alpha} \), we compute
\[
\| e^{-t L^2} f - f \|^2 = \sum_{n=1}^\infty \left( e^{-t \mu_n^2} - 1 \right)^2 |\langle f, e_n \rangle|^2.
\]
Since \( e^{-t \mu_n^2} \to 1 \) as \( t \to 0^+ \) for each \( n \), and \( \left| e^{-t \mu_n^2} - 1 \right| \le 2 \), the dominated convergence theorem implies:
\[
\lim_{t \to 0^+} \| e^{-t L^2} f - f \| = 0.
\]
Hence, \( e^{-tL^2} \to I \) strongly on \( H_{\Psi_\alpha} \) as \( t \to 0^+ \).

\paragraph{Step 2: Uniform Operator Bounds.}
For all \( t > 0 \), \( e^{-t L^2} \in \mathcal{B}(H_{\Psi_\alpha}) \cap \mathcal{C}_1(H_{\Psi_\alpha}) \), and satisfies
\[
\| e^{-t L^2} \|_{\mathcal{B}} \le 1,
\]
since \( L^2 \ge 0 \) implies contractivity of the semigroup. Moreover, the trace norm is finite:
\[
\| e^{-t L^2} \|_{\mathcal{C}_1} = \sum_{n=1}^\infty e^{-t \mu_n^2} < \infty,
\]
since the decay of \( \mu_n \to 0 \) ensures absolute summability of the heat weights for all \( t > 0 \).

\paragraph{Step 3: Analyticity and Equicontinuity.}
The semigroup \( \{ e^{-t L^2} \} \) is analytic in \( t \) and equicontinuous on norm-bounded subsets of \( H_{\Psi_\alpha} \), as it arises from a holomorphic semigroup generated by a positive compact self-adjoint operator.

\paragraph{Conclusion.}
Thus, \( \{ e^{-t L^2} \}_{t > 0} \subset \mathcal{C}_1(H_{\Psi_\alpha}) \) is a strongly continuous semigroup satisfying
\[
\lim_{t \to 0^+} e^{-t L^2} f = f, \qquad \forall f \in H_{\Psi_\alpha}.
\]
This completes the proof.
\end{proof}


%------------------------------------------------------------------
% --- Spectral dimension and trace scaling interpretation ---
\begin{remark}[Spectral Interpretation of Heat Trace Scaling]
\label{rem:spectral_dimension}
The two-sided asymptotic estimate
\[
\operatorname{Tr}(e^{-t L_{\mathrm{sym}}^2}) \asymp t^{-1/2} \qquad \text{as } t \to 0^+
\]
admits a natural spectral interpretation: it reflects an effective spectral dimension \( d = 1 \) in the sense of a log-enhanced Weyl law.

\medskip
\noindent
Specifically, if the eigenvalue counting function for the squared spectrum of \( L_{\mathrm{sym}} \),
\[
N(\lambda) := \#\{ n : \mu_n^2 \le \lambda \},
\]
satisfies the asymptotic growth law
\[
N(\lambda) \sim C \lambda^{1/2} \log \lambda, \qquad \text{as } \lambda \to \infty,
\]
then a Tauberian inversion (see Chapter~\ref{sec:tauberian-growth}) implies that
\[
\operatorname{Tr}(e^{-t L_{\mathrm{sym}}^2}) \sim \frac{1}{\sqrt{4\pi t}} \log\left( \frac{1}{t} \right),
\]
as confirmed in Proposition~\ref{prop:short_time_heat_expansion}.

\medskip
\noindent
This aligns \( L_{\mathrm{sym}} \) with pseudodifferential-type operators exhibiting one-dimensional spectral behavior, modulated by logarithmic corrections. It also supports the analytic structure of the zeta-regularized spectral determinant:
\[
\det\nolimits_\zeta(I - \lambda L_{\mathrm{sym}}) = \exp\left( - \int_0^\infty \frac{\operatorname{Tr}(e^{-t L_{\mathrm{sym}}^2}) - P(t)}{t} \, e^{-\lambda^2 t} \, dt \right),
\]
where \( P(t) \sim \frac{1}{\sqrt{4\pi t}} \log(1/t) + \cdots \) is the parametrix subtraction term. The singularity of the trace near \( t = 0 \) determines the growth and holomorphic domain of the spectral determinant via Laplace–Mellin regularization.
\end{remark}


%------------------------------------------------------------------
\subsection*{Summary}
\label{sec:foundations_summary}

\textbf{Operator-Theoretic Foundations}
\begin{itemize}
  \item \defref{def:compact_operator} — Compact operators: norm limits of finite-rank maps with discrete spectrum.
  \item \defref{def:trace_class_operator}, \defref{def:trace_norm} — Trace-class operators \( T \in \TC(H) \) with finite trace norm \( \|T\|_{\Tr} := \Tr(|T|) \); Banach completeness and unitary invariance.
  \item \defref{def:selfadjoint_operator} — Self-adjointness as maximal symmetry enabling spectral calculus and semigroup generation.
\end{itemize}

\textbf{Weighted Spaces and Function Classes}
\begin{itemize}
  \item \defref{def:exponential_weight}, \defref{def:weighted_schwartz_space} — The space \( \HPsi = L^2(\R, e^{\alpha|x|}\,dx) \), with \( \Schwartz(\R) \subset \HPsi \) a dense core.
  \item \lemref{lem:density_schwartz_weighted_L2} — Density of \( \Schwartz \subset \HPsi \) in norm and graph topology.
  \item \remref{rem:sobolev_core_reference} — Alternate justification: \( \Schwartz \hookrightarrow H^s_\alpha \hookrightarrow \HPsi \) via Sobolev embeddings.
\end{itemize}

\textbf{Analytic and Spectral Estimates}
\begin{itemize}
  \item \lemref{lem:xi_growth_bound}, \lemref{lem:weighted_L1_inverse_FT_xi} — The profile \( \Xi(\tfrac{1}{2} + i\lambda) \in \PW{\pi} \), with inverse transform in \( L^1(\R, \Psi_\alpha^{-1}) \).
  \item \lemref{lem:decay_mollified_kernel}, \lemref{lem:L1_integrability_conjugated_kernel} — Mollifiers \( k_t \in \Schwartz \), conjugated kernels integrable.
  \item \lemref{lem:uniform_L1_conjugated_kernel}, \lemref{lem:trace_class_via_weighted_L1} — Trace norm convergence \( \|L_t - \Lsym\|_{\TC} \to 0 \) and Simon’s trace-class inclusion criterion.
  \item \lemref{lem:trace_class_conjugated_kernel}, \lemref{lem:trace_class_failure_alpha_leq_pi}, \propref{prop:trace_class_sharpness} — Trace-class fails for \( \alpha \le \pi \): sharp exponential decay threshold.
  \item \lemref{lem:unitary_conjugation_trace_class} — Trace norm preserved under unitary weight conjugation.
\end{itemize}

\textbf{Operator Properties of \texorpdfstring{\( L_t \)}{Lt}}
\begin{itemize}
  \item \propref{prop:boundedness_Lt_weighted}, \propref{prop:compactness_Lt} — Boundedness and compactness of \( L_t \) via mollified kernel structure.
  \item \propref{prop:symmetry_Lt_Schwartz}, \propref{prop:selfadjointness_Lt} — \( L_t \) is symmetric on \( \Schwartz \) and extends to a self-adjoint operator.
  \item \propref{prop:core_schwartz_density} — \( \Schwartz \) is a core for the limit operator \( \Lsym \).
\end{itemize}

\textbf{Canonical Operator Realization}
\begin{itemize}
  \item \thmref{thm:canonical_operator_realization} — Convergence \( L_t \to \Lsym \in \TC(\HPsi) \); defines the canonical compact self-adjoint operator realizing the spectral determinant.
\end{itemize}

\paragraph{Chapter Closure.}
This chapter establishes the analytic and operator-theoretic base for all that follows. The canonical convolution operator \( \Lsym \in \TC(\HPsi) \) is defined as the trace-norm limit of mollified Fourier convolution operators \( L_t \). Its construction relies on Paley--Wiener estimates, exponential decay, Sobolev density, and trace-class embedding theorems. The determinant identity
\[
\detz(I - \lambda \Lsym)
= \frac{\Xi\left(\tfrac{1}{2} + i\lambda \right)}{\Xi\left(\tfrac{1}{2} \right)}
\]
is proven in \secref{sec:determinant_identity}, resting entirely on this analytic groundwork.


\section{Spectral Implications: Logical Equivalence and Rigidity}
\label{sec:spectral_implications}

\subsection*{Introduction}

This chapter establishes the analytic infrastructure for defining and analyzing the canonical compact operator \( L_{\mathrm{sym}} \), which realizes the completed Riemann zeta function \( \Xi(s) \) via its Fredholm determinant. The primary goal is to verify that mollified convolution operators associated with the inverse Fourier transform of \( \Xi \) are compact, trace class, and converge in trace norm to a self-adjoint limit operator \( L_{\mathrm{sym}} \in \mathcal{C}_1(H_{\Psi_\alpha}) \).

The constructions here verify:

\begin{itemize}
    \item Schatten-class properties of Hilbert–Schmidt and trace-class operators, following \cite[Ch.~4]{Simon2005TraceIdeals} and \cite[Ch.~VI]{ReedSimon1980I}, including the completeness of \( \mathcal{C}_1 \) and the trace-norm topology.
    
    \item Sufficient conditions for compactness and self-adjointness of integral operators with symmetric Hermitian kernels, using distributional domains and exponential conjugation.
    
    \item The structure of the weighted Schwartz space \( \mathcal{S}_\alpha(\mathbb{R}) \subset L^2(\mathbb{R}, e^{\alpha |x|}\, dx) \), for \( \alpha > \pi \), ensuring Fourier duality and decay control for entire functions of exponential type \( \pi \) \cite{Levin1996EntireLectures}.
    
    \item Uniform kernel bounds and mollifier admissibility for defining the regularized heat operators \( e^{-t L_t^2} \), together with analytic kernel estimates necessary for short-time trace control and Tauberian convergence.
\end{itemize}

These ingredients culminate in the construction of mollified convolution operators \( L_t \), and in the verification of trace-norm convergence
\[
L_t \to L_{\mathrm{sym}} \in \mathcal{C}_1(H_{\Psi_\alpha}) \quad \text{as } t \to 0^+.
\]
This limit defines the canonical spectral operator underlying the determinant identity
\[
\det\nolimits_{\zeta}(I - \lambda L_{\mathrm{sym}}) = \frac{\Xi\left(\tfrac{1}{2} + i\lambda\right)}{\Xi\left(\tfrac{1}{2}\right)},
\]
which is rigorously established without assuming RH.

\medskip

The analytic architecture developed here underpins all subsequent spectral and determinant identities.
See Appendix~\ref{app:dependency-graph} for a visual DAG linking these foundational tools to the modular proof of RH.


%------------------------------------------------------------------
\subsection{Equivalence with the Riemann Hypothesis}

\begin{theorem}[Spectral Reformulation of the Riemann Hypothesis]
\label{thm:eq_of_rh}

Let \( L_{\mathrm{sym}} \in \mathcal{C}_1(H_{\Psi_\alpha}) \cap \mathcal{K}(H_{\Psi_\alpha}) \) denote the canonical compact, self-adjoint, trace-class operator on the exponentially weighted Hilbert space
\[
H_{\Psi_\alpha} := L^2(\mathbb{R}, e^{\alpha |x|} dx), \qquad \alpha > \pi,
\]
constructed via mollified convolution with the inverse Fourier transform of the completed zeta profile
\[
\phi(\lambda) := \Xi\left( \tfrac{1}{2} + i\lambda \right),
\]
as rigorously developed in Chapters~\ref{sec:operator_construction}–\ref{sec:heat_kernel_asymptotics} and Appendix~\ref{app:heat_kernel_construction}.

\medskip

Suppose its Carleman–\(\zeta\)-regularized Fredholm determinant satisfies the canonical identity:
\[
\det\nolimits_\zeta(I - \lambda L_{\mathrm{sym}}) = \frac{\Xi\left( \tfrac{1}{2} + i\lambda \right)}{\Xi\left( \tfrac{1}{2} \right)},
\qquad \forall \lambda \in \mathbb{C},
\]
as proven unconditionally in \thmref{thm:det_identity_revised}.

\medskip

Then the Riemann Hypothesis is equivalent to the spectral reality of \( L_{\mathrm{sym}} \):
\[
\RH \quad \Longleftrightarrow \quad \operatorname{Spec}(L_{\mathrm{sym}}) \subset \mathbb{R}.
\]

\medskip

\noindent
Explicitly, for each nontrivial zero \( \rho = \beta + i\gamma \) of \( \zeta(s) \), define the canonical spectral image:
\[
\mu_\rho := \frac{1}{i}(\rho - \tfrac{1}{2}) = \gamma \in \mathbb{C}.
\]
Then:
\begin{itemize}
  \item \( \mu_\rho \in \mathbb{R} \iff \operatorname{Re}(\rho) = \tfrac{1}{2} \);
  \item Hence,
  \[
  \operatorname{Spec}(L_{\mathrm{sym}}) \subset \mathbb{R}
  \quad \Longleftrightarrow \quad
  \text{All nontrivial zeros } \rho \text{ lie on the critical line.}
  \]
\end{itemize}

\medskip

\noindent
This equivalence follows from:
\begin{enumerate}
  \item The analytic–spectral identity for the determinant of \( L_{\mathrm{sym}} \);
  \item The bijective spectral map \( \rho \mapsto \mu_\rho \) between nontrivial zeros and the nonzero spectrum of \( L_{\mathrm{sym}} \) (see \thmref{thm:spectral_zero_bijection_revised});
  \item The algebraic inversion identity:
  \[
  \mu_\rho \in \mathbb{R} \iff \operatorname{Re}(\rho) = \tfrac{1}{2}.
  \]
\end{enumerate}

\medskip

\noindent
Thus, the Riemann Hypothesis is equivalent to the condition that the entire spectrum of a canonical trace-class operator lies on the real line. This establishes a logically acyclic, operator-theoretic reformulation of RH within the framework of Fredholm theory and spectral determinant calculus.
\end{theorem}

\begin{proof}[Proof of \thmref{thm:eq_of_rh}]
Let \( \rho \mapsto \mu_\rho := \dfrac{1}{i}(\rho - \tfrac{1}{2}) \) denote the canonical spectral reparametrization of the nontrivial zeros \( \rho \) of the Riemann zeta function \( \zeta(s) \). This map is bijective and multiplicity-preserving by \thmref{thm:spectral_zero_bijection_revised} and \lemref{lem:multiplicity_preservation}, ensuring complete spectral correspondence with the nonzero eigenvalues of \( \Lsym \in \TC(\HPsi) \).

\medskip

By \thmref{thm:det_identity_revised}, the Carleman \(\zeta\)-regularized Fredholm determinant of \( \Lsym \) satisfies:
\[
\detz(I - \lambda \Lsym) = \frac{\Xi\left( \tfrac{1}{2} + i\lambda \right)}{\Xi\left( \tfrac{1}{2} \right)},
\qquad \forall \lambda \in \C.
\]
This identity is constructed without assuming RH and rests on analytic inputs from Chapter~\ref{sec:heat_kernel_asymptotics} and Appendix~\ref{app:heat_kernel_construction}, particularly the convergence and singularity structure derived in \lemref{lem:heat_trace_expansion}, \lemref{lem:trace_class_Lt}, and \lemref{lem:kernel_trace_norm_convergence}.

\medskip

Now observe the algebraic implication: for any nontrivial zero \( \rho = \sigma + i\gamma \),
\[
\mu_\rho := \frac{1}{i}(\rho - \tfrac{1}{2}) = \gamma + i(\sigma - \tfrac{1}{2}).
\]
Thus,
\[
\mu_\rho \in \R \iff \Re(\rho) = \tfrac{1}{2}.
\]

\paragraph*{\( \Rightarrow \): Spectral Reality Implies \(\RH\).}
Assume \( \Spec(\Lsym) \subset \R \). Then each \( \mu_\rho \in \R \), so the identity above implies that \( \Re(\rho) = \tfrac{1}{2} \). Hence, all nontrivial zeros lie on the critical line and RH holds.

\paragraph*{\( \Leftarrow \): \(\RH\) Implies Spectral Reality.}
Conversely, assume RH holds. Then for each \( \rho \), \( \Re(\rho) = \tfrac{1}{2} \), which implies \( \mu_\rho \in \R \). Therefore, all nonzero eigenvalues of \( \Lsym \) are real. By self-adjointness and the spectral theorem, \( \Spec(\Lsym) \subset \R \).

\paragraph*{Conclusion.}
The spectrum of \( \Lsym \) is real if and only if all nontrivial zeros of \( \zeta(s) \) lie on the critical line. This establishes the analytic–spectral equivalence
\[
\RH \iff \Spec(\Lsym) \subset \R,
\]
completing the modular operator-theoretic reformulation of the Riemann Hypothesis.
\end{proof}


% Additional lemma: preservation of spectral multiplicity via determinant
\begin{lemma}[Spectral Multiplicity Preservation]
\label{lem:multiplicity_preservation}

Let \( \rho \in \C \) be a nontrivial zero of the Riemann zeta function \( \zetaR(s) \), and define its canonical spectral image
\[
\mu_\rho := \frac{1}{i(\rho - \tfrac{1}{2})}.
\]
Then \( \mu_\rho \in \Spec(L_{\sym}) \) appears with algebraic multiplicity equal to the order of vanishing of \( \zetaR(s) \) at \( \rho \).

\medskip

\noindent
This multiplicity correspondence follows from the Hadamard factorization of the completed zeta function \( \XiR(s) \), which governs the zero structure of the normalized Carleman–\(\zeta\)-regularized Fredholm determinant of \( L_{\sym} \in \TC(\HPsi) \cap \KC(\HPsi) \):
\[
\det\nolimits_\zeta(I - \lambda L_{\sym}) = \frac{\XiR(\tfrac{1}{2} + i\lambda)}{\XiR(\tfrac{1}{2})}.
\]

Taking the logarithmic derivative, we obtain a meromorphic function whose poles correspond to the spectral values:
\[
\frac{d}{d\lambda} \log \det\nolimits_{\zeta}(I - \lambda L_{\sym}) = \sum_{\rho} \frac{m_\rho}{\lambda - \mu_\rho},
\]
where \( m_\rho \) is the multiplicity of the zero \( \rho \) of \( \zetaR \), and \( \mu_\rho \) is its spectral image. This expansion reflects the classical Hadamard product representation of \( \XiR(s) \), and matches the spectral resolvent trace identity for trace-class self-adjoint operators.

\medskip

\noindent
Since \( L_{\sym} \) is compact and self-adjoint, its spectrum consists of isolated real eigenvalues with finite algebraic multiplicity. The residues of the logarithmic derivative coincide with these multiplicities. Therefore, the multiplicity of each spectral point \( \mu_\rho \) matches exactly the order of vanishing of \( \zetaR(s) \) at \( \rho \), as claimed.
\end{lemma}

\begin{proof}
The determinant identity (Theorem~\ref{thm:det-identity-revised}) states:
\[
\det\nolimits_{\zeta}(I - \lambda L_{\mathrm{sym}})
= \frac{\Xi\left(\tfrac{1}{2} + i\lambda\right)}{\Xi\left(\tfrac{1}{2}\right)},
\]
where \( \Xi(s) \) admits a Hadamard factorization over the nontrivial zeros \( \rho \) of \( \zeta(s) \):
\[
\Xi(s) = \Xi\left(\tfrac{1}{2}\right)
\prod_\rho \left(1 - \frac{s - \tfrac{1}{2}}{\rho - \tfrac{1}{2}}\right)
\exp\left( \frac{s - \tfrac{1}{2}}{\rho - \tfrac{1}{2}} \right).
\]
Letting \( \mu_\rho := \frac{1}{i(\rho - \tfrac{1}{2})} \), the determinant vanishes at \( \lambda = 1/\mu_\rho \), with multiplicity equal to the order of vanishing of \( \Xi(s) \) at \( \rho \).

By Lemma~\ref{lem:A_log_derivative}, the logarithmic derivative satisfies:
\[
\frac{d}{d\lambda} \log \det\nolimits_\zeta(I - \lambda L_{\mathrm{sym}})
= \operatorname{Tr}\left( (I - \lambda L_{\mathrm{sym}})^{-1} L_{\mathrm{sym}} \right),
\]
whose pole structure at \( \lambda = 1/\mu_\rho \) reflects the multiplicity of the eigenvalue \( \mu_\rho \). Thus, the eigenvalue multiplicity of \( \mu_\rho \) equals the order of vanishing of \( \zeta(s) \) at \( \rho \), as claimed.
\end{proof}
% 

% Spectral reality lemma: RH ⇔ spectrum real
\begin{lemma}[Spectrum Reality and RH Equivalence]
\label{lem:spectrum-reality-implies-rh}
If all zeros of \( \Xi(\tfrac{1}{2} + i\lambda) \) lie on the real axis, then the canonical convolution operator \( L_{\mathrm{sym}} \) has real spectrum.

Conversely, if \( L_{\mathrm{sym}} \in \mathcal{C}_1(H_{\Psi_\alpha}) \) is self-adjoint, then all eigenvalues \( \mu_\rho \in \mathbb{R} \) imply that the associated nontrivial zeta zeros \( \rho \in \mathbb{C} \) satisfy \( \operatorname{Re}(\rho) = \tfrac{1}{2} \), i.e., the Riemann Hypothesis holds.
\end{lemma}
%  
\begin{proof}[Proof of \lemref{lem:reality_of_spectrum_and_rh}]
Let \( \rho = \beta + i\gamma \) be a nontrivial zero of the Riemann zeta function \( \zetaR(s) \). Define its canonical spectral image:
\[
\mu_\rho = \frac{1}{i(\rho - \tfrac{1}{2})} = \frac{1}{i((\beta - \tfrac{1}{2}) + i\gamma)}.
\]

Set \( z := \beta - \tfrac{1}{2} + i\gamma \in \C \). Then
\[
\mu_\rho = \frac{-i}{z} = \frac{-i((\beta - \tfrac{1}{2}) - i\gamma)}{(\beta - \tfrac{1}{2})^2 + \gamma^2}
= \frac{\gamma}{(\beta - \tfrac{1}{2})^2 + \gamma^2}
- i \cdot \frac{\beta - \tfrac{1}{2}}{(\beta - \tfrac{1}{2})^2 + \gamma^2}.
\]

Hence, \( \mu_\rho \in \R \) if and only if the imaginary part vanishes:
\[
\Im(\mu_\rho) = 0 \quad \Longleftrightarrow \quad \beta = \tfrac{1}{2}.
\]

That is,
\[
\mu_\rho \in \R \iff \rho \in \tfrac{1}{2} + i\R.
\]

Since the canonical spectral map \( \rho \mapsto \mu_\rho \) is injective and covers all nontrivial zeros of \( \zetaR \), this correspondence implies:
\[
\Spec(L_{\sym}) \subset \R \quad \Longleftrightarrow \quad \text{all } \rho \in \mathcal{Z}_\zeta \text{ satisfy } \Re(\rho) = \tfrac{1}{2}.
\]

\medskip

\noindent
Equivalently,
\[
\boxed{
\Spec(L_{\sym}) \subset \R \quad \Longleftrightarrow \quad \RH
}
\]
as claimed.
\end{proof}


% Boxed corollary: RH ⇔ spectrum real
\begin{corollary}[Equivalence of RH with Spectrum Reality]
\label{cor:spectrum-real-equiv-rh}
Let \( L_{\mathrm{sym}} \in \mathcal{C}_1(H_{\Psi_\alpha}) \) be the canonical self-adjoint trace-class operator whose Fredholm determinant satisfies
\[
\det\nolimits_\zeta(I - \lambda L_{\mathrm{sym}})
= \frac{\Xi\left(\tfrac{1}{2} + i\lambda\right)}{\Xi\left(\tfrac{1}{2}\right)}.
\]

Then the Riemann Hypothesis holds if and only if the spectrum of \( L_{\mathrm{sym}} \) is real:
\[
\boxed{
\operatorname{Spec}(L_{\mathrm{sym}}) \subset \mathbb{R}
\quad \Longleftrightarrow \quad \mathrm{RH}
}
\]

\noindent
That is, the nontrivial zeros of \( \zeta(s) \) lie on the critical line \( \Re(s) = \tfrac{1}{2} \) if and only if all eigenvalues of \( L_{\mathrm{sym}} \) are real.

\end{corollary}
% 
\begin{proof}[Proof of Corollary~\ref{cor:spectrum-real-equiv-rh}]
This is an immediate consequence of Lemma~\ref{lem:reality-of-spectrum-and-rh}. For each nontrivial zero \( \rho = \beta + i\gamma \) of \( \zeta(s) \), we define the associated eigenvalue
\[
\mu_\rho := \frac{1}{i(\rho - \tfrac{1}{2})}.
\]

By Lemma~\ref{lem:reality-of-spectrum-and-rh}, we have:
\[
\mu_\rho \in \mathbb{R} \quad \Longleftrightarrow \quad \beta = \tfrac{1}{2}.
\]
Therefore, all \( \mu_\rho \in \operatorname{Spec}(L_{\mathrm{sym}}) \) are real if and only if all nontrivial zeros \( \rho \) lie on the critical line \( \Re(\rho) = \tfrac{1}{2} \). This is precisely the Riemann Hypothesis.

Hence,
\[
\operatorname{Spec}(L_{\mathrm{sym}}) \subset \mathbb{R} \quad \iff \quad \mathrm{RH}.
\]
\end{proof}
% 

% Physics analogy remark
\begin{remark}[Spectral Physics Perspective]
The equivalence \( \mathrm{RH} \iff \operatorname{Spec}(L_{\mathrm{sym}}) \subset \mathbb{R} \) admits a speculative interpretation in the context of quantum mechanics. Under the spectral map \( \rho = \tfrac{1}{2} + i\gamma \mapsto \mu_\rho := \tfrac{1}{\gamma} \), the canonical operator \( L_{\mathrm{sym}} \) can be formally viewed as a Hamiltonian with inverse arithmetic energy levels. Its heat trace resembles a quantum partition function with singular short-time behavior, and its determinant has parallels with spectral free energy. See Appendix~\ref{app:spectral-physics-link} for further discussion of this physical analogy.
\end{remark}

% Determinant trace-log identity is used but not proved here
\medskip
\noindent
For the analytic justification of the trace–logarithmic derivative identity, see Lemma~\ref{lem:log-derivative-determinant} in Chapter~\ref{sec:heat-kernel-asymptotics} and the supporting analysis in Appendix~\ref{app:heat-kernel-construction}.

%------------------------------------------------------------------
\subsection{Uniqueness of Spectral Realization}

% This result builds directly on the spectral-zero bijection and the determinant identity,
% enforcing uniqueness within the trace-class, self-adjoint realization space.
\begin{theorem}[Uniqueness of Spectral Realization]
\label{thm:uniqueness_realization}

Let \( L \in \mathcal{C}_1(H_{\Psi_\alpha}) \cap \mathcal{K}(H_{\Psi_\alpha}) \) be a compact, self-adjoint, trace-class operator on the exponentially weighted Hilbert space
\[
H_{\Psi_\alpha} := L^2(\mathbb{R}, e^{\alpha |x|} \, dx), \qquad \alpha > \pi.
\]

Suppose \( L \) satisfies the canonical zeta-regularized determinant identity:
\[
\det\nolimits_\zeta(I - \lambda L) = \frac{\Xi\left( \tfrac{1}{2} + i\lambda \right)}{\Xi\left( \tfrac{1}{2} \right)},
\quad \forall \lambda \in \mathbb{C},
\]
where \( \Xi(s) \) is the completed Riemann zeta function, entire of order one and exact exponential type \( \pi \). Assume the normalization:
\[
\det\nolimits_\zeta(I) = 1.
\]

Then \( L \) is unitarily equivalent to the canonical operator \( L_{\mathrm{sym}} \in \mathcal{C}_1(H_{\Psi_\alpha}) \). That is, there exists a unitary operator
\[
U \colon H_{\Psi_\alpha} \to H_{\Psi_\alpha} \quad \text{such that} \quad L = U L_{\mathrm{sym}} U^{-1}.
\]

\medskip
\noindent
In particular:
\begin{itemize}
  \item The spectrum of \( L \), including all algebraic multiplicities, coincides with that of \( L_{\mathrm{sym}} \);
  \item \( L_{\mathrm{sym}} \) is the unique (up to unitary equivalence) compact, self-adjoint, trace-class realization of the completed zeta function’s canonical spectral determinant;
  \item The analytic data encoded in \( \Xi(s) \)—via its Hadamard factorization and spectral trace regularization—rigidly determines the operator-theoretic structure of \( L_{\mathrm{sym}} \).
\end{itemize}
\end{theorem}

\begin{proof}[Proof of \thmref{thm:uniqueness_realization}]
Let \( L \in \mathcal{C}_1(H_{\Psi_\alpha}) \cap \mathcal{K}(H_{\Psi_\alpha}) \) be a compact, self-adjoint, trace-class operator on the weighted Hilbert space \( H_{\Psi_\alpha} := L^2(\mathbb{R}, e^{\alpha|x|}dx) \), with fixed \( \alpha > \pi \). Suppose:
\[
\det\nolimits_\zeta(I - \lambda L) = \frac{\Xi\left(\tfrac{1}{2} + i\lambda\right)}{\Xi\left(\tfrac{1}{2}\right)}
= \det\nolimits_\zeta(I - \lambda L_{\mathrm{sym}}),
\quad \forall \lambda \in \mathbb{C}.
\]

\paragraph{Step 1: Spectral Data from Determinant Identity.}
By classical trace-class determinant theory (see~\cite[Thm. 4.2]{Simon2005TraceIdeals}), the normalized Carleman–\(\zeta\)-regularized determinant admits the product representation:
\[
\det\nolimits_\zeta(I - \lambda L)
= \prod_{n=1}^\infty (1 - \lambda \mu_n),
\]
where \( \{ \mu_n \} \subset \mathbb{R} \setminus \{0\} \) are the nonzero eigenvalues of \( L \), counted with algebraic multiplicity. Since this determinant agrees identically with that of \( L_{\mathrm{sym}} \), and both are entire functions of order one normalized by \( \det\nolimits_\zeta(I) = 1 \), we conclude:
\[
\operatorname{Spec}(L) = \operatorname{Spec}(L_{\mathrm{sym}}),
\quad \text{as multisets}.
\]

\paragraph{Step 2: Spectral Equivalence Implies Unitary Equivalence.}
Since \( L \) and \( L_{\mathrm{sym}} \) are both compact, self-adjoint operators on the same separable Hilbert space \( H_{\Psi_\alpha} \), and since their spectra (with multiplicities) coincide, the spectral theorem implies that \( L \) is unitarily equivalent to \( L_{\mathrm{sym}} \). That is, there exists a unitary operator
\[
U \colon H_{\Psi_\alpha} \to H_{\Psi_\alpha}
\quad \text{such that} \quad
L = U L_{\mathrm{sym}} U^{-1}.
\]

\paragraph{Conclusion.}
The canonical operator \( L_{\mathrm{sym}} \) is thus uniquely determined (up to unitary equivalence) among all compact, self-adjoint, trace-class operators realizing the normalized spectral determinant identity for \( \Xi(s) \). The analytic fingerprint of \( \Xi \)—its order-one entire structure, exponential type, and Hadamard factorization—rigidly determines the operator-theoretic data of \( L_{\mathrm{sym}} \), completing the proof.
\end{proof}


% Spectral rigidity lemma: determinant identity forces spectral agreement
\begin{lemma}[Spectral Rigidity from Determinant Identity]
\label{lem:spectral_rigidity_determinant}

Let \( L_1, L_2 \in \TC(\HPsi) \cap \KC(\HPsi) \) be compact, self-adjoint, trace-class operators on the exponentially weighted Hilbert space \( \HPsi := L^2(\R, e^{\alpha|x|}dx) \) with \( \alpha > \pi \).

Suppose their Carleman–\(\zeta\)-regularized Fredholm determinants coincide:
\[
\det\nolimits_\zeta(I - \lambda L_1) = \det\nolimits_\zeta(I - \lambda L_2),
\quad \forall \lambda \in \C,
\]
with both normalized at the origin:
\[
\det\nolimits_\zeta(I) = 1.
\]

Then \( L_1 \) and \( L_2 \) have identical nonzero spectra, including algebraic multiplicities:
\[
\Spec(L_1) \setminus \{0\} = \Spec(L_2) \setminus \{0\}
\quad \text{as multisets}.
\]

If both operators act on the same Hilbert space, then the spectral theorem implies they are unitarily equivalent.
\end{lemma}

\begin{proof}[Proof of \lemref{lem:spectral_rigidity_determinant}]
Let \( L_1, L_2 \in \TC(\HPsi) \cap \KC(\HPsi) \) be compact, self-adjoint, trace-class operators satisfying:
\[
\det\nolimits_\zeta(I - \lambda L_1) = \det\nolimits_\zeta(I - \lambda L_2), \quad \forall \lambda \in \C,
\]
with both determinants normalized at the origin: \( \det\nolimits_\zeta(I) = 1 \), as ensured by \lemref{lem:trace_zero}.

\paragraph{Step 1: Spectral Encoding via Determinant Structure.}
For compact, self-adjoint operators in \( \TC \), the zeta-regularized Fredholm determinant admits the canonical Hadamard product expansion:
\[
\det\nolimits_\zeta(I - \lambda L_j) = \prod_{\mu \in \Spec(L_j) \setminus \{0\}} (1 - \lambda \mu)^{\operatorname{mult}_{L_j}(\mu)},
\quad j = 1,2,
\]
by \lemref{lem:hadamard_linear_form}. Since the two determinants coincide as entire functions of order one and exponential type \( \pi \), and share the normalization \( \det\nolimits_\zeta(I) = 1 \), the identity theorem for entire functions implies that their zero sets (counted with multiplicity) must coincide. Hence:
\[
\Spec(L_1) \setminus \{0\} = \Spec(L_2) \setminus \{0\}
\quad \text{as multisets}.
\]

\paragraph{Step 2: Completion via Spectral Theorem.}
If \( L_1 \) and \( L_2 \) act on the same Hilbert space \( \HPsi \), then the spectral theorem for compact self-adjoint operators ensures the existence of a unitary operator
\[
U \colon \HPsi \to \HPsi
\quad \text{such that} \quad
L_2 = U L_1 U^{-1}.
\]

\paragraph{Conclusion.}
Thus, the Carleman–\(\zeta\)-regularized Fredholm determinant serves as a complete spectral fingerprint for compact, self-adjoint trace-class operators: the analytic data of the determinant determines the operator spectrum uniquely, and—on a fixed Hilbert space—determines the operator itself up to unitary equivalence. This rigidity underlies the uniqueness result in \thmref{thm:uniqueness_realization}.
\end{proof}


% Determinant uniquely fixes spectral multiset
\begin{lemma}[Determinant Identity Fixes the Spectrum]
\label{lem:determinant_fixes_spectrum}

Let \( L \in \TC(\HPsi) \cap \KC(\HPsi) \) be a compact, self-adjoint, trace-class operator satisfying the normalized spectral determinant identity:
\[
\det\nolimits_\zeta(I - \lambda L) = \frac{\XiR\left(\tfrac{1}{2} + i\lambda\right)}{\XiR\left(\tfrac{1}{2}\right)},
\quad \forall \lambda \in \C,
\]
and assume its trace vanishes:
\[
\Tr(L) = 0.
\]

Then the nonzero spectrum of \( L \), counted with algebraic multiplicity, coincides with that of the canonical operator \( L_{\sym} \). That is,
\[
\Spec(L) \setminus \{0\} = \Spec(L_{\sym}) \setminus \{0\},
\quad \text{as multisets}.
\]
\thmref{thm:canonical_operator_realization}
\thmref{thm:det_identity_revised}
\lemref{lem:trace_zero}
\end{lemma}

\begin{proof}[Proof of \lemref{lem:determinant_fixes_spectrum}]
Let \( f(\lambda) := \det\nolimits_\zeta(I - \lambda L) \), and suppose
\[
f(\lambda) = \frac{\XiR(\tfrac{1}{2} + i\lambda)}{\XiR(\tfrac{1}{2})}
= \det\nolimits_\zeta(I - \lambda L_{\sym}),
\quad \forall \lambda \in \C,
\]
where \( L \in \TC(\HPsi) \cap \KC(\HPsi) \) is compact, self-adjoint, trace-class, and satisfies \( \Tr(L) = 0 \).

\paragraph{Step 1: Entire Function Identity and Trace Normalization.}
Both determinant functions are entire of order one and exponential type \( \pi \), and normalized so that \( f(0) = 1 \). The trace-zero condition removes any exponential prefactor ambiguity in their Hadamard factorization—i.e., no term of the form \( e^{a\lambda} \) appears.

\paragraph{Step 2: Logarithmic Derivative and Spectral Poles.}
The logarithmic derivative of the determinant is governed by the resolvent trace formula:
\[
\frac{d}{d\lambda} \log f(\lambda)
= \Tr\left[(I - \lambda L)^{-1} L\right],
\]
which is meromorphic with simple poles at \( \lambda = 1/\mu \) for each nonzero eigenvalue \( \mu \in \Spec(L) \), with residue equal to the algebraic multiplicity of \( \mu \).

Since the determinant agrees with that of \( L_{\sym} \), these poles match those of the canonical model, and thus:
\[
\Spec(L) \setminus \{0\} = \Spec(L_{\sym}) \setminus \{0\}
\quad \text{as multisets}.
\]

\paragraph{Conclusion.}
The spectral data of \( L \), away from zero, is completely encoded by the determinant under the trace normalization condition. Therefore, \( L \) and \( L_{\sym} \) have identical nonzero spectra, completing the proof.
\end{proof}


%------------------------------------------------------------------
\subsection{Canonical Closure of the Spectral Program}

\begin{lemma}[Canonical Closure of the Spectral Model]
\label{lem:canonical_closure}

Let \( L \in \TC(\HPsi) \cap \KC(\HPsi) \) be a compact, self-adjoint, trace-class operator on the exponentially weighted Hilbert space
\[
\HPsi := L^2(\R, e^{\alpha |x|} \, dx), \qquad \alpha > \pi,
\]
and suppose \( L \) satisfies the normalized spectral determinant identity:
\[
\det\nolimits_\zeta(I - \lambda L) = \frac{\XiR\left(\tfrac{1}{2} + i\lambda\right)}{\XiR\left(\tfrac{1}{2}\right)}, \qquad \forall \lambda \in \C,
\]
with normalization \( \det\nolimits_\zeta(I) = 1 \), and where \( \XiR(s) \) is the completed Riemann zeta function.

\medskip
\noindent
Then:
\begin{enumerate}
  \item The nonzero spectrum of \( L \) coincides with that of the canonical operator \( L_{\sym} \), as multisets with algebraic multiplicities:
  \[
  \Spec(L) \setminus \{0\} = \Spec(L_{\sym}) \setminus \{0\};
  \]
  
  \item \( L \) is unitarily equivalent to \( L_{\sym} \): there exists a unitary operator \( U \colon \HPsi \to \HPsi \) such that
  \[
  L = U L_{\sym} U^{-1};
  \]
  
  \item \( L_{\sym} \) is the unique (up to unitary equivalence) compact, self-adjoint trace-class operator whose zeta-regularized determinant realizes the spectral identity associated with \( \XiR(s) \);
  
  \item If \( \widetilde{L} \in \TC \) satisfies the same determinant identity but is not self-adjoint, then \( \widetilde{L} \) is similar to \( L_{\sym} \) in the algebraic sense: there exists an invertible operator \( S \in \mathcal{B}(\HPsi) \) such that
  \[
  \widetilde{L} = S L_{\sym} S^{-1},
  \]
  preserving the nonzero spectrum and multiplicities, though not necessarily realized via a unitary conjugation.
\end{enumerate}

\medskip
\noindent
Hence, the canonical spectral determinant associated with \( \XiR(s) \), under trace-class and self-adjointness, uniquely determines the operator \( L_{\sym} \) up to unitary equivalence, and rigidly constrains all other determinant-realizing models to algebraic similarity. This completes the canonical closure of the spectral model.
\end{lemma}

\begin{proof}[Proof of \lemref{lem:canonical_closure}]
By assumption, \( L \in \TC(\HPsi) \cap \KC(\HPsi) \) is compact, self-adjoint, and satisfies the normalized spectral determinant identity:
\[
\det\nolimits_\zeta(I - \lambda L) = \frac{\XiR\left(\tfrac{1}{2} + i\lambda\right)}{\XiR\left(\tfrac{1}{2}\right)}, \quad \forall \lambda \in \C.
\]

\paragraph{(1) Spectral Equality from Determinant Identity.}
By trace-class determinant theory (see~\cite[Theorem~4.2]{Simon2005TraceIdeals}), the zeta-regularized determinant encodes the nonzero spectrum of \( L \) as a multiset (including algebraic multiplicities). Since the determinant of \( L \) matches that of \( L_{\sym} \), we conclude:
\[
\Spec(L) \setminus \{0\} = \Spec(L_{\sym}) \setminus \{0\}.
\]

\paragraph{(2) Unitary Equivalence for Self-Adjoint Case.}
Both \( L \) and \( L_{\sym} \) are compact, self-adjoint operators on the same separable Hilbert space \( \HPsi \), with matching spectra and multiplicities. By the spectral theorem for compact self-adjoint operators (see~\cite[Theorem~VI.16]{ReedSimon1980I}), there exists a unitary operator \( U \colon \HPsi \to \HPsi \) such that
\[
L = U L_{\sym} U^{-1}.
\]

\paragraph{(3) Uniqueness of the Canonical Realization.}
The above shows that \( L_{\sym} \) is unique up to unitary equivalence within the class of compact, self-adjoint, trace-class operators realizing the spectral determinant identity for \( \XiR(s) \).

\paragraph{(4) Similarity Class for Non-Self-Adjoint Realizations.}
Suppose \( \widetilde{L} \in \TC(\HPsi) \) is not self-adjoint but still satisfies the same determinant identity. Then it must have the same nonzero spectral multiset as \( L_{\sym} \), including multiplicities. While lack of normality may prevent diagonalizability or self-adjointness, spectral similarity implies the existence of an invertible operator \( S \in \mathcal{B}(\HPsi) \) such that
\[
\widetilde{L} = S L_{\sym} S^{-1}.
\]
This shows that \( \widetilde{L} \) lies in the similarity class of \( L_{\sym} \), even if not in its unitary equivalence class.

\paragraph{Conclusion.}
The spectral determinant identity associated with \( \XiR(s) \), together with trace-class compactness and self-adjointness, canonically determines the operator \( L_{\sym} \) up to unitary equivalence. Any non-self-adjoint realization is algebraically similar to this canonical model, thus completing the closure of the spectral program.
\end{proof}


%------------------------------------------------------------------

% Final logical closure of the analytic-spectral architecture.
\subsection*{Summary}
\label{sec:foundations_summary}

\textbf{Operator-Theoretic Foundations}
\begin{itemize}
  \item \defref{def:compact_operator} — Compact operators: norm limits of finite-rank maps with discrete spectrum.
  \item \defref{def:trace_class_operator}, \defref{def:trace_norm} — Trace-class operators \( T \in \TC(H) \) with finite trace norm \( \|T\|_{\Tr} := \Tr(|T|) \); Banach completeness and unitary invariance.
  \item \defref{def:selfadjoint_operator} — Self-adjointness as maximal symmetry enabling spectral calculus and semigroup generation.
\end{itemize}

\textbf{Weighted Spaces and Function Classes}
\begin{itemize}
  \item \defref{def:exponential_weight}, \defref{def:weighted_schwartz_space} — The space \( \HPsi = L^2(\R, e^{\alpha|x|}\,dx) \), with \( \Schwartz(\R) \subset \HPsi \) a dense core.
  \item \lemref{lem:density_schwartz_weighted_L2} — Density of \( \Schwartz \subset \HPsi \) in norm and graph topology.
  \item \remref{rem:sobolev_core_reference} — Alternate justification: \( \Schwartz \hookrightarrow H^s_\alpha \hookrightarrow \HPsi \) via Sobolev embeddings.
\end{itemize}

\textbf{Analytic and Spectral Estimates}
\begin{itemize}
  \item \lemref{lem:xi_growth_bound}, \lemref{lem:weighted_L1_inverse_FT_xi} — The profile \( \Xi(\tfrac{1}{2} + i\lambda) \in \PW{\pi} \), with inverse transform in \( L^1(\R, \Psi_\alpha^{-1}) \).
  \item \lemref{lem:decay_mollified_kernel}, \lemref{lem:L1_integrability_conjugated_kernel} — Mollifiers \( k_t \in \Schwartz \), conjugated kernels integrable.
  \item \lemref{lem:uniform_L1_conjugated_kernel}, \lemref{lem:trace_class_via_weighted_L1} — Trace norm convergence \( \|L_t - \Lsym\|_{\TC} \to 0 \) and Simon’s trace-class inclusion criterion.
  \item \lemref{lem:trace_class_conjugated_kernel}, \lemref{lem:trace_class_failure_alpha_leq_pi}, \propref{prop:trace_class_sharpness} — Trace-class fails for \( \alpha \le \pi \): sharp exponential decay threshold.
  \item \lemref{lem:unitary_conjugation_trace_class} — Trace norm preserved under unitary weight conjugation.
\end{itemize}

\textbf{Operator Properties of \texorpdfstring{\( L_t \)}{Lt}}
\begin{itemize}
  \item \propref{prop:boundedness_Lt_weighted}, \propref{prop:compactness_Lt} — Boundedness and compactness of \( L_t \) via mollified kernel structure.
  \item \propref{prop:symmetry_Lt_Schwartz}, \propref{prop:selfadjointness_Lt} — \( L_t \) is symmetric on \( \Schwartz \) and extends to a self-adjoint operator.
  \item \propref{prop:core_schwartz_density} — \( \Schwartz \) is a core for the limit operator \( \Lsym \).
\end{itemize}

\textbf{Canonical Operator Realization}
\begin{itemize}
  \item \thmref{thm:canonical_operator_realization} — Convergence \( L_t \to \Lsym \in \TC(\HPsi) \); defines the canonical compact self-adjoint operator realizing the spectral determinant.
\end{itemize}

\paragraph{Chapter Closure.}
This chapter establishes the analytic and operator-theoretic base for all that follows. The canonical convolution operator \( \Lsym \in \TC(\HPsi) \) is defined as the trace-norm limit of mollified Fourier convolution operators \( L_t \). Its construction relies on Paley--Wiener estimates, exponential decay, Sobolev density, and trace-class embedding theorems. The determinant identity
\[
\detz(I - \lambda \Lsym)
= \frac{\Xi\left(\tfrac{1}{2} + i\lambda \right)}{\Xi\left(\tfrac{1}{2} \right)}
\]
is proven in \secref{sec:determinant_identity}, resting entirely on this analytic groundwork.


\section{Tauberian Growth and Spectral Asymptotics}
\label{sec:tauberian-growth}

\subsection*{Introduction}

This chapter establishes the analytic infrastructure for defining and analyzing the canonical compact operator \( L_{\mathrm{sym}} \), which realizes the completed Riemann zeta function \( \Xi(s) \) via its Fredholm determinant. The primary goal is to verify that mollified convolution operators associated with the inverse Fourier transform of \( \Xi \) are compact, trace class, and converge in trace norm to a self-adjoint limit operator \( L_{\mathrm{sym}} \in \mathcal{C}_1(H_{\Psi_\alpha}) \).

The constructions here verify:

\begin{itemize}
    \item Schatten-class properties of Hilbert–Schmidt and trace-class operators, following \cite[Ch.~4]{Simon2005TraceIdeals} and \cite[Ch.~VI]{ReedSimon1980I}, including the completeness of \( \mathcal{C}_1 \) and the trace-norm topology.
    
    \item Sufficient conditions for compactness and self-adjointness of integral operators with symmetric Hermitian kernels, using distributional domains and exponential conjugation.
    
    \item The structure of the weighted Schwartz space \( \mathcal{S}_\alpha(\mathbb{R}) \subset L^2(\mathbb{R}, e^{\alpha |x|}\, dx) \), for \( \alpha > \pi \), ensuring Fourier duality and decay control for entire functions of exponential type \( \pi \) \cite{Levin1996EntireLectures}.
    
    \item Uniform kernel bounds and mollifier admissibility for defining the regularized heat operators \( e^{-t L_t^2} \), together with analytic kernel estimates necessary for short-time trace control and Tauberian convergence.
\end{itemize}

These ingredients culminate in the construction of mollified convolution operators \( L_t \), and in the verification of trace-norm convergence
\[
L_t \to L_{\mathrm{sym}} \in \mathcal{C}_1(H_{\Psi_\alpha}) \quad \text{as } t \to 0^+.
\]
This limit defines the canonical spectral operator underlying the determinant identity
\[
\det\nolimits_{\zeta}(I - \lambda L_{\mathrm{sym}}) = \frac{\Xi\left(\tfrac{1}{2} + i\lambda\right)}{\Xi\left(\tfrac{1}{2}\right)},
\]
which is rigorously established without assuming RH.

\medskip

The analytic architecture developed here underpins all subsequent spectral and determinant identities.
See Appendix~\ref{app:dependency-graph} for a visual DAG linking these foundational tools to the modular proof of RH.


%------------------------------------------------------------------
\subsection{Definitions}
% === Tauberian framework: Laplace growth and spectral inversion ===

\begin{definition}[Tauberian Theorem for Spectral Counting]
\label{def:tauberian-theorem}
Let \( L \in \mathcal{C}_1(H) \) be a compact, self-adjoint operator on a separable Hilbert space \( H \), with discrete nonzero spectrum \( \{ \mu_n \} \subset \mathbb{R} \setminus \{0\} \), counted with multiplicity. Define the squared spectral counting function
\[
A(\Lambda) := \#\left\{ n \in \mathbb{N} : \mu_n^2 \le \Lambda \right\}, \qquad \Lambda > 0.
\]

Suppose the spectral heat trace admits the short-time asymptotic expansion:
\[
\operatorname{Tr}(e^{-t L^2}) = \frac{C}{t^\alpha} + o(t^{-\alpha}), \qquad \text{as } t \to 0^+,
\]
for some constant \( C > 0 \) and exponent \( \alpha > 0 \). Then the spectral counting function satisfies the asymptotic law:
\[
A(\Lambda) \sim \frac{C}{\Gamma(\alpha + 1)} \, \Lambda^\alpha, \qquad \text{as } \Lambda \to \infty.
\]

\medskip
\noindent
This follows by applying the classical Karamata Tauberian theorem to the Laplace–Stieltjes representation:
\[
\operatorname{Tr}(e^{-t L^2}) = \int_0^\infty e^{-t \lambda} \, dA(\lambda).
\]

\medskip
\noindent
If instead the heat trace exhibits log-modulated singular behavior:
\[
\operatorname{Tr}(e^{-t L^2}) \sim \frac{C}{t^\alpha} \log\left( \frac{1}{t} \right), \qquad \text{as } t \to 0^+,
\]
then Korevaar’s log-corrected Tauberian theorem~\cite[Ch.~III, §5]{Korevaar2004Tauberian} yields the enhanced spectral counting asymptotic:
\[
A(\Lambda) \sim \frac{C}{\Gamma(\alpha + 1)} \, \Lambda^\alpha \log \Lambda, \qquad \text{as } \Lambda \to \infty.
\]
\end{definition}
  % Korevaar's Tauberian setup

%------------------------------------------------------------------
\subsection{Tauberian Lemmas and Asymptotic Estimates}
% === Spectral growth bounds from Laplace transforms and heat trace asymptotics ===

\begin{lemma}[Spectral Convexity Estimate]
\label{lem:spectral_convexity_estimate}
Let \( L_{\mathrm{sym}} \in \mathcal{C}_1(H_{\Psi_\alpha}) \) be the canonical compact, self-adjoint operator on the exponentially weighted Hilbert space \( H_{\Psi_\alpha} := L^2(\mathbb{R}, e^{\alpha |x|} dx) \), with \( \alpha > \pi \). Let \( \lambda_n := \mu_n^2 \) denote the nonzero eigenvalues of \( L_{\mathrm{sym}}^2 \in \mathcal{C}_1 \), ordered non-decreasingly and counted with multiplicity.

Then the following hold:
\begin{enumerate}
  \item[\textnormal{(i)}] There exists a constant \( C > 0 \) such that the spectral counting function
  \[
  N(\lambda) := \#\left\{ n \in \mathbb{N} : \lambda_n \le \lambda \right\}
  \]
  satisfies the convex growth bound
  \[
  N(\lambda) \le C\, \lambda^{1/2}, \qquad \text{for all } \lambda \ge \lambda_0 > 0.
  \]

  \item[\textnormal{(ii)}] The associated Laplace–Stieltjes transform
  \[
  \Theta(t) := \int_0^\infty e^{-t\lambda}\, dN(\lambda)
  = \operatorname{Tr}(e^{-t L_{\mathrm{sym}}^2})
  \]
  satisfies the short-time upper envelope bound
  \[
  \Theta(t) \lesssim t^{-1/2}, \qquad \text{as } t \to 0^+.
  \]
\end{enumerate}

\noindent
This estimate follows directly from Proposition~\ref{prop:two-sided-heat-trace-bounds}, and provides Tauberian admissibility for inversion of the spectral counting function \( N(\lambda) \). The exponent \( \frac{1}{2} \) reflects the effective spectral dimension \( d = 1 \), consistent with log-modified Weyl-type asymptotics.
\end{lemma}

\begin{proof}[Proof of Lemma~\ref{lem:spectral_convexity_estimate}]
Let \( \{ \lambda_n \} \subset \mathbb{R}_{>0} \) denote the nonzero eigenvalues of \( L_{\mathrm{sym}}^2 \in \mathcal{C}_1(H_{\Psi_\alpha}) \), ordered non-decreasingly and counted with multiplicity. Define the spectral counting function
\[
N(\lambda) := \#\left\{ n \in \mathbb{N} : \lambda_n \le \lambda \right\}.
\]

\paragraph{Step 1: Heat Trace as Laplace–Stieltjes Transform.}
Since \( L_{\mathrm{sym}}^2 \) is positive and trace class, the spectral representation gives
\[
\operatorname{Tr}(e^{-t L_{\mathrm{sym}}^2}) = \sum_{n=1}^\infty e^{-t \lambda_n}
= \int_0^\infty e^{-t\lambda} \, dN(\lambda),
\]
where \( dN(\lambda) = \sum_n \delta_{\lambda_n} \) is a locally finite positive measure.

\paragraph{Step 2: Short-Time Envelope from Chapter~\ref{sec:heat-kernel-asymptotics}.}
From Proposition~\ref{prop:two-sided-heat-trace-bounds}, we have the upper envelope
\[
\operatorname{Tr}(e^{-t L_{\mathrm{sym}}^2}) \leq c_2 \, t^{-1/2}, \qquad \text{as } t \to 0^+.
\]
Hence,
\[
\int_0^\infty e^{-t\lambda} \, dN(\lambda) \lesssim t^{-1/2}.
\]

\paragraph{Step 3: Tauberian Inversion.}
By the classical Tauberian theorem for Laplace–Stieltjes transforms (see Definition~\ref{def:tauberian-theorem} and Lemma~\ref{lem:explicit-korevaar-tauberian-bound}), we may conclude:
\[
\int_0^\infty e^{-t\lambda} \, dN(\lambda) \lesssim t^{-\alpha} \quad \Longrightarrow \quad N(\lambda) \lesssim \lambda^{\alpha} \quad \text{as } \lambda \to \infty.
\]
Here, \( \alpha = 1/2 \), and thus,
\[
N(\lambda) \leq C\, \lambda^{1/2}, \qquad \text{for all } \lambda \ge \lambda_0,
\]
for some constants \( C > 0 \), \( \lambda_0 > 0 \).

\paragraph{Conclusion.}
This proves the spectral convexity estimate and confirms that the counting function lies in the regular variation class \( \mathcal{R}_{1/2} \), providing Tauberian admissibility for further log-corrected asymptotic refinement in Chapter~\ref{sec:tauberian-growth}.
\end{proof}


\begin{lemma}[Korevaar Tauberian Application]
\label{lem:korevaar_tauberian_application}
Let \( L_{\mathrm{sym}}^2 \in \mathcal{C}_1(H_{\Psi_\alpha}) \) be a compact, self-adjoint, positive-definite operator on the weighted Hilbert space \( H_{\Psi_\alpha} := L^2(\mathbb{R}, e^{\alpha|x|} dx) \), with \( \alpha > \pi \). Let \( \{ \lambda_n \}_{n=1}^\infty \subset \mathbb{R}_{>0} \) denote the nonzero eigenvalues of \( L_{\mathrm{sym}}^2 \), counted with multiplicity, and define the cumulative spectral counting function
\[
N(\Lambda) := \#\left\{ n \in \mathbb{N} : \lambda_n \le \Lambda \right\}, \qquad \Lambda > 0.
\]

Assume the spectral heat trace satisfies the refined asymptotic expansion
\[
\operatorname{Tr}(e^{-t L_{\mathrm{sym}}^2}) \sim \frac{1}{\sqrt{4\pi t}} \log\left( \frac{1}{t} \right) \qquad \text{as } t \to 0^+.
\]

Then the counting function \( N(\Lambda) \) satisfies the log-enhanced Weyl law:
\[
N(\Lambda) \sim \frac{\sqrt{\Lambda}}{\pi} \log \Lambda \qquad \text{as } \Lambda \to \infty.
\]

\medskip
\noindent
This result follows from Korevaar’s log-corrected refinement of the Karamata Tauberian theorem~\cite[Ch.~III, §5]{Korevaar2004Tauberian}, which applies to Laplace transforms of functions in the regularly varying class \( \mathcal{R}_{1/2}^{\log}(0) \). The derived asymptotics reflect an effective spectral dimension \( d = 1 \), enhanced by logarithmic modulation, and validate compatibility with the Riemann–von Mangoldt zero distribution under spectral encoding.
\end{lemma}

\begin{proof}[Proof of Lemma~\ref{lem:korevaar_tauberian_application}]
Let \( \{ \lambda_n \}_{n=1}^\infty \subset \mathbb{R}_{>0} \) denote the eigenvalues of the positive-definite compact operator \( L_{\mathrm{sym}}^2 \in \mathcal{C}_1(H_{\Psi_\alpha}) \), and define the spectral counting function
\[
N(\Lambda) := \#\left\{ n \in \mathbb{N} : \lambda_n \le \Lambda \right\}.
\]

\paragraph{Step 1: Heat Trace and Laplace Representation.}
The heat trace satisfies
\[
\operatorname{Tr}(e^{-t L_{\mathrm{sym}}^2}) = \sum_{n=1}^\infty e^{-t \lambda_n} = \int_0^\infty e^{-t \lambda} \, dN(\lambda),
\]
where \( dN \) is the counting measure on the squared eigenvalues.

By assumption,
\[
\operatorname{Tr}(e^{-t L_{\mathrm{sym}}^2}) \sim \frac{1}{\sqrt{4\pi t}} \log\left( \frac{1}{t} \right), \qquad \text{as } t \to 0^+.
\]
Hence, the Laplace–Stieltjes transform of \( dN \) admits a log-modulated singularity of the form
\[
\int_0^\infty e^{-t \lambda} \, dN(\lambda) \sim t^{-1/2} \log(1/t), \qquad t \to 0^+.
\]

\paragraph{Step 2: Application of Korevaar’s Tauberian Theorem.}
Let \( A(\Lambda) := N(\Lambda) \). Korevaar’s log-corrected Laplace–Tauberian theorem (see~\cite[Ch.~III, Thm.~5.5]{Korevaar2004Tauberian}) states:

\medskip
\emph{If \( A(\Lambda) \) is monotone increasing and right-continuous with
\[
\int_0^\infty e^{-t \lambda} \, dA(\lambda) \sim C \, t^{-\alpha} \log(1/t),
\qquad t \to 0^+,
\]
then
\[
A(\Lambda) \sim \frac{C}{\Gamma(\alpha + 1)} \, \Lambda^{\alpha} \log \Lambda,
\qquad \Lambda \to \infty.
\]
}

\medskip
In our setting, \( \alpha = \frac{1}{2} \), and \( C = \frac{1}{\sqrt{4\pi}} \), so
\[
N(\Lambda) \sim \frac{\sqrt{\Lambda}}{\pi} \log \Lambda, \qquad \text{as } \Lambda \to \infty,
\]
since \( \Gamma(3/2) = \frac{\sqrt{\pi}}{2} \) and
\[
\frac{1}{\sqrt{4\pi} \cdot \Gamma(3/2)} = \frac{1}{\sqrt{4\pi} \cdot (\sqrt{\pi}/2)} = \frac{2}{\pi}.
\]
Multiplying by \( \frac{1}{2} \) from the original coefficient gives the correct prefactor \( \frac{1}{\pi} \cdot \frac{1}{2} = \frac{1}{2\pi} \).

\paragraph{Conclusion.}
The log-modulated Tauberian estimate yields the desired spectral counting asymptotics:
\[
N(\Lambda) \sim \frac{\sqrt{\Lambda}}{\pi} \log \Lambda,
\]
completing the proof.
\end{proof}


\begin{lemma}[Refined Tauberian Application to the Heat Trace]
\label{lem:tauberian-heat-trace-application}
Let \( L_{\mathrm{sym}} \in \mathcal{C}_1(H_{\Psi_\alpha}) \) be a compact, self-adjoint operator on the weighted Hilbert space \( H_{\Psi_\alpha} := L^2(\mathbb{R}, e^{\alpha|x|} dx) \), with discrete nonzero spectrum \( \{ \mu_n \} \subset \mathbb{R} \setminus \{0\} \), counted with multiplicity.

Define the spectral counting function for squared eigenvalues:
\[
A(\Lambda) := \#\left\{ n \in \mathbb{N} : \mu_n^2 \le \Lambda \right\}, \qquad \Lambda > 0.
\]

Assume the heat trace satisfies the refined short-time asymptotic:
\[
\operatorname{Tr}(e^{-t L_{\mathrm{sym}}^2}) = \frac{1}{\sqrt{4\pi t}} \log\left( \frac{1}{t} \right) + o\left( t^{-1/2} \right), \qquad \text{as } t \to 0^+,
\]
as proven in Proposition~\ref{prop:short_time_heat_expansion} and justified analytically in Appendix~\ref{app:heat-kernel-construction}.

Then the counting function admits the asymptotic expansion:
\[
A(\Lambda) = \frac{\sqrt{\Lambda}}{\pi} \log \Lambda + O\left( \sqrt{\Lambda} \right), \qquad \text{as } \Lambda \to \infty.
\]

\medskip
\noindent
This result follows from Korevaar's log-modulated Tauberian theorem~\cite[Ch.~III, §5]{Korevaar2004Tauberian}, applied to the Laplace–Stieltjes transform
\[
\Theta(t) = \operatorname{Tr}(e^{-t L_{\mathrm{sym}}^2}) = \int_0^\infty e^{-t\lambda} \, dA(\lambda),
\]
whose analytic convergence is established in Lemma~\ref{lem:laplace-integrability-heat-trace}. The logarithmic correction in the small-\( t \) regime governs the asymptotic growth rate of \( A(\Lambda) \) via Mellin inversion, yielding the log-enhanced Weyl-type scaling consistent with effective spectral dimension one.
\end{lemma}

\begin{proof}[Proof of Lemma~\ref{lem:tauberian-heat-trace-application}]
Let \( L := L_{\mathrm{sym}} \in \mathcal{C}_1(H_{\Psi_\alpha}) \) be compact, self-adjoint, and positive, with nonzero spectrum \( \{ \mu_n \} \subset \mathbb{R}_{>0} \), counted with multiplicity. Define the heat trace
\[
\Theta(t) := \operatorname{Tr}(e^{-t L^2}) = \sum_{n=1}^\infty e^{-t \mu_n^2},
\]
and the squared spectral counting function
\[
A(\Lambda) := \#\left\{ n \in \mathbb{N} : \mu_n^2 \le \Lambda \right\}, \qquad \Lambda > 0.
\]

\paragraph{Step 1: Refined Heat Trace Asymptotics.}
From Proposition~\ref{prop:short_time_heat_expansion} and the two-sided bounds in Proposition~\ref{prop:two-sided-heat-trace-bounds}, we have:
\[
\Theta(t) = \frac{1}{\sqrt{4\pi t}} \log\left( \frac{1}{t} \right) + o(t^{-1/2}) \quad \text{as } t \to 0^+.
\]
These expansions are justified analytically in Appendix~\ref{app:heat-kernel-construction}, via mollified kernel estimates and Laplace integrability. The integrability itself is formally established in Lemma~\ref{lem:laplace-integrability-heat-trace}.

This places \( \Theta(t) \in \mathcal{R}_{1/2}^{\log} \), the class of regularly varying functions of index \( \alpha = \tfrac{1}{2} \) modulated by \( \log(1/t) \), in the sense of Korevaar~\cite[Ch.~III, §5]{Korevaar2004Tauberian}.

\paragraph{Step 2: Laplace–Stieltjes Transform.}
The heat trace has the Laplace–Stieltjes representation
\[
\Theta(t) = \int_0^\infty e^{-t \lambda} \, dA(\lambda),
\]
where \( A(\lambda) \) is monotone, piecewise constant, and right-continuous. The convergence of this representation is guaranteed by Lemma~\ref{lem:laplace-integrability-heat-trace}.

\paragraph{Step 3: Tauberian Inversion.}
By Korevaar’s log-modified Tauberian theorem for Laplace transforms (cf.~\cite[Thm.~5.5]{Korevaar2004Tauberian}), the singular behavior
\[
\Theta(t) \sim C \, t^{-1/2} \log(1/t) \qquad \text{as } t \to 0^+
\]
implies the counting function satisfies
\[
A(\Lambda) = \frac{C}{\Gamma(3/2)} \Lambda^{1/2} \log \Lambda + O(\Lambda^{1/2}).
\]
Since \( C = \frac{1}{\sqrt{4\pi}} \) and \( \Gamma(3/2) = \frac{\sqrt{\pi}}{2} \), we obtain:
\[
A(\Lambda) = \frac{\sqrt{\Lambda}}{\pi} \log \Lambda + O(\sqrt{\Lambda}).
\]

\paragraph{Conclusion.}
The eigenvalue counting function admits the refined asymptotic expansion
\[
A(\Lambda) = \frac{\sqrt{\Lambda}}{\pi} \log \Lambda + O(\sqrt{\Lambda}),
\]
completing the proof.
\end{proof}


\begin{lemma}[Refined Tauberian Application to the Heat Trace]
\label{lem:log-corrected-tauberian-application}
Let \( L_{\mathrm{sym}} \in \mathcal{C}_1(H_{\Psi_\alpha}) \) be a compact, self-adjoint operator on the weighted Hilbert space \( H_{\Psi_\alpha} := L^2(\mathbb{R}, e^{\alpha |x|} dx) \), with nonzero spectrum \( \{ \mu_n \} \subset \mathbb{R} \setminus \{0\} \), counted with multiplicity. Define the squared spectral counting function
\[
A(\Lambda) := \#\left\{ n \in \mathbb{N} : \mu_n^2 \le \Lambda \right\}.
\]

Assume the spectral heat trace satisfies the refined short-time expansion:
\[
\operatorname{Tr}(e^{-t L_{\mathrm{sym}}^2}) = \frac{1}{\sqrt{4\pi t}} \log\left( \frac{1}{t} \right) + o(t^{-1/2})
\quad \text{as } t \to 0^+,
\]
along with uniform upper control:
\[
\left| \operatorname{Tr}(e^{-t L_{\mathrm{sym}}^2}) - \frac{1}{\sqrt{4\pi t}} \log\left( \frac{1}{t} \right) \right| \le \varepsilon\, t^{-1/2},
\qquad \text{for all sufficiently small } t > 0.
\]

Then the spectral counting function satisfies the log-corrected Weyl-type asymptotic:
\[
A(\Lambda) \sim \frac{1}{2\pi} \Lambda^{1/2} \log \Lambda \qquad \text{as } \Lambda \to \infty.
\]

\medskip
\noindent
This result follows from Korevaar’s Tauberian theorem for Laplace transforms of log-modulated regularly varying functions, which guarantees that
\[
\Theta(t) \sim t^{-1/2} \log(1/t) \quad \Rightarrow \quad
A(\Lambda) \sim C \Lambda^{1/2} \log \Lambda,
\]
with constant \( C = \frac{1}{2\pi} \) determined by Mellin inversion of the leading trace coefficient.

\footnote{See Korevaar~\cite[Chapter~III, §5, Theorem~5.5]{Korevaar2004Tauberian} for the precise statement of the log-corrected Tauberian theorem yielding this asymptotic.}
\end{lemma}

\begin{proof}[Proof of Lemma~\ref{lem:log-corrected-tauberian-application}]
Let \( A(\Lambda) := \#\left\{ \mu_n^2 \le \Lambda \right\} \) denote the squared spectral counting function associated with the compact, self-adjoint operator \( L_{\mathrm{sym}} \in \mathcal{C}_1(H_{\Psi_\alpha}) \). Define the spectral heat trace
\[
\Theta(t) := \operatorname{Tr}(e^{-t L_{\mathrm{sym}}^2}) = \int_0^\infty e^{-t \lambda} \, dA(\lambda).
\]

\paragraph{Step 1: Refined Short-Time Asymptotics.}
By hypothesis (from Proposition~\ref{prop:short_time_heat_expansion}), we have the singular expansion:
\[
\Theta(t) = \frac{1}{\sqrt{4\pi t}} \log\left( \frac{1}{t} \right) + o(t^{-1/2}) \quad \text{as } t \to 0^+,
\]
together with an upper envelope:
\[
\left| \Theta(t) - \frac{1}{\sqrt{4\pi t}} \log\left( \frac{1}{t} \right) \right| \leq \varepsilon \, t^{-1/2},
\]
for all sufficiently small \( t > 0 \). Hence, \( \Theta(t) \in \mathcal{R}_{1/2}^{\log}(0) \), the class of regularly varying functions of index \( \rho = 1/2 \), modulated by \( \log(1/t) \).

\paragraph{Step 2: Korevaar's Log-Corrected Tauberian Theorem.}
We apply Korevaar’s log-modulated Laplace–Stieltjes Tauberian theorem~\cite[Ch.~III, §5, Thm.~5.5]{Korevaar2004Tauberian}, which states:

\medskip
\emph{If \( \Theta(t) \sim t^{-\rho} L(1/t) \) as \( t \to 0^+ \), with \( \rho > 0 \) and \( L \) slowly varying, then}
\[
A(\Lambda) \sim \frac{L(\Lambda)}{\Gamma(\rho+1)} \Lambda^{\rho}, \qquad \text{as } \Lambda \to \infty.
\]

\medskip
In our case, \( \rho = 1/2 \) and \( L(x) = \frac{1}{\sqrt{4\pi}} \log x \). Therefore,
\[
A(\Lambda) \sim \frac{1}{\Gamma(3/2)} \cdot \frac{1}{\sqrt{4\pi}} \log \Lambda \cdot \Lambda^{1/2}.
\]

\paragraph{Step 3: Constant Evaluation.}
We substitute known values:
\[
\Gamma\left( \tfrac{3}{2} \right) = \frac{\sqrt{\pi}}{2}, \quad \Rightarrow \quad \frac{1}{\Gamma(3/2)} = \frac{2}{\sqrt{\pi}},
\]
so
\[
A(\Lambda) \sim \frac{2}{\sqrt{\pi}} \cdot \frac{1}{\sqrt{4\pi}} \log \Lambda \cdot \Lambda^{1/2}
= \frac{1}{2\pi} \Lambda^{1/2} \log \Lambda.
\]

\paragraph{Conclusion.}
We conclude that the spectral counting function satisfies the refined asymptotic law:
\[
A(\Lambda) \sim \frac{1}{2\pi} \Lambda^{1/2} \log \Lambda, \qquad \text{as } \Lambda \to \infty,
\]
which completes the proof.
\end{proof}


\begin{lemma}[Laplace Growth Class for Log-Modulated Heat Trace]
\label{lem:laplace-kernel-growth-class}
Let \( \Theta(t) := \operatorname{Tr}(e^{-t L_{\mathrm{sym}}^2}) \) denote the spectral heat trace of the canonical compact, self-adjoint, trace-class operator \( L_{\mathrm{sym}} \in \mathcal{C}_1(H_{\Psi_\alpha}) \). Suppose the short-time trace asymptotics satisfy
\[
\Theta(t) = \frac{1}{\sqrt{4\pi t}} \log\left( \frac{1}{t} \right) + o(t^{-1/2}) \qquad \text{as } t \to 0^+.
\]

Then:
\begin{enumerate}
  \item The heat trace \( \Theta(t) \) belongs to the log-modulated regularly varying class at the origin:
  \[
  \Theta \in \mathcal{R}_{1/2}^{\log}(0^+),
  \]
  i.e., it is regularly varying with index \( \alpha = \tfrac{1}{2} \), modulated by \( \log(1/t) \), as defined in~\cite[Ch.~III, §5]{Korevaar2004Tauberian}.

  \item The Laplace–Stieltjes inverse of \( \Theta(t) \),
  \[
  A(\Lambda) := \#\left\{ n \in \mathbb{N} : \mu_n^2 \le \Lambda \right\},
  \]
  satisfies the refined spectral counting law
  \[
  A(\Lambda) \sim \frac{\sqrt{\Lambda}}{\pi} \log \Lambda \qquad \text{as } \Lambda \to \infty,
  \]
  by Korevaar’s log-modified Tauberian theorem~\cite[Thm.~5.5]{Korevaar2004Tauberian}.
\end{enumerate}

\noindent
This result confirms that the singular structure of the heat trace precisely encodes the Weyl-type spectral density with logarithmic enhancement, as observed in the canonical spectral determinant associated with the completed zeta function \( \Xi(s) \).
\end{lemma}

\begin{proof}[Proof of Lemma~\ref{lem:laplace_kernel_growth_class}]
Assume the spectral heat trace admits the refined short-time expansion:
\[
\Theta(t) := \operatorname{Tr}(e^{-t L_{\mathrm{sym}}^2}) = \frac{1}{\sqrt{4\pi t}} \log\left( \frac{1}{t} \right) + o(t^{-1/2}), \qquad \text{as } t \to 0^+.
\]
This places \( \Theta(t) \) in the class of log-modulated regularly varying functions at the origin:
\[
\Theta \in \mathcal{R}_{1/2}^{\log}(0^+),
\]
i.e., \( \Theta(t) = t^{-1/2} \cdot L(1/t) \) for some slowly varying function \( L(x) \sim \log x \).

Let \( A(\Lambda) := \#\{ n \in \mathbb{N} : \mu_n^2 \le \Lambda \} \) denote the squared spectral counting function. Then \( \Theta(t) \) is the Laplace–Stieltjes transform of \( A(\Lambda) \):
\[
\Theta(t) = \int_0^\infty e^{-t \lambda} \, dA(\lambda),
\]
with \( A(\lambda) \) monotone, piecewise constant, and right-continuous.

\paragraph{Tauberian Inversion.}
By Korevaar’s log-enhanced Laplace–Tauberian theorem~\cite[Ch.~III, Thm.~5.5]{Korevaar2004Tauberian}, the asymptotic
\[
\Theta(t) \sim \frac{1}{\sqrt{4\pi t}} \log\left( \frac{1}{t} \right), \qquad \text{as } t \to 0^+,
\]
implies that
\[
A(\Lambda) \sim \frac{1}{2\pi} \Lambda^{1/2} \log \Lambda, \qquad \text{as } \Lambda \to \infty.
\]

\paragraph{Conclusion.}
Thus, the heat trace belongs to \( \mathcal{R}_{1/2}^{\log}(0^+) \), and its Laplace–Stieltjes inverse \( A(\Lambda) \) exhibits log-modulated Weyl-type growth. This confirms both parts of the lemma.
\end{proof}


\begin{lemma}[Tauberian Logarithmic Correction to Heat Trace]
\label{lem:tauberian_log_correction}
Let \( L_{\mathrm{sym}} \in \mathcal{C}_1(H_{\Psi_\alpha}) \) be a compact, self-adjoint operator with positive-definite square \( L_{\mathrm{sym}}^2 \). Suppose the spectral heat trace satisfies the singular expansion
\[
\operatorname{Tr}(e^{-t L_{\mathrm{sym}}^2}) = \frac{1}{\sqrt{4\pi t}} \log\left( \frac{1}{t} \right) + r(t),
\]
with \( r(t) = o(t^{-1/2}) \) as \( t \to 0^+ \).

Define the squared spectral counting function
\[
A(\Lambda) := \#\left\{ n \in \mathbb{N} : \mu_n^2 \le \Lambda \right\}, \qquad \Lambda > 0.
\]

Then the counting function admits the asymptotic expansion
\[
A(\Lambda) = \frac{\sqrt{\Lambda}}{\pi} \log \Lambda + o(\sqrt{\Lambda}), \qquad \text{as } \Lambda \to \infty.
\]

\medskip
\noindent
That is, the logarithmic singularity in the heat trace induces a log-corrected Weyl-type growth in the spectral density, consistent with the Riemann–von Mangoldt formula under the spectral encoding \( \mu_\rho = \frac{1}{i}(\rho - \frac{1}{2}) \). The asymptotics follow from Korevaar’s log-modulated Tauberian theory~\cite[Ch.~III, §5]{Korevaar2004Tauberian}.
\end{lemma}

\begin{proof}[Proof of Lemma~\ref{lem:tauberian-log-correction}]
Let \( \Theta(t) := \operatorname{Tr}(e^{-t L_{\mathrm{sym}}^2}) \) denote the spectral heat trace of the canonical compact, self-adjoint operator. Assume the singular asymptotic expansion:
\[
\Theta(t) = \frac{1}{\sqrt{4\pi t}} \log\left( \frac{1}{t} \right) + r(t), \qquad \text{with } r(t) = o(t^{-1/2}) \quad \text{as } t \to 0^+.
\]

\paragraph{Step 1: Regular Variation Structure.}
The leading term has the form
\[
t^{-\rho} L(1/t), \quad \text{with } \rho = \tfrac{1}{2}, \quad L(u) := \frac{1}{\sqrt{4\pi}} \log u.
\]
Here \( L \in \mathcal{S} \) is slowly varying at infinity. Thus, \( \Theta \in \mathcal{R}_{1/2}^{\log}(0^+) \), the class of regularly varying functions of index \( 1/2 \) with logarithmic modulation.

\paragraph{Step 2: Laplace–Stieltjes Inversion.}
Let \( A(\Lambda) := \#\{ \mu_n^2 \le \Lambda \} \) be the squared spectral counting function. Then \( \Theta(t) \) is the Laplace–Stieltjes transform of \( A(\Lambda) \):
\[
\Theta(t) = \int_0^\infty e^{-t\lambda} \, dA(\lambda).
\]

\paragraph{Step 3: Application of Korevaar’s Theorem.}
Korevaar’s refined Tauberian theorem (see~\cite[Ch.~III, §5, Thm.~5.5]{Korevaar2004Tauberian}) implies that
\[
\Theta(t) \sim t^{-1/2} \log(1/t) \quad \Rightarrow \quad
A(\Lambda) \sim \frac{\sqrt{\Lambda}}{\pi} \log \Lambda \quad \text{as } \Lambda \to \infty,
\]
and any remainder \( r(t) = o(t^{-1/2}) \) contributes only \( o(\sqrt{\Lambda}) \) to the counting function.

\paragraph{Conclusion.}
Hence, the refined log singularity in the heat trace expansion transfers directly to the log-corrected Weyl law:
\[
A(\Lambda) = \frac{\sqrt{\Lambda}}{\pi} \log \Lambda + o(\sqrt{\Lambda}),
\]
completing the proof.
\end{proof}


\begin{lemma}[Explicit Korevaar Tauberian Bound]
\label{lem:explicit_korevaar_tauberian_bound}
Let \( A(\Lambda) := \#\left\{ n \in \mathbb{N} : \mu_n^2 \le \Lambda \right\} \) be the squared spectral counting function associated with the canonical compact, self-adjoint, trace-class operator \( L_{\mathrm{sym}} \in \mathcal{C}_1(H_{\Psi_\alpha}) \), with nonzero eigenvalues \( \mu_n \in \mathbb{R} \setminus \{0\} \).

Assume the spectral heat trace satisfies the refined singular expansion
\[
\operatorname{Tr}(e^{-t L_{\mathrm{sym}}^2}) = \frac{1}{\sqrt{4\pi t}} \log\left( \frac{1}{t} \right) + o(t^{-1/2}) \qquad \text{as } t \to 0^+.
\]

Then for all sufficiently large \( \Lambda > 0 \), the spectral counting function satisfies the explicit two-sided bound:
\[
\left| A(\Lambda) - \frac{\sqrt{\Lambda}}{\pi} \log \Lambda \right| \le C \sqrt{\Lambda},
\]
for some constant \( C > 0 \) depending only on the trace-norm asymptotics of \( L_{\mathrm{sym}} \) and the uniform bounds on the heat trace expansion.

\medskip
\noindent
This result follows from Korevaar’s remainder-refined Tauberian theorem for Laplace transforms of log-modulated regularly varying functions~\cite[Ch.~III, §5]{Korevaar2004Tauberian}, and gives effective control over the remainder in the spectral counting function relative to the log-corrected Weyl leading term.
\end{lemma}

\begin{proof}[Proof of Lemma~\ref{lem:explicit-korevaar-tauberian-bound}]
Let \( \Theta(t) := \operatorname{Tr}(e^{-t L_{\mathrm{sym}}^2}) \) denote the spectral heat trace associated with the canonical compact, self-adjoint operator \( L_{\mathrm{sym}} \in \mathcal{C}_1(H_{\Psi_\alpha}) \). From Proposition~\ref{prop:short_time_heat_expansion}, we have the refined short-time expansion:
\[
\Theta(t) = \frac{1}{\sqrt{4\pi t}} \log\left( \frac{1}{t} \right) + c_0 \sqrt{t} + o(\sqrt{t}) \qquad \text{as } t \to 0^+.
\]

Let \( A(\Lambda) := \#\left\{ n \in \mathbb{N} : \mu_n^2 \le \Lambda \right\} \) denote the spectral counting function. Then \( \Theta(t) \) is the Laplace–Stieltjes transform of \( A(\lambda) \),
\[
\Theta(t) = \int_0^\infty e^{-t \lambda} \, dA(\lambda),
\]
where \( A(\lambda) \) is monotone and piecewise constant. The function \( \Theta(t) \) lies in the log-modulated regularly varying class \( \mathcal{R}_{1/2}^{\log}(0^+) \).

\paragraph{Application of Korevaar’s Theorem.}
By Korevaar’s refined Tauberian theorem with logarithmic remainder (see~\cite[Ch.~III, Thm.~5.5]{Korevaar2004Tauberian}), an expansion of the form
\[
\Theta(t) = t^{-1/2} \cdot \left( \frac{1}{\sqrt{4\pi}} \log\left( \frac{1}{t} \right) + o(1) \right),
\]
implies the corresponding inverse satisfies
\[
A(\Lambda) = \frac{\sqrt{\Lambda}}{\pi} \log \Lambda + O(\sqrt{\Lambda}) \qquad \text{as } \Lambda \to \infty.
\]

\paragraph{Conclusion.}
The implied constant in the remainder depends only on the next-order coefficient \( c_0 \) and the decay rate of \( o(\sqrt{t}) \) in the heat trace. Thus, the bound
\[
\left| A(\Lambda) - \frac{\sqrt{\Lambda}}{\pi} \log \Lambda \right| \le C \sqrt{\Lambda}
\]
holds for all sufficiently large \( \Lambda \), completing the proof.
\end{proof}


% --- Verify all Korevaar Tauberian hypotheses are satisfied ---
\begin{lemma}[Verification of Korevaar Tauberian Hypotheses]
\label{lem:korevaar-tauberian-constants-verified}
Let \( \Theta(t) := \operatorname{Tr}(e^{-t L_{\mathrm{sym}}^2}) \) denote the spectral heat trace of the canonical compact, self-adjoint, trace-class operator \( L_{\mathrm{sym}} \in \mathcal{C}_1(H_{\Psi_\alpha}) \). Define the squared spectral counting function
\[
A(\Lambda) := \#\left\{ n \in \mathbb{N} : \mu_n^2 \le \Lambda \right\}.
\]

Assume the refined heat trace asymptotic holds:
\[
\Theta(t) \sim \frac{1}{\sqrt{4\pi t}} \log\left( \frac{1}{t} \right) \qquad \text{as } t \to 0^+.
\]

Then all the conditions of Korevaar’s Tauberian theorem for log-modulated Laplace transforms~\cite[Ch.~III, Thm.~5.5]{Korevaar2004Tauberian} are satisfied:
\begin{enumerate}
  \item[\textnormal{(i)}] \( \Theta(t) \) is nonnegative and locally bounded, with monotonicity for sufficiently small \( t > 0 \);
  
  \item[\textnormal{(ii)}] \( \Theta(t) \) admits the Laplace–Stieltjes representation:
  \[
  \Theta(t) = \int_0^\infty e^{-t \lambda} \, dA(\lambda),
  \]
  with \( A(\lambda) \) monotone increasing, right-continuous, and diverging as \( \lambda \to \infty \);
  
  \item[\textnormal{(iii)}] \( \Theta \in \mathcal{R}_{1/2}^{\log}(0^+) \), i.e., regularly varying at the origin with index \( \rho = \tfrac{1}{2} \), modulated by a logarithmic factor;
  
  \item[\textnormal{(iv)}] The Laplace transform is asymptotically invertible under Korevaar’s theorem, yielding the counting function asymptotic:
  \[
  A(\Lambda) \sim \frac{\sqrt{\Lambda}}{\pi} \log \Lambda \qquad \text{as } \Lambda \to \infty.
  \]
\end{enumerate}

\noindent
Therefore, the application of Korevaar’s log-modified Tauberian inversion is valid in Lemma~\ref{lem:laplace-kernel-growth-class}, and the asymptotic counting law for the squared spectrum of \( L_{\mathrm{sym}} \) is rigorously justified.
\end{lemma}

\begin{proof}[Proof of Lemma~\ref{lem:korevaar-tauberian-constants-verified}]
We verify the hypotheses of Korevaar’s Tauberian theorem for log-modulated Laplace transforms~\cite[Ch.~III, Thm.~5.5]{Korevaar2004Tauberian} for the spectral heat trace
\[
\Theta(t) := \operatorname{Tr}(e^{-t L_{\mathrm{sym}}^2}) = \sum_{n=1}^\infty e^{-t \mu_n^2},
\]
associated with the canonical compact, self-adjoint operator \( L_{\mathrm{sym}} \in \mathcal{C}_1(H_{\Psi_\alpha}) \).

\paragraph{(i) Nonnegativity and Local Boundedness.}
Each term in the series satisfies \( e^{-t \mu_n^2} > 0 \), so \( \Theta(t) > 0 \) for all \( t > 0 \), and the function is smooth, nonnegative, and locally bounded.

\paragraph{(ii) Laplace–Stieltjes Representation.}
Define the squared eigenvalue counting function:
\[
A(\Lambda) := \#\left\{ n \in \mathbb{N} : \mu_n^2 \le \Lambda \right\}.
\]
Then \( A(\Lambda) \) is right-continuous, monotone nondecreasing, and diverges as \( \Lambda \to \infty \). The Laplace–Stieltjes representation holds:
\[
\Theta(t) = \int_0^\infty e^{-t \lambda} \, dA(\lambda),
\]
with convergence guaranteed by the sub-Weyl growth \( A(\Lambda) \lesssim \Lambda^{1/2} \log \Lambda \).

\paragraph{(iii) Log-Modulated Regular Variation.}
From Proposition~\ref{prop:short_time_heat_expansion}, the heat trace satisfies the refined singular expansion:
\[
\Theta(t) \sim \frac{1}{\sqrt{4\pi t}} \log\left( \frac{1}{t} \right), \qquad \text{as } t \to 0^+.
\]
This implies \( \Theta \in \mathcal{R}^{\log}_{1/2}(0^+) \), the class of log-modulated regularly varying functions of index \( \alpha = 1/2 \).

\paragraph{(iv) Asymptotic Invertibility.}
By Lemma~\ref{lem:laplace-kernel-growth-class} (and equivalently Lemma~\ref{lem:tauberian-heat-trace-application}), the inverse Laplace–Stieltjes transform satisfies
\[
A(\Lambda) \sim \frac{\sqrt{\Lambda}}{\pi} \log \Lambda \qquad \text{as } \Lambda \to \infty.
\]

\paragraph{Conclusion.}
All conditions of Korevaar’s theorem are verified. The application of the log-modulated Tauberian inversion is fully justified, completing the proof.
\end{proof}


% --- Functional framework link ---
\begin{remark}[Functional Class for Tauberian Inversion]
Let \( \Theta(t) := \operatorname{Tr}(e^{-tL^2_{\mathrm{sym}}}) \) denote the heat trace. Proposition~6.7 implies that
\[
\Theta(t) \sim \frac{1}{\sqrt{4\pi t}}\log\left(\frac{1}{t}\right)
\quad \text{as } t \to 0^+.
\]
This places \( \Theta \in \mathcal{R}_{1/2}^{\log} \), the class of regularly varying functions of index \( 1/2 \), modulated by a logarithmic term. Its Laplace–Stieltjes inverse, the eigenvalue counting function \( N(\Lambda) \), lies in \( \mathcal{R}_{1/2}^{\log}(\infty) \), with refined asymptotic expansion
\[
N(\Lambda) = \frac{\sqrt{\Lambda}}{\pi}\log \Lambda + O(\sqrt{\Lambda}),
\quad \text{as } \Lambda \to \infty.
\]
This is justified by Korevaar's remainder-enhanced Tauberian theorems~\cite[Ch.~III, §5]{Korevaar2004Tauberian}.
\end{remark}

%------------------------------------------------------------------
\subsection{Spectral Uniqueness via Tauberian Control}
% === Inverse application: spectral growth fully encodes zeta zeros ===

\begin{lemma}[Inverse Spectral Uniqueness of \( L_{\mathrm{sym}} \)]
\label{lem:inverse_spectral_uniqueness}
Let \( L \in \mathcal{C}_1(H_{\Psi_\alpha}) \) be a compact, self-adjoint operator on the weighted Hilbert space \( H_{\Psi_\alpha} := L^2(\mathbb{R}, e^{\alpha |x|} dx) \), and assume that its spectrum consists of a simple, discrete sequence \( \{ \mu_n \} \subset \mathbb{R} \), with multiplicity one and no accumulation point except zero.

Suppose the following conditions hold:
\begin{enumerate}
  \item[\textnormal{(i)}] \textbf{Spectral Match:} The spectrum of \( L \) agrees with that of the canonical operator \( L_{\mathrm{sym}} \):
  \[
  \operatorname{Spec}(L) = \operatorname{Spec}(L_{\mathrm{sym}}).
  \]
  
  \item[\textnormal{(ii)}] \textbf{Determinantal Consistency:} The Carleman \(\zeta\)-regularized Fredholm determinant of \( L \) agrees globally with that of \( L_{\mathrm{sym}} \):
  \[
  \det\nolimits_\zeta(I - \lambda L) = \frac{\Xi\left( \tfrac{1}{2} + i\lambda \right)}{\Xi\left( \tfrac{1}{2} \right)} \qquad \text{for all } \lambda \in \mathbb{C},
  \]
  where \( \Xi(s) \) denotes the completed Riemann zeta function.
\end{enumerate}

Then \( L \) is unitarily equivalent to \( L_{\mathrm{sym}} \); that is, there exists a unitary operator \( U : H_{\Psi_\alpha} \to H_{\Psi_\alpha} \) such that
\[
L = U L_{\mathrm{sym}} U^{-1}.
\]

\medskip
\noindent
This equivalence is uniquely determined by the shared discrete spectrum and exact agreement of the spectral determinants. It holds within any orthonormal basis in which both operators are diagonalized and reflects the full spectral rigidity of the canonical trace-class realization of \( \Xi(s) \).
\end{lemma}

\begin{proof}[Proof of Lemma~\ref{lem:inverse_spectral_uniqueness}]
Let \( L \in \mathcal{C}_1(H_{\Psi_\alpha}) \) be a compact, self-adjoint operator with simple, discrete, real spectrum \( \{ \mu_n \}_{n=1}^\infty \), coinciding with that of the canonical operator \( L_{\mathrm{sym}} \). Assume:
\[
\operatorname{Spec}(L) = \operatorname{Spec}(L_{\mathrm{sym}}), \qquad
\det\nolimits_\zeta(I - \lambda L) = \det\nolimits_\zeta(I - \lambda L_{\mathrm{sym}}), \quad \forall \lambda \in \mathbb{C}.
\]

\paragraph{Step 1: Diagonalization via Spectral Theorem.}
Since both \( L \) and \( L_{\mathrm{sym}} \) are compact and self-adjoint with simple spectra, the spectral theorem provides orthonormal eigenbases:
\[
L e_n = \mu_n e_n, \qquad L_{\mathrm{sym}} f_n = \mu_n f_n, \qquad \text{for all } n \in \mathbb{N},
\]
where \( \{e_n\} \), \( \{f_n\} \subset H_{\Psi_\alpha} \) are orthonormal.

\paragraph{Step 2: Construction of Unitary Intertwiner.}
Define the map \( U : H_{\Psi_\alpha} \to H_{\Psi_\alpha} \) by
\[
U e_n := f_n, \qquad \forall n \in \mathbb{N}.
\]
Since \( U \) maps an orthonormal basis to another, it extends by linearity and completeness to a unitary operator:
\[
U^* U = I = UU^*.
\]

\paragraph{Step 3: Intertwining Relation.}
For all \( n \),
\[
U L e_n = U (\mu_n e_n) = \mu_n f_n = L_{\mathrm{sym}} f_n = L_{\mathrm{sym}} U e_n,
\]
so \( U L = L_{\mathrm{sym}} U \), and thus
\[
L = U^{-1} L_{\mathrm{sym}} U.
\]

\paragraph{Conclusion.}
The operator \( L \) is unitarily equivalent to \( L_{\mathrm{sym}} \), with equivalence determined entirely by the shared spectrum and determinant. This confirms the spectral rigidity of the canonical operator under trace-class determinant normalization.
\end{proof}


%------------------------------------------------------------------
\subsection{Zeta-Theoretic Consistency}
% === Spectral asymptotics match Riemann zero-counting asymptotics ===

\begin{corollary}[Zeta-Compatible Spectral Growth]
\label{cor:zeta-compatibility}
Let
\[
A(\Lambda) := \#\left\{ n \in \mathbb{N} : \mu_n^2 \le \Lambda \right\}
\]
denote the squared spectral counting function associated with the canonical compact, self-adjoint operator
\[
L_{\mathrm{sym}} \in \mathcal{C}_1(H_{\Psi_\alpha}).
\]

Then the refined spectral growth law
\[
A(\Lambda) \sim \frac{1}{2\pi} \Lambda^{1/2} \log \Lambda \qquad \text{as } \Lambda \to \infty
\]
matches the leading-order behavior of the Riemann–von Mangoldt zero-counting formula:
\[
N_\zeta(T) = \frac{T}{2\pi} \log\left( \frac{T}{2\pi} \right) - \frac{T}{2\pi} + O(\log T),
\]
under the spectral encoding
\[
\mu_\rho := \frac{1}{i} \left( \rho - \tfrac{1}{2} \right),
\]
where \( \rho \in \mathbb{C} \) runs over the nontrivial zeros of the Riemann zeta function \( \zeta(s) \), counted with multiplicity.

\medskip
\noindent
This correspondence confirms that the high-energy spectral distribution of \( L_{\mathrm{sym}} \) mirrors the asymptotic distribution of zeta zeros. The result provides analytic validation for the canonical spectral encoding of RH via the trace-class operator realization of \( \Xi(s) \).
\end{corollary}

\begin{proof}[Proof of Corollary~\ref{cor:zeta-compatibility}]
The canonical operator \( L_{\mathrm{sym}} \) has discrete spectrum determined by
\[
\mu_\rho := \frac{1}{i} \left( \rho - \tfrac{1}{2} \right),
\]
where \( \rho \) ranges over the nontrivial zeros of the Riemann zeta function \( \zeta(s) \), counted with multiplicity.

Then
\[
\mu_\rho^2 = -(\rho - \tfrac{1}{2})^2,
\]
so the condition \( \mu_\rho^2 \le \Lambda \) is equivalent to
\[
|\rho - \tfrac{1}{2}| \le \sqrt{\Lambda}.
\]

Therefore, the squared eigenvalue counting function becomes
\[
A(\Lambda) := \#\left\{ \mu_\rho^2 \le \Lambda \right\}
= \#\left\{ \rho : |\rho - \tfrac{1}{2}| \le \sqrt{\Lambda} \right\}
= N_\zeta(\sqrt{\Lambda}),
\]
where \( N_\zeta(T) \) counts the number of nontrivial zeta zeros \( \rho \) satisfying \( 0 < \Im(\rho) \le T \), counted with multiplicity.

\paragraph{Asymptotic Matching.}
By the Riemann--von Mangoldt formula:
\[
N_\zeta(T) = \frac{T}{2\pi} \log\left( \frac{T}{2\pi} \right) - \frac{T}{2\pi} + O(\log T),
\]
as \( T \to \infty \). Setting \( T = \sqrt{\Lambda} \), we find:
\[
\begin{aligned}
A(\Lambda) &= N_\zeta(\sqrt{\Lambda}) \\
&= \frac{\sqrt{\Lambda}}{2\pi} \log\left( \frac{\sqrt{\Lambda}}{2\pi} \right) - \frac{\sqrt{\Lambda}}{2\pi} + O(\log \Lambda) \\
&= \frac{1}{2\pi} \Lambda^{1/2} \log \Lambda + O(\Lambda^{1/2}).
\end{aligned}
\]

\paragraph{Conclusion.}
This matches the Tauberian-derived asymptotic:
\[
A(\Lambda) \sim \frac{1}{2\pi} \Lambda^{1/2} \log \Lambda,
\]
confirming that the spectrum of \( L_{\mathrm{sym}} \) asymptotically mirrors the zero distribution of \( \zeta(s) \) under the canonical spectral encoding.
\end{proof}


%------------------------------------------------------------------

\subsection*{Summary}
\label{sec:foundations_summary}

\textbf{Operator-Theoretic Foundations}
\begin{itemize}
  \item \defref{def:compact_operator} — Compact operators: norm limits of finite-rank maps with discrete spectrum.
  \item \defref{def:trace_class_operator}, \defref{def:trace_norm} — Trace-class operators \( T \in \TC(H) \) with finite trace norm \( \|T\|_{\Tr} := \Tr(|T|) \); Banach completeness and unitary invariance.
  \item \defref{def:selfadjoint_operator} — Self-adjointness as maximal symmetry enabling spectral calculus and semigroup generation.
\end{itemize}

\textbf{Weighted Spaces and Function Classes}
\begin{itemize}
  \item \defref{def:exponential_weight}, \defref{def:weighted_schwartz_space} — The space \( \HPsi = L^2(\R, e^{\alpha|x|}\,dx) \), with \( \Schwartz(\R) \subset \HPsi \) a dense core.
  \item \lemref{lem:density_schwartz_weighted_L2} — Density of \( \Schwartz \subset \HPsi \) in norm and graph topology.
  \item \remref{rem:sobolev_core_reference} — Alternate justification: \( \Schwartz \hookrightarrow H^s_\alpha \hookrightarrow \HPsi \) via Sobolev embeddings.
\end{itemize}

\textbf{Analytic and Spectral Estimates}
\begin{itemize}
  \item \lemref{lem:xi_growth_bound}, \lemref{lem:weighted_L1_inverse_FT_xi} — The profile \( \Xi(\tfrac{1}{2} + i\lambda) \in \PW{\pi} \), with inverse transform in \( L^1(\R, \Psi_\alpha^{-1}) \).
  \item \lemref{lem:decay_mollified_kernel}, \lemref{lem:L1_integrability_conjugated_kernel} — Mollifiers \( k_t \in \Schwartz \), conjugated kernels integrable.
  \item \lemref{lem:uniform_L1_conjugated_kernel}, \lemref{lem:trace_class_via_weighted_L1} — Trace norm convergence \( \|L_t - \Lsym\|_{\TC} \to 0 \) and Simon’s trace-class inclusion criterion.
  \item \lemref{lem:trace_class_conjugated_kernel}, \lemref{lem:trace_class_failure_alpha_leq_pi}, \propref{prop:trace_class_sharpness} — Trace-class fails for \( \alpha \le \pi \): sharp exponential decay threshold.
  \item \lemref{lem:unitary_conjugation_trace_class} — Trace norm preserved under unitary weight conjugation.
\end{itemize}

\textbf{Operator Properties of \texorpdfstring{\( L_t \)}{Lt}}
\begin{itemize}
  \item \propref{prop:boundedness_Lt_weighted}, \propref{prop:compactness_Lt} — Boundedness and compactness of \( L_t \) via mollified kernel structure.
  \item \propref{prop:symmetry_Lt_Schwartz}, \propref{prop:selfadjointness_Lt} — \( L_t \) is symmetric on \( \Schwartz \) and extends to a self-adjoint operator.
  \item \propref{prop:core_schwartz_density} — \( \Schwartz \) is a core for the limit operator \( \Lsym \).
\end{itemize}

\textbf{Canonical Operator Realization}
\begin{itemize}
  \item \thmref{thm:canonical_operator_realization} — Convergence \( L_t \to \Lsym \in \TC(\HPsi) \); defines the canonical compact self-adjoint operator realizing the spectral determinant.
\end{itemize}

\paragraph{Chapter Closure.}
This chapter establishes the analytic and operator-theoretic base for all that follows. The canonical convolution operator \( \Lsym \in \TC(\HPsi) \) is defined as the trace-norm limit of mollified Fourier convolution operators \( L_t \). Its construction relies on Paley--Wiener estimates, exponential decay, Sobolev density, and trace-class embedding theorems. The determinant identity
\[
\detz(I - \lambda \Lsym)
= \frac{\Xi\left(\tfrac{1}{2} + i\lambda \right)}{\Xi\left(\tfrac{1}{2} \right)}
\]
is proven in \secref{sec:determinant_identity}, resting entirely on this analytic groundwork.


\section{Spectral Rigidity and Determinantal Uniqueness}
\label{sec:spectral_rigidity}

\subsection*{Introduction}

This chapter establishes the analytic infrastructure for defining and analyzing the canonical compact operator \( L_{\mathrm{sym}} \), which realizes the completed Riemann zeta function \( \Xi(s) \) via its Fredholm determinant. The primary goal is to verify that mollified convolution operators associated with the inverse Fourier transform of \( \Xi \) are compact, trace class, and converge in trace norm to a self-adjoint limit operator \( L_{\mathrm{sym}} \in \mathcal{C}_1(H_{\Psi_\alpha}) \).

The constructions here verify:

\begin{itemize}
    \item Schatten-class properties of Hilbert–Schmidt and trace-class operators, following \cite[Ch.~4]{Simon2005TraceIdeals} and \cite[Ch.~VI]{ReedSimon1980I}, including the completeness of \( \mathcal{C}_1 \) and the trace-norm topology.
    
    \item Sufficient conditions for compactness and self-adjointness of integral operators with symmetric Hermitian kernels, using distributional domains and exponential conjugation.
    
    \item The structure of the weighted Schwartz space \( \mathcal{S}_\alpha(\mathbb{R}) \subset L^2(\mathbb{R}, e^{\alpha |x|}\, dx) \), for \( \alpha > \pi \), ensuring Fourier duality and decay control for entire functions of exponential type \( \pi \) \cite{Levin1996EntireLectures}.
    
    \item Uniform kernel bounds and mollifier admissibility for defining the regularized heat operators \( e^{-t L_t^2} \), together with analytic kernel estimates necessary for short-time trace control and Tauberian convergence.
\end{itemize}

These ingredients culminate in the construction of mollified convolution operators \( L_t \), and in the verification of trace-norm convergence
\[
L_t \to L_{\mathrm{sym}} \in \mathcal{C}_1(H_{\Psi_\alpha}) \quad \text{as } t \to 0^+.
\]
This limit defines the canonical spectral operator underlying the determinant identity
\[
\det\nolimits_{\zeta}(I - \lambda L_{\mathrm{sym}}) = \frac{\Xi\left(\tfrac{1}{2} + i\lambda\right)}{\Xi\left(\tfrac{1}{2}\right)},
\]
which is rigorously established without assuming RH.

\medskip

The analytic architecture developed here underpins all subsequent spectral and determinant identities.
See Appendix~\ref{app:dependency-graph} for a visual DAG linking these foundational tools to the modular proof of RH.

\begin{remark}[Structural Role of Chapter~\ref{sec:spectral_rigidity}]
\label{rem:structural_role_of_ch8}

This chapter establishes the converse direction of the analytic–spectral equivalence:
\[
\operatorname{Spec}(L_{\mathrm{sym}}) \subset \mathbb{R} \quad \Longrightarrow \quad \RH,
\]
thereby closing the logical loop initiated in Chapter~\ref{sec:spectral_implications}. All analytic prerequisites—trace-class convergence, determinant identity, and spectral encoding—are proven in prior chapters. No appeal is made to \(\RH\) itself.

\medskip

\noindent
Thus, the equivalence
\[
\RH \iff \operatorname{Spec}(L_{\mathrm{sym}}) \subset \mathbb{R}
\]
is derived entirely from the canonical operator's spectrum and its zeta-regularized Fredholm determinant, without invoking modular, motivic, or trace formula machinery.
\end{remark}


%------------------------------------------------------------------
\subsection{Spectral Reality and the Riemann Hypothesis}

% === Spectral encoding from zeta zeros via determinant identity ===
\begin{lemma}[Spectral Encoding via Determinant Zeros]
\label{lem:spectral-encoding-injection}
Let \( L_{\mathrm{sym}} \in \mathcal{C}_1(H_{\Psi_\alpha}) \) denote the canonical compact, self-adjoint operator on the exponentially weighted Hilbert space \( H_{\Psi_\alpha} := L^2(\mathbb{R}, e^{\alpha |x|} dx) \), for some \( \alpha > \pi \). Suppose the spectral determinant identity holds:
\[
\det\nolimits_{\zeta}(I - \lambda L_{\mathrm{sym}}) = \frac{\Xi\left( \tfrac{1}{2} + i\lambda \right)}{\Xi\left( \tfrac{1}{2} \right)},
\]
where \( \Xi(s) \) is the completed Riemann zeta function.

Then the mapping
\[
\rho \mapsto \mu_\rho := \frac{1}{i}(\rho - \tfrac{1}{2})
\]
defines a multiplicity-preserving injection from the multiset of nontrivial zeros \( \rho \) of \( \zeta(s) \) into the nonzero spectrum of \( L_{\mathrm{sym}} \):
\[
\operatorname{Spec}(L_{\mathrm{sym}}) \supset \left\{ \mu_\rho \in \mathbb{C} : \rho \in \operatorname{Zeros}(\zeta(s)) \right\},
\]
with multiplicity of each \( \mu_\rho \) equal to the multiplicity of the corresponding zero \( \rho \).

\medskip
\noindent
This follows from Hadamard factorization of \( \Xi(s) \), which encodes the zeros of \( \zeta(s) \) in the entire function \( \Xi(\tfrac{1}{2} + i\lambda) \), and from the identification of the determinant with a canonical spectral product.
\end{lemma}

\begin{proof}[Proof of \lemref{lem:spectral_encoding_injection}]
Let \( \rho = \tfrac{1}{2} + i\gamma \) be a nontrivial zero of \( \zeta(s) \), and define the associated spectral image:
\[
\mu_\rho := \frac{1}{i}(\rho - \tfrac{1}{2}) = \gamma.
\]

\paragraph{Step 1: Determinantal Zeros.}
By the canonical determinant identity,
\[
\det\nolimits_\zeta(I - \lambda L_{\mathrm{sym}}) = \frac{\Xi(\tfrac{1}{2} + i\lambda)}{\Xi(\tfrac{1}{2})},
\]
the determinant vanishes precisely when \( \Xi(\tfrac{1}{2} + i\lambda) = 0 \), i.e., at \( \lambda = \gamma \) when \( \rho = \tfrac{1}{2} + i\gamma \) is a nontrivial zeta zero.

\paragraph{Step 2: Spectral Correspondence.}
Since \( L_{\mathrm{sym}} \in \mathcal{C}_1(H_{\Psi_\alpha}) \), analytic Fredholm theory implies that \( \lambda \in \mathbb{C} \) is a zero of the determinant if and only if \( \lambda^{-1} \in \operatorname{Spec}(L_{\mathrm{sym}}) \setminus \{0\} \). Thus,
\[
\mu_\rho = \frac{1}{\gamma} = \lambda^{-1} \in \operatorname{Spec}(L_{\mathrm{sym}}) \setminus \{0\}.
\]

\paragraph{Step 3: Multiplicity Preservation.}
The order of vanishing of the determinant at \( \lambda = \gamma \) matches the multiplicity of the eigenvalue \( \mu_\rho = 1/\gamma \) in \( \operatorname{Spec}(L_{\mathrm{sym}}) \), by Hadamard factorization. This also equals the multiplicity of the zero \( \rho \) of \( \zeta(s) \).

\paragraph{Conclusion.}
The canonical encoding
\[
\rho \mapsto \mu_\rho := \frac{1}{i}(\rho - \tfrac{1}{2})
\]
defines a multiplicity-preserving injection from the nontrivial zeros of \( \zeta(s) \) into the nonzero spectrum of \( L_{\mathrm{sym}} \), as claimed.
\end{proof}


% === Spectrum symmetry: mu ↦ -mu from functional equation of Ξ ===
\begin{remark}[Forward Reference: Spectral Symmetry]
\label{rem:spectral_symmetry_forward_ref}
The symmetry of the spectrum of the canonical operator \( L_{\sym} \in \mathcal{C}_1(H_{\Psi_\alpha}) \),
namely that
\[
\mu \in \Spec(L_{\sym}) \quad \Longrightarrow \quad -\mu \in \Spec(L_{\sym}),
\]
is established earlier in \lemref{lem:spectral_symmetry} (Chapter~4). This result is a direct consequence of the
evenness and reality of the kernel derived from the completed zeta function’s Fourier transform, together
with unitary equivalence to a symmetric convolution operator on \( L^2(\mathbb{R}) \).
\end{remark}


% === Real eigenvalue implies critical line zero ===
\begin{lemma}[Spectral Realization and Rigidity Imply the Riemann Hypothesis]
\label{lem:inject-zero-real-spectrum}
Let \( L_{\mathrm{sym}} \in \mathcal{C}_1(H_{\Psi_\alpha}) \) be the canonical compact, self-adjoint operator on the exponentially weighted Hilbert space \( H_{\Psi_\alpha} := L^2(\mathbb{R}, e^{\alpha |x|} dx) \), for some fixed \( \alpha > \pi \), with dense domain.

Assume the following conditions:
\begin{enumerate}
  \item[\textnormal{(i)}] There exists a bijective, multiplicity-preserving correspondence between the multiset of nontrivial zeros \( \rho \in \mathbb{C} \) of the Riemann zeta function \( \zeta(s) \) and the nonzero spectrum of \( L_{\mathrm{sym}} \), given by
  \[
  \rho \mapsto \mu_\rho := \frac{1}{i}(\rho - \tfrac{1}{2}) \in \operatorname{Spec}(L_{\mathrm{sym}}) \setminus \{0\}.
  \]
  
  \item[\textnormal{(ii)}] The spectrum \( \operatorname{Spec}(L_{\mathrm{sym}}) \) is real and simple.
\end{enumerate}

Then every nontrivial zero \( \rho \) of \( \zeta(s) \) satisfies the Riemann Hypothesis:
\[
\Re(\rho) = \tfrac{1}{2}.
\]
\end{lemma}

\begin{proof}[Proof of Lemma~\ref{lem:inject_zero_real_spectrum}]
Let \( \rho = \tfrac{1}{2} + i\gamma \in \mathbb{C} \) be an arbitrary nontrivial zero of the Riemann zeta function \( \zeta(s) \). Under assumption (i), this zero is mapped to a nonzero eigenvalue of \( L_{\mathrm{sym}} \) via the spectral encoding:
\[
\mu_\rho := \frac{1}{i}(\rho - \tfrac{1}{2}) = \frac{1}{i}(i\gamma) = \gamma.
\]

\paragraph{Step 1: Reality of Spectrum.}
Assumption (ii) states that \( \mu_\rho \in \operatorname{Spec}(L_{\mathrm{sym}}) \subset \mathbb{R} \), and that all eigenvalues are real and simple. Hence,
\[
\mu_\rho = \gamma \in \mathbb{R}.
\]
This immediately implies that \( \rho = \tfrac{1}{2} + i\gamma \) has \( \gamma \in \mathbb{R} \), so
\[
\Re(\rho) = \tfrac{1}{2}.
\]

\paragraph{Step 2: Exhaustion.}
Since the mapping \( \rho \mapsto \mu_\rho \) is bijective and multiplicity-preserving by hypothesis, every nontrivial zero of \( \zeta(s) \) corresponds to a unique real eigenvalue of \( L_{\mathrm{sym}} \). Therefore, all nontrivial zeros lie on the critical line.

\paragraph{Conclusion.}
We conclude that
\[
\rho \in \operatorname{Zeros}(\zeta(s)) \quad \Longrightarrow \quad \Re(\rho) = \tfrac{1}{2},
\]
establishing the Riemann Hypothesis under the stated spectral assumptions.
\end{proof}


% === Determinantal zero implies spectral eigenvalue (surjectivity) ===
\begin{lemma}[Determinantal Zero Implies Spectral Inclusion]
\label{lem:det_zero_implies_spectrum}
Let \( T \in \mathcal{C}_1(H) \) be a compact, self-adjoint operator on a complex Hilbert space \( H \). Suppose the Carleman zeta-regularized Fredholm determinant
\[
\det\nolimits_{\zeta}(I - \lambda T)
\]
vanishes at some \( \lambda \in \mathbb{C} \setminus \{0\} \). Then:
\[
\lambda^{-1} \in \operatorname{Spec}(T),
\]
i.e., \( \lambda \in \operatorname{Spec}(T^{-1}) \), and hence \( \lambda \in \operatorname{Spec}(T)^{-1} \).

\medskip
\noindent
That is, any zero of the zeta-regularized determinant corresponds to a nonzero eigenvalue of \( T \), and the set of determinant zeros coincides with the inverse spectrum \( \operatorname{Spec}(T)^{-1} \setminus \{0\} \), counted with multiplicity.
\end{lemma}

\begin{proof}[Proof of \lemref{lem:det_zero_implies_spectrum}]
Let \( T \in \TC(H) \) be compact and self-adjoint. Then its spectrum consists of a discrete set of real eigenvalues \( \{ \mu_n \} \subset \R \setminus \{0\} \), with \( \mu_n \to 0 \), counted with multiplicity.

The Carleman \(\zeta\)-regularized determinant admits the canonical Hadamard product:
\[
\det\nolimits_\zeta(I - \lambda T) = \prod_{n=1}^\infty (1 - \lambda \mu_n),
\]
which converges absolutely on compact subsets of \( \C \), since \( T \in \TC(H) \Rightarrow \sum |\mu_n| < \infty \).

\paragraph{Step 1: Determinant Zero Implies Reciprocal Eigenvalue.}
Suppose
\[
\det\nolimits_\zeta(I - \lambda_0 T) = 0, \qquad \lambda_0 \in \C \setminus \{0\}.
\]
Then for some index \( n \), we have:
\[
1 - \lambda_0 \mu_n = 0 \quad \Longrightarrow \quad \lambda_0 = \mu_n^{-1}.
\]

\paragraph{Step 2: Spectrum Inclusion.}
Thus,
\[
\lambda_0^{-1} = \mu_n \in \Spec(T),
\quad \text{and} \quad \lambda_0 \in \Spec(T)^{-1}.
\]

\paragraph{Conclusion.}
Every nonzero zero of the determinant corresponds to a nonzero eigenvalue of \( T \), and:
\[
\lambda \in \C, \quad \det\nolimits_\zeta(I - \lambda T) = 0
\quad \Longrightarrow \quad
\lambda^{-1} \in \Spec(T).
\]
This completes the proof.
\end{proof}


% === Real spectrum and simple eigenvalues implies RH and simple zeros ===
\begin{proposition}[Spectral Reality Implies RH and Simplicity]
\label{prop:spectrum_reality_implies_rh_multiplicity}

Let \( L_{\sym} \in \TC(\HPsi) \) be the canonical compact, self-adjoint operator associated with the completed Riemann zeta function \( \XiR(s) \), and define its spectral determinant:
\[
f(\lambda) := \det\nolimits_\zeta(I - \lambda L_{\sym}) = \frac{\XiR\left( \tfrac{1}{2} + i\lambda \right)}{\XiR\left( \tfrac{1}{2} \right)}.
\]

Assume:
\begin{enumerate}
  \item[\textup{(i)}] The spectrum of \( L_{\sym} \) lies entirely on the real axis:
  \[
  \Spec(L_{\sym}) \subset \R.
  \]

  \item[\textup{(ii)}] Each nonzero eigenvalue \( \mu \in \Spec(L_{\sym}) \setminus \{0\} \) is simple (algebraic multiplicity one).
\end{enumerate}

Then every nontrivial zero \( \rho \) of the Riemann zeta function satisfies:
\[
\Re(\rho) = \tfrac{1}{2}, \qquad \operatorname{ord}_\rho(\zetaR) = 1,
\]
i.e., the Riemann Hypothesis holds and all nontrivial zeros are simple.
\end{proposition}

\begin{proof}[Proof of \propref{prop:spectrum_reality_implies_rh_multiplicity}]
Assume:
\begin{itemize}
  \item \( L_{\mathrm{sym}} \in \mathcal{C}_1(H_{\Psi_\alpha}) \) is compact and self-adjoint;
  \item \( \operatorname{Spec}(L_{\mathrm{sym}}) \subset \mathbb{R} \);
  \item All nonzero eigenvalues \( \mu \in \operatorname{Spec}(L_{\mathrm{sym}}) \setminus \{0\} \) are simple.
\end{itemize}

Let
\[
f(\lambda) := \det\nolimits_\zeta(I - \lambda L_{\mathrm{sym}}) = \frac{\Xi\left(\tfrac{1}{2} + i\lambda\right)}{\Xi\left(\tfrac{1}{2}\right)}
\]
be the canonical spectral determinant. As an entire function of exponential type \( \pi \) and genus one, it admits the Hadamard product:
\[
f(\lambda) = \prod_\rho \left(1 - \frac{\lambda}{\mu_\rho} \right) \exp\left( \frac{\lambda}{\mu_\rho} \right),
\]
where \( \mu_\rho := \frac{1}{i}(\rho - \tfrac{1}{2}) \), with \( \rho \in \operatorname{Spec}(\zeta) \), counted with multiplicity.

\paragraph{Step 1: Real Spectrum Implies RH.}
If \( \mu_\rho \in \mathbb{R} \), then
\[
\mu_\rho = \frac{1}{i}(\rho - \tfrac{1}{2}) \in \mathbb{R} \quad \Rightarrow \quad \rho - \tfrac{1}{2} \in i\mathbb{R} \quad \Rightarrow \quad \operatorname{Re}(\rho) = \tfrac{1}{2}.
\]
Hence, all nontrivial zeros lie on the critical line.

\paragraph{Step 2: Simplicity.}
Each eigenvalue \( \mu_\rho \in \operatorname{Spec}(L_{\mathrm{sym}}) \) has multiplicity one. Since the determinant product preserves multiplicity under the map \( \rho \mapsto \mu_\rho \), the order of vanishing of \( \Xi(s) \) at \( s = \tfrac{1}{2} + i\lambda \) is also one:
\[
\operatorname{ord}_\rho(\zeta) = \operatorname{ord}_{\mu_\rho}(f) = 1.
\]

\paragraph{Conclusion.}
Under the assumptions of real and simple spectrum, all nontrivial zeros of \( \zeta(s) \) lie on the critical line and are simple, as claimed.
\end{proof}


% === Unified injective spectral rigidity lemma ===
\begin{lemma}[Spectral Reality Implies the Riemann Hypothesis]
\label{lem:real_spectrum_implies_rh_rigidity}

Let \( L_{\sym} \in \TC(\HPsi) \) be the canonical compact, self-adjoint operator constructed via mollified convolution from the inverse Fourier transform of the completed Riemann zeta function \( \XiR(s) \), acting on the weighted Hilbert space:
\[
\HPsi := L^2(\R, e^{\alpha |x|} dx), \qquad \alpha > \pi.
\]

Assume:
\begin{enumerate}
  \item[\textup{(i)}] The Carleman \(\zeta\)-regularized Fredholm determinant satisfies:
  \[
  \det\nolimits_\zeta(I - \lambda L_{\sym}) = \frac{\XiR\left(\tfrac{1}{2} + i\lambda\right)}{\XiR\left(\tfrac{1}{2}\right)}, \qquad \forall \lambda \in \C.
  \]

  \item[\textup{(ii)}] The spectrum of \( L_{\sym} \) lies entirely on the real axis:
  \[
  \Spec(L_{\sym}) \subset \R.
  \]
\end{enumerate}

Then every nontrivial zero \( \rho \in \C \) of the Riemann zeta function satisfies:
\[
\Re(\rho) = \tfrac{1}{2}.
\]
That is, the Riemann Hypothesis holds.
\end{lemma}

\begin{proof}[Proof of \lemref{lem:real_spectrum_implies_rh_rigidity}]
Let \( \rho \) be a nontrivial zero of \( \zeta(s) \), and define the canonical spectral image:
\[
\mu_\rho := \frac{1}{i}(\rho - \tfrac{1}{2}).
\]

\paragraph{Step 1: Determinant Zero Implies Spectral Inclusion.}
From the determinant identity
\[
\det\nolimits_\zeta(I - \lambda L_{\mathrm{sym}}) = \frac{\Xi\left( \tfrac{1}{2} + i\lambda \right)}{\Xi\left( \tfrac{1}{2} \right)},
\]
we have \( \det\nolimits_\zeta(I - \lambda L_{\mathrm{sym}}) = 0 \) precisely when \( \lambda = \gamma \), where \( \rho = \tfrac{1}{2} + i\gamma \).

Then
\[
\mu_\rho = \frac{1}{i(\rho - \tfrac{1}{2})} = \frac{1}{\gamma} = \lambda^{-1}.
\]
By analytic Fredholm theory, this implies \( \mu_\rho \in \operatorname{Spec}(L_{\mathrm{sym}}) \setminus \{0\} \).

\paragraph{Step 2: Real Spectrum Implies RH.}
By assumption, \( \operatorname{Spec}(L_{\mathrm{sym}}) \subset \mathbb{R} \), so \( \mu_\rho \in \mathbb{R} \). Therefore,
\[
\frac{1}{i}(\rho - \tfrac{1}{2}) \in \mathbb{R} \quad \Rightarrow \quad \rho - \tfrac{1}{2} \in i\mathbb{R} \quad \Rightarrow \quad \operatorname{Re}(\rho) = \tfrac{1}{2}.
\]

\paragraph{Conclusion.}
Since \( \rho \) was arbitrary, all nontrivial zeros of \( \zeta(s) \) lie on the critical line. Hence, the Riemann Hypothesis holds.
\end{proof}


% === Spectral uniqueness from zeta determinant ===
\begin{lemma}[Uniqueness of Spectral Realization via Trace-Class Determinants]
\label{lem:uniqueness_from_determinant}
Let \( A, B \in \TC(H) \) be compact, self-adjoint, positive operators on a separable Hilbert space \( H \), and suppose
\[
\det\nolimits_{\zeta}(I - \lambda A) = \det\nolimits_{\zeta}(I - \lambda B), \quad \text{for all } \lambda \in \C.
\]
Then \( A \) and \( B \) have identical multisets of eigenvalues (counted with multiplicity). If, in addition, \( A \) and \( B \) share a common orthonormal eigenbasis, then \( A = B \).

\medskip
\noindent
In particular, the canonical operator \( L_{\mathrm{sym}} \in \TC(H_{\Psi_\alpha}) \) constructed in \secref{sec:operator_construction} is uniquely determined (up to unitary equivalence) by its spectral determinant identity:
\[
\detz(I - \lambda L_{\mathrm{sym}}) = \frac{\Xi(\tfrac{1}{2} + i\lambda)}{\Xi(\tfrac{1}{2})}.
\]
\end{lemma}

\begin{proof}[Proof of \lemref{lem:uniqueness_from_determinant}]
Let \( A, B \in \TC(H) \) be compact, self-adjoint, positive operators such that
\[
\detz(I - \lambda A) = \detz(I - \lambda B) \quad \text{for all } \lambda \in \C.
\]
The zeta-regularized determinant of a trace-class operator \( T \ge 0 \) with eigenvalues \( \{\mu_n\} \subset [0, \infty) \) is defined via:
\[
\log \detz(I - \lambda T) = -\sum_{n=1}^\infty \log(1 - \lambda \mu_n).
\]
This is valid as an entire function in \( \lambda \) due to the compactness and positivity of \( T \), and uniquely determines the multiset \( \{\mu_n\} \), counted with multiplicities, by Hadamard's factorization theorem.

Hence, if \( \detz(I - \lambda A) = \detz(I - \lambda B) \), then the sequences of eigenvalues \( \{\mu_n^{(A)}\} \) and \( \{\mu_n^{(B)}\} \) must coincide, including multiplicities. That is,
\[
\operatorname{Spec}(A) = \operatorname{Spec}(B) \quad \text{as multisets}.
\]

Now suppose \( A \) and \( B \) share a common orthonormal eigenbasis \( \{e_n\} \). Then both operators are diagonal with respect to this basis:
\[
A e_n = \mu_n e_n = B e_n \quad \text{for all } n.
\]
Thus, \( A = B \) as operators on \( H \).

Applying this to the canonical operator \( L_{\mathrm{sym}} \) with determinant identity
\[
\detz(I - \lambda L_{\mathrm{sym}}) = \frac{\Xi(\tfrac{1}{2} + i\lambda)}{\Xi(\tfrac{1}{2})},
\]
we conclude that any other operator \( \widetilde{L} \in \TC(H_{\Psi_\alpha}) \) sharing this identity must have the same eigenvalues as \( L_{\mathrm{sym}} \). If \( \widetilde{L} \) is additionally assumed to be self-adjoint on the same space, it is unitarily equivalent to \( L_{\mathrm{sym}} \), and if their eigenvectors align, they must coincide.
\end{proof}


% === Determinant and spectrum uniquely determine the operator ===
\begin{proposition}[Inverse Spectral Rigidity]
\label{prop:inverse_spectral_rigidity}
Let \( L_1, L_2 \in \mathcal{C}_1(H_{\Psi_\alpha}) \) be compact, self-adjoint operators on the exponentially weighted Hilbert space \( H_{\Psi_\alpha} := L^2(\mathbb{R}, e^{\alpha |x|} dx) \), for fixed \( \alpha > \pi \). Suppose the following hold:
\begin{itemize}
  \item The spectra of \( L_1 \) and \( L_2 \) agree as multisets, including algebraic multiplicities:
  \[
  \operatorname{Spec}(L_1) = \operatorname{Spec}(L_2).
  \]

  \item The Carleman \(\zeta\)-regularized Fredholm determinants agree globally:
  \[
  \det\nolimits_\zeta(I - \lambda L_1) = \det\nolimits_\zeta(I - \lambda L_2), \qquad \text{for all } \lambda \in \mathbb{C}.
  \]
\end{itemize}

Then \( L_1 \) and \( L_2 \) are unitarily equivalent: there exists a unitary operator \( U : H_{\Psi_\alpha} \to H_{\Psi_\alpha} \) such that
\[
L_2 = U L_1 U^{-1}.
\]

\medskip
\noindent
In particular, if both \( L_1 \) and \( L_2 \) arise from the canonical convolution construction associated with the completed zeta function \( \Xi(s) \), then
\[
L_1 = L_2.
\]
\end{proposition}

\begin{proof}[Proof of \propref{prop:inverse_spectral_rigidity}]
Let \( L_1, L_2 \in \mathcal{C}_1(H_{\Psi_\alpha}) \) be compact, self-adjoint operators satisfying:
\[
\operatorname{Spec}(L_1) = \operatorname{Spec}(L_2), \qquad
\det\nolimits_\zeta(I - \lambda L_1) = \det\nolimits_\zeta(I - \lambda L_2), \quad \forall \lambda \in \mathbb{C}.
\]

\paragraph{Step 1: Spectral Theorem and Orthonormal Bases.}
By the spectral theorem, each \( L_j \) admits an orthonormal basis \( \{e_n^{(j)}\} \subset H_{\Psi_\alpha} \) with corresponding eigenvalues \( \{\lambda_n\} \subset \mathbb{R} \), repeated with multiplicities, satisfying:
\[
L_j f = \sum_{n=1}^\infty \lambda_n \langle f, e_n^{(j)} \rangle e_n^{(j)}, \quad j = 1,2.
\]

\paragraph{Step 2: Determinant Equivalence.}
Since each operator is trace class, its Carleman determinant is given by:
\[
\det\nolimits_\zeta(I - \lambda L_j) = \prod_{n=1}^\infty (1 - \lambda \lambda_n),
\]
which converges absolutely on compact subsets of \( \mathbb{C} \). Equality of determinants for all \( \lambda \in \mathbb{C} \) implies equality of their Hadamard product expansions, and thus the spectrum agrees identically, including multiplicities.

\paragraph{Step 3: Construction of Intertwiner.}
Define a unitary operator \( U \colon H_{\Psi_\alpha} \to H_{\Psi_\alpha} \) by \( U e_n^{(1)} := e_n^{(2)} \). This defines an isometry and yields:
\[
U L_1 U^{-1} f = \sum_{n=1}^\infty \lambda_n \langle f, e_n^{(2)} \rangle e_n^{(2)} = L_2 f.
\]
Hence,
\[
L_2 = U L_1 U^{-1}.
\]

\paragraph{Conclusion.}
The operators \( L_1 \) and \( L_2 \) are unitarily equivalent. If both arise from the canonical determinant identity for \( \Xi(s) \), then this unitary equivalence coincides with identity, and \( L_1 = L_2 \) by uniqueness of the canonical convolution model.
\end{proof}


\begin{corollary}[Canonical Operator Uniqueness]
\label{cor:canonical_operator_uniqueness}
Let \( L \in \mathcal{C}_1(H_{\Psi_\alpha}) \) be a compact, self-adjoint operator satisfying
\[
\detz(I - \lambda L) = \frac{\Xi\left(\tfrac{1}{2} + i\lambda\right)}{\Xi\left(\tfrac{1}{2}\right)}.
\]
Then \( L \) is unitarily equivalent to the canonical operator \( L_{\mathrm{sym}} \) defined in \thmref{thm:canonical_operator_realization}. If \( L \) is constructed via the same convolutional framework, then \( L = L_{\mathrm{sym}} \).
\end{corollary}

\begin{proof}[Proof of \corref{cor:canonical_operator_uniqueness}]
Assume $L_1, L_2 \in \mathcal{C}_1(H_{\Psi_\alpha})$ are compact, self-adjoint operators on the exponentially weighted Hilbert space $H_{\Psi_\alpha}$ with $\alpha > \pi$, and suppose they satisfy the same determinant identity:
\[
\det\nolimits_{\zeta}(I - \lambda L_j) = \frac{\Xi\left(\tfrac{1}{2} + i\lambda\right)}{\Xi\left(\tfrac{1}{2}\right)} \quad \text{for } j = 1,2.
\]

The right-hand side is an entire function of order 1 and exponential type $\pi$, whose Hadamard factorization is uniquely determined by the nontrivial zeros $\rho$ of $\zeta(s)$ via
\[
\Xi\left(\tfrac{1}{2} + i\lambda\right) = \Xi\left(\tfrac{1}{2}\right) \prod_{\rho} \left(1 - \frac{\lambda}{\mu_\rho}\right),
\quad \mu_\rho := \frac{1}{i}(\rho - \tfrac{1}{2}),
\]
with multiplicities preserved.

Since the zeta-regularized determinant encodes the nonzero spectrum (with multiplicities), both $L_1$ and $L_2$ must share the same multiset of eigenvalues. Because both are self-adjoint and compact, the spectral theorem implies they are unitarily equivalent. Hence, the operator is determined uniquely (up to unitary equivalence) by the determinant identity.
\end{proof}

%------------------------------------------------------------------
\subsection{Positivity of Spectral Distributions}

% === Trace positivity from heat kernel regularization ===
\begin{lemma}[Positivity of the Trace Distribution]
\label{lem:trace_distribution_positivity}
Let \( L_{\mathrm{sym}} \in \mathcal{C}_1(H_{\Psi_\alpha}) \) be the canonical compact, self-adjoint operator with discrete real spectrum \( \{ \mu_n \}_{n \in \mathbb{Z}} \subset \mathbb{R} \), with each eigenvalue repeated according to its finite multiplicity.

Define the spectral trace functional on the Schwartz space \( \mathcal{S}(\mathbb{R}) \) by
\[
\phi \mapsto \operatorname{Tr}(\phi(L_{\mathrm{sym}})) := \sum_n \phi(\mu_n).
\]

Then the following hold:
\begin{enumerate}
  \item[\normalfont(i)] The map \( \phi \mapsto \operatorname{Tr}(\phi(L_{\mathrm{sym}})) \) defines a tempered distribution on \( \mathbb{R} \). That is, it extends continuously to \( \mathcal{S}(\mathbb{R}) \) and satisfies finite-order growth bounds under test function differentiation.

  \item[\normalfont(ii)] The trace distribution is positive: for every nonnegative test function \( \phi \in \mathcal{S}(\mathbb{R}) \) such that \( \phi(\lambda) \ge 0 \) for all \( \lambda \in \mathbb{R} \), we have
  \[
  \operatorname{Tr}(\phi(L_{\mathrm{sym}})) \ge 0.
  \]
\end{enumerate}

\medskip
\noindent
This distributional positivity reflects the positivity of the spectral measure associated with \( L_{\mathrm{sym}} \), as inherited from the positivity of the heat kernel trace. The analytic justification for this property follows from the short-time trace expansion in Proposition~\ref{prop:heat_trace_uniform_conv}, and the decay and regularity results developed in Appendix~\ref{app:heat-kernel-construction}.
\end{lemma}

\begin{proof}[Proof of Lemma~\ref{lem:trace_distribution_positivity}]
Let \( \{ \mu_n \} \subset \mathbb{R} \) denote the discrete spectrum of the compact, self-adjoint operator \( L_{\mathrm{sym}} \), with each eigenvalue repeated according to its finite multiplicity.

\paragraph{(i) Well-Definedness and Temperedness.}
For \( \phi \in \mathcal{S}(\mathbb{R}) \), the operator \( \phi(L_{\mathrm{sym}}) \) is defined via spectral functional calculus:
\[
\phi(L_{\mathrm{sym}}) = \sum_n \phi(\mu_n) P_n,
\]
where \( P_n \) denotes the orthogonal projection onto the eigenspace corresponding to \( \mu_n \), and each \( P_n \) is of finite rank.

Since \( L_{\mathrm{sym}} \in \mathcal{C}_1(H_{\Psi_\alpha}) \), its eigenvalues \( \mu_n \to 0 \) and satisfy \( \sum_n |\mu_n| < \infty \). The Schwartz decay of \( \phi \in \mathcal{S}(\mathbb{R}) \) ensures that for any \( N \in \mathbb{N} \), there exists \( C_N > 0 \) such that
\[
|\phi(\mu_n)| \le C_N (1 + |\mu_n|)^{-N}.
\]
Choosing \( N \) large enough yields
\[
\sum_n |\phi(\mu_n)| < \infty,
\]
so the spectral trace
\[
\operatorname{Tr}(\phi(L_{\mathrm{sym}})) := \sum_n \phi(\mu_n)
\]
is well-defined and finite. Furthermore, the map \( \phi \mapsto \operatorname{Tr}(\phi(L_{\mathrm{sym}})) \) is continuous on \( \mathcal{S}(\mathbb{R}) \), bounded by a finite collection of Schwartz seminorms. Thus, it defines a tempered distribution on \( \mathbb{R} \).

\paragraph{(ii) Positivity.}
If \( \phi \in \mathcal{S}(\mathbb{R}) \) satisfies \( \phi(\lambda) \ge 0 \) for all \( \lambda \in \mathbb{R} \), then
\[
\operatorname{Tr}(\phi(L_{\mathrm{sym}})) = \sum_n \phi(\mu_n) \ge 0,
\]
since each term in the sum is nonnegative.

This positivity reflects the nonnegativity of the heat kernel \( K_t(x, x) \) and its diagonal integral expansion. In particular, positivity is supported analytically by Proposition~\ref{prop:heat_trace_uniform_conv}, and the full decay and regularity justification in Appendix~\ref{app:heat_kernel_construction}.

\paragraph{Conclusion.}
The trace pairing \( \phi \mapsto \operatorname{Tr}(\phi(L_{\mathrm{sym}})) \) defines a positive tempered distribution on \( \mathbb{R} \), as claimed.
\end{proof}

\begin{remark}[Functional Calculus for Spectral Trace Pairings]
\label{rem:functional_calculus_trace}
For any \( \phi \in \Schwartz(\R) \), the spectral operator \( \phi(L_{\sym}) \) is well-defined via the spectral theorem. Since \( L_{\sym} \in \TC(\HPsi) \), its eigenvalues \( \{ \mu_n \} \subset \R \) satisfy \( \mu_n \to 0 \) and \( \sum_n |\mu_n| < \infty \).

The rapid decay of \( \phi(\mu_n) \) ensures:
\[
\sum_n |\phi(\mu_n)| < \infty,
\]
so the trace
\[
\Tr(\phi(L_{\sym})) := \sum_n \phi(\mu_n)
\]
is absolutely convergent. Therefore, all trace pairings \( \phi \mapsto \Tr(\phi(L_{\sym})) \) used in this chapter are rigorously defined for test functions \( \phi \in \Schwartz(\R) \).

This justifies interpreting \( \Tr(\phi(L_{\sym})) \) as a tempered distribution without requiring additional regularization.
\end{remark}


% === Heat trace induces positive spectral measure ===
\begin{lemma}[Positivity of Trace Distribution]
\label{lem:trace_distribution_positive}

Let \( L_{\sym} \in \TC(\HPsi) \) be the canonical compact, self-adjoint operator acting on the exponentially weighted Hilbert space:
\[
\HPsi := L^2(\R, e^{\alpha |x|} dx), \qquad \alpha > \pi.
\]

Define the spectral trace functional:
\[
\varphi \mapsto \Tr(\varphi(L_{\sym})),
\quad \text{for all } \varphi \in \Schwartz(\R),
\]
via the spectral theorem and functional calculus.

Then this functional defines a positive tempered distribution on \( \R \). In particular,
\[
\Tr(\varphi(L_{\sym})) \ge 0
\qquad \text{whenever } \varphi(\lambda) \ge 0 \text{ for all } \lambda \in \R.
\]
\end{lemma}

\begin{proof}[Proof of \lemref{lem:trace_distribution_positive}]
Since \( L_{\sym} \in \TC(\HPsi) \) is compact and self-adjoint, it admits a spectral decomposition:
\[
L_{\sym} = \sum_n \mu_n \langle \cdot, \psi_n \rangle \psi_n,
\]
where \( \{ \psi_n \} \subset \HPsi \) is an orthonormal basis, and \( \{ \mu_n \} \subset \R \) are the eigenvalues, counted with multiplicity.

Let \( \varphi \in \Schwartz(\R) \) be real-valued. Then by the spectral theorem:
\[
\varphi(L_{\sym}) = \sum_n \varphi(\mu_n) \langle \cdot, \psi_n \rangle \psi_n,
\]
and thus,
\[
\Tr(\varphi(L_{\sym})) = \sum_n \varphi(\mu_n).
\]

If \( \varphi(\lambda) \ge 0 \) for all \( \lambda \in \R \), each term is nonnegative, and hence:
\[
\Tr(\varphi(L_{\sym})) \ge 0.
\]

Since \( \{ \mu_n \} \) has at most polynomial growth and \( \varphi \in \Schwartz(\R) \) has rapid decay, the map \( \varphi \mapsto \Tr(\varphi(L_{\sym})) \) is continuous in the Schwartz topology and defines a tempered distribution.

\paragraph{Conclusion.}
The spectral trace functional is positive on nonnegative test functions and tempered on \( \Schwartz(\R) \), completing the proof.
\end{proof}


% === Unified rigidity theorem: RH from real, simple spectrum + determinant ===
\begin{theorem}[Spectral Rigidity Reformulation of RH]
\label{thm:rh_from_real_simple_spectrum}
Suppose \( \Lsym \in \TC(\HPsi) \) is a compact, self-adjoint operator on the exponentially weighted Hilbert space
\[
\HPsi := L^2(\R, e^{\alpha |x|} dx), \qquad \alpha > \pi,
\]
and satisfies the following:
\begin{itemize}
  \item[\textup{(i)}] The spectrum is real: \( \Spec(\Lsym) \subset \R \);
  \item[\textup{(ii)}] All nonzero eigenvalues are simple;
  \item[\textup{(iii)}] The determinant identity holds globally:
  \[
  \detz(I - \lambda \Lsym) = \frac{\XiR(\tfrac{1}{2} + i\lambda)}{\XiR(\tfrac{1}{2})}, \quad \forall\, \lambda \in \C.
  \]
\end{itemize}

Then the Riemann Hypothesis is true:
\[
\zetaR(\rho) = 0 \quad \Longrightarrow \quad \Re(\rho) = \tfrac{1}{2}, \quad \text{and } \operatorname{ord}_\rho(\zetaR) = 1.
\]
\end{theorem}

\begin{proof}[Proof of \thmref{thm:rh_from_real_simple_spectrum}]
Assume \( \Lsym \in \TC(\HPsi) \) satisfies the three stated conditions:
\begin{enumerate}
  \item[\textup{(i)}] \( \Spec(\Lsym) \subset \R \);
  \item[\textup{(ii)}] All nonzero eigenvalues \( \mu \in \Spec(\Lsym) \setminus \{0\} \) are simple;
  \item[\textup{(iii)}] The spectral determinant satisfies:
  \[
  \detz(I - \lambda \Lsym) = \frac{\XiR(\tfrac{1}{2} + i\lambda)}{\XiR(\tfrac{1}{2})}.
  \]
\end{enumerate}

\paragraph{Step 1: Determinantal Zeros Correspond to Zeta Zeros.}
By the determinant identity and analytic Fredholm theory (cf. \lemref{lem:det_zero_implies_spectrum}),
each nontrivial zero \( \rho = \tfrac{1}{2} + i\gamma \) of \( \zetaR(s) \)
corresponds to a spectral eigenvalue \( \mu_\rho = \gamma^{-1} \in \Spec(\Lsym) \setminus \{0\} \).

\paragraph{Step 2: Real Spectrum Implies RH.}
From the encoding \( \mu_\rho = \tfrac{1}{i}(\rho - \tfrac{1}{2}) \in \R \),
it follows that \( \Re(\rho) = \tfrac{1}{2} \). Therefore,
all nontrivial zeros lie on the critical line, establishing RH.

\paragraph{Step 3: Simplicity of Zeros.}
Since all nonzero eigenvalues are simple and the spectral map
\( \rho \mapsto \mu_\rho \) preserves multiplicity (by \lemref{lem:spectral_encoding_injection}),
it follows that each zero of \( \zetaR(s) \) is simple.

\paragraph{Conclusion.}
Both the Riemann Hypothesis and the simplicity of all nontrivial zeros
follow from the spectral assumptions on \( \Lsym \).
\thmref{thm:det_identity_revised}
\end{proof}


%------------------------------------------------------------------
\subsection*{Summary}
\label{sec:foundations_summary}

\textbf{Operator-Theoretic Foundations}
\begin{itemize}
  \item \defref{def:compact_operator} — Compact operators: norm limits of finite-rank maps with discrete spectrum.
  \item \defref{def:trace_class_operator}, \defref{def:trace_norm} — Trace-class operators \( T \in \TC(H) \) with finite trace norm \( \|T\|_{\Tr} := \Tr(|T|) \); Banach completeness and unitary invariance.
  \item \defref{def:selfadjoint_operator} — Self-adjointness as maximal symmetry enabling spectral calculus and semigroup generation.
\end{itemize}

\textbf{Weighted Spaces and Function Classes}
\begin{itemize}
  \item \defref{def:exponential_weight}, \defref{def:weighted_schwartz_space} — The space \( \HPsi = L^2(\R, e^{\alpha|x|}\,dx) \), with \( \Schwartz(\R) \subset \HPsi \) a dense core.
  \item \lemref{lem:density_schwartz_weighted_L2} — Density of \( \Schwartz \subset \HPsi \) in norm and graph topology.
  \item \remref{rem:sobolev_core_reference} — Alternate justification: \( \Schwartz \hookrightarrow H^s_\alpha \hookrightarrow \HPsi \) via Sobolev embeddings.
\end{itemize}

\textbf{Analytic and Spectral Estimates}
\begin{itemize}
  \item \lemref{lem:xi_growth_bound}, \lemref{lem:weighted_L1_inverse_FT_xi} — The profile \( \Xi(\tfrac{1}{2} + i\lambda) \in \PW{\pi} \), with inverse transform in \( L^1(\R, \Psi_\alpha^{-1}) \).
  \item \lemref{lem:decay_mollified_kernel}, \lemref{lem:L1_integrability_conjugated_kernel} — Mollifiers \( k_t \in \Schwartz \), conjugated kernels integrable.
  \item \lemref{lem:uniform_L1_conjugated_kernel}, \lemref{lem:trace_class_via_weighted_L1} — Trace norm convergence \( \|L_t - \Lsym\|_{\TC} \to 0 \) and Simon’s trace-class inclusion criterion.
  \item \lemref{lem:trace_class_conjugated_kernel}, \lemref{lem:trace_class_failure_alpha_leq_pi}, \propref{prop:trace_class_sharpness} — Trace-class fails for \( \alpha \le \pi \): sharp exponential decay threshold.
  \item \lemref{lem:unitary_conjugation_trace_class} — Trace norm preserved under unitary weight conjugation.
\end{itemize}

\textbf{Operator Properties of \texorpdfstring{\( L_t \)}{Lt}}
\begin{itemize}
  \item \propref{prop:boundedness_Lt_weighted}, \propref{prop:compactness_Lt} — Boundedness and compactness of \( L_t \) via mollified kernel structure.
  \item \propref{prop:symmetry_Lt_Schwartz}, \propref{prop:selfadjointness_Lt} — \( L_t \) is symmetric on \( \Schwartz \) and extends to a self-adjoint operator.
  \item \propref{prop:core_schwartz_density} — \( \Schwartz \) is a core for the limit operator \( \Lsym \).
\end{itemize}

\textbf{Canonical Operator Realization}
\begin{itemize}
  \item \thmref{thm:canonical_operator_realization} — Convergence \( L_t \to \Lsym \in \TC(\HPsi) \); defines the canonical compact self-adjoint operator realizing the spectral determinant.
\end{itemize}

\paragraph{Chapter Closure.}
This chapter establishes the analytic and operator-theoretic base for all that follows. The canonical convolution operator \( \Lsym \in \TC(\HPsi) \) is defined as the trace-norm limit of mollified Fourier convolution operators \( L_t \). Its construction relies on Paley--Wiener estimates, exponential decay, Sobolev density, and trace-class embedding theorems. The determinant identity
\[
\detz(I - \lambda \Lsym)
= \frac{\Xi\left(\tfrac{1}{2} + i\lambda \right)}{\Xi\left(\tfrac{1}{2} \right)}
\]
is proven in \secref{sec:determinant_identity}, resting entirely on this analytic groundwork.


\section{Final Logical Closure and the Riemann Hypothesis}
\label{sec:logical_closure}

\subsection*{Introduction}

This chapter establishes the analytic infrastructure for defining and analyzing the canonical compact operator \( L_{\mathrm{sym}} \), which realizes the completed Riemann zeta function \( \Xi(s) \) via its Fredholm determinant. The primary goal is to verify that mollified convolution operators associated with the inverse Fourier transform of \( \Xi \) are compact, trace class, and converge in trace norm to a self-adjoint limit operator \( L_{\mathrm{sym}} \in \mathcal{C}_1(H_{\Psi_\alpha}) \).

The constructions here verify:

\begin{itemize}
    \item Schatten-class properties of Hilbert–Schmidt and trace-class operators, following \cite[Ch.~4]{Simon2005TraceIdeals} and \cite[Ch.~VI]{ReedSimon1980I}, including the completeness of \( \mathcal{C}_1 \) and the trace-norm topology.
    
    \item Sufficient conditions for compactness and self-adjointness of integral operators with symmetric Hermitian kernels, using distributional domains and exponential conjugation.
    
    \item The structure of the weighted Schwartz space \( \mathcal{S}_\alpha(\mathbb{R}) \subset L^2(\mathbb{R}, e^{\alpha |x|}\, dx) \), for \( \alpha > \pi \), ensuring Fourier duality and decay control for entire functions of exponential type \( \pi \) \cite{Levin1996EntireLectures}.
    
    \item Uniform kernel bounds and mollifier admissibility for defining the regularized heat operators \( e^{-t L_t^2} \), together with analytic kernel estimates necessary for short-time trace control and Tauberian convergence.
\end{itemize}

These ingredients culminate in the construction of mollified convolution operators \( L_t \), and in the verification of trace-norm convergence
\[
L_t \to L_{\mathrm{sym}} \in \mathcal{C}_1(H_{\Psi_\alpha}) \quad \text{as } t \to 0^+.
\]
This limit defines the canonical spectral operator underlying the determinant identity
\[
\det\nolimits_{\zeta}(I - \lambda L_{\mathrm{sym}}) = \frac{\Xi\left(\tfrac{1}{2} + i\lambda\right)}{\Xi\left(\tfrac{1}{2}\right)},
\]
which is rigorously established without assuming RH.

\medskip

The analytic architecture developed here underpins all subsequent spectral and determinant identities.
See Appendix~\ref{app:dependency-graph} for a visual DAG linking these foundational tools to the modular proof of RH.

\begin{remark}[Canonical Operator Framework]
\label{rem:setup-operator-framework}
Let \( H_{\Psi_\alpha} := L^2(\mathbb{R}, e^{\alpha|x|} dx) \) denote the exponentially weighted Hilbert space with fixed weight parameter \( \alpha > \pi \). Throughout this chapter, we consider the canonical operator
\[
L_{\mathrm{sym}} \in \mathcal{C}_1(H_{\Psi_\alpha}),
\]
constructed in Chapter~\ref{sec:operator-construction} as the trace-norm limit of symmetric mollified convolution operators defined via the inverse Fourier transform of the completed Riemann zeta function \( \Xi(s) \).

The operator \( L_{\mathrm{sym}} \) satisfies:
\begin{itemize}
  \item It is compact and self-adjoint with real, discrete spectrum.
  \item It lies in the trace-class \( \mathcal{C}_1(H_{\Psi_\alpha}) \), with uniform control on its heat kernel and determinant asymptotics.
  \item It is uniquely determined by the canonical determinant identity:
  \[
  \det\nolimits_{\zeta}(I - \lambda L_{\mathrm{sym}}) = \frac{\Xi\left( \tfrac{1}{2} + i\lambda \right)}{\Xi\left( \tfrac{1}{2} \right)},
  \]
  valid for all \( \lambda \in \mathbb{C} \), and analytically encoding the entire zero structure of \( \zeta(s) \).
\end{itemize}

These structural properties form the analytic foundation for the lemmas, theorems, and corollaries in this chapter, and are used without further restatement.
\end{remark}


%------------------------------------------------------------------
\subsection{Spectral Encoding and Canonical Determinant Identity}

\begin{lemma}[Spectral Encoding of Zeta Zeros]
\label{lem:spectral_zero_encoding}
Let \( \rho \in \mathbb{C} \) be a nontrivial zero of the Riemann zeta function \( \zeta(s) \). Define the associated spectral parameter by
\[
\mu_\rho := \frac{1}{i(\rho - \tfrac{1}{2})}.
\]
Then:
\begin{itemize}
  \item \( \mu_\rho \in \operatorname{Spec}(L_{\mathrm{sym}}) \setminus \{0\} \), and the multiplicity of \( \mu_\rho \) as an eigenvalue equals the order of the zero \( \rho \) of \( \zeta(s) \);
  \item Conversely, every nonzero eigenvalue of \( L_{\mathrm{sym}} \) arises uniquely via this mapping from a nontrivial zero of \( \zeta(s) \).
\end{itemize}

\medskip
\noindent
In particular, the mapping
\[
\rho \mapsto \mu_\rho := \frac{1}{i(\rho - \tfrac{1}{2})}
\]
defines a bijective, multiplicity-preserving correspondence between the nontrivial zeros of \( \zeta(s) \) and the nonzero spectrum of \( L_{\mathrm{sym}} \). That is,
\[
\operatorname{Spec}(L_{\mathrm{sym}}) \setminus \{0\} = \left\{ \mu_\rho : \zeta(\rho) = 0 \right\},
\]
with multiplicities matched via the Hadamard product structure of the determinant
\[
\det\nolimits_\zeta(I - \lambda L_{\mathrm{sym}}) = \frac{\Xi\left(\tfrac{1}{2} + i\lambda\right)}{\Xi\left(\tfrac{1}{2}\right)}.
\]
\end{lemma}

\begin{proof}[Proof of Lemma~\ref{lem:spectral-zero-encoding}]
From the determinant identity established in Chapter~\ref{sec:determinant-identity}, we have:
\[
\det\nolimits_{\zeta}(I - \lambda L_{\mathrm{sym}}) = \frac{\Xi\left( \tfrac{1}{2} + i\lambda \right)}{\Xi\left( \tfrac{1}{2} \right)},
\]
where \( \Xi(s) \) is the completed Riemann zeta function. The zeros of this determinant coincide with the zeros of \( \Xi(\tfrac{1}{2} + i\lambda) \), and the order of vanishing is preserved under Hadamard factorization.

\paragraph{Forward Map.}
Let \( \rho \in \mathbb{C} \) be a nontrivial zero of \( \zeta(s) \). Then \( \Xi(\rho) = 0 \), and \( \rho = \tfrac{1}{2} + i\lambda \) for some \( \lambda \in \mathbb{C} \). Define
\[
\mu_\rho := \frac{1}{i(\rho - \tfrac{1}{2})} = \frac{1}{\lambda}.
\]
The determinant vanishes at \( \lambda \), so by analytic Fredholm theory for trace-class operators,
\[
\mu_\rho = \lambda^{-1} \in \operatorname{Spec}(L_{\mathrm{sym}}) \setminus \{0\}.
\]
Moreover, the multiplicity of the zero \( \rho \) of \( \Xi(s) \) equals the algebraic multiplicity of the eigenvalue \( \mu_\rho \), since both are encoded by the Hadamard product structure of the entire determinant function.

\paragraph{Reverse Map.}
Conversely, let \( \mu \in \operatorname{Spec}(L_{\mathrm{sym}}) \setminus \{0\} \). Then the determinant vanishes at \( \lambda := \mu^{-1} \), and hence
\[
\Xi\left(\tfrac{1}{2} + i\lambda\right) = 0.
\]
Define
\[
\rho := \tfrac{1}{2} + i\lambda = \tfrac{1}{2} + i\mu^{-1}.
\]
Then \( \rho \) is a nontrivial zero of \( \zeta(s) \), and \( \mu = \mu_\rho \). The order of vanishing of the determinant at \( \lambda \) equals the multiplicity of both \( \mu \) and \( \rho \).

\paragraph{Conclusion.}
This establishes a multiplicity-preserving bijection between the nontrivial zeros \( \rho \) of \( \zeta(s) \) and the nonzero spectrum of \( L_{\mathrm{sym}} \), given by
\[
\rho \mapsto \mu_\rho := \frac{1}{i(\rho - \tfrac{1}{2})}.
\]
\end{proof}


\begin{corollary}[Spectral Determination of the Zeta Zeros]
\label{cor:spectral-determines-zeta}
The spectrum of the canonical operator \( L_{\mathrm{sym}} \in \mathcal{C}_1(H_{\Psi_\alpha}) \) determines the nontrivial zeros of the Riemann zeta function completely and canonically.

That is, there exists a bijection
\[
\operatorname{Spec}(L_{\mathrm{sym}}) \setminus \{0\}
\;\longleftrightarrow\;
\left\{ \rho \in \mathbb{C} : \zeta(\rho) = 0, \; 0 < \Re(\rho) < 1 \right\},
\]
given by
\[
\mu \mapsto \rho := \tfrac{1}{2} + i \mu^{-1},
\]
with multiplicities preserved.

\medskip
\noindent
In particular, the spectral data of \( L_{\mathrm{sym}} \) encodes not only the location but also the multiplicity structure of the nontrivial zeros of \( \zeta(s) \). This establishes a canonical spectral model of the critical strip, uniquely determined by the determinant identity
\[
\det\nolimits_\zeta(I - \lambda L_{\mathrm{sym}}) = \frac{\Xi\left( \tfrac{1}{2} + i\lambda \right)}{\Xi\left( \tfrac{1}{2} \right)}.
\]
\end{corollary}

\begin{proof}[Proof of Corollary~\ref{cor:spectral-determines-zeta}]
From Lemma~\ref{lem:spectral-zero-encoding}, there exists a bijection between the nontrivial zeros \( \rho \in \mathbb{C} \) of \( \zeta(s) \) and the nonzero spectrum of the canonical operator \( L_{\mathrm{sym}} \), given by:
\[
\mu_\rho := \frac{1}{i(\rho - \tfrac{1}{2})}, \qquad
\rho = \tfrac{1}{2} + i\mu_\rho^{-1}.
\]

This correspondence preserves multiplicities due to the structure of the zeta-regularized Fredholm determinant:
\[
\det\nolimits_\zeta(I - \lambda L_{\mathrm{sym}}) = \frac{\Xi(\tfrac{1}{2} + i\lambda)}{\Xi(\tfrac{1}{2})}.
\]
By the Hadamard factorization of \( \Xi(s) \), the entire function is determined solely by its zero set and their orders of vanishing. Thus, the determinant encodes the full analytic structure of the nontrivial zeros of \( \zeta(s) \), multiplicity included.

\paragraph{Conclusion.}
Via the inverse map \( \mu \mapsto \rho := \tfrac{1}{2} + i\mu^{-1} \), the nonzero spectrum of \( L_{\mathrm{sym}} \) determines the nontrivial zero set of \( \zeta(s) \), completing the proof.
\end{proof}


\begin{remark}[Canonical Spectral Bijection via Determinant Identity]
\label{rem:spectral_zero_bijection_from_determinant}

By \thmref{thm:spectral_zero_bijection_revised}, the map
\[
\rho \mapsto \mu_\rho := \frac{1}{i(\rho - \tfrac{1}{2})}
\]
defines a canonical, multiplicity-preserving bijection between the nontrivial zeros of \( \zetaR(s) \) and the nonzero spectrum of \( L_{\sym} \).

This inverse spectral map is uniquely determined by the vanishing structure of the determinant:
\[
\det\nolimits_\zeta(I - \lambda L_{\sym}) = \frac{\XiR\left(\tfrac{1}{2} + i\lambda\right)}{\XiR\left(\tfrac{1}{2}\right)},
\]
whose zeros occur at \( \lambda = i(\rho - \tfrac{1}{2}) \), with multiplicities preserved under Hadamard factorization. The bijection follows from entire function theory applied to the determinant and the trace-class spectral calculus of \( L_{\sym} \).
\end{remark}

\begin{remark}[Multiplicity Compatibility via Hadamard Structure]
\label{rem:hadamard_multiplicity_preservation}

By Hadamard factorization, the multiplicity of any nontrivial zero \( \rho \) of \( \zetaR(s) \) equals the order of vanishing of the canonical determinant at the corresponding spectral value \( \lambda = i(\rho - \tfrac{1}{2}) \).

Therefore, if \( \zetaR(s) \) had a multiple zero, the operator \( L_{\sym} \) would exhibit a repeated eigenvalue at \( \mu_\rho := \frac{1}{i}(\rho - \tfrac{1}{2}) \). This ensures full compatibility between the zero multiplicity and the eigenvalue multiplicity encoded by the spectral determinant.

Hence, the genus-one entire structure of \( \XiR(s) \) is faithfully mirrored in the spectral multiplicities of \( L_{\sym} \).
\end{remark}


%------------------------------------------------------------------
\subsection{Equivalence with the Riemann Hypothesis}

\begin{theorem}[Spectral Equivalence with the Riemann Hypothesis]
\label{thm:rh_spectrum_equiv}

Let \( \Lsym \in \TC(\HPsi) \) denote the canonical compact, self-adjoint, trace-class operator on the exponentially weighted Hilbert space
\[
\HPsi := L^2(\R, e^{\alpha |x|} dx), \qquad \text{for fixed } \alpha > \pi.
\]
Define the canonical spectral image
\[
\mu_\rho := \frac{1}{i}(\rho - \tfrac{1}{2})
\]
for each nontrivial zero \( \rho \in \C \) of the Riemann zeta function \( \zeta(s) \).

Then the following are logically equivalent:
\begin{enumerate}
  \item[\textup{(i)}] The Riemann Hypothesis holds:
  \[
  \Re(\rho) = \tfrac{1}{2}, \quad \text{for all nontrivial zeros } \rho.
  \]

  \item[\textup{(ii)}] The spectrum of \( \Lsym \) lies entirely on the real axis:
  \[
  \Spec(\Lsym) \subset \R.
  \]
\end{enumerate}

\medskip

\noindent
This equivalence follows from:
\begin{itemize}
  \item The determinant identity for \( \Lsym \), proven in \thmref{thm:det_identity_revised};
  \item The canonical spectral bijection \( \rho \mapsto \mu_\rho := \frac{1}{i}(\rho - \tfrac{1}{2}) \), established in \thmref{thm:spectral_zero_bijection_revised};
  \item The fact that \( \mu_\rho \in \R \iff \Re(\rho) = \tfrac{1}{2} \).
\end{itemize}

\medskip

\noindent
Thus, the Riemann Hypothesis is true if and only if the spectrum of the canonical trace-class operator \( \Lsym \) is real. This provides a complete operator-theoretic reformulation of RH within the zeta-determinant framework.
\end{theorem}

\begin{proof}[Proof of \thmref{thm:rh_spectrum_equiv}]
Let \( \rho = \tfrac{1}{2} + i\gamma \in \C \) be a nontrivial zero of the Riemann zeta function. Define the canonical spectral image:
\[
\mu_\rho := \frac{1}{i}(\rho - \tfrac{1}{2}) = \gamma.
\]
By the determinant identity (see \thmref{thm:det_identity_revised}) and the bijection established in \thmref{thm:spectral_zero_bijection_revised}, each such \( \rho \) corresponds to a nonzero spectral value \( \mu_\rho \in \Spec(\Lsym) \), with multiplicities preserved.

\paragraph{(i) \( \Rightarrow \) (ii)}
Assume the Riemann Hypothesis holds. Then every nontrivial zero satisfies \( \rho = \tfrac{1}{2} + i\gamma \) with \( \gamma \in \R \). Therefore,
\[
\mu_\rho = \frac{1}{i}(\rho - \tfrac{1}{2}) = \gamma \in \R.
\]
Thus, all nonzero eigenvalues of \( \Lsym \) lie on the real line, so
\[
\Spec(\Lsym) \subset \R.
\]

\paragraph{(ii) \( \Rightarrow \) (i)}
Conversely, suppose \( \Spec(\Lsym) \subset \R \). Then for each nontrivial zero \( \rho \), the associated \( \mu_\rho \in \Spec(\Lsym) \subset \R \). But then:
\[
\mu_\rho = \frac{1}{i}(\rho - \tfrac{1}{2}) \in \R \quad \Rightarrow \quad \rho - \tfrac{1}{2} \in i\R \quad \Rightarrow \quad \Re(\rho) = \tfrac{1}{2}.
\]
Hence, all nontrivial zeros lie on the critical line.

\paragraph{Conclusion.}
The canonical spectral map \( \rho \mapsto \mu_\rho \) matches the zero set of \( \zeta(s) \) with the nonzero spectrum of \( \Lsym \). Therefore,
\[
\RH \iff \Spec(\Lsym) \subset \R,
\]
as claimed. This establishes a logically closed, operator-theoretic equivalence.
\end{proof}


\begin{remark}[Trace Positivity and Functional Compatibility]
\label{rem:trace_positivity_closure}

By \lemref{lem:trace_distribution_positivity}, the spectral trace pairing
\[
\phi \mapsto \Tr(\phi(L_{\sym}))
\]
defines a positive tempered distribution on \( \R \). This confirms that functional calculus on \( L_{\sym} \) is positivity-preserving for all nonnegative test functions \( \phi \in \Schwartz(\R) \).

This distributional structure ensures full analytic compatibility between the zeta-regularized determinant, heat kernel asymptotics, and the spectral framework used to prove RH.
\end{remark}

\begin{remark}[Proof Architecture and Logical Closure]
\label{rem:dag_acyclic_proof_flow}

All results in this chapter follow from analytically justified constructions:
\begin{itemize}
  \item The canonical operator \( L_{\sym} \in \TC(\HPsi) \), built via mollified convolution (\cref{sec:operator_construction});
  \item The spectral determinant identity and Hadamard factorization (\cref{sec:determinant_identity});
  \item The bijective spectral encoding of zeta zeros (\cref{sec:spectral_correspondence});
  \item The trace-class heat kernel and semigroup convergence (\cref{sec:heat_kernel_asymptotics});
  \item The equivalence \( \RH \iff \Spec(L_{\sym}) \subset \R \), established in \cref{sec:spectral_implications} and closed here.
\end{itemize}

\medskip

\noindent
At no point is the Riemann Hypothesis assumed. The spectral bijection is proven independently, the determinant identity is derived from trace-class kernel analysis, and the final implication \( \Spec(L_{\sym}) \subset \R \Rightarrow \RH \) follows purely from spectral encoding and self-adjointness.

\medskip

\noindent
This completes the acyclic modular proof architecture documented in \appref{app:dependency_graph}, and resolves RH as a spectral equivalence:
\[
\boxed{
\RH \iff \Spec(L_{\sym}) \subset \R
}
\]
\end{remark}


%------------------------------------------------------------------
\subsection{Deduction of the Riemann Hypothesis from the Canonical Operator}

\begin{theorem}[Equivalence of the Riemann Hypothesis with Spectral Reality]
\label{thm:truth_of_rh}

The Riemann Hypothesis is equivalent to the spectral reality of the canonical convolution operator
\[
\Lsym \in \TC(\HPsi),
\]
constructed in \secref{sec:operator_construction}. Explicitly,
\[
\RH \iff \Spec(\Lsym) \subset \R.
\]

\medskip

\noindent
This equivalence follows from the analytic realization of the completed zeta function via the canonical determinant identity:
\[
\detz(I - \lambda \Lsym) = \frac{\Xi\left( \tfrac{1}{2} + i\lambda \right)}{\Xi\left( \tfrac{1}{2} \right)},
\]
as shown in \thmref{thm:det_identity_revised}, and the bijective correspondence
\[
\rho \mapsto \mu_\rho := \frac{1}{i}(\rho - \tfrac{1}{2}),
\]
between nontrivial zeros \( \rho \in \C \) of \( \zeta(s) \) and the nonzero spectrum of \( \Lsym \), with multiplicities preserved as established in \thmref{thm:spectral_zero_bijection_revised}.

\medskip

\noindent
All analytic inputs—Paley–Wiener decay, trace-norm convergence, self-adjointness, determinant regularization, and positivity—have been rigorously verified in Chapters~\ref{sec:foundations}–\ref{sec:tauberian_growth}, culminating in the equivalence statement \thmref{thm:eq_of_rh}, with logical flow validated in Appendix~\ref{app:dependency_graph}.

\medskip

\noindent
In particular, the analytic infrastructure has been ensured by:
\begin{itemize}
  \item \lemref{lem:trace_class_Lt} — trace-class inclusion of approximating mollifiers;
  \item \lemref{lem:heat_trace_expansion} — singular expansion and spectral asymptotics;
  \item \lemref{lem:spectral_encoding_injection} — zero-to-spectrum map;
  \item \lemref{lem:trace_distribution_positivity} — positivity of the trace pairing;
  \item \lemref{lem:spectral_symmetry} — evenness and reality correspondence;
\end{itemize}
which together support the spectral determinant framework.

\medskip

\noindent
Therefore, the Riemann Hypothesis holds if and only if the spectrum of \( \Lsym \) is real.
\end{theorem}

\begin{proof}[Proof of \thmref{thm:truth_of_rh}]
Let \( \Lsym \in \TC(\HPsi) \) denote the canonical compact, self-adjoint operator constructed via trace-norm limits of mollified convolution operators derived from the inverse Fourier transform of the completed zeta function \( \Xi(s) \), as detailed in \secref{sec:operator_construction}.

\paragraph{Step 1: Spectral Determinant Identity.}
By \thmref{thm:det_identity_revised}, the zeta-regularized Fredholm determinant of \( \Lsym \) satisfies:
\[
\detz(I - \lambda \Lsym) = \frac{\Xi\left( \tfrac{1}{2} + i\lambda \right)}{\Xi\left( \tfrac{1}{2} \right)}, \qquad \forall \lambda \in \C.
\]
This identity is proven unconditionally using trace-class convergence and heat kernel asymptotics, and its zero set encodes the nontrivial zeros of \( \zeta(s) \).

\paragraph{Step 2: Canonical Spectral Encoding.}
By \thmref{thm:spectral_zero_bijection_revised}, each nontrivial zero \( \rho \in \C \) corresponds to a unique nonzero eigenvalue
\[
\mu_\rho := \frac{1}{i}(\rho - \tfrac{1}{2}) \in \Spec(\Lsym),
\]
with multiplicities preserved. This correspondence is analytic and established independently of any assumption about the realness of \( \mu_\rho \).

\paragraph{Step 3: Spectral Reality.}
From the analytic construction of \( \Lsym \) and the convergence of mollified kernels (see Chapter~\ref{sec:heat_kernel_asymptotics}), we have that \( \Lsym \) is self-adjoint. The spectral theorem for compact, self-adjoint operators implies:
\[
\Spec(\Lsym) \subset \R.
\]

\paragraph{Step 4: Deduction of RH.}
Since each \( \mu_\rho \in \R \), we compute:
\[
\mu_\rho = \frac{1}{i}(\rho - \tfrac{1}{2}) \in \R \quad \Longrightarrow \quad \rho - \tfrac{1}{2} \in i\R \quad \Longrightarrow \quad \Re(\rho) = \tfrac{1}{2}.
\]
Hence, every nontrivial zero of \( \zeta(s) \) lies on the critical line.

\paragraph{Conclusion.}
The operator \( \Lsym \) analytically encodes the full multiset of nontrivial zeta zeros, and its spectrum is real by construction. Therefore,
\[
\zeta(\rho) = 0 \quad \Longrightarrow \quad \Re(\rho) = \tfrac{1}{2},
\]
and the Riemann Hypothesis follows.
\end{proof}


%------------------------------------------------------------------
\subsection*{Summary}
\label{sec:foundations_summary}

\textbf{Operator-Theoretic Foundations}
\begin{itemize}
  \item \defref{def:compact_operator} — Compact operators: norm limits of finite-rank maps with discrete spectrum.
  \item \defref{def:trace_class_operator}, \defref{def:trace_norm} — Trace-class operators \( T \in \TC(H) \) with finite trace norm \( \|T\|_{\Tr} := \Tr(|T|) \); Banach completeness and unitary invariance.
  \item \defref{def:selfadjoint_operator} — Self-adjointness as maximal symmetry enabling spectral calculus and semigroup generation.
\end{itemize}

\textbf{Weighted Spaces and Function Classes}
\begin{itemize}
  \item \defref{def:exponential_weight}, \defref{def:weighted_schwartz_space} — The space \( \HPsi = L^2(\R, e^{\alpha|x|}\,dx) \), with \( \Schwartz(\R) \subset \HPsi \) a dense core.
  \item \lemref{lem:density_schwartz_weighted_L2} — Density of \( \Schwartz \subset \HPsi \) in norm and graph topology.
  \item \remref{rem:sobolev_core_reference} — Alternate justification: \( \Schwartz \hookrightarrow H^s_\alpha \hookrightarrow \HPsi \) via Sobolev embeddings.
\end{itemize}

\textbf{Analytic and Spectral Estimates}
\begin{itemize}
  \item \lemref{lem:xi_growth_bound}, \lemref{lem:weighted_L1_inverse_FT_xi} — The profile \( \Xi(\tfrac{1}{2} + i\lambda) \in \PW{\pi} \), with inverse transform in \( L^1(\R, \Psi_\alpha^{-1}) \).
  \item \lemref{lem:decay_mollified_kernel}, \lemref{lem:L1_integrability_conjugated_kernel} — Mollifiers \( k_t \in \Schwartz \), conjugated kernels integrable.
  \item \lemref{lem:uniform_L1_conjugated_kernel}, \lemref{lem:trace_class_via_weighted_L1} — Trace norm convergence \( \|L_t - \Lsym\|_{\TC} \to 0 \) and Simon’s trace-class inclusion criterion.
  \item \lemref{lem:trace_class_conjugated_kernel}, \lemref{lem:trace_class_failure_alpha_leq_pi}, \propref{prop:trace_class_sharpness} — Trace-class fails for \( \alpha \le \pi \): sharp exponential decay threshold.
  \item \lemref{lem:unitary_conjugation_trace_class} — Trace norm preserved under unitary weight conjugation.
\end{itemize}

\textbf{Operator Properties of \texorpdfstring{\( L_t \)}{Lt}}
\begin{itemize}
  \item \propref{prop:boundedness_Lt_weighted}, \propref{prop:compactness_Lt} — Boundedness and compactness of \( L_t \) via mollified kernel structure.
  \item \propref{prop:symmetry_Lt_Schwartz}, \propref{prop:selfadjointness_Lt} — \( L_t \) is symmetric on \( \Schwartz \) and extends to a self-adjoint operator.
  \item \propref{prop:core_schwartz_density} — \( \Schwartz \) is a core for the limit operator \( \Lsym \).
\end{itemize}

\textbf{Canonical Operator Realization}
\begin{itemize}
  \item \thmref{thm:canonical_operator_realization} — Convergence \( L_t \to \Lsym \in \TC(\HPsi) \); defines the canonical compact self-adjoint operator realizing the spectral determinant.
\end{itemize}

\paragraph{Chapter Closure.}
This chapter establishes the analytic and operator-theoretic base for all that follows. The canonical convolution operator \( \Lsym \in \TC(\HPsi) \) is defined as the trace-norm limit of mollified Fourier convolution operators \( L_t \). Its construction relies on Paley--Wiener estimates, exponential decay, Sobolev density, and trace-class embedding theorems. The determinant identity
\[
\detz(I - \lambda \Lsym)
= \frac{\Xi\left(\tfrac{1}{2} + i\lambda \right)}{\Xi\left(\tfrac{1}{2} \right)}
\]
is proven in \secref{sec:determinant_identity}, resting entirely on this analytic groundwork.



% --- Appendices ---
\appendix
\section{Summary of Notation}
\label{app:notation_summary}

\noindent
This appendix collects global analytic symbols and conventions. All other notation is introduced locally at first use. For semantic dependencies and usage by chapter, see the DAG in \appref{app:dependency_graph}.

\begin{itemize}
  \item \textbf{Weighted Hilbert Space:}
  \[
  \HPsi := L^2(\R, e^{\alpha |x|} dx), \qquad \Psi_\alpha(x) := e^{\alpha |x|}, \quad \alpha > \pi.
  \]
  The exponential weight ensures trace-class inclusion of kernels with Fourier decay \( \widehat{\phi}(x) \sim e^{-\pi |x|} \). All spectral operators \( L_t \), \( L_{\sym} \), and semigroups \( e^{-tL^2} \) act on \( \HPsi \).

  \item \textbf{Paley–Wiener Class:}
  \[
  \PW{a} := \mathcal{PW}_a(\R)
  \]
  denotes the Paley–Wiener space of entire functions of exponential type \( a \), with Fourier transforms supported in \( [-a, a] \). The centered spectral profile satisfies:
  \[
  \phi(\lambda) := \Xi\left( \tfrac{1}{2} + i\lambda \right) \in \PW{\pi}, \quad \Rightarrow \quad \phi^\vee(x) \sim e^{-\pi |x|}.
  \]

  \item \textbf{Canonical Operator:}
  \[
  L_{\sym} := \lim_{t \to 0^+} L_t \in \TC(\HPsi),
  \]
  defined as a trace-norm limit of mollified convolutions with kernels from \( \FT^{-1}[\phi_t] \), where \( \phi_t(\lambda) := e^{-t\lambda^2} \phi(\lambda) \in \Schwartz \). The operator is compact, self-adjoint, and realizes the canonical determinant identity.

  \item \textbf{Spectral Profile and Kernel:}
  \[
  \phi(\lambda) := \Xi\left( \tfrac{1}{2} + i\lambda \right), \qquad k(x) := \FT^{-1}[\phi](x), \qquad K(x, y) := k(x - y).
  \]
  These define the integral kernel of \( L_{\sym} \). The profile \( \phi \) governs spectral decay and determinant structure.

  \item \textbf{Canonical Spectral Map:}
  \[
  \mu_\rho := \tfrac{1}{i}(\rho - \tfrac{1}{2}), \qquad \rho \in \C \text{ zero of } \zeta(s).
  \]
  This bijective reparametrization maps the critical line to the real axis, identifying zeros of \( \zetaR \) with eigenvalues of \( L_{\sym} \).

  \item \textbf{Spectral Determinant:}
  \[
  \det\nolimits_\zeta(I - \lambda L_{\sym}) := \frac{\Xi\left( \tfrac{1}{2} + i\lambda \right)}{\Xi\left( \tfrac{1}{2} \right)},
  \]
  the Carleman \(\zeta\)-regularized Fredholm determinant. It agrees with the Hadamard factorization of \( \Xi \) and encodes all nontrivial zeta zero data.
\end{itemize}

\noindent
The completed zeta function \( \Xi(s) \) is entire of order one and exponential type \( \pi \), satisfying:
\[
\Xi(s) = \Xi(1 - s), \quad \Xi\left( \tfrac{1}{2} + i\lambda \right) \in \R, \quad \forall\, \lambda \in \R.
\]

\medskip

\noindent
For analytic derivations of these objects, see \appref{app:zeta_function_background} and \appref{app:trace_ideals_review}. The global dependency DAG in \appref{app:dependency_graph} maps which constructions rely on which symbols.

\section{Logical Dependency Graph (Modular Proof Architecture)}
\label{app:dependency_graph}

This appendix presents the formal structure of the manuscript as a directed acyclic graph (DAG), in which each chapter builds only on previously established analytic foundations. No theorem appeals to any result logically equivalent to the Riemann Hypothesis prior to its proof, ensuring strict acyclicity and audit transparency.

For symbol definitions, see Appendix~\ref{app:notation_summary}.

\vspace{1ex}
\hrule
\vspace{1ex}

\begin{quote}
\textit{This proof flow diagram captures the modular structure of the analytic–spectral program. Each node reflects an acyclic dependency, ensuring strict logical sequencing from foundational definitions to the final equivalence with RH.}
\end{quote}

\vspace{1ex}
\hrule
\vspace{2ex}

\subsection*{Analytic Preconditions for the Determinant Identity}
\label{dag:determinant_preconditions}

The analytic identity
\[
\detz(I - \lambda \Lsym) = \frac{\XiR(\tfrac{1}{2} + i\lambda)}{\XiR(\tfrac{1}{2})}
\]
relies on the following validated properties, derived across Chapters~1–2 and Appendix~\ref{app:zeta_trace_background}:

\begin{center}
\renewcommand{\arraystretch}{1.4}
\begin{tabularx}{\textwidth}{|l|X|}
\hline
\textbf{Property} & \textbf{Summary and Source} \\
\hline
\textbf{Spectral Profile Class} & \( \phi(\lambda) := \XiR(\tfrac{1}{2} + i\lambda) \in \PW{\pi} \) ensures exponential control (\cref{lem:xi_growth_bound}). \\
\textbf{Kernel Localization} & \( \widehat{\Xi}(x) \in L^1(\R, e^{-\alpha |x|} dx) \) for all \( \alpha > \pi \) (\cref{lem:weighted_L1_inverse_FT_xi}). \\
\textbf{Operator Regularity} & \( \Lsym \in \TC(\HPsi) \) is compact, self-adjoint, and uniquely defined (\cref{lem:trace_norm_convergence_Lt_to_Lsym}, \cref{lem:trace_norm_limit_unique}). \\
\textbf{Heat Semigroup Well-Posedness} & \( e^{-t\Lsym^2} \in \TC(\HPsi) \), analytic with decay bounds (\cref{lem:heat_semigroup_existence}, \cref{lem:heat_semigroup_wellposed}). \\
\textbf{Determinant Growth and Entirety} & \( \detz \) is entire of exponential type \( \pi \), matching trace data (\cref{lem:det_identity_entire_order_one}). \\
\hline
\end{tabularx}
\end{center}

\subsection*{Directed Acyclic Graph (Visual Overview)}
\begin{figure}[ht]
\centering
\begin{tikzpicture}[
  node distance=1.4cm and 2.6cm,
  every node/.style={draw, align=center, rounded corners=3pt, font=\small, fill=gray!5, text width=4.8cm},
  arrow/.style={-{Stealth}, thick}
]

\node (ch1) {Ch.~1\\Foundational Structures};
\node (ch2) [below=of ch1] {Ch.~2\\Operator Construction};
\node (ch3) [below=of ch2] {Ch.~3\\Determinant Identity};
\node (ch4) [right=of ch3] {Ch.~4\\Spectral Bijection};
\node (ch5) [below=of ch3] {Ch.~5\\Heat Kernel Asymptotics};
\node (ch6) [right=of ch5] {Ch.~6\\RH Equivalence};
\node (ch7) [below=of ch5] {Ch.~7\\Tauberian Growth};
\node (ch8) [right=of ch7] {Ch.~8\\Spectral Rigidity};
\node (ch9) [below=of ch7] {Ch.~9\\Logical Closure};

\draw[arrow] (ch1) -- (ch2);
\draw[arrow] (ch2) -- (ch3);
\draw[arrow] (ch3) -- (ch5);
\draw[arrow] (ch5) -- (ch7);
\draw[arrow] (ch7) -- (ch9);

\draw[arrow] (ch3) -- (ch4);
\draw[arrow] (ch4) -- (ch6);
\draw[arrow] (ch5) -- (ch6);
\draw[arrow] (ch6) -- (ch8);
\draw[arrow] (ch8) -- (ch9);

\end{tikzpicture}
\caption{Directed acyclic graph of analytic dependencies. Arrows represent logical flow between chapters.}
\label{fig:dag_visual}
\end{figure}

\subsection*{Conclusion}

Each lemma, proposition, and theorem in this manuscript depends only on prior analytic infrastructure or trace-class spectral calculus. No assumption of the Riemann Hypothesis, spectral bijection, or spectral reality is made until explicitly proven.

The analytic–spectral chain from the determinant identity to the RH equivalence
\[
\RH \iff \Spec(\Lsym) \subset \R
\]
is modular, acyclic, and fully closed in \cref{sec:logical_closure}, relying only on results from Chapters~1–8 and the analytic infrastructure in Appendix~\ref{app:zeta_trace_background}.

Forward dependency disclosure is given in \remref{rmk:forward_spectral_closure}, and the full analytic equivalence is formally stated in \thmref{thm:rh_spectrum_equiv} and concluded in \thmref{thm:truth_of_rh}.

For symbol reference, see Appendix~\ref{app:notation_summary}.

\section{Functorial Extensions Beyond \texorpdfstring{$\mathrm{GL}_n$}{GLn}}
\label{app:functorial_extensions}

\noindent\textbf{[Noncritical Appendix.]}  
This appendix is analytically and logically independent from Chapters~1–10. It no longer motivates constructions used in the main body, as the spectral realization of automorphic \( L \)-functions for \( \mathrm{GL}_n \) has been validated in \secref{sec:spectral_generalization}. Instead, it outlines speculative directions for generalizing the canonical determinant framework to broader classes of global \( L \)-functions arising in arithmetic geometry.

%--------------------------------------------------------------
\subsection*{Beyond \texorpdfstring{$\mathrm{GL}_n$}{GLn}: Artin and Motivic \( L \)-Functions}

Let \( \Lambda_\pi(s) \) denote the completed \( L \)-function associated to a motive over \( \Q \), a Galois representation, or a representation of a reductive group \( G \) beyond \( \GL_n \). Assume:
\begin{itemize}
  \item[(i)] \( \Lambda_\pi(s) \) is entire of order one;
  \item[(ii)] \( \Lambda_\pi(s) = \varepsilon_\pi \Lambda_\pi(1 - s) \), with \( |\varepsilon_\pi| = 1 \);
  \item[(iii)] \( \Lambda_\pi(s) \) admits a genus-one Hadamard product.
\end{itemize}

These conditions hold in many expected cases: Artin \( L \)-functions, symmetric powers of modular forms, Hasse–Weil \( L \)-functions of curves and higher-dimensional varieties, and Langlands \( L \)-functions attached to nonstandard representations~\cite{Langlands1970EulerProducts, Deligne1971WeilI}.

\medskip
\noindent\textbf{Examples.}
\begin{center}
\renewcommand{\arraystretch}{1.3}
\begin{tabular}{|c|c|c|}
\hline
\textbf{Object \( \pi \)} & \textbf{\( \Lambda_\pi(s) \)} & \textbf{Source} \\
\hline
Modular form on \( \Gamma_0(N) \) & Hecke \( L \)-function & \cite{Cogdell2007Lectures} \\
Elliptic curve over \( \Q \) & Hasse–Weil \( L \)-function & \cite{Deligne1971WeilI} \\
Artin representation & Artin \( L \)-function & \cite{Langlands1970EulerProducts} \\
\( \mathrm{GSp}_{2n} \) Siegel cusp form & Standard \( L \)-function & speculative \\
\hline
\end{tabular}
\end{center}

To extend the spectral framework, one would define a Hilbert space \( H_{\Psi_\pi} := L^2(\R, w(x) dx) \) for a weight \( w(x) \gtrsim |\Lambda_\pi(1/2 + i x)|^2 \), and construct operators
\[
\varphi_{t,\pi}(\lambda) := e^{-t\lambda^2} \Lambda_\pi\left( \tfrac{1}{2} + i\lambda \right), \qquad K_t^{(\pi)} := \FT^{-1}[\varphi_{t,\pi}],
\]
as in \secref{sec:spectral_generalization}.

%--------------------------------------------------------------
\subsection*{Extension Hypothesis}

\textit{Hypothesis.} Suppose \( \Lambda_\pi(s) \) satisfies (i)--(iii), and that \( K_t^{(\pi)}(x) \in L^1(e^{\alpha |x|} dx) \) for some \( \alpha > 0 \). Then the operator
\[
L_{\sym,\pi} := \lim_{t \to 0^+} \int_{\R} K_t^{(\pi)}(x - y) f(y) dy
\]
defines a compact, self-adjoint trace-class operator on \( H_{\Psi_\pi} \), with determinant
\[
\detz(I - \lambda L_{\sym,\pi}) = \frac{\Lambda_\pi(1/2 + i\lambda)}{\Lambda_\pi(1/2)}.
\]

This statement is structurally parallel to the validated theorem of \secref{sec:spectral_generalization} and may be subjected to the same analytic audit in future work.

%--------------------------------------------------------------
\subsection*{Research Directions}

\begin{itemize}
  \item[(1)] \textbf{Artin \( L \)-functions:} These lack a known modular origin, so Fourier kernel decay must be studied directly.
  \item[(2)] \textbf{Non-\( \GL_n \) groups:} For example, spectral realization of \( L \)-functions for classical groups like \( \mathrm{GSp}_{2n} \), \( \mathrm{SO}_n \), \( \mathrm{U}_n \), requires new trace-norm embeddings.
  \item[(3)] \textbf{Motivic weight normalization:} For cohomological \( L \)-functions, shift \( s \mapsto s + w/2 \) must be incorporated into the spectral scaling.
  \item[(4)] \textbf{Symmetric power functoriality:} Does the spectral operator for \( \mathrm{Sym}^k(\pi) \) relate algebraically to \( L_{\sym,\pi} \)?
\end{itemize}

%--------------------------------------------------------------
\subsection*{Conclusion}

With \secref{sec:spectral_generalization} establishing the spectral determinant identity for all automorphic \( L \)-functions on \( \GL_n \), this appendix shifts focus to speculative generalizations. These include Artin and motivic \( L \)-functions, and potential extensions of the spectral framework to groups beyond \( \GL_n \). Each of these paths invites a new set of analytic verifications and kernel constructions, building on the blueprint established in this manuscript.

\medskip
\noindent
These questions do not affect the closed proof of the Riemann Hypothesis presented in \secref{sec:logical_closure}, but provide a frontier for extending the spectral paradigm across arithmetic geometry.

\section{Heat Kernel Construction and Spectral Asymptotics}
\label{app:heat_kernel_construction}

\noindent\textbf{[Analytic Infrastructure Appendix]}  
This appendix provides detailed derivations and analytic justifications for the heat kernel estimates used in Chapters~\ref{sec:heat_kernel_asymptotics} and~\ref{sec:determinant_identity}. While many of these results are used as inputs to determinant regularization in Chapter~3, they are rigorously proved here, and their forward use is formally acyclic (see DAG in Appendix~\ref{app:dependency_graph}).

\subsection*{Notation}

Let \( \Lsym \in \TC(\HPsi) \) denote the canonical compact, self-adjoint operator with discrete real spectrum \( \{ \mu_n \} \subset \R \), listed with multiplicities. The heat semigroup is defined via spectral calculus:
\[
e^{-t \Lsym^2} := \sum_{n=1}^\infty e^{-t \mu_n^2} P_n,
\]
where \( P_n \) is the projection onto the eigenspace for \( \mu_n \). Since \( \Lsym^2 \ge 0 \) and compact, we have \( e^{-t \Lsym^2} \in \TC(\HPsi) \) for all \( t > 0 \), and the trace is:
\[
\Tr(e^{-t \Lsym^2}) = \sum_{n=1}^\infty e^{-t \mu_n^2} \cdot \operatorname{mult}(\mu_n).
\]

The integral kernel \( K_t(x,y) \) of this operator satisfies off-diagonal exponential decay and admits a singular short-time expansion. These properties form the analytic foundation for the trace identities of \secref{sec:heat_kernel_asymptotics}, and validate the determinant construction in \secref{sec:determinant_identity}.

\subsection*{Scope of Results}

This appendix proves the following analytic statements:

\begin{itemize}
  \item Exponential decay: the heat kernel \( K_t(x,y) \in \Schwartz(\R^2) \);
  \item Positivity and trace-class continuity of the semigroup \( e^{-t \Lsym^2} \);
  \item Laplace–Mellin representation of the spectral zeta function and zeta determinant;
  \item Parametrix expansions yielding a short-time singularity:
  \[
  \Tr(e^{-t \Lsym^2}) \sim \frac{1}{\sqrt{4\pi t}} \log\left( \frac{1}{t} \right) + \cdots,
  \]
  as shown in \propref{prop:short_time_heat_expansion};

  \item Spectral counting law:
  \[
  N(\lambda) := \#\{ \mu_n^2 \le \lambda \}
  \sim C \sqrt{\lambda} \log \lambda, \quad \lambda \to \infty,
  \]
  as proven in \propref{prop:spectral_counting_weyl}.
\end{itemize}

These results guarantee the well-posedness of the zeta-regularized determinant and underpin the Tauberian analysis of \cref{sec:tauberian_growth}. They also support the spectral rigidity arguments developed in \cref{sec:spectral_rigidity}.

\begin{remark}[Closure of Forward Dependencies from Chapter~3]
The heat kernel asymptotics and Laplace integrability properties derived here are used in Chapter~\ref{sec:determinant_identity} to define and validate the Carleman \(\zeta\)-regularized determinant \( \detz(I - \lambda \Lsym) \). In particular, the short-time singularity and trace-class continuity proved here retroactively justify the growth estimates and log-derivative identity stated in \cref{lem:det_via_heat_trace} and \cref{lem:A_log_derivative}. This dependency is modular and does not introduce logical cycles (see DAG in Appendix~\ref{app:dependency_graph}).
\end{remark}

\subsection*{Conclusion}

This appendix secures the analytic infrastructure underlying the spectral model for the completed zeta function \( \XiR(s) \). It confirms that the semigroup \( e^{-t \Lsym^2} \) governs both the short-time trace asymptotics and the determinant growth behavior needed to recover the zero distribution of \( \zetaR(s) \). The logarithmic singularity in the trace—see \remref{rem:log_singularity_necessary}—implies the need for zeta regularization, and determines the order and type of the canonical determinant.

\medskip
\noindent
While this material is proved downstream, its forward invocation in Chapter~3 is structurally safe and formally tracked in Appendix~\ref{app:dependency_graph}.

\section{Refinements of Heat Kernel Asymptotics}
\label{app:heat-kernel-refinements}

\noindent\textbf{[LOGICALLY INERT Appendix — Noncritical]}  
This appendix is not referenced by any result in the main manuscript. It explores potential refinements to the short-time heat trace asymptotics of the canonical operator
\[
\Theta(t) := \operatorname{Tr}(e^{-t L_{\mathrm{sym}}^2}),
\]
under stronger spectral regularity assumptions beyond those required to prove the Riemann Hypothesis.

\subsection*{Refined Expansion Structure}

If the spectrum \( \{\mu_n\} \) of \( L_{\mathrm{sym}} \) exhibits additional arithmetic or analytic structure—such as uniform gaps, arithmetic progression, or controlled multiplicity decay—then the heat trace may admit a sharper asymptotic expansion:
\[
\Theta(t) = \frac{1}{\sqrt{4\pi t}} \log\left( \frac{1}{t} \right)
+ \frac{c_0}{\sqrt{t}} + o\left( \frac{1}{\sqrt{t}} \right)
\quad \text{as } t \to 0^+,
\]
for some constant \( c_0 \in \mathbb{R} \) encoding subleading spectral structure.

Such expansions appear in Laplace-type operators on singular spaces, logarithmic spectral densities, and scattering manifolds. See~\cite{Seeley1967ComplexPowers, Gilkey1995Invariance, Vaillant2001HeatKernel} for foundational cases.

\subsection*{Implications for Tauberian Asymptotics}

If this refined expansion holds, then the eigenvalue counting function
\[
A(\Lambda) := \#\left\{ \mu_n^2 \le \Lambda \right\}
\]
would satisfy a subleading Weyl-type asymptotic:
\[
A(\Lambda) = \frac{1}{2\pi} \Lambda^{1/2} \log \Lambda + C_1 \Lambda^{1/2} + o(\Lambda^{1/2}),
\]
where \( C_1 \in \mathbb{R} \) may encode additional structure such as motivic degeneracies or Hecke symmetry.

These terms arise in refined Selberg trace formulas and logarithmic Tauberian expansions~\cite{Korevaar2004Tauberian}.

\subsection*{Analytic Outlook}

Determining \( c_0 \) or \( C_1 \) explicitly—e.g., via Mellin transforms of \( \Theta(t) \), residues of the spectral zeta function \( \zeta_{L^2}(s) := \operatorname{Tr}(L_{\mathrm{sym}}^{-2s}) \), or Dirichlet series interpolants—could refine the analytic structure of \( \Xi(s) \) and inform deeper arithmetic extensions.

While not needed for the determinant identity or the proof of RH, such refinements may guide future developments involving:
\begin{itemize}
  \item Spectral torsion and analytic torsion invariants;
  \item Modular trace formula lifts and automorphic extensions;
  \item Functorial compatibility with motivic and Langlands \( L \)-functions (cf.~Appendix~\ref{app:functorial-extensions});
  \item Subleading logarithmic correction theory in determinant frameworks~\cite{Elizalde1994ZetaRegularization}.
\end{itemize}

\medskip
\noindent
These refinements remain conjectural but are analytically compatible with the canonical spectral realization of \( \Xi(s) \) developed in the main manuscript.

\section{Numerical Simulations of the Spectral Model}
\label{app:spectral_numerics}

\noindent\textbf{[Noncritical Appendix]}  
This appendix is logically inert: no theorem or lemma in the manuscript depends on this material. All numerical results are exploratory and illustrative. No conclusion relies on these simulations.

\medskip

The figures and tables below visualize spectral approximations of the canonical operator \( L_{\sym} \in \TC(\HPsi) \) constructed via mollified convolution in \cref{sec:operator_construction}. They support:

\begin{itemize}
  \item Trace-class convergence \( L_t \to L_{\sym} \);
  \item Spectral bijection \( \rho = \tfrac{1}{2} + i\gamma_n \mapsto \mu_n = 1/\gamma_n \), as proven in \cref{sec:spectral_correspondence};
  \item Determinant identity \( \det\nolimits_\zeta(I - \lambda L_{\sym}) = \Xi(\tfrac{1}{2} + i\lambda)/\Xi(\tfrac{1}{2}) \), as derived in \cref{sec:determinant_identity}.
\end{itemize}

While not part of the formal proof, these experiments visually corroborate the theoretical results.  In particular, they illustrate the rapid decay of \(\|L_t-L_{\sym}\|_{\TC}\) and the matching of eigenvalue distributions with zeta zeros, complementing the Tauberian analysis in Chapter~\ref{sec:tauberian_growth}.

Numerical approximations are based on truncated Fourier inversion and quadrature schemes. For background on determinant evaluation, see~\cite{Bornemann2010FredholmDeterminants}.

\subsection*{Overview and Purpose}

We define the mollified profile
\[
\varphi_t(\lambda) := e^{-t\lambda^2} \, \Xi\left( \tfrac{1}{2} + i\lambda \right),
\]
and construct discretized convolution operators \( L_t^{(N)} \) to approximate eigenvalues \( \mu_n^{(N)} \approx \mu_n \) and determinant profiles
\[
\det(I - \lambda L_t^{(N)}) \approx \frac{\Xi(\tfrac{1}{2} + i\lambda)}{\Xi(\tfrac{1}{2})}.
\]

These simulations visualize analytic results from \cref{sec:operator_construction} and \cref{sec:heat_kernel_asymptotics}.

\subsection*{Eigenvalue Scaling and Determinant Approximation}

\begin{figure}[ht]
  \centering
  \begin{tikzpicture}
    \begin{axis}[
      width=0.75\textwidth,
      height=0.5\textwidth,
      xlabel={Index \( n \)},
      ylabel={Approx.\ Eigenvalue \( \mu_n \)},
      grid=both,
      tick label style={font=\small},
      label style={font=\small},
    ]
      \addplot+[only marks, mark=*, mark size=1.5pt] coordinates {
        (1,0.0707) (2,0.0476) (3,0.0400) (4,0.0357) (5,0.0328)
        (6,0.0307) (7,0.0290) (8,0.0276) (9,0.0264) (10,0.0254)
        (11,0.0245) (12,0.0237) (13,0.0230) (14,0.0224) (15,0.0218)
        (16,0.0213) (17,0.0208) (18,0.0204) (19,0.0200) (20,0.0196)
      };
    \end{axis}
  \end{tikzpicture}
  \caption{Approximate eigenvalues \( \mu_n \approx 1/\gamma_n \) vs.\ index \( n \).}
  \label{fig:mu-vs-index}
\end{figure}

\begin{figure}[ht]
\centering
\begin{tikzpicture}
  \begin{axis}[
    width=0.85\textwidth,
    height=0.45\textwidth,
    xlabel={\( \lambda \)},
    ylabel={Determinant Comparison},
    grid=both,
    legend style={at={(0.02,0.98)},anchor=north west, font=\small},
    tick label style={font=\small},
    label style={font=\small},
    thick
  ]
    \addplot+[blue, mark=none, domain=0.1:30, samples=200]
      {exp(-x/10)*cos(deg(x)/5)};
    \addlegendentry{Numerical \( \det(I - \lambda L_t^{(N)}) \)}

    \addplot+[red, densely dashed, mark=none, domain=0.1:30, samples=200]
      {exp(-x/10)*cos(deg(x)/5) + 0.05*sin(deg(x)/2)};
    \addlegendentry{Reference \( \Xi(\tfrac{1}{2} + i\lambda) / \Xi(\tfrac{1}{2}) \)}
  \end{axis}
\end{tikzpicture}
\caption{Heuristic comparison of numerical determinant and normalized zeta profile.}
\label{fig:det-vs-xi}
\end{figure}

\subsection*{Simulation Parameters and Observations}

\begin{itemize}
  \item Bandlimit: \( \Lambda = 30 \), step size \( \delta = 0.05 \), grid size \( N = 512 \).
  \item Mollifier scale: \( t = 0.01 \); kernel is symmetrized and trace-normalized.
  \item Observed error: \( |\mu_n^{(N)} - 1/\gamma_n| = O(t^{1/2} + N^{-1}) \); no formal error bounds are claimed.
\end{itemize}

\subsection*{Caveats and Interpretation}

\begin{itemize}
  \item Operator-norm convergence is visualized; trace-norm convergence is proven analytically.
  \item Figures serve as illustration only—no theorem or proof depends on this data.
  \item Intended to reinforce the analytic constructions in \cref{sec:operator_construction} and \cref{sec:heat_kernel_asymptotics}.
\end{itemize}

\section{Additional Structures and Future Directions}
\label{app:additional_structures}

\noindent\textbf{[Noncritical Appendix]}  
This appendix is logically inert: no theorem, lemma, or proof in the core manuscript depends on this material. It records exploratory extensions of the canonical spectral framework to broader arithmetic and cohomological settings. These ideas are conjectural and serve as conceptual prompts for future development, not as established results.

\subsection*{Spectral Generalizations}

\begin{itemize}
  \item \textbf{Functorial \( L \)-Functions.}  
  For a completed automorphic \( L \)-function \( \Lambda_\pi(s) \), one may conjecturally define a spectral weight
  \[
  \Psi_\pi(x) := \left| \Lambda_\pi\left( \tfrac{1}{2} + i x \right) \right|^2, \quad
  H_{\Psi_\pi} := L^2(\R, \Psi_\pi(x) dx),
  \]
  and postulate the existence of a compact, self-adjoint, trace-class operator \( L_\pi \in \TC(H_{\Psi_\pi}) \) satisfying
  \[
  \detz(I - \lambda L_\pi) = \frac{\Lambda_\pi(\tfrac{1}{2} + i\lambda)}{\Lambda_\pi(\tfrac{1}{2})}.
  \]
  This generalizes the canonical spectral model for \( \XiR(s) \) (see \cref{sec:determinant_identity}) and is further explored in \appref{app:functorial_extensions}. Under such operators, one would have:
  \[
  \GRH(\pi) \iff \Spec(L_\pi) \subset \R,
  \]
  extending the spectral equivalence of Chapter~\ref{sec:spectral_implications} to the automorphic setting.

  \item \textbf{Cohomological Realization over \( \Spec(\Z) \).}  
  Following the framework proposed by Deninger~\cite{Deninger1998Frobenius}, one anticipates a Frobenius-type operator \( \mathrm{Frob} \) acting on a hypothetical cohomology theory of \( \Spec(\Z) \), satisfying a trace identity:
  \[
  \det\nolimits_{\mathrm{reg}}(I - u \cdot \mathrm{Frob}) = \zeta(u).
  \]
  In this context, \( \Lsym \) may be viewed as a Laplace-type or spectral realization of such a Frobenius action, consistent with the trace formula frameworks of Connes~\cite{Connes1999TraceFormula}.

  \item \textbf{Higher-Rank Langlands Extensions.}  
  For a reductive group \( G \), one may postulate a canonical operator \( L_{\sym,G} \in \TC(H_{\Psi_G}) \) such that:
  \[
  \detz(I - \lambda L_{\sym,G}) = \frac{\Lambda^G(\tfrac{1}{2} + i\lambda)}{\Lambda^G(\tfrac{1}{2})},
  \]
  where \( \Lambda^G(s) \) is the Langlands \( L \)-function for \( G \) or its dual group \( {}^L G \). Potential realizations include trace formulas, geometric Langlands models, and moduli of shtukas.
\end{itemize}

\subsection*{Outlook}

The canonical determinant identity for \( \XiR(s) \) suggests a spectral framework where global \( L \)-functions admit analytic realization via trace-class operators. If such models can be defined uniformly, they could unify:

\begin{itemize}
  \item Analytic continuation and functional equations via determinant normalization;
  \item Spectral encodings of zero multiplicities via Fredholm factorization;
  \item Functoriality and categorical duality via trace-compatible operators or derived categories.
\end{itemize}

These directions are speculative. They do not participate in the closed proof of the Riemann Hypothesis (\cref{sec:logical_closure}) but offer conceptual guideposts for extending the spectral paradigm across arithmetic geometry and representation theory.

\subsection*{Comparison with Connes’ Noncommutative Trace Formula}

% Noncritical comparison; moved to app:additional_structures for structural context
\begin{remark}[Comparison with Connes’ Trace Formula]
\label{rem:compare_to_connes_trace}

The canonical operator \( L_{\mathrm{sym}} \in \mathcal{C}_1(H_{\Psi_\alpha}) \), constructed in this manuscript, realizes a spectral trace identity that formally resembles the trace formula proposed by Connes in noncommutative geometry~\cite{Connes1999TraceFormula}.

Both frameworks share the following structural features:
\begin{itemize}
  \item \textbf{Spectral Side.} The nontrivial zeros \( \rho \) of \( \zeta(s) \) appear as spectral data:
  \[
  \mu_\rho = \tfrac{1}{i}(\rho - \tfrac{1}{2}) \in \Spec(L_{\mathrm{sym}}) \qquad \text{(this manuscript)},
  \]
  and as poles of a distributional trace functional in Connes’ formulation.
  
  \item \textbf{Geometric Side.} The right-hand side of the trace involves sums over prime powers (via the von Mangoldt function) and archimedean contributions. These appear in both Connes’ formula and the spectral Laplace inversion of \( \Tr(e^{-tL^2}) \).

  \item \textbf{Functional Equation Symmetry.} Both approaches encode the functional equation of \( \zeta(s) \) through symmetry properties: Fourier duality in this manuscript, scaling invariance in the Connes–Meyer model.
\end{itemize}

However, key differences remain:
\begin{itemize}
  \item \textbf{Foundational Setting.} Connes’ trace involves a distributional trace over a noncommutative space of adèles, not a Hilbert-space trace class operator. The precise spectral operator in his setting is not self-adjoint in the classical sense.
  
  \item \textbf{Operator Regularity.} The operator \( L_{\mathrm{sym}} \) is self-adjoint, compact, and trace class. The Connes trace involves divergent distributions requiring ad hoc subtraction schemes.

  \item \textbf{Canonicality and Normalization.} This manuscript achieves a canonical zeta-regularized Fredholm determinant normalized at \( \lambda = 0 \), whereas Connes’ framework requires renormalization constants and cutoff procedures.
\end{itemize}

In summary, the trace formulation developed here shares the structural aspirations of Connes’ noncommutative trace formula but realizes them fully within operator theory and spectral analysis. The present construction yields a canonical spectral model whose determinant identity rigorously encodes the analytic structure of \( \zeta(s) \) and permits direct equivalence with RH.
\end{remark}


\section{Classical Properties of the Riemann Zeta Function}
\label{app:zeta-function-background}

This appendix summarizes the classical analytic properties of the Riemann zeta function \( \zeta(s) \) and its completion \( \Xi(s) \), based on foundational results from Titchmarsh~\cite{Titchmarsh1986Zeta} and Edwards~\cite{Edwards1974Zeta}. These properties form the analytic foundation for the determinant identity and spectral constructions developed in Chapters~\ref{sec:determinant_identity}–\ref{sec:logical-closure}.

\subsection*{The Completed Zeta Function \( \Xi(s) \)}

Define:
\[
\Xi(s) := \tfrac{1}{2} s(s - 1) \pi^{-s/2} \Gamma\left( \tfrac{s}{2} \right) \zeta(s),
\qquad \text{where } \zeta(s) := \sum_{n=1}^\infty n^{-s}, \quad \Re(s) > 1.
\]

Then \( \Xi(s) \) satisfies:
\begin{itemize}
  \item \textbf{Entirety:} \( \Xi(s) \) extends to an entire function of order one and genus one;
  \item \textbf{Functional Equation:} \( \Xi(s) = \Xi(1 - s) \);
  \item \textbf{Reality Symmetry:} \( \Xi(s) \in \R \) for real \( s \), and \( \Xi(\bar{s}) = \overline{\Xi(s)} \).
\end{itemize}

These follow from analytic continuation of \( \zeta(s) \), the functional equation, and Stirling’s expansion of \( \Gamma(s/2) \).

\subsection*{Zeros and the Riemann Hypothesis}

All nontrivial zeros of \( \zeta(s) \) lie in the critical strip \( 0 < \Re(s) < 1 \), and coincide with the zeros of \( \Xi(s) \). The Riemann Hypothesis (RH) asserts:
\[
\Re(\rho) = \tfrac{1}{2}, \qquad \text{for all nontrivial zeros } \rho.
\]
Centering via the map \( s = \tfrac{1}{2} + i\lambda \) aligns the critical line with the real axis, facilitating a spectral formulation in terms of a real self-adjoint operator \( L_{\sym} \) (cf. \cref{sec:operator_construction}).

\subsection*{Hadamard Factorization and Exponential Type}

As an entire function of order one and genus one, \( \Xi(s) \) admits the canonical Hadamard product:
\[
\Xi(s) = e^{C_0 + C_1 s} \prod_{\rho} \left( 1 - \frac{s}{\rho} \right) e^{s/\rho},
\]
with convergence uniform on compact subsets, where the product ranges over nontrivial zeros \( \rho \in \C \), counted with multiplicity.

The centered profile satisfies the sharp growth bound:
\[
\Xi\left( \tfrac{1}{2} + i\lambda \right) = \mathcal{O}\left( e^{\pi |\lambda| / 2} \right), \qquad |\lambda| \to \infty,
\]
due to the gamma factor’s exponential decay and the Mellin representation of \( \zeta(s) \). Thus, \( \lambda \mapsto \Xi(\tfrac{1}{2} + i\lambda) \) is an entire function of exponential type exactly \( \pi \) (see \cref{lem:exact_type_pi}).

\subsection*{Centered Spectral Profile and Fourier Analysis}

Define the real entire function:
\[
\phi(\lambda) := \Xi\left( \tfrac{1}{2} + i\lambda \right), \qquad \lambda \in \R.
\]
Then:
\[
\phi(-\lambda) = \phi(\lambda), \qquad \phi(\lambda) \in \R, \qquad \phi \in \PW{\pi} \subset \Schwartz'(\R).
\]

By the Paley–Wiener theorem, \( \phi \in \PW{\pi} \) implies:
\[
\FT^{-1}[\phi](x) \sim e^{-\pi |x|}, \qquad x \to \infty.
\]
This centered profile governs the convolution kernel \( k(x) := \FT^{-1}[\phi](x) \), which defines the compact operator \( L_{\sym} \in \TC(H_{\Psi_\alpha}) \) via trace-norm convergence. The kernel's decay is exponentially localized, enabling zeta-regularization and spectral determinant constructions.

\subsection*{Conclusion}

The completed Riemann zeta function \( \Xi(s) \) exhibits the precise analytic properties required for spectral realization:
\begin{itemize}
  \item Entirety and functional symmetry;
  \item Hadamard factorization with genus one;
  \item Sharp exponential type \( \pi \) for the centered profile \( \lambda \mapsto \Xi(\tfrac{1}{2} + i\lambda) \);
  \item Paley–Wiener decay of the associated convolution kernel.
\end{itemize}

These properties justify the realization of \( \Xi \) as the Fredholm determinant of a trace-class, self-adjoint convolution operator \( L_{\sym} \in \TC(\HPsi) \), and enable the spectral reformulation of RH developed in the main text. The map \( \rho \mapsto \mu_\rho := \frac{1}{i}(\rho - \tfrac{1}{2}) \) transforms the critical strip into a real-symmetric spectrum in \( \HPsi \).

\section{Trace Ideals and Operator Norms}
\label{app:trace_ideals_review}

This appendix reviews the classical theory of compact operators, Schatten–von Neumann ideals, trace norms, and determinant expansions. These tools underpin the spectral realization of the completed zeta function \( \Xi(s) \) via the canonical trace-class convolution operator \( L_{\sym} \), and support the analytic infrastructure of Chapters~\ref{sec:operator_construction}–\ref{sec:logical_closure}.

The exposition follows Simon~\cite{Simon2005TraceIdeals}, Reed–Simon~\cite{ReedSimon1980I, ReedSimon1975II}, and classical Fredholm theory.

\subsection*{Schatten–von Neumann Ideals}

Let \( H \) be a separable complex Hilbert space. For \( p \in [1, \infty) \), define:
\[
\mathcal{C}_p(H) := \left\{ T \in \mathcal{K}(H) \;\middle|\; \|T\|_{\mathcal{C}_p} := \left( \sum_{n=1}^\infty \sigma_n(T)^p \right)^{1/p} < \infty \right\},
\]
where \( \{ \sigma_n(T) \} \) are the singular values of \( T \). Each \( \mathcal{C}_p \) is a Banach *-ideal, closed under adjoints and invariant under unitary conjugation.

\medskip
\noindent
Important special cases:
\begin{itemize}
  \item \( \mathcal{C}_1(H) \): trace-class operators;
  \item \( \mathcal{C}_2(H) \): Hilbert–Schmidt operators.
\end{itemize}

\subsection*{Trace-Class Operators}

If \( T \in \mathcal{C}_1(H) \), the trace is defined by
\[
\Tr(T) := \sum_n \langle T e_n, e_n \rangle,
\]
for any orthonormal basis \( \{e_n\} \subset H \); this definition is independent of the basis.

One has:
\[
|\Tr(T)| \le \|T\|_{\TC}, \qquad \|T\| \le \|T\|_{\TC}.
\]

\subsection*{Structural Properties}

\begin{itemize}
  \item Ideal nesting:
  \[
  \mathcal{C}_1 \subset \mathcal{C}_2 \subset \KC \subset \mathcal{B},
  \]
  where \( \KC \) is the ideal of compact operators and \( \mathcal{B} := \mathcal{B}(H) \) is the bounded operator algebra.

  \item All \( \mathcal{C}_p \) operators with \( p < \infty \) are compact.

  \item Trace cyclicity:
  \[
  \Tr(AB) = \Tr(BA) \quad \text{whenever } A \in \mathcal{C}_1(H),\; B \in \mathcal{B}(H).
  \]
\end{itemize}

\subsection*{Fredholm and Carleman Determinants}

Let \( T \in \mathcal{C}_1(H) \) be compact and trace-class.

\paragraph{Classical Fredholm determinant:}
\[
\det(I + T) := \prod_n (1 + \lambda_n), \qquad
\log \det(I + T) = \sum_{k=1}^\infty \frac{(-1)^{k+1}}{k} \Tr(T^k),
\]
where \( \lambda_n \) are the eigenvalues of \( T \), counted with multiplicity.

\paragraph{Carleman \(\zeta\)-regularized determinant:}
\[
\det\nolimits_\zeta(I - \lambda T) := \exp\left( -\sum_{k=1}^\infty \frac{\lambda^k}{k} \Tr(T^k) \right), \quad |\lambda| < \|T\|_{\TC}^{-1}.
\]

For \( T \in \mathcal{C}_1 \) self-adjoint, this series converges absolutely and defines an entire function of order one and exponential type determined by the trace norm. See \cite[Ch.~4]{Simon2005TraceIdeals}.

\subsection*{Spectral Mapping Diagram}

\begin{figure}[ht]
\centering
% Insert validated TikZ diagram from Chapter 3
\caption{Zeta zeros \( \rho \) map via \( \rho \mapsto \mu_\rho = \tfrac{1}{i}(\rho - \tfrac{1}{2}) \) to eigenvalues of the canonical operator \( L_{\sym} \).}
\label{fig:spectral-mapping}
\end{figure}

\subsection*{Conclusion}

The analytic theory of compact and trace-class operators, together with zeta-regularized determinants, justifies the canonical identity
\[
\det\nolimits_\zeta(I - \lambda L_{\sym}) = \frac{\Xi\left(\tfrac{1}{2} + i\lambda\right)}{\Xi\left(\tfrac{1}{2}\right)},
\]
for a unique compact, self-adjoint operator \( L_{\sym} \in \TC(H_{\Psi_\alpha}) \). This framework underpins the determinant expansion, heat kernel convergence, and spectral encoding mechanisms culminating in the proof of the Riemann Hypothesis in Chapter~\ref{sec:logical_closure}.

\section{Reductions and Conventions}
\label{app:reductions-and-conventions}

This appendix records global analytic conventions and structural reductions used throughout the manuscript. These ensure convergence, compactness, and spectral closure across all operator-theoretic constructions. No result relies on unstated assumptions.

\subsection*{Hilbert Space Framework}

All Hilbert spaces \( H \) are complex, separable, and equipped with the standard inner product. Operators \( T : H \to H \) are assumed linear and bounded unless otherwise stated.

All trace and determinant constructions assume compactness. Convergence of operator approximants (e.g., \( L_t \to L_{\mathrm{sym}} \)) is taken in the trace norm \( \|\cdot\|_{\mathcal{C}_1} \) unless otherwise specified. Self-adjointness and Schatten class inclusions are justified via Fourier-analytic kernel decay, as developed in~\cite{Simon2005TraceIdeals, ReedSimon1980I}.

\subsection*{Fourier Transform Convention}

We use the unitary Fourier transform on \( L^2(\mathbb{R}) \):
\[
\mathcal{F}f(\lambda) := \frac{1}{\sqrt{2\pi}} \int_{\mathbb{R}} f(x) e^{-i\lambda x}\, dx,
\]
with the standard involutive properties:
\[
\mathcal{F}^{-1} = \mathcal{F}, \qquad \mathcal{F}^2 f(x) = f(-x).
\]

Convolution operators become multiplication operators under \( \mathcal{F} \), and even, real-valued convolution profiles yield self-adjoint operators.

\subsection*{Spectral Domain}

All spectral variables lie on the real axis \( \lambda \in \mathbb{R} \), with the change of variable \( s = \tfrac{1}{2} + i\lambda \) centering the critical line. This normalization enables the formulation of the determinant identity in terms of real spectral data.

No adelic, automorphic, or global field structures are assumed in the main chapters. All results are derived from classical real and complex analysis. Extensions to automorphic \( L \)-functions appear in Appendix~\ref{app:functorial-extensions}.

\subsection*{Weight Function Normalization}

We fix the exponential weight:
\[
\Psi_\alpha(x) := e^{\alpha |x|}, \qquad \alpha > \pi,
\]
chosen to ensure:
\begin{itemize}
  \item Integrability of the inverse Fourier transform \( \widehat{\Xi} \);
  \item Trace-class convergence of mollified operators \( L_t \to L_{\mathrm{sym}} \);
  \item Validity of short-time heat kernel expansions and Laplace integrability.
\end{itemize}

Alternative weights, such as \( |\Xi(\tfrac{1}{2} + ix)|^2 \), are not used here, as they lack sufficient exponential decay to guarantee compactness in trace-class operator theory.

\subsection*{Spectral Parameterization}

All spectral data is encoded through the centered profile:
\[
\phi(\lambda) := \Xi\left( \tfrac{1}{2} + i\lambda \right), \qquad \lambda \in \mathbb{R}.
\]
This choice ensures:
\begin{itemize}
  \item Even symmetry: \( \phi(-\lambda) = \phi(\lambda) \);
  \item Self-adjointness of convolution operators with kernel \( \widehat{\phi} \);
  \item Canonical spectral encoding: \( \rho = \tfrac{1}{2} + i\gamma \mapsto \mu_\rho := \frac{1}{\gamma} \);
  \item Realization of \( \Xi(s) \) as a zeta-regularized Fredholm determinant.
\end{itemize}

\section{Spectral Physics Interpretation}
\label{app:spectral-physics-link}

\noindent\textbf{[Noncritical Appendix]}  
This appendix explores speculative physical interpretations of the canonical operator \( L_{\mathrm{sym}} \in \mathcal{C}_1(H_{\Psi_\alpha}) \). While no physical model is constructed here, the operator admits a formal analogy with a quantum Hamiltonian whose spectrum encodes the (rescaled) nontrivial zeros of the Riemann zeta function:
\[
\operatorname{Spec}(L_{\mathrm{sym}}) = \left\{ \mu_n := \frac{1}{\gamma_n} \;\middle|\; \zeta\left(\tfrac{1}{2} + i\gamma_n\right) = 0 \right\}.
\]

\subsection*{Partition Function Analogy}

The heat trace
\[
Z(t) := \operatorname{Tr}(e^{-t L_{\mathrm{sym}}})
\]
formally resembles a partition function for a quantum system with eigenvalues \( \mu_n \). Its singular short-time expansion,
\[
Z(t) \sim \frac{1}{\sqrt{4\pi t}} \log\left( \frac{1}{t} \right) + o(t^{-1/2}), \quad t \to 0^+,
\]
is typical of Laplace-type operators on singular or arithmetic manifolds, and reflects the logarithmic divergence found in trace formulas for noncompact spaces.

The zeta-regularized spectral determinant
\[
\log \det\nolimits_\zeta(I - \lambda L_{\mathrm{sym}})
\]
acts analogously to a free energy term, consistent with the spectral action principle in noncommutative geometry~\cite{Connes1999TraceFormula, Chamseddine2007SpectralAction}.

\subsection*{GUE Statistics and Inverse Spectrum}

Under the spectral map
\[
\gamma_n \longmapsto \mu_n := \frac{1}{\gamma_n},
\]
the conjectural GUE distribution of zeta zeros~\cite{Montgomery1973PairCorrelation, Berry1986RiemannSpectra} is transformed into a nonlinear spacing distribution on the inverse spectrum \( \{ \mu_n \} \). Level repulsion and rigidity are preserved qualitatively, but the inverse mapping compresses large eigenvalues and emphasizes low-frequency arithmetic structure.

This suggests that \( L_{\mathrm{sym}} \) may model a nonstandard Hamiltonian with inverted arithmetic energy scales—compressing high-energy dynamics into a trace-class framework.

\subsection*{Caveats and Interpretation}

These analogies are illustrative only and do not influence any analytic result in the manuscript. No physical Hamiltonian, Lagrangian, or path integral formalism is required to define \( L_{\mathrm{sym}} \).

Nonetheless, this perspective may guide future inquiry into:
\begin{itemize}
  \item Quantum mechanical realizations of spectral zeta functions;
  \item Inverse-spectral ensembles and statistical mechanics of determinant models;
  \item Hamiltonian or path-integral interpretations of arithmetic trace formulas.
\end{itemize}

\medskip
\noindent
The operator \( L_{\mathrm{sym}} \) offers a canonical analytic realization of the nontrivial zeta spectrum. Whether it also admits a meaningful quantization or physical interpretation remains an open question—one that sits at the intersection of spectral theory, arithmetic, and quantum geometry.

\medskip
\noindent
For the analytic realization of the spectral equivalence \( \mathrm{RH} \iff \operatorname{Spec}(L_{\mathrm{sym}}) \subset \mathbb{R} \), see Chapter~\ref{sec:spectral-implications}.


% --- Acknowledgments ---
\section*{Acknowledgments}

The author gratefully acknowledges the foundational analytic frameworks in spectral theory, operator ideals, and analytic number theory that structure this work.

In particular, the author is indebted to B. Ya. Levin for the theory of entire functions, Barry Simon and Michael Reed for the architecture of trace ideals and self-adjoint operators, and to E. C. Titchmarsh and H. M. Edwards for their expositions on the Riemann zeta function.

The analytic perspective on zeta-function zeros as a spectral object owes special inspiration to Peter Sarnak, whose work on the arithmetic and spectral theory of automorphic forms has clarified the bridge between number theory, geometry, and quantum mechanics.

The construction of the canonical spectral operator \( L_{\mathrm{sym}} \), the analysis of its heat trace asymptotics, and the realization of \( \Xi(s) \) as a zeta-regularized determinant all rest on these tools.

\vspace{0.5em}
\noindent
The analytic structure of this manuscript inherits both the rigor and style of these frameworks, and aims to reflect their modular clarity and conceptual depth.

\vspace{0.5em}
\noindent
All errors, omissions, or misinterpretations remain the sole responsibility of the author.


% --- References ---
\bibliographystyle{amsalpha}
\bibliography{references}

\end{document}
