\subsection*{Chapter Summary}

This chapter establishes a canonical spectral encoding of the nontrivial zeros of the Riemann zeta function via the compact, self-adjoint operator \( \Lsym \in \TC(\HPsi) \). The central results organize the bijection:
\[
\rho = \tfrac{1}{2} + i\gamma \quad \leftrightarrow \quad \mu_\rho := \tfrac{1}{i}(\rho - \tfrac{1}{2}) = \gamma \in \Spec(\Lsym),
\]
into a rigorous analytic framework:

\begin{itemize}
  \item \lemref{lem:zero_to_eigenvalue_injection} — Every nontrivial zero \( \rho \in \mathcal{Z}(\zeta) \) maps to a unique eigenvalue \( \mu_\rho \in \Spec(\Lsym) \setminus \{0\} \).

  \item \lemref{lem:spectral_exhaustivity} — Surjectivity: every nonzero spectral value \( \mu \in \Spec(\Lsym) \) corresponds to some zeta zero \( \rho \).

  \item \lemref{lem:spectral_multiplicity_matching} — Multiplicities match exactly:
  \[
  \operatorname{mult}_{\mathrm{spec}}(\mu_\rho) = \operatorname{ord}_\rho(\zeta).
  \]

  \item \lemref{lem:spectral_symmetry} — The spectrum is symmetric:
  \[
  \mu \in \Spec(\Lsym) \Rightarrow -\mu \in \Spec(\Lsym),
  \]
  reflecting the functional equation \( \zeta(s) = \zeta(1 - s) \).

  \item \lemref{lem:spectral_bijection_consistency} — The map \( \rho \mapsto \mu_\rho \) is a multiplicity-preserving bijection from nontrivial zeta zeros to the nonzero spectrum of \( \Lsym \).

  \item \thmref{thm:spectral_zero_bijection_revised} — Main consolidation:
  \[
  \zeta(\rho) = 0 \quad \Longleftrightarrow \quad \mu_\rho := \tfrac{1}{i}(\rho - \tfrac{1}{2}) \in \Spec(\Lsym),
  \]
  with multiplicities exactly matched.
\end{itemize}

\begin{quote}
  \textbf{Remark (Spectral Encoding as Analytic Dual).}~
  The bijection is canonical: the full spectrum of \( \Lsym \) is uniquely determined by the analytic structure of \( \Xi(s) \) via the determinant identity (\thmref{thm:det_identity_revised}). Conversely, the spectrum, including multiplicities, reconstructs \( \Xi(s) \) (see \corref{cor:spectrum_determines_zeta}).
\end{quote}

A schematic table summarizing the map \( \rho \mapsto \mu_\rho \), the symmetry \( \mu \mapsto -\mu \), and spectral bijection structure is provided in Table~\ref{fig:schematic_bijection_table}.

\medskip

This canonical spectral correspondence forms the analytic basis of the Riemann Hypothesis equivalence
\[
\mathrm{RH} \quad \Longleftrightarrow \quad \Spec(\Lsym) \subset \R,
\]
which is rigorously proven in Chapter~\ref{sec:spectral_implications} (\thmref{thm:eq_of_rh}).
