\begin{remark}[Canonical Operator Framework]
\label{rem:setup_operator_framework}

Let \( \HPsi := L^2(\R, e^{\alpha|x|} dx) \) denote the exponentially weighted Hilbert space, with fixed weight parameter \( \alpha > \pi \). Throughout this chapter, we work with the canonical operator
\[
L_{\sym} \in \TC(\HPsi),
\]
constructed in \cref{sec:operator_construction} as the trace-norm limit of mollified symmetric convolution operators derived from the inverse Fourier transform of the completed zeta function \( \XiR(s) \).

The operator \( L_{\sym} \) satisfies:
\begin{itemize}
  \item It is compact and self-adjoint with real, discrete spectrum;
  \item It lies in the trace-class \( \TC(\HPsi) \), with analytic control on heat kernel asymptotics and spectral determinant growth;
  \item It satisfies the canonical determinant identity:
  \[
  \det\nolimits_\zeta(I - \lambda L_{\sym}) = \frac{\XiR\left( \tfrac{1}{2} + i\lambda \right)}{\XiR\left( \tfrac{1}{2} \right)},
  \qquad \forall\, \lambda \in \C,
  \]
  where the right-hand side is normalized to 1 at \( \lambda = 0 \). This identity analytically encodes all nontrivial zeros of \( \zetaR(s) \) via spectral calculus.
\end{itemize}

\medskip

\noindent
These structural properties are analytically proven in Chapters~\ref{sec:determinant_identity}–\ref{sec:spectral_implications}, and are assumed throughout this chapter without further restatement. No appeal is made to RH or any unproven zero-distribution assumption.
\end{remark}
