\begin{proof}[Proof of Lemma~\ref{lem:tauberian_heat_trace_application}]
Let \( L := L_{\mathrm{sym}} \in \mathcal{C}_1(H_{\Psi_\alpha}) \) be compact, self-adjoint, and positive, with nonzero spectrum \( \{ \mu_n \} \subset \mathbb{R}_{>0} \), counted with multiplicity. Define the heat trace
\[
\Theta(t) := \operatorname{Tr}(e^{-t L^2}) = \sum_{n=1}^\infty e^{-t \mu_n^2},
\]
and the squared spectral counting function
\[
A(\Lambda) := \#\left\{ n \in \mathbb{N} : \mu_n^2 \le \Lambda \right\}, \qquad \Lambda > 0.
\]

\paragraph{Step 1: Refined Heat Trace Asymptotics.}
From Proposition~\ref{prop:short_time_heat_expansion} and the two-sided bounds in Proposition~\ref{prop:two_sided_heat_trace_bounds}, we have:
\[
\Theta(t) = \frac{1}{\sqrt{4\pi t}} \log\left( \frac{1}{t} \right) + o(t^{-1/2}) \quad \text{as } t \to 0^+.
\]
These expansions are justified analytically in Appendix~\ref{app:heat_kernel_construction}, via mollified kernel estimates and Laplace integrability. The integrability itself is formally established in Lemma~\ref{lem:laplace_integrability_heat_trace}.

This places \( \Theta(t) \in \mathcal{R}_{1/2}^{\log} \), the class of regularly varying functions of index \( \alpha = \tfrac{1}{2} \) modulated by \( \log(1/t) \), in the sense of Korevaar~\cite[Ch.~III, §5]{Korevaar2004Tauberian}.

\paragraph{Step 2: Laplace–Stieltjes Transform.}
The heat trace has the Laplace–Stieltjes representation
\[
\Theta(t) = \int_0^\infty e^{-t \lambda} \, dA(\lambda),
\]
where \( A(\lambda) \) is monotone, piecewise constant, and right-continuous. The convergence of this representation is guaranteed by Lemma~\ref{lem:laplace_integrability_heat_trace}.

\paragraph{Step 3: Tauberian Inversion.}
By Korevaar’s log-modified Tauberian theorem for Laplace transforms (cf.~\cite[Thm.~5.5]{Korevaar2004Tauberian}), the singular behavior
\[
\Theta(t) \sim C \, t^{-1/2} \log(1/t) \qquad \text{as } t \to 0^+
\]
implies the counting function satisfies
\[
A(\Lambda) = \frac{C}{\Gamma(3/2)} \Lambda^{1/2} \log \Lambda + O(\Lambda^{1/2}).
\]
Since \( C = \frac{1}{\sqrt{4\pi}} \) and \( \Gamma(3/2) = \frac{\sqrt{\pi}}{2} \), we obtain:
\[
A(\Lambda) = \frac{\sqrt{\Lambda}}{\pi} \log \Lambda + O(\sqrt{\Lambda}).
\]

\paragraph{Conclusion.}
The eigenvalue counting function admits the refined asymptotic expansion
\[
A(\Lambda) = \frac{\sqrt{\Lambda}}{\pi} \log \Lambda + O(\sqrt{\Lambda}),
\]
completing the proof.
\end{proof}
