\subsection*{Introduction}

This chapter establishes the analytic infrastructure for defining and analyzing the canonical compact operator \( L_{\sym} \), which realizes the completed Riemann zeta function \( \Xi(s) \) via its Fredholm determinant. The primary goal is to verify that mollified convolution operators associated with the inverse Fourier transform of \( \Xi \) are compact, trace class, and converge in trace norm to a self-adjoint limit operator \( L_{\sym} \in \TC(\HPsi) \), where the weighted Hilbert space \( \HPsi := L^2(\R, \Psi_\alpha(x) \, dx) \) is equipped with exponential weight \( \Psi_\alpha(x) = e^{\alpha|x|} \) for \( \alpha > \pi \).

\begin{center}
\begin{tikzpicture}[node distance=20pt, every node/.style={align=center},>=Stealth]
  \node (phi) [draw, rectangle] {\( \phi_t(\lambda) := e^{-t\lambda^2} \Xi(\tfrac{1}{2} + i\lambda) \)};
  \node (kt) [draw, rectangle, below=of phi] {\( k_t(x) := \FT^{-1}[\phi_t](x) \)};
  \node (kernel) [draw, rectangle, below=of kt] {\( K_t(x,y) := k_t(x - y) \)};
  \node (op) [draw, rectangle, below=of kernel] {\( (L_t f)(x) := \int_{\R} K_t(x,y) f(y)\,dy \)};
  \node (limit) [draw, rectangle, below=of op, yshift=-0cm] {\( L_{\sym} := \lim_{t \to 0^+} L_t \in \TC(\HPsi) \)};
  \node (det) [draw, rectangle, below=of limit, yshift=-0cm] {\( \det\nolimits_\zeta(I - \lambda L_{\sym}) = \frac{\Xi(\tfrac{1}{2} + i\lambda)}{\Xi(\tfrac{1}{2})} \)};

  \draw[->] (phi) -- (kt);
  \draw[->] (kt) -- (kernel);
  \draw[->] (kernel) -- (op);
  \draw[->] (op) -- (limit);
  \draw[->] (limit) -- (det);
\end{tikzpicture}
\end{center}

\noindent
This pipeline links mollified spectral data to compact operator realization and determinant recovery.

\medskip

The constructions in this chapter verify:

\begin{itemize}
    \item \textbf{Trace-class regularity:} The canonical kernel \( k(x) = \FT^{-1}[\Xi(\tfrac{1}{2} + i\lambda)](x) \) decays like \( e^{-\pi |x|} \), and trace-class inclusion holds if and only if \( \alpha > \pi \); this bound is sharp (\cref{prop:trace_class_sharpness}).

    \item \textbf{Schatten-class and determinant control:} Operators \( L_t \in \TC(\HPsi) \) admit uniform bounds in the trace-norm ideal, and determinant convergence is ensured by compact kernel decay.

    \item \textbf{Essential self-adjointness:} The limit \( L_{\sym} \) is symmetric and essentially self-adjoint on \( \Schwartz(\R) \), by Nelson’s analytic vector theorem (\cref{rem:selfadjoint_analytic_vectors}).

    \item \textbf{Heat semigroup generation:} The semigroup \( \{e^{-t L_{\sym}^2}\}_{t > 0} \subset \TC(\HPsi) \) exists, is holomorphic in \( t \), and trace-class, with precise exponential bounds (\cref{lem:heat_semigroup_existence}).

    \item \textbf{Paley--Wiener analytic control:} The spectral profile \( \phi(\lambda) := \Xi(\tfrac{1}{2} + i\lambda) \) lies in the Paley--Wiener class \( \PW{\pi} \), and its inverse Fourier transform belongs to \( L^1(\R, e^{-\alpha |x|} dx) \) for all \( \alpha > \pi \). This ensures kernel localization and enables rigorous determinant regularization.
\end{itemize}

These ingredients culminate in the construction of mollified convolution operators \( L_t \), and in the verification of trace-norm convergence
\[
L_t \to L_{\sym} \in \TC(\HPsi) \quad \text{as } t \to 0^+.
\]
This limit defines the canonical spectral operator underlying the determinant identity
\[
\det\nolimits_\zeta(I - \lambda L_{\sym}) = \frac{\Xi\left(\tfrac{1}{2} + i\lambda\right)}{\Xi\left(\tfrac{1}{2}\right)},
\]
which is rigorously established in the next chapter, without assuming RH or spectral symmetry.

\medskip

The analytic architecture developed here underpins all subsequent spectral and determinant identities.
See Appendix~\ref{app:dependency_graph} for a visual DAG linking these foundational tools to the modular proof of RH.
