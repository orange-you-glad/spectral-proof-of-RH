\begin{lemma}[Positivity of the Trace Distribution]
\label{lem:trace_distribution_positivity}

Let \( L_{\sym} \in \TC(\HPsi) \) be the canonical compact, self-adjoint operator with discrete real spectrum \( \{ \mu_n \}_{n \in \Z} \subset \R \), where each eigenvalue appears with its finite multiplicity.

Define the spectral trace functional on the Schwartz space \( \Schwartz(\R) \) by:
\[
\phi \mapsto \Tr(\phi(L_{\sym})) := \sum_n \phi(\mu_n).
\]

Then:
\begin{enumerate}
  \item[\textup{(i)}] The map \( \phi \mapsto \Tr(\phi(L_{\sym})) \) defines a tempered distribution on \( \R \). That is, it extends continuously on \( \Schwartz(\R) \) and satisfies finite-order growth bounds under differentiation.

  \item[\textup{(ii)}] The distribution is positive: for every \( \phi \in \Schwartz(\R) \) with \( \phi(\lambda) \ge 0 \) for all \( \lambda \in \R \), we have:
  \[
  \Tr(\phi(L_{\sym})) \ge 0.
  \]
\end{enumerate}

\medskip

\noindent
This positivity reflects the spectral measure structure of \( L_{\sym} \), and is inherited from the positivity of the heat trace \( \Tr(e^{-t L_{\sym}^2}) \) and associated kernel. The analytic justification follows from the short-time convergence results in \propref{prop:heat_trace_uniform_conv} and kernel estimates developed in \appref{app:heat_kernel_construction}.
\end{lemma}
