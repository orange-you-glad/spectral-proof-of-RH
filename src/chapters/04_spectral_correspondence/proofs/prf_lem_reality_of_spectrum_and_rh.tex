\begin{proof}[Proof of \lemref{lem:reality_of_spectrum_and_rh}]
Assume first that all zeros of \( \lambda \mapsto \XiR(\tfrac{1}{2} + i\lambda) \) lie on the real axis. Then the canonical determinant identity
\[
\det\nolimits_\zeta(I - \lambda L_{\sym}) = \frac{\XiR(\tfrac{1}{2} + i\lambda)}{\XiR(\tfrac{1}{2})}
\]
has zeros only for real \( \lambda \). Since this determinant arises from the spectrum of a compact, self-adjoint operator \( L_{\sym} \), the location of its zeros implies that all eigenvalues \( \mu_\rho \in \R \), by the spectral theorem.

Conversely, suppose \( L_{\sym} \in \TC(\HPsi) \) is self-adjoint and all eigenvalues \( \mu_\rho \in \R \). Then by the canonical spectral encoding
\[
\mu_\rho = \frac{1}{i(\rho - \tfrac{1}{2})},
\]
it follows that
\[
\rho = \tfrac{1}{2} + \frac{1}{i\mu_\rho}.
\]
Since \( \mu_\rho \in \R \), this implies \( \Re(\rho) = \tfrac{1}{2} \). Therefore, all nontrivial zeros of \( \zetaR(s) \) lie on the critical line, and the Riemann Hypothesis holds.
\end{proof}
