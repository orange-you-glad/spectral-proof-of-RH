\begin{lemma}[Mollifier Independence of Canonical Kernel Limit]
\label{lem:mollifier_independence_kernel_limit}
Let \( \alpha > \pi \), and define the exponentially weighted Hilbert space
\[
H_{\Psi_\alpha} := L^2(\R, \Psi_\alpha(x)\, dx), \qquad \Psi_\alpha(x) := e^{\alpha |x|}.
\]

Let \( \widehat{\Xi} \in \Schwartz' \) denote the inverse Fourier transform of the canonical spectral profile
\[
\phi(\lambda) := \Xi\left( \tfrac{1}{2} + i\lambda \right),
\]
where \( \Xi \) is the completed Riemann zeta function.

Let \( \{\varphi_t\}_{t > 0} \subset \Schwartz \) be any mollifier family satisfying:
\begin{itemize}
  \item \textnormal{(Normalization)}: \( \int_{\R} \varphi_t(x)\, dx = 1 \);
  \item \textnormal{(Approximate Identity)}: \( \varphi_t \to \delta \) in \( \Schwartz' \) as \( t \to 0^+ \);
  \item \textnormal{(Symmetry)}: \( \varphi_t(x) = \varphi_t(-x) \);
  \item \textnormal{(Decay)}: \( \varphi_t \in L^1 \cap L^2 \cap L^1(\Psi_\alpha\, dx) \) for all \( t > 0 \).
\end{itemize}

Define the mollified kernels and corresponding convolution operators:
\[
\widehat{\Xi}_t := \varphi_t * \widehat{\Xi}, \qquad
(L_t^{(\varphi)} f)(x) := \int_{\R} \widehat{\Xi}_t(x - y)\, f(y)\, dy.
\]

Then:

\begin{enumerate}
  \item[\textnormal{(i)}] \textbf{Trace-Class Structure.}  
  For each \( t > 0 \), the mollified kernel \( \widehat{\Xi}_t \in \Schwartz \cap L^1(\R, \Psi_\alpha(x)\, dx) \), and
  \[
  L_t^{(\varphi)} \in \TC(H_{\Psi_\alpha}).
  \]
  This follows from classical decay and Simon’s weighted trace-norm kernel criterion~\cite[Ch.~4]{Simon2005TraceIdeals}.

  \item[\textnormal{(ii)}] \textbf{Trace-Norm Convergence and Uniqueness.}  
  The limit
  \[
  L_{\sym} := \lim_{t \to 0^+} L_t^{(\varphi)} \quad \text{in } \TC(H_{\Psi_\alpha})
  \]
  exists and is independent of the mollifier family \( \{\varphi_t\} \). Specifically, for any two mollifiers \( \varphi_t \) and \( \tilde{\varphi}_t \) satisfying the above properties,
  \[
  \lim_{t \to 0^+} \| L_t^{(\varphi)} - L_t^{(\tilde{\varphi})} \|_{\TC} = 0.
  \]
  This follows from convolution continuity in trace-norm, mollifier convergence in \( \Schwartz' \), and uniform exponential envelope control.
\end{enumerate}

\noindent
Hence, the canonical operator \( L_{\sym} \in \TC(H_{\Psi_\alpha}) \) is uniquely determined by the analytic data \( (\phi, \Psi_\alpha) \), and is independent of the mollifier family. This analytic rigidity confirms the well-posedness of the spectral model and supports the determinant identity in \defref{def:convolution_operators_Lt_Lsym}.
\end{lemma}
