\subsection*{Introduction}
\label{sec:intro_spectral_correspondence}

This chapter rigorously establishes the canonical bijection between the nontrivial zeros of the Riemann zeta function \( \zetaR(s) \) and the nonzero spectrum of the trace-class operator \( \Lsym \in \TC(\HPsi) \), constructed in \secref{sec:operator_construction} and analytically normalized in \secref{sec:determinant_identity}. The spectral identification map
\[
\rho := \tfrac{1}{2} + i\gamma \quad \longmapsto \quad \mu_\rho := \frac{1}{i}(\rho - \tfrac{1}{2}) = \gamma
\]
assigns to each nontrivial zero \( \rho \in \C \) a nonzero eigenvalue \( \mu_\rho \in \R \setminus \{0\} \), and realizes the completed Riemann zeta function \( \XiR(s) \) as the \(\zeta\)-regularized Fredholm determinant:
\[
\detz(I - \lambda \Lsym) = \frac{\XiR\left( \tfrac{1}{2} + i\lambda \right)}{\XiR\left( \tfrac{1}{2} \right)}.
\]

\medskip

The correspondence between zeros and spectral values is developed through a chain of analytically independent results:

\begin{itemize}
  \item \textbf{Injection:} Every nontrivial zero \( \rho \in \mathcal{Z}(\zetaR) \) induces a nonzero eigenvalue \( \mu_\rho \in \Spec(\Lsym) \) (\lemref{lem:zero_to_eigenvalue_injection}).

  \item \textbf{Surjection:} Every nonzero eigenvalue arises from a unique zero \( \rho \in \mathcal{Z}(\zetaR) \) (\lemref{lem:spectral_exhaustivity}).

  \item \textbf{Multiplicity Matching:} Spectral multiplicities coincide with zero multiplicities (\lemref{lem:spectral_multiplicity_matching}).

  \item \textbf{Spectral Symmetry:} The spectrum satisfies \( \mu \in \Spec(\Lsym) \Rightarrow -\mu \in \Spec(\Lsym) \) (\lemref{lem:spectral_symmetry}), reflecting functional equation symmetry.

  \item \textbf{Bijection Consistency:} The map \( \rho \mapsto \mu_\rho \) defines a bijection between the zero multiset and the nonzero spectrum of \( \Lsym \) (\lemref{lem:spectral_bijection_consistency}).

  \item \textbf{Spectral Trace Realization:} The spectral trace \( \Tr(e^{-t \Lsym^2}) \) admits a Laplace transform encoding of \( \XiR(s) \), justifying analytic continuation (\lemref{lem:spectral_measure_heat_semigroup}).

  \item \textbf{Reality–RH Equivalence:} The Riemann Hypothesis is equivalent to spectral reality:
  \[
  \Spec(\Lsym) \subset \R \quad \Longleftrightarrow \quad \text{RH},
  \]
  as proven in \lemref{lem:reality_of_spectrum_and_rh}.
\end{itemize}

\medskip

The central result of this chapter, \thmref{thm:spectral_zero_bijection_revised}, consolidates the analytic chain into the canonical identity:
\[
\Spec(\Lsym) \setminus \{0\} = \left\{ \mu_\rho := \frac{1}{i}(\rho - \tfrac{1}{2}) \,\middle|\, \zetaR(\rho) = 0 \right\},
\]
with exact preservation of multiplicities. This bijection is proven unconditionally and without assuming RH, and establishes the spectral encoding framework used in subsequent equivalence theorems in \cref{sec:spectral_implications}.
