\begin{proof}
The canonical determinant identity (\thmref{thm:det_identity_revised}) asserts:
\[
\det\nolimits_{\zeta}(I - \lambda L_{\sym})
= \frac{\XiR\left(\tfrac{1}{2} + i\lambda\right)}{\XiR\left(\tfrac{1}{2}\right)},
\]
where \( \XiR(s) \) admits the classical Hadamard factorization over the nontrivial zeros \( \rho \) of \( \zetaR(s) \):
\[
\XiR(s) = \XiR\left(\tfrac{1}{2}\right)
\prod_\rho \left(1 - \frac{s - \tfrac{1}{2}}{\rho - \tfrac{1}{2}}\right)
\exp\left( \frac{s - \tfrac{1}{2}}{\rho - \tfrac{1}{2}} \right).
\]

\medskip

Define the canonical spectral parameter \( \mu_\rho := \dfrac{1}{i(\rho - \tfrac{1}{2})} \). Then:
\[
\det\nolimits_\zeta(I - \lambda L_{\sym}) = 0 \quad \Longleftrightarrow \quad \lambda = \frac{1}{\mu_\rho},
\]
and the order of vanishing of the determinant at \( \lambda = 1/\mu_\rho \) matches the order of vanishing of \( \zetaR(s) \) at \( \rho \).

\medskip

By \lemref{lem:log_derivative_determinant}, the logarithmic derivative of the determinant satisfies the trace identity:
\[
\frac{d}{d\lambda} \log \det\nolimits_\zeta(I - \lambda L_{\sym})
= \Tr\left( (I - \lambda L_{\sym})^{-1} L_{\sym} \right),
\]
which is meromorphic with poles precisely at the reciprocal spectral values \( \lambda = 1/\mu_\rho \). The residue at each such pole equals the algebraic multiplicity of the corresponding eigenvalue \( \mu_\rho \in \Spec(L_{\sym}) \).

\medskip

Since \( L_{\sym} \in \TC(\HPsi) \cap \KC(\HPsi) \) is compact and self-adjoint, its spectrum is real and discrete, and the eigenvalues have finite algebraic multiplicities. The spectral trace calculus thus confirms that the poles of the logarithmic derivative coincide (in both location and multiplicity) with those arising from the Hadamard factorization of \( \XiR(s) \).

\medskip

Therefore, the order of vanishing of \( \zetaR(s) \) at each nontrivial zero \( \rho \) equals the algebraic multiplicity of the spectral point \( \mu_\rho \in \Spec(L_{\sym}) \), completing the analytic–spectral correspondence.
\end{proof}
