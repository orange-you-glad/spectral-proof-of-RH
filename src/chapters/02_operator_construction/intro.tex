\subsection*{Introduction}

The goal of this chapter is to construct the canonical compact, self-adjoint, trace-class operator
\[
L_{\sym} \in \TC(H_{\Psi_\alpha}),
\]
on the exponentially weighted Hilbert space \( H_{\Psi_\alpha} \), designed to mirror the spectral structure of the completed Riemann zeta function \( \Xi(s) \).

We proceed through five analytically rigorous stages:

\begin{itemize}
    \item We define the weighted Hilbert space
    \[
    H_{\Psi_\alpha} := L^2(\R, \Psi_\alpha(x)\, dx), \quad \Psi_\alpha(x) := e^{\alpha |x|}, \quad \alpha > \pi,
    \]
    chosen so that the inverse Fourier transform of the canonical spectral profile
    \[
    \phi(\lambda) := \Xi\left(\tfrac{1}{2} + i\lambda\right)
    \]
    lies in \( L^1(\R, \Psi_\alpha^{-1}\, dx) \). This ensures that convolution against \( \phi^\vee \) defines a trace-class operator on \( H_{\Psi_\alpha} \).

    \item We define mollified spectral profiles
    \[
    \varphi_t(\lambda) := e^{-t\lambda^2} \phi(\lambda),
    \]
    which lie in the Schwartz space \( \Schwartz \), and induce a family of convolution operators \( L_t \in \TC(H_{\Psi_\alpha}) \). Each \( L_t \) is bounded, self-adjoint, and satisfies the uniform estimate
    \[
    \| L_t \|_{\TC} \le C(\alpha), \quad \forall t \in (0,1],
    \]
    with \( C(\alpha) \) depending explicitly on decay bounds from mollifier profiles and exponential weights.

    \item We prove trace-norm convergence
    \[
    \| L_t - L_{\sym} \|_{\TC} \to 0 \quad \text{as } t \to 0^+,
    \]
    and define the canonical operator via limit:
    \[
    L_{\sym} := \lim_{t \to 0^+} L_t,
    \]
    which exists uniquely in \( \TC(H_{\Psi_\alpha}) \) and is independent of mollifier choice.

    \item We prove that \( L_{\sym} \) is compact, self-adjoint, and densely defined on \( H_{\Psi_\alpha} \), with \( \Schwartz \subset H_{\Psi_\alpha} \) serving as a symmetric core. Essential self-adjointness of both \( L_{\sym} \) and \( L_{\sym}^2 \) is established via kernel decay and Nelson’s analytic vector theorem.

    \item We prove that the trace vanishes:
    \[
    \Tr(L_{\sym}) = 0,
    \]
    ensuring canonical normalization in the determinant identity. This removes ambiguity in the exponential factor of Hadamard-type entire functions.
\end{itemize}

These results culminate in \thmref{thm:existence_Lsym}, \thmref{thm:sa_trace_class_Lsym}, and \thmref{thm:trace_zero_Lsym}, which collectively certify the existence, uniqueness, and analytic characterization of \( L_{\sym} \) as the canonical trace-class convolution operator associated with \( \Xi(s) \).

\medskip

This operator underlies the Carleman–zeta-regularized determinant identity
\[
\det\nolimits_{\zeta}(I - \lambda^2 L_{\sym}^2) = \Xi\left(\tfrac{1}{2} + i\lambda\right),
\]
to be rigorously derived in \secref{sec:determinant_identity}.

\medskip

All convergence results and trace-class bounds are derived from explicit kernel estimates, using Gaussian mollification and exponential decay. Every analytic property—trace-class inclusion, compactness, symmetry, essential self-adjointness, and spectral normalization—is rigorously verified. This construction lays the spectral groundwork for encoding the nontrivial zeros of \( \zeta \) via canonical Fredholm theory.

\paragraph{Contrast with Prior Spectral Proposals.}
In contrast to heuristic or speculative frameworks—such as the Hilbert–Pólya conjecture, Connes’s noncommutative trace formulas~\cite{Connes1999TraceFormula}, or Deninger’s cohomological program~\cite{Deninger1998Frobenius}—the operator \( L_{\sym} \) is constructed concretely within classical Hilbert space analysis. It satisfies verified analytic properties: trace-class compactness, self-adjointness, exponential kernel decay, and determinant compatibility with \( \Xi \). This resolves major technical gaps in previous proposals, which lacked either a spectral determinant, a compact operator realization, or analytic control over convergence.
