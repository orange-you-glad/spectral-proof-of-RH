\begin{lemma}[Uniqueness of Construction from Fixed Analytic Data]
\label{lem:construction_canonical_data}
Let \( H_{\Psi_\alpha} := L^2(\R, e^{\alpha |x|}\, dx) \) for fixed \( \alpha > \pi \), and let \( \widehat{\Xi} \in \mathcal{S}'(\R) \) denote the inverse Fourier transform of the canonical spectral profile
\[
\phi(\lambda) := \Xi\left( \tfrac{1}{2} + i\lambda \right).
\]

Let \( \eta \in \Schwartz \) be a non-negative mollifier satisfying
\[
\eta \ge 0, \qquad \int_{\R} \eta(x)\, dx = 1,
\]
and define its rescaled family:
\[
\eta_\epsilon(x) := \frac{1}{\epsilon} \eta\left( \frac{x}{\epsilon} \right), \qquad \epsilon > 0.
\]
Define the mollified spatial kernels via convolution:
\[
\widehat{\Xi}_\epsilon(x) := (\eta_\epsilon * \widehat{\Xi})(x),
\]
and the corresponding convolution operators:
\[
(L_\epsilon f)(x) := \int_{\R} \widehat{\Xi}_\epsilon(x - y)\, f(y)\, dy.
\]

Then:
\begin{enumerate}
  \item[\textnormal{(i)}] \textbf{Trace-Class Structure.}  
  For each \( \epsilon > 0 \), we have
  \[
  \widehat{\Xi}_\epsilon \in \Schwartz \cap L^1(\R, e^{\alpha |x|} dx),
  \]
  and hence \( L_\epsilon \in \TC(H_{\Psi_\alpha}) \), self-adjoint and compact.  
  This follows from Simon’s trace-norm kernel criterion under exponential conjugation~\cite[Thm.~3.1]{Simon2005TraceIdeals}.

  \item[\textnormal{(ii)}] \textbf{Trace-Norm Convergence and Uniqueness.}  
  The trace-norm limit
  \[
  L_{\sym} := \lim_{\epsilon \to 0^+} L_\epsilon \quad \text{in } \TC(H_{\Psi_\alpha})
  \]
  exists uniquely and is independent of the choice of mollifier \( \eta \), provided it satisfies the standard conditions above.

  This convergence follows from:
  \begin{itemize}
    \item Pointwise convergence: \( \widehat{\Xi}_\epsilon(x) \to \widehat{\Xi}(x) \) almost everywhere;
    \item Uniform exponential decay: \( \widehat{\Xi}_\epsilon \in L^1(\R, e^{\alpha |x|} dx) \) with common bound;
    \item Dominated convergence of conjugated kernels in \( L^1(\R^2) \), implying trace-norm convergence.
  \end{itemize}

  The resulting operator \( L_{\sym} \in \TC(H_{\Psi_\alpha}) \) depends only on the analytic input \( (\Xi, \alpha) \), and not on the mollifier \( \eta \). This confirms that the canonical operator \( L_{\sym} \) arises intrinsically from the spectral data of the Riemann zeta function.
\end{enumerate}
\end{lemma}
