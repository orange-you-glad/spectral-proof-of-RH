\begin{theorem}[Spectral Reformulation of the Riemann Hypothesis]
\label{thm:eq_of_rh}

Let \( L_{\mathrm{sym}} \in \mathcal{C}_1(H_{\Psi_\alpha}) \cap \mathcal{K}(H_{\Psi_\alpha}) \) denote the canonical compact, self-adjoint, trace-class operator on the exponentially weighted Hilbert space
\[
H_{\Psi_\alpha} := L^2(\mathbb{R}, e^{\alpha |x|} dx), \qquad \alpha > \pi,
\]
constructed via mollified convolution with the inverse Fourier transform of the completed zeta profile
\[
\phi(\lambda) := \Xi\left( \tfrac{1}{2} + i\lambda \right),
\]
as rigorously developed in Chapters~\ref{sec:operator_construction}–\ref{sec:heat_kernel_asymptotics} and Appendix~\ref{app:heat_kernel_construction}.

\medskip

Suppose its Carleman–\(\zeta\)-regularized Fredholm determinant satisfies the canonical identity:
\[
\det\nolimits_\zeta(I - \lambda L_{\mathrm{sym}}) = \frac{\Xi\left( \tfrac{1}{2} + i\lambda \right)}{\Xi\left( \tfrac{1}{2} \right)},
\qquad \forall \lambda \in \mathbb{C},
\]
as proven unconditionally in \thmref{thm:det_identity_revised}.

\medskip

Then the Riemann Hypothesis is equivalent to the spectral reality of \( L_{\mathrm{sym}} \):
\[
\RH \quad \Longleftrightarrow \quad \operatorname{Spec}(L_{\mathrm{sym}}) \subset \mathbb{R}.
\]

\medskip

\noindent
Explicitly, for each nontrivial zero \( \rho = \beta + i\gamma \) of \( \zeta(s) \), define the canonical spectral image:
\[
\mu_\rho := \frac{1}{i}(\rho - \tfrac{1}{2}) = \gamma \in \mathbb{C}.
\]
Then:
\begin{itemize}
  \item \( \mu_\rho \in \mathbb{R} \iff \operatorname{Re}(\rho) = \tfrac{1}{2} \);
  \item Hence,
  \[
  \operatorname{Spec}(L_{\mathrm{sym}}) \subset \mathbb{R}
  \quad \Longleftrightarrow \quad
  \text{All nontrivial zeros } \rho \text{ lie on the critical line.}
  \]
\end{itemize}

\medskip

\noindent
This equivalence follows from:
\begin{enumerate}
  \item The analytic–spectral identity for the determinant of \( L_{\mathrm{sym}} \);
  \item The bijective spectral map \( \rho \mapsto \mu_\rho \) between nontrivial zeros and the nonzero spectrum of \( L_{\mathrm{sym}} \) (see \thmref{thm:spectral_zero_bijection_revised});
  \item The algebraic inversion identity:
  \[
  \mu_\rho \in \mathbb{R} \iff \operatorname{Re}(\rho) = \tfrac{1}{2}.
  \]
\end{enumerate}

\medskip

\noindent
Thus, the Riemann Hypothesis is equivalent to the condition that the entire spectrum of a canonical trace-class operator lies on the real line. This establishes a logically acyclic, operator-theoretic reformulation of RH within the framework of Fredholm theory and spectral determinant calculus.
\end{theorem}
