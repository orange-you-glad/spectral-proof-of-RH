\begin{proof}[Proof of \lemref{lem:Lsym_square_selfadjoint}]
We establish essential self-adjointness of \( L_{\mathrm{sym}}^2 \) on the Schwartz core \( \mathcal{S}(\mathbb{R}) \subset H_{\Psi_\alpha} \).

\paragraph{Step 1: Invariance of the Core.}
Since \( L_{\mathrm{sym}} \) is defined via convolution with kernel \( k \in \mathcal{S}(\mathbb{R}) \), and convolution preserves the Schwartz space, we have:
\[
L_{\mathrm{sym}} : \mathcal{S}(\mathbb{R}) \to \mathcal{S}(\mathbb{R}), \quad \Rightarrow \quad L_{\mathrm{sym}}^2 : \mathcal{S}(\mathbb{R}) \to \mathcal{S}(\mathbb{R}).
\]
Thus, the domain \( \mathcal{S}(\mathbb{R}) \) is invariant under \( L_{\mathrm{sym}} \) and its square.

\paragraph{Step 2: Symmetry.}
Because \( L_{\mathrm{sym}} \) is self-adjoint, it follows that
\[
\langle L_{\mathrm{sym}}^2 f, g \rangle_{H_{\Psi_\alpha}} = \langle f, L_{\mathrm{sym}}^2 g \rangle_{H_{\Psi_\alpha}}, \qquad \forall f, g \in \mathcal{S}(\mathbb{R}),
\]
so \( L_{\mathrm{sym}}^2 \) is symmetric on \( \mathcal{S}(\mathbb{R}) \).

\paragraph{Step 3: Nelson's Analytic Vector Criterion.}
Since \( L_{\mathrm{sym}}^2 \) preserves \( \mathcal{S}(\mathbb{R}) \), and each \( f \in \mathcal{S}(\mathbb{R}) \) satisfies
\[
\| (L_{\mathrm{sym}}^2)^n f \| < C_n,
\]
the elements of \( \mathcal{S}(\mathbb{R}) \) are analytic vectors for \( L_{\mathrm{sym}}^2 \). Thus, by Nelson’s analytic vector theorem~\cite[Thm.~X.36]{ReedSimon1975II}, we conclude:
\[
L_{\mathrm{sym}}^2 \text{ is essentially self-adjoint on } \mathcal{S}(\mathbb{R}).
\]

\paragraph{Step 4: Spectral Discreteness.}
Since \( L_{\mathrm{sym}} \in \mathcal{C}_1(H_{\Psi_\alpha}) \), we have \( L_{\mathrm{sym}}^2 \in \mathcal{C}_1(H_{\Psi_\alpha}) \) as well. Then, by spectral theory of compact self-adjoint operators, the spectrum satisfies:
\[
\operatorname{Spec}(L_{\mathrm{sym}}^2) \subset [0, \infty),
\]
and consists of discrete real eigenvalues with finite multiplicity, accumulating only at zero.

\paragraph{Conclusion.}
The operator \( L_{\mathrm{sym}}^2 \) is essentially self-adjoint on \( \mathcal{S}(\mathbb{R}) \), with closure admitting a discrete spectral resolution. This confirms the analytic setup for functional calculus, heat kernel expansion, and spectral determinant theory.
\end{proof}
