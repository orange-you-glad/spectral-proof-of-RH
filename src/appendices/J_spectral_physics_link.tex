\section{Spectral Physics Interpretation}
\label{app:spectral_physics_link}

\noindent\textbf{[Noncritical Appendix]}  
This appendix explores heuristic physical interpretations of the canonical operator \( L_{\sym} \in \TC(\HPsi) \). No physical model is constructed. The analogy is strictly formal and logically inert. However, \( L_{\sym} \) admits a structure reminiscent of a quantum Hamiltonian whose spectrum encodes the rescaled nontrivial zeros of the Riemann zeta function:
\[
\Spec(L_{\sym}) = \left\{ \mu_n := \frac{1}{\gamma_n} \;\middle|\; \zetaR\left(\tfrac{1}{2} + i\gamma_n\right) = 0 \right\}.
\]

\subsection*{Partition Function Analogy}

The spectral trace
\[
Z(t) := \Tr(e^{-t L_{\sym}})
\]
resembles a partition function for a quantum system with energy levels \( \mu_n \). Its singular expansion,
\[
Z(t) \sim \frac{1}{\sqrt{4\pi t}} \log\left( \frac{1}{t} \right) + o(t^{-1/2}), \quad t \to 0^+,
\]
is typical of Laplace-type operators on singular spaces and reflects divergences observed in trace formulas on noncompact domains.

The associated zeta-regularized determinant behaves analogously to a regularized free energy in statistical mechanics. For its analytic definition and connection to \( \Xi(s) \), see \cref{thm:det_identity_revised} and Chapter~\ref{sec:determinant_identity}.

\subsection*{GUE Statistics and Inverse Spectrum}

Under the map
\[
\gamma_n \longmapsto \mu_n := \frac{1}{\gamma_n},
\]
the conjectured GUE distribution of zeta zeros~\cite{Montgomery1973PairCorrelation, Berry1986RiemannSpectra} transforms into a nonlinear spacing distribution on \( \Spec(L_{\sym}) \). This map compresses high-energy modes and magnifies low-frequency arithmetic structure.

From this viewpoint, \( L_{\sym} \) functions as a trace-class compression of an arithmetic Hamiltonian—capturing global zeta statistics in an analytic setting.

\subsection*{Caveats and Interpretation}

These analogies do not influence any analytic result in the manuscript. No Hamiltonian, Lagrangian, or path integral is defined here.

Nonetheless, this perspective may motivate further inquiry into:
\begin{itemize}
  \item Quantum realizations of spectral zeta functions;
  \item Inverse-spectral statistics and trace-class ensembles;
  \item Hamiltonian interpretations of arithmetic trace formulas.
\end{itemize}

\medskip
\noindent
The operator \( L_{\sym} \) is a canonical analytic realization of the nontrivial zeta spectrum. Whether it admits a deeper physical interpretation—perhaps via quantization or arithmetic gauge theory—remains open.

\medskip
\noindent
For the analytic proof that
\[
\RH \iff \Spec(L_{\sym}) \subset \R,
\]
see Chapter~\ref{sec:spectral_implications}.
