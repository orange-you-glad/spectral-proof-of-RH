\begin{proof}[Proof of \lemref{lem:spectral_decay_bounds}]
Let \( \rho = \tfrac{1}{2} + i\gamma \) be a nontrivial zero of the Riemann zeta function \( \zeta(s) \), and define the corresponding eigenvalue of the canonical operator \( L_{\mathrm{sym}} \in \mathcal{C}_1(H_{\Psi_\alpha}) \) by
\[
\mu_\rho := \frac{1}{i}(\rho - \tfrac{1}{2}) = \frac{1}{\gamma}.
\]

\paragraph{Step 1: Zero Counting and Asymptotic Spacing.}
Let \( N(T) \) denote the number of nontrivial zeros \( \rho = \tfrac{1}{2} + i\gamma \) with \( 0 < \gamma \leq T \). Classical estimates (see \cite{Titchmarsh1986Zeta}) give:
\[
N(T) = \frac{T}{2\pi} \log \left( \frac{T}{2\pi} \right) + O(T).
\]
This implies the average spacing between consecutive \( \gamma_n \) behaves like \( \log \gamma_n \), so \( \gamma_n \to \infty \) sublinearly.

\paragraph{Step 2: Spectral Decay via Inversion.}
Since \( \mu_\rho = 1/\gamma \), the small eigenvalues of \( L_{\mathrm{sym}} \) correspond to large \( \gamma \). From the estimate above, the number of zeros with \( \gamma \geq T \) is
\[
\#\left\{ \rho : \gamma \geq T \right\} = N(\infty) - N(T) \sim O\left( \frac{T}{\log T} \right).
\]
Therefore, the number of eigenvalues with \( |\mu_\rho| \leq x \) satisfies
\[
\#\left\{ \mu_\rho : |\mu_\rho| \leq x \right\} = O\left( \frac{1}{x \log(1/x)} \right)
\quad \text{as } x \to 0^+,
\]
by substituting \( \gamma = 1/x \). Equivalently, this implies
\[
N(x) := \#\left\{ \mu_\rho : |\mu_\rho| \geq x \right\} = O\left( \frac{1}{x \log(1/x)} \right).
\]

\paragraph{Step 3: Membership in Schatten Ideals.}
This decay implies the eigenvalue sequence \( \{ \mu_\rho \} \) lies in \( \ell^p(\mathbb{R}) \) for all \( p > 1 \), and specifically in \( \ell^2 \), verifying that \( L_{\mathrm{sym}} \in \mathcal{C}_2(H_{\Psi_\alpha}) \subset \mathcal{C}_1(H_{\Psi_\alpha}) \), consistent with earlier analysis.
\end{proof}
