\begin{proof}[Proof of \lemref{lem:spectral_rigidity_determinant}]
Let \( L_1, L_2 \in \TC(\HPsi) \cap \KC(\HPsi) \) be compact, self-adjoint, trace-class operators satisfying:
\[
\det\nolimits_\zeta(I - \lambda L_1) = \det\nolimits_\zeta(I - \lambda L_2), \quad \forall \lambda \in \C,
\]
with both determinants normalized at the origin: \( \det\nolimits_\zeta(I) = 1 \).

\paragraph{Step 1: Spectral Encoding via Determinant Structure.}
For compact, self-adjoint operators in \( \TC \), the zeta-regularized Fredholm determinant admits the canonical Hadamard product expansion:
\[
\det\nolimits_\zeta(I - \lambda L_j) = \prod_{\mu \in \Spec(L_j) \setminus \{0\}} (1 - \lambda \mu)^{\operatorname{mult}_{L_j}(\mu)},
\quad j = 1,2.
\]
Since the two determinants coincide as entire functions of order one and exponential type \( \pi \), and share the normalization \( \det\nolimits_\zeta(I) = 1 \), the identity theorem for entire functions implies that their zero sets (counted with multiplicity) must coincide. Hence:
\[
\Spec(L_1) \setminus \{0\} = \Spec(L_2) \setminus \{0\}
\quad \text{as multisets}.
\]

\paragraph{Step 2: Completion via Spectral Theorem.}
If \( L_1 \) and \( L_2 \) act on the same Hilbert space \( \HPsi \), then the spectral theorem for compact self-adjoint operators ensures the existence of a unitary operator
\[
U \colon \HPsi \to \HPsi
\quad \text{such that} \quad
L_2 = U L_1 U^{-1}.
\]

\paragraph{Conclusion.}
Thus, the Carleman–\(\zeta\)-regularized Fredholm determinant serves as a complete spectral fingerprint for compact, self-adjoint trace-class operators: the analytic data of the determinant determines the operator spectrum uniquely, and—on a fixed Hilbert space—determines the operator itself up to unitary equivalence.
\end{proof}
