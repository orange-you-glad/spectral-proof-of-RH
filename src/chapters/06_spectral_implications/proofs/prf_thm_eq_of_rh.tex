\begin{proof}[Proof of \thmref{thm:eq_of_rh}]
Let \( \rho \mapsto \mu_\rho := \dfrac{1}{i}(\rho - \tfrac{1}{2}) \) denote the canonical spectral reparametrization of the nontrivial zeros \( \rho \) of the Riemann zeta function \( \zeta(s) \). This map is bijective and multiplicity-preserving by \thmref{thm:spectral_zero_bijection_revised} and \lemref{lem:multiplicity_preservation}, ensuring complete spectral correspondence with the nonzero eigenvalues of \( \Lsym \in \TC(\HPsi) \).

\medskip

By \thmref{thm:det_identity_revised}, the Carleman \(\zeta\)-regularized Fredholm determinant of \( \Lsym \) satisfies:
\[
\detz(I - \lambda \Lsym) = \frac{\Xi\left( \tfrac{1}{2} + i\lambda \right)}{\Xi\left( \tfrac{1}{2} \right)},
\qquad \forall \lambda \in \C.
\]
This identity is constructed without assuming RH and rests on analytic inputs from Chapter~\ref{sec:heat_kernel_asymptotics} and Appendix~\ref{app:heat_kernel_construction}, particularly the convergence and singularity structure derived in \lemref{lem:heat_trace_expansion}, \lemref{lem:trace_class_Lt}, and \lemref{lem:kernel_trace_norm_convergence}.

\medskip

Now observe the algebraic implication: for any nontrivial zero \( \rho = \sigma + i\gamma \),
\[
\mu_\rho := \frac{1}{i}(\rho - \tfrac{1}{2}) = \gamma + i(\sigma - \tfrac{1}{2}).
\]
Thus,
\[
\mu_\rho \in \R \iff \Re(\rho) = \tfrac{1}{2}.
\]

\paragraph*{\( \Rightarrow \): Spectral Reality Implies \(\RH\).}
Assume \( \Spec(\Lsym) \subset \R \). Then each \( \mu_\rho \in \R \), so the identity above implies that \( \Re(\rho) = \tfrac{1}{2} \). Hence, all nontrivial zeros lie on the critical line and RH holds.

\paragraph*{\( \Leftarrow \): \(\RH\) Implies Spectral Reality.}
Conversely, assume RH holds. Then for each \( \rho \), \( \Re(\rho) = \tfrac{1}{2} \), which implies \( \mu_\rho \in \R \). Therefore, all nonzero eigenvalues of \( \Lsym \) are real. By self-adjointness and the spectral theorem, \( \Spec(\Lsym) \subset \R \).

\paragraph*{Conclusion.}
The spectrum of \( \Lsym \) is real if and only if all nontrivial zeros of \( \zeta(s) \) lie on the critical line. This establishes the analytic–spectral equivalence
\[
\RH \iff \Spec(\Lsym) \subset \R,
\]
completing the modular operator-theoretic reformulation of the Riemann Hypothesis.
\end{proof}
