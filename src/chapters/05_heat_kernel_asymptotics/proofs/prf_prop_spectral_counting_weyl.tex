\begin{proof}[Proof of Proposition~\ref{prop:spectral_counting_weyl}]
From the refined short-time expansion of the heat trace (see Proposition~\ref{prop:short_time_heat_expansion}), we have:
\[
\operatorname{Tr}(e^{-t L_{\mathrm{sym}}^2}) \sim \frac{1}{\sqrt{4\pi t}} \log\left( \frac{1}{t} \right), \qquad \text{as } t \to 0^+.
\]
Let \( \{ \mu_n^2 \} \subset (0, \infty) \) denote the nonzero eigenvalues of \( L_{\mathrm{sym}}^2 \), counted with multiplicity, and define the counting function
\[
N(\lambda) := \#\{ n : \mu_n^2 \leq \lambda \}.
\]

By classical Tauberian theorems (cf.~\cite[Thm.~4.12.9]{Korevaar2004Tauberian}), and as made precise in Lemma~\ref{lem:tauberian_heat_trace_application}, a heat trace expansion of the form
\[
\Theta(t) \sim \frac{1}{\sqrt{4\pi t}} \log\left( \frac{1}{t} \right)
\]
as \( t \to 0^+ \) implies that
\[
N(\lambda) \sim C \, \lambda^{1/2} \log \lambda, \qquad \text{as } \lambda \to \infty,
\]
for some constant \( C > 0 \). The logarithmic enhancement reflects the leading singularity in the trace and modifies the usual Weyl law for spectral dimension one.

\paragraph{Conclusion.}
This Tauberian correspondence translates the singular trace asymptotics into a spectral density estimate for the eigenvalues of \( L_{\mathrm{sym}}^2 \), yielding the claimed log-modified growth for the counting function.
\end{proof}
