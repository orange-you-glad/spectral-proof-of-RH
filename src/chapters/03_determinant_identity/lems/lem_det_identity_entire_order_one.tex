\begin{lemma}[Determinant Identity Defines an Entire Function of Order One and Type \texorpdfstring{\( \pi \)}{π}]
\label{lem:det_identity_entire_order_one}

Let \( L_{\sym} \in \TC(\HPsi) \) be a compact, self-adjoint, trace-class operator.

Then the map
\[
\lambda \mapsto \det\nolimits_\zeta(I - \lambda L_{\sym})
\]
extends to an entire function on \( \C \) of order one and exponential type \( \pi \). That is, there exists a constant \( C > 0 \) such that for all \( \lambda \in \C \),
\[
\left| \det\nolimits_\zeta(I - \lambda L_{\sym}) \right| \le C e^{\pi |\lambda|}.
\]
In particular,
\[
\det\nolimits_\zeta(I - \lambda L_{\sym}) \in \mathcal{E}_1^\pi,
\]
the Hadamard class of entire functions of order one and exact exponential type \( \pi \).

\medskip
\noindent
The exponential type is governed by the Paley--Wiener theorem~\cite[Ch.~9]{Levin1996EntireLectures}, which bounds the growth of entire functions whose Fourier transforms have compact support. In this case, the support of the kernel’s Fourier transform \( \ft{\phi} \), inherited from the exponential type of \( \Xi(s) \), restricts the convolution kernel \( K(x - y) \) to type \( \pi \). This control on kernel growth ensures that the Fredholm determinant exhibits exactly exponential type \( \pi \).
\end{lemma}
