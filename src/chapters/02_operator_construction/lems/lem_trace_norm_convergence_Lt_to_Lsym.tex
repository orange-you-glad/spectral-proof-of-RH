\begin{lemma}[Trace-Norm Convergence \( L_t \to L_{\sym} \)]
\label{lem:trace_norm_convergence_Lt_to_Lsym}
Fix \( \alpha > \pi \), and define the exponentially weighted Hilbert space
\[
H_{\Psi_\alpha} := L^2(\R, e^{\alpha|x|}\, dx).
\]
Let
\[
\phi(\lambda) := \Xi\left( \tfrac{1}{2} + i\lambda \right)
\]
be the canonical Fourier profile.

Define mollified spectral profiles and convolution kernels by
\[
\varphi_t(\lambda) := e^{-t\lambda^2} \phi(\lambda), \qquad
K_t(x,y) := \frac{1}{2\pi} \int_{\R} e^{i(x - y)\lambda} \varphi_t(\lambda)\, d\lambda.
\]
Then for each \( t > 0 \), the convolution operator
\[
(L_t f)(x) := \int_{\R} K_t(x, y)\, f(y)\, dy
\]
defines a trace-class operator:
\[
L_t \in \TC(H_{\Psi_\alpha}), \qquad \sup_{0 < t \le 1} \| L_t \|_{\TC} < \infty.
\]
This follows from the exponential decay of \( K_t(x,y) \), implied by Paley--Wiener theory applied to \( \varphi_t \in PW_\pi \cap \Schwartz \)~\cite[Thm.~IX.12]{ReedSimon1975II}, and classical trace-norm criteria for weighted Hilbert spaces~\cite[Ch.~4]{Simon2005TraceIdeals}.

\medskip

Define the canonical convolution operator as the Schatten-1 limit:
\[
L_{\sym} := \lim_{t \to 0^+} L_t \quad \text{in } \TC(H_{\Psi_\alpha}),
\]
i.e.,
\[
\| L_t - L_{\sym} \|_{\TC} \to 0 \quad \text{as } t \to 0^+.
\]

This convergence is ensured by the following:

\begin{itemize}
  \item Pointwise convergence: \( \varphi_t(\lambda) \to \phi(\lambda) \) for all \( \lambda \in \R \);
  \item Local \( L^1 \)-convergence: \( \varphi_t \to \phi \) in \( L^1_{\mathrm{loc}}(\R) \);
  \item Uniform trace-norm bounds on \( L_t \), and exponential decay of the mollified kernels \( K_t \);
  \item The limiting kernel
  \[
  K_{\sym}(x,y) := \frac{1}{2\pi} \int_{\R} e^{i(x-y)\lambda} \phi(\lambda)\, d\lambda
  \]
  satisfies \( K_{\sym} \in L^1(\R^2, \Psi_\alpha(x)\Psi_\alpha(y)\, dx\, dy) \), so
  \[
  L_{\sym} \in \TC(H_{\Psi_\alpha}).
  \]
\end{itemize}

\paragraph{Spectral Implications.}
Since \( L_t \to L_{\sym} \) in trace norm, we have:
\[
\Tr\left(e^{-tL_t^2}\right) \to \Tr\left(e^{-tL_{\sym}^2}\right) \quad \text{as } t \to 0^+,
\]
and for all \( \lambda \in \R \),
\[
\det\nolimits_{\zeta}(I - \lambda L_t) \to \det\nolimits_{\zeta}(I - \lambda L_{\sym}).
\]
These analytic consequences underpin the spectral determinant identity, rigorously developed in \secref{sec:determinant_identity}.
\end{lemma}
