\begin{lemma}[Verification of Korevaar Tauberian Hypotheses]
\label{lem:korevaar_tauberian_constants_verified}

Let \( \Theta(t) := \Tr(e^{-t L_{\sym}^2}) \) denote the spectral heat trace of the canonical compact, self-adjoint, trace-class operator \( L_{\sym} \in \TC(\HPsi) \). Define the squared spectral counting function:
\[
A(\Lambda) := \#\left\{ \mu \in \Spec(L_{\sym}) : \mu^2 \le \Lambda \right\}.
\]

Assume the refined short-time heat trace asymptotic holds:
\[
\Theta(t) \sim \frac{1}{\sqrt{4\pi t}} \log\left( \frac{1}{t} \right), \qquad \text{as } t \to 0^+.
\]

Then all the conditions of Korevaar’s Tauberian theorem for log-modulated Laplace transforms~\cite[Ch.~III, Thm.~5.5]{Korevaar2004Tauberian} are satisfied:

\begin{enumerate}
  \item[\textup{(i)}] \( \Theta(t) \) is nonnegative and locally bounded, and becomes monotonic on some interval \( t \in (0, \varepsilon) \);

  \item[\textup{(ii)}] \( \Theta \) admits the Laplace–Stieltjes representation:
  \[
  \Theta(t) = \int_0^\infty e^{-t \lambda} \, dA(\lambda),
  \]
  where \( A(\lambda) \) is right-continuous, monotone increasing, and diverges as \( \lambda \to \infty \);

  \item[\textup{(iii)}] \( \Theta \in \mathcal{R}_{1/2}^{\log}(0^+) \), i.e., it is regularly varying at the origin with index \( \tfrac{1}{2} \), modulated by a slowly varying term \( \log(1/t) \);

  \item[\textup{(iv)}] The Laplace transform satisfies the inversion hypotheses of Korevaar’s theorem, and therefore:
  \[
  A(\Lambda) \sim \frac{\sqrt{\Lambda}}{\pi} \log \Lambda, \qquad \text{as } \Lambda \to \infty.
  \]
\end{enumerate}

\noindent
Hence, all prerequisites for Korevaar’s log-corrected Tauberian inversion are met, and the spectral counting law stated in \lemref{lem:laplace_kernel_growth_class} and \lemref{lem:log_corrected_tauberian_estimate} follows rigorously.
\end{lemma}

\begin{remark}[Verification of Tauberian Conditions]
The function \( \Theta(t) = \Tr(e^{-tL^2}) \) satisfies \( \Theta(t) \sim t^{-1/2} \log(1/t) \), with \( \log(1/t) \) slowly varying. All regular variation and convexity conditions in Korevaar's Theorem III.5.1 are met, as shown in \lemref{lem:laplace_kernel_growth_class}.
\end{remark}
