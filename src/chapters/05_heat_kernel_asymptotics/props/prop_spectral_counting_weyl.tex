\begin{proposition}[Spectral Counting Function]
\label{prop:spectral_counting_weyl}
Let \( \{ \mu_n \} \subset \mathbb{R} \setminus \{0\} \) denote the nonzero eigenvalues of the canonical compact, self-adjoint operator \( L_{\mathrm{sym}} \in \mathcal{C}_1(H_{\Psi_\alpha}) \), ordered by increasing absolute value and counted with multiplicity. Define the spectral counting function
\[
N(\lambda) := \#\{ n \in \mathbb{N} : \mu_n^2 \leq \lambda \}, \qquad \lambda > 0.
\]
Then, as \( \lambda \to \infty \), the function \( N(\lambda) \) satisfies the asymptotic growth law
\[
N(\lambda) \sim C\, \lambda^{1/2} \log \lambda,
\]
for some constant \( C > 0 \) determined by the leading singularity in the short-time heat trace expansion.

\medskip
\noindent
This result follows from the singular expansion
\[
\operatorname{Tr}(e^{-t L_{\mathrm{sym}}^2}) \sim \frac{1}{\sqrt{4\pi t}} \log\left( \frac{1}{t} \right) + \cdots \quad \text{as } t \to 0^+,
\]
via a Tauberian inversion argument. In particular, the spectral counting law exhibits sub-Weyl growth with a logarithmic enhancement, reflecting the non-classical scaling of the canonical convolution operator \( L_{\mathrm{sym}} \) on the weighted space \( H_{\Psi_\alpha} \).
\end{proposition}
