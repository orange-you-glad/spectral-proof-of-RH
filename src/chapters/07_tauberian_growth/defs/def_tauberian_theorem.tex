\begin{definition}[Tauberian Theorem for Spectral Counting]
\label{def:tauberian_theorem}

Let \( L \in \TC(H) \) be a compact, self-adjoint operator on a separable Hilbert space \( H \), with discrete nonzero spectrum \( \{ \mu_n \} \subset \R \setminus \{0\} \), counted with multiplicity. Define the squared spectral counting function
\[
A(\Lambda) := \#\left\{ n \in \N : \mu_n^2 \le \Lambda \right\}, \qquad \Lambda > 0.
\]

Suppose the spectral heat trace satisfies the short-time asymptotic expansion:
\[
\Tr(e^{-t L^2}) = \frac{C}{t^\alpha} + o(t^{-\alpha}), \qquad \text{as } t \to 0^+,
\]
for some constant \( C > 0 \) and exponent \( \alpha > 0 \). Then the spectral counting function obeys:
\[
A(\Lambda) \sim \frac{C}{\Gamma(\alpha + 1)} \, \Lambda^\alpha, \qquad \text{as } \Lambda \to \infty.
\]

This follows from the classical Karamata Tauberian theorem applied to the Laplace–Stieltjes representation:
\[
\Tr(e^{-t L^2}) = \int_0^\infty e^{-t \lambda} \, dA(\lambda).
\]

\medskip

If instead the heat trace exhibits logarithmic modulation:
\[
\Tr(e^{-t L^2}) \sim \frac{C}{t^\alpha} \log\left( \frac{1}{t} \right), \qquad \text{as } t \to 0^+,
\]
then Korevaar’s log-corrected Tauberian theorem~\cite[Ch.~III, §5]{Korevaar2004Tauberian} yields the refined asymptotic:
\[
A(\Lambda) \sim \frac{C}{\Gamma(\alpha + 1)} \, \Lambda^\alpha \log \Lambda, \qquad \text{as } \Lambda \to \infty.
\]

\medskip

\noindent
\emph{Remark.} See Appendix~\ref{app:notation_summary} for definitions of the regularly varying classes \( \mathcal{R}_\alpha \) and \( \mathcal{R}_\alpha^{\log} \). Korevaar’s framework guarantees that the leading growth law is uniquely determined by the trace singularity, ensuring spectral asymptotic rigidity.
\end{definition}
