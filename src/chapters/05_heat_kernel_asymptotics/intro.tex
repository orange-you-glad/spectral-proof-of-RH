\subsection*{Introduction}
\label{sec:intro_heat_kernel_asymptotics}

This chapter develops a detailed short-time analysis of the spectral heat trace
\[
\Tr(e^{-t \Lsym^2}),
\]
where \( \Lsym \in \TC(\HPsi) \) is the canonical compact, self-adjoint operator constructed in \secref{sec:operator_construction}. The operator \( \Lsym \) generates a strongly continuous, holomorphic, trace-class contraction semigroup
\[
e^{-t \Lsym^2} \in \TC(\HPsi) \cap \mathcal{B}(\HPsi), \qquad t > 0,
\]
defined via the spectral theorem and classical heat kernel calculus~\cite[Ch.~X]{ReedSimon1978IV}, \cite[Ch.~3]{Simon2005TraceIdeals}.

\paragraph{Two-Sided Heat Trace Bounds.}
We establish that for some constants \( c_1, c_2 > 0 \) and \( t_0 > 0 \),
\[
c_1\, t^{-1/2} \le \Tr(e^{-t \Lsym^2}) \le c_2\, t^{-1/2}, \qquad \text{for all } 0 < t \le t_0.
\]
These estimates reflect a spectral dimension of one, with inverse-square spectral density and a logarithmic singularity governed by the analytic structure of \( \Xi(s) \).

\paragraph{Analytic Framework.}
The asymptotic expansion is derived using operator-theoretic tools, kernel analysis, and Paley–Wiener theory. Key analytic components include:
\begin{itemize}
  \item Gaussian mollification and exponential decay of the Fourier profile \( \phi(\lambda) = \Xi(\tfrac{1}{2} + i\lambda) \), ensuring Schwartz-class inverse transforms~\cite[Ch.~IX]{ReedSimon1975II}, \cite[Ch.~9]{Levin1996EntireLectures}.
  \item Positivity and diagonal structure of the kernel \( K_t(x, y) \), including \( K_t(x, x) > 0 \) from \lemref{lem:heat_kernel_diagonal_positivity}, and off-diagonal Gaussian decay.
  \item Trace-norm convergence \( L_t \to \Lsym \) and heat trace convergence under dominated convergence on \( \HPsi \).
  \item Pointwise asymptotic expansion of the kernel and its trace:
  \[
  \begin{aligned}
  K_t(x,x) &\sim \sum_{n=0}^\infty a_n(x)\, t^{n - 1/2}, \\
  \Tr(e^{-t \Lsym^2}) &\sim \sum_{n=0}^\infty A_n\, t^{n - 1/2}, \quad A_n := \int_{\R} a_n(x)\, dx,
  \end{aligned}
  \]
  with an explicit logarithmic singularity controlled in \lemref{lem:heat_trace_expansion} and \propref{prop:short_time_heat_expansion}.
\end{itemize}

\paragraph{Spectral Class and Regular Variation.}
The singular behavior of the trace function \( \Theta(t) := \Tr(e^{-t \Lsym^2}) \) places it in the log-modulated regularly varying class
\[
\Theta(t) \in \mathcal{R}_{1/2}^{\log}(0^+),
\]
which is essential for Tauberian inversion and spectral growth analysis.

\paragraph{Determinant and Spectral Implications.}
The refined expansion
\[
\Tr(e^{-t \Lsym^2}) = \frac{\log(1/t)}{\sqrt{4\pi t}} + c_0 \sqrt{t} + o(\sqrt{t}),
\]
determines the Hadamard type and normalization of the canonical zeta-regularized determinant \( \detz(I - \lambda \Lsym) \). This structure underlies the Laplace–Mellin definition (\lemref{lem:log_derivative_determinant}) and guarantees the log-derivative identity required for analytic continuation. The singularity is necessary to match the spectral zeta pole at \( s = 0 \) and reflects the logarithmic trace anomaly predicted by Hadamard factorization~\cite[Ch.~III]{Korevaar2004Tauberian}.

\begin{remark}[Modular Validation of Determinant Structure]
Several heat trace asymptotics developed in this chapter—especially the singular expansion and Laplace integrability—were invoked in Chapter~\ref{sec:determinant_identity} to justify the structure of the canonical determinant. These references are modular and acyclic, as tracked in the DAG (\cref{app:dependency_graph}), and this chapter completes the analytic justification of the determinant’s growth rate and spectral zeta continuation.
\end{remark}

\paragraph{Outlook.}
These asymptotic results provide the analytic foundation for Chapter~\ref{sec:tauberian_growth}, where Korevaar’s log-corrected Tauberian theorem is applied to invert the trace expansion and derive the spectral counting law
\[
N(\Lambda) = \frac{\sqrt{\Lambda}}{\pi} \log \Lambda + O(\sqrt{\Lambda}),
\]
thereby confirming that the spectrum of \( \Lsym \) encodes the density and growth rate of the nontrivial zeros of \( \zeta(s) \).
