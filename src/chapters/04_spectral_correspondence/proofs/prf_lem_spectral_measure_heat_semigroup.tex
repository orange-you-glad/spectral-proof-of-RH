\begin{proof}[Proof of \lemref{lem:spectral_measure_heat_semigroup}]
We prove the result using the spectral theorem and trace properties of compact, self-adjoint operators.

Since \( L_{\sym} \in \TC(\HPsi) \) is compact and self-adjoint, it admits a discrete spectral decomposition:
\[
L_{\sym} = \sum_{\rho} \mu_\rho P_\rho,
\]
where each \( \mu_\rho \in \R \) is an eigenvalue with finite multiplicity and \( P_\rho \) denotes the corresponding orthogonal projection.

\paragraph{Heat Trace via Spectral Functional Calculus.}
Using the spectral calculus, the heat semigroup satisfies:
\[
e^{-t L_{\sym}^2} = \sum_{\rho} e^{-t\mu_\rho^2} P_\rho,
\]
and thus the trace is given by:
\[
\Tr(e^{-t L_{\sym}^2}) = \sum_{\rho} e^{-t\mu_\rho^2} \Tr(P_\rho) = \sum_{\rho} \operatorname{mult}_{\mathrm{spec}}(\mu_\rho) \, e^{-t\mu_\rho^2}.
\]
Define the spectral counting measure
\[
dN(\lambda) := \sum_{\rho} \operatorname{mult}_{\mathrm{spec}}(\mu_\rho) \, \delta_{\mu_\rho}(\lambda),
\]
so the trace becomes the Laplace-type integral:
\[
\Tr(e^{-t L_{\sym}^2}) = \int_{\R} e^{-t\lambda^2} \, dN(\lambda).
\]

\paragraph{Spectral Trace Identity for Test Functions.}
For any test function \( \phi \in \Schwartz \), the spectral theorem gives
\[
\phi(L_{\sym}) = \sum_{\rho} \phi(\mu_\rho) P_\rho,
\]
and taking the trace yields:
\[
\Tr(\phi(L_{\sym})) = \sum_{\rho} \phi(\mu_\rho) \Tr(P_\rho) = \int_{\R} \phi(\lambda) \, dN(\lambda).
\]

\paragraph{Fredholm Logarithmic Expansion.}
The Carleman \(\zeta\)-regularized determinant admits the logarithmic trace representation:
\[
\log \det\nolimits_\zeta(I - \lambda L_{\sym}) = - \sum_{\rho} \log\left(1 - \lambda \mu_\rho^{-1} \right) = - \int_{\R} \log\left(1 - \frac{\lambda}{\lambda'} \right) \, dN(\lambda'),
\]
valid for \( \lambda \in \C \setminus \{ \mu_\rho \} \). This matches the Hadamard representation for entire functions of order one, whose zero set corresponds to \( \Spec(L_{\sym}) \setminus \{0\} \).

\paragraph{Conclusion.}
The trace of the heat semigroup \( e^{-t L_{\sym}^2} \) and the logarithmic expansion of the determinant are both encoded by the counting measure \( dN \), which captures spectral multiplicity. This completes the proof.
\end{proof}
