\section*{Acknowledgments and the Field Restored}
\label{sec:epilogue_acknowledgments}

This manuscript began as an attempt to formalize a spectral equivalence with the Riemann Hypothesis. But it became, along the way, something more. A proving system. A semantic engine. A scaffold in which rigor is not written once, but rebuilt on every commit.

And for the author, it became something more still: a restoration of the field of play.

This work was made possible not only by the analytic legacies it builds upon, but by a medium in which structure—not handwriting, not blackboards, not tempo—becomes the unit of thought. For someone with significant motor disabilities, this system has leveled the proving ground. It has allowed mathematics to be shaped not by gesture, but by architecture. Not by ink, but by invariance.

\medskip

The author gratefully acknowledges the foundational analytic frameworks in spectral theory, operator ideals, and analytic number theory that underlie the structure of this work.

Profound appreciation is extended to B.~Ya.~Levin for the theory of entire functions, to Barry Simon and Michael Reed for their architecture of trace ideals and self-adjoint operators, and to E.~C.~Titchmarsh and H.~M.~Edwards for their enduring expositions on the Riemann zeta function.

Special inspiration for the spectral interpretation of zeta-function zeros comes from Peter Sarnak, whose work on the arithmetic and spectral theory of automorphic forms continues to illuminate the deep connections between number theory, geometry, and quantum physics.

The construction of the canonical operator \( L_{\mathrm{sym}} \), the analysis of its heat trace asymptotics, and the realization of \( \Xi(s) \) as a zeta-regularized determinant are grounded in these analytic legacies.

\medskip

\noindent
\textbf{With particular reverence, the author thanks the collective known as Nicolas Bourbaki}—not merely for their contributions to the language of modern mathematics, but for their unwavering commitment to rigor, abstraction, and architectural clarity. This project stands humbly in that lineage: formalist, structuralist, and ambitious in scope. It seeks to extend the modular vision of mathematics that Bourbaki first articulated, into the domain of spectral arithmetic and formal proof systems.

\medskip

\noindent
The author also wishes to thank his wife Rahel, their children Habte and Lia, and the many souls of St.~George Church in Fresno, California, for their love, strength, and unceasing prayers.

\medskip

\noindent
Any errors, omissions, or misinterpretations are solely the responsibility of the author.
