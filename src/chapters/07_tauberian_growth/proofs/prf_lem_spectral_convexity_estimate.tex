\begin{proof}[Proof of Lemma~\ref{lem:spectral_convexity_estimate}]
Let \( \{ \lambda_n \} \subset \mathbb{R}_{>0} \) denote the nonzero eigenvalues of \( L_{\mathrm{sym}}^2 \in \mathcal{C}_1(H_{\Psi_\alpha}) \), ordered non-decreasingly and counted with multiplicity. Define the spectral counting function
\[
N(\lambda) := \#\left\{ n \in \mathbb{N} : \lambda_n \le \lambda \right\}.
\]

\paragraph{Step 1: Heat Trace as Laplace–Stieltjes Transform.}
Since \( L_{\mathrm{sym}}^2 \) is positive and trace class, the spectral representation gives
\[
\operatorname{Tr}(e^{-t L_{\mathrm{sym}}^2}) = \sum_{n=1}^\infty e^{-t \lambda_n}
= \int_0^\infty e^{-t\lambda} \, dN(\lambda),
\]
where \( dN(\lambda) = \sum_n \delta_{\lambda_n} \) is a locally finite positive measure.

\paragraph{Step 2: Short-Time Envelope from Chapter~\ref{sec:heat_kernel_asymptotics}.}
From Proposition~\ref{prop:two_sided_heat_trace_bounds}, we have the upper envelope
\[
\operatorname{Tr}(e^{-t L_{\mathrm{sym}}^2}) \leq c_2 \, t^{-1/2}, \qquad \text{as } t \to 0^+.
\]
Hence,
\[
\int_0^\infty e^{-t\lambda} \, dN(\lambda) \lesssim t^{-1/2}.
\]

\paragraph{Step 3: Tauberian Inversion.}
By the classical Tauberian theorem for Laplace–Stieltjes transforms (see Definition~\ref{def:tauberian_theorem} and Lemma~\ref{lem:explicit_korevaar_tauberian_bound}), we may conclude:
\[
\int_0^\infty e^{-t\lambda} \, dN(\lambda) \lesssim t^{-\alpha} \quad \Longrightarrow \quad N(\lambda) \lesssim \lambda^{\alpha} \quad \text{as } \lambda \to \infty.
\]
Here, \( \alpha = 1/2 \), and thus,
\[
N(\lambda) \leq C\, \lambda^{1/2}, \qquad \text{for all } \lambda \ge \lambda_0,
\]
for some constants \( C > 0 \), \( \lambda_0 > 0 \).

\paragraph{Conclusion.}
This proves the spectral convexity estimate and confirms that the counting function lies in the regular variation class \( \mathcal{R}_{1/2} \), providing Tauberian admissibility for further log-corrected asymptotic refinement in Chapter~\ref{sec:tauberian_growth}.
\end{proof}
