\begin{corollary}[Spectral Determination of the Zeta Zeros]
\label{cor:spectral_determines_zeta}

The spectrum of the canonical operator \( L_{\sym} \in \TC(\HPsi) \) determines the nontrivial zeros of the Riemann zeta function completely and canonically.

That is, there exists a bijection:
\[
\Spec(L_{\sym}) \setminus \{0\}
\;\longleftrightarrow\;
\left\{ \rho \in \C : \zetaR(\rho) = 0, \; 0 < \Re(\rho) < 1 \right\},
\]
given by the canonical inverse map:
\[
\mu \mapsto \rho := \tfrac{1}{2} + i \mu^{-1},
\]
with multiplicities preserved.

\medskip

\noindent
In particular, the spectral data of \( L_{\sym} \) encodes both the location and the order of all nontrivial zeros of \( \zetaR(s) \). This confirms that \( L_{\sym} \) provides a canonical spectral model for the critical strip, uniquely determined by the determinant identity from \thmref{thm:det_identity_revised} and the bijection of \thmref{thm:spectral_zero_bijection_revised}:
\[
\det\nolimits_\zeta(I - \lambda L_{\sym}) = \frac{\XiR\left( \tfrac{1}{2} + i\lambda \right)}{\XiR\left( \tfrac{1}{2} \right)}.
\]
\end{corollary}
