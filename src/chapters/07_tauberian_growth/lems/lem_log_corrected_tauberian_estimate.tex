\begin{lemma}[Log-Corrected Tauberian Estimate for Spectral Growth]
\label{lem:log_corrected_tauberian_estimate}

Let \( L_{\mathrm{sym}} \in \mathcal{C}_1(H_{\Psi_\alpha}) \) be the canonical compact, self-adjoint, trace-class operator whose spectral determinant satisfies
\[
\det\nolimits_\zeta(I - \lambda L_{\mathrm{sym}}) = \frac{\Xi\left( \tfrac{1}{2} + i\lambda \right)}{\Xi\left( \tfrac{1}{2} \right)}.
\]
Let \( \Theta(t) := \operatorname{Tr}(e^{-t L_{\mathrm{sym}}^2}) \) denote its spectral heat trace, and define the squared spectral counting function
\[
N(\Lambda) := \#\left\{ \mu \in \operatorname{Spec}(L_{\mathrm{sym}}) : \mu^2 \le \Lambda \right\}.
\]

Then:

\begin{enumerate}
  \item[\textup{(i)}] The heat trace satisfies the short-time asymptotic:
  \[
  \Theta(t) \sim \frac{1}{\sqrt{4\pi t}} \log\left(\frac{1}{t}\right), \qquad \text{as } t \to 0^+.
  \]

  \item[\textup{(ii)}] The function \( \Theta \) lies in the logarithmically modified regular variation class:
  \[
  \Theta(t) \in \mathcal{R}_{1/2}^{\log}(0^+),
  \]
  where \( \mathcal{R}_{\alpha}^{\log} \) denotes the class of functions regularly varying with index \( \alpha \), modulated by a slowly varying logarithmic factor. See Appendix~\ref{app:notation_summary}.

  \item[\textup{(iii)}] The eigenvalue counting function satisfies the asymptotic expansion:
  \[
  N(\Lambda) = \frac{\sqrt{\Lambda}}{\pi} \log \Lambda + O(\sqrt{\Lambda}),
  \qquad \text{as } \Lambda \to \infty,
  \]
  i.e., \( N \in \mathcal{R}_{1/2}^{\log}(\infty) \).

  \item[\textup{(iv)}] These results follow from Korevaar’s log-corrected Tauberian theorem~\cite[Ch.~III, §5]{Korevaar2004Tauberian}, under analytic input from Proposition~\ref{prop:short_time_heat_expansion} and regularity bounds verified in \lemref{lem:korevaar_tauberian_constants_verified}.
\end{enumerate}

\medskip

\noindent
\emph{Uniqueness:} The log-modulated spectral asymptotic in (iii) is uniquely determined by the trace singularity in (i). Korevaar's theorem guarantees that no alternative growth profile is compatible with the stated short-time behavior of \( \Theta(t) \).
\end{lemma}
