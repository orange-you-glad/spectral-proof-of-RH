\begin{lemma}[Tauberian Inversion for Logarithmic Heat Trace]
\label{lem:tauberian_heat_trace_application}
Let \( L_{\sym} \in \TC(\HPsi) \) be a canonical compact, self-adjoint operator with squared spectrum \( \{ \mu_n^2 \} \subset (0, \infty) \). Define the eigenvalue counting function
\[
N(\lambda) := \#\{ n \in \N : \mu_n^2 \le \lambda \},
\]
and the spectral heat trace
\[
\Theta(t) := \Tr(e^{-t L_{\sym}^2}) = \sum_{n=1}^\infty e^{-t \mu_n^2}.
\]

\medskip
\noindent
Suppose \( \Theta(t) \sim \frac{1}{\sqrt{4\pi t}} \log(1/t) \) as \( t \to 0^+ \). Then the counting function satisfies
\[
N(\lambda) \sim C \, \lambda^{1/2} \log \lambda \quad \text{as } \lambda \to \infty,
\]
for some constant \( C > 0 \). That is, the singular trace behavior implies a sub-Weyl counting law with logarithmic enhancement.

\medskip
\noindent
This implication follows by Laplace–Mellin inversion and classical Tauberian theory for Laplace transforms of regularly varying functions (see Korevaar~\cite[Thm.~4.12.9]{Korevaar2004Tauberian}, Bingham–Goldie–Teugels~\cite[Thm.~1.7.1]{BGT1989RegularVariation}).
\end{lemma}
