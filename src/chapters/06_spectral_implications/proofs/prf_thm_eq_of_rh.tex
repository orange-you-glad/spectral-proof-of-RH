\begin{proof}[Proof of Theorem~\ref{thm:eq_of_rh}]
Let \( \rho \mapsto \mu_\rho := \dfrac{1}{i(\rho - \tfrac{1}{2})} \) denote the canonical bijection between the nontrivial zeros \( \rho \) of the Riemann zeta function \( \zeta(s) \) and the nonzero eigenvalues of the canonical compact, self-adjoint trace-class operator \( L_{\mathrm{sym}} \in \mathcal{S}_1(H_{\Psi_\alpha}) \), as established in Theorem~\ref{thm:spectral_zero_bijection_revised}. This correspondence is multiplicity-preserving by Lemma~\ref{lem:multiplicity_preservation}, and the determinant identity
\[
\det\nolimits_\zeta(I - \lambda L_{\mathrm{sym}}) = \frac{\Xi\left(\tfrac{1}{2} + i\lambda\right)}{\Xi\left(\tfrac{1}{2}\right)}
\]
encodes the zero structure of \( \Xi(s) \) exactly, up to canonical normalization.

The key identity in the proof is that:
\[
\mu_\rho \in \mathbb{R}
\quad \Longleftrightarrow \quad
\Re(\rho) = \tfrac{1}{2}.
\]
This follows by direct computation:
\[
\mu_\rho = \frac{1}{i(\rho - \tfrac{1}{2})} \in \mathbb{R}
\quad \Longleftrightarrow \quad
\frac{1}{i(\sigma + i\gamma - \tfrac{1}{2})} \in \mathbb{R}
\quad \Longleftrightarrow \quad
\sigma = \tfrac{1}{2}.
\]

\paragraph{\( \Rightarrow \)}  
Assume \( \Spec(L_{\mathrm{sym}}) \subset \mathbb{R} \). Then all eigenvalues \( \mu_\rho \in \mathbb{R} \). By the canonical correspondence, this implies \( \Re(\rho) = \tfrac{1}{2} \) for all nontrivial \( \rho \), hence RH holds.

\paragraph{\( \Leftarrow \)}  
Conversely, assume the Riemann Hypothesis is true. Then each nontrivial zero \( \rho \) satisfies \( \Re(\rho) = \tfrac{1}{2} \), so each eigenvalue \( \mu_\rho \in \mathbb{R} \), and hence \( \Spec(L_{\mathrm{sym}}) \subset \mathbb{R} \).

\paragraph{Spectral and Functional Conclusion.}
Because \( L_{\mathrm{sym}} \) is self-adjoint (Theorem~\ref{thm:sa_trace_class_Lsym}) and trace-class, the spectral theorem ensures that its spectrum is real-valued if and only if all canonical \( \mu_\rho \in \mathbb{R} \). Therefore, the Riemann Hypothesis is logically equivalent to the spectral realness of \( L_{\mathrm{sym}} \), as claimed.
\end{proof}
%  