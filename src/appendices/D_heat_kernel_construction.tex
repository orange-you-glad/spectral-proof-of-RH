\section{Heat Kernel Construction and Spectral Asymptotics}
\label{app:heat_kernel_construction}

\noindent\textbf{[Analytic Infrastructure Appendix]}  
This appendix provides detailed derivations and analytic justifications for the heat kernel estimates used in Chapters~\ref{sec:heat_kernel_asymptotics} and~\ref{sec:determinant_identity}. While many of these results are used as inputs to determinant regularization in Chapter~3, they are rigorously proved here, and their forward use is formally acyclic (see DAG in Appendix~\ref{app:dependency_graph}).

\subsection*{Notation}

Let \( \Lsym \in \TC(\HPsi) \) denote the canonical compact, self-adjoint operator with discrete real spectrum \( \{ \mu_n \} \subset \R \), listed with multiplicities. The heat semigroup is defined via spectral calculus:
\[
e^{-t \Lsym^2} := \sum_{n=1}^\infty e^{-t \mu_n^2} P_n,
\]
where \( P_n \) is the projection onto the eigenspace for \( \mu_n \). Since \( \Lsym^2 \ge 0 \) and compact, we have \( e^{-t \Lsym^2} \in \TC(\HPsi) \) for all \( t > 0 \), and the trace is:
\[
\Tr(e^{-t \Lsym^2}) = \sum_{n=1}^\infty e^{-t \mu_n^2} \cdot \operatorname{mult}(\mu_n).
\]

The integral kernel \( K_t(x,y) \) of this operator satisfies off-diagonal exponential decay and admits a singular short-time expansion. These properties form the analytic foundation for the trace identities of \secref{sec:heat_kernel_asymptotics}, and validate the determinant construction in \secref{sec:determinant_identity}.

\subsection*{Scope of Results}

This appendix proves the following analytic statements:

\begin{itemize}
  \item Exponential decay: the heat kernel \( K_t(x,y) \in \Schwartz(\R^2) \);
  \item Positivity and trace-class continuity of the semigroup \( e^{-t \Lsym^2} \);
  \item Laplace–Mellin representation of the spectral zeta function and zeta determinant;
  \item Parametrix expansions yielding a short-time singularity:
  \[
  \Tr(e^{-t \Lsym^2}) \sim \frac{1}{\sqrt{4\pi t}} \log\left( \frac{1}{t} \right) + \cdots,
  \]
  as shown in \propref{prop:short_time_heat_expansion};

  \item Spectral counting law:
  \[
  N(\lambda) := \#\{ \mu_n^2 \le \lambda \}
  \sim C \sqrt{\lambda} \log \lambda, \quad \lambda \to \infty,
  \]
  as proven in \propref{prop:spectral_counting_weyl}.
\end{itemize}

These results guarantee the well-posedness of the zeta-regularized determinant and underpin the Tauberian analysis of \cref{sec:tauberian_growth}. They also support the spectral rigidity arguments developed in \cref{sec:spectral_rigidity}.

\begin{remark}[Closure of Forward Dependencies from Chapter~3]
The heat kernel asymptotics and Laplace integrability properties derived here are used in Chapter~\ref{sec:determinant_identity} to define and validate the Carleman \(\zeta\)-regularized determinant \( \detz(I - \lambda \Lsym) \). In particular, the short-time singularity and trace-class continuity proved here retroactively justify the growth estimates and log-derivative identity stated in \cref{lem:det_via_heat_trace} and \cref{lem:A_log_derivative}. This dependency is modular and does not introduce logical cycles (see DAG in Appendix~\ref{app:dependency_graph}).
\end{remark}

\subsection*{Conclusion}

This appendix secures the analytic infrastructure underlying the spectral model for the completed zeta function \( \XiR(s) \). It confirms that the semigroup \( e^{-t \Lsym^2} \) governs both the short-time trace asymptotics and the determinant growth behavior needed to recover the zero distribution of \( \zetaR(s) \). The logarithmic singularity in the trace—see \remref{rem:log_singularity_necessary}—implies the need for zeta regularization, and determines the order and type of the canonical determinant.

\medskip
\noindent
While this material is proved downstream, its forward invocation in Chapter~3 is structurally safe and formally tracked in Appendix~\ref{app:dependency_graph}.
