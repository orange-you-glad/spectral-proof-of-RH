\subsection*{Introduction}
\label{sec:det_intro}

This chapter establishes the canonical identity
\begin{equation}
\label{eq:zeta_det_identity}
\detz(I - \lambda \Lsym) 
= \frac{\Xi\left( \tfrac{1}{2} + i\lambda \right)}{\Xi\left( \tfrac{1}{2} \right)},
\end{equation}
realizing the completed Riemann zeta function \( \Xi(s) \) as the Carleman \(\zeta\)-regularized Fredholm determinant of the canonical convolution operator
\[
\Lsym \in \TC(\HPsi),
\]
constructed in \cref{sec:operator_construction} as the trace-norm limit of mollified operators \( L_t \) defined by convolution with smoothed inverse Fourier transforms of \( \Xi \).

\begin{previewbox}
\textbf{Analytic Preconditions (Summary).}
The results in this chapter rely on the following rigorously established inputs, proved across Chapters~1–2 and verified in Chapter~\ref{sec:heat_kernel_asymptotics} and Appendix~\ref{app:heat_kernel_construction}:

\begin{itemize}
  \item \( \phi(\lambda) := \Xi(\tfrac{1}{2} + i\lambda) \) is real, even, entire of exponential type \( \pi \) (\cref{lem:exact_type_pi}), hence \( \phi \in \PW{\pi} \).
  \item Its inverse Fourier transform \( k := \FT^{-1}[\phi] \) lies in \( L^1(\R, \Psi_\alpha^{-1}) \) for all \( \alpha > \pi \), enabling trace-class convolution on \( \HPsi \).
  \item The operator \( \Lsym \in \TC(\HPsi) \) is compact and self-adjoint, defined as the trace-norm limit \( L_t \to \Lsym \) (\cref{lem:trace_norm_convergence_Lt_to_Lsym}, \cref{lem:trace_norm_limit_unique}).
  \item The heat semigroup \( \{ e^{-t \Lsym^2} \}_{t > 0} \subset \TC(\HPsi) \) exists, is holomorphic in \( t \), and satisfies Laplace-integrable bounds (\cref{lem:heat_semigroup_wellposed}).
  \item Short-time singularity:
  \[
  \Tr(e^{-t \Lsym^2}) \sim \tfrac{\log(1/t)}{\sqrt{4\pi t}} + \cdots,
  \]
  governs the order-one Hadamard type of the determinant (\cref{lem:det_identity_entire_order_one}).
  \item The determinant is defined locally via power series (\cref{lem:det_via_heat_trace}) and globally by Laplace–Mellin continuation using the trace asymptotics above.
\end{itemize}

All analytic dependencies are modular and tracked in the DAG (Appendix~\ref{app:dependency_graph}). No assumptions regarding RH, spectral reality, or zero bijection are used in this chapter.
\end{previewbox}

\begin{remark}[Forward Use of Analytic Results from Chapter~5 and Appendix~D]
While the present chapter proves the determinant identity \eqref{eq:zeta_det_identity} without assuming the Riemann Hypothesis, it does rely on heat trace asymptotics and Laplace integrability properties rigorously derived in Chapter~\ref{sec:heat_kernel_asymptotics} and Appendix~\ref{app:heat_kernel_construction}. These forward references are explicitly tracked in the DAG (\cref{app:dependency_graph}), and no circular logic arises. The analytic inputs are modular and acyclic, but readers should note that their full derivation appears later in the manuscript.
\end{remark}

\medskip

\paragraph{Structure of the Proof.}
The argument proceeds through four main analytic modules, grounded in trace-class determinant theory~\cite[Ch.~4]{Simon2005TraceIdeals} and the Paley--Wiener theory of entire functions~\cite[Ch.~9]{Levin1996EntireLectures}:

\begin{itemize}
  \item \textbf{Heat trace construction and analytic continuation:} Define the Carleman determinant via Laplace--Mellin transform of the trace \( \Tr(e^{-t \Lsym^2}) \), analytically continued via Tauberian expansion (\cref{lem:heat_semigroup_wellposed}, \cref{lem:det_via_heat_trace}).

  \item \textbf{Short-time singularity and growth control:} As \( t \to 0^+ \), the expansion
  \[
  \Tr(e^{-t \Lsym^2}) = \frac{1}{\sqrt{4\pi t}} \log\left(\frac{1}{t}\right) + \mathcal{O}(t^{-1/2})
  \]
  controls the logarithmic singularity and determines the genus-one Hadamard type of the determinant (\cref{lem:det_identity_entire_order_one}).

  \item \textbf{Hadamard uniqueness and spectral identification:} The determinant \( \lambda \mapsto \detz(I - \lambda \Lsym) \in \mathcal{E}_1^\pi \), the space of entire functions of order one and type \( \pi \). Via Hadamard's factorization and uniqueness (\cref{lem:hadamard_uniqueness_E1pi}), it must match the normalized zeta profile, established via logarithmic derivative matching (\cref{lem:A_log_derivative}) and trace centering (\cref{lem:trace_zero}).

  \item \textbf{Canonical normalization:} The trace vanishing
  \[
  \Tr(\Lsym) = 0
  \]
  ensures the normalization
  \[
  \detz(I) = 1,
  \]
  eliminating ambiguity in the Hadamard exponential prefactor and uniquely anchoring the determinant.
\end{itemize}

\medskip

This analytic phase is fully self-contained, strictly modular, and free of spectral assumptions (e.g., no reliance on RH or spectral bijection). The spectral encoding
\[
\rho \mapsto \mu_\rho := \tfrac{1}{i}(\rho - \tfrac{1}{2}) \in \Spec(\Lsym)
\]
is developed independently in Chapter~\ref{sec:spectral_correspondence}, after the determinant structure is canonically established.
