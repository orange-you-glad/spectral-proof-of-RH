\section{Zeta Functions and Trace-Class Operators}
\label{app:zeta_trace_background}

\noindent
This appendix summarizes analytic properties of the completed Riemann zeta function and trace-class operator theory used throughout the spectral determinant construction. For detailed background, see~\cite{Titchmarsh1986Zeta, Simon2005TraceIdeals, Bornemann2010FredholmDeterminants}.

\subsection*{The Completed Zeta Function \( \Xi(s) \)}

The completed Riemann zeta function is defined by
\[
\Xi(s) := \tfrac{1}{2}s(s-1)\pi^{-s/2}\Gamma(s/2)\zeta(s),
\]
and satisfies the functional equation
\[
\Xi(s) = \Xi(1 - s).
\]
It is entire of order one and exponential type \( \pi \), real-valued on the real axis, and admits a Hadamard factorization:
\[
\Xi(s) = \Xi\left(\tfrac{1}{2}\right) \prod_{\rho} \left(1 - \frac{s - \tfrac{1}{2}}{\rho - \tfrac{1}{2}}\right) e^{(s - \tfrac{1}{2})/(\rho - \tfrac{1}{2})},
\]
where the product is over nontrivial zeros \( \rho \in \mathbb{C} \), symmetric about \( \Re(s) = \tfrac{1}{2} \).

Let
\[
\phi(\lambda) := \Xi\left( \tfrac{1}{2} + i\lambda \right).
\]
Then \( \phi \) is real, even, entire of exponential type \( \pi \), and belongs to the Paley–Wiener class \( \PW{\pi} \). Its inverse Fourier transform decays exponentially:
\[
\widehat{\phi}(x) := \int_{\R} \phi(\lambda) e^{2\pi i x \lambda} \, d\lambda
\in L^1(\R, e^{\alpha|x|}dx) \quad \text{for all } \alpha < \pi.
\]

\subsection*{Fredholm and Carleman Determinants}

Let \( T \in \TC(H) \) be trace class. The Fredholm determinant is defined by
\[
\det(I - \lambda T) := \prod_{n=1}^\infty (1 - \lambda \mu_n),
\]
where \( \mu_n \in \C \) are the eigenvalues of \( T \), counted with multiplicities. This determinant is entire in \( \lambda \), and satisfies
\[
\frac{d}{d\lambda} \log \det(I - \lambda T) = \Tr\left((I - \lambda T)^{-1} T\right).
\]

For unbounded nonnegative operators \( L \) with compact resolvent (e.g., \( \Lsym^2 \)), the zeta-regularized determinant is defined by
\[
\log \det\nolimits_\zeta(L) := - \left. \frac{d}{ds} \zeta_L(s) \right|_{s=0}, \quad
\zeta_L(s) := \sum_{\lambda > 0} \lambda^{-s}.
\]

If \( L = T^2 \) with \( T \in \TC(H) \), and
\[
\Tr(e^{-tL}) \sim \frac{\log(1/t)}{\sqrt{4\pi t}} + \cdots,
\]
then \( \log \det_\zeta(L) \) grows as a genus-one Hadamard product, matching the structure of \( \Xi \).

\paragraph{Comparison: Elliptic vs. Convolutional Regularization.}
For elliptic operators (e.g., Laplacians on compact manifolds), the classical zeta function arises from:
\[
\Tr(e^{-t\Delta}) \sim \sum_{n=0}^\infty a_n t^{(n-d)/2}.
\]
In contrast, \( \Lsym^2 \) is convolutional and admits a leading singularity:
\[
\Tr(e^{-t \Lsym^2}) \sim \frac{\log(1/t)}{\sqrt{4\pi t}} + \cdots.
\]
This divergence requires Laplace–Carleman regularization, subtracting a singular parametrix (see \remref{rem:spectral_dimension}). The result matches the growth and structure of the canonical determinant for \( \Xi(s) \).

\subsection*{Schatten Classes and Trace Ideals}

Let \( \mathcal{H} \) be a separable Hilbert space, and \( T \in \mathcal{B}(\mathcal{H}) \). The singular values \( s_n(T) \) are the eigenvalues of \( |T| := (T^*T)^{1/2} \). The Schatten class \( \mathcal{C}_p(\mathcal{H}) \) is defined by
\[
\|T\|_{\mathcal{C}_p} := \left( \sum_{n=1}^\infty s_n(T)^p \right)^{1/p} < \infty.
\]
We have inclusions
\[
\TC(\mathcal{H}) := \mathcal{C}_1(\mathcal{H}) \subset \mathcal{C}_2(\mathcal{H}) \subset \mathcal{K}(\mathcal{H}),
\]
and trace-class operators satisfy
\[
\Tr(T) = \sum_{n=1}^\infty \langle T e_n, e_n \rangle, \quad \text{for any orthonormal basis } \{ e_n \}.
\]

If \( T \in \TC(\HPsi) \) and \( T = T^* \), the trace functional \( \phi \mapsto \Tr(\phi(T)) \) extends to a tempered distribution on \( \R \), for all \( \phi \in \mathcal{S}(\R) \). This underlies the spectral positivity results in Chapter~\ref{sec:spectral_rigidity}.
