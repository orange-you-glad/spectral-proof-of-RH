\begin{remark}[Canonical Spectral Bijection via Determinant Identity]
\label{rem:spectral_zero_bijection_from_determinant}

By \thmref{thm:spectral_zero_bijection_revised}, the map
\[
\rho \mapsto \mu_\rho := \frac{1}{i(\rho - \tfrac{1}{2})}
\]
defines a canonical, multiplicity-preserving bijection between the nontrivial zeros of \( \zetaR(s) \) and the nonzero spectrum of \( L_{\sym} \).

This inverse spectral map is uniquely determined by the vanishing structure of the determinant:
\[
\det\nolimits_\zeta(I - \lambda L_{\sym}) = \frac{\XiR\left(\tfrac{1}{2} + i\lambda\right)}{\XiR\left(\tfrac{1}{2}\right)},
\]
whose zeros occur at \( \lambda = i(\rho - \tfrac{1}{2}) \), with multiplicities preserved under Hadamard factorization. The bijection follows from entire function theory applied to the determinant and the trace-class spectral calculus of \( L_{\sym} \).
\end{remark}
