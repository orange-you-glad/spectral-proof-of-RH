\section*{Prologue: Bourbaki.RH and the Future of Proof}
\label{sec:prologue}

\paragraph*{1.1 The Autonomous Mathematician.}
Bourbaki.RH is not a project about the Riemann Hypothesis. It is a prototype for what mathematics becomes when its proofs are no longer written, but built. Autonomous. Compositional. Self-verifying. At its heart is not an answer, but an architecture—a formal system that doesn’t just construct proofs, but refactors them, audits them, and compiles them into semantic artifacts. This is not LLM-as-calculator. It is LLM as collaborator, as interlocutor, as dialectical engine in a formal loop:
\[
\text{conjecture} \rightarrow \text{attempt} \rightarrow \text{audit} \rightarrow \text{refactor} \rightarrow \text{validate} \rightarrow \text{render}.
\]
The output is not a paper. It is a reproducible build. Every DAG edge in this manuscript is re-proven on every commit. Every identity is tracked to its analytic kernel. Every expansion is recursively justified. This is not a new proof of RH. It is a new epistemology of proof.

\paragraph*{1.2 Why RH Became the Test Case.}
The Riemann Hypothesis was not chosen for celebrity, but for constraint. RH does not tolerate intuition loosely held. It rewards only structure. It enforces a discipline of equivalence. In a mathematical ecosystem designed to be modular, local, and formally audited, RH serves not as a summit but as a stress test. Its sharp edge—real spectrum versus critical zeros—is the kind of binary condition that a formal pipeline can test, stabilize, and expose without infinite guesswork. RH demanded a proving system mature enough to confront necessity. And Bourbaki.RH responded.

\paragraph*{1.3 Workflow in Miniature.}
At the core is a Directed Acyclic Graph (DAG) of irreducible semantic units—lemmas that are not merely steps in a proof, but composable modules in a formal structure. Each is typed, audited, and connected by analytic invariants: spectral bijections, determinant identities, kernel decay, Tauberian asymptotics. The DAG compiles into a Lean 4 formalization (in progress), a typeset manuscript, and a containerized CI pipeline that re-verifies the logical closure on every push. Nothing is assumed. Nothing is trusted. Everything is reconstructed. From the Fourier kernel up.

\paragraph*{1.4 Mathematical Result, Briefly.}
The central analytic construct is a canonical trace-class operator
\[
\Lsym \in \TC(\HPsi), \quad \HPsi := L^2(\R, e^{\alpha|x|} dx), \quad \alpha > \pi,
\]
whose zeta-regularized Fredholm determinant satisfies the identity:
\[
\detz(I - \lambda \Lsym) = \frac{\Xi\left(\tfrac{1}{2} + i\lambda\right)}{\Xi\left(\tfrac{1}{2}\right)}.
\]
Through this identity, the nontrivial zeros \( \rho \in \C \) of \( \zeta(s) \) are encoded as spectral data:
\[
\rho \mapsto \mu_\rho := \tfrac{1}{i}(\rho - \tfrac{1}{2}) \in \Spec(\Lsym).
\]
By the spectral theorem, this yields:
\[
\RH \iff \Spec(\Lsym) \subset \R.
\]
But again: this is not about novelty. It is about closure. Formal, auditable, repeatable closure—where arithmetic truth emerges as a spectral invariant.

\paragraph*{1.5 Invitation to Audit and Extend.}
Every lemma is modular. Every operator is reproducible. The full DAG structure, Lean verification suite (in progress), and CI runner are open source:
\begin{itemize}
  \item GitHub: \texttt{https://github.com/orange-you-glad/spectral-proof-of-RH}
  \item Formal interlocutor: \texttt{https://chat.openai.com/g/g-WkZZP7ywr-bourbaki-rh}
\end{itemize}
The real experiment is not whether RH is true, but whether a formal ecosystem can grow. Can you extend it? Introduce new operators? Lift to automorphic \( L \)-functions? Generalize the determinant theory? Bourbaki.RH wasn’t designed to resolve the Riemann Hypothesis. It was designed to build the proving systems of the next century.

\subsection*{Main Equivalence — Spectral RH Reformulation}
\hfill \break

\begin{tcolorbox}[colback=gray!3!white, colframe=black!75!white]
We construct a canonical operator \( \Lsym \in \TC(\HPsi) \) on
\[
\HPsi := L^2\big(\mathbb{R}, e^{\alpha |x|} \, dx\big), \qquad \alpha > \pi,
\]
such that:
\begin{enumerate}
  \item The zeta-regularized determinant identity holds:
  \[
  \detz(I - \lambda \Lsym) = \frac{\Xi\left(\tfrac{1}{2} + i\lambda\right)}{\Xi\left(\tfrac{1}{2}\right)}.
  \]
  \item The spectrum of \( \Lsym \) satisfies the bijective correspondence:
  \[
  \rho \mapsto \mu_\rho := \tfrac{1}{i}(\rho - \tfrac{1}{2}) \in \Spec(\Lsym),
  \]
  with multiplicities preserved.
  \item The Riemann Hypothesis holds if and only if
  \[
  \Spec(\Lsym) \subset \mathbb{R}.
  \]
\end{enumerate}
All analytic and operator-theoretic infrastructure—Paley–Wiener kernel decay, trace-norm convergence, heat semigroup regularity, determinant growth—is rigorously modularized and verified across Chapters~1–10. See \thmref{thm:truth_of_rh} for formal equivalence.
\end{tcolorbox}

\subsection*{Disclosure of Large-Language-Model Assistance}

The author used OpenAI’s \emph{ChatGPT} (GPT-4o, May 2025 release) as a research and writing assistant. This collaboration included:

\begin{itemize}
  \item Structural scaffolding for chapters and DAG architecture;
  \item Drafting provisional \LaTeX{} for definitions, lemmas, and operator identities;
  \item Symbolic verification of transforms and expansions;
  \item Refactoring of prose for clarity, precision, and formal alignment.
\end{itemize}

All mathematics was independently verified by the author. The LLM is not an author and bears no responsibility for any claim made herein.
