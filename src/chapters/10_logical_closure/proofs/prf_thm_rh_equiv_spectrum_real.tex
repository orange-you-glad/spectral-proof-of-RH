\begin{proof}[Proof of \thmref{thm:rh_spectrum_equiv}]
Let \( \rho = \tfrac{1}{2} + i\gamma \in \C \) be a nontrivial zero of the Riemann zeta function. Define the canonical spectral image:
\[
\mu_\rho := \frac{1}{i}(\rho - \tfrac{1}{2}) = \gamma.
\]
By the determinant identity (see \thmref{thm:det_identity_revised}) and the bijection established in \thmref{thm:spectral_zero_bijection_revised}, each such \( \rho \) corresponds to a nonzero spectral value \( \mu_\rho \in \Spec(\Lsym) \), with multiplicities preserved.

\paragraph{(i) \( \Rightarrow \) (ii)}
Assume the Riemann Hypothesis holds. Then every nontrivial zero satisfies \( \rho = \tfrac{1}{2} + i\gamma \) with \( \gamma \in \R \). Therefore,
\[
\mu_\rho = \frac{1}{i}(\rho - \tfrac{1}{2}) = \gamma \in \R.
\]
Thus, all nonzero eigenvalues of \( \Lsym \) lie on the real line, so
\[
\Spec(\Lsym) \subset \R.
\]

\paragraph{(ii) \( \Rightarrow \) (i)}
Conversely, suppose \( \Spec(\Lsym) \subset \R \). Then for each nontrivial zero \( \rho \), the associated \( \mu_\rho \in \Spec(\Lsym) \subset \R \). But then:
\[
\mu_\rho = \frac{1}{i}(\rho - \tfrac{1}{2}) \in \R \quad \Rightarrow \quad \rho - \tfrac{1}{2} \in i\R \quad \Rightarrow \quad \Re(\rho) = \tfrac{1}{2}.
\]
Hence, all nontrivial zeros lie on the critical line.

\paragraph{Conclusion.}
The canonical spectral map \( \rho \mapsto \mu_\rho \) matches the zero set of \( \zeta(s) \) with the nonzero spectrum of \( \Lsym \). Therefore,
\[
\RH \iff \Spec(\Lsym) \subset \R,
\]
as claimed. This establishes a logically closed, operator-theoretic equivalence.
\end{proof}
