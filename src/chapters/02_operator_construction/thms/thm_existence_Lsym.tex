\begin{theorem}[Existence of the Canonical Operator \( \Lsym \)]
\label{thm:existence_Lsym}
Let
\[
\varphi_t(\lambda) := e^{-t\lambda^2} \, \Xi\left( \tfrac{1}{2} + i\lambda \right)
\]
be the mollified Fourier profiles, and let \( L_t \) denote the corresponding convolution operators acting on
\[
H_{\Psi_\alpha} := L^2(\R, \Psi_\alpha(x)\, dx),
\quad \text{where } \Psi_\alpha(x) := e^{\alpha |x|}, \quad \alpha > \pi.
\]

Then:
\begin{enumerate}
  \item[\textnormal{(i)}] For each \( t > 0 \), the operator \( L_t \in \TC(H_{\Psi_\alpha}) \) is compact and trace class (see \lemref{lem:trace_class_Lt}).

  \item[\textnormal{(ii)}] The trace-norm limit
  \[
  \Lsym := \lim_{t \to 0^+} L_t \quad \text{in } \TC(H_{\Psi_\alpha})
  \]
  exists and defines a compact trace-class operator:
  \[
  \Lsym \in \TC(H_{\Psi_\alpha}) \cap \KC(H_{\Psi_\alpha}),
  \]
  as established in \lemref{lem:trace_norm_convergence_Lt_to_Lsym}.

  \item[\textnormal{(iii)}] The operator \( \Lsym \) is self-adjoint on \( H_{\Psi_\alpha} \), with domain given by the closure of \( \Schwartz \subset H_{\Psi_\alpha} \) (see \lemref{lem:core_essential_sa}).

  \item[\textnormal{(iv)}] The trace of \( \Lsym \) vanishes:
  \[
  \Tr(\Lsym) = 0,
  \]
  ensuring canonical normalization of the determinant (see \thmref{thm:trace_zero_Lsym}).
\end{enumerate}

\medskip
\noindent
This theorem establishes the existence of a canonical compact operator associated with the analytic structure of the completed Riemann zeta function \( \Xi(s) \), realized as the trace-norm limit of mollified spectral convolution operators. The operator \( \Lsym \) provides the analytic foundation for the zeta-regularized determinant identity and the spectral encoding of the nontrivial zeros of \( \zeta(s) \), developed in Chapters~\ref{sec:determinant_identity} and~\ref{sec:spectral_correspondence}.
\end{theorem}
