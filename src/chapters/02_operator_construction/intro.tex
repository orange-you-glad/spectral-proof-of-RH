\subsection*{Introduction}

The goal of this chapter is to construct the canonical compact, self-adjoint, trace-class operator
\[
L_{\sym} \in \TC(H_{\Psi_\alpha}),
\]
on the exponentially weighted Hilbert space \( H_{\Psi_\alpha} \), designed to mirror the spectral structure of the completed Riemann zeta function \( \Xi(s) \).

We proceed through five analytically rigorous stages:

\begin{itemize}
    \item \textbf{Weighted space construction:} We define the Hilbert space
    \[
    H_{\Psi_\alpha} := L^2(\R, \Psi_\alpha(x)\, dx), \quad \Psi_\alpha(x) := e^{\alpha |x|}, \quad \alpha > \pi,
    \]
    chosen so that the inverse Fourier transform of the canonical spectral profile
    \[
    \phi(\lambda) := \Xi\left(\tfrac{1}{2} + i\lambda\right)
    \]
    lies in \( L^1(\R, \Psi_\alpha^{-1}\, dx) \), enabling convolution operators to be trace-class on \( H_{\Psi_\alpha} \). This threshold \( \alpha > \pi \) is shown to be sharp in \cref{prop:trace_class_sharpness}.

    \item \textbf{Mollifier family construction:} We define mollified spectral profiles
    \[
    \varphi_t(\lambda) := e^{-t\lambda^2} \phi(\lambda),
    \]
    which lie in \( \Schwartz \) and generate convolution operators \( L_t \in \TC(H_{\Psi_\alpha}) \) with symmetric, compact kernels. These satisfy
    \[
    \| L_t \|_{\TC} \le C(\alpha), \quad \forall t \in (0,1],
    \]
    where \( C(\alpha) \) depends on uniform kernel bounds (see \cref{lem:trace_class_Lt}).

    \item \textbf{Trace-norm convergence and canonical limit:} We prove that
    \[
    \| L_t - L_{\sym} \|_{\TC} \to 0 \quad \text{as } t \to 0^+,
    \]
    and define
    \[
    L_{\sym} := \lim_{t \to 0^+} L_t,
    \]
    which exists uniquely in trace norm (\cref{lem:trace_norm_limit_unique}) and is independent of mollifier choice. The operator \( L_{\sym} \) inherits compactness and self-adjointness via kernel symmetry and convergence (\cref{thm:existence_Lsym}, \cref{thm:sa_trace_class_Lsym}).

    \item \textbf{Core domain and spectral closure:} The Schwartz space \( \Schwartz(\R) \subset H_{\Psi_\alpha} \) is a core for \( L_{\sym} \), and
    \[
    L_{\sym} \text{ and } L_{\sym}^2 \text{ are both essentially self-adjoint on } \Schwartz,
    \]
    by Nelson’s analytic vector theorem and the mollifier bounds (\cref{rem:selfadjoint_analytic_vectors}).

    \item \textbf{Trace vanishing and normalization:} We prove
    \[
    \Tr(L_{\sym}) = 0,
    \]
    eliminating ambiguity in the exponential prefactor of the determinant and ensuring canonical normalization (\cref{thm:trace_zero_Lsym}).
\end{itemize}

These results culminate in \thmref{thm:existence_Lsym}, \thmref{thm:sa_trace_class_Lsym}, and \thmref{thm:trace_zero_Lsym}, which collectively certify the existence, uniqueness, and analytic properties of \( L_{\sym} \) as the canonical convolution operator associated with \( \Xi(s) \).

\medskip

This operator underlies the Carleman–zeta-regularized determinant identity
\[
\det\nolimits_{\zeta}(I - \lambda^2 L_{\sym}^2) = \Xi\left(\tfrac{1}{2} + i\lambda\right),
\]
which will be rigorously derived in \secref{sec:determinant_identity}, building on the analytic semigroup \( e^{-t L_{\sym}^2} \) constructed here (\cref{lem:heat_semigroup_existence}).

\medskip

All convergence and boundedness claims are grounded in explicit kernel estimates and Paley--Wiener decay. The operator \( L_{\sym} \) satisfies all required properties for spectral encoding of the nontrivial zeros of \( \zeta \), and does so within the classical framework of Hilbert space spectral theory.

\paragraph{Contrast with Prior Spectral Proposals.}
Unlike speculative constructions—such as the Hilbert–Pólya heuristic, Connes’s noncommutative geometry~\cite{Connes1999TraceFormula}, or Deninger’s dynamical program~\cite{Deninger1998Frobenius}—the operator \( L_{\sym} \) is defined concretely in \( \TC(\HPsi) \), with trace-class compactness, self-adjointness, exponential kernel decay, and a zeta-compatible determinant. This fills a critical analytic gap in prior proposals.
