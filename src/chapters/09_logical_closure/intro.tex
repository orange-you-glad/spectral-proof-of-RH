\subsection*{Closure of the Spectral Program}

This chapter concludes the analytic reformulation of the Riemann Hypothesis as a statement of spectral rigidity. We synthesize the spectral, determinantal, and trace-theoretic constructions developed in preceding chapters into a formally closed and logically acyclic equivalence.

\medskip
\noindent
The logical framework proceeds through the following modular components:

\begin{itemize}
  \item \textbf{Canonical Operator Construction.}  
  A compact, self-adjoint trace-class operator
  \[
  L_{\mathrm{sym}} \in \mathcal{C}_1(H_{\Psi_\alpha})
  \]
  is constructed as the trace-norm limit of mollified symmetric convolution operators derived from the inverse Fourier transform of the completed zeta function \( \Xi(s) \). Mollifier decay guarantees convergence in Schatten norm.

  \item \textbf{Determinant Identity.}  
  The zeta-regularized Fredholm determinant of \( L_{\mathrm{sym}} \) satisfies:
  \[
  \det\nolimits_\zeta(I - \lambda L_{\mathrm{sym}}) = \frac{\Xi\left(\tfrac{1}{2} + i\lambda\right)}{\Xi\left(\tfrac{1}{2}\right)},
  \]
  valid for all \( \lambda \in \mathbb{C} \). This identity transfers the zero structure of \( \Xi(s) \) to the spectral data of \( L_{\mathrm{sym}} \).

  \item \textbf{Spectral Multiplicity Matching.}  
  The Hadamard factorization of \( \Xi(s) \) ensures that the order of each zero matches the algebraic multiplicity of the corresponding eigenvalue:
  \[
  \operatorname{ord}_{\rho} \zeta = \operatorname{mult}_{\mu_\rho} L_{\mathrm{sym}},
  \]
  where \( \mu_\rho := \frac{1}{i}(\rho - \tfrac{1}{2}) \). Thus, the determinant encodes both location and multiplicity of the zeros.

  \item \textbf{Spectral Realization.}  
  As established in Lemma~\ref{lem:spectral-zero-encoding}, every nontrivial zero \( \rho \) of \( \zeta(s) \) maps to a nonzero spectral point:
  \[
  \mu_\rho := \frac{1}{i(\rho - \tfrac{1}{2})} \in \operatorname{Spec}(L_{\mathrm{sym}}).
  \]

  \item \textbf{Spectral Symmetry.}  
  By Lemma~\ref{lem:spectral-symmetry-lsym}, the spectrum of \( L_{\mathrm{sym}} \) satisfies \( \mu \in \operatorname{Spec}(L_{\mathrm{sym}}) \Rightarrow -\mu \in \operatorname{Spec}(L_{\mathrm{sym}}) \), reflecting the evenness of \( \Xi(\tfrac{1}{2} + i\lambda) \).

  \item \textbf{Spectral Rigidity and Logical Equivalence.}  
  The condition
  \[
  \operatorname{Spec}(L_{\mathrm{sym}}) \subset \mathbb{R}
  \]
  is logically equivalent to the Riemann Hypothesis:
  \[
  \mathrm{RH} \iff \Re(\rho) = \tfrac{1}{2} \; \forall \rho \in \operatorname{Spec}(\zeta)
       \iff \mu_\rho \in \mathbb{R} \; \forall \mu_\rho \in \operatorname{Spec}(L_{\mathrm{sym}}).
  \]
  This equivalence holds regardless of spectral multiplicity.

  \item \textbf{Spectral Completeness.}  
  Corollary~\ref{cor:spectral-determines-zeta} confirms that the spectrum of \( L_{\mathrm{sym}} \) completely determines the nontrivial zero set of \( \zeta(s) \), giving a canonical operator model of the critical line.

  \item \textbf{Trace Positivity and Functional Calculus.}  
  By Lemma~\ref{lem:trace_distribution_positivity}, the map
  \[
  \phi \mapsto \operatorname{Tr}(\phi(L_{\mathrm{sym}}))
  \]
  defines a positive tempered distribution for all \( \phi \in \mathcal{S}(\mathbb{R}) \) with \( \phi \ge 0 \). This validates the zeta-regularized determinant as a spectral observable and confirms the positivity of the induced spectral measure.

  \item \textbf{Analytic Closure.}  
  The RH equivalence is deduced entirely from:
  \begin{itemize}
    \item the spectral theorem for compact self-adjoint operators;
    \item short-time heat trace asymptotics and Gaussian kernel decay;
    \item the Hadamard factorization of \( \Xi(s) \), normalized at \( s = \tfrac{1}{2} \);
    \item positivity and convergence of the trace distribution.
  \end{itemize}
  No appeal is made to modular, adelic, or automorphic techniques.
\end{itemize}

\medskip
\noindent
The full dependency graph supporting this conclusion is depicted in Figure~\ref{fig:dag_appendix_b}, and formally documented in Appendix~\ref{app:dependency-graph}. The equivalence is established in the main theorem that follows.
