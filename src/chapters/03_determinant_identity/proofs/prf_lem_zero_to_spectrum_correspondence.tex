\begin{proof}[Proof of Lemma~\ref{lem:zero_to_spectrum_correspondence}]
Let \( f(\lambda) := \det\nolimits_\zeta(I - \lambda L_{\mathrm{sym}}) \). By Theorem~\ref{thm:det_identity_revised}, we have
\[
f(\lambda) = \frac{\Xi\left( \tfrac{1}{2} + i\lambda \right)}{\Xi\left( \tfrac{1}{2} \right)}.
\]

Let \( \rho \in \mathbb{C} \setminus \{\tfrac{1}{2}\} \) be a nontrivial zero of \( \zeta(s) \), so that \( \Xi(\rho) = 0 \). Then \( \lambda_0 := i(\rho - \tfrac{1}{2}) \in \mathbb{C} \) is a zero of \( \Xi\left( \tfrac{1}{2} + i\lambda \right) \), hence of \( f(\lambda) \).

But since
\[
f(\lambda) = \prod_n \left( 1 - \frac{\lambda}{\mu_n} \right),
\]
with \( \{ \mu_n \} \subset \mathbb{R} \) the nonzero eigenvalues of \( L_{\mathrm{sym}} \), the zeros of \( f \) occur precisely at \( \lambda = \mu_n \), counted with algebraic multiplicity.

Thus, if \( f(\lambda_0) = 0 \), then \( \lambda_0 \in \operatorname{Spec}(L_{\mathrm{sym}})^{-1} \), and
\[
\mu := \lambda_0^{-1} = \frac{1}{i(\rho - \tfrac{1}{2})}
\]
is an eigenvalue of \( L_{\mathrm{sym}} \), with multiplicity equal to the order of the zero of \( \Xi(s) \) at \( \rho \), inherited via the Fredholm determinant expansion.

\paragraph{Symmetry.}
Since \( \Xi(s) = \Xi(1 - s) \), the zeros \( \rho \) occur in pairs \( \rho \), \( 1 - \rho \), which induce
\[
\mu_\rho = \frac{1}{i(\rho - \tfrac{1}{2})}, \qquad \mu_{1 - \rho} = \frac{1}{i(1 - \rho - \tfrac{1}{2})} = -\mu_\rho.
\]
So the spectrum is symmetric under \( \mu \mapsto -\mu \), as expected for a self-adjoint convolution operator arising from an even kernel.

\paragraph{Conclusion.}
Each nontrivial zero \( \rho \ne \tfrac{1}{2} \) gives rise to a real nonzero eigenvalue \( \mu_\rho \in \mathbb{R} \setminus \{0\} \) of \( L_{\mathrm{sym}} \), counted with multiplicity. This establishes the claimed correspondence.
\end{proof}
% 