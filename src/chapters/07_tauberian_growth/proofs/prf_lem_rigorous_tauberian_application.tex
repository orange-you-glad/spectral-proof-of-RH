\begin{proof}[Proof of Lemma~\ref{lem:log-corrected-tauberian-application}]
Let \( A(\Lambda) := \#\left\{ \mu_n^2 \le \Lambda \right\} \) denote the squared spectral counting function associated with the compact, self-adjoint operator \( L_{\mathrm{sym}} \in \mathcal{C}_1(H_{\Psi_\alpha}) \). Define the spectral heat trace
\[
\Theta(t) := \operatorname{Tr}(e^{-t L_{\mathrm{sym}}^2}) = \int_0^\infty e^{-t \lambda} \, dA(\lambda).
\]

\paragraph{Step 1: Refined Short-Time Asymptotics.}
By hypothesis (from Proposition~\ref{prop:short_time_heat_expansion}), we have the singular expansion:
\[
\Theta(t) = \frac{1}{\sqrt{4\pi t}} \log\left( \frac{1}{t} \right) + o(t^{-1/2}) \quad \text{as } t \to 0^+,
\]
together with an upper envelope:
\[
\left| \Theta(t) - \frac{1}{\sqrt{4\pi t}} \log\left( \frac{1}{t} \right) \right| \leq \varepsilon \, t^{-1/2},
\]
for all sufficiently small \( t > 0 \). Hence, \( \Theta(t) \in \mathcal{R}_{1/2}^{\log}(0) \), the class of regularly varying functions of index \( \rho = 1/2 \), modulated by \( \log(1/t) \).

\paragraph{Step 2: Korevaar's Log-Corrected Tauberian Theorem.}
We apply Korevaar’s log-modulated Laplace–Stieltjes Tauberian theorem~\cite[Ch.~III, §5, Thm.~5.5]{Korevaar2004Tauberian}, which states:

\medskip
\emph{If \( \Theta(t) \sim t^{-\rho} L(1/t) \) as \( t \to 0^+ \), with \( \rho > 0 \) and \( L \) slowly varying, then}
\[
A(\Lambda) \sim \frac{L(\Lambda)}{\Gamma(\rho+1)} \Lambda^{\rho}, \qquad \text{as } \Lambda \to \infty.
\]

\medskip
In our case, \( \rho = 1/2 \) and \( L(x) = \frac{1}{\sqrt{4\pi}} \log x \). Therefore,
\[
A(\Lambda) \sim \frac{1}{\Gamma(3/2)} \cdot \frac{1}{\sqrt{4\pi}} \log \Lambda \cdot \Lambda^{1/2}.
\]

\paragraph{Step 3: Constant Evaluation.}
We substitute known values:
\[
\Gamma\left( \tfrac{3}{2} \right) = \frac{\sqrt{\pi}}{2}, \quad \Rightarrow \quad \frac{1}{\Gamma(3/2)} = \frac{2}{\sqrt{\pi}},
\]
so
\[
A(\Lambda) \sim \frac{2}{\sqrt{\pi}} \cdot \frac{1}{\sqrt{4\pi}} \log \Lambda \cdot \Lambda^{1/2}
= \frac{1}{2\pi} \Lambda^{1/2} \log \Lambda.
\]

\paragraph{Conclusion.}
We conclude that the spectral counting function satisfies the refined asymptotic law:
\[
A(\Lambda) \sim \frac{1}{2\pi} \Lambda^{1/2} \log \Lambda, \qquad \text{as } \Lambda \to \infty,
\]
which completes the proof.
\end{proof}
