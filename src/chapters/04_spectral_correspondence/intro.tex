\subsection*{Introduction}

This chapter establishes the spectral bijection between the nontrivial zeros of the Riemann zeta function and the nonzero spectrum of the canonical trace-class operator \( L_{\mathrm{sym}} \). Each nontrivial zero
\[
\rho = \tfrac{1}{2} + i\gamma
\]
is mapped to an eigenvalue
\[
\mu_\rho := \frac{1}{i(\rho - \tfrac{1}{2})} = \frac{1}{\gamma} \in \mathbb{R},
\]
realizing the completed zeta function \( \Xi(s) \) spectrally through the canonical determinant identity
\[
\det\nolimits_\zeta(I - \lambda L_{\mathrm{sym}}) = \frac{\Xi\left( \tfrac{1}{2} + i\lambda \right)}{\Xi\left( \tfrac{1}{2} \right)}.
\]

\medskip
\noindent
This correspondence is developed through the following modular structure:
\begin{itemize}
  \item \textbf{Injection:} Every nontrivial zero of \( \zeta(s) \) yields a nonzero eigenvalue of \( L_{\mathrm{sym}} \) \textbf{(Lemma~\ref{lem:zero-to-eigenvalue-injection})}.
  \item \textbf{Surjection:} Every nonzero eigenvalue of \( L_{\mathrm{sym}} \) arises from such a zero \textbf{(Lemma~\ref{lem:spectral_exhaustivity})}.
  \item \textbf{Multiplicity:} The order of each zero matches the algebraic multiplicity of the corresponding eigenvalue \textbf{(Lemma~\ref{lem:spectral-multiplicity-matching})}.
  \item \textbf{Symmetry:} The spectrum of \( L_{\mathrm{sym}} \) is symmetric about the origin \textbf{(Lemma~\ref{lem:spectral-symmetry})}, reflecting the functional identity \( \zeta(s) = \zeta(1 - s) \).
  \item \textbf{Bijection:} The map \( \rho \mapsto \mu_\rho \) defines a multiplicity-preserving bijection between the nontrivial zeros and the nonzero spectrum \textbf{(Lemma~\ref{lem:spectral_bijection_consistency})}.
  \item \textbf{Spectral Measure:} The spectral heat trace expansion recovers \( \Xi \) via integration over the spectral measure \textbf{(Lemma~\ref{lem:spectral-heat-trace-representation})}.
  \item \textbf{Spectrum-Reality Equivalence:} The spectrum of \( L_{\mathrm{sym}} \) is real if and only if the Riemann Hypothesis holds \textbf{(Lemma~\ref{lem:spectrum-reality-implies-rh})}.
\end{itemize}

\medskip
\noindent
Theorem~\ref{thm:spectral-zero-bijection-revised} consolidates these results into a canonical spectral identity:
\[
\mathrm{Spec}(L_{\mathrm{sym}}) \setminus \{0\} = \left\{ \mu_\rho = \frac{1}{i(\rho - \tfrac{1}{2})} : \zeta(\rho) = 0 \right\},
\]
with algebraic multiplicities preserved. This bijection underpins the spectral reformulation of the Riemann Hypothesis in Chapter~\ref{sec:spectral-implications}.

\medskip
\noindent\textbf{Canonical Spectral Schematic:} See Figure~\ref{fig:spectral-bijection-table}.

\begin{figure}[ht]
\centering
\renewcommand{\arraystretch}{1.25}
\begin{tabular}{c|c|c}
\toprule
\textbf{Zeta Zero \( \rho_k \)} & \textbf{Canonical Map} & \textbf{Eigenvalue \( \mu_{\rho_k} \)} \\
\midrule
\( \tfrac{1}{2} + i\gamma_1 \) & \( \mu_{\rho_1} = \dfrac{1}{i(\rho_1 - \tfrac{1}{2})} \) & \( \dfrac{1}{\gamma_1} \) \\
\( \tfrac{1}{2} + i\gamma_2 \) & \( \mu_{\rho_2} = \dfrac{1}{i(\rho_2 - \tfrac{1}{2})} \) & \( \dfrac{1}{\gamma_2} \) \\
\( \tfrac{1}{2} + i\gamma_3 \) & \( \mu_{\rho_3} = \dfrac{1}{i(\rho_3 - \tfrac{1}{2})} \) & \( \dfrac{1}{\gamma_3} \) \\
\( \vdots \) & \( \vdots \) & \( \vdots \) \\
\bottomrule
\end{tabular}
\caption{
Canonical spectral encoding of the nontrivial zeros \( \rho_k = \tfrac{1}{2} + i\gamma_k \) of the Riemann zeta function via the real eigenvalues \( \mu_{\rho_k} = 1/\gamma_k \) of the compact self-adjoint operator \( L_{\mathrm{sym}} \). This bijection is established in Theorem~\ref{thm:spectral-zero-bijection-revised} and follows analytically from the Fredholm determinant identity.
}
\label{fig:spectral_bijection_table}
\end{figure}
%  
%  