\begin{proof}[Proof of \thmref{thm:uniqueness_realization}]
Let \( L \in \TC(\HPsi) \cap \KC(\HPsi) \) be a compact, self-adjoint, trace-class operator on the weighted Hilbert space \( \HPsi := L^2(\R, e^{\alpha|x|}dx) \), with fixed \( \alpha > \pi \). Suppose:
\[
\det\nolimits_\zeta(I - \lambda L) = \frac{\XiR\left(\tfrac{1}{2} + i\lambda\right)}{\XiR\left(\tfrac{1}{2}\right)}
= \det\nolimits_\zeta(I - \lambda L_{\sym}),
\quad \forall \lambda \in \C.
\]

\paragraph{Step 1: Spectral Data from Determinant Identity.}
By classical trace-class determinant theory (see~\cite[Thm. 4.2]{Simon2005TraceIdeals}), the normalized Carleman–\(\zeta\)-regularized determinant admits the product representation:
\[
\det\nolimits_\zeta(I - \lambda L)
= \prod_{n=1}^\infty (1 - \lambda \mu_n),
\]
where \( \{ \mu_n \} \subset \R \setminus \{0\} \) are the nonzero eigenvalues of \( L \), counted with algebraic multiplicity. Since this determinant agrees identically with that of \( L_{\sym} \), and both are entire functions of order one normalized by \(\det\nolimits_\zeta(I) = 1\), we conclude:
\[
\Spec(L) = \Spec(L_{\sym}),
\quad \text{as multisets}.
\]

\paragraph{Step 2: Spectral Equivalence Implies Unitary Equivalence.}
Since \( L \) and \( L_{\sym} \) are both compact, self-adjoint operators on the same separable Hilbert space \( \HPsi \), and since their spectra (with multiplicities) coincide, the spectral theorem implies that \( L \) is unitarily equivalent to \( L_{\sym} \). That is, there exists a unitary operator
\[
U \colon \HPsi \to \HPsi
\quad \text{such that} \quad
L = U L_{\sym} U^{-1}.
\]

\paragraph{Conclusion.}
The canonical operator \( L_{\sym} \) is thus uniquely determined (up to unitary equivalence) among all compact, self-adjoint, trace-class operators realizing the normalized spectral determinant identity for \( \XiR(s) \). The analytic fingerprint of \( \XiR \)—its order-one entire structure, exponential type, and Hadamard factorization—rigidly determines the operator-theoretic data of \( L_{\sym} \), completing the proof.
\end{proof}
