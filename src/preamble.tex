%==============================
% Preamble: Core Packages and Custom Definitions
%==============================

%--- Essential Math Packages ---
\usepackage{amsmath, amssymb, amsfonts, mathrsfs}
\usepackage{graphicx}
\usepackage{lmodern}
\usepackage{microtype}
\usepackage{tikz}
\usetikzlibrary{arrows.meta, positioning, cd}

%--- Table and Caption Formatting ---
\usepackage{booktabs}
\usepackage{tabularx}
\usepackage{caption}

%--- Hyperlinking (black links, PDF metadata) ---
\usepackage{hyperref}
\hypersetup{
    bookmarksdepth=section,
    breaklinks=true,
    colorlinks=true,
    linkcolor=black,
    citecolor=black,
    urlcolor=black,
    pdfauthor={R.A. Jacob Martone},
    pdftitle={A Canonical Spectral Proof of the Riemann Hypothesis},
    pdfsubject={Mathematics -- Spectral Theory, Zeta Functions, Trace-Class Operators},
    pdfkeywords={Riemann Hypothesis, Fredholm Determinant, Trace-Class Operator, Hilbert Space, Zeta Regularization, Heat Kernel, Spectral Theory}
}

%==============================
% Layout and Font Adjustments
%==============================
\linespread{1.12}  % Improves vertical spacing for readability
\setlength{\headheight}{18pt}

%==============================
% Custom Column Types
%==============================
\newcolumntype{L}{>{\raggedright\arraybackslash}X}

%==============================
% Number Systems
%==============================
\newcommand{\R}{\mathbb{R}}
\newcommand{\C}{\mathbb{C}}
\newcommand{\Z}{\mathbb{Z}}
\newcommand{\N}{\mathbb{N}}

%==============================
% Function Spaces
%==============================
\newcommand{\Lp}[1]{L^{#1}(\R)}
\newcommand{\Schwartz}{\mathcal{S}(\R)}
\newcommand{\HPsi}{H_{\Psi}}
\newcommand{\Hilb}{\mathcal{H}}

%==============================
% Operator-Theoretic Objects
%==============================
\newcommand{\TC}{\mathcal{B}_1}       % Trace-class operators
\newcommand{\KC}{\mathcal{K}}         % Compact operators
\newcommand{\SC}{\mathcal{S}_2}       % Hilbert-Schmidt
\newcommand{\Tr}{\operatorname{Tr}}   % Trace
\newcommand{\Spec}{\operatorname{Spec}} 
\newcommand{\Ran}{\operatorname{Ran}} 
\newcommand{\Ker}{\operatorname{Ker}} 
\newcommand{\HS}{\mathrm{HS}}         
\newcommand{\Lop}{L_t}                % Mollified operator

%==============================
% Special Functions and Weights
%==============================
\newcommand{\XiR}{\Xi}
\newcommand{\zetaR}{\zeta}
\newcommand{\PsiW}{\Psi}
\newcommand{\PsiAlpha}[1]{\exp\!\bigl(\alpha |#1|\bigr)}

%==============================
% Fourier Transform Convention
%==============================
\newcommand{\FT}{\mathcal{F}}
\newcommand{\ft}[1]{\widehat{#1}}

%==============================
% Math Utilities
%==============================
\newcommand{\abs}[1]{\left|#1\right|}
\newcommand{\norm}[1]{\left\|#1\right\|}
\newcommand{\ip}[2]{\left\langle #1, #2 \right\rangle}

%==============================
% Cross-Reference Shortcuts
%==============================
\newcommand{\thmref}[1]{Theorem~\ref{#1}}
\newcommand{\lemref}[1]{Lemma~\ref{#1}}
\newcommand{\corref}[1]{Corollary~\ref{#1}}
\newcommand{\propref}[1]{Proposition~\ref{#1}}
\newcommand{\defref}[1]{Definition~\ref{#1}}
\newcommand{\secref}[1]{Section~\ref{#1}}
\newcommand{\appref}[1]{Appendix~\ref{#1}}
\newcommand{\eqnref}[1]{\eqref{#1}}

%==============================
% Additional Math Labels
%==============================
\newcommand{\sym}{\mathrm{sym}}
\newcommand{\op}{\mathrm{op}}

%==============================
% Weighted Spaces (exponential profile)
%==============================
\newcommand{\PsiAlphaSpace}[1]{L^2(\R, \PsiAlpha{#1}\,dx)}

%==============================
% Theorem Environments
%==============================

\theoremstyle{plain}
\newtheorem{theorem}{Theorem}[section]
\newtheorem{lemma}[theorem]{Lemma}
\newtheorem{proposition}[theorem]{Proposition}
\newtheorem{corollary}[theorem]{Corollary}
\newtheorem{conjecture}[theorem]{Conjecture}

\theoremstyle{definition}
\newtheorem{definition}[theorem]{Definition}
\newtheorem{example}[theorem]{Example}

\theoremstyle{remark}
\newtheorem{remark}[theorem]{Remark}
\newtheorem*{auditnote}{Audit Note}

% ===== Preview Box Environment =====
\usepackage[most]{tcolorbox}

\newtcolorbox{previewbox}{
  colback=gray!5,
  colframe=black,
  fonttitle=\bfseries,
  coltitle=black,
  sharp corners,
  boxrule=0.4pt,
  title={Spectral Preview},
  before skip=10pt,
  after skip=10pt,
  breakable
}

\newenvironment{references}{
  \medskip
  \noindent\textbf{References.}
  \begin{list}{}{\leftmargin=1.5em \itemindent=-1.5em \itemsep=0pt}
  \item[]
}{
  \end{list}
  \medskip
}

\usepackage{pgfplots}
\pgfplotsset{compat=1.17}

% In preamble
\makeatletter
\DeclareRobustCommand{\widecheck}[1]{
  \mathpalette\@widecheck{#1}
}
\def\@widecheck#1#2{%
  \setbox\z@\hbox{\m@th#1#2}%
  \setbox\tw@\hbox{\m@th#1\widehat{\box\z@}}%
  \setbox\tw@\hbox to \wd\z@{\hss\box\tw@\hss}%
  {\vbox{\offinterlineskip
      \box\tw@
      \vskip -\ht\z@
      \box\z@}}}
\makeatother


\newcommand{\remref}[1]{Remark~\ref{#1}}
