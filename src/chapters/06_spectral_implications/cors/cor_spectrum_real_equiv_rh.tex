\begin{corollary}[Equivalence of RH with Spectrum Reality]
\label{cor:spectrum_real_equiv_rh}

Let \( L_{\sym} \in \TC(\HPsi) \cap \KC(\HPsi) \) be the canonical compact, self-adjoint, trace-class operator whose Carleman–\(\zeta\)-regularized Fredholm determinant satisfies:
\[
\det\nolimits_\zeta(I - \lambda L_{\sym})
= \frac{\XiR\left(\tfrac{1}{2} + i\lambda\right)}{\XiR\left(\tfrac{1}{2}\right)},
\qquad \forall \lambda \in \C.
\]

Then the Riemann Hypothesis is equivalent to the spectral reality of \( L_{\sym} \):
\[
\boxed{
\Spec(L_{\sym}) \subset \R
\quad \Longleftrightarrow \quad \RH
}
\]

\noindent
That is, all nontrivial zeros of the Riemann zeta function \( \zetaR(s) \) lie on the critical line \( \Re(s) = \tfrac{1}{2} \) if and only if every eigenvalue of \( L_{\sym} \) is real.

\medskip

\noindent
This equivalence follows directly from \thmref{thm:eq_of_rh}, and rests analytically on the determinant identity in \thmref{thm:det_identity_revised}, the spectral bijection in \thmref{thm:spectral_zero_bijection_revised}, and the multiplicity preservation proven in \lemref{lem:multiplicity_preservation}.

\medskip

It constitutes the operator-theoretic core of the analytic–spectral reformulation of the Riemann Hypothesis.
\end{corollary}
