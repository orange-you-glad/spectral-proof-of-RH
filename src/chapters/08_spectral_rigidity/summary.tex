\subsection*{Summary}

This chapter reformulates the Riemann Hypothesis as a statement of spectral rigidity: the spectrum and determinant of the canonical operator \( L_{\sym} \in \TC(\HPsi) \) determine both the location and multiplicity of all nontrivial zeros of \( \zetaR(s) \).

\begin{itemize}
  \item \lemref{lem:spectral_encoding_injection} — The canonical determinant identity defines a multiplicity-preserving injection
  \[
  \rho \mapsto \mu_\rho := \frac{1}{i}(\rho - \tfrac{1}{2}),
  \]
  mapping nontrivial zeros of \( \zetaR(s) \) to eigenvalues of \( L_{\sym} \).

  \item \lemref{lem:det_zero_implies_spectrum} — Each zero of the determinant \( \det\nolimits_\zeta(I - \lambda L_{\sym}) \) corresponds to a reciprocal eigenvalue \( \mu := \lambda^{-1} \in \Spec(L_{\sym}) \).

  \item \lemref{lem:inject_zero_real_spectrum} — If \( \Spec(L_{\sym}) \subset \R \), then all nontrivial zeros lie on the critical line:
  \[
  \Spec(L_{\sym}) \subset \R \quad \Rightarrow \quad \Re(\rho) = \tfrac{1}{2}.
  \]

  \item \lemref{lem:real_spectrum_implies_rh_rigidity} — The above implication holds without assuming simplicity or bijectivity, using only the determinant identity and spectral reality.

  \item \propref{prop:spectrum_reality_implies_rh_multiplicity} — Real, simple spectrum implies both RH and simplicity of all nontrivial zeta zeros.

  \item \thmref{thm:rh_from_real_simple_spectrum} — If \( L_{\sym} \) has real, simple spectrum and satisfies the canonical determinant identity, then:
  \[
  \RH \text{ holds and all nontrivial zeros are simple.}
  \]

  \item \lemref{lem:spectral_symmetry} — Spectrum symmetry under \( \mu \mapsto -\mu \) follows from the functional identity of \( \XiR(s) \).

  \item \lemref{lem:trace_distribution_positivity} — The trace pairing
  \[
  \phi \mapsto \Tr(\phi(L_{\sym}))
  \]
  defines a positive tempered distribution. In particular:
  \[
  \Tr(e^{-t L_{\sym}^2}) \ge 0.
  \]

  \item \lemref{lem:trace_distribution_positive} — Positivity extends to all nonnegative Schwartz functions. See \remref{rem:functional_calculus_trace} for domain justification.

  \item \propref{prop:inverse_spectral_rigidity} — Any two compact, self-adjoint trace-class operators sharing the same spectrum and determinant identity are unitarily equivalent.
\end{itemize}

These results confirm that the Riemann Hypothesis can be derived purely from the spectral and determinant data of \( L_{\sym} \). This chapter completes the analytic implication
\[
\Spec(L_{\sym}) \subset \R \quad \Longrightarrow \quad \RH,
\]
and closes the logical loop initiated in Chapter~\ref{sec:spectral_implications}. This closure is encoded structurally in the dependency graph (\appref{app:dependency_graph}).

\medskip

\noindent
\textbf{Remarks.}
\begin{itemize}
  \item Spectral simplicity suffices (but is not required) for RH.
  \item The determinant–spectrum map arises intrinsically from Fredholm theory and the Hadamard structure of \( \XiR(s) \).
  \item Positivity of the trace distribution reflects harmonic-analytic regularity of the spectral measure.
  \item Under the diagonalization \( \FT_L \), the trace pairing becomes integration against the spectral measure.
\end{itemize}

\medskip

\noindent
\textbf{Forward Link.}  
The final chapter (\cref{sec:logical_closure}) consolidates all analytic implications into a logically complete equivalence:
\[
\RH \iff \Spec(L_{\sym}) \subset \R.
\]
