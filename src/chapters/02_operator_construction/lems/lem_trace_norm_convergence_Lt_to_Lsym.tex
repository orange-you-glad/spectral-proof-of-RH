\begin{lemma}[Trace-Norm Convergence \( L_t \to L_{\mathrm{sym}} \)]
\label{lem:trace-norm-convergence-Lt-to-Lsym}
Let \( \alpha > \pi \) be fixed, and define the weighted Hilbert space
\[
H_{\Psi_\alpha} := L^2(\mathbb{R}, e^{\alpha|x|}\, dx).
\]
Let
\[
\phi(\lambda) := \Xi\left( \tfrac{1}{2} + i\lambda \right)
\]
be the canonical Fourier profile.

Define the mollified profiles and associated convolution kernels by
\[
\varphi_t(\lambda) := e^{-t\lambda^2} \phi(\lambda), \qquad
K_t(x,y) := \frac{1}{2\pi} \int_{\mathbb{R}} e^{i(x - y)\lambda} \varphi_t(\lambda)\, d\lambda.
\]
Then each \( L_t \colon f \mapsto \int K_t(x,y) f(y) dy \) defines a trace-class operator on \( H_{\Psi_\alpha} \):
\[
L_t \in \mathcal{C}_1(H_{\Psi_\alpha}), \quad \text{with } \sup_{0 < t \le 1} \|L_t\|_{\mathcal{C}_1} < \infty.
\]
This follows from the exponential decay of \( K_t(x,y) \), implied by Paley--Wiener theory applied to \( \varphi_t \in PW_\pi \cap \mathcal{S}(\mathbb{R}) \) \cite[Thm.~IX.12]{ReedSimon1975II}, and from classical results on Hilbert--Schmidt kernels in weighted \( L^2 \) spaces \cite[Ch.~4]{Simon2005TraceIdeals}.

\medskip

Define the canonical operator \( L_{\mathrm{sym}} \) as the unique Schatten-1 limit:
\[
L_{\mathrm{sym}} := \lim_{t \to 0^+} L_t \quad \text{in } \mathcal{C}_1(H_{\Psi_\alpha}).
\]

Then:
\[
\lim_{t \to 0^+} \| L_t - L_{\mathrm{sym}} \|_{\mathcal{C}_1(H_{\Psi_\alpha})} = 0.
\]

\medskip

\noindent
This convergence is guaranteed by the following:
\begin{itemize}
  \item Pointwise convergence \( \varphi_t(\lambda) \to \phi(\lambda) \) for all \( \lambda \in \mathbb{R} \);
  \item Local \( L^1 \)-convergence \( \varphi_t \to \phi \) in \( L^1_{\mathrm{loc}} \);
  \item Uniform trace-norm bounds on \( L_t \), and exponential decay of the mollified kernels \( K_t \);
  \item The limiting kernel
  \begin{multline*}
  K_{\mathrm{sym}}(x,y) := \frac{1}{2\pi} \int_{\mathbb{R}} 
  e^{i(x-y)\lambda} \phi(\lambda)\, d\lambda
  \end{multline*}
  also defines a convolution operator \( L_{\mathrm{sym}} \in \mathcal{C}_1(H_{\Psi_\alpha}) \), since
  \[
  K_{\mathrm{sym}} \in L^1(\mathbb{R}^2; \Psi_\alpha(x)\Psi_\alpha(y)\,dx\,dy).
  \]
\end{itemize}

\paragraph{Spectral Implications.}
Since the Schatten-1 convergence \( L_t \to L_{\mathrm{sym}} \) holds, we have:
\[
\mathrm{Tr}\left(e^{-tL_t^2}\right) \to \mathrm{Tr}\left(e^{-tL_{\mathrm{sym}}^2}\right), \quad \text{as } t \to 0^+,
\]
and for all \( \lambda \in \mathbb{R} \),
\[
\det\nolimits_{\zeta}(I - \lambda L_t) \to \det\nolimits_{\zeta}(I - \lambda L_{\mathrm{sym}}).
\]
These analytic properties are foundational for the spectral determinant framework and are rigorously developed in Chapter~\ref{sec:determinant-identity}.
\end{lemma}
% 