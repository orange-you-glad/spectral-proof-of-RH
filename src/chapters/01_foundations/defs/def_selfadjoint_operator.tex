\begin{definition}[Self-Adjoint Operators]\label{def:selfadjoint_operator}
Let \( H \) be a separable complex Hilbert space, and let \( T \colon \mathcal{D}(T) \subset H \to H \) be a densely defined linear operator.

The operator \( T \) is \emph{self-adjoint} if
\[
T = T^*, \qquad \text{and} \qquad \mathcal{D}(T) = \mathcal{D}(T^*),
\]
where the adjoint \( T^* \colon \mathcal{D}(T^*) \to H \) is defined as follows: a vector \( g \in H \) lies in \( \mathcal{D}(T^*) \) if there exists \( h \in H \) such that
\[
\langle T f, g \rangle = \langle f, h \rangle \quad \text{for all } f \in \mathcal{D}(T),
\]
in which case \( T^* g := h \). The Riesz representation theorem ensures that \( T^* \) is well-defined and closed.

\medskip
\noindent\textbf{Bounded Case.}
If \( T \in \mathcal{B}(H) \) is bounded and everywhere defined, then self-adjointness reduces to symmetry:
\[
\langle T f, g \rangle = \langle f, T g \rangle \quad \text{for all } f, g \in H.
\]

\medskip
\noindent\textbf{Symmetric Operators.}
A densely defined operator \( T \colon \mathcal{D}(T) \to H \) is \emph{symmetric} if
\[
\langle T f, g \rangle = \langle f, T g \rangle \quad \text{for all } f, g \in \mathcal{D}(T),
\]
i.e., \( T \subseteq T^* \). Such an operator is self-adjoint if and only if \( \mathcal{D}(T) = \mathcal{D}(T^*) \).

\medskip
\noindent\textbf{Graph Characterization.}
\( T \) is self-adjoint if and only if its graph
\[
\operatorname{graph}(T) := \{ (f, Tf) \in H \times H : f \in \mathcal{D}(T) \}
\]
is closed and coincides with the graph of its adjoint \( T^* \). In particular, self-adjoint operators are always closed and symmetric, with no proper symmetric extensions.

\medskip
\noindent\textbf{Spectral Theorem.}
Every self-adjoint operator \( T \colon \mathcal{D}(T) \to H \) admits a unique spectral decomposition:
\[
T = \int_{\sigma(T)} \lambda \, dE_\lambda,
\]
where \( \{ E_\lambda \} \) is a projection-valued measure (PVM) on the Borel subsets of \( \R \), supported on the spectrum \( \sigma(T) \subset \R \). Consequently:
\begin{itemize}
    \item \( \sigma(T) \subset \R \) is closed and non-empty;
    \item The spectral calculus \( f(T) := \int f(\lambda)\, dE_\lambda \) is well-defined for all bounded Borel functions \( f \colon \R \to \C \);
    \item The one-parameter unitary group \( \{ e^{itT} \}_{t \in \R} \subset \mathcal{U}(H) \) is strongly continuous.
\end{itemize}

\medskip
\noindent\textbf{Compact Self-Adjoint Operators.}
If \( T \in \mathcal{C}_1(H) \cap \operatorname{SA}(H) \), then \( \sigma(T) \subset \R \) consists entirely of isolated eigenvalues of finite multiplicity, with \( \lambda_n \to 0 \). The eigenfunctions form a complete orthonormal basis of \( H \).

\medskip
\noindent\textbf{Remarks.}
\begin{itemize}
    \item Self-adjointness guarantees real spectrum, a complete functional calculus, and unitary time evolution.
    \item A symmetric operator \( T_0 \colon \mathcal{D}_0 \to H \) is said to be \emph{essentially self-adjoint} if its closure \( \overline{T_0} \) is self-adjoint. This is essential in spectral theory and quantum mechanics.
    \item In this manuscript, all mollified convolution operators \( L_t \) and the canonical limit \( L_{\sym} \) are constructed to be essentially self-adjoint on the Schwartz core \( \Schwartz \subset \HPsi \) (cf.~\secref{sec:operator_construction}).
    \item For symmetric integral operators with Hermitian kernels and exponential damping, self-adjointness follows from symmetry and essential self-adjointness on the core via Friedrichs extensions.
\end{itemize}

\medskip
\noindent\textbf{References.}
\begin{itemize}
    \item M.~Reed and B.~Simon, \emph{Methods of Modern Mathematical Physics I: Functional Analysis}, Chapter~VI \cite{ReedSimon1980I}.
    \item B.~Simon, \emph{Trace Ideals and Their Applications}, Chapter~3 \cite{Simon2005TraceIdeals}.
\end{itemize}
\end{definition}
