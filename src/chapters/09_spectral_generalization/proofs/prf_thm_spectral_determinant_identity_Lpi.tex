\begin{proof}[Proof of \thmref{thm:spectral_determinant_identity_Lpi}]
Let \( \phi_t^{(\pi)}(\lambda) := e^{-t\lambda^2} \, \Xi\left(\tfrac{1}{2} + i\lambda, \pi\right) \), and let \( k_t^{(\pi)} := \FT^{-1}[\phi_t^{(\pi)}] \) be its inverse Fourier transform. By the exponential decay assumption in \defref{def:kernel_decay_condition}, the associated convolution operator \( \Ltpi \in \TC(\Hpsipi) \) for all \( t > 0 \), and \( \Ltpi \to \Lsympi \in \TC(\Hpsipi) \) as \( t \to 0^+ \), by \lemref{lem:trace_class_convergence_general}.

By the spectral theorem, since each \( \Ltpi \) is trace class and self-adjoint, the Fredholm determinant \( \detz(I - \lambda \Ltpi) \) exists and defines an entire function in \( \lambda \in \C \). Moreover, for each \( t > 0 \), the mollified identity
\[
\detz(I - \lambda \Ltpi) = \frac{\Xi\left(\tfrac{1}{2} + i\lambda, \pi\right) \cdot e^{-t\lambda^2}}{\Xi\left(\tfrac{1}{2}, \pi\right)}
\]
holds by direct computation of the Fourier–Laplace transform.

Taking the trace-norm limit \( t \to 0^+ \), we have
\[
\lim_{t \to 0^+} \detz(I - \lambda \Ltpi)
= \frac{\Xi\left(\tfrac{1}{2} + i\lambda, \pi\right)}{\Xi\left(\tfrac{1}{2}, \pi\right)},
\]
since \( \exp(-t\lambda^2) \to 1 \) uniformly on compact subsets of \( \C \), and the determinant is continuous under trace-norm convergence~\cite[Thm.~6.5]{Simon2005TraceIdeals}.

Thus, the determinant identity holds for the limit operator \( \Lsympi \):
\[
\detz(I - \lambda \Lsympi) = \frac{\Xi\left(\tfrac{1}{2} + i\lambda, \pi\right)}{\Xi\left(\tfrac{1}{2}, \pi\right)}.
\]
\end{proof}
