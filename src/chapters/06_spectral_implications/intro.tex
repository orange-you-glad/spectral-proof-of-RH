\subsection*{Introduction}
\label{sec:intro_spectral_implications}

This chapter concludes the analytic–spectral phase of the manuscript by proving two core consequences of the canonical determinant identity established in \thmref{thm:det_identity_revised}:

\begin{itemize}
  \item \textbf{Spectral Equivalence with RH:} The Riemann Hypothesis is equivalent to the reality of the spectrum of \( \Lsym \), i.e.,
  \[
  \RH \iff \Spec(\Lsym) \subset \R,
  \]
  as proven in \lemref{lem:reality_of_spectrum_and_rh} and summarized in \corref{cor:spectrum_real_equiv_rh}.
  
  \item \textbf{Spectral Uniqueness:} Any compact, self-adjoint, trace-class operator whose zeta-regularized determinant equals the normalized \( \Xi(s) \) determinant is unitarily equivalent to \( \Lsym \), as shown in \thmref{thm:uniqueness_realization}.
\end{itemize}

These results are analytically grounded in the trace-norm heat kernel asymptotics, Laplace-integrable semigroup theory, and spectral zeta regularization developed in Chapter~\ref{sec:heat_kernel_asymptotics} and Appendix~\ref{app:heat_kernel_construction}. Crucially, \textit{no assumption of the Riemann Hypothesis} is used in their derivation.

\paragraph{RH–Spectrum Equivalence.}
The core logical equivalence
\[
\RH \iff \Spec(\Lsym) \subset \R
\]
follows from the determinant identity, the bijection \( \rho \mapsto \mu_\rho \), and the spectral trace properties of \( L_{\mathrm{sym}} \). The spectrum is fully determined by the zeros of \( \zeta(s) \), and their location governs the reality of the spectrum.

\paragraph{Uniqueness of the Canonical Operator.}
Let \( L \in \TC(\HPsi) \) be any compact, self-adjoint operator satisfying the same determinant identity:
\[
\det_\zeta(I - \lambda L) = \frac{\XiR\left(\tfrac{1}{2} + i\lambda\right)}{\XiR\left(\tfrac{1}{2}\right)}, \qquad \forall \lambda \in \C.
\]
Then \( L \) is unitarily equivalent to \( \Lsym \). This rigidity result shows that \( \Lsym \) is uniquely determined—up to unitary conjugation—by the analytic structure of \( \Xi(s) \).

\paragraph{Canonical Operator-Theoretic Principle.}
\begin{quote}
The completed zeta function \( \Xi(s) \), through its Hadamard factorization and functional symmetry, canonically determines a unique compact, self-adjoint, trace-class operator whose spectrum encodes RH via the reality condition \( \Spec(\Lsym) \subset \R \).
\end{quote}

This chapter completes the analytic chain initiated in Chapter~\ref{sec:determinant_identity} and prepares the logical closure in Chapter~\ref{sec:logical_closure}.

\paragraph{Analytic Closure.}
All arguments in this chapter—determinant identity, trace class convergence, spectral zeta analysis—are rigorously constructed using operator semigroup theory and heat kernel asymptotics. All analytic dependencies are modular, acyclic, and traced in the DAG (\appref{app:dependency_graph}).

\paragraph{Logical Completeness.}
Every cited estimate, trace identity, and spectral implication is proven explicitly or cited with proof. The logical structure of this chapter is closed, canonical, and consistent with the global framework.
