\begin{proof}[Proof of Lemma~\ref{lem:spectrum-reality-implies-rh}]
We prove both directions of the equivalence.

First, assume that all zeros of \( \Xi(\tfrac{1}{2} + i\lambda) \) lie on the real axis. Then the canonical determinant identity
\[
\det\nolimits_\zeta(I - \lambda L_{\mathrm{sym}}) = \frac{\Xi(\tfrac{1}{2} + i\lambda)}{\Xi(\tfrac{1}{2})}
\]
has zeros only for real \( \lambda \). Since this determinant arises from the spectrum of the compact, self-adjoint operator \( L_{\mathrm{sym}} \), the location of its zeros implies that all eigenvalues \( \mu_\rho \in \mathbb{R} \), as guaranteed by the spectral theorem.

Conversely, assume that \( L_{\mathrm{sym}} \in \mathcal{C}_1(H_{\Psi_\alpha}) \) is self-adjoint with discrete spectrum \( \{\mu_\rho\} \subset \mathbb{R} \). Then by the canonical eigenvalue encoding
\[
\mu_\rho = \frac{1}{i(\rho - \tfrac{1}{2})},
\]
it follows that
\[
\rho = \tfrac{1}{2} + \frac{1}{i\mu_\rho}.
\]
Since \( \mu_\rho \in \mathbb{R} \), this implies \( \operatorname{Re}(\rho) = \tfrac{1}{2} \). Hence, all nontrivial zeros of \( \zeta(s) \) lie on the critical line, and the Riemann Hypothesis holds.
\end{proof}
%  