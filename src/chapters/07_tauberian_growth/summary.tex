\subsection*{Summary}

\begin{itemize}
  % --- Tauberian Principle ---
  \item \defref{def:tauberian_theorem} — Classical Laplace–Tauberian framework: short-time behavior of the heat trace \( \Tr(e^{-tL^2}) \) determines high-energy spectral growth \( N(\Lambda) \) via inverse Laplace analysis.

  % --- Asymptotic Bounds and Growth Envelopes ---
  \item \lemref{lem:spectral_convexity_estimate} — From the envelope \( \Tr(e^{-t L^2}) = O(t^{-1/2}) \), we deduce the coarse growth bound:
  \[
  N(\lambda) = O(\lambda^{1/2+\varepsilon}),
  \]
  confirming trace-class compactness and effective dimension \( d = 1 \).

  \item \lemref{lem:log_corrected_tauberian_estimate} — From the refined expansion
  \[
  \Tr(e^{-t L^2}) \sim \frac{1}{\sqrt{4\pi t}} \log\left( \frac{1}{t} \right),
  \]
  Korevaar’s Tauberian theory yields the sharp asymptotic:
  \[
  N(\lambda) \sim \frac{\sqrt{\lambda}}{\pi} \log \lambda.
  \]

  \item \corref{cor:zeta_compatibility} — The spectral growth law for \( L_{\sym} \) matches the Riemann–von Mangoldt formula:
  \[
  N_\zeta(T) = \frac{T}{2\pi} \log\left( \frac{T}{2\pi} \right) - \frac{T}{2\pi} + O(\log T),
  \]
  confirming analytic agreement between the spectrum of \( L_{\sym} \) and the zero-counting function for \( \zeta(s) \).
\end{itemize}

\medskip
\noindent
(For higher-order heat kernel corrections and refined Tauberian remainder terms, see Appendix~\ref{app:heat_kernel_refinements}.)

\vspace{0.5em}

\noindent\textbf{Analytic Flow Diagram:}
\[
\begin{aligned}
\text{Heat Kernel}
&\xrightarrow{\text{Laplace}} \text{Trace Asymptotics} \\
&\xrightarrow{\text{Tauberian}} \text{Spectral Growth} \\
&\xrightarrow{\text{Hadamard}} \text{Zeta Determinant}
\end{aligned}
\]

\begin{remark}[Transition to Spectral Rigidity]
We have shown that the asymptotic eigenvalue distribution of \( L_{\sym} \) encodes the nontrivial zeros of the Riemann zeta function. The next natural question is rigidity: if an operator shares both the spectrum and the determinant identity, is it unitarily equivalent to \( L_{\sym} \)?

This question leads directly to the spectral rigidity results in Chapter~\ref{sec:spectral_rigidity}.
\end{remark}

\paragraph{Logical Closure.}
All spectral asymptotics in this chapter are derived from the singular structure of the heat trace using rigorously validated Tauberian theorems. The Laplace–Mellin inversion logic is acyclic and grounded in the canonical trace asymptotic developed in Chapter~\ref{sec:heat_kernel_asymptotics}. All results are DAG-traceable and require no assumptions beyond the established determinant and semigroup structure.
