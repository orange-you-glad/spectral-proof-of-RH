\begin{theorem}[Self-Adjointness and Trace-Class Structure of \( L_{\sym} \)]
\label{thm:sa_trace_class_Lsym}
Let \( L_{\sym} \) be the canonical convolution operator defined as the trace-norm limit of the mollified operators \( L_t \), acting on the exponentially weighted Hilbert space
\[
H_{\Psi_\alpha} := L^2(\R, \Psi_\alpha(x)\, dx), \qquad \Psi_\alpha(x) := e^{\alpha |x|}, \quad \alpha > \pi.
\]

Then:
\begin{enumerate}
  \item[\textnormal{(i)}] \( L_{\sym} \in \TC(H_{\Psi_\alpha}) \); that is, it is trace class and realized as
  \[
  L_{\sym} = \lim_{t \to 0^+} L_t \quad \text{in } \TC(H_{\Psi_\alpha}).
  \]

  \item[\textnormal{(ii)}] \( L_{\sym} \in \KC(H_{\Psi_\alpha}) \); that is, it is compact, since every trace-class operator is compact.

  \item[\textnormal{(iii)}] \( L_{\sym} \) is self-adjoint:
  \[
  L_{\sym} = L_{\sym}^*,
  \]
  with domain closure obtained from the symmetric core \( \Schwartz \subset H_{\Psi_\alpha} \), as established in \lemref{lem:core_essential_sa}.
\end{enumerate}

\medskip

\noindent
This spectral classification guarantees:
\begin{itemize}
  \item The spectrum \( \Spec(L_{\sym}) \subset \R \) is discrete, consisting of real eigenvalues of finite multiplicity accumulating only at zero.
  \item The spectral theorem applies to \( L_{\sym} \), enabling functional calculus and definition of the semigroup \( e^{-tL_{\sym}^2} \).
  \item The Fredholm determinant identity derived in Chapter~\ref{sec:determinant_identity} is rigorously valid and encodes the nontrivial zeros of the completed Riemann zeta function \( \Xi(s) \).
\end{itemize}
\end{theorem}
