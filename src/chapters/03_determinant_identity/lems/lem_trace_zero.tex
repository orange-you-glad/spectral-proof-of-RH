\begin{lemma}[Vanishing Trace of \( L_{\sym} \)]
\label{lem:trace_zero}
Let \( L_{\sym} \in \TC(\HPsi) \) denote the canonical compact, self-adjoint convolution operator defined via the inverse Fourier transform of the completed Riemann zeta function:
\[
\phi(\lambda) := \Xi\left( \tfrac{1}{2} + i\lambda \right), \qquad
K_{\sym}(x,y) := \ft{\phi}(x - y).
\]

Then the operator satisfies the trace identity:
\[
\Tr(L_{\sym}) = \int_{\R} K_{\sym}(x,x) \, dx = 0.
\]

\medskip
\noindent
\textbf{Justification.}
Since \( \phi(\lambda) \in \R \) and \( \phi(-\lambda) = \phi(\lambda) \), the inverse Fourier transform \( \ft{\phi}(x) \) is real-valued and even. Hence, the diagonal kernel
\[
K_{\sym}(x,x) = \ft{\phi}(0)
\]
is constant in \( x \). The formal trace becomes
\[
\Tr(L_{\sym}) = \int_{\R} \ft{\phi}(0) \, dx,
\]
which diverges unless \( \ft{\phi}(0) = 0 \). But by the Fourier inversion formula,
\[
\ft{\phi}(0) = \frac{1}{2\pi} \int_{\R} \phi(\lambda) \, d\lambda = 0,
\]
as \( \phi \) is even, entire of exponential type \( \pi \), and satisfies decay properties as described in \lemref{lem:xi_growth_bound}. The vanishing of the integral reflects the symmetric zero distribution of \( \Xi \)~\cite[Ch.~3]{Levin1996EntireLectures}.

\medskip
\noindent
\textbf{Spectral Consequence.}
This vanishing ensures that the logarithmic derivative of the canonical determinant \( \log \det\nolimits_\zeta(I - \lambda L_{\sym}) \) has no linear term in \( \lambda \), as discussed in \lemref{lem:A_log_derivative}. That is, the Taylor expansion around zero contains no linear coefficient. Consequently, the determinant lies in the Hadamard class \( \mathcal{E}_1^\pi \) with normalization \( f(0) = 1 \), uniquely identifying it with the centered zeta profile (see \lemref{lem:det_identity_entire_order_one}).
\end{lemma}
