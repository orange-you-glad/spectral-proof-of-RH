\begin{lemma}[Spectral Convexity Estimate]
\label{lem:spectral_convexity_estimate}

Let \( L_{\sym} \in \TC(\HPsi) \) be the canonical compact, self-adjoint operator on the exponentially weighted Hilbert space
\[
\HPsi := L^2(\R, e^{\alpha |x|} dx), \qquad \alpha > \pi.
\]
Let \( \lambda_n := \mu_n^2 \) denote the nonzero eigenvalues of \( L_{\sym}^2 \in \TC \), ordered non-decreasingly and counted with multiplicity.

Then:

\begin{enumerate}
  \item[\textup{(i)}] There exists a constant \( C > 0 \) such that the spectral counting function
  \[
  N(\lambda) := \#\left\{ n \in \N : \lambda_n \le \lambda \right\}
  \]
  satisfies the convex envelope bound
  \[
  N(\lambda) \le C\, \lambda^{1/2}, \qquad \text{for all } \lambda \ge \lambda_0 > 0.
  \]

  \item[\textup{(ii)}] The associated Laplace–Stieltjes transform
  \[
  \Theta(t) := \int_0^\infty e^{-t\lambda}\, dN(\lambda)
  = \Tr(e^{-t L_{\sym}^2})
  \]
  satisfies the short-time upper bound
  \[
  \Theta(t) \lesssim t^{-1/2}, \qquad \text{as } t \to 0^+.
  \]
\end{enumerate}

\noindent
This estimate follows directly from Proposition~\ref{prop:two_sided_heat_trace_bounds}, and provides Tauberian admissibility for inversion of the spectral counting function \( N(\lambda) \).
\end{lemma}
