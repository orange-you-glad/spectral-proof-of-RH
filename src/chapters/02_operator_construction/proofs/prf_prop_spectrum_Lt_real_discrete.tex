\begin{proof}[Proof of Proposition~\ref{prop:spectrum-Lt-real-discrete}]
Let \( H_{\Psi_\alpha} := L^2(\mathbb{R}, \Psi_\alpha(x)\, dx) \), with \( \Psi_\alpha(x) := e^{\alpha |x|} \) and \( \alpha > \pi \). Let \( L_t \in \mathcal{B}(H_{\Psi_\alpha}) \) denote the mollified convolution operator with kernel \( K_t(x,y) := k_t(x - y) \), where \( k_t \in \mathcal{S}(\mathbb{R}) \) is the inverse Fourier transform of
\[
\phi_t(\lambda) := e^{-t\lambda^2} \, \Xi\left( \tfrac{1}{2} + i\lambda \right).
\]

\paragraph{(i) Trace-Class and Compactness.}
By Lemma~\ref{lem:trace-class-Lt}, each \( L_t \in \mathcal{C}_1(H_{\Psi_\alpha}) \). Since \( \mathcal{C}_1 \subset \mathcal{K} \), the operator is compact:
\[
L_t \in \mathcal{C}_1(H_{\Psi_\alpha}) \cap \mathcal{K}(H_{\Psi_\alpha}).
\]

\paragraph{(ii) Self-Adjointness.}
By Proposition~\ref{prop:selfadjointness-Lt}, the operator \( L_t \) is bounded and self-adjoint on \( H_{\Psi_\alpha} \), with symmetric core \( \mathcal{S}(\mathbb{R}) \subset H_{\Psi_\alpha} \).

\paragraph{(iii) Real and Discrete Spectrum.}
By the spectral theory of compact, self-adjoint operators on Hilbert spaces (see \cite[Thm.~VI.16]{ReedSimon1980I}), it follows that:
\[
\operatorname{Spec}(L_t) = \{ \lambda_n \} \subset \mathbb{R},
\]
where \( \lambda_n \to 0 \), and each eigenvalue \( \lambda_n \in \mathbb{R} \) has finite multiplicity. The only possible accumulation point is 0.

\paragraph{(iv) Orthonormal Basis of Eigenfunctions.}
Again by the compact self-adjoint spectral theorem, there exists an orthonormal basis \( \{ \psi_n \} \subset H_{\Psi_\alpha} \) such that
\[
L_t \psi_n = \lambda_n \psi_n.
\]
These \( \psi_n \) are eigenfunctions associated to the eigenvalues \( \lambda_n \).

\paragraph{Conclusion.}
The operator \( L_t \in \mathcal{C}_1(H_{\Psi_\alpha}) \cap \mathcal{K}(H_{\Psi_\alpha}) \) is self-adjoint and admits a discrete spectral resolution with real eigenvalues and a complete orthonormal basis of eigenfunctions.
\end{proof}
% 