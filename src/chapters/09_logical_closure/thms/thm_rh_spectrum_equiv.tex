\begin{theorem}[Spectral Equivalence with the Riemann Hypothesis]
\label{thm:rh_spectrum_equiv}

Let \( L_{\sym} \in \TC(\HPsi) \) denote the canonical compact, self-adjoint, trace-class operator on the exponentially weighted Hilbert space
\[
\HPsi := L^2(\R, e^{\alpha |x|} dx),
\qquad \text{for fixed } \alpha > \pi.
\]
Define the canonical spectral map
\[
\mu_\rho := \frac{1}{i(\rho - \tfrac{1}{2})}
\]
for each nontrivial zero \( \rho \in \C \) of the Riemann zeta function \( \zetaR(s) \).

Then the following are equivalent:
\begin{enumerate}
  \item[\textup{(i)}] The Riemann Hypothesis holds:
  \[
  \Re(\rho) = \tfrac{1}{2}, \quad \text{for all nontrivial zeros } \rho.
  \]

  \item[\textup{(ii)}] The spectrum of \( L_{\sym} \) lies entirely on the real line:
  \[
  \Spec(L_{\sym}) \subset \R.
  \]
\end{enumerate}

\medskip

\noindent
This equivalence follows from the bijective spectral encoding
\[
\rho \mapsto \mu_\rho := \frac{1}{i(\rho - \tfrac{1}{2})},
\]
which matches nontrivial zeta zeros with the nonzero spectrum of \( L_{\sym} \), as established in \thmref{thm:spectral_zero_bijection_revised}. The Carleman determinant identity
\[
\det\nolimits_\zeta(I - \lambda L_{\sym}) = \frac{\XiR\left( \tfrac{1}{2} + i\lambda \right)}{\XiR\left( \tfrac{1}{2} \right)}
\]
ensures that reality of spectrum corresponds exactly to the critical-line condition for all \( \rho \in \Spec(\zetaR) \).
\end{theorem}
