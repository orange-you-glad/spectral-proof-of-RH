\begin{definition}[Exponential Weight and Weighted Hilbert Space]\label{def:exponential_weight}
Fix \( \alpha > \pi \), and define the exponential weight
\[
\Psi_\alpha(x) := e^{\alpha |x|}, \qquad x \in \mathbb{R}.
\]
Then \( \Psi_\alpha \in C^\infty(\mathbb{R}) \) is strictly positive, even, convex, and satisfies:
\begin{itemize}
    \item Super-exponential growth: \( \Psi_\alpha(x) \to \infty \) as \( |x| \to \infty \);
    \item Rapid decay: \( \Psi_\alpha^{-1}(x) = e^{-\alpha |x|} \in L^1(\mathbb{R}) \) for all \( \alpha > 0 \).
\end{itemize}

Define the weighted Hilbert space
\[
H_{\Psi_\alpha} := L^2(\mathbb{R}, \Psi_\alpha(x)\, dx) = \left\{ f \in L^2_{\mathrm{loc}}(\mathbb{R}) \ \middle|\ \int_{\mathbb{R}} |f(x)|^2 e^{\alpha |x|}\, dx < \infty \right\}.
\]

\medskip
\noindent\textbf{Paley–Wiener Control.}
Let \( F \colon \mathbb{C} \to \mathbb{C} \) be entire of exponential type \( \tau < \alpha \). Then by the Paley–Wiener theorem \cite[Thm.~3.2.4]{Levin1996EntireLectures}, the inverse Fourier transform satisfies
\[
\mathscr{F}^{-1}[F](x) \in L^1(\mathbb{R}, \Psi_\alpha^{-1}(x)\, dx),
\]
i.e., it decays faster than \( e^{-\alpha |x|} \). This provides precise decay estimates for convolution kernels constructed from entire spectral data.

\medskip
\noindent\textbf{Application to \(\Xi\) and Canonical Kernels.}
Let
\[
\phi(\lambda) := \Xi\left( \tfrac{1}{2} + i\lambda \right), \quad
k := \mathscr{F}^{-1}[\phi], \quad
K(x,y) := k(x - y).
\]
Since \( \Xi(s) \) is entire of exponential type \( \pi \), it follows that for all \( \alpha > \pi \),
\[
K \in L^1(\mathbb{R}^2, \Psi_\alpha^{-1}(x)\Psi_\alpha^{-1}(y)\, dx\,dy),
\]
and the convolution operator
\[
(Lf)(x) := \int_{\mathbb{R}} k(x - y) f(y)\, dy
\]
lies in \( \mathcal{C}_1(H_{\Psi_\alpha}) \) by Simon's trace-class criterion \cite[Thm.~4.2]{Simon2005TraceIdeals}.

\medskip
\noindent\textbf{Mollified Heat Kernels and Trace-Norm Limit.}
Define the mollified spectral profile:
\[
\phi_t(\lambda) := e^{-t\lambda^2} \phi(\lambda), \quad
k_t := \mathscr{F}^{-1}[\phi_t], \quad
K_t(x,y) := k_t(x - y).
\]
Then for all \( t > 0 \), we have \( k_t \in \mathcal{S}(\mathbb{R}) \subset L^1(\mathbb{R}, \Psi_\alpha^{-1}(x)\, dx) \), and the associated convolution operator
\[
L_t f(x) := \int_{\mathbb{R}} k_t(x - y) f(y)\, dy
\]
lies in \( \mathcal{C}_1(H_{\Psi_\alpha}) \). Moreover, there exists a canonical trace-norm limit:
\[
L_{\mathrm{sym}} := \lim_{t \to 0^+} L_t \in \mathcal{C}_1(H_{\Psi_\alpha}).
\]

\medskip
\noindent\textbf{Sharpness of \(\alpha > \pi\).}
The condition \( \alpha > \pi \) is sharp: the Paley–Wiener bound for \( k \) yields \( |k(x)| \approx e^{-\pi|x|} \) as \( |x| \to \infty \), so for \( \alpha \le \pi \), the weighted norm
\[
\iint_{\mathbb{R}^2} |K(x,y)| \, \Psi_\alpha(x)\Psi_\alpha(y)\, dx\,dy = \infty.
\]
Thus, \( L \notin \mathcal{C}_1(H_{\Psi_\alpha}) \) unless \( \alpha > \pi \).

\medskip
\noindent\textbf{Spectral and Analytic Consequences.}
The Hilbert space \( H_{\Psi_\alpha} \), with \( \alpha > \pi \), provides the analytic framework for the determinant and trace theory:
\begin{itemize}
    \item Heat trace finiteness: \( \operatorname{Tr}(e^{-tL^2}) < \infty \) for all \( t > 0 \), enabling short-time expansion;
    \item Spectral zeta function: \( \zeta_L(s) = \sum \lambda_n^{-s} \) admits analytic continuation via Tauberian theory \cite{Korevaar2004Tauberian};
    \item Determinant identity: \( \det\nolimits_\zeta(I - \lambda L) \) is entire of order one and recovers \( \Xi(\tfrac{1}{2} + i\lambda) \) up to normalization.
\end{itemize}

\medskip
\noindent\textbf{References.}
\begin{itemize}
    \item B. Ya. Levin, \emph{Lectures on Entire Functions}, Theorem~3.2.4 \cite{Levin1996EntireLectures}.
    \item B. Simon, \emph{Trace Ideals and Their Applications}, Theorem~4.2 \cite{Simon2005TraceIdeals}.
    \item J. Korevaar, \emph{Tauberian Theory}, Chapter~III \cite{Korevaar2004Tauberian}.
\end{itemize}
\end{definition}
