\begin{remark}[Trace Kernel Decay and Spectral Regularization]
\label{rem:generalized_trace_decay}
The exponential decay condition of the kernel \( k_t^{(\pi)} \), as defined in \defref{def:kernel_decay_condition}, is critical for ensuring that the mollified operators \( \Ltpi \in \TC(\Hpsipi) \) and that their trace-norm limit \( \Lsympi \) exists and retains compactness and spectral regularity.

This decay permits well-defined Laplace and zeta-regularizations of the spectral data. In particular, it guarantees that the heat semigroup \( e^{-t (\Lsympi)^2} \) exists as a trace-class semigroup for all \( t > 0 \), and enables the analytic continuation of the associated spectral zeta function:
\[
\zeta_{\Lsympi}(s) := \Tr\left( (\Lsympi)^{-s} \right)
\]
in a half-plane \( \Re(s) > \sigma_0 \). These ingredients, familiar from the Riemann case, extend to the automorphic setting once analytic control of the kernel decay is established.

In the absence of this exponential decay, the spectral trace and determinant constructions would fail to converge, and no generalization of the Fredholm identity would be valid. Thus, the entire formalism relies on the spectral moderation of \( \Xi(s, \pi) \) through its inverse Fourier profile.
\end{remark}
