\begin{proof}[\ref{lem:spectral_measure_heat_semigroup}]
We prove the result using the spectral theorem and properties of trace-class semigroups.

Since \( L_{\mathrm{sym}} \) is self-adjoint and compact, it admits a discrete spectral decomposition:
\[
L_{\mathrm{sym}} = \sum_{\rho} \mu_\rho P_\rho,
\]
where each \( \mu_\rho \in \mathbb{R} \) is an eigenvalue of finite multiplicity and \( P_\rho \) is the corresponding orthogonal projection.

By the spectral functional calculus, the heat semigroup is given by:
\[
e^{-tL_{\mathrm{sym}}^2} = \sum_{\rho} e^{-t\mu_\rho^2} P_\rho,
\]
and taking the trace yields:
\[
\operatorname{Tr}(e^{-tL_{\mathrm{sym}}^2}) = \sum_\rho e^{-t\mu_\rho^2}.
\]
Define the spectral measure \( \mu := \sum_{\rho} \delta_{\mu_\rho} \), accounting for multiplicity. Then the heat trace becomes the Laplace transform:
\[
\operatorname{Tr}(e^{-tL_{\mathrm{sym}}^2}) = \int_{\mathbb{R}} e^{-t\lambda^2} \, d\mu(\lambda).
\]

For test functions \( \phi \in \mathcal{S}(\mathbb{R}) \), the spectral theorem gives:
\[
\phi(L_{\mathrm{sym}}) = \sum_{\rho} \phi(\mu_\rho) P_\rho,
\]
and taking the trace:
\[
\operatorname{Tr}(\phi(L_{\mathrm{sym}})) = \sum_{\rho} \phi(\mu_\rho) = \int_{\mathbb{R}} \phi(\lambda) \, d\mu(\lambda).
\]

Finally, for the Carleman zeta-regularized determinant, we recall the logarithmic trace identity valid for compact operators:
\[
\log \det\nolimits_{\zeta}(I - \lambda L_{\mathrm{sym}}) = - \sum_\rho \log\left(1 - \lambda \mu_\rho^{-1} \right) = - \int_{\mathbb{R}} \log\left(1 - \frac{\lambda}{\lambda'} \right) \, d\mu(\lambda'),
\]
which matches the Hadamard representation for entire functions of order one with zero set \( \{\mu_\rho\} \). This completes the proof.
\end{proof}
s