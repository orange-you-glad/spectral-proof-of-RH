\begin{proof}[Proof of \thmref{thm:truth_of_rh_from_trace_asymptotics}]
From \thmref{thm:spectral_bijection_from_trace_asymptotics}, the nontrivial zeros $\rho$ of $\zeta(s)$ are in bijective, multiplicity-preserving correspondence with the spectrum of $\tilde{L}_{\mathrm{sym}}$ via
\[
\mu_\rho := \frac{1}{i}(\rho - \tfrac{1}{2}) \in \operatorname{Spec}(\tilde{L}_{\mathrm{sym}}).
\]

Assume $\tilde{L}_{\mathrm{sym}}$ is self-adjoint. Then by the spectral theorem for bounded linear operators on a Hilbert space, we have:
\[
\operatorname{Spec}(\tilde{L}_{\mathrm{sym}}) \subset \mathbb{R}.
\]
Hence, every $\mu_\rho \in \mathbb{R}$, which implies:
\[
\operatorname{Im}(\mu_\rho) = 0 \quad \Rightarrow \quad \operatorname{Re}(\rho) = \tfrac{1}{2}.
\]
Therefore, all nontrivial zeros of $\zeta(s)$ lie on the critical line, and the Riemann Hypothesis holds.
\end{proof}
