\section{Spectral Physics Interpretation}
\label{app:spectral_physics_link}

\noindent\textbf{[Noncritical Appendix]}  
This appendix explores speculative physical analogies of the canonical operator \( \Lsym \in \TC(\HPsi) \). No physical model is constructed or required for any analytic result. However, the spectral structure of \( \Lsym \) formally resembles quantum Hamiltonians and thermodynamic systems, offering a conceptual bridge between analytic number theory and statistical mechanics.

\subsection*{Partition Function Analogy}

The spectral trace
\[
Z(t) := \Tr(e^{-t \Lsym})
\]
resembles a thermal partition function for a quantum system with discrete energy levels \( \mu_n \). Its short-time singular expansion,
\[
Z(t) \sim \frac{1}{\sqrt{4\pi t}} \log\left( \frac{1}{t} \right) + o(t^{-1/2}), \qquad t \to 0^+,
\]
mirrors ultraviolet divergences common in heat kernel regularization of noncompact spectral problems.

The zeta-regularized Fredholm determinant
\[
\detz(I - \lambda \Lsym)
\]
may then be interpreted as a spectral free energy, with its logarithm playing the role of a generating function or thermodynamic potential. This analogy is formalized in \cref{thm:det_identity_revised}.

\subsection*{GUE Statistics and Inverse Spectrum}

Under the nonlinear spectral map
\[
\gamma_n \longmapsto \mu_n := \frac{1}{\gamma_n},
\]
the conjectured GUE distribution of \( \zeta \)-zeros~\cite{Montgomery1973PairCorrelation, Berry1986RiemannSpectra} transforms into a compressed spectrum \( \mu_n \in \Spec(\Lsym) \). This inversion:
\begin{itemize}
  \item Compresses high-energy behavior while enhancing resolution at low \( \gamma \);
  \item Yields a finite trace-class operator with spectrally regulated infrared properties;
  \item Suggests \( \Lsym \) models a thermodynamically regularized arithmetic Hamiltonian.
\end{itemize}

\subsection*{Interpretation and Outlook}

These analogies do not impact the analytic construction or RH equivalence results. No Lagrangian, path integral, or quantization framework is proposed. Nonetheless, they may inspire future investigation into:

\begin{itemize}
  \item Quantum mechanical models of arithmetic spectra;
  \item Spectral trace-class Hamiltonians in arithmetic QFT;
  \item Regularized determinants as partition functions in nonlocal field theories;
  \item Categorical or gauge-theoretic representations of zeta spectra.
\end{itemize}

\noindent
While purely heuristic here, the canonical spectral encoding via \( \Lsym \) provides a natural testing ground for connections between number theory, spectral geometry, and statistical mechanics.

\subsection*{Analogy Table: Zeta–Physics Correspondence}

\begin{center}
\small
\renewcommand{\arraystretch}{1.3}
\begin{tabularx}{\textwidth}{|X|X|X|}
\hline
\textbf{Zeta-Theoretic Object} & \textbf{Physics Interpretation} & \textbf{Formal Analog} \\
\hline
\( \gamma_n \) (zeta zero ordinates) &
Quantum momentum or energy eigenvalues &
\( E_n \sim \gamma_n \) \\
\( \mu_n := \gamma_n^{-1} \) &
Inverse energy; infrared spectral scale &
Bound state or regulated mode \\
\( Z(t) := \Tr(e^{-t \Lsym}) \) &
Partition function at temperature \( t^{-1} \) &
Heat kernel trace \\
\( \detz(I - \lambda \Lsym) \) &
Regularized spectral free energy &
\( \log \mathcal{Z}(\lambda) \) in QFT \\
\( \Theta(t) \sim t^{-1/2} \log(1/t) \) &
UV divergence in partition function &
Short-time blowup; spectral singularity \\
\( \zeta_{\Lsym^2}(s) \) &
Spectral zeta function; internal energy &
Mellin transform of heat kernel \\
\( \Xi(s) \) &
Canonical spectral partition function &
Determinant over arithmetic spectrum \\
\hline
\end{tabularx}
\end{center}

\noindent
This table heuristically aligns the canonical analytic structures with thermodynamic and quantum analogs. These correspondences are not used in any formal argument but may guide speculative unification between spectral number theory and quantum physics.

\medskip
\noindent
For the formal analytic proof of
\[
\RH \iff \Spec(\Lsym) \subset \R,
\]
see Chapter~\ref{sec:spectral_implications}.
