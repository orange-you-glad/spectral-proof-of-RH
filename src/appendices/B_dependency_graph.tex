\section{Logical Dependency Graph (Modular Proof Architecture)}
\label{app:dependency_graph}

\noindent
This appendix presents the formal structure of the manuscript as a directed acyclic graph (DAG), in which each chapter builds only on previously established analytic foundations. No theorem appeals to any result logically equivalent to the Riemann Hypothesis prior to its proof, ensuring strict acyclicity and audit transparency.

For symbol definitions, see Appendix~\ref{app:notation_summary}.

\vspace{1ex}
\hrule
\vspace{1ex}

\begin{quote}
\textit{This proof flow diagram captures the modular structure of the analytic–spectral program. Each node reflects an acyclic dependency, ensuring strict logical sequencing from foundational definitions to the final equivalence with RH.}
\end{quote}

\vspace{1ex}
\hrule
\vspace{2ex}

\subsection*{Analytic Preconditions for the Determinant Identity}
\label{dag:determinant_preconditions}

The analytic identity
\[
\detz(I - \lambda \Lsym) = \frac{\XiR(\tfrac{1}{2} + i\lambda)}{\XiR(\tfrac{1}{2})}
\]
is built on the following validated properties, derived across Chapters~1–2 and Appendix~\ref{app:zeta_trace_background}:

\begin{center}
\renewcommand{\arraystretch}{1.4}
\begin{tabularx}{\textwidth}{|l|X|}
\hline
\textbf{Property} & \textbf{Summary and Source} \\
\hline
\textbf{Exponential Weight Threshold} & \( \alpha > \pi \) is necessary and sufficient for trace-class kernel inclusion; sharp cutoff in \cref{lem:trace_class_failure_alpha_leq_pi}. \\
\textbf{Spectral Profile Class} & \( \phi(\lambda) := \XiR(\tfrac{1}{2} + i\lambda) \in \PW{\pi} \) ensures exponential type and kernel decay (\cref{lem:xi_growth_bound}). \\
\textbf{Kernel Localization} & \( \widehat{\Xi}(x) \in L^1(\R, e^{-\alpha |x|} dx) \) for all \( \alpha > \pi \); see \cref{lem:weighted_L1_inverse_FT_xi}. \\
\textbf{Operator Regularity} & \( \Lsym \in \TC(\HPsi) \) is compact, self-adjoint, and trace-norm constructed (\cref{lem:trace_norm_convergence_Lt_to_Lsym}, \cref{lem:trace_norm_limit_unique}). \\
\textbf{Heat Semigroup Well-Posedness} & \( e^{-t\Lsym^2} \in \TC(\HPsi) \) is holomorphic and integrable (\cref{lem:heat_semigroup_existence}, \cref{lem:heat_semigroup_wellposed}). \\
\textbf{Determinant Growth and Entirety} & \( \detz \) is entire of exponential type \( \pi \); growth bounds from trace asymptotics (\cref{lem:det_identity_entire_order_one}). \\
\hline
\end{tabularx}
\end{center}

\subsection*{Directed Acyclic Graph (Visual Overview)}

\begin{figure}[ht]
\centering
\scalebox{0.87}{
\begin{tikzpicture}[
  node distance=1.4cm and 2.6cm,
  every node/.style={draw, align=center, rounded corners=3pt, font=\small, fill=gray!5, text width=4.8cm},
  arrow/.style={-{Stealth}, thick}
]

\node (ch1) {Ch.~1\\Foundational Structures};
\node (ch2) [below=of ch1] {Ch.~2\\Operator Construction};
\node (ch3) [below=of ch2] {Ch.~3\\Determinant Identity};
\node (ch4) [right=of ch3] {Ch.~4\\Spectral Bijection};
\node (ch5) [below=of ch3] {Ch.~5\\Heat Kernel Asymptotics};
\node (ch6) [right=of ch5] {Ch.~6\\RH Equivalence};
\node (ch7) [below=of ch5] {Ch.~7\\Tauberian Growth};
\node (ch8) [right=of ch7] {Ch.~8\\Spectral Rigidity};
\node (ch9) [below=of ch7] {Ch.~9\\Logical Closure};

\draw[arrow] (ch1) -- (ch2);
\draw[arrow] (ch2) -- (ch3);
\draw[arrow] (ch3) -- (ch5);
\draw[arrow] (ch5) -- (ch7);
\draw[arrow] (ch7) -- (ch9);

\draw[arrow] (ch3) -- (ch4);
\draw[arrow] (ch4) -- (ch6);
\draw[arrow] (ch5) -- (ch6);
\draw[arrow] (ch6) -- (ch8);
\draw[arrow] (ch8) -- (ch9);

\end{tikzpicture}
}
\caption{Directed acyclic graph of analytic dependencies. Arrows represent logical flow between chapters.}
\label{fig:dag_visual}
\end{figure}

\paragraph{Global Structure.}
Each lemma, proposition, and theorem depends only on prior analytic inputs or trace-class spectral calculus. No assumption of RH, spectral bijection, or zero reality is made prior to its proof. The analytic chain from determinant to RH equivalence,
\[
\RH \iff \Spec(\Lsym) \subset \R,
\]
is modular, acyclic, and closed in \cref{sec:logical_closure}, grounded in results from Chapters~1–8 and Appendix~\ref{app:zeta_trace_background}.

\paragraph{Base Constraint.}
The exponential weight threshold \( \alpha > \pi \) governs all trace-class constructions and appears as the root node of the DAG (\cref{lem:trace_class_failure_alpha_leq_pi}).

\begin{remark}[Closure of Forward Dependencies]
The determinant identity in Chapter~\ref{sec:determinant_identity} invokes heat trace asymptotics and Laplace integrability results proved in Chapter~\ref{sec:heat_kernel_asymptotics} and Appendix~\ref{app:heat_kernel_construction}. These forward uses are modular, formally declared, and induce no logical cycles.
\end{remark}

\medskip

Forward dependency annotations are given in \remref{rem:forward_spectral_closure}. The equivalence
\[
\RH \iff \Spec(\Lsym) \subset \R
\]
is fully formalized in \thmref{thm:rh_spectrum_equiv} and concluded in \thmref{thm:truth_of_rh}.

The DAG structure also extends conjecturally to the automorphic setting (Chapter~\ref{sec:spectral_generalization}), where one postulates analogous trace-class operators \( L_\pi \) satisfying:
\[
\GRH(\pi) \iff \Spec(L_\pi) \subset \R.
\]

For notation, see Appendix~\ref{app:notation_summary}.

\begin{tcolorbox}[colback=gray!2!white, colframe=black!50, title={\textbf{Canonical Equivalence — RH via Spectral Reality}}]
The canonical operator \( \Lsym \in \TC(\HPsi) \), constructed in Chapter~\ref{sec:operator_construction} and analytically normalized in Chapter~\ref{sec:determinant_identity}, satisfies the determinant identity
\[
\detz(I - \lambda \Lsym) = \frac{\Xi(\tfrac{1}{2} + i\lambda)}{\Xi(\tfrac{1}{2})},
\]
whose spectral zeros match the nontrivial zeros of \( \zeta(s) \). This determinant structure, combined with the bijection \( \rho \mapsto \mu_\rho := \tfrac{1}{i}(\rho - \tfrac{1}{2}) \), canonically encodes zeta zeros into the spectrum of \( \Lsym \). The Riemann Hypothesis is then equivalent to spectral reality:
\[
\RH \iff \Spec(\Lsym) \subset \R,
\]
as formally proven in \thmref{thm:truth_of_rh}. This equivalence is derived solely from analytic spectral theory and zeta determinant regularization.
\end{tcolorbox}
